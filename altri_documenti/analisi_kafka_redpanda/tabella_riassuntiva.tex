\section{Tabella riassuntiva}

\begin{longtable}{|>{\centering\arraybackslash}p{0.30\textwidth}|>{\centering\arraybackslash}p{0.30\textwidth}|>{\centering\arraybackslash}p{0.30\textwidth}|}
	\hline
	\textbf{Paragone}                           & \textbf{Apache Kafka}                                                                                                                                       & \textbf{Redpanda}                                                                                                                                                      \\
	\hline
	\endfirsthead
	\hline
	\textbf{Paragone}                           & \textbf{Apache Kafka}                                                                                                                                       & \textbf{Redpanda}                                                                                                                                                      \\
	\endhead
	\hline
	\textbf{Adozione}                           & Utilizzato da migliaia di compagnie (tra cui LinkedIn, Airbnb, e Netflix)                                                                                   & Non chiaro quante organizzazioni lo usino. Adottato da Cisco e Vodafone.                                                                                               \\
	\hline
	\textbf{Community}                          & Migliaia di contributori                                                                                                                                    & \textit{Community} più piccola ed emergente.                                                                                                                           \\
	\hline
	\textbf{Maturità}                           & Stabile, sviluppato dal 2011                                                                                                                                & Emergente, lanciato nel 2019.                                                                                                                                          \\
	\hline
	\textbf{Documentazione, risorse}            & Documentazione dettagliata, forum, tutorial, e corsi online                                                                                                 & Documentazione dettagliata, ma non altrettante risorse. Tutorial creati dal team di Redpanda.                                                                          \\
	\hline
	\textbf{\textit{Client}}                    & Ampia varietà di \textit{client} per i principali linguaggi di programmazione                                                                               & Lista di client ufficialmente testati, ma \href{https://docs.redpanda.com/current/develop/kafka-clients/}{qualsiasi client Kafka è compatibile}.                       \\
	\hline
	\textbf{CLIs}                               & Include un set di strumenti per gestire i topic, messaggi, cluster...                                                                                       & Include \texttt{rpk} , un'interfaccia per gestire topic, messaggi, debugging, interazione con Redpanda Cloud.                                                          \\
	\hline
	\textbf{Monitoraggio}                       & Richiede configurazioni di sistemi di monitoraggio (JMX, Grafana, Prometheus)                                                                               & Integrato direttamente con Prometheus e Grafana.                                                                                                                       \\
	\hline
	\textbf{Architettura}                       & Complesso da configurare e gestire su larga scala. Solo a partire dalla versione 3.4.0 è possibile eseguirlo senza ZooKeeper.                               & Facile da installare e configurare, indipendente da Zookeeper, integrato con una web UI (\href{https://redpanda.com/redpanda-console-kafka-ui}{Redpanda Console}).     \\
	\hline
	\textbf{Licenza}                            & Open source, Apache 2.0                                                                                                                                     & Edizioni \textit{Community} e \textit{Enterprise}, BSL (Business Source License).                                                                                      \\
	\hline
	\textbf{\textit{Deploy self-hosted}}        & \textit{Bare-metal}, macchine virtuali, \textit{cloud}, Docker, Kubernetes & \textit{Bare-metal}, macchine virtuali, \textit{cloud}, Docker, Kubernetes            \\
	\hline
	\textbf{\textit{Managed deploy}}            & Numerosi servizi di terze parti, come Confluent Cloud, AWS MSK...                                                                                           & Offre 3 opzioni: \textit{cluster} dedicati gestiti da Redpanda, BYOC (\textit{Bring Your Own Cloud}), \textit{cluster serverless} su architettura gestita da Redpanda. \\
	\hline
	\textbf{\textit{Schema registry} integrato} & No                                                                                                                                                          & Sì                                                                                                                                                                     \\
	\hline
	\textbf{Protocollo di replicazione}         & Sincrono o asincrono                                                                                                                                        & Sincrono                                                                                                                                                               \\
	\hline
	\textbf{Modello di contribuzione}           & Open source, supporto dalla community e da aziende                                                                                                          & Sviluppato solamente dal \textit{team} di Redpanda                                                                                                                     \\
	\hline
	\caption{Riassunto del confronto tra \textit{Apache Kafka} e \textit{Redpanda}}
\end{longtable}





















