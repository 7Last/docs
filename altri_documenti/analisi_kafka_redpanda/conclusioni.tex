\section{Conclusioni}
Apache Kafka e Redpanda sono due strumenti molto simili, ma rispondono ad esigenze differenti. Nel caso si debba gestire un progetto in ambiente di produzione, Apache Kafka è la scelta ottimale, in quanto è più stabile, testato e affidabile. Redpanda invece si presta meglio per progetti più semplici e con carichi di dati minori. Inoltre, risulta maggiormente adatto a utenti più inesperti, in quanto richiede meno configurazioni. \\
Un altro aspetto da considerare è la licenza: Apache Kafka è \textit{open source}, mentre Redpanda è un prodotto commerciale; nel caso di budget limitato, Apache Kafka risulta dunque più conveniente.\\
Nelle valutazioni per la scelta dello strumento più adatto, è importante tenere conto che i \textit{benchmark} sono stati eseguiti dai creatori dei \textit{software}, perciò potrebbero essere stati studiati in modo da favorire uno strumento rispetto all'altro.\\
Ai fini della realizzazione del \textit{Proof of Concept} e del \textit{Minimum Viable Product} non sono richieste prestazioni elevate in quanto il carico di dati sarà limitato, perciò pensiamo che sia sufficiente utilizzare Redpanda. Essendo il progetto didattico il primo approccio a questo tipo di tecnologia per alcuni membri del gruppo, Redpanda permetterebbe a tutti i componenti di apprenderne il funzionamento in modo più semplice e veloce.\\
Data la compatibilità tra le due tecnologie, in un secondo momento si potrebbe facilmente passare ad Apache Kafka, senza dover riscrivere il codice.\\
Infine, nel caso in cui il progetto dovesse evolvere oltre il \textit{Minimum Viable Product}, riterremmo più opportuno passare ad Apache Kafka.




