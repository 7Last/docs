\documentclass[italian,12pt]{article} %tipo di documento

%--------------variabili------------------%
\def\Title{Norme di Progetto}
\def\Author{7Last}
\def\Version{v0.2}
%-----------------------------------------%


\usepackage[left=2cm, right=2cm, bottom=3cm, top=3cm]{geometry}
\usepackage{fancyhdr}
\usepackage{graphicx}
\graphicspath{ {../../logo/} }
\usepackage{href-ul}
\usepackage{tikz}
\usepackage{tgadventor}
\usepackage[useregional=numeric,showseconds=true,showzone=false]{datetime2}
\usepackage{caption}
\usepackage{longtable}
\usepackage{xcolor}




\linespread{1.2}
\captionsetup[table]{labelformat=empty}

\renewcommand{\contentsname}{Indice}%imposto il nome dell'indice
\renewcommand\familydefault{\sfdefault}
%-------------------INIZIO DOCUMENTO--------------
\begin{document}

\newgeometry{left=2cm,right=2cm,bottom=2.1cm,top=2.1cm}
\begin{titlepage}
	\vspace*{.5cm}

	\vspace{2cm}
	{
		\centering
		{\bfseries\huge \Title\par}
		\bigbreak
		{\bfseries\Large \Subtitle\par}
		\bigbreak
		{\bfseries\large \Author\par}
		\bigbreak
		{\Date\;-\;\Version\par}
		\vfill

		\begin{center}
			\begin{tikzpicture}
				\clip (0,0) circle (2cm) node {\includegraphics[width=4cm]{logo.jpg}};
			\end{tikzpicture}
		\end{center}
	}

	\vfill

\end{titlepage}

\restoregeometry






















\newpage

\pagestyle{fancy}
\fancyhead{}
\lhead{
	\begin{tikzpicture}
		\clip (0,0) circle (0.5cm);
		\node at (0,0) {\includegraphics[width=1cm]{./../logo/logo.png}};
	\end{tikzpicture}%
}
\chead{\vspace{\fill}\Title\vspace{\fill}}
\rhead{\vspace{\fill}\Version\vspace{\fill}}




%-----------tabella versioni-----------%
\begin{table}[!h]
	\caption{Versioni}
	\begin{center}
		\begin{tabular}{ c c c c c c }
			\hline \\[-2ex]
			Ver. & Data & Autore & Descrizione \\
			\\[-2ex] \hline \\[-1.5ex]
			1.0 & 05/03/2024 & Leonardo Baldo& prima stesura \\
			1.1 & 15/03/2024 & Valerio Occhinegro& rifinitura  \\
			\\[-1.5ex] \hline
		\end{tabular}
	\end{center}
\end{table}
%---------------------------------------%
\newpage

\tableofcontents

\newpage

\section{Registro presenze}

\paragraph{Dettagli}

\begin{itemize}
	\item Sede della riunione: on-line sulla piattaforma discord
	\item Orario di inizio: 21:00
	\item Orario di fine: 22:00
\end{itemize}

\paragraph{Componenti del gruppo}

\begin{flushleft}
\begin{table}[!h]
\begin{tabular}{ |c|c|c| } 
	\hline
	Partecipanti & Presenza \\
	\hline 
	Leonardo Baldo 		 & 1:00 h \\ 
	Antonio Benetazzo 	 & 1:00 h \\
	Elena Ferro 		 & 1:00 h \\
	Davide Malgarise 	 & 1:00 h \\
	Valerio Occhinegro 	 & 1:00 h \\
	Raul Seganfreddo 	 & 1:00 h \\
	Matteo Tiozzo 		 & 1:00 h \\ 
	\hline
\end{tabular}
\end{table}
\end{flushleft}


\newpage

\section{Verbale}
\begin{flushleft}

Prima riunione ufficiale del gruppo 7Last effettuata in modalità remota utilizzando REDATTO. 
(Capire se Discord può essere nominato o meno)

Il primo incontro del gruppo ha  inizio con la presentazione di ciascun membro appartenente allo stesso. 
In seguito a questa parte introduttiva si avvia un processo di brainstorming relativo, inizialmente, all’individuazione di un nome che 
possa risultare accattivante. 
	
Al termine di questo processo il nome con più votazioni, e dunque quello scelto per rappresentare il nostro team, è “7Last”: un nominativo 
che riassume al suo interno due delle caratteristiche base del gruppo, quali il numero di componenti (7), e l’ordine posizionale di creazione del team (19° gruppo su 19).
	
Lo step successivo è caratterizzato dalla ideazione e creazione di un logo, focalizzate sulla scelta di: colori, stile, forma e contenuto. 
Viene dunque incaricato Matteo Tiozzo per la generazione, tramite AI, di un logo coerente con le preferenze espresse.
	
Il gruppo poi si esprime circa i capitolati proposti dalle aziende, partecipanti al progetto formativo universitario di 
“Ingegneria del Software”. Da questa discussione emerge un interesse comune relativo al capitolato 6 “SyncCity”, proposto dall’azienda 
SyncLab. Nonostante ciò, tutti i partecipanti del gruppo sono concordi nel non prendere decisioni affrettate, decidono dunque di richiedere 
un incontro conoscitivo ed esplicativo con le tre aziende per approfondire ulteriormente le proposte offerte e descritte nei vari progetti.
	
Il membro Valerio Occhinegro si occupa e gestisce la creazione di una mail di rappresentanza per poter interagire in maniera più agevole, 
funzionale e formale con le varie imprese. In un secondo momento, Antonio Benetazzo redige una mail indirizzata alle tre aziende sopra 
citate per la richiesta dei colloqui.
	
L’incontro si è concluso con la scelta di LaTeX per la redazione  della documentazione , Git come strumento di versionamento  e viene 
posticipata la scelta di un ITS poichè è presente una forte indecisione tra l’utilizzo di Jira e Github
	
(valutare se aggiungere Discord, telegram e Notion come altri supporti)

\end{flushleft}

\end{document}

























