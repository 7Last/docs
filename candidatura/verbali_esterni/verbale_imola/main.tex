\documentclass[italian,12pt]{article} %tipo di documento

%--------------variabili------------------%
\def\Title{Norme di Progetto}
\def\Author{7Last}
\def\Version{v0.2}
%-----------------------------------------%


\usepackage[left=2cm, right=2cm, bottom=3cm, top=3cm]{geometry}
\usepackage{fancyhdr}
\usepackage{graphicx}
\graphicspath{ {../../logo/} }
\usepackage{href-ul}
\usepackage{tikz}
\usepackage{tgadventor}
\usepackage[useregional=numeric,showseconds=true,showzone=false]{datetime2}
\usepackage{caption}
\usepackage{longtable}
\usepackage{xcolor}




\linespread{1.2}
\captionsetup[table]{labelformat=empty}

\renewcommand{\contentsname}{Indice}%imposto il nome dell'indice
\renewcommand\familydefault{\sfdefault}
%-------------------INIZIO DOCUMENTO--------------
\begin{document}

\newgeometry{left=2cm,right=2cm,bottom=2.1cm,top=2.1cm}
\begin{titlepage}
	\vspace*{.5cm}

	\vspace{2cm}
	{
		\centering
		{\bfseries\huge \Title\par}
		\bigbreak
		{\bfseries\Large \Subtitle\par}
		\bigbreak
		{\bfseries\large \Author\par}
		\bigbreak
		{\Date\;-\;\Version\par}
		\vfill

		\begin{center}
			\begin{tikzpicture}
				\clip (0,0) circle (2cm) node {\includegraphics[width=4cm]{logo.jpg}};
			\end{tikzpicture}
		\end{center}
	}

	\vfill

\end{titlepage}

\restoregeometry






















\newpage

\pagestyle{fancy}
\fancyhead{}
\lhead{
	\begin{tikzpicture}
		\clip (0,0) circle (0.5cm);
		\node at (0,0) {\includegraphics[width=1cm]{./../logo/logo.png}};
	\end{tikzpicture}%
}
\chead{\vspace{\fill}\Title\vspace{\fill}}
\rhead{\vspace{\fill}\Version\vspace{\fill}}




%-----------tabella versioni-----------%
\begin{table}[!h]
	\caption{Versioni}
	\begin{center}
		\begin{tabular}{ c c c c c c }
			\hline \\[-2ex]
			Ver. & Data & Autore & Descrizione \\
			\\[-2ex] \hline \\[-1.5ex]
			1.0 & 15/03/2024 & Leonardo Baldo& prima stesura \\
			1.1 & 16/03/2024 & Valerio Occhinegro& rifinitura  \\
			\\[-1.5ex] \hline
		\end{tabular}
	\end{center}
\end{table}
%---------------------------------------%
\newpage

\tableofcontents

\newpage

\section{Registro presenze}

\paragraph{Dettagli}

\begin{itemize}
	\item Sede della riunione: on-line sulla piattaforma teams
	\item Orario di inizio: 14:30
	\item Orario di fine: 14:50
\end{itemize}

\paragraph{Componenti del gruppo}

\begin{flushleft}
\begin{table}[!h]
\begin{tabular}{ |c|c|c| } 
	\hline
	Partecipanti & Presenza \\
	\hline 
	Leonardo Baldo 		 & 0:20 h \\ 
	Antonio Benetazzo 	 & 0:20 h \\
	Elena Ferro 		 & 0:20 h \\
	Davide Malgarise 	 & 0:20 h \\
	Valerio Occhinegro 	 & 0:20 h \\
	Raul Seganfreddo 	 & 0:20 h \\
	Matteo Tiozzo 		 & 0:20 h \\ 
	\hline
\end{tabular}
\end{table}
\end{flushleft}

\paragraph{Componenti dell'azienda}

\begin{itemize}
	\item Federico Bernacca
	\item Stefan Glamocak
\end{itemize}

\newpage

\section{Verbale}
\begin{flushleft}

	All’incontro on-line sulla piattaforma Teams con l’azienda Imola Informatica sono presenti, 
	come riportato in tabella, i 7 membri del gruppo e due rappresentanti dell’azienda stessa.

	Il colloquio ha inizio alle ore 14:30 ed ha come finalità la conoscenza della natura del progetto proposto da Imola Informatica.
	Viene quindi richiesto un rapido recap riguardante i contenuti e le finalità del progetto che prevede la facile ed immediata 
	prenotazione di un tavolo, ordinazione di un pasto (anche per soggetti non registrati) e creazione di una admin page per i ristoranti, 
	il tutto tramite web-app.
	
	7Last richiede alcune delucidazioni riguardanti le tecnologie da utilizzare per la riuscita positiva del progetto e se verrà fornita 
	inizialmente una struttura per il backend o se dovrà essere sviluppata. Gli esponenti dell’azienda affermano che la scelta potrà 
	ricadere liberamente su qualsiasi tecnologia e che sarà necessario occuparsi anche della progettazione backend e della crezione 
	di un mock per il processo di pagamento.
	
	L’azienda offre un incontro settimanale dedicato alla formazione e all’affiancamento pratico nel progetto nel caso in cui vi 
	siano dubbi o emergano problematiche; afferma inoltre che le comunicazioni potranno essere gestite tramite la creazione di un 
	gruppo Telegram per velocizzare le tempistiche e rendere più semplici gli scambi. Imola Informatica offre un supporto per 
	l’organizzazione e l’dentificazione dei ruoli necessari per la riuscita del lavoro e la frammentazione dello stesso; 
	sarà dunque utile fornire delle preferenze riguardanti il ruolo da investire, anche in base alle conoscenze personali 
	relative alla padronanza e conoscenza dei linguaggi.
	
	L’incontro giunge dunque al termine alle ore 14:50.

\end{flushleft}

\begin{table}[b]
\begin{tabular}{@{}p{.5in}p{4in}@{}}
	Data:  & \hrulefill \\
		   &     		\\
		   &     		\\
	Firma: & \hrulefill \\
\end{tabular}
\end{table}
	

\end{document}

























