\documentclass[italian,12pt]{article} %tipo di documento

%--------------variabili------------------%
\def\Title{Norme di Progetto}
\def\Author{7Last}
\def\Version{v0.2}
%-----------------------------------------%


\usepackage[left=2cm, right=2cm, bottom=3cm, top=3cm]{geometry}
\usepackage{fancyhdr}
\usepackage{graphicx}
\graphicspath{ {../../logo/} }
\usepackage{href-ul}
\usepackage{tikz}
\usepackage{tgadventor}
\usepackage[useregional=numeric,showseconds=true,showzone=false]{datetime2}
\usepackage{caption}
\usepackage{longtable}
\usepackage{xcolor}




\linespread{1.2}
\captionsetup[table]{labelformat=empty}

\renewcommand{\contentsname}{Indice}%imposto il nome dell'indice
\renewcommand\familydefault{\sfdefault}
%-------------------INIZIO DOCUMENTO--------------
\begin{document}

\newgeometry{left=2cm,right=2cm,bottom=2.1cm,top=2.1cm}
\begin{titlepage}
	\vspace*{.5cm}

	\vspace{2cm}
	{
		\centering
		{\bfseries\huge \Title\par}
		\bigbreak
		{\bfseries\Large \Subtitle\par}
		\bigbreak
		{\bfseries\large \Author\par}
		\bigbreak
		{\Date\;-\;\Version\par}
		\vfill

		\begin{center}
			\begin{tikzpicture}
				\clip (0,0) circle (2cm) node {\includegraphics[width=4cm]{logo.jpg}};
			\end{tikzpicture}
		\end{center}
	}

	\vfill

\end{titlepage}

\restoregeometry






















\newpage

\pagestyle{fancy}
\fancyhead{}
\lhead{
	\begin{tikzpicture}
		\clip (0,0) circle (0.5cm);
		\node at (0,0) {\includegraphics[width=1cm]{./../logo/logo.png}};
	\end{tikzpicture}%
}
\chead{\vspace{\fill}\Title\vspace{\fill}}
\rhead{\vspace{\fill}\Version\vspace{\fill}}




%-----------tabella versioni-----------%
\begin{table}[!h]
	\caption{Versioni}
	\begin{center}
		\begin{tabular}{ c c c c c c }
			\hline \\[-2ex]
			Ver. & Data & Autore & Descrizione \\
			\\[-2ex] \hline \\[-1.5ex]
			1.0 & 08/03/2024 & Leonardo Baldo& prima stesura \\
			1.1 & 15/03/2024 & Valerio Occhinegro& rifinitura  \\
			\\[-1.5ex] \hline
		\end{tabular}
	\end{center}
\end{table}
%---------------------------------------%
\newpage

\tableofcontents

\newpage

\section{Registro presenze}

\paragraph{Dettagli}

\begin{itemize}
	\item Sede della riunione: on-line sulla piattaforma zoom
	\item Orario di inizio: 16:00
	\item Orario di fine: 16:30
\end{itemize}

\paragraph{Componenti del gruppo}

\begin{flushleft}
\begin{table}[!h]
\begin{tabular}{ |c|c|c| } 
	\hline
	Partecipanti & Presenza \\
	\hline 
	Leonardo Baldo 		 & 0:30 h \\ 
	Antonio Benetazzo 	 & 0:30 h \\
	Elena Ferro 		 & 0:30 h \\
	Valerio Occhinegro 	 & 0:30 h \\
	Raul Seganfreddo 	 & 0:30 h \\
	Matteo Tiozzo 		 & 0:30 h \\ 
	\hline
\end{tabular}
\end{table}
\end{flushleft}

\paragraph{Componenti dell'azienda}

\begin{itemize}
	\item Gregorio Piccoli
\end{itemize}

\newpage

\section{Verbale}
\begin{flushleft}

All’incontro on-line sulla piattaforma Zoom con l’azienda Zucchetti sono presenti, come riportato in tabella, 6 membri del gruppo 
(un assente) e un rappresentante dell’azienda stessa.

Il colloquio ha inizio alle ore 16:00 ed ha come finalità la conoscenza della natura del progetto proposto da Zucchetti.
	
7Last richiede alcune delucidazioni riguardanti le tecnologie da utilizzare per la riuscita positiva del progetto; 
l’esponente dell’azienda afferma che, sulla base di esperienze pregresse con altri gruppi di lavoro universitari, 
sono consigliabili le seguenti: Python, interfaccia in HTML, JS, CSS.
	
Emerge l’interesse da parte del gruppo di comprendere in maniera più specifica quale struttura di prompt sia necessario ottenere 
ai fini del progetto. Gregorio Piccoli mostra, tramite l’utilizzo di diversi modelli di AI, i possibili tipi di prompt e la reazione 
dell’intelligenza artificiale ad essi per poi terminare con la visione della tipologia di output richiesta. Quest’ultima consiste nel 
descrivere la struttura, alcuni contenuti specifici del database e la richiesta da tradurre in linguaggio SQL.
	
Per far comprendere al meglio la necessità di un prodotto di questo tipo, l’esponente dell’azienda ha illustrato numerosi esempi 
di quesiti ai quali l’AI non è stata in grado di rispondere in maniera adeguata. Si è dunque compreso che maggiore è la specificità 
del prompt, migliore sarà la risposta ottenuta tramite AI.
	
In caso di necessità di incontri futuri, il gruppo chiede se verranno utilizzate le stesse modalità (incontro on-line tramite Zoom); 
l’azienda risponde che vi è anche la possibilità di visita presso la sede di Padova in via G. Cittadella 7.
	
L’incontro giunge dunque al termine alle ore 16:30.

\end{flushleft}

\begin{table}[b]
	\begin{tabular}{@{}p{.5in}p{4in}@{}}
		Data:  & \hrulefill \\
			   &     		\\
			   &     		\\
		Firma: & \hrulefill \\
	\end{tabular}
	\end{table}

\end{document}

























