\documentclass[italian,12pt]{article} %tipo di documento

%--------------variabili------------------%
\def\Title{Norme di Progetto}
\def\Author{7Last}
\def\Version{v0.2}
%-----------------------------------------%


\usepackage[left=2cm, right=2cm, bottom=3cm, top=3cm]{geometry}
\usepackage{fancyhdr}
\usepackage{graphicx}
\graphicspath{ {../../logo/} }
\usepackage{href-ul}
\usepackage{tikz}
\usepackage{tgadventor}
\usepackage[useregional=numeric,showseconds=true,showzone=false]{datetime2}
\usepackage{caption}
\usepackage{longtable}
\usepackage{xcolor}




\linespread{1.2}
\captionsetup[table]{labelformat=empty}

\renewcommand{\contentsname}{Indice}%imposto il nome dell'indice
\renewcommand\familydefault{\sfdefault}
%-------------------INIZIO DOCUMENTO--------------
\begin{document}

\newgeometry{left=2cm,right=2cm,bottom=2.1cm,top=2.1cm}
\begin{titlepage}
	\vspace*{.5cm}

	\vspace{2cm}
	{
		\centering
		{\bfseries\huge \Title\par}
		\bigbreak
		{\bfseries\Large \Subtitle\par}
		\bigbreak
		{\bfseries\large \Author\par}
		\bigbreak
		{\Date\;-\;\Version\par}
		\vfill

		\begin{center}
			\begin{tikzpicture}
				\clip (0,0) circle (2cm) node {\includegraphics[width=4cm]{logo.jpg}};
			\end{tikzpicture}
		\end{center}
	}

	\vfill

\end{titlepage}

\restoregeometry






















\newpage

\pagestyle{fancy}
\fancyhead{}
\lhead{
	\begin{tikzpicture}
		\clip (0,0) circle (0.5cm);
		\node at (0,0) {\includegraphics[width=1cm]{./../logo/logo.png}};
	\end{tikzpicture}%
}
\chead{\vspace{\fill}\Title\vspace{\fill}}
\rhead{\vspace{\fill}\Version\vspace{\fill}}




%-----------tabella versioni-----------%
\begin{table}[!h]
	\caption{Versioni}
	\begin{center}
		\begin{tabular}{ c c c c c c }
			\hline \\[-2ex]
			Ver. & Data & Autore & Descrizione \\
			\\[-2ex] \hline \\[-1.5ex]
			1.3 & 18/03/2024 & Elena Ferro & Revisione finale\\
			1.2 & 18/03/2024 & Matteo Tiozzo & Prima revisione \\
			1.1 & 15/03/2024 & Valerio Occhinegro& Sistemazione ortografica e di linguaggio \\
			1.0 & 08/03/2024 & Leonardo Baldo& Prima stesura del documento \\
			\\[-1.5ex] \hline
		\end{tabular}
	\end{center}
\end{table}
%---------------------------------------%
\newpage

\tableofcontents

\newpage

\section{Partecipanti}

\textbf{Sede della riunione}: piattaforma Zoom\\
\textbf{Orario di inizio}: 16:00\\
\textbf{Orario di fine}: 16:30\\


\paragraph{Componenti di 7Last}

\begin{flushleft}
\begin{table}[!h]
\begin{tabular}{ |c|c|c| } 
	\hline
	\textbf{Partecipanti} & \textbf{Presenza} \\
	\hline 
	Leonardo Baldo 		 & 0:30 h \\ 
	Antonio Benetazzo 	 & 0:30 h \\
	Elena Ferro 		 & 0:30 h \\
	Valerio Occhinegro 	 & 0:30 h \\
	Raul Seganfreddo 	 & 0:30 h \\
	Matteo Tiozzo 		 & 0:30 h \\ 
	\hline
\end{tabular}
\end{table}
\end{flushleft}

\paragraph{Componenti di Zucchetti SpA}
\begin{itemize}
	\item Gregorio Piccoli
\end{itemize}

\newpage

\begin{flushleft}
\section{Verbale dell'incontro}
\subsection{Tecnologie utilizzate}
	\textbf{Domanda}: che tecnologie consiglia l'azienda per lo sviluppo del progetto?\\
	L'azienda suggerisce, in base alle esperienze con i gruppi precedenti, l'utilizzo di Python, interfaccia in HTML, JS, CSS.

\subsection{Chiarimenti sulla struttura del prompt}
	\textbf{Domanda}: qual è la struttura finale del prompt?\\
	L'azienda mostra la tipologia di output richiesta, che consiste nel descrivere la struttura, alcuni contenuti specifici del database e la richiesta da tradurre in linguaggio SQL. Per far comprendere al meglio la necessità di un prodotto di questo tipo, l’esponente dell’azienda ha illustrato numerosi esempi 
	di quesiti ai quali l’AI non è stata in grado di rispondere in maniera adeguata. Si è dunque compreso che maggiore è la specificità 
	del prompt, migliore sarà la risposta ottenuta tramite AI.

\subsection{Disponibilità di Zucchetti SpA}
	\textbf{Domanda}: l'azienda sarà disponibile ad incontri in caso di necessità da parte del gruppo? Se si, in che modo?\\
	In caso di necessità l'azienda si rende disponibile ad incontri futuri, sia in sede che online. 

\subsection{Vantaggi offerti da questo progetto}
	\textbf{Domanda}: quali sono i vantaggi che questo progetto comporta rispetto all'interazione diretta con LLM?\\
	Memorizzando la struttura del database e includendo solo le tabelle di cui si necessita, è possibile ottenere un risultato più preciso e immediato, evitando allucinazioni e risposte non pertinenti.
	

\end{flushleft}
\vspace*{3cm}

\begin{table}[b]
	\begin{tabular}{@{}p{.5in}p{4in}@{}}
		Data:  & \hrulefill \\
			   &     		\\
			   &     		\\
		Firma: & \hrulefill \\
	\end{tabular}
	\end{table}

\end{document}

























