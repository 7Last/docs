\documentclass[italian,12pt]{article} %tipo di documento

%--------------variabili------------------%
\def\Title{Norme di Progetto}
\def\Author{7Last}
\def\Version{v0.2}
%-----------------------------------------%


\usepackage[left=2cm, right=2cm, bottom=3cm, top=3cm]{geometry}
\usepackage{fancyhdr}
\usepackage{graphicx}
\graphicspath{ {../../logo/} }
\usepackage{href-ul}
\usepackage{tikz}
\usepackage{tgadventor}
\usepackage[useregional=numeric,showseconds=true,showzone=false]{datetime2}
\usepackage{caption}
\usepackage{longtable}
\usepackage{xcolor}




\linespread{1.2}
\captionsetup[table]{labelformat=empty}

\renewcommand{\contentsname}{Indice}%imposto il nome dell'indice
\renewcommand\familydefault{\sfdefault}
%-------------------INIZIO DOCUMENTO--------------
\begin{document}

\newgeometry{left=2cm,right=2cm,bottom=2.1cm,top=2.1cm}
\begin{titlepage}
	\vspace*{.5cm}

	\vspace{2cm}
	{
		\centering
		{\bfseries\huge \Title\par}
		\bigbreak
		{\bfseries\Large \Subtitle\par}
		\bigbreak
		{\bfseries\large \Author\par}
		\bigbreak
		{\Date\;-\;\Version\par}
		\vfill

		\begin{center}
			\begin{tikzpicture}
				\clip (0,0) circle (2cm) node {\includegraphics[width=4cm]{logo.jpg}};
			\end{tikzpicture}
		\end{center}
	}

	\vfill

\end{titlepage}

\restoregeometry






















\newpage

\pagestyle{fancy}
\fancyhead{}
\lhead{
	\begin{tikzpicture}
		\clip (0,0) circle (0.5cm);
		\node at (0,0) {\includegraphics[width=1cm]{./../logo/logo.png}};
	\end{tikzpicture}%
}
\chead{\vspace{\fill}\Title\vspace{\fill}}
\rhead{\vspace{\fill}\Version\vspace{\fill}}




%-----------tabella versioni-----------%
\begin{table}[!h]
	\caption{Versioni}
	\begin{center}
		\begin{tabular}{ c c c c c c }
			\hline \\[-2ex]
			Ver. & Data & Autore & Descrizione \\
			\\[-2ex] \hline \\[-1.5ex]
			1.3 & 18/03/2024 & Elena Ferro & Revisione finale\\
			1.2 & 18/03/2024 & Matteo Tiozzo & Prima revisione \\
			1.1 & 15/03/2024 & Valerio Occhinegro& Sistemazione ortografica e di linguaggio  \\
			1.0 & 08/03/2024 & Leonardo Baldo& Prima stesura del documento \\
			\\[-1.5ex] \hline
		\end{tabular}
	\end{center}
\end{table}
%---------------------------------------%
\newpage

\tableofcontents

\newpage
\section{Partecipanti}

\textbf{Sede della riunione}: piattaforma meet\\
\textbf{Orario di inizio}: 15:00\\
\textbf{Orario di fine}: 15:30\\


\paragraph{Componenti di 7Last}

\begin{flushleft}
\begin{table}[!h]
\begin{tabular}{ |c|c|c| } 
	\hline
	\textbf{Partecipanti} & \textbf{Presenza} \\
	\hline 
	Leonardo Baldo 		 & 0:30 h \\ 
	Antonio Benetazzo 	 & 0:30 h \\
	Elena Ferro 		 & 0:30 h \\
	Davide Malgarise 	 & 0:30 h \\
	Valerio Occhinegro 	 & 0:30 h \\
	Raul Seganfreddo 	 & 0:30 h \\
	Matteo Tiozzo 		 & 0:30 h \\ 
	\hline
\end{tabular}
\end{table}
\end{flushleft}

\paragraph{Componenti di Sync Lab S.r.l.}

\begin{itemize}
	\item Fabio Pallaro
	\item Daniele Zorzi
	\item Andrea Dorigo
\end{itemize}

\newpage

\begin{flushleft}
\section{Verbale dell'incontro}
\subsection{Linguaggi di programmazione e tecnologie utilizzate}
	\textbf{Domanda}: è possibile utilizzare linguaggi di programmazione differenti da Python per la simulazione dei dispositivi IOT?\\
	È possibile utilizzare altri linguaggi di programmazione, l'utilizzo di Python è consigliato per la facilità del codice e per la popolarità stessa del linguaggio.


\subsection{Quantità di dati da gestire}
	\textbf{Domanda}: la quantità dei dati da gestire sarà paragonabile a quella effettivamente prodotta dai sensori?\\
	L'archittettura dovrà essere scalabile, ma per questo progetto si tratterà di una simulazione, sarà dunque sufficiente analizzarne un numero simbolico.

\subsection{Elementi di complessità}
	\textbf{Domanda}: quali sono gli elementi di complessità maggiore del progetto?\\
	Gli elementi che richiederanno maggiore impegno sono l'integrazione delle varie tecnologie tra loro, la creazione dell'architettura (networking, containerizzazione) e la configurazione (devops) dei prodotti. La parte di programmazione sarà limitata allo script python per la generazione dei dati.

\subsection{Strumenti consigliati}
	\textbf{Domanda}: quali strumenti consigliate per la gestione del progetto?\\
	Per la gestione del progetto l'azienda consiglia l'utilizzo di ClickUp, Discord e Trello.

\subsection{Disponibilità e organizzazione Sync Lab S.r.l.}
	\textbf{Domanda}: quale supporto verrà fornito dall'azienda? Come saranno organizzate le riunioni e gli incontri?\\
	L'azienda fornirà una formazione per le tecnologie che andranno utilizzate e metterà a disposizione un tutor per la risoluzione di eventuali problemi. Sottolinea che sarà disponibile ad effettuare un incontro di routine una volta a settimana, da concordare con il gruppo, e sarà disponibile in caso di necessità. 

\end{flushleft}

\begin{table}[b]
	\begin{tabular}{@{}p{.5in}p{4in}@{}}
		Data:  & \hrulefill \\
			   &     		\\
			   &     		\\
		Firma: & \hrulefill \\
	\end{tabular}
	\end{table}

\end{document}

























