\documentclass[italian,12pt]{article} %tipo di documento

%--------------variabili------------------%
\def\Title{Norme di Progetto}
\def\Author{7Last}
\def\Version{v0.2}
%-----------------------------------------%


\usepackage[left=2cm, right=2cm, bottom=3cm, top=3cm]{geometry}
\usepackage{fancyhdr}
\usepackage{graphicx}
\graphicspath{ {../../logo/} }
\usepackage{href-ul}
\usepackage{tikz}
\usepackage{tgadventor}
\usepackage[useregional=numeric,showseconds=true,showzone=false]{datetime2}
\usepackage{caption}
\usepackage{longtable}
\usepackage{xcolor}




\linespread{1.2}
\captionsetup[table]{labelformat=empty}

\renewcommand{\contentsname}{Indice}%imposto il nome dell'indice
\renewcommand\familydefault{\sfdefault}
%-------------------INIZIO DOCUMENTO--------------
\begin{document}

\newgeometry{left=2cm,right=2cm,bottom=2.1cm,top=2.1cm}
\begin{titlepage}
	\vspace*{.5cm}

	\vspace{2cm}
	{
		\centering
		{\bfseries\huge \Title\par}
		\bigbreak
		{\bfseries\Large \Subtitle\par}
		\bigbreak
		{\bfseries\large \Author\par}
		\bigbreak
		{\Date\;-\;\Version\par}
		\vfill

		\begin{center}
			\begin{tikzpicture}
				\clip (0,0) circle (2cm) node {\includegraphics[width=4cm]{logo.jpg}};
			\end{tikzpicture}
		\end{center}
	}

	\vfill

\end{titlepage}

\restoregeometry






















\newpage

\pagestyle{fancy}
\fancyhead{}
\lhead{
	\begin{tikzpicture}
		\clip (0,0) circle (0.5cm);
		\node at (0,0) {\includegraphics[width=1cm]{./../logo/logo.png}};
	\end{tikzpicture}%
}
\chead{\vspace{\fill}\Title\vspace{\fill}}
\rhead{\vspace{\fill}\Version\vspace{\fill}}




%-----------tabella versioni-----------%
\begin{table}[!h]
	\caption{Versioni}
	\begin{center}
		\begin{tabular}{ c c c c c c }
			\hline \\[-2ex]
			Ver. & Data & Autore & Descrizione \\
			\\[-2ex] \hline \\[-1.5ex]
			1.0 & 08/03/2024 & Leonardo Baldo& prima stesura \\
			1.1 & 15/03/2024 & Valerio Occhinegro& rifinitura  \\
			\\[-1.5ex] \hline
		\end{tabular}
	\end{center}
\end{table}
%---------------------------------------%
\newpage

\tableofcontents

\newpage

\section{Registro presenze}

\paragraph{Dettagli}

\begin{itemize}
	\item Sede della riunione: on-line sulla piattaforma meet
	\item Orario di inizio: 15:00
	\item Orario di fine: 15:30
\end{itemize}

\paragraph{Componenti del gruppo}

\begin{flushleft}
\begin{table}[!h]
\begin{tabular}{ |c|c|c| } 
	\hline
	Partecipanti & Presenza \\
	\hline 
	Leonardo Baldo 		 & 0:30 h \\ 
	Antonio Benetazzo 	 & 0:30 h \\
	Elena Ferro 		 & 0:30 h \\
	Davide Malgarise 	 & 0:30 h \\
	Valerio Occhinegro 	 & 0:30 h \\
	Raul Seganfreddo 	 & 0:30 h \\
	Matteo Tiozzo 		 & 0:30 h \\ 
	\hline
\end{tabular}
\end{table}
\end{flushleft}

\paragraph{Componenti dell'azienda}

\begin{itemize}
	\item Fabio Pallaro
	\item Daniele Zorzi
	\item Andrea Dorigo
\end{itemize}

\newpage

\section{Verbale}
\begin{flushleft}

All’incontro on-line sulla piattaforma meet con l’azienda SyncLab sono presenti, 
come riportato in tabella, i 7 membri del gruppo e tre rappresentanti dell’azienda.

Il colloquio ha inizio alle ore 15:00 ed ha come obiettivo la conoscenza approfondita del progetto proposto dall’azienda.

Gli esponenti di SyncLab chiedono le motivazioni che hanno portato il team alla scelta del capitolato in questione.  
Il gruppo afferma che la decisione è stata dettata dal desiderio di poter  imparare ad utilizzare alcune tecnologie, 
necessarie alla creazione della piattaforma SyncCity e utili per un futuro lavorativo, che non sono state affrontate 
nel corso di laurea, come ad esempio la gestione di ingenti quantità di dati e la creazione di database non relazionali.

I rappresentanti dell’azienda consigliano di confrontarsi con i gruppi che in passato hanno già affrontato tale progetto per avere un 
riscontro relativo alle possibili difficoltà e alle modalità di attuazione.

7Last chiede, successivamente, se la quantità dei dati da gestire sarà paragonabile a quella effettivamente prodotta dai sensori e, dunque, 
se sarà necessario lavorare con grandi quantità di dati o basterà una numero “dimostrativo”. Viene spiegato che l’obiettivo del programma 
è proprio l’analisi di un vasto numero di data, ma per questo progetto si tratterà di una simulazione, sarà dunque suffieciente 
analizzarne un numero simbolico.

L’azienda, a questo punto, richiede la data del kick-off per potersi organizzare e dare avvio  nel migliore dei modi al lavoro; 
il periodo indicativo di inizio è successivo al giorno 22/03/2024.

La parte successiva dell’incontro è dedicata alla richiesta di alcune delucidazioni e curiosiosità da parte del gruppo, 
riguardanti le applicazioni effettivamente realizzate da SyncLab.

Viene quindi chiesto se l’azienda ha già dei riscontri effettivi di interesse da alcune città per l’utilizzo e l’integrazione di questa 
tecnologia; SyncLab si è già occupata di gestire la funzione di smart-parking (monitoraggio e gestione parcheggi) tramite una tecnologia 
affine e attualmente sta sviluppando un sistema che verte sulla sincronizzazione semaforica.

Il gruppo vorrebbe capire a quali aspetti principali dare più risalto e come utilizzare in maniera simbiotica e contemporanea le 
varie tecnologie; gli esponenti dell’azienda affermano che vi sarà un maggiore impegno da dedicare all’integrazione di esse tra loro, 
alla creazione dell’architettura (networking, containerizzazione) e alla configurazione (devops) dei prodotti, 
piuttosto che di *programmazione* vera e propria (fatta eccezione per lo script python). 

Gli strumenti consigliati per raggiungere l’obiettivo ultimo del progetto sono ClickUp 
(perchè di più semplice utilizzo rispetto a Jira), Discord e Trello.

L’incontro giunge dunque al termine alle ore 15:30.

\end{flushleft}

\begin{table}[b]
	\begin{tabular}{@{}p{.5in}p{4in}@{}}
		Data:  & \hrulefill \\
			   &     		\\
			   &     		\\
		Firma: & \hrulefill \\
	\end{tabular}
	\end{table}

\end{document}

























