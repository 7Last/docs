\documentclass[italian,12pt]{article}

%--------------variabili------------------%
\def\Title{Norme di Progetto}
\def\Author{7Last}
\def\Version{v0.2}
%-----------------------------------------%


\usepackage[left=2cm, right=2cm, bottom=3cm, top=3cm]{geometry}
\usepackage{fancyhdr}
\usepackage{graphicx}
\graphicspath{ {../../logo/} }
\usepackage{href-ul}
\usepackage{tikz}
\usepackage{tgadventor}
\usepackage[useregional=numeric,showseconds=true,showzone=false]{datetime2}
\usepackage{caption}
\usepackage{longtable}
\usepackage{xcolor}




\linespread{1.2}
\captionsetup[table]{labelformat=empty}
\geometry{headsep=1.5cm}

\renewcommand{\contentsname}{Indice}
\renewcommand\familydefault{\sfdefault}

\let\oldthepage\thepage
\renewcommand{\thepage}{\sffamily\oldthepage}


\begin{document}

\newgeometry{left=2cm,right=2cm,bottom=2.1cm,top=2.1cm}
\begin{titlepage}
	\vspace*{.5cm}

	\vspace{2cm}
	{
		\centering
		{\bfseries\huge \Title\par}
		\bigbreak
		{\bfseries\Large \Subtitle\par}
		\bigbreak
		{\bfseries\large \Author\par}
		\bigbreak
		{\Date\;-\;\Version\par}
		\vfill

		\begin{center}
			\begin{tikzpicture}
				\clip (0,0) circle (2cm) node {\includegraphics[width=4cm]{logo.jpg}};
			\end{tikzpicture}
		\end{center}
	}

	\vfill

\end{titlepage}

\restoregeometry






















\newpage

\pagestyle{fancy}
\fancyhead{}
\lhead{
	\begin{tikzpicture}
		\clip (0,0) circle (0.5cm);
		\node at (0,0) {\includegraphics[width=1cm]{./../logo/logo.png}};
	\end{tikzpicture}%
}
\chead{\vspace{\fill}\Title\vspace{\fill}}
\rhead{\vspace{\fill}\Version\vspace{\fill}}


\begin{table}[!h]
	\caption{Versioni}
	\footnotesize
	\begin{center}
		\begin{tabular}{ l l l l l }
			\hline                                                             \\[-2ex]
			Ver. & Data       & Redattore   & Verificatore   & Descrizione     \\
			\\[-2ex] \hline \\[-1.5ex]
			1.0  & 2024-05-12 & Elena Ferro & Leonardo Baldo & Stesura verbale \\
			\\[-1.5ex] \hline
		\end{tabular}
	\end{center}
\end{table}

\newpage

\tableofcontents

\newpage

\section{Dettagli della riunione}


\textbf{Sede della riunione}: Piattaforma Discord\\
\textbf{Orario di inizio}: 15:30\\
\textbf{Orario di fine}: 16:30\\


\begin{flushleft}
	\begin{table}[!h]
		\begin{tabular}{ |l|l|l| }
			\hline
			\textbf{Partecipante} & \textbf{Ruolo} & \textbf{Presenza} \\
			\hline
			Antonio Benetazzo     &                & Presente          \\
			Davide Malgarise      &                & Presente          \\
			Elena Ferro           & Redattore      & Presente          \\
			Leonardo Baldo        & Verificatore   & Presente          \\
			Matteo Tiozzo         &                & Assente           \\
			Raul Seganfreddo      & Amministratore & Presente          \\
			Valerio Occhinegro    &                & Presente          \\
			\hline
		\end{tabular}
	\end{table}
\end{flushleft}

\section{Ordine del giorno}
\begin{itemize}
	\item Cosa è stato fatto fino ad ora
	\item Cosa rimane da fare
	\item Varie ed eventuali
	\item Decisioni prese e conclusioni
\end{itemize}

\newpage

\section{Verbale}

\subsection{Cosa è stato fatto fino ad ora}
È stato fatto un riepilogo di quanto completato fino ad ora.
\begin{itemize}
	\item \textbf{Verbali} \\
	      Revisionato e approvato il verbale della riunione esterna del 2024-05-06.
	\item \textbf{Avanzamento Documentazione} \\
	      Completata la stesura della prima versione dei documenti \textit{Analisi dei Requisiti} e
	      \textit{Norme di Progetto}. Sono stati discussi assieme alcuni dubbi
	      riguardanti il cruscotto di valutazione della qualità all'interno del \textit{Piano di Qualifica}.
	      È stato inoltre aggiornato il \textit{Piano di Progetto} con i dati relativi al secondo sprint.
	\item \textbf{\textit{Proof of Concept}}\\
	      Il \textit{Proof of Concept} giunge ad una prima versione stabile ed è stato presentato al proponente,
	      la quale si ritiene soddisfatta del lavoro svolto e ha fornito dei feedback per il miglioramento del prodotto.
\end{itemize}

\subsection{Cosa rimane da fare}
Per questo sprint ci sono ancora alcune attività da completare:
\begin{itemize}
	\item \textbf{Documentazione}: completare il \textit{Piano di Qualifica}, proseguire con la stesura del \textit{Piano di Progetto}.
	\item \textbf{Proof of Concept}: raffinare la visualizzazione dei dati su \textit{Grafana}, secondo quanto scritto nel verbale esterno del 2024-05-06.
\end{itemize}

\subsection{Varie ed eventuali}
Sono sorti alcuni dubbi e difficoltà riguardanti:
\begin{itemize}
	\item la configurazione del progetto su sistemi operativi diversi;
	\item la gestione del tempo e dei compiti da svolgere con sprint di durata minore.
\end{itemize}

\subsection{Decisioni prese e conclusioni}
Si è infine discusso il calendario dei lavori definito in fase di candidatura e si è stabilito di ripianificarlo con le seguenti scadenze:
% table:
\begin{table}[!h]
	\begin{center}
		\begin{tabular}{ | l | l | }
			\hline
			\textbf{Data} & \textbf{Scadenza} \\
			\hline
			2024-05-22    & RTB               \\
			2024-07-24    & PB                \\
			2024-08-28    & CA                \\
			\hline
		\end{tabular}
	\end{center}
\end{table}


\end{document}
