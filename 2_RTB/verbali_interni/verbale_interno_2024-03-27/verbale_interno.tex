\documentclass[italian,12pt]{article} %tipo di documento

%--------------variabili------------------%
\def\Title{Norme di Progetto}
\def\Author{7Last}
\def\Version{v0.2}
%-----------------------------------------%


\usepackage[left=2cm, right=2cm, bottom=3cm, top=3cm]{geometry}
\usepackage{fancyhdr}
\usepackage{graphicx}
\graphicspath{ {../../logo/} }
\usepackage{href-ul}
\usepackage{tikz}
\usepackage{tgadventor}
\usepackage[useregional=numeric,showseconds=true,showzone=false]{datetime2}
\usepackage{caption}
\usepackage{longtable}
\usepackage{xcolor}




\linespread{1.2}
\captionsetup[table]{labelformat=empty}
\geometry{headsep=1.5cm}

\renewcommand{\contentsname}{Indice}%imposto il nome dell'indice
\renewcommand\familydefault{\sfdefault}

%-------------------INIZIO DOCUMENTO--------------
\begin{document}

\newgeometry{left=2cm,right=2cm,bottom=2.1cm,top=2.1cm}
\begin{titlepage}
	\vspace*{.5cm}

	\vspace{2cm}
	{
		\centering
		{\bfseries\huge \Title\par}
		\bigbreak
		{\bfseries\Large \Subtitle\par}
		\bigbreak
		{\bfseries\large \Author\par}
		\bigbreak
		{\Date\;-\;\Version\par}
		\vfill

		\begin{center}
			\begin{tikzpicture}
				\clip (0,0) circle (2cm) node {\includegraphics[width=4cm]{logo.jpg}};
			\end{tikzpicture}
		\end{center}
	}

	\vfill

\end{titlepage}

\restoregeometry






















\newpage

\pagestyle{fancy}
\fancyhead{}
\lhead{
	\begin{tikzpicture}
		\clip (0,0) circle (0.5cm);
		\node at (0,0) {\includegraphics[width=1cm]{./../logo/logo.png}};
	\end{tikzpicture}%
}
\chead{\vspace{\fill}\Title\vspace{\fill}}
\rhead{\vspace{\fill}\Version\vspace{\fill}}


%-----------tabella revisioni-----------%
\begin{table}[!h]
	\caption{Versioni}
	\begin{center}
		\begin{tabular}{ c c c p{9cm} }
			\hline                                                                                 \\[-2ex]
			Ver. & Data       & Autore             & Descrizione                                   \\
			\\[-2ex] \hline \\[-1.5ex]
			0.1  & 28/03/2024 & Antonio Benetazzo & Prima stesura                                  \\
			\\[-1.5ex] \hline
		\end{tabular}
	\end{center}
\end{table}
%---------------------------------------%

\newpage

\tableofcontents

\newpage

\section{Dettagli della riunione}


\textbf{Sede della riunione}: piattaforma Discord\\
\textbf{Orario di inizio}: 21:00\\
\textbf{Orario di fine}: 23:00\\


\begin{flushleft}
	\begin{table}[!h]
	\begin{tabular}{ |l|l|l| } 
		\hline
		\textbf{Componente} & \textbf{Ruolo} & \textbf{Presenza} \\
		\hline 
		Antonio Benetazzo 	& Redattore           & Presente \\
		Davide Malgarise 	& Verificatore        & Assente  \\
		Elena Ferro 		& Redattore           & Presente \\
		Leonardo Baldo 		& Verificatore        & Presente \\ 
		Matteo Tiozzo 		& Responsabile        & Presente \\ 
		Raul Seganfreddo 	& Redattore           & Presente \\
		Valerio Occhinegro 	& Amministratore      & Presente \\
		\hline
	\end{tabular}
	\end{table}
\end{flushleft}

\section{Ordine del giorno}
\begin{itemize}
	\itemsep0em
	\item \textbf{}Approvazione della candidatura
	\item \textbf{}Prossimi step e documentazione da produrre
	\item \textbf{}Aggiornamento Way of Working
	\item \textbf{}Modifiche al template per la documentazione
	\item \textbf{}Varie ed eventuali
\end{itemize}

\newpage

\section{Verbale}

\subsection{Approvazione della candidatura}
Prendiamo atto dell'approvazione della candidatura da noi presentata per il progetto 
\textit{SyncCity} da parte del professor Vardanega. \\
Nella valutazione viene evidenziato come la pianificazione sia ancora molto "ingenua". 
Discutiamo assieme su tali osservazioni e ci riserviamo di rivalutare tale 
pianificazione in occasione della pubblicazione della prossima documentazione, 
prevista per il prossimo periodo. Ci viene inoltre suggerito dal professore di 
non tardare nell'impostazione del documento \textit{Piano di Progetto} di cui 
discuteremo in uno dei punti successivi. \\
Siamo comunque soddisfatti del risultato raggiunto e ci impegniamo a migliorare la 
pianificazione per le prossime scadenze. \\
Provvediamo anche a comunicare via e-mail a \textit{Sync Lab} l'approvazione della 
candidatura e alle altre due aziende proponenti gli altri capitolati la nostra 
rinuncia.

\subsection{Prossimi step e documentazione da produrre}
Discutiamo assieme dei prossimi step: le successive revisioni saranno l'RTB 
(Requirements and Technology Baseline), il PB (Product Baseline) e l'eventuale 
CA (Customer Acceptance). \\
Verifichiamo assieme la documentazione necessaria per la prossima revisione, l'RTB 
(e che ci accompagneranno per tutto il resto del progetto): 
\textit{Analisi dei Requisiti}, \textit{Piano di Progetto}, 
\textit{Piano di Qualifica}, \textit{Norme di Progetto}, \textit{Glossario}, 
oltre alla \textit{Lettera di Presentazione}, al POC (Proof of Concept) e a tutti 
i verbali interni ed esterni. Ci impegniamo nel prossimo periodo a studiare 
questi documenti per comprenderne la struttura e il contenuto attesi e ad iniziare 
a lavorarci il prima possibile. \\
In particolare, in merito al \textit{Glossario} riteniamo sarebbe utile trovare 
un modo per automatizzare la messa in evidenza di ciascun vocabolo in tutti i 
documenti in cui appare, in modo che risulti evidente al lettore che si tratta di 
un termine presente nel \textit{Glossario}, ma senza doverlo gestire manualmente, 
cosa che porterebbe inevitabilmente a numerose sviste ed errori. \\
Valutiamo sia opportuno fissare una call con l'azienda per confrontarci anche 
con loro su come affrontare le prossime fasi del progetto.

\subsection{Aggiornamento Way of Working}
Risulta necessario e sempre più urgente anche procedere con l'aggiornamento del 
\textit{Way of Working} per la pianificazione e la gestione del lavoro nelle fasi 
successive. \\
In particolare, nei prossimi giorni valuteremo insieme l'utilizzo di \textit{ClickUp} 
come strumento di gestione del lavoro e delle scadenze, come suggeritoci 
dall'azienda proponente. Valuteremo anche eventuali altri strumenti a supporto della 
gestione del progetto. Provvederemo inoltre, come già detto, alla stesura del 
documento \textit{Norme di Progetto}, dove potremo definire meglio le modalità di 
lavoro e le regole da seguire.

\subsection{Modifiche al template per la documentazione}
Decidiamo di apportare alcune modifiche al template per la documentazione. \\
Oltre ad alcuni accorgimenti grafici, valutiamo sia il caso di dividerlo 
in tre differenti template: uno per la documentazione in genere, uno specifico per 
i verbali interni (che riporteranno anche la data, il registro delle presenze, 
l'ordine del giorno e una struttura definita per il verbale) e uno specifico per 
i verbali esterni (che conterrà in aggiunta la struttura necessaria per la firma 
da parte dell'azienda proponente).

\subsection{Conclusioni}
La riunione termina, ci diamo appuntamento alla prossima settimana, riteniamo sia 
importante trovarci possibilmente almeno una volta a settimana per valutare assieme 
l'andamento del progetto e per discutere eventuali problematiche o dubbi.

\end{document}