\documentclass[italian,12pt]{article}

%--------------variabili------------------%
\def\Title{Norme di Progetto}
\def\Author{7Last}
\def\Version{v0.2}
%-----------------------------------------%


\usepackage[left=2cm, right=2cm, bottom=3cm, top=3cm]{geometry}
\usepackage{fancyhdr}
\usepackage{graphicx}
\graphicspath{ {../../logo/} }
\usepackage{href-ul}
\usepackage{tikz}
\usepackage{tgadventor}
\usepackage[useregional=numeric,showseconds=true,showzone=false]{datetime2}
\usepackage{caption}
\usepackage{longtable}
\usepackage{xcolor}




\linespread{1.2}
\captionsetup[table]{labelformat=empty}
\geometry{headsep=1.5cm}

\renewcommand{\contentsname}{Indice}
\renewcommand\familydefault{\sfdefault}

\begin{document}

\newgeometry{left=2cm,right=2cm,bottom=2.1cm,top=2.1cm}
\begin{titlepage}
	\vspace*{.5cm}

	\vspace{2cm}
	{
		\centering
		{\bfseries\huge \Title\par}
		\bigbreak
		{\bfseries\Large \Subtitle\par}
		\bigbreak
		{\bfseries\large \Author\par}
		\bigbreak
		{\Date\;-\;\Version\par}
		\vfill

		\begin{center}
			\begin{tikzpicture}
				\clip (0,0) circle (2cm) node {\includegraphics[width=4cm]{logo.jpg}};
			\end{tikzpicture}
		\end{center}
	}

	\vfill

\end{titlepage}

\restoregeometry






















\newpage

\pagestyle{fancy}
\fancyhead{}
\lhead{
	\begin{tikzpicture}
		\clip (0,0) circle (0.5cm);
		\node at (0,0) {\includegraphics[width=1cm]{./../logo/logo.png}};
	\end{tikzpicture}%
}
\chead{\vspace{\fill}\Title\vspace{\fill}}
\rhead{\vspace{\fill}\Version\vspace{\fill}}


\begin{table}[!h]
	\caption{Versioni}
	\footnotesize
	\begin{center}
		\begin{tabular}{ l l l l l }
			\hline \\[-2ex]
			Ver. & Data       & Redattore      & Verificatore      & Descrizione     \\
			\\[-2ex] \hline \\[-1.5ex]
			1.0  & 2024-04-24 & Leonardo Baldo & Antonio Benetazzo & Stesura verbale \\
			\\[-1.5ex] \hline
		\end{tabular}
	\end{center}
\end{table}

\newpage

\tableofcontents

\newpage

\section{Dettagli della riunione}


\textbf{Sede della riunione}: Piattaforma Discord\\
\textbf{Orario di inizio}: 15:00\\
\textbf{Orario di fine}: 16:00\\


\begin{flushleft}
	\begin{table}[!h]
	\begin{tabular}{ |l|l|l| } 
		\hline
		\textbf{Partecipante} & \textbf{Ruolo}       & \textbf{Presenza} \\
		\hline 
		Antonio Benetazzo     & Verificatore         & Presente          \\
		Davide Malgarise      &                      & Presente          \\
		Elena Ferro           &                      & Presente          \\
		Leonardo Baldo        & Redattore            & Presente          \\
		Matteo Tiozzo         &                      & Presente          \\
		Raul Seganfreddo      & Amministratore       & Presente          \\
		Valerio Occhinegro    &                      & Presente          \\
		\hline
	\end{tabular}
	\end{table}
\end{flushleft}

\section{Ordine del giorno}
\begin{itemize}
	\item Cosa è stato fatto fino ad ora
	\item Cosa rimane da fare
	\item Varie ed eventuali
	\item Decisioni prese e conclusioni
\end{itemize}

\newpage

\section{Verbale}

\subsection{Cosa è stato fatto fino ad ora}
Abbiamo fatto un riepilogo di quanto completato fino ad ora.
\begin{itemize}
	\item \textbf{Verbali} \\
	Redatto, revisionato e approvato il verbale della riunione esterna del 2024/04/22.
	\item \textbf{Avanzamento Documentazione} \\
	Abbiamo effettuato una prima stesura della \href{https://7last.github.io/docs/rtb/documentazione-interna/glossario#analisi-dei-requisiti}{\textit{Analisi dei Requisiti}\textsubscript{G}}, e la continuazione del \href{https://7last.github.io/docs/rtb/documentazione-interna/glossario#piano-di-progetto}{\textit{Piano di Progetto}\textsubscript{G}}, del \href{https://7last.github.io/docs/rtb/documentazione-interna/glossario#piano-di-qualifica}{\textit{Piano di Qualifica}\textsubscript{G}}, delle \href{https://7last.github.io/docs/rtb/documentazione-interna/glossario#norme-di-progetto}{\textit{Norme di Progetto}\textsubscript{G}}, e del \href{https://7last.github.io/docs/rtb/documentazione-interna/glossario#glossario}{\textit{Glossario}\textsubscript{G}}.
	\item \textbf{Richieste dell'azienda} \\
	Abbiamo iniziato lo studio e l'analisi dei prossimi strumenti da utilizzare, ovvero \textbf{ClickHouse} e \textbf{Grafana}. Al fine di migliorare la persistenza dati su \textbf{ClickHouse} e integrare \textbf{Grafana}. \\
	Prodotto inoltre il documento con vantaggi e svantaggi tra \textbf{Apache Kafka} e \textbf{Redpanda}.
\end{itemize}

\subsection{Cosa rimane da fare}
Per questo sprint ci sono ancora alcune attività da completare:
\begin{itemize}
	\item finire la stesure delle \href{https://7last.github.io/docs/rtb/documentazione-interna/glossario#analisi-dei-requisiti}{\textit{Analisi dei Requisiti}\textsubscript{G}};
	\item avanzamento e aggiornamento del \href{https://7last.github.io/docs/rtb/documentazione-interna/glossario#piano-di-progetto}{\textit{Piano di Progetto}\textsubscript{G}}, del \href{https://7last.github.io/docs/rtb/documentazione-interna/glossario#piano-di-qualifica}{\textit{Piano di Qualifica}\textsubscript{G}}, e della \href{https://7last.github.io/docs/rtb/documentazione-interna/glossario#analisi-dei-requisiti}{\textit{Analisi dei Requisiti}\textsubscript{G}}, e del \href{https://7last.github.io/docs/rtb/documentazione-interna/glossario#glossario}{\textit{Glossario}\textsubscript{G}};
	\item effettuare la stesura del verbale della riunione in corso.
\end{itemize}

\subsection{Varie ed eventuali}
Riguardo al documento di valutazione dei 2 strumenti, l'azienda ha deciso che faremo delle considerazioni, durante il prossimo SAL, previsto per il giorno 2024/05/06. In modo da parlarne assieme e esporre ulteriormente il nostro punto di vista.

\subsection{Decisioni prese e conclusioni}
Concludiamo la riunione impegnandoci a completare le attività rimanenti entro il termine del secondo sprint e ci diamo appuntamento per il prossimo incontro interno previsto per il 2024/04/29.

\end{document}