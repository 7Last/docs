\documentclass[italian,12pt]{article}

%--------------variabili------------------%
\def\Title{Norme di Progetto}
\def\Author{7Last}
\def\Version{v0.2}
%-----------------------------------------%


\usepackage[left=2cm, right=2cm, bottom=3cm, top=3cm]{geometry}
\usepackage{fancyhdr}
\usepackage{graphicx}
\graphicspath{ {../../logo/} }
\usepackage{href-ul}
\usepackage{tikz}
\usepackage{tgadventor}
\usepackage[useregional=numeric,showseconds=true,showzone=false]{datetime2}
\usepackage{caption}
\usepackage{longtable}
\usepackage{xcolor}




\linespread{1.2}
\captionsetup[table]{labelformat=empty}
\geometry{headsep=1.5cm}

\renewcommand{\contentsname}{Indice}
\renewcommand\familydefault{\sfdefault}

\begin{document}

\newgeometry{left=2cm,right=2cm,bottom=2.1cm,top=2.1cm}
\begin{titlepage}
	\vspace*{.5cm}

	\vspace{2cm}
	{
		\centering
		{\bfseries\huge \Title\par}
		\bigbreak
		{\bfseries\Large \Subtitle\par}
		\bigbreak
		{\bfseries\large \Author\par}
		\bigbreak
		{\Date\;-\;\Version\par}
		\vfill

		\begin{center}
			\begin{tikzpicture}
				\clip (0,0) circle (2cm) node {\includegraphics[width=4cm]{logo.jpg}};
			\end{tikzpicture}
		\end{center}
	}

	\vfill

\end{titlepage}

\restoregeometry






















\newpage

\pagestyle{fancy}
\fancyhead{}
\lhead{
	\begin{tikzpicture}
		\clip (0,0) circle (0.5cm);
		\node at (0,0) {\includegraphics[width=1cm]{./../logo/logo.png}};
	\end{tikzpicture}%
}
\chead{\vspace{\fill}\Title\vspace{\fill}}
\rhead{\vspace{\fill}\Version\vspace{\fill}}


\begin{table}[!h]
	\caption{Versioni}
	\footnotesize
	\begin{center}
		\begin{tabular}{ l l l l l }
			\hline                                                                              \\[-2ex]
			Ver. & Data       & Redattore          & Verificatore       & Descrizione           \\
			\\[-2ex] \hline \\[-1.5ex]
			1.0  & 10/04/2024 & Antonio Benetazzo    & Valerio Occhinegro & Approvazione template \\
			\\[-1.5ex] \hline
		\end{tabular}
	\end{center}
\end{table}

\newpage

\tableofcontents

\newpage

\section{Dettagli della riunione}


\textbf{Sede della riunione}: Piattaforma Discord\\
\textbf{Orario di inizio}: 15:00\\
\textbf{Orario di fine}: 16:00\\


\begin{flushleft}
	\begin{table}[!h]
	\begin{tabular}{ |l|l|l| } 
		\hline
		\textbf{Partecipante} & \textbf{Ruolo}       & \textbf{Presenza} \\
		\hline 
		Antonio Benetazzo     & Redattore            & Presente          \\
		Davide Malgarise      &                      & Presente          \\
		Elena Ferro           &                      & Assente           \\
		Leonardo Baldo        &                      & Presente          \\
		Matteo Tiozzo         & Amministratore       & Presente          \\
		Raul Seganfreddo      &                      & Presente          \\
		Valerio Occhinegro    & Verificatore         & Presente          \\
		\hline
	\end{tabular}
	\end{table}
\end{flushleft}

\section{Ordine del giorno}
\subsection{Cosa è stato fatto fino ad ora}
\subsection{Cosa rimane da fare}
\subsection{Problemi riscontrati}
\subsection{Dubbi emersi}
\subsection{Proposta di modifica al template dei verbali}
\subsection{Revisione e approvazione verbali interno ed esterno 3 aprile}
\subsection{Modello da utilizzare per il documento Piano di Progetto}
\subsection{Pubblicazione nel branch develop della documentazione}
\subsection{Varie ed eventuali}

\newpage

\section{Verbale}

\subsection{Cosa è stato fatto fino ad ora}
Abbiamo fatto un riepilogo di quanto completato fino ad ora.
\textbf{Verbali}: sono stati redatti i verbali delle riunioni interne e esterne del 3 aprile. Si procede con la revisione e l'approvazione.
\textbf{Script Glossario}: è stato completato lo sviluppo dello script sviluppato in \textit{Python} per contrassegnare in automatico nei vari documenti le parole presenti nel \textit{Glossario}. Nei porrimi giorni si provvederà a testare lo script in modo da verificarne il corretto funzionamento.
\textbf{Simulatore di dati}: è stato implementato il simulatore che verrà utilizzato nel \textit{POC} per generare dati casuali ma verosimili. Attualmente produce due tipi di dato: \textbf{temperatura}, valore generato tramite una sinusoide che simula l'escursione termica giornaliera in modo parametrico rispetto al mese del timestamp, per cui sarà in grado di generare dati in modo coerente rispetto alla stagione in corso; \textbf{traffico}, l'idea è quella di misurare il numero di veicoli che passano davanti al sensore in un certo periodo di tempo e la loro velocità media (così si dovrebbe riuscire a stimare più o meno il traffico), questo avviene mediante l'utilizzo di una gaussiana multimodale (in altri termini, è una funzione abbastanza pianeggiante tranne in determinati punti dove si hanno dei picchi, che rappresentano i picchi di traffico di mattina/sera).
\textbf{Struttura del documento Analisi dei Requisiti}: è stata definita la struttura del documento \textit{Analisi dei Requisiti}, in modo da avere un'idea chiara di come verrà strutturato il documento.
\textbf{}

\subsection{Argomento 2}
bla bla bla

\subsection{Decisioni prese e conclusioni}
bla bla bla

\end{document}