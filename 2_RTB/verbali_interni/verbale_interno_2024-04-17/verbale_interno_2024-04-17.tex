\documentclass[italian,12pt]{article}

%--------------variabili------------------%
\def\Title{Norme di Progetto}
\def\Author{7Last}
\def\Version{v0.2}
%-----------------------------------------%


\usepackage[left=2cm, right=2cm, bottom=3cm, top=3cm]{geometry}
\usepackage{fancyhdr}
\usepackage{graphicx}
\graphicspath{ {../../logo/} }
\usepackage{href-ul}
\usepackage{tikz}
\usepackage{tgadventor}
\usepackage[useregional=numeric,showseconds=true,showzone=false]{datetime2}
\usepackage{caption}
\usepackage{longtable}
\usepackage{xcolor}




\linespread{1.2}
\captionsetup[table]{labelformat=empty}
\geometry{headsep=1.5cm}

\renewcommand{\contentsname}{Indice}
\renewcommand\familydefault{\sfdefault}

\begin{document}

\newgeometry{left=2cm,right=2cm,bottom=2.1cm,top=2.1cm}
\begin{titlepage}
	\vspace*{.5cm}

	\vspace{2cm}
	{
		\centering
		{\bfseries\huge \Title\par}
		\bigbreak
		{\bfseries\Large \Subtitle\par}
		\bigbreak
		{\bfseries\large \Author\par}
		\bigbreak
		{\Date\;-\;\Version\par}
		\vfill

		\begin{center}
			\begin{tikzpicture}
				\clip (0,0) circle (2cm) node {\includegraphics[width=4cm]{logo.jpg}};
			\end{tikzpicture}
		\end{center}
	}

	\vfill

\end{titlepage}

\restoregeometry






















\newpage

\pagestyle{fancy}
\fancyhead{}
\lhead{
	\begin{tikzpicture}
		\clip (0,0) circle (0.5cm);
		\node at (0,0) {\includegraphics[width=1cm]{./../logo/logo.png}};
	\end{tikzpicture}%
}
\chead{\vspace{\fill}\Title\vspace{\fill}}
\rhead{\vspace{\fill}\Version\vspace{\fill}}


\begin{table}[!h]
	\caption{Versioni}
	\footnotesize
	\begin{center}
		\begin{tabular}{ l l l l l }
			\hline                                                                       \\[-2ex]
			Ver. & Data       & Redattore         & Verificatore       & Descrizione     \\
			\\[-2ex] \hline \\[-1.5ex]
			1.0  & 2024-04-17 & Antonio Benetazzo & Valerio Occhinegro & Stesura verbale \\
			\\[-1.5ex] \hline
		\end{tabular}
	\end{center}
\end{table}

\newpage

\tableofcontents

\newpage

\section{Dettagli della riunione}


\textbf{Sede della riunione}: Piattaforma Discord\\
\textbf{Orario di inizio}: 15:00\\
\textbf{Orario di fine}: 16:00\\


\begin{flushleft}
	\begin{table}[!h]
		\begin{tabular}{ |l|l|l| }
			\hline
			\textbf{Partecipante} & \textbf{Ruolo} & \textbf{Presenza} \\
			\hline
			Antonio Benetazzo     & Redattore      & Presente          \\
			Davide Malgarise      &                & Presente          \\
			Elena Ferro           &                & Presente          \\
			Leonardo Baldo        &                & Presente          \\
			Matteo Tiozzo         & Amministratore & Presente          \\
			Raul Seganfreddo      &                & Presente          \\
			Valerio Occhinegro    & Verificatore   & Presente          \\
			\hline
		\end{tabular}
	\end{table}
\end{flushleft}

\section{Ordine del giorno}
\subsection{Cosa è stato fatto fino ad ora}
\subsection{Cosa rimane da fare}
\subsection{Revisione pianificazione scadenze}
\subsection{Varie ed eventuali}

\newpage

\section{Verbale}

\subsection{Cosa è stato fatto fino ad ora}
Abbiamo fatto un riepilogo di quanto completato fino ad ora.
\begin{itemize}
	\item \textbf{Verbali} \\
	      Redatto, revisionato e approvato il verbale della riunione interna del 10 aprile.
	\item \textbf{Avanzamento Documentazione} \\
	      Abbiamo effettuato una prima stesura e una prima revisione del \textit{Piano di Progetto}, del \textit{Piano di Qualifica}, delle \textit{Norme di Progetto}, e del \textit{Glossario}.
	\item \textbf{Script per \textit{Glossario}} \\
	      Abbiamo completato lo sviluppo dello script in \textit{Python} per automatizzare il processo di marcatura dei termini all'interno dei documenti che sono presenti nel \textit{Glossario}.
	\item \textbf{Richieste dell'azienda} \\
	      Abbiamo sviluppato una prima versione dello script in \textit{Python} che verrà utilizzato per generare dati simulati. Rispetto alle richieste dell'azienda, ovvero di generare in questa fase una sola tipologia di dati, lo script è in grado di generare due tipi di dato, uno relativo all'intensità del traffico e uno relativo alla temperatura ambientale. Abbiamo anche effettuato una prima configurazione di \textit{Docker} per l'ambiente di sviluppo e di \textit{Kafka} (ritenuto dall'azienda opzionale per questo sprint) dove vengono inviati i dati dallo script. Abbiamo previsto due possibilità: una versione che utilizza \textit{Apache Kafka}, come suggerito dal proponente, e una seconda versione con \textit{Redpanda}, un tool alternativo compatibile con \textit{Kafka} che sembra offrire delle performance migliori.
\end{itemize}

\subsection{Cosa rimane da fare}
Per questo sprint ci sono ancora alcune ultime attività da completare:
\begin{itemize}
	\item effettuare alcune modifiche ai documenti e completare la loro prima stesura per poterli pubblicare nel branch \texttt{develop};
	\item includere l'esecuzione dello script per il \textit{Glossario} all'interno del workflow di \textit{GitHub};
	\item effettuare la stesura del verbale della riunione in corso;
	\item cominciare l'\textit{Analisi dei Requisiti}.
\end{itemize}

\subsection{Revisione pianificazione scadenze}
Si è discusso della possibilità di rivedere la pianificazione delle scadenze effettuata durante la candidatura al capitolato. Molti membri del gruppo non hanno altri esami da sostenere che porterebbero via tempo, potrebbero quindi dedicarne di più per questo progetto. Al tempo stesso, però, non vorremmo sottovalutare l'entità del lavoro da svolgere, e ritrovarci magari più avanti a dover posticipare delle scadenze precedentemente anticipate. Teniamo quindi in considerazione la possibilità di rivedere la pianificazione, con l'impegno da parte di tutti di pensarci più approfonditamente e di discuterne in seguito avendo modo di effettuare una più attenta valutazione.

\subsection{Varie ed eventuali}
In occasione del primo SAL con l'azienda proponente previsto per venerdì 19 aprile avremo modo di chiedere chiarimenti e consigli sulle metriche da adottare per il controllo di qualità nel corso del progetto. Valuteremo assieme a loro anche l'inizio dell'\textit{Analisi dei Requisiti}. Infine chiederemo anche una loro opinione in merito alla possibile rivalutazione delle scadenze, per avere un parere sulla possibilità di anticipare la previsione delle varie consegne. \\
Viene deciso l'utilizzo di \textit{Star UML} per la stesura dei diagrammi UML, in quanto è uno strumento gratuito e open source, e permette di esportare i diagrammi in formato \textit{.pdf}.

\subsection{Decisioni prese e conclusioni}
Concludiamo la riunione impegnandoci a completare le attività rimanenti entro il termine del primo sprint e ci diamo appuntamento per il prossimo incontro previsto per venerdì 19 aprile in occasione del primo SAL con l'azienda.

\end{document}
