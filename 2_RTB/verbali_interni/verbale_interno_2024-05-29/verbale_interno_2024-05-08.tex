\documentclass[italian,12pt]{article}

%--------------variabili------------------%
\def\Title{Norme di Progetto}
\def\Author{7Last}
\def\Version{v0.2}
%-----------------------------------------%


\usepackage[left=2cm, right=2cm, bottom=3cm, top=3cm]{geometry}
\usepackage{fancyhdr}
\usepackage{graphicx}
\graphicspath{ {../../logo/} }
\usepackage{href-ul}
\usepackage{tikz}
\usepackage{tgadventor}
\usepackage[useregional=numeric,showseconds=true,showzone=false]{datetime2}
\usepackage{caption}
\usepackage{longtable}
\usepackage{xcolor}




\linespread{1.2}
\captionsetup[table]{labelformat=empty}
\geometry{headsep=1.5cm}

\renewcommand{\contentsname}{Indice}
\renewcommand\familydefault{\sfdefault}

\let\oldthepage\thepage
\renewcommand{\thepage}{\sffamily\oldthepage}


\begin{document}

\newgeometry{left=2cm,right=2cm,bottom=2.1cm,top=2.1cm}
\begin{titlepage}
	\vspace*{.5cm}

	\vspace{2cm}
	{
		\centering
		{\bfseries\huge \Title\par}
		\bigbreak
		{\bfseries\Large \Subtitle\par}
		\bigbreak
		{\bfseries\large \Author\par}
		\bigbreak
		{\Date\;-\;\Version\par}
		\vfill

		\begin{center}
			\begin{tikzpicture}
				\clip (0,0) circle (2cm) node {\includegraphics[width=4cm]{logo.jpg}};
			\end{tikzpicture}
		\end{center}
	}

	\vfill

\end{titlepage}

\restoregeometry






















\newpage

\pagestyle{fancy}
\fancyhead{}
\lhead{
	\begin{tikzpicture}
		\clip (0,0) circle (0.5cm);
		\node at (0,0) {\includegraphics[width=1cm]{./../logo/logo.png}};
	\end{tikzpicture}%
}
\chead{\vspace{\fill}\Title\vspace{\fill}}
\rhead{\vspace{\fill}\Version\vspace{\fill}}


\begin{table}[!h]
	\caption{Versioni}
	\footnotesize
	\begin{center}
		\begin{tabular}{ l l l l l }
			\hline                                                             \\[-2ex]
			Ver. & Data       & Redattore   & Verificatore   & Descrizione     \\
			\\[-2ex] \hline \\[-1.5ex]
			1.0  & 2024-06-02 & Leonardo Baldo & Raul Seganfreddo & Stesura verbale \\
			\\[-1.5ex] \hline
		\end{tabular}
	\end{center}
\end{table}

\newpage

\tableofcontents

\newpage

\section{Dettagli della riunione}


\textbf{Sede della riunione}: Piattaforma Discord\\
\textbf{Orario di inizio}: 15:30\\
\textbf{Orario di fine}: 16:30\\


\begin{flushleft}
	\begin{table}[!h]
		\begin{tabular}{ |l|l|l| }
			\hline
			\textbf{Partecipante} & \textbf{Ruolo} & \textbf{Presenza} \\
			\hline
			Antonio Benetazzo     &                & Presente          \\
			Davide Malgarise      &                & Assente           \\
			Elena Ferro           &                & Presente          \\
			Leonardo Baldo        & Redattore      & Presente          \\
			Matteo Tiozzo         &                & Presente          \\
			Raul Seganfreddo      & Verificatore   & Presente          \\
			Valerio Occhinegro    & Amministratore & Presente          \\
			\hline
		\end{tabular}
	\end{table}
\end{flushleft}

\subsection{Suddivisione dei ruoli per il sesto sprint}
Come prima cosa il gruppo stabilisce la suddivisione dei ruoli per il quarto periodo, cercando di assegnare a ciascun membro un compito non ancora svolto
nei precedenti sprint. Ne risulta la seguente suddivisione:
\begin{itemize}
	\item \textbf{Responsabile}: Elena Ferro
	\item \textbf{Amministratore}: Valerio Occhinegro
	\item \textbf{Analista}: Leonardo Baldo, Matteo Tiozzo
	\item \textbf{Progettista}: Antonio Benetazo, Raul Seganfreddo
	\item \textbf{Verificatore}: Davide Malgarise, Raul Seganfreddo
\end{itemize}
Definiamo, inoltre, i ruoli assegnati per ogni riunione interna od esterna 
di questo quarto sprint (da riportare successivamente nei relativi verbali):
\begin{itemize}
	\item \textbf{Amministratore}: Valerio Occhinegro
	\item \textbf{Redattore}: Leonardo Baldo
	\item \textbf{Verificatore}: Davide Malgarise, Raul Seganfreddo
\end{itemize}

\section{Ordine del giorno}
\begin{itemize}
	\item Cosa è stato fatto fino ad ora
	\item Cosa rimane da fare
	\item Varie ed eventuali
	\item Decisioni prese e conclusioni
\end{itemize}

\newpage

\section{Verbale}

\subsection{Cosa è stato fatto fino ad ora}
È stato fatto un riepilogo di quanto completato fino ad ora.
\begin{itemize}
	\item \textbf{Verbali} \\
	      Revisionati e approvati il verbali delle riunioni interna ed esterna del 2024-05-22.
	\item \textbf{Documentazione} \\
	      Revisione generale documentazione. Popolazione grafici nel \textit{Piano di Qualfifica}.
	\item Raffinate le dashboard.
\end{itemize}

\subsection{Cosa rimane da fare}
Per questo sprint ci sono ancora alcune attività da completare:
\begin{itemize}
	\item \textbf{Verbali} \\
		  Stesura del verbale della riunione in corso.
	\item \textbf{Documenti} \\
		  Compilare il \textit{Piano di Progetto} per questo sprint e i grafici del \textit{Piano di Qualifica}. Stesura del manuale utente e della scheda tecnica. 
	\item \textbf{Studio} \\
		  Studio e analisi di: architettura e pattern, \textit{Business Case}, \textit{Apache Flink}, KPI.
	\item Per ogni metrica del cruscotto trovare un sistema automatico per calcolarlo.
\end{itemize}

\subsection{Varie ed eventuali}
Per una migliore pianificazione abbiamo deciso di definire le attività previste da qui a fine progetto e suddividerle nei vari sprint.

\subsection{Decisioni prese e conclusioni}
Ci diamo appuntamento per il prossimo incontro interno previsto per il 2024-06-05.

\end{document}