\documentclass[italian,12pt]{article}

%--------------variabili------------------%
\def\Title{Norme di Progetto}
\def\Author{7Last}
\def\Version{v0.2}
%-----------------------------------------%


\usepackage[left=2cm, right=2cm, bottom=3cm, top=3cm]{geometry}
\usepackage{fancyhdr}
\usepackage{graphicx}
\graphicspath{ {../../logo/} }
\usepackage{href-ul}
\usepackage{tikz}
\usepackage{tgadventor}
\usepackage[useregional=numeric,showseconds=true,showzone=false]{datetime2}
\usepackage{caption}
\usepackage{longtable}
\usepackage{xcolor}




\linespread{1.2}
\captionsetup[table]{labelformat=empty}
\geometry{headsep=1.5cm}

\renewcommand{\contentsname}{Indice}
\renewcommand\familydefault{\sfdefault}

\let\oldthepage\thepage
\renewcommand{\thepage}{\sffamily\oldthepage}


\begin{document}

\newgeometry{left=2cm,right=2cm,bottom=2.1cm,top=2.1cm}
\begin{titlepage}
	\vspace*{.5cm}

	\vspace{2cm}
	{
		\centering
		{\bfseries\huge \Title\par}
		\bigbreak
		{\bfseries\Large \Subtitle\par}
		\bigbreak
		{\bfseries\large \Author\par}
		\bigbreak
		{\Date\;-\;\Version\par}
		\vfill

		\begin{center}
			\begin{tikzpicture}
				\clip (0,0) circle (2cm) node {\includegraphics[width=4cm]{logo.jpg}};
			\end{tikzpicture}
		\end{center}
	}

	\vfill

\end{titlepage}

\restoregeometry






















\newpage

\pagestyle{fancy}
\fancyhead{}
\lhead{
	\begin{tikzpicture}
		\clip (0,0) circle (0.5cm);
		\node at (0,0) {\includegraphics[width=1cm]{./../logo/logo.png}};
	\end{tikzpicture}%
}
\chead{\vspace{\fill}\Title\vspace{\fill}}
\rhead{\vspace{\fill}\Version\vspace{\fill}}


\begin{table}[!h]
	\caption{Versioni}
	\footnotesize
	\begin{center}
		\begin{tabular}{ l l l l l }
			\hline                                                             \\[-2ex]
			Ver. & Data       & Redattore   & Verificatore   & Descrizione     \\
			\\[-2ex] \hline \\[-1.5ex]
			1.0  & 2024-05-18 & Davide Malgarise & Raul Seganfreddo & Stesura verbale \\
			\\[-1.5ex] \hline
		\end{tabular}
	\end{center}
\end{table}

\newpage

\tableofcontents

\newpage

\section{Dettagli della riunione}


\textbf{Sede della riunione}: Piattaforma Discord\\
\textbf{Orario di inizio}: 16:30\\
\textbf{Orario di fine}: 17:30\\


\begin{flushleft}
	\begin{table}[!h]
		\begin{tabular}{ |l|l|l| }
			\hline
			\textbf{Partecipante} & \textbf{Ruolo} & \textbf{Presenza} \\
			\hline
			Antonio Benetazzo     &                & Presente          \\
			Davide Malgarise      & Redattore      & Presente          \\
			Elena Ferro           & 		       & Presente          \\
			Leonardo Baldo        & 		       & Presente          \\
			Matteo Tiozzo         & Amministratore & Presente          \\
			Raul Seganfreddo      & Verificatore   & Presente          \\
			Valerio Occhinegro    &                & Presente          \\
			\hline
		\end{tabular}
	\end{table}
\end{flushleft}

\section{Ordine del giorno}
\begin{itemize}
	\item Suddivisione dei ruoli per il quarto sprint
	\item Riepilogo dei compiti da svolgere
	\item Decisioni prese e conclusioni
\end{itemize}

\newpage

\section{Verbale}

\subsection{Suddivisione dei ruoli per il quarto sprint}
Come prima cosa il gruppo stabilisce la suddivisione dei ruoli per il quarto periodo, cercando di assegnare a ciascun membro un compito non ancora svolto
nei precedenti sprint. Ne risulta la seguente suddivisione:
\begin{itemize}
	\item \textbf{Amministratore}: Raul Seganfreddo
	\item \textbf{Progettista}: Davide Malgarise
	\item \textbf{Programmatore}: Elena Ferro
	\item \textbf{Programmatore}: Antonio Benetazzo
	\item \textbf{Responsabile}: Valerio Occhinegro
	\item \textbf{Analista}: Leonardo Baldo
	\item \textbf{Verificatore}: Matteo Tiozzo
	% attenzione: nel documento verbale_interno_2024-05.15 i ruoli sono stati descritti in modo diverso, come si nota sotto
\end{itemize}
Definiamo, inoltre, i ruoli assegnati per ogni riunione interna od esterna 
di questo quarto sprint (da riportare successivamente nei relativi verbali):
\begin{itemize}
	\item \textbf{Amministratore}: Matteo Tiozzo
	\item \textbf{Redattore}: Davide Malgarise
	\item \textbf{Verificatore}: Raul Seganfreddo
\end{itemize}

\subsection{Riepilogo dei compiti da svolgere}
In seguito ad un'analisi e un confronto riguardo agli obiettivi fissati con il proponente,
vengono stabilite le attività da svolgere durante il quarto sprint:
\begin{itemize}
	\item raffinamento delle dashboard in \textit{Grafana};
	\item popolamento dei grafici del documento \textit{Piano di Qualifica};
	\item stesura del quarto sprint all'interno del documento \textit{Piano di Progetto};
	\item revisione dei documenti \textit{Piano di Qualifica} e \textit{Piano di Progetto};
	\item formazione e analisi dei business case e relativi strumenti.
\end{itemize}

\subsection{Decisioni prese e conclusioni}
Infine il gruppo ha deciso di richiedere due incontri, il primo con l'azienda per chiarire i dubbi sorti riguardo
il \textit{PoC} (Proof of Concept) e il secondo con il professore per avere un dialogo di confronto riguardo a quanto
fatto fino ad ora.

\end{document}
