\documentclass[italian,12pt]{article}

%--------------variabili------------------%
\def\Title{Norme di Progetto}
\def\Author{7Last}
\def\Version{v0.2}
%-----------------------------------------%


\usepackage[left=2cm, right=2cm, bottom=3cm, top=3cm]{geometry}
\usepackage{fancyhdr}
\usepackage{graphicx}
\graphicspath{ {../../logo/} }
\usepackage{href-ul}
\usepackage{tikz}
\usepackage{tgadventor}
\usepackage[useregional=numeric,showseconds=true,showzone=false]{datetime2}
\usepackage{caption}
\usepackage{longtable}
\usepackage{xcolor}




\linespread{1.2}
\captionsetup[table]{labelformat=empty}
\geometry{headsep=1.5cm}

\renewcommand{\contentsname}{Indice}
\renewcommand\familydefault{\sfdefault}

\begin{document}

\newgeometry{left=2cm,right=2cm,bottom=2.1cm,top=2.1cm}
\begin{titlepage}
	\vspace*{.5cm}

	\vspace{2cm}
	{
		\centering
		{\bfseries\huge \Title\par}
		\bigbreak
		{\bfseries\Large \Subtitle\par}
		\bigbreak
		{\bfseries\large \Author\par}
		\bigbreak
		{\Date\;-\;\Version\par}
		\vfill

		\begin{center}
			\begin{tikzpicture}
				\clip (0,0) circle (2cm) node {\includegraphics[width=4cm]{logo.jpg}};
			\end{tikzpicture}
		\end{center}
	}

	\vfill

\end{titlepage}

\restoregeometry






















\newpage

\pagestyle{fancy}
\fancyhead{}
\lhead{
	\begin{tikzpicture}
		\clip (0,0) circle (0.5cm);
		\node at (0,0) {\includegraphics[width=1cm]{./../logo/logo.png}};
	\end{tikzpicture}%
}
\chead{\vspace{\fill}\Title\vspace{\fill}}
\rhead{\vspace{\fill}\Version\vspace{\fill}}


\begin{table}[!h]
	\caption{Versioni}
	\footnotesize
	\begin{center}
		\begin{tabular}{ l l l l l }
			\hline                                                                        \\[-2ex]
			Ver. & Data       & Redattore          & Verificatore       & Descrizione     \\
			\\[-2ex] \hline \\[-1.5ex]
			1.0  & 09/04/2024 & Antonio Benetazzo  & Valerio Occhinegro & Stesura verbale \\
			\\[-1.5ex] \hline
		\end{tabular}
	\end{center}
\end{table}

\newpage

\tableofcontents

\newpage

\section{Dettagli della riunione}


\textbf{Sede della riunione}: Piattaforma Discord\\
\textbf{Orario di inizio}: 18:00\\
\textbf{Orario di fine}: 19:00\\


\begin{flushleft}
	\begin{table}[!h]
	\begin{tabular}{ |l|l|l| } 
		\hline
		\textbf{Partecipante} & \textbf{Ruolo}       & \textbf{Presenza} \\
		\hline 
		Antonio Benetazzo     & Redattore            & Presente          \\
		Davide Malgarise      &                      & Presente          \\
		Elena Ferro           &                      & Presente          \\
		Leonardo Baldo        &                      & Presente          \\
		Matteo Tiozzo         & Amministratore       & Presente          \\
		Raul Seganfreddo      &                      & Presente          \\
		Valerio Occhinegro    & Verificatore         & Presente          \\
		\hline
	\end{tabular}
	\end{table}
\end{flushleft}

\section{Ordine del giorno}
\subsection{Suddivisione dei ruoli per il primo sprint}
\subsection{Aggiornamento del Way of Working}
\subsection{Riepilogo compiti da svolgere}
\subsection{Varie ed eventuali}

\newpage

\section{Verbale}

\subsection{Suddivisione dei ruoli per il primo sprint}
Decidiamo insieme la prima suddivisione dei ruoli. Per il primo sprint i ruoli sono stati assegnati come segue:
\begin{itemize}
	\item \textbf{Amministratore}: Antonio Benetazzo
	\item \textbf{Analista}: Davide Malgarise
	\item \textbf{Programmatore}: Elena Ferro
	\item \textbf{Programmatore}: Leonardo Baldo
	\item \textbf{Responsabile}: Matteo Tiozzo
	\item \textbf{Analista}: Raul Seganfreddo
	\item \textbf{Verificatore}: Valerio Occhinegro
\end{itemize}
Definiamo, inoltre, i ruoli assegnati per ogni riunione interna od esterna 
di questo primo sprint (da riportare successivamente nei relativi verbali):
\begin{itemize}
	\item \textbf{Amministratore}: Matteo Tiozzo
	\item \textbf{Redattore}: Antonio Benetazzo
	\item \textbf{Verificatore}: Valerio Occhinegro
\end{itemize}

\subsection{Aggiornamento del Way of Working}
Decidiamo di aggiornare il Way of Working. \\
Come già anticipato durante l'incontro con l'azienda, decidiamo di utilizzare \textit{Discord} come strumento di comunicazione principale, in quanto permette di creare canali dedicati per ogni argomento e di avere una chat vocale, oltre ad essere uno strumento già ampiamente utilizzato dalla proponente. \\
Come consigliato sempre dall'azienda, decidiamo di cominciare ad usare \textit{ClickUp} per la gestione delle attività e delle scadenze. All'interno di esso sarà anche possibile gestire i documenti relativi alle annotazioni, precedentemente gestiti con \textit{Notion}, oltre ad avere accesso alla funzionalità di tracciamento delle ore spese per ciascun task (ci potrà essere d'aiuto nella rendicontazione delle ore), alla possibilità di integrazione con \textit{GitHub} per poter collegare a ciascuna attività, branch e pull request e ad un'ampia gamma di reportistica. \\
Valutiamo sia opportuno implementare un modo automatico per poter gestire i file e la loro pubblicazione nel branch main, in modo che tutti i sorgenti relativi alla documentazione e le cartelle di servizio vengano rimossi, così da rendere più semplice il mantenimento del repository. \\
Decidiamo di prendere come modello di riferimento lo standard \textit{ISO/IEC 12207:2017} per la gestione del ciclo di vita del software. Per questo motivo verranno cambiate le date presenti nella documentazione, in modo che rispettino lo standard \textit{ISO 8601}. \\
Verrà creato un repository aggiuntivo dove verranno pubblicati i file (sorgenti e documentazione) relativi al prodotto 
richiesto dal capitolato che andremo a sviluppare, denominato \textit{SyncCity}. Valuteremo successivamente 
la possibilità di attuare politiche di protezione dei branch in questo e in quello relativo alla documentazione. \\
Ci accordiamo per trovarci in riunione interna ogni settimana, indicativamente ogni mercoledì alle ore 15; riteniamo sia importante confrontarci regolarmente per valutare assieme quanto fatto, quanto rimane da fare, risolvere tutte le problematiche occorse in maniera tempestiva e eventuali dubbi o discutere di qualunque argomento venga ritenuto opportuno.

\subsection{Riepilogo compiti da svolgere}
Vengono riepilogati i compiti da svolgere in questo primo sprint:
\begin{itemize}
	\item ingresso da parte di tutti i componenti nei canali di \textit{Discord} messi a disposizione da parte dell'azienda;
	\item configurazione di \textit{ClickUp} e creazione dei task relativi al primo sprint;
	\item implementazione di un sistema automatico per la pubblicazione dei file nel branch main;
	\item aggiornamento dei template relativi ai documenti e ai verbali interni ed esterni;
	\item creazione repository \textit{SyncCity};
	\item creazione del template da usare per le presentazioni da mostrare durante le attività \textit{Diario di Bordo};
	\item studio delle tecnologie \textit{Docker Compose} e \textit{Apache Kafka} proposte dalla proponente e valutazione di eventuali alternative;
	\item sviluppo di un semplice programma in \textit{Python} per la generazione dei dati;
	\item prima stesura dei documenti \textit{Analisi dei Requisiti}, \textit{Piano di Progetto}, \textit{Piano di Qualifica}, \textit{Norme di Progetto}, \textit{Glossario}.
\end{itemize}

\subsection{Decisioni prese e conclusioni}
Una volta definito quanto riportato nel presente verbale ci salutiamo e ci diamo appuntamento al prossimo incontro programmato 
per il giorno 10 aprile alle ore 15.

\end{document}