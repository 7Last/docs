\section{Processi di supporto}


\subsection{Documentazione}

\subsubsection{Introduzione}
Il processo di documentazione è una componente fondamentale nella realizzazione e nel rilascio di un prodotto \textcolor{red}{\uline{\textit{software}}},
poichè fornisce informazioni utili alle parti coinvolte e tiene traccia di tutte le attività relative al ciclo di vita del software,
comprese scelte e \textcolor{red}{\uline{\textit{norme}}} adottate dal gruppo durante lo svolgimento del progetto. In particolare, la documentazione è utile per:
\begin{itemize}
    \item Permettere una comprensione profonda del prodotto e delle sue funzionalità;
    \item Tracciare un confine tra disciplina e creatività;
    \item Garantire uno standard di qualità all'interno dei \textcolor{red}{\uline{\textit{processi}}} produttivi.
\end{itemize} 
Lo scopo della sezione è:
\begin{itemize}
    \item Fornire una raccolta esasutiva di regole che i membri del gruppo devono seguire per agevolare la stesura della documentazione;
    \item Definire delle procedure ripetibili per uniformare la redazione, la verifica e l'approvazione dei documenti;
    \item Creare template per ogni tipologia di documento così da garantire omogeneità e coerenza.
\end{itemize}

\subsubsection{Documentation as Code}
\textbf{TODO adozione di questo approccio?}
\begin{comment}  
    L’approccio che si intende adottare è quello di "Documentation as Code" (Documentazione
    come Codice) che consiste nel trattare la documentazione di un progetto software allo
    stesso modo in cui si tratta il codice sorgente.
    Questo approccio è incentrato sull’utilizzo di pratiche e strumenti tipici dello sviluppo
    software per creare, gestire e distribuire la documentazione. Alcuni aspetti chiave della
    "Documentation as Code" includono:
        • Versionamento;
        • Scrittura in formato testuale;
        • Automazione;
        • Collaborazione;
        • Integrazione Continua;
        • Distribuzione.
    Questo approccio porta diversi vantaggi, tra cui una maggiore coerenza, una migliore
    tracciabilità delle modifiche e facilità di manutenzione. Inoltre, il concetto di "Documentation
    as Code" si allinea con la filosofia DevOps, dove la collaborazione e l’automazione sono valori
    chiave.
\end{comment}

\subsubsection{Tipografia e sorgente documenti}
Per la redazione dei documenti abbiamo deciso di utilizzare il linguaggio di markup \LaTeX{}, in quanto semplifica la creazione e la manutenzione
dei documenti, liberando i redattori dall'onere della visualizzazione grafica e garantendo coerenza nella documentazione del progetto.
Inoltre, per favorire una migliore collaborazione tra i diversi autori, abbiamo scelto di scomporre ogni documento in più file,
ciascuno per una specifica sezione, in modo da permettere a più persone di lavorare sulle singole sezioni o sottosezioni.
Il risultato finale sarà ottenuto tramite l'assemblaggio di tutti i file sorgenti in un file principale, attraverso l'uso del comando 
\textit{input{Sezione.tex}} o, nel caso delle sottosezioni, \textit{input{Sottosezione.tex}}.
Solo nel caso di documenti di piccole dimensioni, come i verbali, si potrà optare per la scrittura di un unico file. 


\subsubsection{Ciclo di vita}
Il ciclo di vita di un documento è composto dalle seguenti fasi:
\begin{enumerate}
    \item Pianificazione della stesura e suddivisione in sezioni: tramite confronto con il gruppo, le sezioni del documento vengono stabilite e assegnate ai Redattori.
    Essi sono responsabili della stesura delle proprie sezioni in conformità con le Norme di Progetto.
    \item Stesura del contenuto e creazione della bozza iniziale: i Redattori realizzano il documento redigendone il contenuto e creano una prima bozza
    che viene utilizzata come punto di partenza per la discussione e la revisione.
    \item Controllo dei contenuti: dopo la stesura effettiva del documento, i Redattori verificano che e il contenuto delle proprie sezioni
    sia conforme alle norme definite e non contenga errori di compilazione.
    \item Revisione: quando la redazione del documento è conclusa, questo viene revisionato dai Verificatori incaricati.
    \item Approvazione e rilascio: nell'ultima fase il documento viene approvato da un Responsabile e rilasciato in versione finale.
\end{enumerate}

\subsubsection{Procedure correlate alla redazione di documenti}

\subsubsubsection{I redattori}
Il redattore è colui che si occupa di scrivere e curare il contenuto di un documento o di una sua sezione in modo chiaro, accurato e comprensibile.
Nel farlo deve seguire lo stesso approccio impiegato nella codifica del \textcolor{red}{\uline{\textit{software}}}, adottando il
\textcolor{red}{\uline{\textit{workflow}}} noto come \textit{feature branch}.
\textbf{Caso redazione nuovo documento/sezione o modifica dei precedenti già verificati}
In queste situazioni il redattore dovrà creare un nuovo \textcolor{red}{\uline{\textit{branch}}} Git in locale e posizionarsi su di esso con i seguenti comandi:
\begin{itemize}
    \item \textit{git checkout main}
    \item \textit{git checkout -b nomeBranch}
\end{itemize}
In particolare l'identificativo del branch deve essere 'parlante', ossia descrittivo e significativo così da consentire una compresione immediata
del documento o della sezione che si sta redigendo. Dunque il redattore deve adottare le specifiche 
\href{#convenzioni_nomenclatura}{convenzioni per la nomenclatura dei branch}.
Una volta terminata la redazione del documento o della sezione assegnata, è necessario rendere disponibile il branch
nella \textcolor{red}{\uline{\textit{repository}}} remota seguento la seguente procedura:
\begin{enumerate}
    \item Eseguire il \textcolor{red}{\uline{\textit{push}}} delle modifiche nel \textcolor{red}{\uline{\textit{branch}}}:
    \begin{itemize}
        \item \textit{git add .}
        \item \textit{git commit -m "Descrizione delle modifiche apportate"}
        \item \textit{git push origin nomeBranch}
    \end{itemize}
    \item Se riscontriamo problemi nel punto 1:
    \textit{git pull origin nomeBranch}
    \item Risolviamo i conflitti e ripetiamo il punto 1.
\end{enumerate}
\textbf{Caso modifica documento in fase di redazione}
Per continuare la redazione di un documento o di una sezione già in fase di stesura, sono necessari i seguenti comandi
\begin{itemize}
    \item \textit{git pull}
    \item \textit{git checkout nomeBranch}
\end{itemize}
\textbf{Completamento redazione documento}
Dopo aver completato la redazione del documento o della sezione, il redattore deve procedere nel seguente modo:
\begin{enumerate}
    \item Spostare l'\textcolor{red}{\uline{\textit{issue}}} relativa all'\textcolor{red}{\uline{\textit{attività}}} assegnata nella colonna "Review" della 
    \href{https://github.com/orgs/7Last/projects/1/views/1}{DashBoard} del progetto, così da comunicare il completamento dell'incarico ai Verificatori.
    \item Aggiornare la tabella contenente il versionamento del documento, inserendo le informazioni richieste e incrementando la versione.
    \item Creare una \href{#pull_request}{Pull Request}: 
    \begin{enumerate}
        \item Accedere alla \href{https://github.com/7Last/docs}{Repository} GitHub, spostarsi nella sezione "Pull Request" e cliccare su "New Pull Request";
        \item Selezionare come \textcolor{red}{\uline{\textit{branch}}} di destinazione "main" e come \textcolor{red}{\uline{\textit{branch}}} sorgente il ramo creato appositamente per la redazione del documento/sezione;
        \item Cliccare su "Create Pull Request";
        \item Dare un titolo significativo e, se necessario, una descrizione alla Pull Request, selezionare i Verificatori e cliccare su "Create Pull Request".
    \end{enumerate}
\end{enumerate}

\subsubsubsection{I verificatori}
Il ruolo e la procedura dei Verificatori sono descritti in dettaglio al paragrafo \href{#verifica_dei_documenti}{3.2.2}.

\subsubsubsection{Il responsabile}
Per quanto riguarda la redazione dei documenti, il Responsabile ha il dovere di:
\begin{itemize}
    \item Identificare i documenti da redigere;
    \item Assegnare le task a Redattori e Verificatori;
    \item Stabilire la scadenza per il completamento delle attività;
    \item Approvare o richiedere eventuali modifiche ai documenti.
\end{itemize}

\subsubsubsection{L'amministratore}
L'amministratore è responsabile della gestione delle \textcolor{red}{\uline{\textit{attività}}} richieste dal Responsabile all'interno dell'ITS.

\subsubsection{Struttura del documento}
Tutta la documentazione prodotta segue uno schema strutturale ben definito e uniforme.
\subsubsubsection{Prima pagina}
Nella prima pagina di ogni documento è presente un'intestazione contenente le seguenti informazioni:
\begin{itemize}
    \item Nome del documento;
    \item Versione del documento;
    \item Logo del gruppo;
    \item Nome del gruppo.
\end{itemize}

\subsubsubsection{Registro delle modifiche}
La seconda pagina è dedicata al registro delle modifiche in formato tabellare e permette di tenere traccia delle modifiche apportate al documento.
La tabella riporta i seguenti dati:
\begin{itemize}
    \item Versione del documento;
    \item Data di rilascio;
    \item Nome dell'autore;
    \item Nome del Verificatore;
    \item Descrizione della modifica.
\end{itemize}

\subsubsubsection{Indice}
Ogni documento contiene un indice delle sezioni e delle sottosezioni presenti al suo interno, in modo da facilitare la consultazione e la navigazione.

\subsubsubsection{Intestazione}
Ogni pagina del documento, ad eccezione della prima, contiene un'intestazione che riporta il nome del documento, la versione e il logo del team.

\subsubsubsection{Verbali: struttura generale}
I verbali costituiscono un report dettagliato dei meeting, con lo scopo di tenere traccia degli argomenti trattati, delle decisioni adottate e le
azioni da intraprendere. Essi si suddividono in esterni o interni, a seconda che il meeting sia con persone esterne al gruppo o con i soli membri del team.
La struttura in ogni caso è la medesima e prevede le seguenti sezioni:
\begin{itemize}
    \item \textbf{Informazioni sulla riunione}:
    \begin{itemize}
        \item Sede del meeting;
        \item Orario di inizio e fine;
        \item Partecipanti del gruppo;
        \item Partecipanti esterni.
    \end{itemize}
    \item \textbf{Corpo del documento}
    \begin{itemize}
        \item \textbf{Revisione del periodo precedente}:
            Analisi dello stato delle \textcolor{red}{\uline{\textit{attività}}} e dell'approccio lavorativo, si discute di eventuali problemi riscontrati
            ma anche degli aspetti positivi in modo da incrementare e migliorare il way of working. 
        \item \textbf{Ordine del giorno}:
            Elenco di ciò che verrà discusso durante la riunione.
        \item \textbf{Sintesi dell'incontro}:
            Breve riassunto delle discussioni e dei temi affrontati surante l'incontro.
        \item \textbf{Decisioni prese}:
            Sezione che elenca in formato testuale le decisioni prese durante il meeting. Alcune di queste potrebbero risultare anche in "Attività individuate".        
        \item \textbf{Attività individuate}:
            Illustrazione dettagliata delle attività assegnate ai diversi membri del gruppo in forma tabellare, sono presenti le seguenti informazioni:
            \begin{itemize}
                \item Nome della task;
                \item ID dell'\textcolor{red}{\uline{\textit{issue}}} su GitHub;
                \item Assegnatari.
            \end{itemize}
    \end{itemize}
    \item \textbf{Ultima pagina}:
    Solo nel caso di verbale esterno, è presente una sezione dedicata alla data e alla firma delle terze parti coinvolte.
\end{itemize}
            
\subsubsection{Norme tipografiche}
\textbf{Nomi assegnati ai file} Il nome dei documenti deve essere omogogeneo alla tipologia di appartenenza, deve essere in minuscolo
e contenere un riferimento alla versione del documento. In particolare la nominazione dei file deve seguire la convenzione:
\begin{itemize}
    \item \textbf{Verbali}: verbale\_esterno/interno\_AA\_MM\_DD\_vX.Y;
    \item \textbf{Norme di Progetto}: norme\_di\_progetto\_vX.Y;
    \item \textbf{Analisi dei Requisiti}: analisi\_dei\_requisiti\_vX.Y;
    \item \textbf{Piano di Progetto}: piano\_di\_progetto\_vX.Y;
    \item \textbf{Glossario}: glossario\_vX.Y.
\end{itemize}
\textbf{Stile del testo}
\begin{itemize}
    \item \textbf{Grassetto}:
    \begin{itemize}
        \item Titoli di sezione;
        \item Termini importanti;
        \item Parole seguite da descrzione o elenchi puntati.
    \end{itemize}
    \item \textbf{Corsivo}:
    \begin{itemize}
        \item Nome del gruppo e dell'azienda proponente;
        \item Termini presenti nel glossario;
        \item Riferimenti a documenti esterni.
    \end{itemize}
    \item \textbf{Maiuscolo}:
    \begin{itemize}
        \item Acronimi;
        \item Iniziali dei nomi;
        \item Iniziali dei ruoli svolti dai membri del gruppo.
    \end{itemize}
\end{itemize}
\textbf{Regole sintattiche}:
\begin{itemize}
    \item Negli elenchi ogni voce deve terminare con ";", ad eccezione dell'ultima che prevede ".";
    \item I numeri razionali si scrivono utilizzando la virgola come separatore tra parte intera e parte decimale;
    \item Le date devono seguire lo standard internazionale ISO 8601, ossia YYYY-MM-DD.
\end{itemize}

\subsubsection{Abbreviazioni}
Segue un elenco delle abbreviazioni più comuni utilizzate nei documenti:
\begin{center}
    \begin{tabular}{|c|c|}
    \hline
    \textbf{Abbreviazione} & \textbf{Scrittura Estesa} \\
    \hline
    RTB & Requirements and Technology Baseline \\
    PB & Product Baseline \\
    CA & Customer Acceptance \\
    ITS & Issue Tracking System \\
    CI & Configuration Item \\
    SAL & Stato Avanzamento Lavori \\
    \hline
    \end{tabular}
\end{center}

\subsubsection{Strumenti}
Gli strumenti utilizzati per la redazione dei documenti sono:
\begin{itemize}
    \item \textbf{\LaTeX{}}: utilizzato per la stesura dei documenti;
    \item \textbf{GitHub}: utilizzato per la gestione del versionamento e per la condivisione dei documenti;
    \item \textbf{Visual Studio Code}: utilizzato come editor di testo per la scrittura dei documenti attraverso l'estensione \LaTeX{} Workshop.
\end{itemize}

\subsection{Verifica}
\subsubsection{Introduzione}
Il processo di verifica è fondamentale durante tutto il ciclo di vita del \textcolor{red}{\uline{\textit{software}}}, a partire dall'iniziale fase
di progettazione fino alla sucessiva manutenzione. La verifica ha lo scopo di garantire che ciascuna \textcolor{red}{\uline{\textit{attività}}}
sia corretta ed efficiente, identificando un processo di controllo per ogni prodotto realizzato. In particolare, ci si preoccupa che gli output
del software (documentazione, codice sorgente, test...) siano conformi alle aspettative e ai requisiti specificati. Nel farlo è fondamentale
applicare tecniche e analisi di test seguendo procedure definite e adottando criteri affidabili.
Le attività di verifica sono svolte dai Verificatori, i quali sono responsabili di analizzare i prodotti e valutare la loro aderenza agli standard
stabiliti. Il fulcro di questo processo è il \textit{Piano di Qualifica}, un documento dettagliato che traccia il percorso della verifica.
Questo fornisce linee guida per una valutazione accurata della qualità, delineando chiaramente gli obiettivi da raggiungere e i criteri di
accettazione da rispettare.

\subsubsection{Verifica dei documenti} \label{verifica_dei_documenti}
Nell'ambito della documentazione, la verifica è un'attività cruciale per garantire la correttezza e l'accuratezza dei contenuti. Essa si suddivide in:
\begin{itemize}
    \item \textbf{Revisione della correttezza tecnica}: assicura che tutte le informazioni siano corrette e coerenti con le norme stabilite;
    \item \textbf{Conformità alle norme}: verifica che il documento segua le linee guida e gli \textcolor{red}{\uline{\textit{standard}}}
        stabiliti per la formattazione, la struttura e lo stile;
    \item \textbf{Revisione ortografica e grammaticale}: controlla che il testo sia privo di errori ortografici, grammaticali e di punteggiatura.
    \item \textbf{Chiarezza e comprensibilità}: valuta la leggibilità del documento, verificando che il contenuto sia chiaro, comprensibile e privo di ambiguità;
    \item \textbf{Coerenza}: verifica che il documento sia omogeneo e coerente, sia internamente che con i documenti correlati.
\end{itemize}
% Specificare procedura GitHub? Pull request, branch, spostare la issue ecc.

\subsubsection{Analisi}
L'analisi è un processo che si occupa di valutare la qualità degli ogetti statici (documenti e codice sorgente) e dinamici
(test ed esecuzione del software).

\subsubsubsection{Analisi statica}
L'analisi statica è un'\textcolor{red}{\uline{\textit{attività}}} di controllo che prescinde dall'esecuzione del prodotto e si basa su una revisione manuale o automatica
del codice e della documentazione. Essa è fondamentale per verificare la presenza di proprietà desiderate e la conformità ai vincoli e
per garantire che non siano presenti errori o difetti. L'analisi statica prevede due metodi di lettura:

\subsubsubsection{Walkthrough}
Questa tecnica prevede una lettura integrale e approfondita del prodotto, con l'obiettivo di individuare errori e difetti. Dunque lo scopo della verifica
non è specifica per un determinato tipo di errore, ma generale. Inoltre il walkthrough è un approccio collaborativo che coinvolge il Verificatore
e l'autore del prodotto, in particolare esso si svolge in quattro fasi:
\begin{enumerate}
    \item \textbf{Pianificazione}: il Verificatore e l'autore si confrontano per individuare le proprietà e i vincoli che il prodotto deve soddisfare;
    \item \textbf{Lettura}: il Verificatore esamina il prodotto, annotando errori e verificando la conformità ai vincoli;
    \item \textbf{Discussione}: il Verificatore e l'autore discutono degli errori riscontrati e valutano le possibili soluzioni;
    \item \textbf{Correzione}: l'autore apporta le modifiche concordate.
\end{enumerate}
Nel caso di prodotti particolarmente complessi o di grandi dimensioni, il walkthrough può risultare dispendioso in termini di risorse,
per questo motivo è più probabile adottarne l'impiego nelle fasi iniziali del progetto.

\subsubsubsection{Inspection}
Al contrario del walkthrough, l'inspection prevede una conoscenza preventiva degli elementi da verificare, i quali vengono organizzati in
liste di controllo specifiche (\textit{checklist}). Di conseguenza questo approccio risulta più rapido ed efficiente nel contesto di documenti
o codice complessi e strutturati poichè consente di identificare tempestivamente e risolvere potenziali problematiche.

\subsubsubsection{Analisi dinamica}
L'analisi statica è un'\textcolor{red}{\uline{\textit{attività}}} di controllo che richiede l'esecuzione effettiva del codice con lo scopo di
individuare discordanze tra i risultati ottenuti e il comportamento atteso del software. Il \textcolor{red}{\uline{\textit{test}}} costituisce
la principale tecnica di analisi dinamica, rappresentato da esecuzioni del codice in un dominio di casi definito in precedenza dal Verificatore.
Quest'ultimo è composto da tutti i possibili casi e dati di input che possono far emergere difetti o eventuali problemi di funzionamento e
garantire la qualità del prodotto finale. Per assicurare l'\textcolor{red}{\uline{\textit{efficacia}}} di un test è necessario
che esso sia ripetibile e decidibile, ossia che produca risultati coerenti e che possa essere eseguito più volte senza che 
i risultati siano influenzati da fattori esterni. Un altro aspetto importante è l'automazione del processo, realizzabile tramite l'uso di
strumenti specifici (driver, stub, logger) che consentono di eseguire i test in modo automatico e di monitorare i risultati ottenuti.

\subsubsection{Testing}


\subsubsubsection{Test di unità}
\subsubsubsection{Test di integrazione}
\subsubsubsection{Test di sistema}
\subsubsubsection{Test di regressione}
\subsubsubsection{Test di accettazione}
\subsubsubsection{Sequenza delle fasi di test}
\subsubsubsection{Codici dei test}
\subsubsubsection{Stato dei test}
\subsubsubsection{Continuous integration}
\subsubsection{Strumenti}
\subsection{Validazione}
\subsubsection{Introduzione}
\subsubsection{Procedura di validazione}
\subsubsection{Strumenti}
\subsection{Gestione della configurazione}
\subsubsection{Introduzione}
\subsubsection{Versionamento}
\subsubsection{Repository}
\subsubsubsection{Struttura repository}
\subsubsection{Sincronizzazione e branching}

\subsubsubsection{Documentazione}
\textbf{Convenzioni per la nomenclatura dei branch relativi alle attività di redazione o modifica di documenti} \label{convenzioni_nomenclatura}

\subsubsubsection{Sviluppo}

\subsubsubsection{Pull Request}
\textbf{Procedura per la creazione di una Pull Request} \label{pull_request}

\subsubsection{Controllo di configurazione}
\subsubsubsection{Change Request}
\subsubsection{Configuration Status Accounting}
\subsubsection{Release management and delivery}
\subsubsubsection{Procedura per la creazione di una release}
\subsubsection{Strumenti}
\subsection{Joint review}
\subsubsection{Introduzione}
\subsubsection{Implementazione del processo}
\subsubsubsection{Revisioni periodiche}
\subsubsubsection{SAL}
\subsubsubsection{Revisioni ad hoc}
\subsubsubsection{Risorse per le revisioni}
\subsubsubsection{Elementi da concordare}
\subsubsubsection{Documenti e distribuzione dei risultati}
\subsubsection{Project management reviews}
\subsubsubsection{Introduzione}
\subsubsubsection{Stato del progetto}
\subsubsection{Revisioni tecniche}
\subsubsection{Strumenti}
\subsection{Risoluzione dei problemi}
\subsubsection{Introduzione}
\subsubsection{Gestione dei rischi}
\subsubsubsection{Codifica dei rischi}
\subsubsubsection{Metriche}
\subsubsection{Identificazione dei problemi}
\subsubsection{Strumenti}
\subsection{Gestione della qualità}
\subsubsection{Introduzione}
\subsubsection{Attività}
\subsubsection{Piano di qualifica}
\subsubsection{PDCA}
\subsubsection{Strumenti}
\subsubsection{Struttura e identificazioni metriche}
\subsubsection{Criteri di accettazione}
\subsubsection{Metriche}
