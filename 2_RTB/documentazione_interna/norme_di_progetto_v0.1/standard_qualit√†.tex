\section{Standard per la qualità}

ADOTTIAMO STANDARD 12207:2017 per la suddivisione dei processi
ADOTTIAMO STANDARD 25010:2023 per la definizione delle metriche di qualità

Nel corso dell’analisi e della valutazione della qualità dei processi e del software, adotteremo standard internazionali ben definiti per garantire una valutazione rigorosa e conforme agli standard globali. In particolare, l’utilizzo dello standard ISO/IEC 9126 fornirà una solida struttura per valutare la qualità del software, concentrandosi su attributi quali la funzionalità, l’affidabilità, l’usabilità, l’efficienza, la manutenibilità e la portabilità. Questo framework ci consentirà di misurare in modo accurato e completo la qualità del prodotto software .
Parallelamente, la suddivisione dei processi in primari, di supporto e organizzativi sarà guidata dall’adozione dello standard ISO/IEC 12207:2017. Infine, l’adozione dello standard ISO/IEC 25010 ci fornirà un quadro completo per la definizione e la suddivisione delle metriche di qualità del software. L’utilizzo congiunto di questi standard consentirà un approccio completo e strutturato alla valutazione della qualità dei processi e del software, assicurando un’elevata coerenza, affidabilità e conformità agli standard riconosciuti a livello internazionale.
\subsection{Caratteristiche del sistema ISO/IEC 25010}
\subsubsection{Funzionalità}
\subsubsection{Affidabilità}
\subsubsection{Usabilità}
\subsubsection{Efficienza}
\subsubsection{Manutenibilità}
\subsubsection{Portabilità}
\subsection{Suddivisione secondo standard ISO/IEC 12207:2017}
\subsubsection{Processi primari}
Fondamentali per lo sviluppo del software e comprendono:
\begin{itemize}
    \item \textbf{Acquisizione}: coinvolge le attività necessarie per acquisire i componenti software esterni necessari al progetto;
    \item \textbf{Fornitura}: coinvolge la consegna del software al cliente o all'utente finale;
    \item \textbf{Sviluppo}: include tutte le attività associate alla progettazione, implementazione e verifica del software;
    \item \textbf{Operazione}: riguarda le attività coinvolte nell'operare il software in un ambiente operativo reale;
    \item \textbf{Manutenzione}: coinvolge tutte le attività necessarie per mantenere e migliorare il software dopo la sua consegna.
\end{itemize}
\subsubsection{Processi di supporto}
Questi processi forniscono il supporto necessario per i processi primari e comprendono:
\begin{itemize}
    \item \textbf{Documentazione}: coinvolge la produzione e la manutenzione della documentazione associata al software;
    \item \textbf{Gestione della configurazione}: include le attività per la gestione delle configurazioni software, come il controllo delle versioni e la gestione delle modifiche;
    \item \textbf{Assicurazione della qualità}: questo processo riguarda le attività per garantire che il software soddisfi i requisiti di qualità stabiliti;
    \item \textbf{Verifica}: coinvolge la revisione e la valutazione dei prodotti software per garantire che soddisfino i requisiti specificati;
    \item \textbf{Validazione}: questo processo si concentra sulla conferma che il software soddisfi le esigenze dell'utente e si integri correttamente nell'ambiente operativo.
\end{itemize}
\subsubsection{Processi organizzativi}
Questi processi supportano l'organizzazione nel suo insieme e si compongono di:
\begin{itemize}
    \item \textbf{Gestione}: coinvolge la pianificazione, l'organizzazione e il controllo delle attività di ingegneria del software;
    \item \textbf{Gestione delle infrastrutture}: include la gestione delle risorse, delle strutture e degli strumenti necessari per lo sviluppo del software;
    \item \textbf{Gestione dei processi}: questo processo riguarda il miglioramento continuo dei processi utilizzati nell'ingegneria del software;
    \item \textbf{Formazione}: coinvolge la formazione del personale per sviluppare le competenze necessarie per svolgere efficacemente i compiti assegnati;
    \item \textbf{Amministrazione}: questo processo riguarda l'amministrazione generale dei processi e delle risorse necessarie per il loro funzionamento.
\end{itemize}