\section{Introduzione}
\subsection{Scopo del documento}
Questo documento ha lo scopo di descrivere le regole e le procedure che il gruppo adotterà per lo svolgimento del progetto. Lo scopo quindi è quello di definire il \textit{Way of Working} del gruppo, in modo da garantire un lavoro efficiente e di qualità.\\
Il processo seguirà le linee guida descritte dallo standard ISO/IEC 12207:1995.

\subsection{Scopo del progetto}
Lo scopo del progetto è quello di realizzare una piattaforma di monitoraggio per una smart city, in grado di raccogliere e analizzare in tempo reale dati provenienti da diverse fonti, come sensori, dispositivi indossabili e macchine. La piattaforma avrà lo scopo di:
\begin{itemize}
	\itemsep0em
    \item Migliorare la qualità della vita dei cittadini: la piattaforma consentirà alle autorità locali di prendere decisioni informate e tempestive sulla gestione delle risorse e sull'implementazione di servizi, basandosi su dati reali e aggiornati.
    \item Coinvolgere i cittadini: i dati monitorati saranno resi accessibili al pubblico attraverso portali online e applicazioni mobili, permettendo ai cittadini di essere informati sullo stato della loro città e di partecipare attivamente alla sua gestione.
    \item Gestire il big data: la piattaforma sarà in grado di gestire grandi volumi di dati provenienti da diverse fonti, aggregandoli, normalizzandoli e analizzandoli per estrarre informazioni significative.
\end{itemize}
La piattaforma si baserà su tecnologie di data streaming processing per l'analisi in tempo reale dei dati e su una piattaforma OLAP per la loro archiviazione e visualizzazione. La parte "IoT" del progetto sarà simulata attraverso tool di generazione di dati realistici.

In sintesi, il progetto mira a creare una piattaforma che sia:
\begin{itemize}
    \item Efficiente: in grado di raccogliere e analizzare grandi volumi di dati in tempo reale.
    \item Efficace: in grado di fornire informazioni utili per la gestione della città e il miglioramento della qualità della vita dei cittadini.
    \item Accessibile: in grado di rendere i dati disponibili al pubblico in modo chiaro e comprensibile.
\end{itemize}
Il progetto si pone come obiettivo di contribuire allo sviluppo di città più intelligenti e sostenibili, in cui la tecnologia viene utilizzata per migliorare il benessere dei cittadini.

\subsection{Glossario}
Al fine di evitare ambiguità e di facilitare la comprensione del documento, si allega il \textit{Glossario} contenente la definizione dei termini tecnici e degli acronimi utilizzati.

\subsection{Riferimenti}
\subsubsection{Riferimenti normativi}
\begin{itemize}
	\item Glossario: \textbf{TODO inserire link}
	\item \textit{ISO/IEC 12207:1995}: \textbf{TODO inserire link}
\end{itemize}
\subsubsection{Riferimenti informativi}

