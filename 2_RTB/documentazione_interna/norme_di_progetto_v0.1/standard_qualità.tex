\section{Standard per la qualità}
Nel corso dell’analisi e della valutazione della qualità dei processi e del software, adotteremo standard internazionali ben definiti per garantire una valutazione rigorosa e conforme agli standard globali. In particolare, l’utilizzo dello standard ISO/IEC 9126 fornirà una solida struttura per valutare la qualità del software, concentrandosi su attributi quali la funzionalità, l’affidabilità, l’usabilità, l’efficienza, la manutenibilità e la portabilità. Questo framework ci consentirà di misurare in modo accurato e completo la qualità del prodotto software .
Parallelamente, la suddivisione dei processi in primari, di supporto e organizzativi sarà guidata dall’adozione dello standard ISO/IEC 12207:1995. Infine, l’adozione dello standard ISO/IEC 25010 ci fornirà un quadro completo per la definizione e la suddivisione delle metriche di qualità del software. L’utilizzo congiunto di questi standard consentirà un approccio completo e strutturato alla valutazione della qualità dei processi e del software, assicurando un’elevata coerenza, affidabilità e conformità agli standard riconosciuti a livello internazionale.

\subsection{Caratteristiche del sistema, STANDARD CHE USEREMO NOI}
\subsubsection{Funzionalità}
\subsubsection{Affidabilità}
\subsubsection{Usabilità}
\subsubsection{Efficienza}
\subsubsection{Manutenibilità}
\subsubsection{Portabilità}
\subsection{Suddivisione secondo standard, STANDARD CHE USEREMO NOI}
\subsubsection{Processi primari}
\subsubsection{Processi di supporto}
\subsubsection{Processi organizzativi}
