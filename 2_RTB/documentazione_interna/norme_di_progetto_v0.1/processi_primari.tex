\section{Processi primari}
\subsection{Fornitura}
\subsubsection{Introduzione}
Il processo di fornitura ha lo scopo di stabilire un accordo contrattuale tra il fornitore e il cliente, in cui vengono definiti i servizi che il fornitore si impegna a fornire e le condizioni di fornitura. 
Il processo di fornitura comprende le seguenti attività:
\subsubsection{Attività}
\begin{itemize}
	\item \textbf{Preparazione della proposta}
	\item \textbf{Contrattazione}
	\item \textbf{Pianificazione}
	\item \textbf{Esecuzione}
	\item \textbf{Revisione}
	\item \textbf{Consegna}
\end{itemize}

\subsubsection{Comunicazioni con l'azienda proponente}
L'azienda proponente SyncLab mette a disposizione l'indirizzo di posta elettronica e il suo canale Discord per la comunicazione tramite messaggi e Google Meet per la comunzicazione attraverso incontri telematici.
Gli incontri telematici hanno una cadenza regolare di due settimane, con possibili incontri aggiuntivi richiesti dal gruppo in caso di necessità, come ad esempio chiarimenti riguardo al capitolato o alle tecnologie utilizzate.
Per ogni colloquio avvenuto con l'azienda proponente verrà fornito un verbale esterno che riporterà i vari argomenti discussi durante il colloquio.
I verbali saranno disponibili all'interno del \textit{repository} \url{https://github.com/7Last/docs}
\subsubsection{Documentazione fornita}
Di seguito saranno elencati i documenti che il gruppo \textit{7last} consegnerà all'azienda \textit{SyncLab} e ai committenti \textit{Prof. Tullio Vardanega} e \textit{Prof. Riccardo Cardin}.
\subsubsubsection{Valutazione dei capitolati}
Il presente documento offre una valutazione approfondita dei capitolati d'appalto presentati in data 2023-10-17. Per ciascun progetto, vengono esaminate le richieste del proponente, le possibili soluzioni e le eventuali criticità.
La valutazione si articola nelle seguenti sezioni:
\begin{itemize}
	\item \textbf{Descrizione}: viene elencato il nome del progetto, l'azienda proponente, i committenti e l'obbiettivo del progetto;
	\item \textbf{Dominio Applicativo}: viene descritto il contesto del progetto;
	\item \textbf{Dominio Tecnologico}: vengono descritte le tecnologie utilizzate per lo sviluppo del progetto;
	\item \textbf{Aspetti Positivi};
	\item \textbf{Aspetti Negativi};
\end{itemize}

\subsubsubsection{Analisi dei requisiti}
L'Analisi dei Requisiti è un documento completo che delinea i casi d'uso, i requisiti e le funzionalità necessarie per il prodotto software.
Il suo scopo principale è chiarire qualsiasi incertezza o ambiguità che potrebbe sorgere dopo la lettura del capitolato.
Questo documento include:
\begin{itemize}
	\item Una \textbf{descrizione dettagliata del prodotto};
	\item Un \textbf{elenco dei casi d'uso}: riporta tutti gli scenari possibili in cui il sistema software potrebbe essere utilizzato dagli utenti finali, descrivendo le azioni che gli utenti compiono nel sistema in modo da identificare requisiti non ovvi inizialmente;
	\item Un \textbf{elenco dei requisiti}: specifica tutti i vincoli richiesti dal proponente o dedotti in base all'analisi dei casi d'uso associati ad essi.
\end{itemize}

\subsubsubsection{Piano di progetto}
Il Piano di Progetto è un documento che si propone di delineare la pianificazione e la gestione delle attività necessarie per portare a termine il progetto.
Esso comprende le seguenti informazioni:
\begin{itemize}
	\item \textbf{Analisi dei Rischi}: identificazione delle potenziali problematiche che potrebbero emergere durante lo sviluppo e che potrebbero rallentare o ostacolare il progresso del progetto. Il gruppo si impegna a fornire soluzioni per tali problemi il prima possibile. I rischi sono classificati in due categorie principali: rischi organizzativi e rischi tecnologici;
	\item \textbf{Modello di sviluppo}: descrizione dell'approccio metodologico e strutturato adottato dal gruppo per lo sviluppo del prodotto;
	\item \textbf{Pianificazione}: delineamento dei periodi temporali, con gli eventi e le attività correlate, all'interno di un calendario. Per ogni periodo, saranno specificate le attribuzioni dei ruoli e una stima dell'impegno richiesto da ciascun membro del gruppo per svolgere le rispettive attività;
	\item \textbf{Preventivo}: stima della durata di ciascun periodo, indicando il tempo necessario per completare tutte le attività pianificate;
	\item \textbf{Consultivo}: analisi del lavoro effettivamente svolto rispetto a quanto preventivato, al fine di ottenere uno stato di avanzamento del progetto al termine di ciascun periodo.
\end{itemize}

\subsubsubsection{Piano di qualifica}
Il Piano di Qualifica è un documento che dettaglia le responsabilità e le attività del Verificatore all'interno del progetto.
Queste attività sono cruciali per garantire la qualità del prodotto software in fase di sviluppo.
Il Piano di Qualifica funge da guida essenziale per la gestione del processo di sviluppo, poiché assicura che il prodotto finale soddisfi le specifiche richieste e le aspettative del committente, monitorando il suo progresso rispetto agli obiettivi stabiliti.
Ogni membro del team coinvolto nel progetto farà riferimento a questo documento per garantire il raggiungimento della qualità desiderata.
Il Piano di Qualifica è strutturato in diverse sezioni, tra cui:
\begin{itemize}
	\item \textbf{Qualità di processo}: definisce i parametri e le metriche per garantire processi di alta qualità;
	\item \textbf{Qualità del prodotto}: stabilisce i parametri e le metriche per assicurare un prodotto finale di alta qualità;
	\item \textbf{Test}: descrive i test necessari per verificare il soddisfacimento dei requisiti nel prodotto;
	\item \textbf{Valutazioni per il miglioramento}: riporta le attività di verifica svolte e le problematiche riscontrate durante lo sviluppo del software, con l'obiettivo di identificare aree di miglioramento.
\end{itemize}

\subsubsubsection{Glossario}
Il \textit{Glossario} è una raccolta di termini presenti nei documenti, accompagnati dalle relative definizioni, specialmente quando il loro significato potrebbe non essere immediatamente chiaro.
Serve a prevenire eventuali ambiguità e facilitare la comunicazione tra i membri del gruppo.
\subsubsubsection{Lettera di presentazione}
La Lettera di Presentazione è il documento attraverso il quale il gruppo \textit{7Last} manifesta l'intenzione di partecipare alla fase di revisione del prodotto software.
Questo documento elenca la documentazione disponibile per i committenti e il proponente, nonché i termini concordati per la consegna del prodotto finito.


\subsubsection{Strumenti}
Di seguito sono descritti gli strumenti software impiegati nel processo di fornitura.
\subsubsubsection{Discord}
Il gruppo utilizza Discord come piattaforma per le riunioni interne e come un metodo informale per contattare l'azienda proponente tramite messaggistica e videochat.

\subsubsubsection{Latex}
LaTeX è un sistema di preparazione di documenti utilizzato principalmente per la creazione di documenti tecnici e scientifici.

\subsubsubsection{Git}
Git è un software per il controllo di versione.

\subsubsection{GitHub}
GitHub è un servizio di hosting per progetti software.

\subsection{Sviluppo}
\subsubsection{Introduzione}
L'ISO/IEC 12207:1995 fornisce le linee guida per il processo di sviluppo, che comprende attività cruciali come analisi, progettazione, codifica, integrazione, testing, installazione e accettazione.
È essenziale eseguire tali attività in stretta conformità alle linee guida e ai requisiti stabiliti nel contratto con il cliente, garantendo così un'implementazione accurata e conforme alle specifiche richieste.
\subsubsection{Analisi dei requisiti}

\subsubsubsection{Descrizione}
L'analisi dei requisiti rappresenta un'attività cruciale nello sviluppo del software poiché fornisce le fondamenta per il design, l'implementazione e i test del sistema.
Secondo lo standard ISO/IEC 12207:1995, l'obiettivo dell'analisi dei requisiti è comprendere e definire in modo completo le necessità del cliente e del sistema.
Questa attività richiede di rispondere a domande fondamentali come "Qual è il contesto?", "Quali sono i requisiti essenziali del cliente?", e implica una comprensione approfondita del contesto e la definizione chiara degli obiettivi, dei vincoli e dei requisiti sia tecnici che funzionali.

\subsubsubsection{Obiettivi}
\begin{itemize}
	\item Collaborare con la proponente per definire gli obiettivi del prodotto al fine di soddisfare le aspettative, includendo l'identificazione, la documentazione e la validazione dei requisiti funzionali e non funzionali;
	\item Promuovere una comprensione condivisa tra tutte le parti interessate;
	\item Consentire una stima accurata delle tempistiche e dei costi del progetto;
	\item Fornire ai progettisti requisiti chiari e facilmente comprensibili;
	\item Agevolare l'attività di verifica e di test fornendo indicazioni pratiche di riferimento.
\end{itemize}

\subsubsubsection{Documentazione}
È responsabilità degli analisti condurre l'analisi dei requisiti, redigendo un documento omonimo che deve comprendere i seguenti elementi:
\begin{itemize}
	\item \textbf{Introduzione}: presentazione e scopo del documento stesso;
	\item \textbf{Descrizione}:  analisi approfondita del prodotto, includendo:
	\begin{itemize}
		\item Obbiettivi del prodotto;
		\item Funzionalità del prodotto;
		\item Caratteristiche utente;
		\item Tecnologie impiegate.
	\end{itemize}
	\item Casi d'uso: descrizione delle funzionalità offerte dal sistema dal punto di vista dell'utente, includendo:
	\begin{itemize}
		\item utenti esterni al sistema;
		\item Elenco dei casi d'uso, comprensivo di:
		\begin{itemize}
			\item Descrizioni dei casi d'uso in formato testuale;
			\item Diagrammid dei casi d'uso.
		\end{itemize}
		\item Eventuali diagrammi di attività per facilitare la comprensione dei processi relativi alle funzionalità.
	\end{itemize}
	\item \textbf{Requisiti}:
	\begin{itemize}
		\item Requisiti funzionali;
		\item Requisiti qualitativi;
		\item Requisiti di vincolo.
	\end{itemize}
\end{itemize}

\subsubsubsection{Casi d'uso}
I casi d'uso forniscono una dettagliata descrizione delle funzionalità del sistema dal punto di vista degli utenti, delineando come il sistema risponde a specifiche azioni o scenari.
Essenzialmente, i casi d'uso sono strumenti utilizzati nell'analisi dei requisiti per catturare e illustrare chiaramente e comprensibilmente come gli utenti interagiranno con il software e quali saranno i risultati di tali interazioni. \\
\\
Ogni caso d'uso testuale deve includere:
\begin{enumerate}
	\item \textbf{Identificativo}: \begin{center}\textbf{UC [Numero caso d'uso].[Numero sotto caso d'uso] - [Titolo]}\end{center} ad esempio \textbf{TODO: inserire esempio} \\ con
	\begin{itemize}
		\item \textbf{Numero caso d'uso}: identificativo numerico del caso d'uso;
		\item \textbf{Numero sotto caso d'uso}: identificativo numerico del sotto caso d'uso (presente solo se si tratta di un sotto caso d'uso);
		\item \textbf{Titolo}: breve e chiaro titolo del caso d'uso.
	\end{itemize}
	\item \textbf{Attore principale}: entità esterna che interagisce attivamente con il sistema per soddisfare una propria necessità.
	\item \textbf{Attore secondario}: eventualmente, un'entità esterna che non interagisce attivamente con il sistema, ma all'interno del caso d'uso consente al sistema di soddisfare la necessità dell'attore principale.
	\item \textbf{Descrizione}: una breve descrizione della funzionalità, se necessaria.
	\item \textbf{Scenario principale}: una sequenza di eventi che si verificano quando un attore interagisce con il sistema per raggiungere l'obiettivo del caso d'uso (postcondizioni).
	\item \textbf{Estensioni}: eventuali scenari alternativi che si verificano in seguito a una o più specifiche condizioni, portando il flusso del caso d'uso a non raggiungere le postcondizioni.
	\item \textbf{Precondizioni}: lo stato in cui deve trovarsi il sistema affinché la funzionalità sia disponibile per l'attore.
	\item \textbf{Postcondizioni}: lo stato in cui si trova il sistema dopo l'esecuzione dello scenario principale.
	\item \textbf{User story associata}: una breve descrizione di una funzionalità del software, scritta dal punto di vista dell'utente, che fornisce contesto, obiettivi e valore.
	\begin{itemize}
		\item L'user story viene scritta nella forma: "Come [utente] desidero poter [funzionalità] per [valore aggiunto]".
	\end{itemize}
\end{enumerate}

\subsubsubsection{Diagrammi dei casi d'uso}
I diagrammi dei casi d'uso sono strumenti grafici che consentono di rappresentare in modo chiaro e intuitivo le funzionalità fornite dal sistema dal punto di vista dell'utente. Inoltre, permettono di individuare e comprendere rapidamente le relazioni e le interazioni tra i diversi casi d'uso, offrendo una visione generale delle funzionalità del sistema.
Questi diagrammi si concentrano sulla descrizione delle funzionalità del sistema dal punto di vista degli utenti, senza entrare nei dettagli implementativi. La loro principale finalità è quella di evidenziare le interazioni esterne al sistema, fornendo una visione focalizzata sulle funzionalità e sull'interazione dell'utente con il sistema stesso.
Un diagramma dei casi d'uso fornisce una panoramica visuale delle principali interazioni tra gli attori e il sistema, agevolando la comprensione dei requisiti funzionali del sistema e la comunicazione tra le parti interessate del progetto.
Di seguito sono elencati i principali componenti di un diagramma dei casi d'uso:
\begin{itemize}
	\item \textbf{Attori}: Gli attori sono rappresentati come entità esterne al sistema con cui interagisce e possono includere utenti umani, altri software o componenti esterni. Sono simboleggiati come "stickman" al di fuori del rettangolo che delimita il sistema.
	\begin{center}
		\includegraphics*[width=4cm]{../../../images/norme_di_progetto/attore.png}
	\end{center} \newpage
	\item \textbf{Casi d'uso}: I casi d'uso sono rappresentati come ovali all'interno del rettangolo che delimita il sistema e descrivono le funzionalità offerte dal sistema dal punto di vista dell'utente. Ogni caso d'uso è associato a uno o più attori e descrive uno scenario specifico in cui l'utente interagisce con il sistema per raggiungere un obiettivo specifico.
	\begin{center}
		\includegraphics*[width=10cm]{../../../images/norme_di_progetto/casoDiUso.png}
	\end{center}
	\item \textbf{Sottocasi d'uso}: I sottocasi d'uso sono casi d'uso che rappresentano scenari specifici all'interno di un caso d'uso principale. Sono rappresentati come ovali all'interno del caso d'uso principale e descrivono azioni o funzionalità aggiuntive necessarie per completare il caso d'uso principale.
	\begin{center}
		\includegraphics*[width=17cm]{../../../images/norme_di_progetto/sottocasiDiUso.png}
	\end{center} \newpage
	\item \textbf{Sistema}: Il sistema è rappresentato come un rettangolo che delimita i casi d'uso e gli attori. Questo simbolo rappresenta il sistema software che offre le funzionalità descritte dai casi d'uso.
	\begin{center}
		\includegraphics*[width=8cm]{../../../images/norme_di_progetto/sistema.png}
	\end{center}
	\item \textbf{Relazioni tra Attori e Casi d'Uso}
	\begin{itemize}
		\item \textbf{Associazione}: Una linea tratteggiata tra un attore e un caso d'uso indica un'associazione tra l'attore e il caso d'uso, che indica che l'attore è coinvolto nel caso d'uso.
		\begin{center}
			\includegraphics*[width=15cm]{../../../images/norme_di_progetto/associazione.png}
		\end{center}
	\end{itemize} \newpage
	\item \textbf{Relazioni tra Attori}
	\begin{itemize}
		\item \textbf{Generalizzazione}: Una freccia con una linea continua tra due attori indica una relazione di generalizzazione, che indica che un attore è un tipo specializzato di un altro attore.
		\begin{center}
			\includegraphics*[width=2cm]{../../../images/norme_di_progetto/generalizzazioneTraAttori.png}
		\end{center}
	\end{itemize} \newpage
	\item \textbf{Relazioni tra Casi d'Uso}
	\begin{itemize}
		\item \textbf{Inclusione}: La relazione di inclusione indica che un caso d'uso (chiamato "includente") incorpora l'esecuzione di un altro caso d'uso (detto "incluso"). In pratica, quando un attore interagisce con il caso d'uso includente, il caso d'uso incluso viene attivato come parte integrante del primo. Questo meccanismo è utile per favorire il riutilizzo di funzionalità e evitare la duplicazione di logica in diversi casi d'uso. La relazione di inclusione è simboleggiata da una freccia tratteggiata che collega il caso d'uso incluso al caso d'uso includente.
		\begin{center}
			\includegraphics*[width=15cm]{../../../images/norme_di_progetto/inclusione.png}
		\end{center} \newpage
		\item \textbf{Estensione}: La relazione di estensione indica che un caso d'uso (chiamato "estendente") può estendere il comportamento di un altro caso d'uso (detto "esteso") in determinate circostanze. In pratica, il caso d'uso estendente può aggiungere funzionalità opzionali o alternative al caso d'uso esteso, senza modificarne il comportamento principale. La relazione di estensione è simboleggiata da una freccia tratteggiata che collega il caso d'uso esteso al caso d'uso estendente.
		\begin{center}
			\includegraphics*[width=15cm]{../../../images/norme_di_progetto/estensione.png}
		\end{center} \newpage
		\item \textbf{Generalizzazione casi d'uso}: Una freccia con una linea continua tra due casi d'uso indica una relazione di generalizzazione, che indica che un caso d'uso è un tipo specializzato di un altro caso d'uso.
		\begin{center}
			\includegraphics*[width=15cm]{../../../images/norme_di_progetto/generalizzazioneCasiDiUso.png}
		\end{center}
	\end{itemize}
\end{itemize}

\subsubsubsection{Requisiti}
I requisiti di un software sono dettagliate specifiche documentate che delineano le funzionalità, le prestazioni, i vincoli e altri aspetti critici che il software deve soddisfare. Questi requisiti sono fondamentali per guidare lo sviluppo, il testing e la valutazione del prodotto, garantendo che risponda alle esigenze degli utenti e agli obiettivi del progetto. Essi comprendono sia i \textbf{requisiti funzionali}, che descrivono le funzionalità necessarie, sia i \textbf{requisiti non funzionali}, che definiscono criteri di prestazione, qualità, sicurezza e vincoli del sistema.\\
Una definizione precisa dei requisiti è essenziale: devono essere chiari e rispondere completamente alle aspettative del cliente o del proponente.

Ogni requisito è costituito da:
\begin{enumerate}
	\item \textbf{identificativo} nel formato \begin{center}\textbf{R[Abbreviazione Tipologia Requisito][Codice]}\end{center} con:\\
	\begin{itemize}
		\item \textbf{Abbreviazione Tipologia Requisito}: indica la tipologia del requisito, che può essere:
		\begin{itemize}
			\item \textbf{RF}: requisito funzionale;
			\item \textbf{RQ}: requisito di qualità;
			\item \textbf{RV}: requisito di vincolo.
		\end{itemize}
		\item \textbf{Codice}: numero progressivo che identifica univocamente il requisito.
	\end{itemize}
	\item \textbf{Importanza}: indica il grado di importanza del requisito, che può essere:
	\begin{itemize}
		\item \textbf{Obbligatorio}: requisito essenziale per il funzionamento del sistema;
		\item \textbf{Desiderabile}: requisito che apporta valore aggiunto al sistema, ma non essenziale;
		\item \textbf{Opzionale}: requisito che può essere implementato in un secondo momento.
	\end{itemize}
	\item \textbf{Descrizione}: descrizione chiara e dettagliata che fornisce una spiegazione del comportamento o della funzionalità richiesta.
	\item \textbf{Fonte}: indica la fonte da cui è stato identificato il requisito, che può essere:\\
	\begin{itemize}
		\item \textbf{Capitolato}: requisito identificato direttamente dal capitolato d'appalto;
		\item \textbf{Verbale Interno}: requisito identificato durante un incontro interno;
		\item \textbf{Verbale Esterno}: requisito identificato durante un incontro con il proponente.
	\end{itemize}
	\item \textbf{Casi d'uso}: elenco dei casi d'uso che soddisfano il requisito.
\end{enumerate}


\subsubsubsection{Metriche}
Nell'analisi dei requisiti, le metriche sono strumenti essenziali per valutare, misurare e gestire diversi aspetti dei requisiti di un sistema o di un progetto. Grazie a queste metriche, è possibile garantire che i requisiti siano esaustivi, precisi, coerenti e comprensibili.
\\
\begin{table}[!h]
    \centering
    \begin{tabular}{|c|c|c|}
    \hline
    \textbf{Metrica} & \textbf{Abbreviazione} & \textbf{Riferimento} \\
    \hline
    0M-CRO & Copertura dei Requisiti Obbligatori & \hyperlink{subsection.6.2}{0M-CRO} \\
    1M-CRD & Copertura dei Requisiti Desiderabili & \hyperlink{subsection.6.2}{1M-CRD} \\
    2M-CROP & Copertura dei Requisiti Opzionali & \hyperlink{subsection.6.2}{2M-CROP} \\
    \hline
    \end{tabular}
    \caption{Metriche per l'analisi dei requisiti (TEST)}
    \label{tab:2}
\end{table}

\subsubsubsection{Strumenti}
\textbf{StarUML} è un'applicazione software impiegata dal team per creare i diagrammi dei casi d'uso.


\subsubsection{Progettazione}
\subsubsubsection{Descrizione}
Il principale obiettivo dell'attività di progettazione è individuare la soluzione implementativa ottimale che soddisfi pienamente le esigenze di tutti gli stakeholder, considerando i requisiti e le risorse disponibili. La progettazione si pone la domanda fondamentale: "Qual è il modo migliore per realizzare ciò di cui c'è bisogno?". È essenziale definire l'architettura del prodotto prima di iniziare la fase di codifica, adottando un approccio basato sulla correttezza per costruzione anziché sulla correzione successiva. Tale approccio consente di gestire efficacemente la complessità del prodotto, garantendo una struttura robusta e coesa durante l'intero processo di sviluppo.

\subsubsubsection{Obiettivi}
L'obiettivo principale è garantire che i requisiti siano soddisfatti attraverso un sistema di qualità definito dall'architettura del prodotto. Ciò comporta:
\begin{itemize}
	\item Individuare componenti modulari che rispettino i requisiti, con specifiche chiare e coerenti, e svilupparle utilizzando risorse sostenibili e costi contenuti;
	\item Organizzare le componenti in modo che siano facilmente comprensibili e manutenibili, garantendo una struttura coesa e ben organizzata;
	\item Definire un'architettura che supporti l'evoluzione del prodotto, consentendo l'aggiunta di nuove funzionalità e la correzione di eventuali errori;
\end{itemize}
Inizialmente, il team di progettazione eseguirà un'analisi approfondita per selezionare con cura le tecnologie più adeguate, valutandone attentamente i vantaggi, i limiti e le eventuali problematiche. Una volta individuate le tecnologie appropriate, si procederà allo sviluppo di un'architettura di alto livello per comprendere e delineare la struttura generale del prodotto, che fungerà da base iniziale per la realizzazione del \textit{Proof of Concept} (PoC). Questa architettura fornirà una visione panoramica del sistema, identificando i principali componenti, i flussi di dati e le interazioni tra di essi, ponendo particolare attenzione alla flessibilità del sistema per eventuali modifiche future. \\Successivamente, si darà il via allo sviluppo del PoC, una parte cruciale della \textit{Technology Baseline}, per valutare le decisioni prese riguardo all'architettura e alle tecnologie adottate, e per verificare la loro congruenza con gli obiettivi e le specifiche del progetto. Dopo lo sviluppo e un'attenta analisi del PoC, si procederà con ulteriori iterazioni, apportando miglioramenti, aggiustamenti e integrazioni fino a raggiungere un design completo. Questo design sarà fondamentale per lo sviluppo del \textit{Minimum Viable Product} (MVP), che rappresenterà una versione essenziale e funzionale del prodotto e sarà parte integrante della \textit{Product Baseline}.

\subsubsubsection{Documentazione}
\textbf{Specifica tecnica}\\ Il documento fornisce una visione dettagliata del design definitivo del prodotto e offre istruzioni chiare agli sviluppatori per implementare correttamente la soluzione software, seguendo i requisiti e le specifiche indicate. Questo aiuta a semplificare il processo di sviluppo del software, riducendo la complessità e le ambiguità, e assicurando che il prodotto finale sia in linea con le aspettative del cliente e funzioni in modo ottimale. Tra gli elementi chiave inclusi in questo documento vi sono:
\begin{itemize}
	\item \textbf{Tecnologie utilizzate}: elenco delle tecnologie, dei framework e degli strumenti impiegati per lo sviluppo del prodotto;
	\item \textbf{Architettura logica}: descrizione dettagliata della struttura logica del sistema, con particolare attenzione ai componenti principali, ai flussi di dati e alle interazioni tra di essi;
	\item \textbf{Architettura di deployment}: rappresentazione grafica dell'architettura del sistema, con indicazioni sulle risorse hardware e software necessarie per il corretto funzionamento del prodotto;
	\item \textbf{Design pattern}: descrizione dei design pattern utilizzati per risolvere problemi comuni e ricorrenti durante lo sviluppo del software;
	\item \textbf{Vincoli e linee guida}: specifiche restrizioni e regole da seguire durante lo sviluppo del prodotto, per garantire coerenza e uniformità nel codice.
	\item \textbf{Procedure di testing e validazione}: indicazioni sulle procedure e gli strumenti da utilizzare per verificare e validare il prodotto, garantendo che soddisfi i requisiti e le aspettative del cliente.
	\item \textbf{Requisiti tecnici}: elenco dettagliato dei requisiti tecnici che il prodotto deve soddisfare, con indicazioni sulle funzionalità, le prestazioni e le caratteristiche richieste.
\end{itemize}

\subsubsubsection{Qualità dell'architettura}
\begin{itemize}
	\item \textbf{Sufficienza}: l'architettura deve soddisfare tutti i requisiti funzionali e non funzionali del sistema, garantendo che tutte le funzionalità richieste siano implementate correttamente e che il sistema funzioni in modo ottimale.
	\item \textbf{Comprensibilità}: l'architettura deve essere chiara, ben strutturata e facilmente comprensibile, in modo che gli sviluppatori possano capire facilmente come il sistema è organizzato e come funziona;
	\item \textbf{Modularità}: l'architettura deve essere modulare, con componenti ben definiti e indipendenti, in modo che possano essere facilmente riutilizzati e sostituiti senza influenzare il resto del sistema;
	\item \textbf{Robustezza}: l'architettura deve essere robusta e resistente agli errori, in modo che il sistema possa gestire eventuali problemi o malfunzionamenti senza interrompere il funzionamento del sistema;
	\item \textbf{Flessibilità}: l'architettura deve essere flessibile e adattabile, in modo che il sistema possa essere facilmente modificato e ampliato per soddisfare nuove esigenze e requisiti;
	\item \textbf{Efficienza}: l'architettura deve essere efficiente e ottimizzata, in modo che il sistema possa funzionare in modo rapido ed efficiente, senza sprechi di risorse;
	\item \textbf{Riusabilità}: l'architettura deve essere progettata per favorire la riutilizzabilità dei componenti, in modo che possano essere facilmente utilizzati in altri contesti e progetti;
	\item \textbf{Affidabilità}: l'architettura deve essere affidabile e sicura, in modo che il sistema possa garantire la corretta esecuzione delle funzionalità e la protezione dei dati e delle informazioni;
	\item \textbf{Disponibilità}: l'architettura deve garantire la disponibilità del sistema, in modo che possa essere sempre accessibile e operativo per gli utenti;
	\item \textbf{Safety}: l'architettura deve garantire la sicurezza del sistema, in modo che possa proteggere i dati e le informazioni sensibili in seguito a malfunzionamenti;
	\item \textbf{Security}: l'architettura deve garantire la sicurezza del sistema, in modo che possa proteggere i dati e le informazioni sensibili da accessi non autorizzati.
	\item \textbf{semplicità}: l'architettura deve essere semplice e intuitiva, in modo che possa essere facilmente compresa e utilizzata dagli sviluppatori e dagli utenti.
	\item \textbf{Coesione}: l'architettura deve essere coesa, con componenti ben definiti e correlati tra loro, in modo che possano lavorare insieme in modo efficace e armonioso;
	\item \textbf{Incapsulazione}: l'architettura deve essere incapsulata, con componenti ben definiti e indipendenti, in modo che possano essere facilmente gestiti e mantenuti senza influenzare il resto del sistema;
	\item \textbf{Basso accoppiamento}: l'architettura deve avere un basso accoppiamento tra i componenti, in modo che possano essere facilmente sostituiti e modificati senza influenzare il resto del sistema;
\end{itemize}

\subsubsubsection{Diagrammi UML}
Vantaggi:
\begin{itemize}
	\item \textbf{Chiarezza nella comunicazione}: i diagrammi UML forniscono una rappresentazione visuale delle informazioni, facilitando la comprensione e la comunicazione tra gli stakeholder;
	\item \textbf{Standardizzazione}: UML è uno standard riconosciuto a livello internazionale, che consente di creare diagrammi coerenti e uniformi, garantendo una maggiore coerenza e comprensibilità;
	\item \textbf{Analisi e progettazione visiva}: i diagrammi UML consentono di analizzare e progettare il sistema in modo visuale, facilitando la comprensione e l'identificazione di problemi e soluzioni;
	\item \textbf{Modellazione e simulazione}: UML consente di modellare e simulare il sistema in modo visuale, facilitando la valutazione delle prestazioni e delle funzionalità del sistema;
	\item \textbf{Manutenzione facilitata}: i diagrammi UML semplificano la manutenzione del sistema, consentendo di identificare e risolvere facilmente problemi e bug;
	\item \textbf{Riduzione degli errori di progettazione}: UML aiuta a ridurre gli errori di progettazione, consentendo di identificare e correggere i problemi in modo rapido ed efficace;
	\item \textbf{Documentazione supportata}: i diagrammi UML forniscono una documentazione visuale del sistema, che facilita la comprensione e la consultazione delle informazioni.
\end{itemize}
A supporto della progettazione, il team utilizzerà i seguenti \textbf{diagrammi delle classi}.
\\\\
\textbf{Diagrammi delle classi}\\
Ogni diagramma delle classi rappresenta le proprietà e le relazioni tra le varie componenti di un sistema, offrendo una visione chiara e dettagliata della struttura del sistema.\\
Le classi sono rappresentate da rettangoli suddivisi in tre sezioni:
\begin{enumerate}
	\item \textbf{Nome della classe}: indica il nome della classe;
	\item \textbf{Attributi}: elenco degli attributi della classe, con il relativo tipo di dato, seguendo il formato: \\ \begin{center}\textbf{Visibilità Nome: Tipo [Molteplicità] = Valore di default}\end{center}
	\begin{itemize}
		\item \textbf{Visibilità}: indica il livello di accesso agli attributi, che può essere:
		\begin{itemize}
			\item \textbf{+}: pubblico;
			\item \textbf{-}: privato;
			\item \textbf{\#}: protetto;
			\item \textbf{\textasciitilde}: package.
		\end{itemize}
		\item \textbf{Nome}: nome dell'attributo. Deve essere rappresentativo, chiaro e deve seguire la notazione \textit{nomeAttributo: tipo}; \\ Se l'attributo è costante, il nome deve essere scritto in maiuscolo (es. \textit{PIGRECO: double});
		\item \textbf{Molteplicità}: nel caso di una sequenza di elementi come liste o array, indica il numero di elementi presenti, se questa non fosse conosciuta si utilizza il simbolo \textit{*} (es \textit{tipoAttributo[*]});
		\item \textbf{default}: valore di default dell'attributo.
	\end{itemize}
	\item \textbf{Metodi}: descrivono il comportamento della classe, seguendo il formato: \\ \begin{center}\textbf{Visibilità Nome(parametri): Tipo di ritorno}\end{center}
	\begin{itemize}
		\item \textbf{Visibilità}: indica il livello di accesso ai metodi, che può essere:
		\begin{itemize}
			\item \textbf{+}: pubblico;
			\item \textbf{-}: privato;
			\item \textbf{\#}: protetto;
			\item \textbf{\textasciitilde}: package.
		\end{itemize}
		\item \textbf{Nome}: nome del metodo. Deve essere rappresentativo, chiaro e deve seguire la notazione \textit{nomeMetodo(parametri): tipoRitorno};
		\item \textbf{Parametri}: elenco dei parametri del metodo, separati tramite virgola. Ogni parametro deve seguire la notazione \textit{nomeParametro: tipo};
		\item \textbf{Tipo di ritorno}: indica il tipo di dato restituito dal metodo.
	\end{itemize}
\end{enumerate}

\textbf{Convenzioni sui metodi}
\begin{itemize}
	\item \textbf{I metodi getter, setter} e i \textbf{costruttori} non vengono inclusi fra i metodi;
	\item \textbf{I metodi statici} sono sottolineati;
	\item \textbf{I metodi astratti} sono scritti in corsivo.
	\item L'\textbf{assenza di attributi o metodi} in una classe determina l'assenza delle relative sezioni nel diagramma.
\end{itemize}
\newpage
\textbf{Relazioni tra le classi}\\
I diagrammi delle classi possono includere diverse relazioni tra le classi, che rappresentano le interazioni e le dipendenze tra le varie componenti del sistema. Le principali relazioni tra le classi sono:
\begin{itemize}
	\item \textbf{Associazione}: indica una relazione tra due classi, in cui un'istanza di una classe è collegata a un'istanza di un'altra classe. L'associazione è rappresentata da una linea che collega le due classi, con una freccia che indica la direzione dell'associazione e le molteplicità possono essere rappresentate con numeri messi agli estremi della freccia.
	\begin{center}
		\includegraphics*[width=12cm]{../../../images/norme_di_progetto/associazioneClassi.png}
	\end{center}
	\item \textbf{Dipendenza}: indica una relazione in cui una classe dipende da un'altra classe, ad esempio se un metodo di una classe utilizza un'istanza di un'altra classe. La dipendenza è rappresentata da una linea tratteggiata che collega le due classi, con una freccia che indica la direzione della dipendenza.
	\begin{center}
		\includegraphics*[width=12cm]{../../../images/norme_di_progetto/dipendenzaClassi.png}
	\end{center}
	\newpage
	\item \textbf{Aggregazione}: indica una relazione in cui una classe è composta da una o più istanze di un'altra classe. L'aggregazione è rappresentata da una linea con un rombo vuoto che collega le due classi, con una freccia che indica la direzione dell'aggregazione.
	\begin{center}
		\includegraphics*[width=12cm]{../../../images/norme_di_progetto/aggregazioneClassi.png}
	\end{center}
	\item \textbf{Composizione}: indica una relazione in cui una classe è composta da una o più istanze di un'altra classe, ma le istanze sono strettamente legate e non possono esistere indipendentemente dalla classe principale. La composizione è rappresentata da una linea con un rombo pieno che collega le due classi, con una freccia che indica la direzione della composizione.
	\begin{center}
		\includegraphics*[width=10cm]{../../../images/norme_di_progetto/composizioneClassi.png}
	\end{center}
	\item \textbf{Generalizzazione}: indica una relazione in cui una classe è un tipo specializzato di un'altra classe. La generalizzazione è rappresentata da una linea con una freccia vuota che collega le due classi.
	\begin{center}
		\includegraphics*[width=10cm]{../../../images/norme_di_progetto/generalizzazioneClassi.png}
	\end{center}
	\newpage
	\item \textbf{Realizzazione}: indica una relazione in cui una classe implementa un'interfaccia o eredita da una classe astratta. La realizzazione è rappresentata da una linea che collega la classe all'interfaccia o alla classe astratta.
	\begin{center}
		\includegraphics*[width=8cm]{../../../images/norme_di_progetto/realizzazioneClassi.png}
	\end{center}
\end{itemize}
\subsubsubsection{Design pattern}
I design pattern rappresentano solide soluzioni a problemi ricorrenti di progettazione in determinati contesti, offrendo un approccio riutilizzabile che assicura la qualità e una rapida implementazione. Si adottano quando una soluzione ha dimostrato efficacia in un contesto specifico. Le guide dettagliate sull'applicazione dei pattern delineano il loro utilizzo ottimale, accompagnate da rappresentazioni grafiche, spiegazioni testuali della logica e descrizioni della loro utilità nell'architettura complessiva. Questa documentazione è essenziale per favorire una comprensione approfondita dell'integrazione dei design pattern nell'architettura generale e per prevenire errori di progettazione.

\subsubsubsection{Test}
Nel processo di sviluppo, il testing è cruciale per garantire la qualità del prodotto finale. Durante questa fase, vengono stabiliti i requisiti di testing, definiti i casi di test e i criteri di accettazione, che servono da strumenti per valutare il software.\\ L'obiettivo principale è individuare e risolvere eventuali problemi o errori nel software prima del rilascio del prodotto finale, assicurando che soddisfi le specifiche e le aspettative del cliente. I progettisti hanno il pieno controllo su questa attività, incluso il definire i test da eseguire. Nella sezione 3.2.4 (\textbf{TODO: controllare se è il numero giusto}) sono fornite descrizioni dettagliate delle varie tipologie di test e della terminologia associata, offrendo ulteriore chiarezza su questa fase critica del processo di sviluppo del software.

\subsubsubsection{Metriche}
\textbf{TODO: inserire metriche}

\subsubsubsection{Strumenti}
\textbf{StarUML} è un'applicazione software impiegata dal team per creare i diagrammi dei casi d'uso.

\subsubsection{Codifica}
\subsubsubsection{Descrizione}
Nel processo di sviluppo del software, la codifica è responsabilità del programmatore ed è il momento cruciale in cui le funzionalità richieste prendono vita. Durante questa fase, le idee e i concetti delineati dai progettisti vengono tradotti in codice, creando istruzioni e procedure eseguibili dai calcolatori.\\ È essenziale che i programmatori rispettino attentamente le linee guida e le norme stabilite per assicurare che il codice sia conforme alle specifiche e rifletta accuratamente le visioni iniziali dei progettisti.

\subsubsubsection{Obiettivi}
La fase di codifica mira a sviluppare un prodotto software che soddisfi le richieste del cliente e sia conforme agli accordi concordati. Il rispetto rigoroso delle norme assicura la creazione di codice di elevata qualità, semplificando la manutenzione, l'espansione e la verifica del software. Questo contribuisce costantemente al miglioramento complessivo della sua qualità.

\subsubsubsection{Norme di codifica}
Le seguenti norme sono state formalizzate in questa maniera:
\begin{itemize}
	\item \textbf{Nomi significativi}: i nomi delle variabili, delle costanti, delle classi e dei metodi devono essere significativi e rappresentativi della loro funzione, in modo da facilitare la comprensione del codice;
	\item \textbf{Indentazione e formattazione consistente}: il codice deve essere correttamente indentato e formattato, con l'uso di spazi e tabulazioni coerenti, per garantire una corretta leggibilità e comprensione;
	\item \textbf{Lunghezza dei metodi}: i metodi devono essere brevi e concisi, con un numero limitato di righe di codice, per garantire una maggiore chiarezza e facilità di manutenzione. I metodi dovrebbero essere lunghi quanto basta per svolgere una singola funzione. Questo favorisce:
	\begin{itemize}
		\item \textbf{Chiarezza}: i metodi brevi sono più facili da comprendere e da seguire;
		\item \textbf{manutenibilità}: i metodi brevi sono più facili da modificare e da aggiornare;
		\item \textbf{Comprensibilità}: i metodi brevi sono più facili da leggere e da interpretare;
		\item \textbf{Testabilità}: i metodi brevi sono più facili da testare e da validare.
		\item \textbf{Lunghezza del codice}: i file di codice sorgente devono essere brevi e concisi, con un numero limitato di righe di codice, per garantire una maggiore chiarezza e facilità di manutenzione.
		\item \textbf{Commenti}: il codice deve essere corredato da commenti chiari e significativi, che spiegano il funzionamento e lo scopo delle varie parti del codice, per facilitare la comprensione e la manutenzione.
		\item \textbf{Conformità ai principi SOLID}: il codice deve rispettare i principi SOLID, che promuovono la scrittura di codice pulito, modulare e manutenibile.
	\end{itemize}
\end{itemize}

\subsubsubsection{Strumenti}
\textbf{Visual Studio Code} è l'editor di codice utilizzato dal team per scrivere e modificare il codice sorgente.

\subsubsubsection{Metriche}
\textbf{TODO: inserire metriche}

\subsubsection{Configurazione dell'ambiente di esecuzione}
\subsubsubsection{Docker}
La redazione dei file Docker è inclusa nel processo di sviluppo del software. Rispettare le regole e le migliori pratiche di codifica per i file Docker è essenziale per assicurare la creazione, la gestione e la distribuzione efficiente dei container.\\
\begin{itemize}
	\item \textbf{Chiarezza e Coerenza}: i file Docker devono essere chiari e coerenti, con una struttura ben definita e una formattazione uniforme, per garantire una corretta organizzazione e facilitare la comprensione;
	\item \textbf{Versionamento}: i file Docker devono includere informazioni dettagliate sulla versione del software e dei componenti utilizzati, per garantire la compatibilità e la coerenza tra i diversi ambienti;
	\item \textbf{Sicurezza}: i file Docker devono essere sicuri e protetti da eventuali vulnerabilità, per garantire la protezione dei dati e delle informazioni sensibili, utilizzando immagini ufficiali e aggiornate, evitando di eseguire comandi con privilegi elevati quando possibile;
	\item \textbf{Efficienza}: i file Docker devono essere efficienti e ottimizzati, con un utilizzo corretto delle risorse e una gestione accurata dei container, per garantire prestazioni elevate e tempi di risposta rapidi;
	\item \textbf{Gestione delle variabili d'ambiente}: i file Docker devono includere variabili d'ambiente ben definite e gestite correttamente, per garantire la configurabilità e la flessibilità del sistema;
	\item \textbf{Logging e monitoraggio}: bisogna configurare i \textit{container} per registrare i log e monitorare le prestazioni, per garantire la visibilità e la tracciabilità delle attività del sistema;
	\item \textbf{Layering}: ridurre il numero di layer e mantenere il più possibile le dipendenze in comune tra i vari layer, per garantire una maggiore efficienza e una migliore gestione delle risorse.
	\item \textbf{Riduzione delle dimensioni delle immagini}: ridurre le dimensioni delle immagini Docker, eliminando i file e le dipendenze non necessarie, per garantire una maggiore efficienza e una migliore gestione delle risorse.
	\item \textbf{Documentazione}: i file Docker devono essere corredati da una documentazione dettagliata e aggiornata, che spieghi il funzionamento e la configurazione del sistema, per facilitare la comprensione e la manutenzione.
	\item \textbf{Testing}: i file Docker devono essere testati e validati accuratamente, per garantire che il sistema funzioni correttamente e soddisfi i requisiti e le aspettative del cliente.
\end{itemize}

\subsubsubsection{Strumenti}
\begin{itemize}
	\item \textbf{Docker}: Docker è una piattaforma open source che semplifica la creazione, la distribuzione e la gestione di container, fornendo un ambiente isolato e sicuro per eseguire applicazioni e servizi.
	\item \textbf{Visual Studio Code}: Visual Studio Code è un editor di codice leggero e versatile, che offre funzionalità avanzate per lo sviluppo del software, tra cui il supporto per Docker e la creazione di file Docker.
\end{itemize}