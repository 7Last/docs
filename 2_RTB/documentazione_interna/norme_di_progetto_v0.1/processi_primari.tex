\section{Processi primari}
\subsection{Fornitura}
\subsubsection{Introduzione}
Il processo di fornitura ha lo scopo di stabilire un accordo contrattuale tra il fornitore e il cliente, in cui vengono definiti i servizi che il fornitore si impegna a fornire e le condizioni di fornitura. 
Il processo di fornitura comprende le seguenti attività:
\subsubsection{Attività}
\begin{itemize}
	\itemsep0em
	\item \textbf{Preparazione della proposta}
	\item \textbf{Contrattazione}
	\item \textbf{Pianificazione}
	\item \textbf{Esecuzione}
	\item \textbf{Revisione}
	\item \textbf{Consegna}
\end{itemize}
TODO Mettere una descrizione di ogni attività?

\subsubsection{Comunicazioni con l'azienda proponente}
L'azienda proponente SyncLab mette a disposizione l'indirizzo di posta elettronica e il suo canale Discord per la comunicazione tramite messaggi e Google Meet per la comunzicazione attraverso incontri telematici.
Gli incontri telematici hanno una cadenza regolare di due settimane, con possibili incontri aggiuntivi richiesti dal gruppo in caso di necessità, come ad esempio chiarimenti riguardo al capitolato o alle tecnologie utilizzate.
Per ogni colloquio avvenuto con l'azienda proponente verrà fornito un verbale esterno che riporterà i vari argomenti discussi durante il colloquio.
I verbali saranno disponibili all'interno del \textit{repository} \url{https://github.com/7Last/docs}
\subsubsection{Documentazione fornita}
Di seguito saranno elencati i documenti che il gruppo \textit{7last} consegnerà all'azienda \textit{SyncLab} e ai committenti \textit{Prof. Tullio Vardanega} e \textit{Prof. Riccardo Cardin}.
\subsubsubsection{Valutazione dei capitolati}
Il presente documento offre una valutazione approfondita dei capitolati d'appalto presentati in data 2023-10-17. Per ciascun progetto, vengono esaminate le richieste del proponente, le possibili soluzioni e le eventuali criticità.
La valutazione si articola nelle seguenti sezioni:
\begin{itemize}
	\item \textbf{Descrizione}: viene elencato il nome del progetto, l'azienda proponente, i committenti e l'obbiettivo del progetto;
	\item \textbf{Dominio Applicativo}: viene descritto il contesto del progetto;
	\item \textbf{Dominio Tecnologico}: vengono descritte le tecnologie utilizzate per lo sviluppo del progetto;
	\item \textbf{Aspetti Positivi};
	\item \textbf{Aspetti Negativi};
\end{itemize}

\subsubsubsection{Analisi dei requisiti}
L'Analisi dei Requisiti è un documento completo che delinea i casi d'uso, i requisiti e le funzionalità necessarie per il prodotto software.
Il suo scopo principale è chiarire qualsiasi incertezza o ambiguità che potrebbe sorgere dopo la lettura del capitolato.
Questo documento include:
\begin{itemize}
	\item Una \textbf{descrizione dettagliata del prodotto};
	\item Un \textbf{elenco dei casi d'uso}: riporta tutti gli scenari possibili in cui il sistema software potrebbe essere utilizzato dagli utenti finali, descrivendo le azioni che gli utenti compiono nel sistema in modo da identificare requisiti non ovvi inizialmente;
	\item Un \textbf{elenco dei requisiti}: specifica tutti i vincoli richiesti dal proponente o dedotti in base all'analisi dei casi d'uso associati ad essi.
\end{itemize}

\subsubsubsection{Piano di progetto}
Il Piano di Progetto è un documento che si propone di delineare la pianificazione e la gestione delle attività necessarie per portare a termine il progetto.
Esso comprende le seguenti informazioni:
\begin{itemize}
	\item \textbf{Analisi dei Rischi}: identificazione delle potenziali problematiche che potrebbero emergere durante lo sviluppo e che potrebbero rallentare o ostacolare il progresso del progetto. Il gruppo si impegna a fornire soluzioni per tali problemi il prima possibile. I rischi sono classificati in due categorie principali: rischi organizzativi e rischi tecnologici;
	\item \textbf{Modello di sviluppo}: descrizione dell'approccio metodologico e strutturato adottato dal gruppo per lo sviluppo del prodotto;
	\item \textbf{Pianificazione}: delineamento dei periodi temporali, con gli eventi e le attività correlate, all'interno di un calendario. Per ogni periodo, saranno specificate le attribuzioni dei ruoli e una stima dell'impegno richiesto da ciascun membro del gruppo per svolgere le rispettive attività;
	\item \textbf{Preventivo}: stima della durata di ciascun periodo, indicando il tempo necessario per completare tutte le attività pianificate;
	\item \textbf{Consultivo}: analisi del lavoro effettivamente svolto rispetto a quanto preventivato, al fine di ottenere uno stato di avanzamento del progetto al termine di ciascun periodo.
\end{itemize}

\subsubsubsection{Piano di qualifica}
Il Piano di Qualifica è un documento che dettaglia le responsabilità e le attività del Verificatore all'interno del progetto.
Queste attività sono cruciali per garantire la qualità del prodotto software in fase di sviluppo.
Il Piano di Qualifica funge da guida essenziale per la gestione del processo di sviluppo, poiché assicura che il prodotto finale soddisfi le specifiche richieste e le aspettative del committente, monitorando il suo progresso rispetto agli obiettivi stabiliti.
Ogni membro del team coinvolto nel progetto farà riferimento a questo documento per garantire il raggiungimento della qualità desiderata.
Il Piano di Qualifica è strutturato in diverse sezioni, tra cui:
\begin{itemize}
	\item \textbf{Qualità di processo}: definisce i parametri e le metriche per garantire processi di alta qualità;
	\item \textbf{Qualità del prodotto}: stabilisce i parametri e le metriche per assicurare un prodotto finale di alta qualità;
	\item \textbf{Test}: descrive i test necessari per verificare il soddisfacimento dei requisiti nel prodotto;
	\item \textbf{Valutazioni per il miglioramento}: riporta le attività di verifica svolte e le problematiche riscontrate durante lo sviluppo del software, con l'obiettivo di identificare aree di miglioramento.
\end{itemize}

\subsubsubsection{Glossario}
Il \textit{Glossario} è una raccolta di termini presenti nei documenti, accompagnati dalle relative definizioni, specialmente quando il loro significato potrebbe non essere immediatamente chiaro.
Serve a prevenire eventuali ambiguità e facilitare la comunicazione tra i membri del gruppo.
\subsubsubsection{Lettera di presentazione}
La Lettera di Presentazione è il documento attraverso il quale il gruppo \textit{7Last} manifesta l'intenzione di partecipare alla fase di revisione del prodotto software.
Questo documento elenca la documentazione disponibile per i committenti e il proponente, nonché i termini concordati per la consegna del prodotto finito.


\subsubsection{Strumenti}
Di seguito sono descritti gli strumenti software impiegati nel processo di fornitura.
\subsubsubsection{Discord}
Il gruppo utilizza Discord come piattaforma per le riunioni interne e come un metodo informale per contattare l'azienda proponente tramite messaggistica e videochat.

\subsubsubsection{Latex}
LaTeX è un sistema di preparazione di documenti utilizzato principalmente per la creazione di documenti tecnici e scientifici.

\subsubsubsection{Git}
Git è un software per il controllo di versione.

\subsubsection{GitHub}
GitHub è un servizio di hosting per progetti software.

\subsection{Sviluppo}
\subsubsection{Introduzione}
L'ISO/IEC 12207:1995 fornisce le linee guida per il processo di sviluppo, che comprende attività cruciali come analisi, progettazione, codifica, integrazione, testing, installazione e accettazione.
È essenziale eseguire tali attività in stretta conformità alle linee guida e ai requisiti stabiliti nel contratto con il cliente, garantendo così un'implementazione accurata e conforme alle specifiche richieste.
\subsubsection{Analisi dei requisiti}

\subsubsubsection{Descrizione}
L'analisi dei requisiti rappresenta un'attività cruciale nello sviluppo del software poiché fornisce le fondamenta per il design, l'implementazione e i test del sistema.
Secondo lo standard ISO/IEC 12207:1995, l'obiettivo dell'analisi dei requisiti è comprendere e definire in modo completo le necessità del cliente e del sistema.
Questa attività richiede di rispondere a domande fondamentali come "Qual è il contesto?", "Quali sono i requisiti essenziali del cliente?", e implica una comprensione approfondita del contesto e la definizione chiara degli obiettivi, dei vincoli e dei requisiti sia tecnici che funzionali.

\subsubsubsection{Obiettivi}
\begin{itemize}
	\item Collaborare con la proponente per definire gli obiettivi del prodotto al fine di soddisfare le aspettative, includendo l'identificazione, la documentazione e la validazione dei requisiti funzionali e non funzionali;
	\item Promuovere una comprensione condivisa tra tutte le parti interessate;
	\item Consentire una stima accurata delle tempistiche e dei costi del progetto;
	\item Fornire ai progettisti requisiti chiari e facilmente comprensibili;
	\item Agevolare l'attività di verifica e di test fornendo indicazioni pratiche di riferimento.
\end{itemize}

\subsubsubsection{Documentazione}
È responsabilità degli analisti condurre l'analisi dei requisiti, redigendo un documento omonimo che deve comprendere i seguenti elementi:
\begin{itemize}
	\item \textbf{Introduzione}: presentazione e scopo del documento stesso;
	\item \textbf{Descrizione}:  analisi approfondita del prodotto, includendo:
	\begin{itemize}
		\item Obbiettivi del prodotto;
		\item Funzionalità del prodotto;
		\item Caratteristiche utente;
		\item Tecnologie impiegate.
	\end{itemize}
	\item Casi d'uso: descrizione delle funzionalità offerte dal sistema dal punto di vista dell'utente, includendo:
	\begin{itemize}
		\item utenti esterni al sistema;
		\item Elenco dei casi d'uso, comprensivo di:
		\begin{itemize}
			\item Descrizioni dei casi d'uso in formato testuale;
			\item Diagrammid dei casi d'uso.
		\end{itemize}
		\item Eventuali diagrammi di attività per facilitare la comprensione dei processi relativi alle funzionalità.
	\end{itemize}
	\item \textbf{Requisiti}:
	\begin{itemize}
		\item Requisiti funzionali;
		\item Requisiti qualitativi;
		\item Requisiti di vincolo.
	\end{itemize}
\end{itemize}

\subsubsubsection{Casi d'uso}
I casi d'uso forniscono una dettagliata descrizione delle funzionalità del sistema dal punto di vista degli utenti, delineando come il sistema risponde a specifiche azioni o scenari.
Essenzialmente, i casi d'uso sono strumenti utilizzati nell'analisi dei requisiti per catturare e illustrare chiaramente e comprensibilmente come gli utenti interagiranno con il software e quali saranno i risultati di tali interazioni. \\
\\
Ogni caso d'uso testuale deve includere:
\begin{enumerate}
	\item \textbf{Identificativo}: \begin{center}\textbf{UC [Numero caso d'uso].[Numero sotto caso d'uso] - [Titolo]}\end{center} ad esempio \textbf{TODO: inserire esempio} \\ con
	\begin{itemize}
		\item \textbf{Numero caso d'uso}: identificativo numerico del caso d'uso;
		\item \textbf{Numero sotto caso d'uso}: identificativo numerico del sotto caso d'uso (presente solo se si tratta di un sotto caso d'uso);
		\item \textbf{Titolo}: breve e chiaro titolo del caso d'uso.
	\end{itemize}
	\item \textbf{Attore principale}: entità esterna che interagisce attivamente con il sistema per soddisfare una propria necessità.
	\item \textbf{Attore secondario}: eventualmente, un'entità esterna che non interagisce attivamente con il sistema, ma all'interno del caso d'uso consente al sistema di soddisfare la necessità dell'attore principale.
	\item \textbf{Descrizione}: una breve descrizione della funzionalità, se necessaria.
	\item \textbf{Scenario principale}: una sequenza di eventi che si verificano quando un attore interagisce con il sistema per raggiungere l'obiettivo del caso d'uso (postcondizioni).
	\item \textbf{Estensioni}: eventuali scenari alternativi che si verificano in seguito a una o più specifiche condizioni, portando il flusso del caso d'uso a non raggiungere le postcondizioni.
	\item \textbf{Precondizioni}: lo stato in cui deve trovarsi il sistema affinché la funzionalità sia disponibile per l'attore.
	\item \textbf{Postcondizioni}: lo stato in cui si trova il sistema dopo l'esecuzione dello scenario principale.
	\item \textbf{User story associata}: una breve descrizione di una funzionalità del software, scritta dal punto di vista dell'utente, che fornisce contesto, obiettivi e valore.
	\begin{itemize}
		\item L'user story viene scritta nella forma: "Come [utente] desidero poter [funzionalità] per [valore aggiunto]".
	\end{itemize}
\end{enumerate}

\subsubsubsection{Diagrammi dei casi d'uso}
I diagrammi dei casi d'uso sono strumenti grafici che consentono di rappresentare in modo chiaro e intuitivo le funzionalità fornite dal sistema dal punto di vista dell'utente. Inoltre, permettono di individuare e comprendere rapidamente le relazioni e le interazioni tra i diversi casi d'uso, offrendo una visione generale delle funzionalità del sistema.
Questi diagrammi si concentrano sulla descrizione delle funzionalità del sistema dal punto di vista degli utenti, senza entrare nei dettagli implementativi. La loro principale finalità è quella di evidenziare le interazioni esterne al sistema, fornendo una visione focalizzata sulle funzionalità e sull'interazione dell'utente con il sistema stesso.
Un diagramma dei casi d'uso fornisce una panoramica visuale delle principali interazioni tra gli attori e il sistema, agevolando la comprensione dei requisiti funzionali del sistema e la comunicazione tra le parti interessate del progetto.
Di seguito sono elencati i principali componenti di un diagramma dei casi d'uso:
\begin{itemize}
	\item 
\end{itemize}

\subsubsubsection{Requisiti}
\subsubsubsection{Metriche}
\subsubsubsection{Strumenti}
\subsubsection{Progettazione}
\subsubsubsection{Descrizione}
\subsubsubsection{Obiettivi}
\subsubsubsection{Documentazione}
\subsubsubsection{Qualità dell'architettura}
\subsubsubsection{Diagrammi UML}
\subsubsubsection{Design pattern}
\subsubsubsection{Test}
\subsubsubsection{Metriche}
\subsubsubsection{Strumenti}
\subsubsection{Codifica}
\subsubsubsection{Descrizione}
\subsubsubsection{Obiettivi}
\subsubsubsection{Norme di codifica}
\subsubsubsection{Strumenti}
\subsubsubsection{Metriche}
\subsubsection{Configurazione dell'ambiente di esecuzione}
\subsubsubsection{Docker}
\subsubsubsection{Strumenti}