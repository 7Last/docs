\documentclass[italian,12pt]{article} %tipo di documento

%--------------variabili------------------%
\def\Title{Norme di Progetto}
\def\Author{7Last}
\def\Version{v0.2}
%-----------------------------------------%


\usepackage[left=2cm, right=2cm, bottom=3cm, top=3cm]{geometry}
\usepackage{fancyhdr}
\usepackage{graphicx}
\graphicspath{ {../../logo/} }
\usepackage{href-ul}
\usepackage{tikz}
\usepackage{tgadventor}
\usepackage[useregional=numeric,showseconds=true,showzone=false]{datetime2}
\usepackage{caption}
\usepackage{longtable}
\usepackage{xcolor}




\linespread{1.2}
\captionsetup[table]{labelformat=empty}
\geometry{headsep=1.5cm}

\renewcommand{\contentsname}{Indice}%imposto il nome dell'indice
\renewcommand\familydefault{\sfdefault}

%-------------------INIZIO DOCUMENTO--------------
\begin{document}

\newgeometry{left=2cm,right=2cm,bottom=2.1cm,top=2.1cm}
\begin{titlepage}
	\vspace*{.5cm}

	\vspace{2cm}
	{
		\centering
		{\bfseries\huge \Title\par}
		\bigbreak
		{\bfseries\Large \Subtitle\par}
		\bigbreak
		{\bfseries\large \Author\par}
		\bigbreak
		{\Date\;-\;\Version\par}
		\vfill

		\begin{center}
			\begin{tikzpicture}
				\clip (0,0) circle (2cm) node {\includegraphics[width=4cm]{logo.jpg}};
			\end{tikzpicture}
		\end{center}
	}

	\vfill

\end{titlepage}

\restoregeometry






















\newpage

\pagestyle{fancy}
\fancyhead{}
\lhead{
	\begin{tikzpicture}
		\clip (0,0) circle (0.5cm);
		\node at (0,0) {\includegraphics[width=1cm]{./../logo/logo.png}};
	\end{tikzpicture}%
}
\chead{\vspace{\fill}\Title\vspace{\fill}}
\rhead{\vspace{\fill}\Version\vspace{\fill}}


%-----------tabella revisioni-----------%
\begin{table}[!h]
	\caption{Versioni}
	\begin{center}
		\begin{tabular}{ c c c p{9cm} }
			\hline\\[-2ex]
			Ver. & Data       & Autore             & Descrizione\\
			\\[-2ex] \hline \\[-1.5ex] \\
			0.1  & 29/03/2024 & Raul Seganfreddo & Creazione documento, impostazione indici e introduzione\\
			\\[-1.5ex] \hline
		\end{tabular}
	\end{center}
\end{table}
%---------------------------------------%

\newpage

\tableofcontents

\newpage

\section{Introduzione}
\subsection{Scopo del documento}
Questo documento ha lo scopo di descrivere le regole e le procedure che il gruppo adotterà per lo svolgimento del progetto. Lo scopo quindi è quello di definire il \textit{Way of Working} del gruppo, in modo da garantire un lavoro efficiente e di qualità.\\
Il processo seguirà le linee guida descritte dallo standard ISO/IEC 12207:1995.

\subsection{Scopo del progetto}
Lo scopo del progetto è quello di realizzare una piattaforma di monitoraggio per una smart city, in grado di raccogliere e analizzare in tempo reale dati provenienti da diverse fonti, come sensori, dispositivi indossabili e macchine. La piattaforma avrà lo scopo di:
\begin{itemize}
	\itemsep0em
    \item Migliorare la qualità della vita dei cittadini: la piattaforma consentirà alle autorità locali di prendere decisioni informate e tempestive sulla gestione delle risorse e sull'implementazione di servizi, basandosi su dati reali e aggiornati.
    \item Coinvolgere i cittadini: i dati monitorati saranno resi accessibili al pubblico attraverso portali online e applicazioni mobili, permettendo ai cittadini di essere informati sullo stato della loro città e di partecipare attivamente alla sua gestione.
    \item Gestire il big data: la piattaforma sarà in grado di gestire grandi volumi di dati provenienti da diverse fonti, aggregandoli, normalizzandoli e analizzandoli per estrarre informazioni significative.
\end{itemize}
La piattaforma si baserà su tecnologie di data streaming processing per l'analisi in tempo reale dei dati e su una piattaforma OLAP per la loro archiviazione e visualizzazione. La parte "IoT" del progetto sarà simulata attraverso tool di generazione di dati realistici.

In sintesi, il progetto mira a creare una piattaforma che sia:
\begin{itemize}
    \item Efficiente: in grado di raccogliere e analizzare grandi volumi di dati in tempo reale.
    \item Efficace: in grado di fornire informazioni utili per la gestione della città e il miglioramento della qualità della vita dei cittadini.
    \item Accessibile: in grado di rendere i dati disponibili al pubblico in modo chiaro e comprensibile.
\end{itemize}
Il progetto si pone come obiettivo di contribuire allo sviluppo di città più intelligenti e sostenibili, in cui la tecnologia viene utilizzata per migliorare il benessere dei cittadini.

\subsection{Glossario}
Al fine di evitare ambiguità e di facilitare la comprensione del documento, si allega il \textit{Glossario} contenente la definizione dei termini tecnici e degli acronimi utilizzati.

\subsection{Riferimenti}
\subsubsection{Riferimenti normativi}
\begin{itemize}
	\item Glossario: \textbf{TODO inserire link}
	\item \textit{ISO/IEC 12207:1995}: \textbf{TODO inserire link}
\end{itemize}
\subsubsection{Riferimenti informativi}


\section{Processi primari}
\subsection{Fornitura}
Il processo di fornitura ha lo scopo di stabilire un accordo contrattuale tra il fornitore e il cliente, in cui vengono definiti i servizi che il fornitore si impegna a fornire e le condizioni di fornitura. 
Il processo di fornitura comprende le seguenti attività:
\begin{itemize}
	\itemsep0em
	\item \textbf{Preparazione della proposta}
	\item \textbf{Contrattazione}
	\item \textbf{Pianificazione}
	\item \textbf{Esecuzione}
	\item \textbf{Revisione}
	\item \textbf{Consegna}
\end{itemize}
TODO Mettere una descrizione di ogni attività?

\subsubsection{Introduzione}
\subsubsection{Attività}
\subsubsection{Comunicazioni con l'azienda proponente}
\subsubsection{Documentazione fornita}

\subsubsection{Aspettative}
Il gruppo \textit{7Last} si impegna ad instaurare un rapporto di collaborazione con il \textit{Sync Lab}, il committente del progetto, basato su trasparenza, comunicazione e rispetto reciproco, in modo da identificare e soddisfare al meglio le esigenze del proponente.
Il gruppo quindi intende ricevere feedback costante sul lavoro svolto, in modo da poter correggere eventuali errori e migliorare la qualità del prodotto finale e verificare che i vincoli individuati siano rispettati.

\subsection{Rapporti con il proponente}
Il proponente \textit{Sync Lab} fornisce al gruppo \textit{7Last} le seguenti informazioni:
\begin{itemize}
	\itemsep0em
	\item un indirizzo mail a cui inviare le comunicazioni;
	\item TODO capire dove fare le chiamate e come comunicare dal prossimo incontro
\end{itemize}

\subsubsection{Documentazione fornita}
Di seguito sono elencati i documenti forniti da \textit{7last} al proponente \textit{Sync Lab} e ai committenti \textit{Prof. Tullio Vardanega} e \textit{Prof. Riccardo Cardin}.
TODO Come si fanno le subsubsubsection?
TODO Elencare e descrivere i documenti

\subsubsection{Strumenti}
Per lo sviluppo del progetto e la gestione del processo di fornitura, il gruppo utilizzerà i seguenti strumenti:
\begin{itemize}
	\itemsep0em
	\item \textbf{GitHub}: per la gestione del codice sorgente e la collaborazione tra i membri del gruppo;
	\item \textbf{Telegram}: per la comunicazione interna tra i membri del gruppo a livello di chat;
	\item \textbf{Discord}: per la comunicazione attraverso chiamate vocali.
\end{itemize}

\subsubsection{Sviluppo}

\section{Processi di supporto}
\subsection{Documentazione}
\subsection{Verifica}
\subsection{Validazione}
\subsection{Gestione della configurazione}
\subsection{Gestione della qualità}

\section{Processi organizzativi}
\subsection{Gestione dei processi}
\subsection{Miglioramnento}
\subsection{Formazione}

\section{Standard ISO}

\section{Metriche di qualità del processo}
\subsection{Processi di supporto}
\subsection{Processi organizzativi}

\section{Metriche di qualità del prodotto}

\end{document}
