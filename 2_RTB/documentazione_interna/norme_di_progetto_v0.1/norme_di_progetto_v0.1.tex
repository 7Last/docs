\documentclass[italian,12pt]{article} %tipo di documento

%--------------variabili------------------%
\def\Title{Norme di Progetto}
\def\Author{7Last}
\def\Version{v0.2}
%-----------------------------------------%


\usepackage[left=2cm, right=2cm, bottom=3cm, top=3cm]{geometry}
\usepackage{fancyhdr}
\usepackage{graphicx}
\graphicspath{ {../../logo/} }
\usepackage{href-ul}
\usepackage{tikz}
\usepackage{tgadventor}
\usepackage[useregional=numeric,showseconds=true,showzone=false]{datetime2}
\usepackage{caption}
\usepackage{longtable}
\usepackage{xcolor}



% Definizione delle nuove classi di titolo
\titleclass{\subsubsubsection}{straight}[\subsection]
\titleclass{\subsubsubsubsection}{straight}[\subsubsubsection]
\titleclass{\subsubsubsubsubsection}{straight}[\subsubsubsubsection] % nuovo livello

% Creazione dei nuovi contatori
\newcounter{subsubsubsection}[subsubsection]
\newcounter{subsubsubsubsection}[subsubsubsection]
\newcounter{subsubsubsubsubsection}[subsubsubsubsection] % nuovo livello

% Rinnovo dei comandi per la formattazione dei numeri delle sezioni
\renewcommand\thesubsubsubsection{\thesubsubsection.\arabic{subsubsubsection}}
\renewcommand\thesubsubsubsubsection{\thesubsubsubsection.\arabic{subsubsubsubsection}}
\renewcommand\thesubsubsubsubsubsection{\thesubsubsubsubsection.\arabic{subsubsubsubsubsection}} % nuovo livello
\renewcommand\theparagraph{\thesubsubsubsubsubsection.\arabic{paragraph}} % opzionale; utile se i paragrafi devono essere numerati

% Formattazione dei titoli delle sezioni
\titleformat{\subsubsubsection}
  {\normalfont\normalsize\bfseries}{\thesubsubsubsection}{1em}{}
\titleformat{\subsubsubsubsection}
  {\normalfont\normalsize\bfseries}{\thesubsubsubsubsection}{1em}{}
\titleformat{\subsubsubsubsubsection} % nuovo livello
  {\normalfont\normalsize\bfseries}{\thesubsubsubsubsubsection}{1em}{} 

% Spaziatura dei titoli delle sezioni
\titlespacing*{\subsubsubsection}
{0pt}{3.25ex plus 1ex minus .2ex}{1.5ex plus .2ex}
\titlespacing*{\subsubsubsubsection}
{0pt}{3.25ex plus 1ex minus .2ex}{1.5ex plus .2ex}
\titlespacing*{\subsubsubsubsubsection} % nuovo livello
{0pt}{3.25ex plus 1ex minus .2ex}{1.5ex plus .2ex}

\makeatletter
% Rinnovo dei comandi per la formattazione dei paragrafi e sottoparagrafi
\renewcommand\paragraph{\@startsection{paragraph}{6}{\z@}%
  {3.25ex \@plus1ex \@minus.2ex}%
  {-1em}%
  {\normalfont\normalsize\bfseries}}
\renewcommand\subparagraph{\@startsection{subparagraph}{7}{\parindent}%
  {3.25ex \@plus1ex \@minus .2ex}%
  {-1em}%
  {\normalfont\normalsize\bfseries}}

% Definizione dei livelli per il Table of Contents
\def\toclevel@subsubsubsection{4}
\def\toclevel@subsubsubsubsection{5}
\def\toclevel@subsubsubsubsubsection{6} % nuovo livello
\def\toclevel@paragraph{7}
\def\toclevel@subparagraph{8}

% Definizione della formattazione per il Table of Contents
\def\l@subsubsubsection{\@dottedtocline{4}{7em}{4em}}
\def\l@subsubsubsubsection{\@dottedtocline{5}{10em}{5em}}
\def\l@subsubsubsubsubsection{\@dottedtocline{6}{14em}{6em}} % nuovo livello
\def\l@paragraph{\@dottedtocline{7}{18em}{7em}}
\def\l@subparagraph{\@dottedtocline{8}{22em}{8em}}
\makeatother

% Impostazione della profondità dei numeri di sezione e del Table of Contents
\setcounter{secnumdepth}{6} % nuovo livello
\setcounter{tocdepth}{6} % nuovo livello


\linespread{1.2}
\captionsetup[table]{name=Tabella}
\geometry{headsep=1.5cm}

\renewcommand{\contentsname}{Indice}%imposto il nome dell'indice
\renewcommand\familydefault{\sfdefault}

\renewcommand{\listtablename}{Indice delle tabelle}%imposto il nome della lista tabelle
\renewcommand\familydefault{\sfdefault}

\renewcommand{\listfigurename}{Indice delle immagini}%imposto il nome della lista immagini
\renewcommand\familydefault{\sfdefault}

\addto\captionsitalian{\renewcommand{\figurename}{Figura}}

%-------------------INIZIO DOCUMENTO--------------
\begin{document}

\newgeometry{left=2cm,right=2cm,bottom=2.1cm,top=2.1cm}
\begin{titlepage}
	\vspace*{.5cm}

	\vspace{2cm}
	{
		\centering
		{\bfseries\huge \Title\par}
		\bigbreak
		{\bfseries\Large \Subtitle\par}
		\bigbreak
		{\bfseries\large \Author\par}
		\bigbreak
		{\Date\;-\;\Version\par}
		\vfill

		\begin{center}
			\begin{tikzpicture}
				\clip (0,0) circle (2cm) node {\includegraphics[width=4cm]{logo.jpg}};
			\end{tikzpicture}
		\end{center}
	}

	\vfill

\end{titlepage}

\restoregeometry






















\newpage

\pagestyle{fancy}
\fancyhead{}
\lhead{
	\begin{tikzpicture}
		\clip (0,0) circle (0.5cm);
		\node at (0,0) {\includegraphics[width=1cm]{./../logo/logo.png}};
	\end{tikzpicture}%
}
\chead{\vspace{\fill}\Title\vspace{\fill}}
\rhead{\vspace{\fill}\Version\vspace{\fill}}


%-----------tabella revisioni-----------%
\begin{table}[!h]
	\caption*{Versioni}
	\begin{center}
		\begin{tabular}{ l l l l l}
			\hline\\[-2ex]
			Ver. & Data       & Autore           & Verificatore   & Descrizione\\
			\\[-2ex] \hline \\[-1.5ex] \\
			0.2  & 05/04/2024 & Matteo Tiozzo    &  & Modificato tabella versioni\\
			0.1  & 29/03/2024 & Raul Seganfreddo &  & Creazione struttura documento\\
			\\[-1.5ex] \hline
		\end{tabular}
	\end{center}
\end{table}
%---------------------------------------%

\newpage

\tableofcontents

\listoftables

\listoffigures

\newpage

\section{Introduzione}
\setcounter{subsection}{0}
\subsection{Scopo del documento}
Il seguente documento si propone di definire la pianificazione e la gestione delle attività richieste per ultimare il progetto. Vengono esaminati in dettaglio elementi cruciali come l’\textit{Analisi dei Rischi}, il \textit{modello di sviluppo adottato}, la \textit{pianificazione delle attività}, la \textit{suddivisione dei ruoli}, oltre a \textit{stime dei costi} e delle \textit{risorse necessarie}.

\subsection{Scopo del prodotto}
Lo scopo principale del prodotto è quello di consentire a \textit{Sync Lab S.r.l.} di valutare la \\\textbf{fattibilità} di investire tempo e risorse nell'implementazione del progetto  \href{https://7last.github.io/docs/rtb/documentazione-interna/glossario\#synccity}{\textit{\textbf{SyncCity} \textsubscript{G}}- A \href{https://7last.github.io/docs/rtb/documentazione-interna/glossario\#smart-city}{smart city\textsubscript{G}} monitoring platform}. Questa soluzione, attraverso l'utilizzo di dispositivi IoT, consente un monitoraggio costante delle città. \href{https://7last.github.io/docs/rtb/documentazione-interna/glossario\#synccity}{SyncCity\textsubscript{G}} avrà lo scopo di monitorare e raccogliere dati da sensori posizionati nelle città, per poi analizzarli e fornire informazioni utili alla gestione della città. Il prodotto finale sarà un prototipo funzionale che consentirà la visualizzazione dei dati raccolti su un cruscotto.

\subsection{Glossario}
Per evitare qualsiasi ambiguità o malinteso sui termini utilizzati nel documento, verrà adottato un \href{https://7last.github.io/docs/rtb/documentazione-interna/glossario\#glossario}{glossario\textsubscript{G}}. Questo \href{https://7last.github.io/docs/rtb/documentazione-interna/glossario\#glossario}{glossario\textsubscript{G}} conterrà varie definizioni. Ogni termine incluso nel \href{https://7last.github.io/docs/rtb/documentazione-interna/glossario\#glossario}{glossario\textsubscript{G}} sarà indicato applicando uno stile specifico:
\begin{itemize}
    \item aggiungendo una "G" al pedice della parola;
    \item fornendo il link al \href{https://7last.github.io/docs/rtb/documentazione-interna/glossario\#glossario}{glossario\textsubscript{G}} online;
\end{itemize}

\subsection{Riferimenti}
    \subsubsection{Normativi}DA SISTEMARE
        \begin{itemize}
            \item \textbf{ISO/IEC 12207:2008} - Systems and software engineering - Software life cycle processes
            \item \textbf{ISO/IEC 31000:2009} - Risk management - Principles and guidelines
        \end{itemize}
    \subsubsection{Informativi}
        \begin{itemize}
            \item \href{https://7last.github.io/docs/rtb/documentazione-interna/glossario\#capitolato}{\textbf{Capitolato \textsubscript{G}}C6 - Sync City}: \textit{A \href{https://7last.github.io/docs/rtb/documentazione-interna/glossario\#smart-city}{smart city\textsubscript{G}} monitoring platform}
            \item \textbf{T2 - Processi di ciclo di vita del software}\\ https://www.math.unipd.it/~tullio/IS-1/2023/Dispense/T2.pdf;
            \item \textbf{T4 - Gestione di progetto}\\ Visibili a questo \uline{\href{https://www.math.unipd.it/~tullio/IS-1/2023/Dispense/T4.pdf}{link}};
            \item \href{https://7last.github.io/docs/rtb/documentazione-interna/glossario\#glossario}{\textbf{Glossario}\textsubscript{G}}\\ Visibile a questo \uline{\href{https://7last.github.io/docs/rtb/documentazione-interna/glossario}{link}};
        \end{itemize}
\subsection{Preventivo iniziale}
Il preventivo iniziale presentato durante la fase di candidatura è disponibile al seguente \uline{\href{https://github.com/7Last/docs/blob/main/1_candidatura/preventivo_costi_assunzione_impegni_v2.0.pdf}{riferimento}}. All'interno di questo documento viene calcolato il preventivo iniziale del progetto, pari a €12.670,00. Inoltre, si specifica che il gruppo \textit{7Last} stima di \textbf{completare} il prodotto entro e non oltre il \textbf{24 Settembre 2024}.

\newpage

\section{Processi primari}
\subsection{Fornitura}
\subsubsection{Introduzione}
Il processo di fornitura ha lo scopo di stabilire un accordo contrattuale tra il fornitore e il cliente, in cui vengono definiti i servizi che il fornitore si impegna a fornire e le condizioni di fornitura.
\subsubsection{Attività}
Il processo di fornitura comprende le seguenti attività:
\begin{itemize}
	\item \textbf{preparazione della proposta};
	\item \textbf{contrattazione};
	\item \textbf{pianificazione};
	\item \textbf{esecuzione};
	\item \textbf{revisione};
	\item \textbf{consegna}.
\end{itemize}

\subsubsection{Comunicazioni con l'azienda proponente}
La proponente \textit{SyncLab S.r.l.} mette a disposizione l'indirizzo di posta elettronica e il suo canale Discord per la comunicazione tramite messaggi e Google Meet per la comunzicazione attraverso incontri telematici.
Gli incontri telematici hanno una cadenza iniziale di due settimane, con possibili incontri aggiuntivi richiesti dal gruppo in caso di necessità, come ad esempio chiarimenti riguardo al capitolato o alle tecnologie utilizzate.
Per ogni colloquio avvenuto con l'azienda proponente verrà fornito un verbale esterno che riporterà i vari argomenti discussi durante il colloquio.
I verbali saranno disponibili all'interno della \href{https://github.com/7Last/docs}{\underline{repository}} del team \textit{7Last}.
\newpage
\subsubsection{Documentazione fornita}
Di seguito saranno elencati i documenti che il gruppo \textit{7last} consegnerà all'azienda \textit{SyncLab S.r.l.} e ai committenti \textit{Prof. Tullio Vardanega} e \textit{Prof. Riccardo Cardin}.
\subsubsubsection{Valutazione dei capitolati}
Il presente documento offre una valutazione approfondita dei capitolati d'appalto presentati in data 2023-10-17. Per ciascun progetto, vengono esaminate le richieste del proponente, le possibili soluzioni e le eventuali criticità.
La valutazione si articola nelle seguenti sezioni:
\begin{itemize}
	\item \textbf{descrizione}: viene elencato il nome del progetto, l'azienda proponente, i committenti e l'obiettivo del progetto;
	\item \textbf{dominio applicativo}: viene descritto il contesto del progetto;
	\item \textbf{dominio tecnologico}: vengono descritte le tecnologie utilizzate per lo sviluppo del progetto;
	\item \textbf{aspetti positivi};
	\item \textbf{aspetti negativi}.
\end{itemize}

\subsubsubsection{Analisi dei requisiti}
L'\textit{Analisi dei Requisiti} è un documento completo che delinea i casi d'uso, i requisiti e le funzionalità necessarie per il prodotto software.
Il suo scopo principale è chiarire qualsiasi incertezza o ambiguità che potrebbe sorgere dopo la lettura del capitolato.
Questo documento include:
\begin{itemize}
	\item \textbf{descrizione dettagliata del prodotto};
	\item \textbf{elenco dei casi d'uso}: riporta tutti gli scenari possibili in cui il sistema software potrebbe essere utilizzato dagli utenti finali, descrivendo le azioni che gli utenti compiono nel sistema in modo da identificare requisiti non ovvi inizialmente;
	\item \textbf{elenco dei requisiti}: specifica tutti i vincoli richiesti dal proponente o dedotti in base all'analisi dei casi d'uso associati ad essi.
\end{itemize}

\subsubsubsection{Piano di progetto}
Il \textit{Piano di Progetto} è un documento che si propone di delineare la pianificazione e la gestione delle attività necessarie per portare a termine il progetto.
Esso comprende le seguenti informazioni:
\begin{itemize}
	\item \textbf{Analisi dei Rischi}: identificazione delle potenziali problematiche che potrebbero emergere durante lo sviluppo e che potrebbero rallentare e/o ostacolare il progresso del progetto. Il gruppo si impegna a fornire soluzioni per tali problemi il prima possibile. I rischi sono classificati in tre categorie principali: rischi organizzativi, rischi tecnologici e rischi comunicativi;
	\item \textbf{Modello di sviluppo}: descrizione dell'approccio metodologico e strutturato adottato dal gruppo per lo sviluppo del prodotto;
	\item \textbf{Pianificazione}: delineamento dei periodi temporali, con gli eventi e le attività correlate, all'interno di un calendario. Per ogni periodo, saranno specificate le attribuzioni dei ruoli e una stima dell'impegno richiesto da ciascun membro del gruppo per svolgere le rispettive attività;
	\item \textbf{Preventivo}: stima della durata di ciascun periodo, indicando il tempo necessario per completare tutte le attività pianificate;
	\item \textbf{Consultivo}: analisi del lavoro effettivamente svolto rispetto a quanto preventivato, al fine di ottenere uno stato di avanzamento del progetto al termine di ciascun periodo.
\end{itemize}

\subsubsubsection{Piano di qualifica}
Il \textit{Piano di Qualifica} è un documento che dettaglia le responsabilità e le attività del Verificatore all'interno del progetto.
Queste attività sono cruciali per garantire la qualità del prodotto software in fase di sviluppo.
Il \textit{Piano di Qualifica} funge da guida essenziale per la gestione del processo di sviluppo, poiché assicura che il prodotto finale soddisfi le specifiche richieste e le aspettative del committente, monitorando il suo progresso rispetto agli obiettivi stabiliti.
Ogni membro del team coinvolto nel progetto farà riferimento a questo documento per garantire il raggiungimento della qualità desiderata.
Tale documento è strutturato in diverse sezioni, tra cui:
\begin{itemize}
	\item \textbf{Qualità di processo}: definisce i parametri e le metriche per garantire processi di alta qualità;
	\item \textbf{Qualità del prodotto}: stabilisce i parametri e le metriche per assicurare un prodotto finale di alta qualità;
	\item \textbf{Test}: descrive i test necessari per verificare il soddisfacimento dei requisiti nel prodotto;
	\item \textbf{Valutazioni per il miglioramento}: riporta le attività di verifica svolte e le problematiche riscontrate durante lo sviluppo del software, con l'obiettivo di identificare aree di miglioramento.
\end{itemize}

\subsubsubsection{Glossario}
Il \textit{Glossario} è una raccolta di termini presenti nei documenti, accompagnati dalle relative definizioni, specialmente quando il loro significato potrebbe non essere immediatamente chiaro.
Serve a prevenire eventuali ambiguità e facilitare la comunicazione tra i membri del gruppo.
\subsubsubsection{Lettera di presentazione}
La \textit{Lettera di Presentazione} è il documento attraverso il quale il gruppo \textit{7Last} manifesta l'intenzione di partecipare alla fase di revisione del prodotto software.
Questo documento elenca la documentazione disponibile per i committenti e il proponente, nonché i termini concordati per la consegna del prodotto finito.

\subsubsection{Strumenti}
Di seguito sono descritti gli strumenti software impiegati nel processo di fornitura.
\begin{itemize}
	\item Discord: il gruppo utilizza Discord come piattaforma per le riunioni interne e come un metodo informale per contattare l'azienda proponente tramite messaggistica e videochat;
	\item LaTeX: un sistema di preparazione di documenti utilizzato principalmente per la creazione di documenti tecnici e scientifici;
	\item Git: un sistema di controllo di versione distribuito utilizzato per il tracciamento delle modifiche ai documenti e per la collaborazione tra i membri del gruppo;
	\item GitHub: un servizio di hosting per progetti software che offre funzionalità di controllo di versione e collaborazione;
\end{itemize}

\subsection{Sviluppo}
\subsubsection{Introduzione}
L'ISO/IEC 12207:1995 fornisce le linee guida per il processo di sviluppo, che comprende attività cruciali come analisi, progettazione, codifica, integrazione, testing, installazione e accettazione.
È essenziale eseguire tali attività in stretta conformità alle linee guida e ai requisiti stabiliti nel contratto con il cliente, garantendo così un'implementazione accurata e conforme alle specifiche richieste.
\subsubsection{Analisi dei requisiti}
\subsubsubsection{Descrizione}
L'analisi dei requisiti rappresenta un'attività cruciale nello sviluppo del software poiché fornisce le fondamenta per il design, l'implementazione e i test del sistema.
Secondo lo standard ISO/IEC 12207:1995, l'obiettivo dell'analisi dei requisiti è comprendere e definire in modo completo le necessità del cliente e del sistema.
Questa attività richiede di rispondere a domande fondamentali come "Qual è il contesto?", "Quali sono i requisiti essenziali del cliente?", e implica una comprensione approfondita del contesto e la definizione chiara degli obiettivi, dei vincoli e dei requisiti sia tecnici che funzionali.

\subsubsubsection{Obiettivi}
\begin{itemize}
	\item Collaborare con la proponente per definire gli obiettivi del prodotto al fine di soddisfare le aspettative, includendo l'identificazione, la documentazione e la validazione dei requisiti funzionali e non funzionali;
	\item promuovere una comprensione condivisa tra tutte le parti interessate;
	\item consentire una stima accurata delle tempistiche e dei costi del progetto;
	\item fornire ai progettisti requisiti chiari e facilmente comprensibili;
	\item agevolare l'attività di verifica e di test fornendo indicazioni pratiche di riferimento.
\end{itemize}
\newpage
\subsubsubsection{Documentazione}
È responsabilità degli analisti condurre l'analisi dei requisiti, redigendo un documento comprendente i seguenti elementi:
\begin{itemize}
	\item \textbf{introduzione}: presentazione e scopo del documento stesso;
	\item \textbf{descrizione}:  analisi approfondita del prodotto, includendo:
	      \begin{itemize}
		      \item obiettivi del prodotto;
		      \item funzionalità del prodotto;
		      \item caratteristiche utente;
		      \item tecnologie impiegate;
	      \end{itemize}
	\item \textbf{casi d'uso}: descrizione delle funzionalità offerte dal sistema dal punto di vista dell'utente, includendo:
	      \begin{itemize}
		      \item utenti esterni al sistema;
		      \item elenco dei casi d'uso, comprensivo di:
		            \begin{itemize}
			            \item descrizioni dei casi d'uso in formato testuale;
			            \item diagrammi dei casi d'uso;
		            \end{itemize}
		      \item eventuali diagrammi di attività per facilitare la comprensione dei processi relativi alle funzionalità;
	      \end{itemize}
	\item \textbf{requisiti}:
	      \begin{itemize}
		      \item requisiti funzionali;
		      \item requisiti qualitativi;
		      \item requisiti di vincolo.
	      \end{itemize}
\end{itemize}

\subsubsubsection{Casi d'uso}
I casi d'uso forniscono una dettagliata descrizione delle funzionalità del sistema dal punto di vista degli utenti, delineando come il sistema risponde a specifiche azioni o scenari.
Essenzialmente, i casi d'uso sono strumenti utilizzati nell'analisi dei requisiti per catturare e illustrare chiaramente e comprensibilmente come gli utenti interagiranno con il software e quali saranno i risultati di tali interazioni. \\
\\
Ogni caso d'uso testuale deve includere:
\begin{enumerate}
	\item \textbf{Identificativo}:
	      \begin{center}
		      \textbf{UC-[identificativo\_caso\_principale].[identificativo\_sotto\_caso]}
	      \end{center}
	      \begin{itemize}
		      \item \textbf{Identificativo sotto caso}: identificativo numerico del sotto caso d'uso (presente solo se si tratta di un sotto caso d'uso);
		      \item \textbf{titolo}: breve e chiaro titolo del caso d'uso.
	      \end{itemize}
	\item \textbf{Attore principale}: entità esterna che interagisce attivamente con il sistema per soddisfare una propria necessità.
	\item \textbf{Scenario principale}: una sequenza di eventi che si verificano quando un attore interagisce con il sistema per raggiungere l'obiettivo del caso d'uso (postcondizioni).
	\item \textbf{Precondizioni}: lo stato in cui deve trovarsi il sistema affinché la funzionalità sia disponibile per l'attore.
	\item \textbf{Postcondizioni}: lo stato in cui si trova il sistema dopo l'esecuzione dello scenario principale.
	\item \textbf{User story}: una breve descrizione di una funzionalità del software, scritta dal punto di vista dell'utente, che fornisce contesto, obiettivi e valore.
	      \begin{itemize}
		      \item L'user story viene scritta nella forma: "Come [utente] desidero poter [funzionalità] per [valore aggiunto]".
	      \end{itemize}
\end{enumerate}

\subsubsubsection{Diagrammi dei casi d'uso}
I diagrammi dei casi d'uso sono strumenti grafici che consentono di rappresentare in modo chiaro e intuitivo le funzionalità fornite dal sistema dal punto di vista dell'utente. Inoltre, permettono di individuare e comprendere rapidamente le relazioni e le interazioni tra i diversi casi d'uso, offrendo una visione generale delle funzionalità del sistema.
Questi diagrammi si concentrano sulla descrizione delle funzionalità del sistema dal punto di vista degli utenti, senza entrare nei dettagli implementativi. La loro principale finalità è quella di evidenziare le interazioni esterne al sistema, fornendo una visione focalizzata sulle funzionalità e sull'interazione dell'utente con il sistema stesso.
Un diagramma dei casi d'uso fornisce una panoramica visuale delle principali interazioni tra gli attori e il sistema, agevolando la comprensione dei requisiti funzionali del sistema e la comunicazione tra le parti interessate del progetto.
Di seguito sono elencati i principali componenti di un diagramma dei casi d'uso:
\begin{itemize}
	\item \textbf{Attori}: Gli attori sono rappresentati come entità esterne al sistema con cui interagisce e possono includere utenti umani, altri software o componenti esterni. Sono simboleggiati come "stickman" al di fuori del rettangolo che delimita il sistema.
	      \begin{center}
		      \includegraphics*[width=4cm]{../../../images/norme_di_progetto/attore.png}
		      \captionof{figure}{Diagramma dei casi d'uso - Attore}
	      \end{center} \newpage
	\item \textbf{Casi d'uso}: I casi d'uso sono rappresentati come ovali all'interno del rettangolo che delimita il sistema e descrivono le funzionalità offerte dal sistema dal punto di vista dell'utente. Ogni caso d'uso è associato a uno o più attori e descrive uno scenario specifico in cui l'utente interagisce con il sistema per raggiungere un obiettivo specifico.
	      \begin{center}
		      \includegraphics*[width=10cm]{../../../images/norme_di_progetto/casoDiUso.png}
		      \captionof{figure}{Diagramma dei casi d'uso - Caso d'uso}
	      \end{center}
	      \newpage
	\item \textbf{Sottocasi d'uso}: I sottocasi d'uso sono casi d'uso che rappresentano scenari specifici all'interno di un caso d'uso principale. Sono rappresentati come ovali all'interno del caso d'uso principale e descrivono azioni o funzionalità aggiuntive necessarie per completare il caso d'uso principale.
	      \begin{center}
		      \includegraphics*[width=17cm]{../../../images/norme_di_progetto/sottocasiDiUso.png}
		      \captionof{figure}{Diagramma dei casi d'uso - Sottocaso d'uso}
	      \end{center} \newpage
	\item \textbf{Sistema}: Il sistema è rappresentato come un rettangolo che delimita i casi d'uso e gli attori. Questo simbolo rappresenta il sistema software che offre le funzionalità descritte dai casi d'uso.
	      \begin{center}
		      \includegraphics*[width=8cm]{../../../images/norme_di_progetto/sistema.png}
		      \captionof{figure}{Diagramma dei casi d'uso - Sistema}
	      \end{center}
	\item \textbf{Relazioni tra Attori e Casi d'Uso}
	      \begin{itemize}
		      \item \textbf{Associazione}: Una linea tratteggiata tra un attore e un caso d'uso indica un'associazione tra l'attore e il caso d'uso, che indica che l'attore è coinvolto nel caso d'uso.
		            \begin{center}
			            \includegraphics*[width=15cm]{../../../images/norme_di_progetto/associazione.png}
			            \captionof{figure}{Diagramma dei casi d'uso - Associazione}
		            \end{center}
	      \end{itemize} \newpage
	\item \textbf{Relazioni tra Attori}
	      \begin{itemize}
		      \item \textbf{Generalizzazione}: Una freccia con una linea continua tra due attori indica una relazione di generalizzazione, che indica che un attore è un tipo specializzato di un altro attore.
		            \begin{center}
			            \includegraphics*[width=2cm]{../../../images/norme_di_progetto/generalizzazioneTraAttori.png}
			            \captionof{figure}{Diagramma dei casi d'uso - Generalizzazione tra attori}
		            \end{center}
	      \end{itemize}
	      \newpage
	\item \textbf{Relazioni tra Casi d'Uso}
	      \begin{itemize}
		      \item \textbf{Inclusione}: La relazione di inclusione indica che un caso d'uso (chiamato "includente") incorpora l'esecuzione di un altro caso d'uso (detto "incluso"). In pratica, quando un attore interagisce con il caso d'uso includente, il caso d'uso incluso viene attivato come parte integrante del primo. Questo meccanismo è utile per favorire il riutilizzo di funzionalità e evitare la duplicazione di logica in diversi casi d'uso. La relazione di inclusione è simboleggiata da una freccia tratteggiata che collega il caso d'uso incluso al caso d'uso includente.
		            \begin{center}
			            \includegraphics*[width=15cm]{../../../images/norme_di_progetto/inclusione.png}
			            \captionof{figure}{Diagramma dei casi d'uso - Inclusione}
		            \end{center} \newpage
		      \item \textbf{Estensione}: La relazione di estensione indica che un caso d'uso (chiamato "estendente") può estendere il comportamento di un altro caso d'uso (detto "esteso") in determinate circostanze. In pratica, il caso d'uso estendente può aggiungere funzionalità opzionali o alternative al caso d'uso esteso, senza modificarne il comportamento principale. La relazione di estensione è simboleggiata da una freccia tratteggiata che collega il caso d'uso esteso al caso d'uso estendente.
		            \begin{center}
			            \includegraphics*[width=15cm]{../../../images/norme_di_progetto/estensione.png}
			            \captionof{figure}{Diagramma dei casi d'uso - Estensione}
		            \end{center} \newpage
		      \item \textbf{Generalizzazione casi d'uso}: Una freccia con una linea continua tra due casi d'uso indica una relazione di generalizzazione, che indica che un caso d'uso è un tipo specializzato di un altro caso d'uso.
		            \begin{center}
			            \includegraphics*[width=15cm]{../../../images/norme_di_progetto/generalizzazioneCasiDiUso.png}
			            \captionof{figure}{Diagramma dei casi d'uso - Generalizzazione tra casi d'uso}
		            \end{center}
	      \end{itemize}
\end{itemize}

\newpage
\subsubsubsection{Requisiti}
I requisiti di un software sono dettagliate specifiche documentate che delineano le funzionalità, le prestazioni, i vincoli e altri aspetti critici che il software deve soddisfare. Questi requisiti sono fondamentali per guidare lo sviluppo, il testing e la valutazione del prodotto, garantendo che risponda alle esigenze degli utenti e agli obiettivi del progetto. Essi comprendono sia i \textbf{requisiti funzionali}, che descrivono le funzionalità necessarie, sia i \textbf{requisiti non funzionali}, che definiscono criteri di prestazione, qualità, sicurezza e vincoli del sistema.\\
Una definizione precisa dei requisiti è essenziale: devono essere chiari e rispondere completamente alle aspettative del cliente o del proponente.\\
Ogni requisito è costituito da:
\begin{enumerate}
	\item \textbf{Codice}: I requisiti sono codificati nel seguente modo:
	      \begin{center}
		      \textbf{R[Tipologia]-[Codice]}
	      \end{center}
	      dove
	      \begin{itemize}
			\item \textbf{Tipologia}: indica la tipologia del requisito, che può essere:
			\begin{itemize}
				\item \textbf{F}: requisito funzionale;
				\item \textbf{Q}: requisito di qualità;
				\item \textbf{V}: requisito di vincolo.
			\end{itemize}
			\item \textbf{Codice}: è un numero progressivo che identifica univocamente il requisito.
	      \end{itemize}
	\item \textbf{Importanza}: indica il grado di importanza del requisito, che può essere:
	      \begin{itemize}
		      \item \textbf{Obbligatorio}: irrinunciabile per il \href{https://7last.github.io/docs/rtb/documentazione-interna/glossario\#committente}{committente\textsubscript{G}};
		      \item \textbf{Desiderabile}: non strettamente necessario, ma che porta valore aggiunto al prodotto;
		      \item \textbf{Opzionale}: relativo a funzionalità aggiuntive.
	      \end{itemize}
	\item \textbf{Fonte}: indica la fonte da cui è stato identificato il requisito, che può essere:
	      \begin{itemize}
		      \item \href{https://7last.github.io/docs/rtb/documentazione-interna/glossario\#capitolato}{\textbf{capitolato}\textsubscript{G}}: requisiti individuati a seguito dell'analisi dello stesso;
		      \item \textbf{interno}: requisiti individuati durante le riunioni interne e da coloro che hanno il ruolo di analista;
		      \item \textbf{esterno}: requisiti aggiuntivi individuati in seguito a incontri con la \href{https://7last.github.io/docs/rtb/documentazione-interna/glossario\#proponente}{proponente\textsubscript{G}};
		      \item \href{https://7last.github.io/docs/rtb/documentazione-interna/glossario\#piano-di-qualifica}{\textbf{piano di qualifica}\textsubscript{G}}: requisiti necessari per adeguare il prodotto agli standard di qualità definiti nel documento \href{https://7last.github.io/docs/rtb/documentazione-interna/glossario\#piano-di-qualifica}{\textit{Piano di Qualifica}\textsubscript{G}}.
		      \item \href{https://7last.github.io/docs/rtb/documentazione-interna/glossario\#norme-di-progetto}{\textbf{norme di progetto}\textsubscript{G}}: requisiti necessari per adeguare il prodotto alle norme stabilite nel documento \href{https://7last.github.io/docs/rtb/documentazione-interna/glossario\#norme-di-progetto}{\textit{Norme di Progetto}\textsubscript{G}}
	      \end{itemize}
\end{enumerate}


\subsubsubsection{Metriche}
Nell'analisi dei requisiti, le metriche sono strumenti essenziali per valutare, misurare e gestire diversi aspetti dei requisiti di un sistema o di un progetto. Grazie a queste metriche, è possibile garantire che i requisiti siano esaustivi, precisi, coerenti e comprensibili.\\
\begin{table}[!h] %? Il riferimento da mettere direttamente sulla colonna "Metrica"?
	\centering
	\begin{tabular}{|c|c|c|}
		\hline
		\textbf{Metrica}                    & \textbf{Abbreviazione}\\
		\hline
		\hyperlink{subsection.6.2}{0M-CRO}  & Copertura dei Requisiti Obbligatori  \\
		\hyperlink{subsection.6.2}{1M-CRD}  & Copertura dei Requisiti Desiderabili \\
		\hyperlink{subsection.6.2}{2M-CROP} & Copertura dei Requisiti Opzionali    \\
		\hline
	\end{tabular}
	\caption{Metriche per l'analisi dei requisiti}
	\label{tab:1}
\end{table}

\subsubsubsection{Strumenti}
\textbf{StarUML} è un'applicazione software impiegata dal team per creare i diagrammi dei casi d'uso. Offre funzionalità avanzate per la modellazione e la progettazione di software, consentendo di creare diagrammi UML chiari e ben strutturati.\\

\subsubsection{Progettazione}
\subsubsubsection{Descrizione}
Il principale obiettivo dell'attività di progettazione è individuare la soluzione implementativa ottimale che soddisfi pienamente le esigenze di tutti gli stakeholder, considerando i requisiti e le risorse disponibili. La progettazione si pone la domanda fondamentale: "Qual è il modo migliore per realizzare ciò di cui c'è bisogno?". È essenziale definire l'architettura del prodotto prima di iniziare la fase di codifica, adottando un approccio basato sulla correttezza per costruzione anziché sulla correzione successiva. Tale approccio consente di gestire efficacemente la complessità del prodotto, garantendo una struttura robusta e coesa durante l'intero processo di sviluppo.

\subsubsubsection{Obiettivi}
L'obiettivo principale è garantire che i requisiti siano soddisfatti attraverso un sistema di qualità definito dall'architettura del prodotto. Ciò comporta:
\begin{itemize}
	\item individuare componenti modulari che rispettino i requisiti, con specifiche chiare e coerenti, e svilupparle utilizzando risorse sostenibili e costi contenuti;
	\item organizzare le componenti in modo che siano facilmente comprensibili e manutenibili, garantendo una struttura coesa e ben organizzata;
	\item definire un'architettura che supporti l'evoluzione del prodotto, consentendo l'aggiunta di nuove funzionalità e la correzione di eventuali errori.
\end{itemize}
Inizialmente, il team di progettazione eseguirà un'analisi approfondita per selezionare con cura le tecnologie più adeguate, valutandone attentamente i vantaggi, i limiti e le eventuali problematiche. Una volta individuate le tecnologie appropriate, si procederà allo sviluppo di un'architettura di alto livello per comprendere e delineare la struttura generale del prodotto, che fungerà da base iniziale per la realizzazione del \textit{Proof of Concept} (PoC). Questa architettura fornirà una visione panoramica del sistema, identificando i principali componenti, i flussi di dati e le interazioni tra di essi, ponendo particolare attenzione alla flessibilità del sistema per eventuali modifiche future. \\Successivamente, si darà il via allo sviluppo del PoC, una parte cruciale della \textit{Technology Baseline}, per valutare le decisioni prese riguardo all'architettura e alle tecnologie adottate, e per verificare la loro congruenza con gli obiettivi e le specifiche del progetto. Dopo lo sviluppo e un'attenta analisi del PoC, si procederà con ulteriori iterazioni, apportando miglioramenti, aggiustamenti e integrazioni fino a raggiungere un design completo. Questo design sarà fondamentale per lo sviluppo del \textit{Minimum Viable Product} (MVP), che rappresenterà una versione essenziale e funzionale del prodotto e sarà parte integrante della \textit{Product Baseline}.
\newpage
\subsubsubsection{Documentazione}
\textbf{Specifica tecnica}\\ Il documento fornisce una visione dettagliata del design definitivo del prodotto e offre istruzioni chiare agli sviluppatori per implementare correttamente la soluzione software, seguendo i requisiti e le specifiche indicate. Questo aiuta a semplificare il processo di sviluppo del software, riducendo la complessità e le ambiguità, e assicurando che il prodotto finale sia in linea con le aspettative del cliente e funzioni in modo ottimale. Tra gli elementi chiave inclusi in questo documento vi sono:
\begin{itemize}
	\item \textbf{tecnologie utilizzate}: elenco delle tecnologie, dei framework e degli strumenti impiegati per lo sviluppo del prodotto;
	\item \textbf{architettura logica}: descrizione dettagliata della struttura logica del sistema, con particolare attenzione ai componenti principali, ai flussi di dati e alle interazioni tra di essi;
	\item \textbf{architettura di deployment}: rappresentazione grafica dell'architettura del sistema, con indicazioni sulle risorse hardware e software necessarie per il corretto funzionamento del prodotto;
	\item \textbf{design pattern}: descrizione dei design pattern utilizzati per risolvere problemi comuni e ricorrenti durante lo sviluppo del software;
	\item \textbf{vincoli e linee guida}: specifiche restrizioni e regole da seguire durante lo sviluppo del prodotto, per garantire coerenza e uniformità nel codice.
	\item \textbf{procedure di testing e validazione}: indicazioni sulle procedure e gli strumenti da utilizzare per verificare e validare il prodotto, garantendo che soddisfi i requisiti e le aspettative del cliente.
	\item \textbf{requisiti tecnici}: elenco dettagliato dei requisiti tecnici che il prodotto deve soddisfare, con indicazioni sulle funzionalità, le prestazioni e le caratteristiche richieste.
\end{itemize}

\subsubsubsection{Qualità dell'architettura}
\begin{itemize}
	\item \textbf{Sufficienza}: l'architettura deve soddisfare tutti i requisiti funzionali e non funzionali del sistema, garantendo che tutte le funzionalità richieste siano implementate correttamente e che il sistema funzioni in modo ottimale;
	\item \textbf{comprensibilità}: l'architettura deve essere chiara, ben strutturata e facilmente comprensibile, in modo che gli sviluppatori possano capire facilmente come il sistema è organizzato e come funziona;
	\item \textbf{modularità}: l'architettura deve essere modulare, con componenti ben definiti e indipendenti, in modo che possano essere facilmente riutilizzati e sostituiti senza influenzare il resto del sistema;
	\item \textbf{robustezza}: l'architettura deve essere robusta e resistente agli errori, in modo che il sistema possa gestire eventuali problemi o malfunzionamenti senza interrompere il funzionamento del sistema;
	\item \textbf{flessibilità}: l'architettura deve essere flessibile e adattabile, in modo che il sistema possa essere facilmente modificato e ampliato per soddisfare nuove esigenze e requisiti;
	\item \textbf{efficienza}: l'architettura deve essere efficiente e ottimizzata, in modo che il sistema possa funzionare in modo rapido ed efficiente, senza sprechi di risorse;
	\item \textbf{riusabilità}: l'architettura deve essere progettata per favorire la riutilizzabilità dei componenti, in modo che possano essere facilmente utilizzati in altri contesti e progetti;
	\item \textbf{affidabilità}: l'architettura deve essere affidabile e sicura, in modo che il sistema possa garantire la corretta esecuzione delle funzionalità e la protezione dei dati e delle informazioni;
	\item \textbf{disponibilità}: l'architettura deve garantire la disponibilità del sistema, in modo che possa essere sempre accessibile e operativo per gli utenti;
	\item \textbf{safety}: l'architettura deve garantire la sicurezza del sistema, in modo che possa proteggere i dati e le informazioni sensibili in seguito a malfunzionamenti;
	\item \textbf{security}: l'architettura deve garantire la sicurezza del sistema, in modo che possa proteggere i dati e le informazioni sensibili da accessi non autorizzati;
	\item \textbf{semplicità}: l'architettura deve essere semplice e intuitiva, in modo che possa essere facilmente compresa e utilizzata dagli sviluppatori e dagli utenti;
	\item \textbf{coesione}: l'architettura deve essere coesa, con componenti ben definiti e correlati tra loro, in modo che possano lavorare insieme in modo efficace e armonioso;
	\item \textbf{incapsulamento}: l'architettura deve essere incapsulata, con componenti ben definiti e indipendenti, in modo che possano essere facilmente gestiti e mantenuti senza influenzare il resto del sistema;
	\item \textbf{basso accoppiamento}: l'architettura deve avere un basso accoppiamento tra i componenti, in modo che possano essere facilmente sostituiti e modificati senza influenzare il resto del sistema.
\end{itemize}

\subsubsubsection{Diagrammi UML}
Vantaggi:
\begin{itemize}
	\item \textbf{chiarezza nella comunicazione}: forniscono una rappresentazione visuale delle informazioni, facilitando la comprensione e la comunicazione tra gli stakeholder;
	\item \textbf{standardizzazione}: UML è uno standard riconosciuto a livello internazionale, che consente di creare diagrammi coerenti e uniformi, garantendo una maggiore coerenza e comprensibilità;
	\item \textbf{analisi e progettazione visiva}: consentono di analizzare e progettare il sistema in modo visuale, facilitando la comprensione e l'identificazione di problemi e soluzioni;
	\item \textbf{modellazione e simulazione}: UML consente di modellare e simulare il sistema in modo visuale, facilitando la valutazione delle prestazioni e delle funzionalità del sistema;
	\item \textbf{manutenzione facilitata}: semplificano la manutenzione del sistema, consentendo di identificare e risolvere facilmente problemi e bug;
	\item \textbf{riduzione degli errori di progettazione}: UML aiuta a ridurre gli errori di progettazione, consentendo di identificare e correggere i problemi in modo rapido ed efficace;
	\item \textbf{documentazione supportata}: forniscono una documentazione visuale del sistema, che facilita la comprensione e la consultazione delle informazioni.
\end{itemize}
A supporto della progettazione, il team utilizzerà i seguenti \textbf{diagrammi delle classi}.
\newpage
\begin{flushleft}
\textbf{Diagrammi delle classi}
\end{flushleft}
Ogni diagramma delle classi rappresenta le proprietà e le relazioni tra le varie componenti di un sistema, offrendo una visione chiara e dettagliata della struttura del sistema.\\
Le classi sono rappresentate da rettangoli suddivisi in tre sezioni:
\begin{enumerate}
	\item \textbf{Nome della classe}: indica il nome della classe;
	\item \textbf{Attributi}: elenco degli attributi della classe, con il relativo tipo di dato, seguendo il formato: 
	\begin{center}
		\textbf{Visibilità Nome: Tipo [Molteplicità] = Valore di default}\end{center}
	      \begin{itemize}
		      \item \textbf{Visibilità}: indica il livello di accesso agli attributi, che può essere:
		            \begin{itemize}
			            \item \textbf{+}: pubblico;
			            \item \textbf{-}: privato;
			            \item \textbf{\#}: protetto;
			            \item \textbf{\textasciitilde}: package.
		            \end{itemize}
		      \item \textbf{Nome}: nome dell'attributo. Deve essere rappresentativo, chiaro e deve seguire la notazione \textit{nomeAttributo: tipo}; \\ Se l'attributo è costante, il nome deve essere scritto in maiuscolo (es. \textit{PIGRECO: double});
		      \item \textbf{Molteplicità}: nel caso di una sequenza di elementi come liste o array, indica il numero di elementi presenti, se questa non fosse conosciuta si utilizza il simbolo \textit{*} (es \textit{tipoAttributo[*]});
		      \item \textbf{default}: valore di default dell'attributo.
	      \end{itemize}
	\item \textbf{Metodi}: descrivono il comportamento della classe, seguendo il formato: \\ \begin{center}\textbf{Visibilità Nome(parametri): Tipo di ritorno}\end{center}
	      \begin{itemize}
		      \item \textbf{Visibilità}: indica il livello di accesso ai metodi, che può essere:
		            \begin{itemize}
			            \item \textbf{+}: pubblico;
			            \item \textbf{-}: privato;
			            \item \textbf{\#}: protetto;
			            \item \textbf{\textasciitilde}: package.
		            \end{itemize}
		      \item \textbf{Nome}: nome del metodo. Deve essere rappresentativo, chiaro e deve seguire la notazione \textit{nomeMetodo(parametri): tipoRitorno};
		      \item \textbf{Parametri}: elenco dei parametri del metodo, separati tramite virgola. Ogni parametro deve seguire la notazione \textit{nomeParametro: tipo};
		      \item \textbf{Tipo di ritorno}: indica il tipo di dato restituito dal metodo.
	      \end{itemize}
\end{enumerate}

\textbf{Convenzioni sui metodi}
\begin{itemize}
	\item \textbf{I metodi getter, setter} e i \textbf{costruttori} non vengono inclusi fra i metodi;
	\item \textbf{I metodi statici} sono sottolineati;
	\item \textbf{I metodi astratti} sono scritti in corsivo.
	\item \textbf{L'assenza di attributi o metodi} in una classe determina l'assenza delle relative sezioni nel diagramma.
\end{itemize}
\textbf{Relazioni tra le classi}\\
I diagrammi delle classi possono includere diverse relazioni tra le classi, che rappresentano le interazioni e le dipendenze tra le varie componenti del sistema. Le principali relazioni tra le classi sono:
\begin{itemize}
	\item \textbf{Associazione}: indica una relazione tra due classi, in cui un'istanza di una classe è collegata a un'istanza di un'altra classe. L'associazione è rappresentata da una linea che collega le due classi, con una freccia che indica la direzione dell'associazione e le molteplicità possono essere rappresentate con numeri messi agli estremi della freccia.
	      \begin{center}
		      \includegraphics*[width=12cm]{../../../images/norme_di_progetto/associazioneClassi.png}
		      \captionof{figure}{Diagramma delle classi - Associazione}
	      \end{center}
	      \newpage
	\item \textbf{Dipendenza}: indica una relazione in cui una classe dipende da un'altra classe, ad esempio se un metodo di una classe utilizza un'istanza di un'altra classe. La dipendenza è rappresentata da una linea tratteggiata che collega le due classi, con una freccia che indica la direzione della dipendenza.
	      \begin{center}
		      \includegraphics*[width=12cm]{../../../images/norme_di_progetto/dipendenzaClassi.png}
		      \captionof{figure}{Diagramma delle classi - Dipendenza}
	      \end{center}
	\item \textbf{Aggregazione}: indica una relazione in cui una classe è composta da una o più istanze di un'altra classe. L'aggregazione è rappresentata da una linea con un rombo vuoto che collega le due classi, con una freccia che indica la direzione dell'aggregazione.
	      \begin{center}
		      \includegraphics*[width=12cm]{../../../images/norme_di_progetto/aggregazioneClassi.png}
		      \captionof{figure}{Diagramma delle classi - Aggregazione}
	      \end{center}
	\item \textbf{Composizione}: indica una relazione in cui una classe è composta da una o più istanze di un'altra classe, ma le istanze sono strettamente legate e non possono esistere indipendentemente dalla classe principale. La composizione è rappresentata da una linea con un rombo pieno che collega le due classi, con una freccia che indica la direzione della composizione.
	      \begin{center}
		      \includegraphics*[width=10cm]{../../../images/norme_di_progetto/composizioneClassi.png}
		      \captionof{figure}{Diagramma delle classi - Composizione}
	      \end{center}
	\item \textbf{Generalizzazione}: indica una relazione in cui una classe è un tipo specializzato di un'altra classe. La generalizzazione è rappresentata da una linea con una freccia vuota che collega le due classi.
	      \begin{center}
		      \includegraphics*[width=10cm]{../../../images/norme_di_progetto/generalizzazioneClassi.png}
		      \captionof{figure}{Diagramma delle classi - Generalizzazione}
	      \end{center}
	      \newpage
	\item \textbf{Realizzazione}: indica una relazione in cui una classe implementa un'interfaccia o eredita da una classe astratta. La realizzazione è rappresentata da una linea che collega la classe all'interfaccia o alla classe astratta.
	      \begin{center}
		      \includegraphics*[width=8cm]{../../../images/norme_di_progetto/realizzazioneClassi.png}
		      \captionof{figure}{Diagramma delle classi - Realizzazione}
	      \end{center}
\end{itemize}
\subsubsubsection{Design pattern}
I design pattern rappresentano solide soluzioni a problemi ricorrenti di progettazione in determinati contesti, offrendo un approccio riutilizzabile che assicura la qualità e una rapida implementazione. Si adottano quando una soluzione ha dimostrato efficacia in un contesto specifico. Le guide dettagliate sull'applicazione dei pattern delineano il loro utilizzo ottimale, accompagnate da rappresentazioni grafiche, spiegazioni testuali della logica e descrizioni della loro utilità nell'architettura complessiva. Questa documentazione è essenziale per favorire una comprensione approfondita dell'integrazione dei design pattern nell'architettura generale e per prevenire errori di progettazione.

\subsubsubsection{Test}
Nel processo di sviluppo, il testing è cruciale per garantire la qualità del prodotto finale. Durante questa fase, vengono stabiliti i requisiti di testing, definiti i casi di test e i criteri di accettazione, che servono da strumenti per valutare il software.\\ L'obiettivo principale è individuare e risolvere eventuali problemi o errori nel software prima del rilascio del prodotto finale, assicurando che soddisfi le specifiche e le aspettative del cliente. I progettisti hanno il pieno controllo su questa attività, incluso il definire i test da eseguire. Nella sezione 3.2.4 sono fornite descrizioni dettagliate delle varie tipologie di test e della terminologia associata, offrendo ulteriore chiarezza su questa fase critica del processo di sviluppo del software.

\subsubsubsection{Strumenti}
\textbf{StarUML} è un'applicazione software impiegata dal team per creare i diagrammi dei casi d'uso.

\subsubsection{Codifica}
\subsubsubsection{Descrizione}
Nel processo di sviluppo del software, la codifica è responsabilità del programmatore ed è il momento cruciale in cui le funzionalità richieste prendono vita. Durante questa fase, le idee e i concetti delineati dai progettisti vengono tradotti in codice, creando istruzioni e procedure eseguibili dai calcolatori.\\ È essenziale che i programmatori rispettino attentamente le linee guida e le norme stabilite per assicurare che il codice sia conforme alle specifiche e rifletta accuratamente le visioni iniziali dei progettisti.

\subsubsubsection{Obiettivi}
La fase di codifica mira a sviluppare un prodotto software che soddisfi le richieste del cliente e sia conforme agli accordi concordati. Il rispetto rigoroso delle norme assicura la creazione di codice di elevata qualità, semplificando la manutenzione, l'espansione e la verifica del software. Questo contribuisce costantemente al miglioramento complessivo della sua qualità.

\subsubsubsection{Norme di codifica}
Le seguenti norme sono state formalizzate in questa maniera:
\begin{itemize}
	\item \textbf{Nomi significativi}: i nomi delle variabili, delle costanti, delle classi e dei metodi devono essere significativi e rappresentativi della loro funzione, in modo da facilitare la comprensione del codice;
	\item \textbf{Indentazione e formattazione consistente}: il codice deve essere correttamente indentato e formattato, con l'uso di spazi e tabulazioni coerenti, per garantire una corretta leggibilità e comprensione;
	\item \textbf{Lunghezza dei metodi}: i metodi devono essere brevi e concisi, con un numero limitato di righe di codice, per garantire una maggiore chiarezza e facilità di manutenzione. I metodi dovrebbero essere lunghi quanto basta per svolgere una singola funzione. Questo favorisce:
	      \begin{itemize}
		      \item \textbf{Chiarezza}: i metodi brevi sono più facili da comprendere e da seguire;
		      \item \textbf{Manutenibilità}: i metodi brevi sono più facili da modificare e da aggiornare;
		      \item \textbf{Comprensibilità}: i metodi brevi sono più facili da leggere e da interpretare;
		      \item \textbf{Testabilità}: i metodi brevi sono più facili da testare e da validare.
		      \item \textbf{Lunghezza del codice}: i file di codice sorgente devono essere brevi e concisi, con un numero limitato di righe di codice, per garantire una maggiore chiarezza e facilità di manutenzione.
		      \item \textbf{Commenti}: il codice deve essere corredato da commenti chiari e significativi, che spiegano il funzionamento e lo scopo delle varie parti del codice, per facilitare la comprensione e la manutenzione.
		      \item \textbf{Conformità ai principi SOLID}: il codice deve rispettare i principi SOLID, che promuovono la scrittura di codice pulito, modulare e manutenibile.
	      \end{itemize}
\end{itemize}

\subsubsubsection{Strumenti}
\textbf{Visual Studio Code} è l'editor di codice utilizzato dal team per scrivere e modificare il codice sorgente.

\subsubsubsection{Metriche}
\begin{table}[!h]
	\centering
	\begin{tabular}{|c|c|}
		\hline
		\textbf{Metrica}                   & \textbf{Abbreviazione} \\
		\hline
		\hyperlink{subsection.6.2}{23M-CC} & Code Coverage          \\
		\hyperlink{subsection.6.2}{24M-BC} & Branch Coverage        \\
		\hyperlink{subsection.6.2}{25M-SC} & Statement Coverage     \\
		\hline
	\end{tabular}
	\caption{Metriche per l'analisi dei requisiti (TEST)}
	\label{tab:2}
\end{table}

\subsubsection{Configurazione dell'ambiente di esecuzione}
\subsubsubsection{Docker}
La redazione dei file Docker è inclusa nel processo di sviluppo del software. Rispettare le regole e le migliori pratiche di codifica per i file Docker è essenziale per assicurare la creazione, la gestione e la distribuzione efficiente dei container.\\
\begin{itemize}
	\item \textbf{Chiarezza e Coerenza}: i file Docker devono essere chiari e coerenti, con una struttura ben definita e una formattazione uniforme, per garantire una corretta organizzazione e facilitare la comprensione;
	\item \textbf{Versionamento}: i file Docker devono includere informazioni dettagliate sulla versione del software e dei componenti utilizzati, per garantire la compatibilità e la coerenza tra i diversi ambienti;
	\item \textbf{Sicurezza}: i file Docker devono essere sicuri e protetti da eventuali vulnerabilità, per garantire la protezione dei dati e delle informazioni sensibili, utilizzando immagini ufficiali e aggiornate, evitando di eseguire comandi con privilegi elevati quando possibile;
	\item \textbf{Efficienza}: i file Docker devono essere efficienti e ottimizzati, con un utilizzo corretto delle risorse e una gestione accurata dei container, per garantire prestazioni elevate e tempi di risposta rapidi;
	\item \textbf{Gestione delle variabili d'ambiente}: i file Docker devono includere variabili d'ambiente ben definite e gestite correttamente, per garantire la configurabilità e la flessibilità del sistema;
	\item \textbf{Logging e monitoraggio}: bisogna configurare i \textit{container} per registrare i log e monitorare le prestazioni, per garantire la visibilità e la tracciabilità delle attività del sistema;
	\item \textbf{Layering}: ridurre il numero di layer e mantenere il più possibile le dipendenze in comune tra i vari layer, per garantire una maggiore efficienza e una migliore gestione delle risorse.
	\item \textbf{Riduzione delle dimensioni delle immagini}: ridurre le dimensioni delle immagini Docker, eliminando i file e le dipendenze non necessarie, per garantire una maggiore efficienza e una migliore gestione delle risorse.
	\item \textbf{Documentazione}: i file Docker devono essere corredati da una documentazione dettagliata e aggiornata, che spieghi il funzionamento e la configurazione del sistema, per facilitare la comprensione e la manutenzione.
	\item \textbf{Testing}: i file Docker devono essere testati e validati accuratamente, per garantire che il sistema funzioni correttamente e soddisfi i requisiti e le aspettative del cliente.
\end{itemize}

\subsubsubsection{Strumenti}
\begin{itemize}
	\item \textbf{Docker}: Docker è una piattaforma open source che semplifica la creazione, la distribuzione e la gestione di container, fornendo un ambiente isolato e sicuro per eseguire applicazioni e servizi.
	\item \textbf{Visual Studio Code}: Visual Studio Code è un editor di codice leggero e versatile, che offre funzionalità avanzate per lo sviluppo del software, tra cui il supporto per Docker e la creazione di file Docker.
\end{itemize}


\newpage


\section{Processi di supporto}
\subsection{Documentazione}
\subsubsection{Introduzione}
Il processo di documentazione è una componente fondamentale nella realizzazione e nel rilascio di un prodotto software,
poiché fornisce informazioni utili alle parti coinvolte e tiene traccia di tutte le attività relative al ciclo di vita del software,
comprese scelte e norme adottate dal gruppo durante lo svolgimento del progetto. In particolare, la documentazione è utile per:
\begin{itemize}
	\item permettere una comprensione profonda del prodotto e delle sue funzionalità;
	\item garantire uno standard di qualità all'interno dei processi produttivi.
\end{itemize}
Lo scopo della sezione è:
\begin{itemize}
	\item definire un insieme di regole e convenzioni per garantire la coerenza e la qualità della documentazione prodotta;
	\item creare template per ogni tipologia di documento così da garantire omogeneità.
\end{itemize}

\subsubsection{Versionamento}
La documentazione viene versionata allo stesso modo del codice sorgente, utilizzando la piattaforma \href{https://7last.github.io/docs/pb/documentazione-interna/glossario\#github}{Github\textsubscript{G}}.
Ciò permette di tenere traccia delle modifiche apportate ai documenti e di realizzare procedure automatiche per l'individuazione degli errori grammaticali,
inserimento della "G" a pedice per i termini presenti nel \href{https://7last.github.io/docs/pb/documentazione-interna/glossario\#glossario}{glossario\textsubscript{G}} e la pubblicazione nel sito del progetto.\\
Abbiamo stabilito di utilizzare il branch \texttt{main} per la pubblicazione dei documenti approvati dal \href{https://7last.github.io/docs/pb/documentazione-interna/glossario\#verificatore}{verificatore\textsubscript{G}} e il branch \texttt{develop} per contenere
i file in formato LaTex da cui derivano i documenti finali.

\subsubsubsection{Branch policy}
Sono state definite le seguenti regole per la gestione dei branch:
\begin{itemize}
	\item \texttt{main}: viene aggiornato solamente dalla \href{https://7last.github.io/docs/pb/documentazione-interna/glossario\#github}{\textit{Github\textsubscript{G}} Action} che si occupa di pubblicare i documenti approvati presenti nel branch \texttt{develop};
	\item \texttt{develop}: per la pubblicazione in questo branch è necessario ricevere un'approvazione da parte del \href{https://7last.github.io/docs/pb/documentazione-interna/glossario\#verificatore}{verificatore\textsubscript{G}} tramite \textit{pull request}.
\end{itemize}

\subsubsection{Tipografia e sorgente documenti}
Per la redazione dei documenti abbiamo deciso di utilizzare il linguaggio di markup LaTeX, grazie alla sua flessibilità e possibilità di creare documenti di alta qualità tipografica.
Inoltre, abbiamo stabilito di suddividere ciascun documento in file differenti, in modo da consentire ai membri del team di lavorare contemporaneamente su sezioni diverse,
riducendo il numero di conflitti e semplificando la gestione del versionamento.

\subsubsection{Ciclo di vita}
Ciascun documento segue un ciclo di vita ben definito, che prevede le seguenti fasi:
\begin{enumerate}
	\item \textbf{Creazione branch} a partire da \texttt{develop}, utilizzando \href{https://www.atlassian.com/git/tutorials/comparing-workflows/gitflow-workflow}{\underline{Gitflow}} [Ultima consultazione 2024-05-13];
	\item \textbf{Scelta del template} da utilizzare per il documento, scegliendo tra \textit{documento}, \textit{verbale interno} o \textit{verbale esterno};
	\item \textbf{Redazione} del documento, seguendo le norme tipografiche e le convenzioni stabilite;
	\item \textbf{Commit} delle modifiche effettuate;
	\item \textbf{Apertura \textit{pull request}} per la revisione del documento, che dovrà essere approvata dal \href{https://7last.github.io/docs/pb/documentazione-interna/glossario\#verificatore}{verificatore\textsubscript{G}} e dal \href{https://7last.github.io/docs/pb/documentazione-interna/glossario\#responsabile}{responsabile\textsubscript{G}} per lo \href{https://7last.github.io/docs/pb/documentazione-interna/glossario\#sprint}{sprint\textsubscript{G}} in corso;
	\item \textbf{Chiusura \textit{pull request}} e \textbf{merge} del branch;
	\item \textbf{Completamento task} su \href{https://7last.github.io/docs/pb/documentazione-interna/glossario\#clickup}{\textit{ClickUp}\textsubscript{G}}.
\end{enumerate}

\subsubsection{Convenzioni nomenclatura e struttura di archiviazione}
Tutti i documenti prodotti devono seguire la seguente convenzione di nomenclatura ben definita, in modo da garantire una corretta organizzazione e archiviazione dei file
\begin{center}
	\texttt{nome\_fase/tipo\_documento/nome\_documento/nome\_documento.tex}
\end{center}
Dove:
\begin{itemize}
	\item \textbf{nome\_fase}: rappresenta la fase di progetto in cui il documento è stato prodotto, può assumere i seguenti valori:
	      \begin{itemize}
		      \item \textbf{1\_candidatura};
		      \item \textbf{2\_RTB};
		      \item \textbf{3\_PB};
		      \item \textbf{4\_CA}.
	      \end{itemize}
	\item \textbf{tipo\_documento}: rappresenta la tipologia di documento, può assumere i seguenti valori:
	      \begin{itemize}
		      \item \textbf{documentazione\_interna};
		      \item \textbf{documentazione\_esterna};
		      \item \textbf{verbali\_interni};
		      \item \textbf{verbali\_esterni}.
	      \end{itemize}
	\item \textbf{nome\_documento}: rappresenta il nome del documento, scritto in snake case.
\end{itemize}

\subsubsection{Procedure correlate alla redazione di documenti}
\subsubsubsection{I redattori}
Il redattore si occupa della stesura del documento, seguendo le norme tipografiche e le convenzioni stabilite, in modo da garantire la
chiarezza e la coerenza del testo. Dovrà rispettare le convenzioni sopracitate.

\subsubsubsection{Il resposabile}
Il \href{https://7last.github.io/docs/pb/documentazione-interna/glossario\#responsabile}{responsabile\textsubscript{G}} coordina e gestisce le attività del gruppo. In particolare:
\begin{itemize}
	\item definire quali documenti devono essere prodotti e quali compiti devono essere svolti;
	\item gestisce le risorse e gli strumenti necessari per ultimare tali attività;
	\item approva la versione finale dei documenti;
	\item comunica con gli \href{https://7last.github.io/docs/pb/documentazione-interna/glossario\#stakeholder}{stakeholder\textsubscript{G}}.
\end{itemize}

\subsubsubsection{L'amministratore}
L’\href{https://7last.github.io/docs/pb/documentazione-interna/glossario\#amministratore}{amministratore\textsubscript{G}} è colui che si occupa della configurazione degli strumenti di supporto alla produzione del software; nel nostro caso,
inserisce le attività nell'ITS \href{https://7last.github.io/docs/pb/documentazione-interna/glossario\#clickup}{ClickUp\textsubscript{G}} e le assegna ai membri del gruppo.

\subsubsubsection{I verificatori}
Il compito del \href{https://7last.github.io/docs/pb/documentazione-interna/glossario\#verificatore}{verificatore\textsubscript{G}} è di controllare la correttezza e la qualità dei documenti prodotti, assicurandosi che siano conformi agli standard e alle norme stabilite.

\subsubsection{Struttura del documento}
Tutti i documenti produtti devono seguire una struttura ben definita, che prevede le componenti in seguito elencate.
\subsubsubsection{Prima pagina}
Nella prima pagina di ogni documento è presente un'intestazione contenente le seguenti informazioni:
\begin{itemize}
	\item titolo del documento;
	\item versione del documento;
	\item logo del gruppo;
	\item nome del gruppo.
\end{itemize}

\subsubsubsection{Registro delle modifiche}
La seconda pagina è dedicata al registro delle modifiche, rappresentato mediante una tabella. Il suo scopo è quello di tenere traccia di tutti i cambiamenti effettuati al documento,
in modo da garantire la rintracciabilità e la trasparenza delle modifiche. Ogni riga della tabella corrisponde a un cambiamento, e contiene le seguenti informazioni:
\begin{itemize}
	\item \textbf{versione} del documento, nel formato vX.Y, dove X rappresenta il numero di versione principale e Y il numero di versione secondaria;
	\item \textbf{data} di ultima modifica;
	\item nome del \textbf{redattore};
	\item nome del \href{https://7last.github.io/docs/pb/documentazione-interna/glossario\#verificatore}{\textbf{verificatore}\textsubscript{G}};
	\item \textbf{descrizione} della modifica.
\end{itemize}

\subsubsubsection{Indice}
Ogni documento contiene un indice delle sezioni e delle sottosezioni presenti al suo interno, in modo da facilitare la consultazione e la navigazione.
Tutte le figure e tabelle presenti nei documenti sono visibili negli appositi indici e sono direttamente accessibili tramite link ipertestuali.

\subsubsubsection{Intestazione}
Ogni pagina del documento, ad eccezione della prima, include un'intestazione che presenta, da sinistra a destra, il logo del gruppo, il titolo del documento e la versione.

\subsubsubsection{Struttura dei verbali}
I verbali rappresentano un resoconto dettagliato degli incontri, tracciando gli argomenti discussi, le decisioni prese e le azioni da intraprendere.
Sono suddivisi in verbali interni ed esterni, a seconda che il meeting coinvolga solo i membri del team o persone esterne al gruppo.
La struttura è la medesima per entrambe le tipologie, ad eccezione dell'ultima pagina.
\begin{itemize}
	\item \textbf{prima pagina}:
	      \begin{itemize}
		      \item tipologia verbale (interno / esterno);
		      \item titolo del documento;
		      \item sottotitolo del documento, "Riunione interna settimanale" per i verbali interni e "Colloquio con \textit{Sync Lab}" per i verbali esterni;
		      \item data e versione del documento;
		      \item logo del gruppo
		      \item nome del gruppo.
	      \end{itemize}
	\item \textbf{Registro delle modifiche};
	\item \textbf{Dettagli sulla riunione}:
	      \begin{itemize}
		      \item sede della riunione (piattaforma utilizzata);
		      \item orario di inizio e fine;
		      \item partecipanti del gruppo (interni) con specificazione del ruolo;
		      \item partecipanti esterni.
	      \end{itemize}
	\item \textbf{Corpo del documento}
	      All'interno del corpo del documento sono presenti le seguenti sezioni:
	      \begin{itemize}
		      \item \textbf{ordine del giorno}: elenco di ciò che verrà discusso durante la riunione;
		      \item \textbf{verbale}: resoconto dettagliato dell'incontro, con una sottosezione per ciascun punto dell'ordine del giorno,
		            gli obiettivi prefissati e le decisioni prese;
	      \end{itemize}
	\item \textbf{Ultima pagina}: solo in caso di verbale esterno, l'ultima pagina contiene data e firma di un componente esterno, presente durante l'incontro.
\end{itemize}
Il template dei verbali è disponibile nella repository del gruppo, più precisamente al seguente \href{https://github.com/7Last/docs/tree/develop/0_template}{\underline{link}}. (Ultima consultazione 2024-05-20).
\newpage
\subsubsection{Norme tipografiche}
\textbf{Nomi dei file}\\ I nomi dei documenti devono essere coerenti con il loro contenuto,
scritti in snake case. Nel branch \texttt{develop} tutti i documenti non contengono la versione nel nome, al fine di evitare conflitti durante la modifica.
Nel momento in cui il documento viene approvato, esso viene spostato nel branch \texttt{main} e la versione viene aggiunta automaticamente al nome del file.\\
In particolare, la denominazione dei file nel branch \texttt{main} deve seguire la seguente convenzione:
\begin{itemize}
	\item \textbf{Verbali}: verbale\_esterno/interno\_YYYY\_MM\_DD\_vA.B;
	\item \href{https://7last.github.io/docs/pb/documentazione-interna/glossario\#norme-di-progetto}{\textbf{Norme di Progetto}\textsubscript{G}}: norme\_di\_progetto\_vA.B;
	\item \href{https://7last.github.io/docs/pb/documentazione-interna/glossario\#analisi-dei-requisiti}{\textbf{Analisi dei Requisiti}\textsubscript{G}}: analisi\_dei\_requisiti\_vA.B;
	\item \href{https://7last.github.io/docs/pb/documentazione-interna/glossario\#piano-di-progetto}{\textbf{Piano di Progetto}\textsubscript{G}}: piano\_di\_progetto\_vA.B;
	\item \textbf{Lettera di Presentazione}: lettera\_di\_presentazione;
	\item \href{https://7last.github.io/docs/pb/documentazione-interna/glossario\#glossario}{\textbf{Glossario}\textsubscript{G}}: \href{https://7last.github.io/docs/pb/documentazione-interna/glossario\#glossario}{glossario\textsubscript{G}}\_vA.B.
\end{itemize}
\textbf{Stile del testo}
\begin{itemize}
	\item \textbf{Grassetto}:
	      \begin{itemize}
		      \item titoli di sezione;
		      \item termini importanti;
		      \item parole seguite da descrizione o elenchi puntati.
	      \end{itemize}
	\item \textbf{Corsivo}:
	      \begin{itemize}
		      \item nome del gruppo e dell'azienda \href{https://7last.github.io/docs/pb/documentazione-interna/glossario\#proponente}{proponente\textsubscript{G}};
		      \item riferimenti a documenti esterni.
	      \end{itemize}
	\item \textbf{Maiuscolo}:
	      \begin{itemize}
		      \item acronimi;
		      \item iniziali dei nomi;
	      \end{itemize}
\end{itemize}
\newpage
\textbf{Regole sintattiche}:
\begin{itemize}
	\item negli elenchi ogni voce deve terminare con ";" mentre l'ultima ".";
	\item i numeri razionali vengono scritti mediante l'uso della virgola come separatore tra parte intera e parte decimale;
	\item le date devono seguire lo standard internazionale ISO 8601, rappresentando la data con YYYY-MM-DD (anno, mese, giorno).
	\item le sezioni di un documento devono essere numerate in modo gerarchico, seguendo la struttura X.Y, dove:
	      \begin{itemize}
		      \item X rappresenta il numero della sezione principale;
		      \item Y rappresenta il numero della sottosezione;
	      \end{itemize}
\end{itemize}

\subsubsection{Metriche}

\begin{table}[!h]
	\centering
	\begin{tabular}{|c|l|}
		\hline
		\textbf{Codice} & \textbf{Nome esteso}               					\\
		\hline
		\underline{\hyperlink{19M}{19M-IG}}     & Indice di Gulpease   			\\
		\underline{\hyperlink{20M}{20M-CO}}     & Correttezza Ortografica  		\\
		\hline
	\end{tabular}
	\caption{Metriche inerenti il processo di documentazione}
\end{table}

\subsubsection{Strumenti}
Gli strumenti utilizzati per la redazione dei documenti sono:
\begin{itemize}
	\item \textbf{LaTeX}: impiegato nella stesura dei documenti;
	\item \href{https://7last.github.io/docs/pb/documentazione-interna/glossario\#github}{\textbf{GitHub}\textsubscript{G}}: utilizzato per la gestione delle versioni e per la condivisione dei documenti;
	\item \textbf{Visual Studio Code}: utilizzato come editor di testo per la scrittura dei documenti.
\end{itemize}



\subsection{Accertamento della qualità}
\subsubsection{Introduzione}
L'accertamento della qualità consiste in un insieme di attività e processi volti a garantire che il software sviluppato soddisfi gli standard di qualitativi richiesti. Questo include l'adesione a requisiti concordati con la \href{https://7last.github.io/docs/pb/documentazione-interna/glossario\#proponente}{proponente\textsubscript{G}}, normative ed è cruciale per assicurare che il prodotto finale sia affidabile e performante.

\subsubsection{Attività}
Il processo di accertamento della qualità comprende le seguenti attività:
\begin{enumerate}
	\item \textbf{sviluppo del \href{https://7last.github.io/docs/pb/documentazione-interna/glossario\#piano-di-qualifica}{piano di qualifica\textsubscript{G}}}: vengono definiti gli standard di qualità e le metriche che il progetto deve raggiungere;
	\item \textbf{monitoraggio continuo della qualità}: consiste in attività sistematiche e pianificate per garantire che i processi utilizzati durante lo sviluppo del software siano adeguati e seguiti correttamente;
	\item \textbf{gestione delle configurazioni}: attraverso un sistema di gestione delle configurazioni ci si assicura che ciascuna di esse venga sottoposta ad opportuni controlli e valutazioni;
	\item \textbf{reporting}: in questa fase vengono generati report che descrivono lo stato della qualità del progetto, inclusi i risultati dei test, i difetti trovati e le azioni correttive intraprese;
	\item \textbf{miglioramento continuo}: vengono applicate le modifiche ai processi basate sui feedback e sulle analisi per migliorare continuamente la qualità del software e dei processi di sviluppo;
	\item \textbf{confronto con la \href{https://7last.github.io/docs/pb/documentazione-interna/glossario\#proponente}{proponente\textsubscript{G}}}: viene instaurata una comunicazione con la \href{https://7last.github.io/docs/pb/documentazione-interna/glossario\#proponente}{proponente\textsubscript{G}} per ricevere feedback costante sul lavoro svolto.
\end{enumerate}

\subsubsection{Piano di qualifica}
Il documento \href{https://7last.github.io/docs/pb/documentazione-interna/glossario\#piano-di-qualifica}{\textit{Piano di Qualifica}\textsubscript{G}} è redatto per garantire che il software sviluppato rispetti gli standard di qualità richiesti e soddisfi le aspettative degli \href{https://7last.github.io/docs/pb/documentazione-interna/glossario\#stakeholder}{stakeholder\textsubscript{G}}. Il suo utilizzo e il suo scopo si estendono a diverse aree critiche del progetto, dalle fasi iniziali di pianificazione fino alla consegna finale del prodotto.

\subsubsection{Ciclo di Deming}
Il \textit{ciclo di Deming}, è un ciclo a 4 stadi che consente di apportare determinate migliorie a specifici processi. È composto da:
\begin{itemize}
	\item \textbf{Plan}: in cui si definiscono le attività, scadenze, responsabilità e risorse per raggiungere gli obiettivi di miglioramento;
	\item \textbf{Do}: esecuzione delle attività definite dal punto precedente;
	\item \textbf{Check}: verifica l'esito delle azioni di miglioramento rispetto alle attese;
	\item \textbf{Act}: consolidare il buono e cercare modi per migliorare il resto.
\end{itemize}

È importante specificare che questo ciclo \textbf{non opera} sul prodotto, bensì sul \textbf{\textit{way of working}} del gruppo, per migliorarlo e renderlo più efficiente.

\subsubsection{Struttura e identificazioni metriche}
Ogni metrica presenta la seguente struttura:
\begin{itemize}
	\item \textbf{Metrica}:
	      codice identificativo nel formato:
	      \begin{center}
		      \textbf{[numero]M-[acronimo]}
	      \end{center}
	      Dove:
	      \begin{itemize}
		      \item \textbf{[numero]}: numero progressivo univoco per ogni metrica;
		      \item \textbf{M}: metrica;
		      \item \textbf{[acronimo]}: abbreviazione del nome della metrica.
	      \end{itemize}
	\item \textbf{Nome}: nome della metrica;
	\item \textbf{Valore accettabile}: valore minimo affinché la metrica sia considerabile soddisfacente e conforme agli obiettivi di qualità;
	\item \textbf{Valore ammissibile}: valore ottimale e ideale che dovrebbe essere raggiunto dalla metrica;
	\item \textbf{Valore ottimo}:
	\item \textbf{Descrizione}: breve descrizione della metrica adottata e delle sue funzionalità;
\end{itemize}

\subsubsection{Metriche}
\begin{table}[!h]
	\centering
	\begin{tabular}{ | c | l | }
		\hline
		\textbf{Codice}                      & \textbf{Nome esteso}            \\
		\hline
		\underline{\hyperlink{21M}{21M-FU}}  & Facilità di utilizzo            \\
        \underline{\hyperlink{22M}{22M-TA}}  & Tempo di apprendimento          \\
        \underline{\hyperlink{23M}{23M-TR}}  & Tempo di risposta               \\
        \underline{\hyperlink{24M}{24M-TE}}  & Tempo di elaborazione           \\
        \underline{\hyperlink{25M}{25M-QMS}} & Metriche di qualità soddisfatte \\
		\hline
	\end{tabular}
	\caption{Metriche relative all'accertamento della qualità}
\end{table}

\subsection{Verifica}
\subsubsection{Introduzione}
La verifica nel ciclo di vita del software garantisce l'efficienza e la correttezza delle attività, assicurando che i prodotti soddisfino i requisiti. I \href{https://7last.github.io/docs/pb/documentazione-interna/glossario\#verificatore}{verificatori\textsubscript{G}} applicano tecniche di test seguendo procedure definite all'interno del \href{https://7last.github.io/docs/pb/documentazione-interna/glossario\#piano-di-qualifica}{Piano di Qualifica\textsubscript{G}}, il quale traccia il percorso di verifica, stabilendo obiettivi e criteri di accettazione. Tutti i processi attivi vengono sottoposti a verifica dopo aver raggiunto un buon grado di completamento o a seguito di modifiche dello stato del processo stesso.

\subsubsection{Verifica dei documenti} \label{verifica_dei_documenti}
Nell'ambito della documentazione, la verifica è un'attività cruciale per garantire la correttezza e l'accuratezza dei contenuti. Questo processo coinvolge una serie di controlli sistematici e revisioni finalizzate a garantire che la documentazione sia accurata, completa e conforme agli standard e ai requisiti specificati.
La verifica dei documenti assicura l'accuratezza, identificando e correggendo errori, omissioni e incongruenze. Questo riduce il rischio di malintesi o errori di implementazione, garantendo che tutte le informazioni tecniche e di progetto siano corrette e precise.
Inoltre, la verifica garantisce che la documentazione rispetti gli standard aziendali, di settore e normativi, mantenendo la coerenza e la qualità attraverso tutti i documenti del progetto. Documenti verificati e accurati migliorano la comunicazione tra i membri del team e con gli \href{https://7last.github.io/docs/pb/documentazione-interna/glossario\#stakeholder}{stakeholder\textsubscript{G}} esterni, facilitando la comprensione e la collaborazione e riducendo i tempi di chiarimento e discussione. Essa si suddivide in:
\begin{itemize}
	\item \textbf{correttezza tecnica}: vengono controllate le informazioni tecniche e i contenuti del documento, verificando che siano corretti e precisi;
	\item \textbf{conformità agli standard}: verifica che il documento segua le linee guida e gli standard stabiliti per la formattazione, la struttura e lo stile;
	\item \textbf{verifica della completezza}: si effettua un controllo per assicurare che tutti i requisiti documentali siano stati soddisfatti e che  tutte le sezioni e le informazioni necessarie siano presenti;
	\item \textbf{revisione linguistica}: viene effettuata una correzione di eventuali errori grammaticali, ortografici e di punteggiatura con l'ausilio di controlli automatici;
	\item \textbf{coerenza}: ci si assicura che i termini e le definizioni siano utilizzati in modo uniforme e coerente in tutto il documento.
\end{itemize}

\subsubsubsection{Processo di revisione}
Al momento di completamento delle modifiche in un documento il redattore è tenuto a creare una pull request dal branch in cui ha effettuato le modifiche verso il branch \texttt{develop}, richiedendo una revisione da parte del \href{https://7last.github.io/docs/pb/documentazione-interna/glossario\#verificatore}{verificatore\textsubscript{G}} e del \href{https://7last.github.io/docs/pb/documentazione-interna/glossario\#responsabile}{responsabile\textsubscript{G}}; in caso di approvazione verrà effettuato il merge e la issue creata su \href{https://7last.github.io/docs/pb/documentazione-interna/glossario\#clickup}{ClickUp\textsubscript{G}} verrà spostata da \textit{in progres} a \textit{done}.

\subsubsection{Analisi}
L'analisi è un processo essenziale che mira a garantire la qualità e l'affidabilità del prodotto finale. Questa fase critica comporta un esame dettagliato e sistematico di tutte le componenti del progetto per identificare e risolvere eventuali problemi prima della consegna. Si suddivide in \textbf{analisi dinamica} e \textbf{analisi statica}.

\subsubsubsection{Analisi dinamica}
L'analisi dinamica riguarda l'esame del comportamento del software durante la sua esecuzione. Questo tipo di analisi viene eseguito a tempo di esecuzione e si concentra sull'osservazione di come il software interagisce con il sistema operativo, le risorse di sistema e altre applicazioni. L'analisi dinamica è essenziale per valutare l'efficienza, le prestazioni e la sicurezza del software in condizioni operative reali; viene effettuata tramite l'utilizzo delle seguenti tipologie di test:
\begin{itemize}
	\item test di unità;
	\item test di integrazione;
	\item test di sistema;
	\item test di regressione;
	\item test di accettazione.
\end{itemize}

I benefici dell'analisi dinamica includono:
\begin{itemize}
	\item \textbf{individuazione di errori e difetti}: consente di identificare problemi e difetti nel software durante l'esecuzione, riducendo i rischi di malfunzionamenti e guasti;
	\item \textbf{ottimizzazione delle prestazioni}: permette di valutare le prestazioni del software e di identificare possibili aree di miglioramento;
	\item \textbf{convalida delle funzionalità}: assicura che il software funzioni correttamente e soddisfi i requisiti operativi attraverso i test elencati in precedenza.
\end{itemize}

\hypertarget{testing}{\subsubsection{Testing}}
Il testing è una fase fondamentale dello sviluppo del software, il cui scopo primario è garantire che il prodotto finale sia di alta qualità, soddisfi i requisiti specificati e funzioni correttamente in tutte le condizioni previste. Questo processo si articola in una serie di attività sistematiche e metodiche per identificare difetti, errori e incongruenze nel software, migliorandone così l'affidabilità e le prestazioni complessive.

Per ogni test è necessario definire i seguenti aspetti:
\begin{itemize}
	\item \textbf{identificazione del test}: un identificatore univoco per il caso di test, che faciliti il tracciamento e la referenza durante il processo di testing;
	\item \textbf{descrizione del test}: una descrizione chiara e concisa del caso di test, che spieghi la funzionalità specifica o il requisito che viene testato;
	\item \textbf{input del test}: dettagli sui dati di input necessari per eseguire il test, inclusi eventuali prerequisiti o condizioni iniziali;
	\item \textbf{output atteso}: descrizione degli output o dei risultati attesi del test, inclusi valori attesi, comportamenti del sistema o condizioni post-esecuzione;
	\item \textbf{stato del test}: indicazione dello stato del test, che può essere "superato", "non superato" o "non implementato";
\end{itemize}

\subsubsubsection{Test di unità}
I test di unità consentono di valutare il funzionamento delle singole unità o componenti del codice in modo isolato, senza dipendere dalle altre parti del sistema. Questo approccio mira a garantire che ogni porzione di software, anche la più piccola, operi correttamente e coerentemente con le specifiche definite, indipendentemente dalle interazioni con altre unità.
L'obiettivo principale dei test di unità è di assicurare che ogni unità funzioni come previsto, restituendo risultati attesi per determinate condizioni di input. Ciò significa che, per ciascuna unità, vengono definiti uno o più scenari di test che includono input specifici e i risultati attesi per tali input. Durante l'esecuzione dei test, il comportamento effettivo dell'unità viene confrontato con i risultati attesi, e qualsiasi discrepanza tra i due indica un potenziale difetto nel codice.
La pratica dei test di unità offre diversi vantaggi significativi. Innanzitutto, permette di individuare e risolvere tempestivamente eventuali difetti nel codice, poiché consente di identificare problemi già durante le prime fasi dello sviluppo, quando sono più facili e meno costosi da correggere. Inoltre, l'esecuzione automatica dei test di unità consente di garantire una verifica continua del codice, permettendo agli sviluppatori di individuare rapidamente eventuali regressioni o effetti collaterali indesiderati derivanti da modifiche al codice.
Inoltre, i test di unità forniscono una documentazione vivente del comportamento del software, in quanto ciascun test rappresenta un caso specifico di utilizzo dell'unità. Questo non solo facilita la comprensione del codice da parte di altri sviluppatori, ma può anche servire come una sorta di contratto che definisce il comportamento atteso dell'unità nel tempo.

\subsubsubsection{Test di integrazione}
I test di integrazione sono una fase nel processo di testing del software durante la quale vengono verificati e validati i collegamenti e le interazioni tra le diverse unità o componenti del sistema. L'obiettivo principale dei test di integrazione è accertarsi che le singole parti del software, già testate individualmente durante i test di unità, funzionino correttamente quando integrate insieme come un sistema completo. In pratica, i test di integrazione esaminano come le varie unità o componenti interagiscono tra loro e se cooperano correttamente per fornire le funzionalità desiderate del software. Questi test possono rivelare problemi come:
\begin{itemize}
	\item \textbf{incompatibilità} tra le interfacce delle diverse unità;
	\item \textbf{errori di comunicazione} tra le componenti;
	\item \textbf{mancata sincronizzazione o coerenza} nei dati scambiati tra le unità;
\end{itemize}

\subsubsubsection{Test di sistema}
Durante questa tipologia di test l'intero sistema viene valutato e verificato per garantire che soddisfi i requisiti funzionali e non funzionali specificati. Vengono eseguiti dopo i test di unità e di integrazione e sono progettati per valutare il sistema nel suo complesso, testando le sue funzionalità e le sue prestazioni rispetto agli obiettivi definiti. In particolare, permettono di:
\begin{itemize}
	\item \textbf{verificare} che il sistema \textbf{soddisfi i requisiti specificati};
	\item \textbf{valutare le prestazioni} del sistema;
	\item \textbf{verificare la sicurezza} e la \textbf{robustezza} del sistema;
	\item \textbf{verificare la compatibilità} del sistema con l'ambiente di esecuzione;
	\item \textbf{verificare la facilità d'uso} e l'\textbf{usabilità} del sistema.
\end{itemize}

\subsubsubsection{Test di regressione}
I test di regressione assicurano che le modifiche apportate al codice non abbiano introdotto nuovi difetti o intaccato funzionalità esistenti nel software. Questi test vengono eseguiti dopo che sono state apportate modifiche al software, come nuove funzionalità, correzioni di bug o aggiornamenti del sistema, per assicurare il corretto funzionamento anche dopo tali modifiche.

\subsubsubsection{Test di accettazione}
I test di accettazione sono una fase nel processo di sviluppo del software durante la quale il prodotto viene testato per valutare se soddisfa i requisiti e le aspettative degli utenti finali, dei clienti o degli \href{https://7last.github.io/docs/pb/documentazione-interna/glossario\#stakeholder}{stakeholder\textsubscript{G}}. Questi test vengono eseguiti dopo che il software è stato completamente sviluppato e prima del suo rilascio ufficiale, consentendo agli utenti finali di valutarlo in un ambiente simile a quello di produzione e fornire feedback sulle sue funzionalità, l'usabilità e la conformità ai requisiti.

\subsubsubsection{Sequenza delle fasi di test}
I test vengono eseguiti in sequenza, partendo da quelli di unità e procedendo verso quelli di integrazione, di sistema, di regressione e di accettazione. Questo approccio graduale consente di identificare e risolvere i difetti in modo sistematico e strutturato, garantendo che il software funzioni correttamente e soddisfi i requisiti specificati.

\subsubsubsection{Codici dei test}
Ciascun test deve essere identificato da un codice univoco nel seguente formato:
\begin{center}
	\textbf{[numero\_test]-[tipologia]}
\end{center}
Dove:
\begin{itemize}
	\item \textbf{[numero\_test]}: rappresenta un numero, in ordine crescente, associato al test, univoco all'interno della tipologia;
	\item \textbf{[tipologia]}: indica l'appartenenza del test, può assumere i seguenti valori:
	      \begin{itemize}
		      \item \textbf{U}: test di unità;
		      \item \textbf{I}: test di integrazione;
		      \item \textbf{S}: test di sistema;
		      \item \textbf{R}: test di regressione;
		      \item \textbf{A}: test di accettazione.
	      \end{itemize}

\end{itemize}

\subsubsubsection{Stato dei test}
A ciascun test è associato uno stato che indica l'esito della sua esecuzione:
\begin{itemize}
	\item \textbf{S}: il test è stato superato;
	\item \textbf{NS}: il test non è stato superato;
	\item \textbf{NI}: il test non è stato implementato.
\end{itemize}
Questi risultati saranno riportati nel documento \textit{"\href{https://7last.github.io/docs/pb/documentazione-interna/glossario\#piano-di-qualifica}{Piano di Qualifica\textsubscript{G}}"}, in particolare nella sezione \textit{"Metodologie di Testing"}.

\subsubsubsection{Analisi statica}
L'analisi statica è il processo mediante il quale si esamina il codice sorgente di un programma senza eseguirlo. Questo tipo di analisi viene effettuato utilizzando strumenti automatizzati che analizzano il codice per individuare possibili errori, violazioni delle best practice di codifica e potenziali vulnerabilità di sicurezza. L'analisi statica si concentra su aspetti come la sintassi, la semantica e le strutture di controllo del codice. I benefici di questo tipo di analisi sono molteplici, tra cui:
\begin{itemize}
	\item \textbf{individuazione precoce degli errori}: consente di rilevare errori e difetti nelle prime fasi dello sviluppo, riducendo i costi e i tempi necessari per le correzioni;
	\item \textbf{miglioramento della qualità del codice}: promuove l'adozione di standard di codifica e best practice, migliorando la manutenibilità e la leggibilità del codice;
	\item \textbf{prevenzione di vulnerabilità di sicurezza}: aiuta a individuare vulnerabilità di sicurezza, come buffer overflow e injection flaws, prima che il software venga eseguito.
\end{itemize}

Nell'ambito dell'analisi statica del software, l'inspection e il walkthrough sono tecniche di revisione del codice e della documentazione, ma presentano differenze significative in termini di obiettivi, formalità e procedure.

\subsubsubsubsection{Inspection}
L'inspection è una tecnica formale e strutturata di revisione del codice. Le sue principali caratteristiche sono:
\begin{itemize}
	\item \textbf{formalità}: l'inspection segue un processo rigoroso con ruoli definiti (moderatore, autore, lettore, ispettore, e registratore) e fasi ben precise (pianificazione, overview, preparazione, meeting di inspection, rework, e follow-up);
	\item \textbf{obiettivi}: l'obiettivo principale è identificare difetti nel codice o nella documentazione. Si concentra su aspetti come errori di logica, violazioni degli standard di codifica, problemi di performance e sicurezza;
	\item \textbf{documentazione}: viene prodotta una documentazione dettagliata delle osservazioni e dei difetti riscontrati, che serve come base per la correzione e per tracciare le azioni successive;
	\item \textbf{ruoli e responsabilità}: ogni partecipante ha un ruolo specifico e le attività sono strettamente coordinate. Il moderatore gestisce il processo, l'autore presenta il lavoro, gli ispettori individuano i difetti, il lettore guida la revisione del materiale, e il registratore annota i difetti trovati;
	\item \textbf{preparazione} I partecipanti devono esaminare il materiale da rivedere in anticipo e preparare una lista di potenziali problemi.
\end{itemize}

\subsubsubsubsection{Walkthrough}
Il walkthrough è una tecnica meno formale e più flessibile rispetto all'inspection. Le sue principali caratteristiche sono:
\begin{itemize}
	\item \textbf{informalità}: il walkthrough è meno strutturato e può essere condotto in modo informale. Non richiede ruoli rigidamente definiti o una procedura rigorosa;
	\item \textbf{obiettivi}: l'obiettivo principale è comprendere il codice o la documentazione, discutere possibili miglioramenti e condividere conoscenze tra i membri del team. Anche se l'individuazione dei difetti è uno degli scopi, non è l'unico focus;
	\item \textbf{documentazione}: la documentazione delle osservazioni è meno formale e dettagliata rispetto all'inspection. Può non essere sempre prodotta una documentazione completa delle discussioni e dei difetti riscontrati;
	\item \textbf{ruoli e responsabilità}: non ci sono ruoli formali assegnati. Solitamente, l'autore del codice o della documentazione guida la revisione e i partecipanti contribuiscono con commenti e suggerimenti;
	\item \textbf{preparazione}: la preparazione è meno intensiva. Non è necessario che tutti i partecipanti esaminino il materiale in anticipo, anche se può essere utile.
\end{itemize}

\subsubsection{Metriche}

\begin{table}[!h]
	\centering
	\begin{tabular}{|c|l|}
		\hline
		\textbf{Codice} & \textbf{Nome esteso}               					\\
		\hline
		\underline{\hyperlink{26M}{26M-CC}}     & Code Coverage   				\\
		\underline{\hyperlink{27M}{27M-BC}}     & Branch Coverage   			\\
		\underline{\hyperlink{28M}{28M-SC}}     & Statement Coverage   			\\
		\underline{\hyperlink{29M}{29M-FD}}     & Failure Density   			\\
		\underline{\hyperlink{30M}{30M-PTCP}}   & Passed Test Case Percentage   \\
		\hline
	\end{tabular}
	\caption{Metriche inerenti il processo di verifica}
\end{table}



\subsubsection{Strumenti}

\begin{itemize}
	\item \href{http://aspell.net/}{Aspell} per il controllo ortografico;
	\item \href{https://docs.python.org/3/library/unittest.html}{unittest} libreria di \href{https://7last.github.io/docs/pb/documentazione-interna/glossario\#python}{Python\textsubscript{G}} per lo sviluppo di test.
\end{itemize}



\subsection{Validazione}
\subsubsection{Introduzione}
La validazione rappresenta un momento critico nel ciclo di sviluppo, in quanto sottopone il software a una serie di controlli dettagliati per assicurare la sua conformità ai requisiti stabiliti e la sua idoneità all'utilizzo da parte degli utenti finali. Questo processo non è solo una verifica formale, ma una fase cruciale che assicura che il software sia costruito in modo tale da rispondere pienamente alle esigenze e alle aspettative degli utenti, nonché agli obiettivi del progetto.

\subsubsection{Procedura di validazione}
In questo processo copre un ruolo fondamentale il test di accettazione che mira a garantire la validazione
del prodotto. Infatti i diversi test elencati nella sezione \underline{\hyperlink{testing}{testing}} costituiscono un input
per la validazione. Essi dovranno verificare:
\begin{itemize}
	\item l'implementazione di tutti i casi d'uso;
	\item la conformità del prodotto ai requisiti obbligatori;
	\item il soddisfacimento di altri requisiti concordati con il \href{https://7last.github.io/docs/pb/documentazione-interna/glossario\#committente}{committente\textsubscript{G}}.
\end{itemize}

\subsection{Gestione della configurazione}
\subsubsection{Introduzione}
La gestione della configurazione è il processo di identificazione, controllo e coordinamento dei componenti software e delle risorse associate durante tutto il ciclo di vita del prodotto. Questo processo assicura che il software e i suoi artefatti correlati siano gestiti in modo coerente e controllato, consentendo agli sviluppatori di tracciare le modifiche, gestire le versioni e garantire l'integrità e la coerenza del sistema nel tempo.
\newpage
\subsubsection{Versionamento}
La convenzione di versionamento adottata è nel formato X.Y dove:
\begin{itemize}
	\item \textbf{X}: rappresenta il completamento in vista di una delle fasi del progetto e dunque viene incrementato al raggiungimento di \href{https://7last.github.io/docs/pb/documentazione-interna/glossario\#requirements-and-technology-baseline}{RTB\textsubscript{G}}, \href{https://7last.github.io/docs/pb/documentazione-interna/glossario\#product-baseline}{PB\textsubscript{G}} ed eventuale \href{https://7last.github.io/docs/pb/documentazione-interna/glossario\#customer-acceptance}{CA\textsubscript{G}}.
	\item \textbf{Y}: rappresenta una versione intermedia e viene incrementata ad ogni modifica significativa del documento.
\end{itemize}

\subsubsection{Repository}
Il team utilizza due repository:
\begin{itemize}
	\item \href{https://github.com/7Last/docs.git}{\underline{docs}}: contenente la documentazione prodotta;
	\item \href{https://github.com/7Last/SyncCity}{\href{https://7last.github.io/docs/pb/documentazione-interna/glossario\#synccity}{\underline{SyncCity}\textsubscript{G}}}: contenente il codice del progetto.
\end{itemize}

\subsubsubsection{Struttura repository}
Il repository inerente alla documentazione è così organizzato:
\begin{enumerate}
	\item \textbf{Candidatura}:
	      \begin{itemize}
		      \item \textbf{verbali\_esterni}: al suo interno sono presenti i verbali delle riunioni avute con i membri di \textit{Sync Lab S.r.l.};
		      \item \textbf{verbali\_interni}: contenente i verbali delle riunioni interne svolte;
		      \item \textbf{lettera\_di\_presentazione}: documento di presentazione per la candidatura al \href{https://7last.github.io/docs/pb/documentazione-interna/glossario\#capitolato}{capitolato\textsubscript{G}} scelto;
		      \item \textbf{preventivo\_costi\_assunzione\_impegni}: documento in cui vengono specificati i costi previsti, il totale delle ore per persona e gli impegni assunti;
		      \item \textbf{valutazione\_dei\_capitolati}: contenente il parere personale riguardo i capitolati offerti dalle varie aziende.
	      \end{itemize}
	\item \href{https://7last.github.io/docs/pb/documentazione-interna/glossario\#requirements-and-technology-baseline}{\textbf{RTB}\textsubscript{G}}:
	      \begin{itemize}
		      \item \textbf{documentazione\_esterna}: contenente i seguenti documenti:
		            \begin{itemize}
			            \item \textbf{analisi\_dei\_requisiti};
			            \item \textbf{piano\_di\_progetto};
			            \item \textbf{piano\_di\_qualifica};
		            \end{itemize}
		      \item \textbf{documentazione\_interna}: contenente i seguenti documenti:
		            \begin{itemize}
			            \item \href{https://7last.github.io/docs/pb/documentazione-interna/glossario\#glossario}{\textbf{glossario}\textsubscript{G}};
			            \item \textbf{norme\_di\_progetto}.
		            \end{itemize}
		      \item \textbf{verbali\_esterni}: al suo interno sono presenti i verbali delle riunioni avute con i membri di \textit{Sync Lab S.r.l.};
		      \item \textbf{verbali\_interni}: contenente i verbali delle riunioni interne svolte.
	      \end{itemize}
	\item \href{https://7last.github.io/docs/pb/documentazione-interna/glossario\#product-baseline}{\textbf{PB}\textsubscript{G}} %todo, da implementare
\end{enumerate}

\subsubsection{Sincronizzazione e branching}

\subsubsubsection{Documentazione}	\label{nomenclatura}
L'approccio adottato per la redazione della documentazione segue il workflow noto come \textit{Gitflow}. Questo workflow è un processo strutturato e collaborativo che consente ai membri del team di lavorare in modo efficace alla creazione, revisione e integrazione di documenti all'interno di un progetto. Questo approccio garantisce una gestione ordinata e controllata della redazione dei documenti, consentendo una migliore organizzazione e una maggiore qualità del lavoro finale.
\begin{flushleft}
	\textbf{Nomenclatura dei branch per le attività di redazione e/o modifica di documenti}: \label{convenzioni_nomenclatura}
\end{flushleft}
\begin{itemize}
	\item il nome del nuovo branch deve riportare il titolo del documento da redarre o modificare;
	\item il nome dei verbali deve presentare anche la data della riunione:
	      \begin{itemize}
		      \item \textit{verbale\_interno\_yy\_mm\_dd} (es. verbale\_interno\_24\_03\_05);
	      \end{itemize}
\end{itemize}

\subsubsubsection{Sviluppo}
\textit{7Last} utilizza \href{https://www.atlassian.com/it/git/tutorials/comparing-workflows/gitflow-workflow}{\underline{{Gitflow}}} come flusso di lavoro.
\begin{flushleft}
	Flusso di lavoro Gitflow:
\end{flushleft}
\begin{enumerate}
    \item \textbf{\texttt{main} branch}
    \begin{itemize}
        \item branch principale e stabile;
        \item contiene il codice in produzione;
        \item ogni commit rappresenta una versione rilasciata.
    \end{itemize}

    \item \textbf{\texttt{develop} branch}
    \begin{itemize}
        \item branch per lo sviluppo integrato;
        \item riceve i merge dai feature branches;
        \item rappresenta lo stato intermedio prima del rilascio.
    \end{itemize}

    \item \textbf{feature branch}
    \begin{itemize}
        \item derivano dal branch \texttt{develop};
        \item utilizzati per sviluppare nuove funzionalità;
        \item dopo il completamento, viene fatto il merge nel branch \texttt{develop}.
    \end{itemize}

    \item \textbf{release branch}
    \begin{itemize}
        \item derivano dal branch \texttt{develop};
        \item preparano una nuova versione per il rilascio;
        \item permettono di effettuare correzioni di bug e preparare la documentazione;
        \item dopo il completamento, viene fatto il merge sia nel branch \texttt{main} che nel branch \texttt{develop}.
    \end{itemize}

    \item \textbf{hotfix branch}
    \begin{itemize}
        \item derivano dal branch master;
        \item utilizzati per correggere rapidamente bug critici in produzione;
        \item dopo il completamento, viene fatto il merge sia nel branch \texttt{main} che nel branch \texttt{develop}.
    \end{itemize}
\end{enumerate}

\subsubsection{Consegna e rilascio}
Il processo di consegna e rilascio secondo lo standard \textit{ISO/IEC 12207:1995} è una componente essenziale del ciclo di vita del software, che si concentra sulla gestione delle versioni del software e sulla distribuzione dei prodotti software agli utenti finali. Questo processo assicura che le versioni del software siano rilasciate in modo controllato, pianificato e che soddisfino i requisiti di qualità e sicurezza. Attraverso una serie di attività ben definite, il processo di gestione del rilascio garantisce che il software sia pronto per l'uso, minimizzando i rischi di problemi post-rilascio e massimizzando l'efficienza operativa. La creazione della prima release deve avvenire in concomitanza con la baseline \href{https://7last.github.io/docs/pb/documentazione-interna/glossario\#requirements-and-technology-baseline}{RTB\textsubscript{G}} (\href{https://7last.github.io/docs/pb/documentazione-interna/glossario\#requirements-and-technology-baseline}{Requirements and Technology Baseline\textsubscript{G}}), mentre la creazione delle successive release deve avvenire in concomitanza con la baseline \href{https://7last.github.io/docs/pb/documentazione-interna/glossario\#product-baseline}{PB\textsubscript{G}} (\href{https://7last.github.io/docs/pb/documentazione-interna/glossario\#product-baseline}{Product Baseline\textsubscript{G}}) e, se prevista, \href{https://7last.github.io/docs/pb/documentazione-interna/glossario\#customer-acceptance}{CA\textsubscript{G}} (\href{https://7last.github.io/docs/pb/documentazione-interna/glossario\#customer-acceptance}{Customer Acceptance\textsubscript{G}}).

\subsubsection{Sito del gruppo}
Il gruppo ha sviluppato un sito disponibile al link: \url{https://7last.github.io/}; che ha lo scopo di facilitare la consultazione dei documenti redatti fino ad ora. Tale pagina web viene aggiornata automaticamente per rispecchiare lo stato dei documenti presenti all'interno del branch \texttt{main}. Inoltre facilita la consultazione del \href{https://7last.github.io/docs/pb/documentazione-interna/glossario\#glossario}{glossario\textsubscript{G}} grazie a dei link, presenti nelle parole distinte dalla lettera G al pedice, diretti alla definizione del termine.

\subsubsection{Strumenti}

\begin{itemize}
	\item \href{https://7last.github.io/docs/pb/documentazione-interna/glossario\#github}{\textbf{Github}\textsubscript{G}}: utilizzato per la gestione delle versioni e per la condivisione dei documenti;
	\item \href{https://7last.github.io/docs/pb/documentazione-interna/glossario\#clickup}{\textbf{ClickUp}\textsubscript{G}}: è una piattaforma di gestione del lavoro all-in-one progettata per pianificare, organizzare e collaborare su progetti e attività.
\end{itemize}


\subsection{Analisi congiunta}
\subsubsection{Introduzione}
L'analisi congiunta è un'attività collaborativa cruciale in cui i recensori e i recensiti si incontrano per esaminare e valutare diversi aspetti del progetto.  Nel nostro caso i recensori sono costituiti da \href{https://7last.github.io/docs/pb/documentazione-interna/glossario\#proponente}{proponente\textsubscript{G}}, \href{https://7last.github.io/docs/pb/documentazione-interna/glossario\#committente}{committente\textsubscript{G}} e \href{https://7last.github.io/docs/pb/documentazione-interna/glossario\#stakeholder}{stakeholder\textsubscript{G}}; mentre i recensiti sono rappresentati dal gruppo \textit{7Last}. Lo scopo principale di questa revisione congiunta è garantire che tutte le parti coinvolte abbiano una comprensione comune dello stato attuale del progetto e dei passi successivi necessari per raggiungere gli obiettivi fissati.

\subsubsection{Realizzazione del processo}
Il processo comprende i seguenti impegni:
\subsubsubsection{Revisioni periodiche}
In corrispondenza delle \href{https://7last.github.io/docs/pb/documentazione-interna/glossario\#milestone}{milestone\textsubscript{G}} stabilite saranno effettuate delle revisioni periodiche, come riportato nel
documento \href{https://7last.github.io/docs/pb/documentazione-interna/glossario\#piano-di-progetto}{\textit{Piano di Progetto}\textsubscript{G}}.

\subsubsubsection{Stato avanzamento lavori}
Al termine di ogni \href{https://7last.github.io/docs/pb/documentazione-interna/glossario\#sprint}{sprint\textsubscript{G}} viene svolta una revisione \href{https://7last.github.io/docs/pb/documentazione-interna/glossario\#stato-avanzamento-lavori}{SAL\textsubscript{G}}
(\href{https://7last.github.io/docs/pb/documentazione-interna/glossario\#stato-avanzamento-lavori}{Stato Avanzamento Lavori\textsubscript{G}}) tra il team e la \href{https://7last.github.io/docs/pb/documentazione-interna/glossario\#proponente}{proponente\textsubscript{G}}. Questa revisione ha lo scopo di valutare il lavoro svolto durante lo \href{https://7last.github.io/docs/pb/documentazione-interna/glossario\#sprint}{sprint\textsubscript{G}} precedente, per verificare che gli obiettivi prefissati siano stati correttamente raggiunti secondo le scadenze prefissate.
Inoltre, durante questo incontro, si pianificano le attività per lo \href{https://7last.github.io/docs/pb/documentazione-interna/glossario\#sprint}{sprint\textsubscript{G}} successivo.

\subsubsubsection{Revisioni straordinarie}
Se uno qualsiasi dei soggetti coinvolti tra gli \href{https://7last.github.io/docs/pb/documentazione-interna/glossario\#stakeholder}{stakeholder\textsubscript{G}} lo ritenga opportuno, è possibile istituire una revisione straordinaria per esaminare attentamente lo stato di avanzamento dei lavori. Durante questa revisione, si discutono eventuali problematiche emerse e le relative soluzioni adottabili.

\subsubsubsection{Risorse per le revisioni}
Le risorse coinvolte nelle revisioni possono essere differenti, ad esempio: strumenti hardware, software, documentazione... è fondamentale che tali risorse siano discusse e concordate tra tutte le parti coinvolte.

\subsubsubsection{Organizzazione degli incontri}
Pochi giorni prima della riunione con la \href{https://7last.github.io/docs/pb/documentazione-interna/glossario\#proponente}{proponente\textsubscript{G}} \textit{7Last} si impegna a consegnare un report con tutte le modifiche effettuate e i campi in cui sono state applicate.
In ciascuna revisione rimane la necessità di concordare i seguenti elementi:
\begin{itemize}
	\item agenda della riunione;
	\item prodotti software risultati dall'attività e relative problematiche;
\end{itemize}

\subsubsubsection{Documenti prodotti e decisioni approvate}
Tramite i \textit{Verbali Esterni} vengono documentati i risultati delle revisioni, compresi i problemi individuati, le soluzioni proposte e le azioni correttive da intraprendere. Questi documenti vengono distribuiti a tutte le parti coinvolte per garantire una comprensione comune e una chiara comunicazione. La parte recensente comunicherà alla parte recensita la veridicità di quanto riportato, approvando o disapprovando i documenti citati.

\subsection{Risoluzione dei problemi}
\subsubsection{Introduzione}
La risoluzione dei problemi è un processo che mira a identificare, analizzare e risolvere le varie problematiche che possono emergere durante lo sviluppo. Si riferisce a un insieme di tecniche e metodologie utilizzate per affrontare e risolvere le difficoltà tecniche, di design, di implementazione o di gestione che possono sorgere durante lo sviluppo del software. Questo processo può includere:
\begin{itemize}
	\item \textbf{identificazione del problema}: consiste nel riconoscere e definire chiaramente il problema;
	\item \textbf{analisi del problema}: nella quale si esamina il problema per capire le cause sottostanti e l'impatto;
	\item \textbf{generazione di soluzioni}: vengono proposte diverse possibili soluzioni al problema;
	\item \textbf{valutazione delle soluzioni}: vengono analizzate le soluzioni proposte per determinarne la fattibilità, l'efficacia e l'efficienza;
	\item \textbf{implementazione della soluzione}: la soluzione adottata viene implementata e testata per verificare che risolva il problema in modo efficace.
\end{itemize}

\subsubsection{Gestione dei rischi}
All'interno del documento \href{https://7last.github.io/docs/pb/documentazione-interna/glossario\#piano-di-progetto}{\textit{Piano di Progetto}\textsubscript{G}}, più precisamente nella sezione \textit{Analisi dei rischi} sono contenuti i rischi che potrebbero emergere durante lo svolgimento del progetto, con relativi approfondimenti e strategie di mitigazione.

\subsubsubsection{Codifica dei rischi}
Ogni rischio è identificato da un codice univoco nel seguente formato:
\begin{center}
	\textbf{R[tipologia]-[indice]}: nome identificativo del rischio.
\end{center}
Dove:
\begin{itemize}
	\item \textbf{[tipologia]}: rappresenta la categoria di rischio, la quale può essere organizzativa, tecnologica o comunicativa;
	\item \textbf{[indice]}: un valore numerico incrementale che identifica univocamente il rischio per ogni tipologia.
\end{itemize}

\subsubsection{Metriche}
\begin{table}[h]
	\centering
	\begin{tabular}{|c|c|}
		\hline
		\textbf{Metrica} & \textbf{Nome}        						\\
		\hline
		\underline{\hyperlink{31M}{31M-RMR}}   &  Risk Mitigation Rate  \\
		\underline{\hyperlink{32M}{32M-NCR}}   &  Rischi Non Calcolati  \\
		\hline
	\end{tabular}
	\caption{Metriche relative alla risoluzione dei problemi}
\end{table}

\subsubsection{Strumenti}
\begin{itemize}
	\item \href{https://7last.github.io/docs/pb/documentazione-interna/glossario\#clickup}{\textbf{ClickUp}\textsubscript{G}}: utilizzato per la gestione delle attività e delle issue;
\end{itemize}


\newpage



\section{Processi organizzativi}
Lo sviluppo software è un processo complesso e multidisciplinare che richiede una pianificazione, una gestione del tempo e delle risorse accurata, efficiente ed efficace. L'adozione di processi organizzativi ben strutturati è punto cruciale per garantire il successo dello sviluppo software. 
\subsection{Gestione dei processi}
\subsubsection{Introduzione}
La gestione dei processi si occupa di determinare, migliorare, ottimizzare i processi che fanno da guida alla realizzazione del software. Le attività di gestione dei processi sono:
\begin{itemize}
    \item \textbf{Definizione dei processi}:
        \begin{itemize}
            \item Identificare e documentare i processi chiave coinvolti nello sviluppo software
            \item Stabilire le linne guida e procedure per l'esecuzione di ciascun processo
        \end{itemize}
    \item \textbf{Pianificazione e monitoraggio}:
        \begin{itemize}
            \item Elaborare piani dettagliati per l'esecuzione dei processi
            \item Monitorare costantemente l'avanzamento, l'efficacia e la conformità ai requisiti pianificati
            \item Stimare i tempi, le risorse ed i costi
        \end{itemize}
    \item \textbf{Valutazione e miglioramento continuo}
        \begin{itemize}
            \item Condurre valutazioni periodiche dei processi per identificare aree di miglioramento
            \item Implementare azioni correttive e preventive per ottimizzare i processi
        \end{itemize}
    \item \textbf{Formazione e Competenze}
        \begin{itemize}
            \item Assicurare che il personale coinvolto nei processi sia adeguatamente formato
            \item Mantenere e sviluppare le competenze necessarie per l'efficace gestione dei processi
        \end{itemize}
    \item \textbf{Gestione dei rischi}
        \begin{itemize}
            \item Identificare e valutare i rischi associati ai processi
            \item Definire le strategie per mitigare o gestire i rischi identificati
        \end{itemize}
\end{itemize}
\subsubsection{Pianificazione}
\subsubsubsection{Descrizione}
La pianificazione riveste un ruolo centrale nella gestione dei processi, poiché mira a creare un piano organizzato e coerente per assicurare un’efficace esecuzione delle attività durante l’intero ciclo di vita del software.
Il responsabile del progetto assume il compito di coordinare ogni aspetto della pianificazione delle attività, che include l’allocazione delle risorse, la definizione dei tempi e la redazione di piani dettagliati. Inoltre, il responsabile si assicura che il piano elaborato sia fattibile e possa essere eseguito correttamente ed efficientemente dai membri del team.
I piani associati all’esecuzione del processo devono comprendere descrizioni dettagliate delle attività e delle risorse necessarie, specificando le tempistiche, le tecnologie impiegate, le infrastrutture coinvolte e il personale assegnato.
\subsubsubsection{Obiettivi}
L’obiettivo primario della pianificazione è assicurare che ciascun membro del team assuma ogni ruolo almeno una volta durante lo svolgimento del progetto, promuovendo così una distribuzione equa delle responsabilità e un arricchimento delle competenze all’interno del team.
La pianificazione, stilata dal responsabile, è integrata nel documento del \href{https://7last.github.io/docs/rtb/glossario#piano di progetto}{Piano di Progetto\textsubscript{G}}. Questo documento fornisce una descrizione completa delle attività e dei compiti necessari per raggiungere gli obiettivi prefissati in ogni periodo del progetto.

\subsubsubsection{Assegnazione dei ruoli}
Durante l’intero periodo del progetto, i membri del gruppo assumeranno sei ruoli distinti, ovvero assumeranno le responsabilità e svolgeranno le mansioni tipiche dei professionisti nel campo dello sviluppo software.
Nei successivi paragrafi sono descritti in dettaglio i seguenti ruoli:
\begin{itemize}
    \item Responsabile
    \item Amministratore
    \item Analista
    \item Progettista
    \item Programmatore
    \item Verificatore
\end{itemize}
\subsubsubsection{Responsabile}
Figura fondamentale che coordina il gruppo, fungendo da punto di riferimento per il committenteG e il team, svolgendo il ruolo di mediatore tra le due parti.
In particolare si occupa di:
\begin{itemize}
    \item Gestire le relazione con l'esterno
    \item Pianificare le attività: quali svolgere, data di inizio e fine, assegnazione delle priorità
    \item Valutare i rischi delle scelte da effettuare
    \item Controllare i progressi del progetto
    \item Gestire le risorse umane
    \item Approvazione della documentazione
\end{itemize}
\subsubsubsection{Amministratore}
Questa figura professionale è incaricata del controllo e dell’amministrazione dell’ambiente di lavoro utilizzato dal gruppo ed è anche il punto di riferimento per quanto concerne le norme di progetto. Le sue mansioni principali sono:
\begin{itemize}
    \item Affrontare e risolvere le problematiche associate alla gestione dei processi
    \item Gestire versionamento della documentazione 
    \item Gestire la configurazione del prodotto
    \item Redigere ed attuare le norme e le procedure per la gestione della qualità
    \item Amministrare le infrastrutture e i servizi per i processi di supporto
\end{itemize}
\subsubsubsection{Analista}
Figura professionale con competenze avanzate riguardo l’attività di analisi dei requisitiG ed il dominio applicativo del problema. Il suo ruolo è quello di identificare, documentare e comprendere a fondo le esigenze e le specifiche del progetto, traducendole in requisiti chiari e dettagliati. Si occupa di:
\begin{itemize}
    \item Analizzare il contesto di riferimento, definire il problema in esame e stabilire gli obiettivi da raggiungere
    \item Comprendere il  problema e definire la complessità e i requisiti
    \item Redigere il documento \textit{Analisi dei requisiti}
    \item Studiare i bisogni espliciti ed impliciti
\end{itemize}
\subsubsubsection{Progettista}
Il progettista è la figura di riferimento per quanto riguarda le scelte progettuali partendo dal lavoro dell’analista. Spetta al progettista assumere decisioni di natura tecnica e tecnologica, oltre a supervisionare il processo di sviluppo. Tuttavia, non è responsabile della manutenzione del prodotto.
In particolare si occupa di:
\begin{itemize}
    \item Progettare l’architetturaG del prodotto secondo specifiche tecniche dettagliate
    \item Prendere decisioni per sviluppare soluzioni che soddisfino i criteri di affidabilità,
    efficienza, sostenibilità e conformità ai requisiti;
    \item Redige la Specifica Architetturale e la parte pragmatica del \href{https://7last.github.io/docs/rtb/glossario#piano di qualifica}{Piano di Qualifica\textsubscript{G}};
\end{itemize}
\subsubsubsection{Programmatore}
Il programmatore è la figura professionale incaricata della scrittura del codice software. Il suo compito primario è implementare il codice conformemente alle specifiche fornite dall’analista e all’architetturaG definita dal progettista.
In particolare, il programmatore:
\begin{itemize}
    \item Scrive codice manutenibile in conformità con le Specifiche Tecniche
    \item Codifica le varie componenti dell’architettura seguendo quanto ideato dai progettisti
    \item Realizza gli strumenti per verificare e validare il codice
    \item Redige il \textit{Manuale Utente}
\end{itemize}
\subsubsubsection{Verificatore}
La principale responsabilità del verificatore consiste nell’ispezionare il lavoro svolto da altri membri del team per assicurare la qualità e la conformità alle attese prefissate. Stabilisce se il lavoro è stato svolto correttamente sulla base delle proprie competenze tecniche, esperienza e conoscenza delle norme.
In particolare il verificatore si occupa di:
\begin{itemize}
    \item Verificare che il lavoro svolto sia conforme alle Norme di progetto
    \item Verificare che il lavoro svolto sia conforme alle Specifiche Tecniche
    \item Ricercare ed in caso segnalare eventuali errori
    \item Redigere la sezione retrospettiva del \textit{Piano di Qualifica}, descrivendo le verifiche e le prove effettuate durante il processo di sviluppo del prodotto
\end{itemize}
\subsubsubsection{Ticketing}
GitHub è adottato come sistema di tracciamento delle issue (ITS), garantendo così una gestione agevole e trasparente delle attività da svolgere.
L’amministratore ha la facoltà di creare e assegnare specifiche issue sulla base delle attività identificate dal responsabile, assicurando chiarezza sulle responsabilità di ciascun membro del team e stabilendo tempi definiti entro cui ciascuna attività deve essere completata. Inoltre, ogni membro del gruppo può monitorare i progressi compiuti nel periodo corrente, consultando lo stato di avanzamento delle varie issue attraverso le Dashboard:
\begin{itemize}
    \item DashBoard: per una panoramica dettagliata sullo stato delle issue
    \item RoadMap: per una panoramica temporale dettagliata delle issue
\end{itemize}
Procedura per la creazione delle issue:\\
Le issue vengono create dall’amministratore e devono essere specificati i seguenti attributi:
\begin{itemize}
    \item Titolo: breve descrizione dell’attività da svolgere
    \item Descrizione:
        \begin{itemize}
            \item Descrizione testuale oppure "to-do" tramite bullet points
            \item Nell'ultima riga viena specificato il verificatore della issue nel formato: "Verificatore: Nome Cognome"
        \end{itemize}
    \item Assegnatario: membro del team responsabile dell'issue
    \item Milestone: periodo di riferimento in cui l’attività deve essere completata
    \item Labels: etichette per categorizzare le issue. \\Per associare ad ogni issue un Configuration Item vengono utilizzati i seguenti label:
        \begin{itemize}
            \item \textbf{NdP}: Norme di Progetto
            \item \textbf{PdP}: Piano di Progetto
            \item \textbf{PdQ}: Piano di Qualifica
            \item \textbf{AdR}: Analisi dei Requisiti
            \item \textbf{PoC}: Proof of Concept
            \item \textbf{Gls}: Glossario
        \end{itemize}
    \item Milestone: milestone associata alla issue
    \item Projects: progetti a cui la issue è associata. Se sono presenti dashboard associate ad un progetto, le issue correlate a tale progetto verranno visualizzate nella relativa/e dashboard di progetto.
    \item Development: branch e Pull Request associate alla issue. Quando una Pull Request viene accettata, la relativa issue viene automaticamente chiusa ed eventualmente spostata nella sezione "Done" della dashboard di progetto.
\end{itemize}
Ciclo di vita di una issue:\\
Il ciclo di vita è il seguente:
\begin{itemize}
    \item Creazione: l’amministratore crea la issue e la assegna al membro del team responsabile
    \item L’amministratore accede alla dashboardG di progetto e sposta la issue dalla colonna "No Status" alla colonna "To Do";    
    \item L’assegnatario apre un branch su GitHub seguendo la denominazione suggerita in "Sincronizzazione e Branching";
    \item Quando la issue viene presa in carico dall’assegnatario, questo accede alla DashBoard e sposta la issue dalla colonna "To Do" alla colonna "In Progress";
    \item Una volta che la issue è considerata terminata, l’assegnatario apre una Pull Request su GitHub seguendo la convenzione descitta in dettaglio nella sezione "Procedura per la creazione di Pull Request".
    \item All’interno della Dashboard GitHub la issue deve essere spostata dalla colonna "In Progress" alla colonna "Da revisionare";
    \item il verificatore o i verificatori designati seguono le procedure esposte nella sezione 3.2 per verificare le modifiche apportate al progetto;
    \item Se la verifica ha esito positivo, la issueG viene trasferita dalla colonna "Da revisionare" alla colonna "Done" della Dashboard di GitHub. Nel caso in cui la issue sia associata ad una Pull Request, una volta che quest’ultima viene accettata dal verificatore, la issue viene automaticamente chiusa e spostata nella colonna "Done" della Dashboard di progetto.
\end{itemize}
\subsubsubsection{Strumenti}
\begin{itemize}
    \item GitHub: piattaforma utilizzata per il tracciamento e la gestione delle issue.
\end{itemize}
\subsubsection{Coordinamento}
\subsubsubsection{Descrizione}
Il coordinamento rappresenta l’attività che sovraintende la gestione della comunicazione e la pianificazione degli incontri tra le diverse parti coinvolte in un progetto di ingegneria del software. Questo comprende sia la gestione della comunicazione interna tra i membri del team del progetto, sia la comunicazione esterna con il proponente e i committenti. Il coordinamento risulta essere cruciale per assicurare che il progetto proceda in modo efficiente e che tutte le parti coinvolte siano informate e partecipino attivamente in ogni fase del progetto.
\subsubsubsection{Obiettivi}
Il coordinamento in un progetto è fondamentale per gestire la comunicazione e pianificare gli incontri tra gli stakeholder. L’obiettivo principale è garantire efficienza, evitando ritardi e confusioni, assicurando che tutte le parti in causa siano informate e coinvolte in ogni fase del progetto. Inoltre, promuove la collaborazione e la coesione nel team, facilitando lo scambio di idee e la risoluzione dei problemi in modo collaborativo, creando un ambiente lavorativo positivo e produttivo.
\textbf{Comunicazione}
Il gruppo 7Last mantiene comunicazioni attive, sia interne che esterne al team, le quali possono essere sincrone o asincrone, a seconda delle necessità.
\subsubsubsection{Comunicazioni sincrone}
\begin{itemize}
    \item \textbf{Comunicazione sincrone interne}\\Per le comunicazioni sincrone interne, il gruppo 7Last, ha scelto di adottare Discord in quanto permette di comunicare tramite chiamate vocali, videochiamate, messaggi di testo, media e file in chat private o come membri di un "server Discord";
    \item \textbf{Comunicazione sincrone esterne}\\Per le comunicazioni sincrone esterne,in accordo con l’azienda proponente si è deciso di utilizzare un canale Discord.
\end{itemize}
\subsubsubsection{Comunicazioni asincrone}
\begin{itemize}
    \item \textbf{Comunicazione asincrone interne}\\Per le comunicazioni asincrone interne, il gruppo 7Last, ha scelto di adottare Telegram in quanto permette di comunicare tramite messaggi di testo, media e file in chat private o come membri di un "gruppo Telegram";
    \item \textbf{Comunicazione asincrone esterne}\\Per le comunicazioni asincrone esterne sono stati adottati due canali differenti:
        \begin{itemize}
            \item \textbf{Email}: per comunicazioni formali e ufficiali
            \item \textbf{Discord}
        \end{itemize}
\end{itemize}
\subsubsubsection{Riunioni interne}
Si è scelto di svolgere i meeting interni a cadenza settimanale, al fine di facilitare una comunicazione costante e coordinare il progresso delle attività.
Generalmente le riunioni sono programmate per ogni venerdi alle ore:
\begin{itemize}
    \item Mercoledì dalle 15:00 - 16:00 per riunioni interne
\end{itemize}
Se qualche membro del gruppo non può partecipare alla riunione nella data e nell’orario stabiliti, si procede programmando un nuovo incontro, concordando data e ora tramite un sondaggio sul canale Telegram dedicato.
Ogni membro del gruppo ha la facoltà di richiedere una riunione supplementare se necessario. In questo caso, la data e l’orario saranno concordati sempre attraverso il canale Telegram dedicato, mediante la creazione di un sondaggio.
Le riunioni interne rivestono un ruolo cruciale nel monitorare il progresso delle mansioni assegnate, valutare i risultati conseguiti e affrontare i dubbi e le difficoltà che possono sorgere. Durante i meeting interni, i membri del team condividono gli aggiornamenti sulle proprie attività, identificano le problematiche riscontrate e discutono di opportunità di miglioramento nei processi di lavoro. Questo ambiente aperto e collaborativo favorisce l’interazione, l’innovazione e la condivisione di nuove prospettive. Per agevolare la comunicazione sincrona, il canale utilizzato per i meeting interni è Discord, ritenuto particolarmente efficace per tali scopi.
Relativamente ai meeting interni, sarà compito del responsabile:
\begin{itemize}
    \item Stabilire preventivamente i principali temi da trattare durante la riunione, considerando la possibilità di aggiungerne di nuovi nel corso della riunione stessa;
    \item Guidare la discussione e raccogliere i pareri dei membri in maniera ordinata;
    \item Nominare un segretario per la riunione;
    \item Pianificare e proporre le nuove attività da svolgere.
\end{itemize}
\textbf{Verbali interni}\\
Lo svolgimento di una riunione interna ha come obiettivo la retrospettiva del periodo precedente, la discussione dei punti stilati nell’ordine del giorno e la pianificazione delle nuove attività.
Alla conclusione di ciascuna riunione, l’amministratore apre un’issue nell’ITS di GitHub e assegna l’incarico di redigere il verbale interno al segretario della riunione. È compito quindi del segretario redigere il verbale, includendo tutte le informazioni rilevanti emerse durante la riunione. Le indicazioni dettagliate per la compilazione dei verbali interni sono disponibili nella sezione 3.1.6.5.

\subsubsubsection{Riunioni esterne}
Durante lo svolgimento del progetto, è essenziale organizzare vari incontri con i Committenti e/o il Proponente al fine di valutare lo stato di avanzamento del prodotto e chiarire eventuali dubbi o questioni.
La responsabilità di convocare tali incontri ricade sul responsabile, il quale è incaricato di pianificarli e agevolarne lo svolgimento in modo efficiente ed efficace.
Sarà compito del responsabile anche l’esposizione dei punti di discussione al proponente/committente, lasciando la parola ai membri del gruppo interessati quando necessario. Questo approccio assicura una comunicazione efficace tra le varie parti in causa, garantendo una gestione ottimale del tempo e una registrazione accurata delle informazioni rilevanti emerse durante gli incontri.
I membri del gruppo si impegnano a garantire la propria presenza in modo costante alle riunioni, facendo il possibile per riorganizzare eventuali altri impegni al fine di partecipare. Nel caso in cui gli obblighi inderogabili di un membro del gruppo rendessero impossibile la partecipazione, il responsabile assicurerà di informare tempestivamente il proponente o i committenti, richiedendo la possibilità di rinviare la riunione ad una data successiva.

\textbf{Riunioni con l'azienda proponente}\\
In accordo con l’azienda proponente, si è stabilito di tenere incontri di stato avanzamento lavori (SAL) con cadenza bisettimanale tramite Google Meet.
Durante tali incontri, si affrontano diversi aspetti, tra cui:
\begin{itemize}
    \item Discussione delle attività svolte nel periodo precedente, valutando l’aderenza a quanto concordato e identificando eventuali problematiche riscontrate
    \item Pianificazione delle attività per il prossimo periodo, definendo gli obiettivi e le azioni necessarie per il loro raggiungimento
    \item Chiarezza e risoluzione di eventuali dubbi emersi nel corso delle attività svolte
\end{itemize}

\textbf{Verbali esterni}\\
Come nel caso delle riunioni interne, anche per le riunioni esterne verrà redatto un verbale con le stesse modalità descritte nella sezione relativa ai Verbali Interni.
Le linee guida per la redazione dei verbali esterni sono reperibili alla sezione 3.1.6.5.


\subsubsubsection{Strumenti}
\begin{itemize}
    \item \textbf{Discord}: impiegato per la comunicazione sincrona e i meeting interni del team e per le riunioni esterne con il proponente
    \item \textbf{Telegram}: utilizzato per la comunicazione asincrona interna
    \item \textbf{Gmail}: come servizio di posta elettronica
\end{itemize}

\subsubsubsection{Metriche}
AGGIUNGERE TABELLA

\subsection{Miglioramento}
\subsubsection{Introduzione}
Secondo lo standard ISO/IEC 12207:1995, il processo di miglioramento nel ciclo di vita del software è finalizzato a stabilire, misurare, controllare e migliorare i processi che lo compongono. L’attività di miglioramento è composta da:
\begin{itemize}
    \item \textbf{Analisi}: valutazione dei processi per identificare le aree di miglioramento
    \item \textbf{Miglioramento}: attuazione di azioni correttive e preventive per ottimizzare i processi
\end{itemize}
\subsubsection{Analisi}
Questa operazione richiede di essere eseguita regolarmente e ad intervalli di tempo appropriati e costanti. L’analisi fornisce un ritorno sulla reale efficacia e correttezza dei processi implementati, permettendo di identificare prontamente quelli che necessitano di miglioramenti.
Durante ogni riunione, il team dedica inizialmente del tempo per condurre una retrospettiva sulle attività svolte nell’ultimo periodo. Questa pratica implica una riflessione approfondita su ciò che è stato realizzato, coinvolgendo tutti i membri nella identificazione delle aree di successo e di possibili miglioramenti.
L’obiettivo principale è formulare azioni correttive da implementare nel prossimo sprint, promuovendo così un costante feedback e un adattamento continuo per migliorare le prestazioni complessive del team nel corso del tempo.
\subsubsection{Miglioramento}
Il team implementa le azioni correttive stabilite durante la retrospettiva, successivamente valuta la loro efficacia e le sottopone nuovamente a esame durante la retrospettiva successiva.
L’esito di ogni azione correttiva sarà documentato nella sezione "Revisione del periodo precedente" di ogni verbale.

\subsection{Formazione}
\subsubsection{Introduzione}
L’obiettivo di questa iniziativa è stabilire standardG per il processo di apprendimento all’interno del team, assicurando la comprensione adeguata delle conoscenze necessarie per la realizzazione del progetto.
Si prevede che il processo di formazione del Team assicuri che ciascun membro acquisisca una competenza adeguata per utilizzare consapevolmente le tecnologie selezionate dal gruppo per la realizzazione del progetto.

\subsubsection{Metodo di formazione}
\subsubsubsection{Individuale}
Ciascun membro del team si impegnerà in un processo di autoformazione per adempiere alle attività assegnate al proprio ruolo. Durante la rotazione dei ruoli, ogni membro del gruppo condurrà una riunione con il successivo occupante del suo attuale ruolo, trasmettendo le conoscenze necessarie. Al contempo, terrà una riunione con chi ha precedentemente svolto il ruolo che esso assumerà, con l’obiettivo di apprendere le competenze richieste.

\subsubsubsection{Gruppo}
Sono programmate sessioni formative, condotte dalla proponente, al fine di trasferire competenze relative alle tecnologie impiegate nel contesto del progetto. La partecipazione del team a tali riunioni è obbligatoria.


\newpage


\section{Standard per la qualità}
Abbiamo scelto di adottare standard internazionali per l'analisi e la valutazione della qualità dei processi e del software. In particolare, la suddivisione dei processi in primari, di supporto e organizzativi sarà guidata dall'adozione dello standard ISO/IEC 12207:1995. Inoltre, l'adozione dello standard ISO/IEC 25010:2023 ci fornirà un quadro completo per la definizione e la classificazione delle metriche di qualità del software. Abbiamo deciso di applicare solo questi due standard, poiché lo standard ISO/IEC 9126:2001 è stato ritirato e sostituito dallo standard ISO/IEC 25010:2023.
\subsection{Caratteristiche del sistema ISO/IEC 25010:2023}
\subsubsection{Appropriatezza funzionale}
\begin{itemize}
    \item \textbf{Completezza}: il prodotto software deve soddisfare tutti i requisiti definiti e attesi dagli utenti;
    \item \textbf{Correttezza}: il prodotto software deve funzionare come previsto e produrre risultati accurati;
    \item \textbf{Appropriatezza}: il prodotto software deve essere adatto allo scopo previsto e al contesto di utilizzo.
\end{itemize}
\subsubsection{Performance}
\begin{itemize}
    \item \textbf{Tempo}: il prodotto software deve rispettare le scadenze e i tempi di consegna previsti;
    \item \textbf{Risorse}: il prodotto software deve utilizzare le risorse di sistema in modo efficiente e ragionevole;
    \item \textbf{Capacità}: il prodotto software deve essere in grado di gestire il carico di lavoro previsto.
\end{itemize}
\subsubsection{Compatibilità}
\begin{itemize}
    \item \textbf{Coesistenza}: il prodotto software deve essere in grado di coesistere con altri software e sistemi sul computer;
    \item \textbf{Interoperabilità}: il prodotto software deve essere in grado di scambiare informazioni e collaborare con altri software e sistemi.
\end{itemize}
\subsubsection{Usabilità}
\begin{itemize}
    \item \textbf{Riconoscibilità}: il prodotto software deve avere un'interfaccia utente intuitiva e facile da usare;
    \item \textbf{Apprendibilità}: gli utenti devono essere in grado di imparare a utilizzare il prodotto software in modo rapido e semplice;
    \item \textbf{Operabilità}: il prodotto software deve essere facile da usare e da controllare;
    \item \textbf{Protezione} errori: il prodotto software deve essere in grado di rilevare e gestire gli errori in modo efficace;
    \item \textbf{Esteticità}: il prodotto software deve avere un'interfaccia utente piacevole e accattivante;
    \item \textbf{Accessibilità}: il prodotto software deve essere accessibile a persone con disabilità.
\end{itemize}
\subsubsection{Affidabilità}
\begin{itemize}
    \item \textbf{Maturità}: il prodotto software deve essere stabile, affidabile e robusto;
    \item \textbf{Disponibilità}: il prodotto software deve essere disponibile quando necessario;
    \item \textbf{Tolleranza}: il prodotto software deve essere in grado di tollerare errori e condizioni inaspettate;
    \item \textbf{Ricoverabilità}: il prodotto software deve essere in grado di ripristinare i dati e le funzionalità in caso di guasto o errore.
\end{itemize}
\subsubsection{Sicurezza}
\begin{itemize}
    \item \textbf{Riservatezza}: il prodotto software deve proteggere i dati sensibili e le informazioni riservate;
    \item \textbf{Integrità}: il prodotto software deve garantire l'accuratezza e la completezza dei dati;
    \item \textbf{Non ripudio}: il prodotto software deve garantire che le transazioni e le comunicazioni non possano essere negate o ripudiate;
    \item \textbf{Autenticazione}: il prodotto software deve verificare l'identità degli utenti e garantire che solo gli utenti autorizzati possano accedere al sistema;
    \item \textbf{Autenticità}: il prodotto software deve garantire che l'origine dei dati e delle informazioni sia verificabile.
\end{itemize}
\subsubsection{Manutenibilità}
\begin{itemize}
    \item \textbf{Modularità}: il prodotto software deve essere progettato in modo modulare, con componenti indipendenti e ben definiti;
    \item \textbf{Riusabilità}: i componenti del prodotto software devono essere progettati per essere riutilizzati in altri progetti;
    \item \textbf{Analizzabilità}: il prodotto software deve essere progettato in modo da essere facilmente analizzabile e comprensibile;
    \item \textbf{Modificabilità}: il prodotto software deve essere progettato in modo da essere facilmente modificabile e adattabile;
    \item \textbf{Testabilità}: il prodotto software deve essere progettato in modo da essere facilmente testabile.
\end{itemize}
\subsubsection{Portabilità}
\begin{itemize}
    \item \textbf{Adattabilità}: il prodotto software deve essere in grado di adattarsi a nuovi ambienti, requisiti e tecnologie;
    \item \textbf{Installabilità}: il prodotto software deve essere facilmente installabile e configurabile;
    \item \textbf{Sostituibilità}: il prodotto software deve essere facilmente sostituibile con altre soluzioni o versioni più recenti.
\end{itemize}
\newpage
\subsection{Suddivisione secondo standard ISO/IEC 12207:1995}
\subsubsection{Processi primari}
Essenziali per lo sviluppo del software e comprendono:
\begin{itemize}
    \item \textbf{acquisizione}: gestione dei propri sotto-fornitori;
    \item \textbf{fornitura}: gestione delle relazioni con il cliente;
    \item \textbf{sviluppo}: comprende tutte le attività legate alla progettazione, implementazione e verifica del software;
    \item \textbf{operazione}: installazione ed fornitura dei prodotti e/o servizi;
    \item \textbf{manutenzione}: correzione, adattamento, progressione.
\end{itemize}
\subsubsection{Processi di supporto}
Questi processi forniscono il supporto necessario per i processi primari e comprendono:
\begin{itemize}
    \item \textbf{documentazione}: comprende la generazione e la cura della documentazione correlata al software;
    \item \textbf{gestione della configurazione}: comprende le operazioni per la gestione delle configurazioni del software, come la gestione delle versioni e il controllo delle modifiche;
    \item \textbf{assicurazione della qualità}: questo processo si occupa delle operazioni per assicurare che il software rispetti gli standard di qualità prefissati;
    \item \textbf{verifica}: implica l’analisi e la valutazione dei prodotti software per assicurare che rispondano ai requisiti definiti;
    \item \textbf{validazione}: questo processo si focalizza sulla verifica che il software risponda alle necessità dell’utente e si integri adeguatamente nell’ambiente di lavoro;
    \item \textbf{revisioni congiunte con il cliente}: questo processo coinvolge il cliente nelle operazioni di analisi e valutazione del software;
    \item \textbf{verifiche ispettive interne}: coinvolge il team di sviluppo nelle attività di revisione e valutazione del software;
    \item \textbf{risoluzione dei problemi}: coinvolge l'identificazione e la risoluzione dei problemi nel software.
\end{itemize}
\subsubsection{Processi organizzativi}
Questi processi supportano l'organizzazione nel suo insieme e si compongono di:
\begin{itemize}
    \item \textbf{gestione}: risoluzione dei problemi dei processi;
    \item \textbf{gestione delle infrastrutture}: disposizione degli strumenti di assistenza ai processi;
    \item \textbf{gestione dei processi}: manutenzione progressiva dei processi;
    \item \textbf{formazione}: supporto, motivazione e integrazione all’auto-apprendimento;
    \item \textbf{amministrazione}: questo processo riguarda l'amministrazione generale dei processi e delle risorse necessarie per il loro funzionamento.
\end{itemize}

\newpage


\section{Metriche di qualità}
\subsection{Metriche per la qualità di processo}
\subsection{Metriche per la qualità di prodotto}



\newpage


%----------------TABLE EXAMPLE----------------%
% \begin{table}[h!]
% 	\centering
% 	\begin{tabular}{|l|l|l|l|} 
% 	 \hline
% 	 Metrica & Nome & Valore ammissibile & Valore ottimo \\  
% 	 \hline
% 	 MPC01 & Earned Value (EV) & $\geq 0$ & $\leq$ EAC \\
% 	 \hline
% 	 MPC02 & Planned Value (PV) & $\geq 0$ & $\leq$ Budget At Completion (BAC) \\ 
% 	 \hline
% 	 MPC03 & Actual Cost (AC) & $\geq 0$ & $\leq$ EAC \\ 
% 	 \hline
% 	 MPC04 & Cost Variance (CV) & $\geq -7.5\%$ & $\geq 0\%$ \\ 
% 	 \hline
% 	 MPC05 & Schedule Variance (SV) & $\geq -7.5\%$ & $\geq 0\%$ \\ 
% 	 \hline
% 	 MPC06 & Estimated at Completion (EAC) & Errore del $\pm 3\%$ rispetto al BAC  & Equivalente al BAC \\ 
% 	 \hline
% 	 MPC07 & Estimate to Complete (ETC)  & $\geq 0$ & $\leq$ EAC \\ 
% 	 \hline
% 	\end{tabular}
%     \caption{Valori delle metriche inerenti al processo di Fornitura}
% 	\label{table:1}
% \end{table}
%---------------------------------------------%


%----------------FIGURE EXAMPLE---------------%

% \begin{figure*}
%     \centering
%     \begin{tikzpicture}
%         \begin{axis}[
%             width=\textwidth,
%             height=0.5\textwidth,
%             ybar,
%             enlargelimits=0,
%             legend style={at={(0.5,1.15)},
%               anchor=north,legend columns=-1},
%             ylabel={Tempo dedicato},    
%             symbolic x coords={Antonio, Davide, Elena, Leonardo, Matteo, Raul, Valerio},
%             xtick=data,
%             % nodes near coords,
%             nodes near coords align={vertical},
%             ]
%         \addplot coordinates {(Antonio,4) (Davide,4) (Elena,4) (Leonardo,4) (Matteo,4) (Raul,4) (Valerio,4)};
%         \addplot coordinates {(Antonio,4) (Davide,4) (Elena,4) (Leonardo,4) (Matteo,4) (Raul,4) (Valerio,4)};
%         \addplot coordinates {(Antonio,4) (Davide,4) (Elena,4) (Leonardo,4) (Matteo,4) (Raul,4) (Valerio,4)};
%         \addplot coordinates {(Antonio,4) (Davide,4) (Elena,4) (Leonardo,4) (Matteo,4) (Raul,4) (Valerio,4)};
%         \addplot coordinates {(Antonio,4) (Davide,4) (Elena,4) (Leonardo,4) (Matteo,4) (Raul,4) (Valerio,4)};
%         \legend{Amministratore, Responsabile, Verificatore, Analista, Progettista}
%         \end{axis}
%     \end{tikzpicture}
% \caption{impegno preventivo per ruolo di ciascun membro del team durante il primo periodo}
% \end{figure*}

%---------------------------------------------%













% \section{Aspettative}
% Il gruppo \textit{7Last} si impegna ad instaurare un rapporto di collaborazione con il \textit{Sync Lab}, il committente del progetto, basato su trasparenza, comunicazione e rispetto reciproco, in modo da identificare e soddisfare al meglio le esigenze del proponente.
% Il gruppo quindi intende ricevere feedback costante sul lavoro svolto, in modo da poter correggere eventuali errori e migliorare la qualità del prodotto finale e verificare che i vincoli individuati siano rispettati.

% \subsection{Rapporti con il proponente}
% Il proponente \textit{Sync Lab} fornisce al gruppo \textit{7Last} le seguenti informazioni:
% \begin{itemize}
% 	\itemsep0em
% 	\item un indirizzo mail a cui inviare le comunicazioni;
% 	\item TODO capire dove fare le chiamate e come comunicare dal prossimo incontro
% \end{itemize}

% \subsubsection{Documentazione fornita}
% Di seguito sono elencati i documenti forniti da \textit{7last} al proponente \textit{Sync Lab} e ai committenti \textit{Prof. Tullio Vardanega} e \textit{Prof. Riccardo Cardin}.
% TODO Come si fanno le subsubsubsection?
% TODO Elencare e descrivere i documenti

% \subsubsection{Strumenti}
% Per lo sviluppo del progetto e la gestione del processo di fornitura, il gruppo utilizzerà i seguenti strumenti:
% \begin{itemize}
% 	\itemsep0em
% 	\item \textbf{GitHub}: per la gestione del codice sorgente e la collaborazione tra i membri del gruppo;
% 	\item \textbf{Telegram}: per la comunicazione interna tra i membri del gruppo a livello di chat;
% 	\item \textbf{Discord}: per la comunicazione attraverso chiamate vocali.
% \end{itemize}

% \subsubsection{Sviluppo}

% \section{Processi di supporto}
% \subsection{Documentazione}
% \subsection{Verifica}
% \subsection{Validazione}
% \subsection{Gestione della configurazione}
% \subsection{Gestione della qualità}

% \section{Processi organizzativi}
% \subsection{Gestione dei processi}
% \subsection{Miglioramnento}
% \subsection{Formazione}

% \section{Standard ISO}

% \section{Metriche di qualità del processo}
% \subsection{Processi di supporto}
% \subsection{Processi organizzativi}

% \section{Metriche di qualità del prodotto}

\end{document}
