\section{Metriche di qualità}
\subsection{Metriche per la qualità di processo}
\begin{itemize}
    \item \textbf{Metrica 1M-EV}:
    \begin{itemize}
        \item \textbf{Nome}: Earned Value;
        \item \textbf{Descrizione}: valore del lavoro effettivamente svolto fino al determinato periodo;
        \item \textbf{Formula}: $EV = EAC \times \%lavoro\:svolto$;
    \end{itemize}
\end{itemize}

\begin{itemize}
    \item \textbf{Metrica 2M-PV}:
    \begin{itemize}
        \item \textbf{Nome}: Planned Value;
        \item \textbf{Descrizione}: stima la somma dei costi realizzativi delle attività imminenti, periodo per periodo;
        \item \textbf{Formula}: $PV = BAC \times \%lavoro\:svolto$;
    \end{itemize}
\end{itemize}

\begin{itemize}
    \item \textbf{Metrica 3M-AC}:
    \begin{itemize}
        \item \textbf{Nome}: Actual Cost;
        \item \textbf{Descrizione}: misura i costi effettivamente sostenuti dall'inizio del progetto fino al presente momento;
        \item \textbf{Formula}: dato reperibile e costantemente aggiornato in "Piano di Progetto v1.0.0";
    \end{itemize}
\end{itemize}

\begin{itemize}
    \item \textbf{Metrica 4M-CV}:
    \begin{itemize}
        \item \textbf{Nome}: Cost Variance;
        \item \textbf{Descrizione}: misura la differenza percentuale di budget tra quanto previsto nella pianificazione di un periodo e l'effettiva realizzazione \textbf{CONTROLLARE};
        \item \textbf{Formula}: $CV = EV - AC$;
    \end{itemize}
\end{itemize}

\begin{itemize}
    \item \textbf{Metrica 5M-SV}:
    \begin{itemize}
        \item \textbf{Nome}: Schedule Variance;
        \item \textbf{Descrizione}: indica in percentuale quanto si è in anticipo o in ritardo rispetto alla pianificazione;
        \item \textbf{Formula}: $SV = EV - PV$;
    \end{itemize}
\end{itemize}

\begin{itemize}
    \item \textbf{Metrica 6M-EAC}:
    \begin{itemize}
        \item \textbf{Nome}: Estimated at Completion;
        \item \textbf{Descrizione}: misura il costo realizzativo stimato per terminare il progetto;
        \item \textbf{Formula}: $EAC = BAC \div CPI$;
    \end{itemize}
\end{itemize}

\begin{itemize}
    \item \textbf{Metrica 7M-ETC}:
    \begin{itemize}
        \item \textbf{Nome}: Estimate to Complete;
        \item \textbf{Descrizione}: Stima dei costi realizzativi fino alla fine del progetto;
        \item \textbf{Formula}: $ETC = EAC - AC$;
    \end{itemize}
\end{itemize}

\begin{itemize}
    \item \textbf{Metrica 8M-RSI}:
    \begin{itemize}
        \item \textbf{Nome}: Requirements Stability Index;
        \item \textbf{Descrizione}: misura impiegata nella quantificazione dell'entità e dell'impatti dei cambiamenti ai requisiti di un progetto;
    \end{itemize}
\end{itemize}

\begin{itemize}
    \item \textbf{Metrica 9M-SFIN}:
    \begin{itemize}
        \item \textbf{Nome}: Structural Fan In;
        \item \textbf{Descrizione}: si riferisce ad una classe che è stata progettata in modo tale da essere utilizzata facilmente da molte altre classi;
    \end{itemize}
\end{itemize}

\begin{itemize}
    \item \textbf{Metrica 10M-SFOUT}:
    \begin{itemize}
        \item \textbf{Nome}: Structural Fan Out;
        \item \textbf{Descrizione}: numero di moduli subordinati immediati di un metodo;
    \end{itemize}
\end{itemize}



\begin{itemize}
    \item \textbf{Metrica 13M-CC}:
    \begin{itemize}
        \item \textbf{Nome}: Code Coverage;
        \item \textbf{Descrizione}: rappresenta il grado in cui il codice sorgente di un programma è testato;
    \end{itemize}
\end{itemize}

\begin{itemize}
    \item \textbf{Metrica 14M-PTCP}:
    \begin{itemize}
        \item \textbf{Nome}: Passed Test Cases Percentage;
        \item \textbf{Descrizione}: percentuale di casi di test superati;
        \item \textbf{Formula}: $PTCP = \frac{Casi\: di\: test\: superati}{Casi\: di\: test\: totali}\: \times \: 100$;
    \end{itemize}
\end{itemize}

\begin{itemize}
    \item \textbf{Metrica 16M-NCR}:
    \begin{itemize}
        \item \textbf{Nome}: Non Calculated Risks;
        \item \textbf{Descrizione}: indica il numero di rischi non calcolati nel documento di "Analisi dei Requisiti v1.0.0";
    \end{itemize}
\end{itemize}

\subsection{Metriche per la qualità di prodotto}

\begin{itemize}
    \item \textbf{Metrica 15M-QMS}:
    \begin{itemize}
        \item \textbf{Nome}: Quality Metrics Satisfied;
        \item \textbf{Descrizione}: misura che valuta quante metriche, tra quelle definite, sono state implementate e soddisfatte;
        \item \textbf{Formula}: $QMS = \frac{Metriche\: soddisfatte}{Metriche\: totali}\: \times \: 100$;
    \end{itemize}
\end{itemize}

\begin{itemize}
    \item \textbf{Metrica 17M-TE}:
    \begin{itemize}
        \item \textbf{Nome}: Time Efficiency;
        \item \textbf{Descrizione}: indicante il livello di efficacia da parte del teamo nello sviluppo di codice di alta qualità;
        \item \textbf{Formula}: $TE = \frac{Tempo\: impiegato\: per\: lo\: sviluppo}{Tempo\: previsto\: per\: lo\: sviluppo}\: \times \: 100$;
    \end{itemize}
\end{itemize}

\begin{itemize}
    \item \textbf{Metrica 18M-CRO}:
    \begin{itemize}
        \item \textbf{Nome}: Copertura dei Requisiti Obbligatori;
        \item \textbf{Descrizione}: metrica che valuta quanto del lavoro svolto durante lo sviluppo corrisponda ai requisiti essenziali o obbligatori definiti in fase di analisi dei requisiti;
        \item \textbf{Formula}: ;
    \end{itemize}
\end{itemize}

\begin{itemize}
    \item \textbf{Metrica 19M-CRD}:
    \begin{itemize}
        \item \textbf{Nome}: Copertura dei Requisiti Desiderabili;
        \item \textbf{Descrizione}: valuta quanti di quei requisiti che, se integrati arricchirebbero l'esperienza utente o fornirebbero vantaggi aggiuntivi non strettamente necessari, sono stati implementati o sodisfatti nel prodotto;
        \item \textbf{Formula}: ;
    \end{itemize}
\end{itemize}

\begin{itemize}
    \item \textbf{Metrica 20M-CROP}:
    \begin{itemize}
        \item \textbf{Nome}: Copertura dei Requisiti Opzionali;
        \item \textbf{Descrizione}: valuta quanti dei requisiti aggiuntivi, non essenziali o di bassa priorità, sono stati implementati o soddisfatti nel prodotto;
        \item \textbf{Formula}: ;
    \end{itemize}
\end{itemize}

\begin{itemize}
    \item \textbf{Metrica 11M-IG}:
    \begin{itemize}
        \item \textbf{Nome}: Indice Gulpease;
        \item \textbf{Descrizione}: misura la leggibilità di un testo in base alla lunghezza delle parole e delle frasi;
        \item \textbf{Formula}: $IG = 89 + \frac{300 \:\times \:Numero\:frasi \:- \:10 \:\times\: Numero\:lettere}{Numero\:parole}$;
    \end{itemize}
\end{itemize}

\begin{itemize}
    \item \textbf{Metrica 12M-CO}:
    \begin{itemize}
        \item \textbf{Nome}: Correttezza Ortografica;
        \item \textbf{Descrizione}: misura la presenza di errori ortografici nei documenti;
        \item \textbf{Caratteristiche}: affidabilità;
    \end{itemize}
\end{itemize}

\begin{itemize}
    \item \textbf{Metrica 21M-CC}:
    \begin{itemize}
        \item \textbf{Nome}: Code Coverage;
        \item \textbf{Descrizione}: fornisce una misura quantitativa del grado o della percentuale di codice eseguito durante i test;
    \end{itemize}
\end{itemize}

\begin{itemize}
    \item \textbf{Metrica 22M-BC}:
    \begin{itemize}
        \item \textbf{Nome}: Branch Coverage;
        \item \textbf{Descrizione}: metrica di copertura del codice che indica la percentuale dei rami decisione del codice coperti dai testi;
        \item \textbf{Formula}: $BC =\frac{Flussi\:funzionali\: testati}{Flussi\:condizionali\: riusciti\: e\: non}\: \times \: 100$;
    \end{itemize}
\end{itemize}

\begin{itemize}
    \item \textbf{Metrica 23M-SC}:
    \begin{itemize}
        \item \textbf{Nome}: Statement Coverage;
        \item \textbf{Descrizione}: metrica di copertura del codice che indica la percentuale degli statement del codice coperti dai test;
        \item \textbf{Formula}: $SC = \frac{Statement\: testati}{Statement\: totali}\: \times \: 100$;
    \end{itemize}
\end{itemize}

\begin{itemize}
    \item \textbf{Metrica 24M-FD}:
    \begin{itemize}
        \item \textbf{Nome}: Failure Density;
        \item \textbf{Descrizione}: misura che indica il numero di difetti trovati in un software o in una parte di esso durante il ciclo di sviluppo;
    \end{itemize}
\end{itemize}

\begin{itemize}
    \item \textbf{Metrica 25M-FU}:
    \begin{itemize}
        \item \textbf{Nome}: Facilità di Utilizzo;
        \item \textbf{Descrizione}: metrica che misura l'usabilità di un sistema software;
        \item \textbf{Caratteristiche}: usabilità;
    \end{itemize}
\end{itemize}

\begin{itemize}
    \item \textbf{Metrica 26M-TA}:
    \begin{itemize}
        \item \textbf{Nome}: Tempo di Apprendimento;
        \item \textbf{Descrizione}: misura il tempo massimo richiesto per apprendere l'utilizzo del prodotto;
        \item \textbf{Caratteristiche}: uasbilità;
    \end{itemize}
\end{itemize}

\begin{itemize}
    \item \textbf{Metrica 27M-UR}:
    \begin{itemize}
        \item \textbf{Nome}: Utilizzo Risorse;
        \item \textbf{Descrizione}: ;
        \item \textbf{Formula}: ;
    \end{itemize}
\end{itemize}

\begin{itemize}
    \item \textbf{Metrica 28M-CCM}:
    \begin{itemize}
        \item \textbf{Nome}: Complessità Ciclomatica;
        \item \textbf{Descrizione}: rappresenta la complessità di un metodo in base ai percorsi possibili \textbf{CONTROLLARE};
        \item \textbf{Formula}: $CCM = e\: -\: n\: +\: 2$;
        \item \textbf{Legenda}: 
        \begin{itemize}
            \item $e$: numero di archi del grafo del flusso di esecuizione del metodo;
            \item $n$: numero di nodi del grafo del flusso di esecuzione del metodo.
        \end{itemize}
        \item \textbf{Caratteristiche}: manutenibilità;
    \end{itemize}
\end{itemize}

\begin{itemize}
    \item \textbf{Metrica 29M-CSM}:
    \begin{itemize}
        \item \textbf{Nome}: Code Smell;
        \item \textbf{Descrizione}: livello di pulizia e manutenibilità del codice sviluppato;
    \end{itemize}
\end{itemize}

\begin{itemize}
    \item \textbf{Metrica 30M-COC}:
    \begin{itemize}
        \item \textbf{Nome}: Coefficient of Coupling;
        \item \textbf{Descrizione}: rappresenta il grado di dipendenza tra diversi moduli o componenti di un sistema software;
        \item \textbf{Formula}: $COC = \frac{Numero\: di\: dipendenze}{Numero\: di\: moduli}$;
    \end{itemize}
\end{itemize}