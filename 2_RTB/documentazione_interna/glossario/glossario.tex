\documentclass[italian,12pt]{article} %tipo di documento

%--------------variabili------------------%
\def\Title{Norme di Progetto}
\def\Author{7Last}
\def\Version{v0.2}
%-----------------------------------------%


\usepackage[left=2cm, right=2cm, bottom=3cm, top=3cm]{geometry}
\usepackage{fancyhdr}
\usepackage{graphicx}
\graphicspath{ {../../logo/} }
\usepackage{href-ul}
\usepackage{tikz}
\usepackage{tgadventor}
\usepackage[useregional=numeric,showseconds=true,showzone=false]{datetime2}
\usepackage{caption}
\usepackage{longtable}
\usepackage{xcolor}




\linespread{1.2}
\captionsetup[table]{labelformat=empty}
\geometry{headsep=1.5cm}

\renewcommand{\contentsname}{Indice}%imposto il nome dell'indice
\renewcommand\familydefault{\sfdefault}
% define a custom command for glossary terms definitions that uses paragraphs
\newcommand{\glossdef}[2]{\paragraph{#1}#2}

%-------------------INIZIO DOCUMENTO--------------
\begin{document}

\newgeometry{left=2cm,right=2cm,bottom=2.1cm,top=2.1cm}
\begin{titlepage}
	\vspace*{.5cm}

	\vspace{2cm}
	{
		\centering
		{\bfseries\huge \Title\par}
		\bigbreak
		{\bfseries\Large \Subtitle\par}
		\bigbreak
		{\bfseries\large \Author\par}
		\bigbreak
		{\Date\;-\;\Version\par}
		\vfill

		\begin{center}
			\begin{tikzpicture}
				\clip (0,0) circle (2cm) node {\includegraphics[width=4cm]{logo.jpg}};
			\end{tikzpicture}
		\end{center}
	}

	\vfill

\end{titlepage}

\restoregeometry






















\newpage

\pagestyle{fancy}
\fancyhead{}
\lhead{
	\begin{tikzpicture}
		\clip (0,0) circle (0.5cm);
		\node at (0,0) {\includegraphics[width=1cm]{./../logo/logo.png}};
	\end{tikzpicture}%
}
\chead{\vspace{\fill}\Title\vspace{\fill}}
\rhead{\vspace{\fill}\Version\vspace{\fill}}


%-----------tabella revisioni-----------%
\begin{table}[!h]
	\caption{Versioni}
	\begin{center}
		\begin{tabular}{ c c c p{9cm} }
			\hline                                                   \\[-2ex]
			Ver. & Data       & Autore         & Descrizione         \\
			\\[-2ex] \hline \\[-1.5ex]
			0.1  & 28/03/2024 & Leonardo Baldo & Creazione documento \\
			\\[-1.5ex] \hline
		\end{tabular}
	\end{center}
\end{table}
%---------------------------------------%

\newpage

\tableofcontents

\addtocontents{toc}{\protect\setcounter{secnumdepth}{-1}} %disattivo la numerazione delle sezioni

\newpage

%-----------A-----------%
\section{A}



\newpage

%-----------B-----------%
\section{B}



\newpage

%-----------C-----------%
\section{C}
\glossdef{ClickUp}{Software di gestione dei progetti che offre varie funzionalità, tra cui la gestione di task, una lavagna virtuale, fogli di calcolo e strumenti collaborativi per la creazione e modifica di documenti, il tutto accessibile da una piattaforma unificata.
}
\glossdef{CMMI}{Capability Maturity Model Integration, modello di riferimento per l'ottimizzazione dei processi di sviluppo del software, che definisce una serie di best practices e linee guida per migliorare la qualità e l'efficienza dei processi aziendali.
}



\newpage

%-----------D-----------%
\section{D}



\newpage

%-----------E-----------%
\section{E}



\newpage

%-----------F-----------%
\section{F}



\newpage

%-----------G-----------%
\section{G}
\glossdef{Glossario}{
	Elenco strutturato di termini tecnici o specializzati, ognuno corredato dalla propria definizione o spiegazione. Questo strumento aiuta a migliorare la comunicazione tra le varie parti coinvolte in un progetto, riducendo le ambiguità e garantendo una comprensione condivisa dei termini utilizzati in un determinato contesto.
}



\newpage

%-----------H-----------%
\section{H}



\newpage

%-----------I-----------%
\section{I}



\newpage

%-----------J-----------%
\section{J}



\newpage

%-----------K-----------%
\section{K}



\newpage

%-----------L-----------%
\section{L}



\newpage

%-----------M-----------%
\section{M}



\newpage

%-----------N-----------%
\section{N}



\newpage

%-----------O-----------%
\section{O}



\newpage

%-----------P-----------%
\section{P}



\newpage

%-----------Q-----------%
\section{Q}



\newpage

%-----------R-----------%
\section{R}



\newpage

%-----------S-----------%
\section{S}



\newpage

%-----------T-----------%
\section{T}



\newpage

%-----------U-----------%
\section{U}



\newpage

%-----------V-----------%
\section{V}



\newpage

%-----------W-----------%
\section{W}



\newpage

%-----------X-----------%
\section{X}



\newpage

%-----------Y-----------%
\section{Y}



\newpage

%-----------Z-----------%
\section{Z}



\end{document}
