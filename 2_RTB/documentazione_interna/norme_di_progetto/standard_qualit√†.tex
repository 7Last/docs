\section{Standard per la qualità}
Nel corso dell’analisi e della valutazione della qualità dei processi e del software, adotteremo standard internazionali ben definiti per garantire una valutazione rigorosa e conforme agli standard globali. In particolare, la suddivisione dei processi in primari, di supporto e organizzativi sarà guidata dall’adozione dello standard ISO/IEC 12207:1995. Infine, l’adozione dello standard ISO/IEC 25010:2023 ci fornirà un quadro completo per la definizione e la suddivisione delle metriche di qualità del software. L’utilizzo congiunto di questi standard consentirà un approccio completo e strutturato alla valutazione della qualità dei processi e del software, assicurando un’elevata coerenza, affidabilità e conformità agli standard riconosciuti a livello internazionale. È stato deciso di applicare solo questi due standard in quando lo standard ISO/IEC 9126:2001 è stato ritirato e sostituito dallo standard ISO/IEC 25010:2023.
\subsection{Caratteristiche del sistema ISO/IEC 25010:2023}
\subsubsection{Appropriatezza funzionale}
\begin{itemize}
    \item \textbf{Completezza}: il prodotto software deve soddisfare tutti i requisiti definiti e attesi dagli utenti;
    \item \textbf{Correttezza}: il prodotto software deve funzionare come previsto e produrre risultati accurati;
    \item \textbf{Appropriatezza}: il prodotto software deve essere adatto allo scopo previsto e al contesto di utilizzo.
\end{itemize}
\subsubsection{Performance}
\begin{itemize}
    \item \textbf{Tempo}: il prodotto software deve rispettare le scadenze e i tempi di consegna previsti;
    \item \textbf{Risorse}: il prodotto software deve utilizzare le risorse di sistema in modo efficiente e ragionevole;
    \item \textbf{Capacità}: il prodotto software deve essere in grado di gestire il carico di lavoro previsto.
\end{itemize}
\subsubsection{Compatibilità}
\begin{itemize}
    \item \textbf{Coesistenza}: il prodotto software deve essere in grado di coesistere con altri software e sistemi sul computer;
    \item \textbf{Interoperabilità}: il prodotto software deve essere in grado di scambiare informazioni e collaborare con altri software e sistemi.
\end{itemize}
\subsubsection{Usabilità}
\begin{itemize}
    \item \textbf{Riconoscibilità}: il prodotto software deve avere un'interfaccia utente intuitiva e facile da usare;
    \item \textbf{Apprendibilità}: gli utenti devono essere in grado di imparare a utilizzare il prodotto software in modo rapido e semplice;
    \item \textbf{Operabilità}: il prodotto software deve essere facile da usare e da controllare;
    \item \textbf{Protezione} errori: il prodotto software deve essere in grado di rilevare e gestire gli errori in modo efficace;
    \item \textbf{Esteticità}: il prodotto software deve avere un'interfaccia utente piacevole e accattivante;
    \item \textbf{Accessibilità}: il prodotto software deve essere accessibile a persone con disabilità.
\end{itemize}
\subsubsection{Affidabilità}
\begin{itemize}
    \item \textbf{Maturità}: il prodotto software deve essere stabile, affidabile e robusto;
    \item \textbf{Disponibilità}: il prodotto software deve essere disponibile quando necessario;
    \item \textbf{Tolleranza}: il prodotto software deve essere in grado di tollerare errori e condizioni inaspettate;
    \item \textbf{Ricoverabilità}: il prodotto software deve essere in grado di ripristinare i dati e le funzionalità in caso di guasto o errore.
\end{itemize}
\subsubsection{Sicurezza}
\begin{itemize}
    \item \textbf{Riservatezza}: il prodotto software deve proteggere i dati sensibili e le informazioni riservate;
    \item \textbf{Integrità}: il prodotto software deve garantire l'accuratezza e la completezza dei dati;
    \item \textbf{Non} ripudio: il prodotto software deve garantire che le transazioni e le comunicazioni non possano essere negate o ripudiate;
    \item \textbf{Autenticazione}: il prodotto software deve verificare l'identità degli utenti e garantire che solo gli utenti autorizzati possano accedere al sistema;
    \item \textbf{Autenticità}: il prodotto software deve garantire che l'origine dei dati e delle informazioni sia verificabile.
\end{itemize}
\subsubsection{Manutenibilità}
\begin{itemize}
    \item \textbf{Modularità}: il prodotto software deve essere progettato in modo modulare, con componenti indipendenti e ben definiti;
    \item \textbf{Riusabilità}: i componenti del prodotto software devono essere progettati per essere riutilizzati in altri progetti;
    \item \textbf{Analizzabilità}: il prodotto software deve essere progettato in modo da essere facilmente analizzabile e comprensibile;
    \item \textbf{Modificabilità}: il prodotto software deve essere progettato in modo da essere facilmente modificabile e adattabile;
    \item \textbf{Testabilità}: il prodotto software deve essere progettato in modo da essere facilmente testabile.
\end{itemize}
\subsubsection{Portabilità}
\begin{itemize}
    \item \textbf{Adattabilità}: il prodotto software deve essere in grado di adattarsi a nuovi ambienti, requisiti e tecnologie;
    \item \textbf{Installabilità}: il prodotto software deve essere facilmente installabile e configurabile;
    \item \textbf{Sostituibilità}: il prodotto software deve essere facilmente sostituibile con altre soluzioni o versioni più recenti.
\end{itemize}
\subsection{Suddivisione secondo standard ISO/IEC 12207:1995}
\subsubsection{Processi primari}
Essenziali per lo sviluppo del software e comprendono:
\begin{itemize}
    \item \textbf{Acquisizione}: gestione dei propri sotto-fornitori;
    \item \textbf{Fornitura}: gestione delle relazioni con il cliente;
    \item \textbf{Sviluppo}: comprende tutte le attività legate alla progettazione, implementazione e verifica del software;
    \item \textbf{Operazione}: installazione ed fornitura dei prodotti e/o servizi;
    \item \textbf{Manutenzione}: correzione, adattamento, progressione.
\end{itemize}
\subsubsection{Processi di supporto}
Questi processi forniscono il supporto necessario per i processi primari e comprendono:
\begin{itemize}
    \item \textbf{Documentazione}: comprende la generazione e la cura della documentazione correlata al software;
    \item \textbf{Gestione della configurazione}: comprende le operazioni per la gestione delle configurazioni del software, come la gestione delle versioni e il controllo delle modifiche;
    \item \textbf{Assicurazione della qualità}: questo processo si occupa delle operazioni per assicurare che il software rispetti gli standard di qualità prefissati;
    \item \textbf{Verifica}: implica l’analisi e la valutazione dei prodotti software per assicurare che rispondano ai requisiti definiti;
    \item \textbf{Validazione}: questo processo si focalizza sulla verifica che il software risponda alle necessità dell’utente e si integri adeguatamente nell’ambiente di lavoro;
    \item \textbf{Revisioni congiunte con il cliente}: questo processo coinvolge il cliente nelle operazioni di analisi e valutazione del software;
    \item \textbf{Verifiche ispettive interne}: coinvolge il team di sviluppo nelle attività di revisione e valutazione del software;
    \item \textbf{Risoluzione dei problemi}: coinvolge l'identificazione e la risoluzione dei problemi nel software.
\end{itemize}
\subsubsection{Processi organizzativi}
Questi processi supportano l'organizzazione nel suo insieme e si compongono di:
\begin{itemize}
    \item \textbf{Gestione}: risoluzione dei problemi dei processi;
    \item \textbf{Gestione delle infrastrutture}: disposizione degli strumenti di assistenza ai processi;
    \item \textbf{Gestione dei processi}: manutenzione progressiva dei processi;
    \item \textbf{Formazione}: supporto, motivazione e integrazione all’auto-apprendimento;
    \item \textbf{Amministrazione}: questo processo riguarda l'amministrazione generale dei processi e delle risorse necessarie per il loro funzionamento.
\end{itemize}