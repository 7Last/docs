\section{Introduzione}
\subsection{Scopo del documento}
Questo documento ha lo scopo di descrivere le regole e le procedure che ogni membro del gruppo è tenuto a rispettare durante lo svolgimento del progetto. Lo scopo quindi è quello di definire il \textit{Way of Working} del team, in modo da garantire un lavoro efficiente e di qualità. La stesura di questo documento avverrà nelle prime fasi dello svolgimento del progetto, in modo tale da aggiornarsi e adattarsi alle esigenze del gruppo.\\ 
Il processo seguirà le linee guida descritte dallo standard ISO/IEC 12207:1995.

\subsection{Scopo del progetto}
Lo scopo del progetto è quello di realizzare una piattaforma di monitoraggio per una città smart, in grado di raccogliere e analizzare in tempo reale dati provenienti da diverse fonti, come sensori, dispositivi indossabili e macchine. La piattaforma avrà i seguenti scopi:
\begin{itemize}
    \item \textbf{migliorare} la qualità della vita dei cittadini: la piattaforma consentirà alle autorità locali di prendere decisioni informate e tempestive sulla gestione delle risorse e sull'implementazione di servizi, basandosi su dati reali e aggiornati;
    \item \textbf{coinvolgere} i cittadini: i dati monitorati saranno resi accessibili al pubblico attraverso portali online e applicazioni mobili, permettendo ai cittadini di essere informati sullo stato della loro città e di partecipare attivamente alla sua gestione;
    \item \textbf{gestire} i big data: sarà in grado di gestire grandi volumi di dati provenienti da diverse tipologie di sensori, aggregandoli, normalizzandoli e analizzandoli per estrarre informazioni significative.
\end{itemize}
La piattaforma si baserà su tecnologie di data streaming processing per l'analisi in tempo reale dei dati e su una piattaforma OLAP per la loro archiviazione e visualizzazione. La parte "IoT" del progetto sarà simulata attraverso tool di generazione di dati realistici.
Il progetto mira dunque a creare una piattaforma che sia \textbf{efficiente}, \textbf{efficace} e \textbf{accessibile}.

\subsection{Glossario}
Per evitare ambiguità e facilitare la comprensione del documento, si farà uso di un \textit{glossario} contenente la definizione dei termini tecnici e degli acronimi utilizzati. I termini di questo documento presenti nel glossario saranno riconoscibili in quanto accompagnati da una "G" a pedice e aventi il link alla relativa definizione nell'apposito documento online.

\subsection{Riferimenti}
\subsubsection{Riferimenti normativi}
\begin{itemize}
	\item Capitolato d'appalto C6 SyncCity: a smart city monitoring platform \\ \url{https://www.math.unipd.it/~tullio/IS-1/2023/Progetto/C6p.pdf}
	\item \textit{ISO/IEC 12207:1995} \\ \url{https://www.math.unipd.it/~tullio/IS-1/2009/Approfondimenti/ISO_12207-1995.pdf}
	
\end{itemize}
\subsubsection{Riferimenti informativi}
\begin{itemize}
    \item Glossario\\ \url{https://7last.github.io/docs/rtb/documentazione-interna/glossario}
    \item Documentazione Git\\ \url{https://git-scm.com/docs}
    \item Documentazione Latex\\ \url{https://www.latex-project.org/help/documentation/}
    \item Documentazione Python\\ \url{https://docs.python.org/3/}
    \item Documentazione Kafka\\ \url{https://kafka.apache.org/documentation/}
    \item Documentazione ClickHouse\\ \url{https://clickhouse.com/docs}
    \item Documentazione Grafana\\ \url{https://grafana.com/docs/grafana/latest/}
    \item Documentazione Docker\\ \url{https://docs.docker.com/}
\end{itemize}