\section{Processi organizzativi}
I processi organizzativi sono fondamentali per garantire che il progetto sia gestito in modo efficace, efficiente e conforme agli standard di qualità. Questi processi servono a coordinare le attività del team, a ottimizzare l'uso delle risorse e a mitigare i rischi, assicurando così il successo del progetto. 
\subsection{Gestione dei processi}
\subsubsection{Introduzione}
Le attività di gestione dei processi sono:
\begin{itemize}
    \item \textbf{pianificazione dei processi}: vengono definiti gli obiettivi, le fasi del progetto, le risorse necessarie e le scadenze. Questa fase stabilisce anche i criteri di successo e il piano di lavoro dettagliato;
    \item \textbf{definizione dei processi}: documentazione dei processi chiave che saranno utilizzati nel progetto, inclusi i processi di sviluppo del software, di controllo di versione, di gestione dei cambiamenti e di assicurazione della qualità;
    \item \textbf{assegnazione delle risorse}: assegnazione dei membri del team alle attività specifiche del progetto in base alle loro competenze e disponibilità;
    \item \textbf{monitoraggio e controllo}: si effettua un monitoraggio continuo del progresso del progetto rispetto al piano stabilito. Questo include il monitoraggio dei tempi, dei costi e della qualità, nonché l'identificazione e la gestione dei rischi;
    \item \textbf{gestione dei cambiamenti}: vengono effettuate una valutazione e gestione delle modifiche richieste durante lo sviluppo del software. Questo può includere modifiche ai requisiti, alla pianificazione o alla distribuzione delle risorse;
    \item \textbf{assicurazione della qualità}: implementazione di processi e procedure per garantire che il prodotto software soddisfi i requisiti e le aspettative del cliente;
    \item \textbf{comunicazione e coordinamento}: facilitazione della comunicazione tra i membri del team, gli stakeholder e altri soggetti interessati al progetto. Questo assicura che tutte le parti coinvolte siano informate sullo stato del progetto e sulle decisioni prese;
    \item \textbf{miglioramento continuo}: consiste in un'analisi dei processi utilizzati nel progetto al fine di identificare aree di miglioramento e implementare azioni correttive per ottimizzare l'efficienza e la qualità complessiva del lavoro svolto.
\end{itemize}
\subsubsection{Pianificazione}
\subsubsubsection{Descrizione}
La pianificazione dei processi implica la definizione, l'organizzazione e il controllo delle attività necessarie per portare a termine con successo il progetto. Si tratta di un'attività strategica che fornisce una guida chiara e una struttura gestionale per tutto il ciclo di vita del progetto. La pianificazione dei processi è essenziale per garantire che il progetto sia completato in tempo, entro il budget e con la qualità richiesta.
\subsubsubsection{Obiettivi}
Lo scopo principale della pianificazione dei processi è garantire che il progetto venga eseguito in modo efficiente, efficace e conforme agli obiettivi e ai requisiti stabiliti, assicurando allo stesso tempo che ciascun membro del team assuma ogni ruolo per almeno una volta. Questo processo serve anche a mitigare i rischi e ad affrontare le sfide in modo proattivo, consentendo al team di affrontare eventuali ostacoli lungo il percorso.

\subsubsubsection{Assegnazione dei ruoli}
Durante lo svolgimento del progetto, i membri di \textit{7Last} assumeranno ruoli distinti, tra cui:
\begin{itemize}
    \item \textbf{responsabile}, il quale si occupa di:
        \begin{itemize}
            \item coordinare il gruppo di lavoro;
            \item pianificare e controllare le attività;
            \item gestire le risorse;
            \item gestire le comunicazioni con l'esterno.
        \end{itemize}
    \item \textbf{Amministratore}, i suoi compiti sono:
        \begin{itemize}
            \item gestire l'ambiente di lavoro;
            \item gestione delle procedure e delle norme;
            \item gestione della configurazione del prodotto.
        \end{itemize}
    \item \textbf{Analista}, che:
        \begin{itemize}
            \item analizza i requisiti del progetto;
            \item redige l'analisi dei requisiti;
            \item studia il dominio applicativo del problema.
        \end{itemize}
    \item \textbf{Progettista}, il quale:
        \begin{itemize}
            \item progetta l'architettura del prodotto;
            \item prende decisioni tecniche e tecnologiche.
        \end{itemize}
    \item \textbf{Programmatore}: si occupa di:
        \begin{itemize}
            \item scrivere il codice del prodotto;
            \item implementare le funzionalità richieste;
            \item codifica le componenti dell'architettura del prodotto;
            \item redige il \textit{Manuale Utente}. 
        \end{itemize}
    \item \textbf{Verificatore}: il cui compito è:
        \begin{itemize}
            \item verificare che il lavoro svolto sia conforme alle norme e alle specifiche tecniche del progetto;
            \item ricercare ed eventualmente segnalare errori;
            \item redigere la sezione retrospettiva del \textit{Piano di Qualifica}.
        \end{itemize}
\end{itemize}

\subsubsubsection{Strumenti}
\begin{itemize}
    \item \textbf{GitHub}: utilizzato per la condivisione del codice tra i membri del gruppo;
    \item \textbf{ClickUp}: piattaforma utilizzata per il tracciamento e la gestione delle issue e dei compiti;
\end{itemize}
\subsubsection{Coordinamento}
\subsubsubsection{Descrizione}
La fase di coordinamento rappresenta il momento in cui convergono le molteplici variabili coinvolte nell'esecuzione del lavoro. Questa fase è come il nodo centrale di un intricato intreccio, in cui le attività, le risorse e le comunicazioni si fondono per creare un ambiente che facilita un'esecuzione efficace del progetto. È il punto di congiunzione vitale in cui le diverse componenti del progetto si uniscono, fornendo un quadro organizzativo che funge da guida e sostegno per il raggiungimento degli obiettivi prefissati.

\subsubsubsection{Obiettivi}
Lo scopo principale della fase di coordinamento è quello di creare armonia tra tutte le attività del progetto, integrando e sincronizzando ogni componente in modo da massimizzare l'efficienza e ottimizzare l'utilizzo delle risorse disponibili. Questo processo implica una gestione attenta delle risorse umane, finanziarie e temporali, assicurando che ogni membro del team sia adeguatamente informato, motivato e allineato sui compiti da svolgere e sugli obiettivi da raggiungere. Inoltre, il coordinamento mira a garantire che le risorse siano allocate in modo appropriato, evitando sovrapposizioni o carenze che potrebbero compromettere il successo del progetto. In definitiva, la fase di coordinamento agisce come il collante che tiene insieme tutte le componenti del progetto, consentendo al team di lavorare in modo collaborativo verso il raggiungimento del successo.\\\\A tal proposito, il gruppo \textit{7Last} si occupa di mantenere comunicazioni attive, sia interne che esterne, che possono essere sincrone o asincrone.
\subsubsubsection{Comunicazioni asincrone}
\begin{itemize}
    \item \textbf{comunicazione asincrone interne}: viene deciso di adottare Telegram, applicazione che consente di comunicare mediante l'utilizzo di messaggi testo, media e file in chat private o all'interno di gruppi;
    \item \textbf{comunicazione asincrone esterne}: vengono adottati due canali differenti per garantire le comunicazioni asincrone esterne, ovvero: 
        \begin{itemize}
            \item \textbf{E-mail}: per comunicazioni formali e ufficiali;
            \item \textbf{Discord}: in caso di necessità di risposta immediata e per comunicazioni informali.
        \end{itemize}
\end{itemize}
\subsubsubsection{Comunicazioni sincrone}
\begin{itemize}
    \item \textbf{comunicazione sincrone interne}: viene adottato Discord per questo scopo, il quale permette di comunicare tramite chiamate vocali, videochiamate, messaggi di testo, media e file in chat private o all'interno di gruppi;
    \item \textbf{comunicazione sincrone esterne}: in accordo con l'azienda \textit{SyncLab S.r.l.} viene scelto di adottare Google Meet per le comunicazioni sincrone esterne. 
\end{itemize}

\subsubsubsection{Riunioni interne}
Le riunioni interne avranno luogo ogni mercoledì, tramite Discord. Queste riunioni serviranno a monitorare il progresso delle attività, a discutere eventuali problematiche riscontrate e a pianificare le attività future. L'orario in cui si terranno queste riunioni è dalle 15:00 alle 16:00. Nel caso in cui un membro del gruppo non possa partecipare alla riunione, è tenuto a comunicarlo tempestivamente al responsabile, il quale provvederà a trovare un'altra data e un altro orario che possa andare bene a tutti i membri del gruppo. Le riunioni interne saranno registrate e il verbale sarà redatto dal redattore della riunione. In queste riunioni i compiti del responsabile sono:
\begin{itemize}
    \item stabilire l'\textbf{\textit{ordine del giorno}} della riunione;
    \item guidare la riunione e assicurarsi che tutti i membri forniscano il loro parere in modo ordinato;
    \item pianificare e assegnare le nuove attività da svolgere.
\end{itemize}

\subsubsubsection{Riunioni esterne}
Durante lo svolgimento del progetto, è essenziale organizzare vari incontri con i Committenti e/o il Proponente al fine di valutare lo stato di avanzamento del prodotto e chiarire eventuali dubbi o questioni.
La responsabilità di convocare tali incontri ricade sul responsabile, il quale è incaricato di pianificarli e agevolarne lo svolgimento in modo efficiente ed efficace.
Sarà compito del responsabile anche l’esposizione dei punti di discussione al proponente/committente, lasciando la parola ai membri del gruppo interessati quando necessario. Questo approccio assicura una comunicazione efficace tra le varie parti in causa, garantendo una gestione ottimale del tempo e una registrazione accurata delle informazioni rilevanti emerse durante gli incontri.
I membri del gruppo si impegnano a garantire la propria presenza in modo costante alle riunioni, facendo il possibile per riorganizzare eventuali altri impegni al fine di partecipare. Nel caso in cui gli obblighi inderogabili di un membro del gruppo rendessero impossibile la partecipazione, il responsabile assicurerà di informare tempestivamente il proponente o i committenti, richiedendo la possibilità di rinviare la riunione ad una data successiva.
\begin{flushleft}
\textbf{Riunioni con l'azienda proponente}:\\
\end{flushleft}
\textit{7Last} si impegna ad organizzare incontri regolari con \textit{SyncLab S.r.l.} per monitorare lo stato di avanzamento del progetto e affrontare eventuali dubbi o questioni. In linea con tale impegno, è stato concordato di tenere incontri di Stato Avanzamento Lavori (SAL) inizialmente ogni due settimane, con l'intenzione di aumentare la frequenza a uno ogni settimana in seguito. Gli incontri si svolgeranno tramite la piattaforma Google Meet e tratteranno i seguenti argomenti:
\begin{itemize}
    \item discussione e valutazione delle attività svolte nel periodo passato;
    \item pianificazione delle attività per il periodo successivo;
    \item chiarimento di eventuali dubbi emersi nel corso del periodo passato.
\end{itemize}
\begin{flushleft}
\textbf{Verbali esterni}
\end{flushleft}
Come avviene per le riunioni interne, anche per quelle esterne verrà stilato un verbale secondo le medesime modalità illustrate nella sezione relativa ai Verbali Interni. Tuttavia, la distinzione risiede nel fatto che il verbale redatto sarà successivamente inviato all'azienda per l'approvazione e la firma.

\subsubsubsection{Strumenti}
\begin{itemize}
    \item \textbf{Discord}: impiegato per la comunicazione sincrona e i meeting interni del team;
    \item \textbf{Telegram}: utilizzato per la comunicazione asincrona interna;
    \item \textbf{Google Meet}: adottato per le comunicazioni sincrone esterne;
    \item \textbf{Gmail}: come servizio di posta elettronica.
\end{itemize}

\subsubsubsection{Metriche}
\begin{table}[h]
	\centering
	\begin{tabular}{|c|c|}
		\hline
		\textbf{Metrica} & \textbf{Nome}                   \\
		\hline
		33M-BV           & Budget Variance                 \\
		32M-SV           & Schedule Variance               \\
		\hline
	\end{tabular}
	\caption{Metriche relative alla gestione della qualità}
    \label{tab:1}
\end{table}

\subsection{Miglioramento}
\subsubsection{Introduzione}
Secondo lo standard ISO/IEC 12207:1995, il processo di miglioramento nel ciclo di vita del software ha lo scopo di stabilire, misurare, controllare e migliorare i processi che lo compongono. L’attività di miglioramento è composta da:
\begin{itemize}
    \item \textbf{analisi dei dati}: questa fase coinvolge la raccolta e l'analisi dei dati pertinenti per identificare le aree che richiedono miglioramento;
    \item \textbf{valutazione delle prestazioni attuali}: consiste nel valutare le prestazioni attuali rispetto agli obiettivi desiderati o agli standard di riferimento. Questo aiuta a determinare dove sono necessari miglioramenti e quali sono le aree critiche che richiedono attenzione;
    \item \textbf{identificazione delle aree di miglioramento}: basandosi sull'analisi dei dati e sulla valutazione delle prestazioni attuali, vengono identificate le specifiche aree o processi che richiedono miglioramento;
    \item \textbf{pianificazione del miglioramento}: una volta identificate le aree di miglioramento, viene sviluppato un piano dettagliato che definisce gli obiettivi del miglioramento, le attività necessarie, le risorse coinvolte, le tempistiche e le metriche per valutare il successo;
    \item \textbf{implementazione delle modifiche}: coinvolge l'attuazione del piano di miglioramento, che può includere l'introduzione di nuovi processi, procedure o strumenti, la formazione del personale, la revisione dei flussi di lavoro esistenti, ecc..;
    \item \textbf{monitoraggio e valutazione}: una volta implementate le modifiche, è necessario monitorare e valutare i risultati per assicurarsi che il miglioramento sia stato efficace e che i risultati attesi siano stati raggiunti;
    \item \textbf{miglioramento}: attuazione di azioni correttive e preventive per ottimizzare i processi.
\end{itemize}

\subsection{Formazione}
\subsubsection{Introduzione}
La formazione è una componente essenziale per garantire che tutti i membri del team siano adeguatamente preparati per affrontare le sfide tecniche e gestionali del progetto. Questa fase del progetto si concentra sullo sviluppo delle competenze e delle conoscenze necessarie per utilizzare strumenti, tecnologie e metodologie specifiche del progetto, promuovendo così l'efficienza e la qualità del lavoro svolto.

\subsubsection{Metodo di formazione}
\subsubsubsection{Individuale}
Ogni inviduo del team dovrà compiere un processo di autoformazione per riuscire a svolgere il ruolo assegnato al meglio. La rotazione dei ruoli permetterà al nuovo occupante di un ruolo di apprendere le competenze necessarie da colui che ha precedentemente svolto il ruolo, nel caso avesse delle lacune. Questo metodo permette di avere una formazione continua e di garantire che ogni membro del team sia in grado di svolgere ogni ruolo.

\subsubsubsection{Gruppo}
\textit{SyncLab S.r.l.} mette a disposizione sessioni formative riguardo le tecnologie adottate nel progetto.