\section{Processi di supporto}

\subsection{Documentazione}
\subsubsection{Introduzione}
Il processo di documentazione è una componente fondamentale nella realizzazione e nel rilascio di un prodotto software,
poiché fornisce informazioni utili alle parti coinvolte e tiene traccia di tutte le attività relative al ciclo di vita del software,
comprese scelte e norme adottate dal gruppo durante lo svolgimento del progetto. In particolare, la documentazione è utile per:
\begin{itemize}
	\item Permettere una comprensione profonda del prodotto e delle sue funzionalità;
	\item Tracciare un confine tra disciplina e creatività;
	\item Garantire uno standard di qualità all'interno dei processi produttivi.
\end{itemize}
Lo scopo della sezione è:
\begin{itemize}
	\item Fornire una raccolta esaustiva di regole che i membri del gruppo devono seguire per agevolare la stesura della documentazione;
	\item Definire delle procedure ripetibili per uniformare la redazione, la verifica e l'approvazione dei documenti;
	\item Creare template per ogni tipologia di documento così da garantire omogeneità e coerenza.
\end{itemize}

\subsubsection{Documentation as Code}
Per la stesura della documentazione viene adottato l'approccio "Documentation as Code, che consiste nel trattare la documentazione di progetto
allo stesso modo del codice sorgente. Dunque vengono impiegati strumenti e metodologie tipiche dello sviluppo software:
\begin{itemize}
	\item \textbf{Versionamento}: la documentazione è versionata tramite il sistema di controllo di versione Git, in modo da tenere traccia
	      delle modifiche e garantire la coerenza tra le varie versioni;
	\item \textbf{Automazione}: l'automazione dei processi di build e di deploy permette di semplificare la gestione della documentazione e
	      di ridurre il rischio di errori;
	\item \textbf{Collaborazione}: l'uso di piattaforme di condivisione e collaborazione (GitHub) favorisce la cooperazione tra i membri
	      del gruppo e semplifica la gestione dei documenti;
	\item \textbf{Integrazione continua}: permette di verificare costantemente la correttezza e la coerenza della documentazione.
\end{itemize}

\subsubsection{Tipografia e sorgente documenti}
Per la redazione dei documenti abbiamo deciso di utilizzare il linguaggio di markup \LaTeX{}, in quanto semplifica la creazione e la manutenzione
dei documenti, liberando i redattori dall'onere della visualizzazione grafica e garantendo coerenza nella documentazione del progetto.
Inoltre, per favorire una migliore collaborazione tra i diversi autori, abbiamo scelto di scomporre ogni documento in più file,
ciascuno per una specifica sezione, in modo da permettere a più persone di lavorare sulle singole sezioni o sottosezioni.
Il risultato finale sarà ottenuto tramite l'assemblaggio di tutti i file sorgenti in un file principale, attraverso l'uso del comando
\textit{input{Sezione.tex}} o, nel caso delle sottosezioni, \textit{input{Sottosezione.tex}}.
Solo nel caso di documenti di piccole dimensioni, come i verbali, si potrà optare per la scrittura di un unico file.

\subsubsection{Ciclo di vita}
Il ciclo di vita di un documento è composto dalle seguenti fasi:
\begin{enumerate}
	\item \textbf{Pianificazione} della stesura e \textbf{suddivisione} in sezioni: tramite confronto con il gruppo, le sezioni del documento vengono stabilite e assegnate ai Redattori.
	      Essi sono responsabili della stesura delle proprie sezioni in conformità con le Norme di Progetto.
	\item \textbf{Stesura} del contenuto e \textbf{creazione} della bozza iniziale: i Redattori realizzano il documento redigendone il contenuto e creano una prima bozza
	      che viene utilizzata come punto di partenza per la discussione e la revisione.
	\item \textbf{Controllo} dei contenuti: dopo la stesura effettiva del documento, i Redattori verificano che e il contenuto delle proprie sezioni
	      sia conforme alle norme definite e non contenga errori di compilazione.
	\item \textbf{Revisione}: quando la redazione del documento è conclusa, questo viene revisionato dai Verificatori incaricati.
	\item \textbf{Approvazione e rilascio}: nell'ultima fase il documento viene approvato da un Responsabile e rilasciato in versione finale.
\end{enumerate}

\subsubsection{Procedure correlate alla redazione di documenti}

\subsubsubsection{I redattori}
Il redattore è colui che si occupa di scrivere e curare il contenuto di un documento o di una sua sezione in modo chiaro, accurato e comprensibile.
Nel farlo deve seguire lo stesso approccio impiegato nella codifica del software, adottando il
workflow noto come \textit{feature branch}.
\textbf{Caso redazione nuovo documento/sezione o modifica dei precedenti già verificati}
In queste situazioni il redattore dovrà creare un nuovo branch Git in locale e posizionarsi su di esso con i seguenti comandi:
\begin{itemize}
	\item \texttt{git checkout main}
	\item \texttt{git checkout -b nomeBranch}
\end{itemize}
In particolare l'identificativo del branch deve essere 'parlante', ossia descrittivo e significativo così da consentire una comprensione immediata
del documento o della sezione che si sta redigendo. Dunque il redattore deve adottare le specifiche
\href{#convenzioni_nomenclatura}{\underline{convenzioni per la nomenclatura dei branch}}.
Una volta terminata la redazione del documento o della sezione assegnata, è necessario rendere disponibile il branch
nella repository remota seguendo la seguente procedura:
\begin{enumerate}
	\item Eseguire il push delle modifiche nel branch:
	      \begin{itemize}
		      \item \texttt{git add .}
		      \item \texttt{git commit -m "Descrizione delle modifiche apportate"}
		      \item \texttt{git push origin nomeBranch}
	      \end{itemize}
	\item Se riscontriamo problemi nel punto 1:
	      \textit{git pull origin nomeBranch}
	\item Risolviamo i conflitti e ripetiamo il punto 1.
\end{enumerate}
\textbf{Caso modifica documento in fase di redazione}
Per continuare la redazione di un documento o di una sezione già in fase di stesura, sono necessari i seguenti comandi
\begin{itemize}
	\item \texttt{git pull}
	\item \texttt{git checkout nomeBranch}
\end{itemize}
\textbf{Completamento redazione documento}
Dopo aver completato la redazione del documento o della sezione, il redattore deve procedere nel seguente modo:
\begin{enumerate}
	\item Spostare l'issue relativa all'attività assegnata nella colonna "Review" della
	      \href{https://github.com/orgs/7Last/projects/1/views/1}{\underline{DashBoard}} del progetto, così da comunicare il completamento dell'incarico ai Verificatori.
	\item Aggiornare la tabella contenente il versionamento del documento, inserendo le informazioni richieste e incrementando la versione.
	\item Creare una Pull Request:
	      \begin{enumerate}
		      \item Accedere alla \href{https://github.com/7Last/docs}{\underline{Repository}} GitHub, spostarsi nella sezione "Pull Request" e cliccare su "New Pull Request";
		      \item Selezionare come branch di destinazione "main" e come branch sorgente il ramo creato appositamente per la redazione del documento/sezione;
		      \item Cliccare su "Create Pull Request";
		      \item Dare un titolo significativo e, se necessario, una descrizione alla Pull Request, selezionare i Verificatori e cliccare su "Create Pull Request".
	      \end{enumerate}
\end{enumerate}

\subsubsubsection{I verificatori}
Il ruolo e la procedura dei Verificatori sono descritti in dettaglio al paragrafo \href{#verifica_dei_documenti}{\underline{3.2.2}}.

\subsubsubsection{Il responsabile}
Per quanto riguarda la redazione dei documenti, il Responsabile ha il dovere di:
\begin{itemize}
	\item Identificare i documenti da redigere;
	\item Assegnare le task a Redattori e Verificatori;
	\item Stabilire la scadenza per il completamento delle attività;
	\item Approvare o richiedere eventuali modifiche ai documenti.
\end{itemize}

\subsubsubsection{L'amministratore}
L'amministratore è responsabile della gestione delle attività richieste dal Responsabile all'interno dell'ITS.

\subsubsection{Struttura del documento}
Tutta la documentazione prodotta segue uno schema strutturale ben definito e uniforme.
\subsubsubsection{Prima pagina}
Nella prima pagina di ogni documento è presente un'intestazione contenente le seguenti informazioni:
\begin{itemize}
	\item Nome del documento;
	\item Versione del documento;
	\item Logo del gruppo;
	\item Nome del gruppo.
\end{itemize}

\subsubsubsection{Registro delle modifiche}
La seconda pagina è dedicata al registro delle modifiche in formato tabellare e permette di tenere traccia delle modifiche apportate al documento.
La tabella riporta i seguenti dati:
\begin{itemize}
	\item Versione del documento;
	\item Data di rilascio;
	\item Nome dell'autore;
	\item Nome del Verificatore;
	\item Descrizione della modifica.
\end{itemize}

\subsubsubsection{Indice}
Ogni documento contiene un indice delle sezioni e delle sottosezioni presenti al suo interno, in modo da facilitare la consultazione e la navigazione.

\subsubsubsection{Intestazione}
Ogni pagina del documento, ad eccezione della prima, contiene un'intestazione che riporta il nome del documento, la versione e il logo del team.

\subsubsubsection{Verbali: struttura generale}
I verbali costituiscono un report dettagliato dei meeting, con lo scopo di tenere traccia degli argomenti trattati, delle decisioni adottate e le
azioni da intraprendere. Essi si suddividono in esterni o interni, a seconda che il meeting sia con persone esterne al gruppo o con i soli membri del team.
La struttura in ogni caso è la medesima e prevede le seguenti sezioni:
\begin{itemize}
	\item \textbf{Informazioni sulla riunione}:
	      \begin{itemize}
		      \item Sede del meeting;
		      \item Orario di inizio e fine;
		      \item Partecipanti del gruppo;
		      \item Partecipanti esterni.
	      \end{itemize}
	\item \textbf{Corpo del documento}
	      \begin{itemize}
		      \item \textbf{Revisione del periodo precedente}:
		            Analisi dello stato delle attività e dell'approccio lavorativo, si discute di eventuali problemi riscontrati
		            ma anche degli aspetti positivi in modo da incrementare e migliorare il way of working.
		      \item \textbf{Ordine del giorno}:
		            Elenco di ciò che verrà discusso durante la riunione.
		      \item \textbf{Sintesi dell'incontro}:
		            Breve riassunto delle discussioni e dei temi affrontati durante l'incontro.
		      \item \textbf{Decisioni prese}:
		            Sezione che elenca in formato testuale le decisioni prese durante il meeting. Alcune di queste potrebbero risultare anche in "Attività individuate".
		      \item \textbf{Attività individuate}:
		            Illustrazione dettagliata delle attività assegnate ai diversi membri del gruppo in forma tabellare, sono presenti le seguenti informazioni:
		            \begin{itemize}
			            \item Nome della task;
			            \item ID dell'issue su GitHub;
			            \item Assegnatari.
		            \end{itemize}
	      \end{itemize}
	\item \textbf{Ultima pagina}:
	      Solo nel caso di verbale esterno, è presente una sezione dedicata alla data e alla firma delle terze parti coinvolte.
\end{itemize}

\subsubsection{Norme tipografiche}
\textbf{Nomi assegnati ai file} Il nome dei documenti deve essere omogeneo alla tipologia di appartenenza, deve essere in minuscolo
e contenere un riferimento alla versione del documento. In particolare la nomenclatura dei file deve seguire la convenzione:
\begin{itemize}
	\item \textbf{Verbali}: verbale\_esterno/interno\_AA\_MM\_DD\_vX.Y;
	\item \textbf{Norme di Progetto}: norme\_di\_progetto\_vX.Y;
	\item \textbf{Analisi dei Requisiti}: analisi\_dei\_requisiti\_vX.Y;
	\item \textbf{Piano di Progetto}: piano\_di\_progetto\_vX.Y;
	\item \textbf{Glossario}: glossario\_vX.Y.
\end{itemize}
\textbf{Stile del testo}
\begin{itemize}
	\item \textbf{Grassetto}:
	      \begin{itemize}
		      \item Titoli di sezione;
		      \item Termini importanti;
		      \item Parole seguite da descrizione o elenchi puntati.
	      \end{itemize}
	\item \textbf{Corsivo}:
	      \begin{itemize}
		      \item Nome del gruppo e dell'azienda proponente;
		      \item Termini presenti nel glossario;
		      \item Riferimenti a documenti esterni.
	      \end{itemize}
	\item \textbf{Maiuscolo}:
	      \begin{itemize}
		      \item Acronimi;
		      \item Iniziali dei nomi;
		      \item Iniziali dei ruoli svolti dai membri del gruppo.
	      \end{itemize}
\end{itemize}
\textbf{Regole sintattiche}:
\begin{itemize}
	\item Negli elenchi ogni voce deve terminare con ";", ad eccezione dell'ultima che prevede ".";
	\item I numeri razionali si scrivono utilizzando la virgola come separatore tra parte intera e parte decimale;
	\item Le date devono seguire lo standard internazionale ISO 8601, ossia YYYY-MM-DD.
\end{itemize}

\subsubsection{Abbreviazioni}
Segue un elenco delle abbreviazioni più comuni utilizzate nei documenti:

\begin{table}[!h]
	\centering
	\begin{tabular}{|c|c|}
		\hline
		\textbf{Abbreviazione} & \textbf{Scrittura Estesa}            \\
		\hline
		RTB                    & Requirements and Technology Baseline \\
		PB                     & Product Baseline                     \\
		CA                     & Customer Acceptance                  \\
		ITS                    & Issue Tracking System                \\
		CI                     & Configuration Item                   \\
		SAL                    & Stato Avanzamento Lavori             \\
		\hline
	\end{tabular}
	\caption{Spiegazione delle abbreviazioni utilizzate nei documenti.}
	\label{tab:1}
\end{table}


\subsubsection{Strumenti}
Gli strumenti utilizzati per la redazione dei documenti sono:
\begin{itemize}
	\item \textbf{\LaTeX{}}: utilizzato per la stesura dei documenti;
	\item \textbf{GitHub}: utilizzato per la gestione del versionamento e per la condivisione dei documenti;
	\item \textbf{Visual Studio Code}: utilizzato come editor di testo per la scrittura dei documenti attraverso l'estensione \LaTeX{} Workshop.
\end{itemize}

\subsection{Verifica}
\subsubsection{Introduzione}
Il processo di verifica è fondamentale durante tutto il ciclo di vita del software, a partire dall'iniziale fase
di progettazione fino alla successiva manutenzione. La verifica ha lo scopo di garantire che ciascuna attività
sia corretta ed efficiente, identificando un processo di controllo per ogni prodotto realizzato. In particolare, ci si preoccupa che gli output
del software (documentazione, codice sorgente, test...) siano conformi alle aspettative e ai requisiti specificati. Nel farlo è fondamentale
applicare tecniche e analisi di test seguendo procedure definite e adottando criteri affidabili.
Le attività di verifica sono svolte dai Verificatori, i quali sono responsabili di analizzare i prodotti e valutare la loro aderenza agli standard
stabiliti. Il fulcro di questo processo è il \textit{Piano di Qualifica}, un documento dettagliato che traccia il percorso della verifica.
Questo fornisce linee guida per una valutazione accurata della qualità, delineando chiaramente gli obiettivi da raggiungere e i criteri di
accettazione da rispettare.

\subsubsection{Verifica dei documenti} \label{verifica_dei_documenti}
Nell'ambito della documentazione, la verifica è un'attività cruciale per garantire la correttezza e l'accuratezza dei contenuti. Essa si suddivide in:
\begin{itemize}
	\item \textbf{Revisione della correttezza tecnica}: assicura che tutte le informazioni siano corrette e coerenti con le norme stabilite;
	\item \textbf{Conformità alle norme}: verifica che il documento segua le linee guida e gli standard
	      stabiliti per la formattazione, la struttura e lo stile;
	\item \textbf{Revisione ortografica e grammaticale}: controlla che il testo sia privo di errori ortografici, grammaticali e di punteggiatura.
	\item \textbf{Chiarezza e comprensibilità}: valuta la leggibilità del documento, verificando che il contenuto sia chiaro, comprensibile e privo di ambiguità;
	\item \textbf{Coerenza}: verifica che il documento sia omogeneo e coerente, sia internamente che con i documenti correlati.
\end{itemize}
% Specificare procedura GitHub? Pull request, branch, spostare la issue ecc.

\subsubsection{Analisi}
L'analisi è un processo che si occupa di valutare la qualità degli oggetti statici (documenti e codice sorgente) e dinamici
(test ed esecuzione del software).

\subsubsubsection{Analisi statica}
L'analisi statica è un'attività di controllo che prescinde dall'esecuzione del prodotto e si basa su una revisione manuale o automatica
del codice e della documentazione. Essa è fondamentale per verificare la presenza di proprietà desiderate e la conformità ai vincoli e
per garantire che non siano presenti errori o difetti. L'analisi statica prevede due metodi di lettura:

\subsubsubsection{Walkthrough}
Questa tecnica prevede una lettura integrale e approfondita del prodotto, con l'obiettivo di individuare errori e difetti. Dunque lo scopo della verifica
non è specifica per un determinato tipo di errore, ma generale. Inoltre il walkthrough è un approccio collaborativo che coinvolge il Verificatore
e l'autore del prodotto, in particolare esso si svolge in quattro fasi:
\begin{enumerate}
	\item \textbf{Pianificazione}: il Verificatore e l'autore si confrontano per individuare le proprietà e i vincoli che il prodotto deve soddisfare;
	\item \textbf{Lettura}: il Verificatore esamina il prodotto, annotando errori e verificando la conformità ai vincoli;
	\item \textbf{Discussione}: il Verificatore e l'autore discutono degli errori riscontrati e valutano le possibili soluzioni;
	\item \textbf{Correzione}: l'autore apporta le modifiche concordate.
\end{enumerate}
Nel caso di prodotti particolarmente complessi o di grandi dimensioni, il walkthrough può risultare dispendioso in termini di risorse,
per questo motivo è più probabile adottarne l'impiego nelle fasi iniziali del progetto.

\subsubsubsection{Inspection}
Al contrario del walkthrough, l'inspection prevede una conoscenza preventiva degli elementi da verificare, i quali vengono organizzati in
liste di controllo specifiche (\textit{checklist}). Di conseguenza questo approccio risulta più rapido ed efficiente nel contesto di documenti
o codice complessi e strutturati poiché consente di identificare tempestivamente e risolvere potenziali problematiche.

\subsubsubsection{Analisi dinamica}
L'analisi statica è un'attività di controllo che richiede l'esecuzione effettiva del codice con lo scopo di
individuare discordanze tra i risultati ottenuti e il comportamento atteso del software. Il test costituisce
la principale tecnica di analisi dinamica, rappresentato da esecuzioni del codice in un dominio di casi definito in precedenza dal Verificatore.
Quest'ultimo è composto da tutti i possibili casi e dati di input che possono far emergere difetti o eventuali problemi di funzionamento e
garantire la qualità del prodotto finale. Per assicurare l'efficacia di un test è necessario
che esso sia ripetibile e decidibile, ossia che produca risultati coerenti e che possa essere eseguito più volte senza che
i risultati siano influenzati da fattori esterni. Un altro aspetto importante è l'automazione del processo, realizzabile tramite l'uso di
strumenti specifici (driver, stub, logger) che consentono di eseguire i test in modo automatico e di monitorare i risultati ottenuti.

\subsubsection{Testing} \label{testing}
Lo scopo del testing è quello di individuare errori e difetti nella componente soggetta a test,
garantendo che il prodotto soddisfi i requisiti specificati e produca i risultati attesi. Per ogni test
è necessario definire i seguenti aspetti:
\begin{itemize}
	\item \textbf{Ambiente}: il sistema hardware e software all'interno del quale viene eseguito il test;
	\item \textbf{Stato iniziale}: i parametri iniziali del sistema prima dell'esecuzione del test;
	\item \textbf{Input}: i dati di input necessari per l'esecuzione del test;
	\item \textbf{Output}: i risultati attesi in relazione ad un determinato input;
	\item \textbf{Commenti}: eventuali note aggiuntive.
\end{itemize}

\subsubsubsection{Test di unità}
I test di unità sono finalizzati alla verifica di componenti software
atomiche, ossia singole unità di codice, come classi, metodi o funzioni. Essi sono implementati principalmente durante la progettazione e
devono essere eseguiti per primi, in quanto verificano il corretto funzionamento prima dell'integrazione con altre unità. Questi
test sono eseguiti in modo isolato e indipendente dal resto del sistema,
al seguente scopo è consentito l'utilizzo di mock e stub per simulare il comportamento di componenti non ancora sviluppate. In base al tipo
di controllo che si vuole effettuare possiamo distinguere due connotazioni differenti:
\begin{itemize}
	\item \textbf{Test funzionali}: verificano che l'unità testata produca i risultati attesi in base ai dati di input;
	\item \textbf{Test strutturali}: verificano la copertura di tutti i possibili cammini di esecuzione del codice.
\end{itemize}

\subsubsubsection{Test di integrazione}
Questi test vengono pianificati durante la fase di progettazione architetturale, successivamente ai
test di unità. Essi verificano la corretta integrazione tra le diverse unità software
precedentemente testate per garantire che lavorino sinergicamente secondo le specifiche del progetto. Inoltre, è possibile annullare le modifiche
apportate in modo da ripristinare uno stato sicuro nel caso si verifichino errori durante l'esecuzione di questo processo. Distinguiamo
due approcci di integrazione:
\begin{itemize}
	\item \textbf{Top-down:} l'integrazione avviene partendo dalle componenti di sistema che hanno più dipendenze e maggiore rilevanza esterna, garantendo la disponibilità immediata delle funzionalità di alto livello. Questo approccio prevedere l'utilizzo di molti oggetti simulati;
	\item \textbf{Bottom-up:} l'integrazione avviene partendo dalle componenti di sistema che hanno meno dipendenze e maggiore valore interno, ovvero quelle meno visibili all'utente. Questo comporta una fase di test più tardiva delle funzionalità utente.
\end{itemize}

\subsubsubsection{Test di sistema}
I test di sistema sono finalizzati alla verifica del corretto funzionamento dell'intero sistema.
Ovvero si assicurano che tutte le componenti siano integrate correttamente e che tutti i requisiti software
siano presenti e funzionanti. Al termine di questa fase l'applicazione esegue le funzioni previste in modo accurato e affidabile.
I test di sistema sono pianificati successivamente ai test di integrazione.

\subsubsubsection{Test di regressione}
I test di regressione devono essere eseguiti ogni qualvolta vengono apportate delle modifiche al codice.
Essi, infatti, hanno l'obiettivo di garantire che queste modifiche non abbiano introdotto nuovi difetti o compromettano le funzionalità
precedentemente testate, evitando così il verificarsi di regressioni. Questi controlli prevedono la ripetizione mirata di
test di unità, d'integrazione e di sistema preservando la stabilità del sistema.

\subsubsubsection{Test di accettazione}
I test di accettazione rappresentano un processo fondamentale prima del rilascio del prodotto finale.
Essi verificano che tutti le aspettative degli utenti e i requisiti richiesti dal committente siano pienamente soddisfatti. Per questo motivo
devono essere svolti necessariamente in presenza del committente.

\subsubsubsection{Sequenza delle fasi di test}
La sequenza ordinata delle fasi di test è la seguente:
\begin{enumerate}
	\item Test di unità;
	\item Test di integrazione;
	\item Test di regressione;
	\item Test di sistema;
	\item Test di accettazione.
\end{enumerate}

\subsubsubsection{Codici dei test}
Ciascun test deve essere identificato da un codice univoco nel seguente formato:
\begin{center}
	\textbf{[tipo]\_[codice]}
\end{center}
Dove:
\begin{itemize}
	\item \textbf{Tipo}: indica la tipologia di appartenenza del test, può assumere i seguenti valori:
	      \begin{itemize}
		      \item \textbf{UT}: test di unità (Unit Test);
		      \item \textbf{IT}: test di integrazione (Integration Test);
		      \item \textbf{RT}: test di regressione (Regression Test);
		      \item \textbf{ST}: test di sistema (System Test);
		      \item \textbf{AT}: test di accettazione (Acceptance test).
	      \end{itemize}
	\item \textbf{Codice}: rappresenta un numero associato al test, univoco all'interno della tipologia:
	      \begin{itemize}
		      \item nel caso in cui il test non abbia un padre, esso è un semplice numero progressivo;
		      \item nel caso in cui il test abbia un padre, il codice è nel formato:
		            \begin{center}
			            \textbf{[codicePadre].[codiceFiglio]}
		            \end{center}
	      \end{itemize}
\end{itemize}

\subsubsubsection{Stato dei test}
A ciascun test è associato uno stato che indica l'esito della sua esecuzione:
\begin{itemize}
	\item \textbf{S}: il test è stato superato;
	\item \textbf{NS}: il test non è stato superato;
	\item \textbf{NI}: il test non è stato implementato.
\end{itemize}
Questi risultati saranno riportati nel documento \textit{"Piano di Qualifica"}, in particolare nella sezione \textit{"Specifica dei test"}.

\subsection{Validazione}
\subsubsection{Introduzione}
La validazione è un processo che si occupa di verificare che il prodotto software sia in linea con i requisiti
e con le aspettative del cliente. Di conseguenza, è fondamentale l'interazione diretta con il committente e il
proponente, al fine di ottenere un feedback immediato e garantire un chiaro allineamento tra ciò che è stato
prodotto e le aspettative degli utenti finali. Dunque lo scopo finale è avere un prodotto pronto per il rilascio, determinando la
conclusione del ciclo di vita del software.

\subsubsection{Procedura di validazione}
In questo processo copre un ruolo fondamentale il test di accettazione che mira a garantire la validazione
del prodotto. Infatti i diversi test elencati nella sezione \href{#testing}{\underline{3.2.4}} \textit{(Testing)} costituiscono un input
per la validazione. Essi dovranno verificare:
\begin{itemize}
	\item Il soddisfacimento dei casi d'uso;
	\item La conformità del prodotto ai requisiti obbligatori;
	\item Il soddisfacimento di altri requisiti concordati con il committente.
\end{itemize}

\subsection{Gestione della configurazione}
\subsubsection{Introduzione}
La gestione della configurazione è un processo attuato durante tutto il ciclo di vita del software, infatti viene
applicata a tutte le categorie di "artefatti" coinvolti. Essa si occupa di tracciare e controllare le modifiche della documentazione e del codice
prodotto, detti Configuration Item (CI). Così facendo le modifiche apportate saranno accessibili in qualsiasi momento, garantendo la
possibilità di verificare le motivazioni alla base dei cambiamenti effettuati e anche il ripristino di versioni precedenti.

\subsubsection{Versionamento}
La convenzione di versionamento adottata è nel formato X.Y dove:
\begin{itemize}
	\item \textbf{X}: rappresenta il completamento in vista di una delle fasi del progetto e dunque viene incrementato al raggiungimento di
	      RTB, PB ed eventuale CA.
	\item \textbf{Y}: rappresenta una versione intermedia e viene incrementata ad ogni modifica significativa del documento.
\end{itemize}

\subsubsection{Repository}
Il team utilizza due repository:
\begin{itemize}
	\item \href{https://github.com/7Last/docs.git}{\underline{Documentazione}}: contenente la documentazione prodotta;
	\item \href{https://github.com/7Last/7Last.github.io.git}{\underline{Codice}}: contenente il codice del progetto.
\end{itemize}

\subsubsubsection{Struttura repository}
Il repository dedicato alla documentazione è organizzato nel seguente modo:
\begin{itemize}
	\item \textbf{Candidatura}:
	      \begin{itemize}
		      \item \textbf{Verbali esterni}: contenente i verbali delle riunioni con le proponenti;
		      \item \textbf{Verbali interni}: contenente i verbali delle riunioni svolte all'interno del team;
		      \item \textbf{Lettera di presentazione};
		      \item \textbf{Preventivo costi e assunzione impegni};
		      \item \textbf{Valutazione dei capitolati}.
	      \end{itemize}
	\item \textbf{RTB}:
	      \begin{itemize}
		      \item \textbf{Verbali}: contenente tutti i verbali prodotti durante il periodo di RTB, distinti tra esterni e interni;
		      \item \textbf{Analisi dei Requisiti};
		      \item \textbf{Piano di Progetto};
		      \item \textbf{Piano di Qualifica};
		      \item \textbf{Glossario};
		      \item \textbf{Norme di Progetto}.
	      \end{itemize}
	\item \textbf{PB}:
\end{itemize}

\subsubsection{Sincronizzazione e branching}

\subsubsubsection{Documentazione}
L'approccio adottato per la redazione della documentazione segue il workflow noto come \textit{feature branch}.
Ossia ogni attività è identificata da una specifica issue di GitHub (ClickUp?) e, prima dello svolgimento di
ciascuna di esse, il componente interessato crea una diramazione del branch "develop". Tale metodologia
permette di lavorare in modo isolato e parallelo nei rispettivi "workspace", massimizzando il lavoro ed evitando sovrascritture indesiderate.
\textbf{Convenzioni per la nomenclatura dei branch relativi alle attività di redazione o modifica di documenti} \label{convenzioni_nomenclatura}
\begin{itemize}
	\item Il nome del branch deve riportare il nome del documento che si vuole redarre o modificare;
	\item Nel caso dei verbali, il nome deve presentare anche la data della riunione: \textit{verbale\_interno\_yy\_mm\_dd} (es. verbale\_interno\_24\_03\_05);
	\item Nel caso di redazione o modifica di una singola sezione di un documento, il nome del branch deve avere il formato:
	      \textit{nomeDocumento\_nomeSezione} (es. norme\_di\_progetto\_introduzione);
\end{itemize}

\subsubsubsection{Sviluppo}
Il team utilizza lo stile di flusso di lavoro \href{https://www.atlassian.com/it/git/tutorials/comparing-workflows/gitflow-workflow}{\underline{{Gitflow}}}.
\textbf{Flusso di lavoro Gitflow}
\begin{enumerate}
	\item \textbf{Branch develop}: è il punto di avvio per nuove attività, si crea a partire dal branch "main";
	\item \textbf{Branch release}: gestisce la preparazione del software per un rilascio e ammette solo
	      modifiche minori e correzione di bug, si crea a partire dal branch "develop";
	\item \textbf{Branch feature}: si occupa dello sviluppo di nuove funzionalità;
	\item \textbf{Merge di feature in develop}: una volta completata un'attività, il branch feature viene
	      unito a develop;
	\item \textbf{Merge di release in develop e main}: dopo il completamento del branch release, esso viene
	      unito sia a develop che a main;
	\item \textbf{Branch hotfix}: si occupa della correzione di bug critici riscontrati nella fase di produzione;
	\item \textbf{Merge di hotfix in develop e main}: dopo il completamento del branch hotfix, esso viene
	      unito sia a develop che a main per garantire coerenza tra le versioni.
\end{enumerate}

\subsubsubsection{Pull Request}
Quando un'attività è completata, il componente che l'ha svolta crea una Pull Request per integrare le modifiche
nel ramo principale. Il Verificatore ha il compito di verificare la correttezza del lavoro svolto e approvare la Pull Request.
\textbf{Procedura per la creazione di una Pull Request} \label{pull_request}
\begin{enumerate}
	\item Accedere alla pagina del repository e spostarsi sulla sezione "Pull Request";
	\item Cliccare su "New Pull Request";
	\item Selezionare il branch di partenza e quello di destinazione;
	\item Cliccare su "Create Pull Request";
	\item Assegnare un titolo alla Pull Request;
	\item Aggiungere una descrizione delle modifiche apportate;
	\item Assegnare un Verificatore;
	\item Per convenzione, l'assegnatario è colui che richiede la Pull Request;
	\item Selezionare le label;
	\item Selezionare il progetto;
	\item Selezionare la milestone;
	\item Cliccare su "Create Pull Request";
	\item Nel caso di conflitti seguire la procedura per la risoluzione proposta da GitHub.
\end{enumerate}

\subsubsection{Controllo di configurazione}
\subsubsubsection{Change Request}
Per lo svolgimento di questo processo in modo ordinato e strutturato seguiamo lo standard
\href{https://www.math.unipd.it/~tullio/IS-1/2009/Approfondimenti/ISO_12207-1995.pdf}{\underline{ISO/IEC 12207:1995}}
che prevede le seguenti attività:
\begin{enumerate}
	\item \textbf{Identificazione e registrazione} Ciascuna change request è identificata, registrata e documentata. L'identificazione avviene
	      tramite la creazione di un'issue con allegata l'etichetta "Change request".
	      Inoltre vengono riportate informazioni come la natura della modifica richiesta, la priorità e l'impatto sul progetto.
	\item \textbf{Valutazione e analisi} La change request viene valutata e analizzata per determinare la fattibilità e l'impatto sul
	      sistema. Vengono analizzati costi e benefici conseguenti alla modifica.
	\item \textbf{Approvazione o rifiuto} La change request viene approvata o respinta in base a criteri come budget, tempo e priorità.
	\item \textbf{Pianificazione delle modifiche} Se la change request viene approvata, viene pianificata la sua implementazione e l'integrazione
	      nel ciclo di sviluppo del software.
	\item \textbf{Implementazione} Le modifiche vengono effettivamente implementate, è fondamentale tenere traccia di ciò che viene fatto così
	      da consentire una corretta documentazione e, se necessario, il rollback allo stato precedente.
	\item \textbf{Verifica e validazione} Le modifiche vengono verificate e validate per garantire che non abbiano introdotto nuovi difetti
	      e che siano conformi agli obiettivi prefissati.
	\item \textbf{Documentazione} Tutte le fasi del progetto descritto vengono documentate in modo accurato per garantire trasparenza e
	      tracciabilità.
	\item \textbf{Comunicazione agli stakeholder} Il confronto con gli stakeholder deve essere mantenuto
	      durante tutto l'arco del processo, in modo da mantenere trasparenza e fiducia.
\end{enumerate}

\subsubsection{Contabilità dello Stato di Configurazione}
La contabilità dello stato di configurazione (o Configuration Status Accounting) è un'attività
che si occupa di tenere traccia e monitorare tutte le configurazioni di un sistema software
durante tutto il ciclo di vita. Nello specifico esso mantiene la trasparenza e la tracciabilità delle modifiche relative ai Configuration Item (CI),
mantenendo un registro accurato di tutte le attività.
\textbf{Registrazione delle configurazioni}: registrazioni di informazioni dettagliate su ogni CI:
\begin{itemize}
	\item \textbf{Documentazione}: nella prima pagina di ciascuno di essi troviamo le informazioni riguardanti la configurazione;
	\item \textbf{Sviluppo}: le informazioni relative alla configurazione sono inserite come prime
	      righe di ciascun file sotto forma di commento.
\end{itemize}
\textbf{Stato e cambiamenti}: tracciamento dello stato attuale di ogni CI e delle modifiche effettuate, ossia versione attuale, revisioni e baseline.
\begin{itemize}
	\item \textbf{Registro delle modifiche}: per tenere traccia dello stato di ciascun elemento di configurazione si utilizza il registro
	      delle modifiche integrato in ognuno di essi.
	\item \textbf{Branching e DashBoard}: dal momento che ciascuna issue è associata ad un CI tramite
	      label, è possibile verificare facilmente se ci sono attività correlate
	      in corso nella colonna "In progress" della Dashboard di progetto.
\end{itemize}
\textbf{Supporto per la gestione delle change request}: registrazione e documentazione di tutte le modifiche effettuate ai CI in risposta alle
richieste di modifica. Per la gestione di queste richieste si utilizza l'Issue Tracking System (ITS) di GitHub (ClickUp?) creando una
issue con la label "Change request".

\subsubsection{Release management and delivery}
Secondo lo standard \href{https://www.math.unipd.it/~tullio/IS-1/2009/Approfondimenti/ISO_12207-1995.pdf}{\underline{ISO/IEC 12207:1995}},
questo processo si occupa di tutto ciò che riguarda il rilascio e la distribuzione del software. Più precisamente,
la Release Management and Delivery gestisce la release, la distribuzione e la documentazione correlata al prodotto software
pronto per l'uso operativo.

Nel nostro caso, la pianificazione della release avviene in concomitanza con le baseline stabilite per il progetto didattico, ossia
RTB (Requirements and Technology Baseline),
PB (Product Baseline) ed eventualmente CA (Customer Acceptance).
Affinché questo processo termini con successo, risulta fondamentale che esso sia preceduto da un'attenta fase di verifica e validazione del prodotto.

\subsubsubsection{Procedura per la creazione di una release}
\begin{enumerate}
	\item Accedere alla pagina del repository;
	\item Spostarsi sulla sezione "Releases";
	\item Cliccare su "Create a new release";
	\item Cliccare su "Choose a tag" e selezionare il tag che identifica la versione da rilasciare. Nel caso non fosse presente, è possibile
	      crearlo cliccando su "Create a new tag";
	\item Selezionare come branch di destinazione il branch main;
	\item Scrivere una descrizione della release;
	\item Cliccare su "Publish release".
\end{enumerate}
Conclusa la procedura, sarà possibile visionare la release nella sezione "Releases" del repository.

\subsection{Joint review}
\subsubsection{Introduzione}
La Joint Review, o revisione congiunta, è un processo che si occupa di valutare lo stato e i risultati di un'attività
all'interno del progetto, prendendo in considerazione sia gli aspetti tecnici che gestionali. Essa coinvolge due parti distinte che hanno entrambe
la facoltà di attivare il processo: la parte recensita e la parte recensente. Nel nostro caso i recensori sono costituiti da
proponente, committente e stakeholder;
mentre i recensiti da noi fornitori.

\subsubsection{Implementazione del processo}
L'implementazione del processo comprende i seguenti impegni:
\subsubsubsection{Revisioni periodiche}
In corrispondenza delle milestone stabilite saranno effettuate delle revisioni periodiche, come riportato nel
documento \textit{Piano di Progetto}.

\subsubsubsection{Stato Avanzamento Lavori}
Al termine di ogni sprint, quindi generalmente ogni due settimane, si svolge una revisione SAL
(Stato Avanzamento Lavori) tra il team e il proponente. Questa revisione ha lo scopo di valutare il lavoro svolto
nell'arco di tempo precedente, per verificare che gli obiettivi prefissati siano stati correttamente raggiunti secondo le scadenze prefissate.
Inoltre, durante questo incontro, si pianificano anche le successive task da portare a termine.

\subsubsubsection{Revisioni ad hoc}
Nel caso in cui una delle parti in gioco lo ritenga necessario, è possibile attivare una revisione ad hoc per valutare attentamente lo stato
di avanzamento del progetto, discutendo di eventuali problematiche e relative soluzioni adottabili.

\subsubsubsection{Risorse per le revisioni}
Le risorse necessarie per svolgere le revisioni sono di svariata natura e possono essere: personale, strumenti, hardware, software, strutture, ecc.
In ogni caso, è importante che queste risorse siano discusse e concordate tra le parti.

\subsubsubsection{Elementi da concordare}
In ciascuna revisione è necessario concordare i seguenti elementi:
\begin{itemize}
	\item Agenda della riunione;
	\item Prodotti software risultati dall'attività e relative problematiche;
	\item Procedure;
	\item Criteri di ingresso e uscita per la revisione.
\end{itemize}

\subsubsubsection{Documenti e distribuzione dei risultati}
I risultati della revisione devono essere riportati e documentati accuratamente nei \textit{Verbali Esterni}. In seguito, la parte recensente
comunicherà alla parte recensita la veridicità di quanto riportato, approvando o disapprovando i risultati citati.

\subsubsection{Project management reviews}
\subsubsubsection{Introduzione}
Secondo lo standard \href{https://www.math.unipd.it/~tullio/IS-1/2009/Approfondimenti/ISO_12207-1995.pdf}{\underline{ISO/IEC 12207:1995}},
il processo di Project Management Reviews si occupa di revisionare e valutare un progetto software, sia esso
in corso o completato. In particolare esso si preoccupa che il progetto venga svolto in modo efficiente e conforme agli obiettivi e ai requisiti
prefissati. Questa attività di revisione viene eseguita periodicamente durante tutto il ciclo di vita del progetto e
coinvolge i fornitori e gli stakeholder.

\subsubsubsection{Stato del progetto}
La revisione valuta lo stato del progetto in relazione al budget, alle tempistiche e agli standard di qualità. I risultati discussi con gli
stakeholder sono:
\begin{itemize}
	\item Le attività progrediscono secondo i piani;
	\item Il progetto è gestito efficientemente attraverso l'allocazione adeguata delle risorse;
	\item Se necessario, viene valutata la possibilità di una pianificazione alternativa;
	\item Eventuali rischi sono identificati e gestiti in modo appropriato.
\end{itemize}

\subsubsection{Revisioni tecniche}
Le revisioni tecniche valutano prodotti e servizi software e forniscono un feedback dei seguenti aspetti:
\begin{itemize}
	\item Completezza;
	\item Conformità agli standard e alle specifiche;
	\item Correttezza nell'implementazione delle modifiche richieste nelle change request;
	\item Coerenza con le linee guida nello sviluppo e nella manutenzione del progetto.
\end{itemize}

\subsection{Risoluzione dei problemi}
\subsubsection{Introduzione}
La risoluzione dei problemi è un processo che si occupa di identificare e risolvere le problematiche che possono emergere durante tutto il ciclo
di vita del prodotto. Nello specifico, lo scopo è quello di fornire un approccio e un mezzo tempestivi ed affidabili per garantire che tutti
i problemi individuati siano analizzati e risolti in modo efficace. All'interno delle problematiche riconosciute includiamo anche i casi di non
conformità. Inoltre, la risoluzione dei problemi si propone anche di studiare le cause alla radice dei problemi, aiutando lo sviluppo di misure
di prevenzione per evitare che si ripresentino in futuro. Dunque  questo processo ricopre un ruolo fondamentale nell'ambito del miglioramento
continuo, dove la comprensione degli errori passati permette la crescita e l'ottimizzazione dei processi. Per farlo è importante seguire
approcci strutturati e metodologie efficaci, che prevedono raccolta dati, analisi profonda delle cause, valutazione dell'impatto e implementazione
di azioni correttive e preventive. In aggiunta, è essenziale anche mantenere una documentazione accurata di tutto il processo, così da
permettere trasparenza e tracciabilità.

\subsubsection{Gestione dei rischi}
Nella sezione \textit{Analisi dei rischi} del documento \textit{Piano di Progetto} sono stati identificati dal Responsabile tutti i rischi
che potrebbero insorgere durante lo svolgimento del progetto, con relative probabilità di occorrenza e misure di mitigazione. Inoltre,
viene condotta un'analisi del loro impatto e una valutazione dell'esito delle misure adottate. Nel caso di esito negativo, sono richieste
delle modifiche per correggere quella determinata mitigazione.

\subsubsubsection{Codifica dei rischi}
Ogni rischio è identificato da un codice univoco nel seguente formato:
\begin{center}
	\textbf{[tipo]\_[probabilità]\_[priorità]\_[indice]}: nomeRischio
\end{center}
Dove:
\begin{itemize}
	\item \textbf{Tipo}: indica la tipologia del rischio:
	      \begin{itemize}
		      \item \textbf{TR}: rischio relativo all'utilizzo delle tecnologie (Technology Risk);
		      \item \textbf{OR}: rischio relativo all'organizzazione del team (Organization Risk);
		      \item \textbf{PR}: rischio relativo agli impegni personali (Personal Risk).
	      \end{itemize}
	\item \textbf{Probabilità}: indica la probabilità di occorrenza del rischio:
	      \begin{itemize}
		      \item \textbf{1}: bassa;
		      \item \textbf{2}: media;
		      \item \textbf{3}: alta.
	      \end{itemize}
	\item \textbf{Priorità}: indica la pericolosità del rischio:
	      \begin{itemize}
		      \item \textbf{B}: bassa;
		      \item \textbf{M}: media;
		      \item \textbf{A}: alta.
	      \end{itemize}
	\item \textbf{Indice}: rappresenta un numero associato al rischio, univoco all'interno della tipologia.
\end{itemize}

\subsubsubsection{Metriche}
\begin{table}[h]
	\caption{Metriche relative alla gestione dei processi}
	\centering
	\begin{tabular}{|c|c|}
		\hline
		\textbf{Metrica} & \textbf{Nome}             \\
		\hline
		16M-NCR          & Non Calculated Risk (NCR) \\
		\hline
	\end{tabular}
\end{table}

\subsubsection{Identificazione dei problemi}
Nel momento in cui insorge un problema, è essenziale comunicarlo immediatamente al resto del team e aprire una segnalazione nel sistema di
issue tracking con label "bug" e opportuna descrizione.

\subsection{Gestione della qualità}
\subsubsection{Introduzione}
Il processo di gestione della qualità si occupa di mantenere la qualità nel flusso operativo del fornitore
durante tutto il ciclo di vita del software. Adottiamo infatti un approccio
olistico che si estende dal concepimento all'implementazione e alla manutenzione del prodotto finale.
Nello specifico, la gestione della qualità mira a garantire che le aspettative del proponente e
degli utenti finali siano pienamente soddisfatte, nonché di rispettare gli standard di qualità prefissati. Come vedremo più nel dettaglio,
questo include:
\begin{itemize}
	\item Definizione degli obiettivi di qualità;
	\item Identificazione delle metriche e dei criteri di qualità;
	\item Pianificazione ed esecuzione delle attività di controllo qualità, attraverso revisioni, ispezioni e test.
\end{itemize}

\subsubsection{Attività}
Le attività di gestione della qualità sono le seguenti:
\begin{enumerate}
	\item \textbf{Definizione degli standard di qualità}: identificazione chiara degli standard di qualità
	      (inclusi eventualmente requisiti funzionali) che il prodotto software deve soddisfare;
	\item \textbf{Pianificazione della qualità}: progettazione di un piano di qualità che definisce le attività,
	      risorse e tempistiche;
	\item \textbf{Assicurazione della qualità}: monitoraggio e valutazione dei processi;
	\item \textbf{Controllo della qualità}: esecuzione di test e verifiche;
	\item \textbf{Gestione delle modifiche}: implementazione di un sistema di gestione modifiche per garantire che ciascuna di esse sia valutata
	      in base all'impatto che ha sulla qualità del prodotto;
	\item \textbf{Miglioramento continuo e correzione}: raccolta di feedback, analisi dell'andamento, e adozione delle
	      best practice per massimizzare la qualità dei processi e dei profitti;
	\item \textbf{Coinvolgimento degli stakeholder}: comunicazione tramite feedback con gli stakeholder
	      per garantire che stakeholder, clienti ed utenti finali siano coinvolti nel processo di gestione della qualità;
	\item \textbf{Formazione del team}: formazione e supporto costante all'interno del gruppo per mantenere un alto livello di qualità.
\end{enumerate}

\subsubsection{Piano di qualifica}
Il documento \textit{Piano di Qualifica} ricopre un ruolo essenziale nel contesto della gestione della qualità in quanto tratta ampiamente
le attività di pianificazione e controllo della qualità. In particolare esso definisce le specifiche di qualità
richieste, definendo i metodi di controllo necessari a garantire il rispetto di tali requisiti.

\subsubsection{PDCA}
Il ciclo PDCA, o ciclo di Deming, è un modello di gestione iterativo che consente il miglioramento continuo dei processi. Esso consta delle seguenti fasi:
\begin{itemize}
	\item \textbf{Plan}: pianificazione delle attività per definire processi e relativa sequenza di esecuzione;
	\item \textbf{Do}: esecuzione effettiva di quanto pianificato;
	\item \textbf{Check}: analisi e verifica di tutte le informazioni raccolte durante la fase di esecuzione;
	\item \textbf{Act}: implementazione di azioni correttive nel caso del mancato raggiungimento dei risultati attesi, oppure riconoscimento e
	      consolidamento delle buone pratiche. Così facendo, il ciclo viene di volta in volta incrementato e perfezionato aggiungendo valore al processo.
\end{itemize}

\subsubsection{Struttura e identificazioni metriche}
Ogni metrica presenta la seguente struttura:
\begin{itemize}
	\item \textbf{Metrica}:
	      codice identificativo nel formato:
	      \begin{center}
		      \textbf{[numero]M-[acronimo]}
	      \end{center}
	      Dove:
	      \begin{itemize}
		      \item \textbf{Numero}: numero progressivo univoco per ogni metrica;
		      \item \textbf{M}: metrica;
		      \item \textbf{Acronimo}: abbreviazione della metrica.
	      \end{itemize}
	\item \textbf{Nome}: nome della metrica;
	\item \textbf{Descrizione}: breve descrizione della metrica adottata e delle sue funzionalità;
	\item \textbf{Formula}: formula matematica per il calcolo della metrica;
\end{itemize}

\subsubsection{Criteri di accettazione}
All'interno del documento \textit{Piano di Qualifica} sono definiti i criteri di accettazione in formato tabellare:
\begin{itemize}
	\item \textbf{Valore accettabile}: valore minimo affinché la metrica sia considerabile soddisfacente e conforme agli obiettivi di qualità;
	\item \textbf{Valore desiderabile}: valore ottimale e ideale che dovrebbe essere raggiunto dalla metrica.
\end{itemize}

\subsubsection{Metriche}
\begin{table}[h]
	\centering
	\caption{Metriche relative alla gestione della qualità}
	\begin{tabular}{|c|c|}
		\hline
		\textbf{Metrica} & \textbf{Nome}                   \\
		\hline
		15M-QMS          & Quality Metrics Satisfied (QMS) \\
		24M-FD           & Failure Density (FD)            \\
		\hline
	\end{tabular}
\end{table}
