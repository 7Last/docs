\section{Svantaggi di Redpanda/Vantaggi di Apache Kafka}
\subsection{Maturità}
Redpanda è stato rilasciato per la prima volta nel 2019, mentre Apache Kafka è stato rilasciato
per la prima volta nel 2011. Questo significa che Kafka ha avuto più tempo per svilupparsi,
maturare e stabilizzarsi, e ha un livello di maturità e stabilità che Redpanda non
ha ancora raggiunto.\\
Per lo stesso motivo, Kafka ha una maggiore adozione e diffusione.

\subsection{Comunità e supporto}
Kafka ha una vasta e attiva comunità di sviluppatori, che forniscono supporto, risorse e strumenti per estendere e migliorare il progetto.
Kafka ha una documentazione molto completa e ben strutturata, e sono forniti molti tutorial, guide e risorse online per imparare ad utilizzarlo.\\
Redpanda, al contrario, ha una comunità più piccola e meno attiva, e quindi meno risorse e strumenti disponibili.\\

\subsection{Integrazione con altri servizi}
Kafka è supportato da una vasta gamma di strumenti e librerie di terze parti che lo integrano con altri sistemi e servizi
(con cui Redpanda dovrebbe essere compatibile, ma non è garantito).

\subsection{Protocollo di replicazione}
Il protocollo Raft utilizzato da Redpanda per la replicazione e la scrittura su disco è sincrono, il che significa che i dati devono essere scritti (fsynced) su disco in modo sincrono.
I dati devono essere scritti (\textit{fsynced}) su disco in modo sincrono, altrimenti è possibile perdere dati durante l'elezione di un nuovo leader.


