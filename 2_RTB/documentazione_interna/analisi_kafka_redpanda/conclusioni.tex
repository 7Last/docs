\section{Conclusioni}
Kafka e Redpanda sono due strumenti molto simili, ma rispondono ad esigenze differenti.
Nel caso si debba gestire un progetto in ambiente di produzione, Kafka è la scelta ottimale, in quanto è più stabile, testato e affidabile.
Redpanda invece si presta meglio per progetti più semplici e con carichi di dati minori. Inoltre, risulta maggiormente adatto a utenti meno esperti, in quanto richiede meno configurazioni. \\
In secondo luogo, un altro aspetto da considerare è la licenza: Kafka è \textit{open source}, mentre Redpanda è un prodotto commerciale; nel caso di
budget limitato, Kafka risulta dunque più conveniente.\\
Nelle valutazioni per la scelta dello strumento più adatto, è importante tenere conto che i \textit{benchmark} sono stati fatti dai creatori dei \textit{software}
perciò potrebbero essere stati effettuati in maniera da favorirli.\\
Ai fini della realizzazione del \textit{Proof of Concept} non sono richieste prestazioni elevate, perciò pensiamo che sia sufficiente utilizzare Redpanda.
Essendo questo il primo approccio a questo tipo di tecnologia per alcuni membri del gruppo, Redpanda permetterebbe a tutti i componenti di apprenderne
il funzionamento in modo più semplice e veloce.\\
Data la compatibilità tra le due tecnologie, in un secondo momento si potrebbe facilmente passare a Kafka, senza dover riscrivere il codice.\\
Infine, nel caso in cui il progetto dovesse evolvere oltre il \textit{Minimum Viable Product}, riterremmo più opportuno passare a Kafka.



