\section{Tabella comparativa}
\begin{longtable}{|p{4cm}||p{5.5cm}|p{5.5cm}|}
	\hline
	\textbf{Paragone}                    & \href{https://7last.github.io/docs/rtb/documentazione-interna/glossario#apache-kafka}{\textbf{Apache Kafka}\textsubscript{G}}                                                         & \textbf{Redpanda}                                                                                                                                                                \\
	\hline
	\endfirsthead
	\hline
	\textbf{Paragone}                    & \href{https://7last.github.io/docs/rtb/documentazione-interna/glossario#apache-kafka}{\textbf{Apache Kafka}\textsubscript{G}}                                                         & \textbf{Redpanda}                                                                                                                                                                \\
	\hline
	\textbf{Stelle su Github}            & 27.5k                                                                         & 8.8k                                                                                                                                                                             \\
	\hline
	\textbf{Fork su Github}              & 13.5k                                                                         & 539                                                                                                                                                                              \\
	\hline
	\textbf{Adozione}                    & Utilizzato da migliaia di compagnie (tra cui LinkedIn, Airbnb, e Netflix)     & Non chiaro quante organizzazioni lo usino. Adottato da Cisco e Vodafone.                                                                                                         \\
	\hline
	\textbf{\textit{Community}}          & Migliaia di contributori                                                      & \textit{Community} più piccola                                        ed emergente.                                                                                              \\
	\hline
	\textbf{Maturità}                    & Stabile, sviluppato dal 2011                                                  & Emergente, lanciato nel 2019.                                                                                                                                                    \\
	\hline
	\textbf{Documentazione, risorse}     & Documentazione dettagliata, forum, tutorial, e corsi online                   & Documentazione dettagliata, ma non altrettante risorse. Tutorial creati dal team di Redpanda.                                                                                    \\
	\hline
	\textbf{\textit{Client}}             & Ampia varietà di \textit{client} per i principali linguaggi di programmazione & Lista di \href{https://docs.redpanda.com/current/develop/kafka-clients/}{client ufficialmente testati}, ma secondo la documentazione qualsiasi client Kafka dovrebbe funzionare. \\
	\hline
	\textbf{CLIs}                        & Include un set di strumenti per gestire i topic, messaggi, cluster...         & Include \texttt{rpk} , un'interfaccia per gestire topic, messaggi, debugging, interazione con Redpanda Cloud.                                                                    \\
	\hline
	\textbf{Monitoraggio}                & Richiede configurazioni di sistemi di monitoraggio (JMX, Grafana, Prometheus) & Integrato direttamente con Prometheus e Grafana.                                                                                                                                 \\
	\hline
	\textbf{Facilità di utilizzo}        & Complesso da configurare e gestire                                            & Facile da installare e configurare, indipendente da Zookeeper                                                                                                                    \\
	\hline
	\textbf{Licenza}                     & Open source, Apache 2.0                                                       & Edizioni \textit{Community} e \textit{Enterprise}, BSL (Business Source License).                                                                                                \\
	\hline
	\textbf{\textit{Deploy self-hosted}} & \textit{Bare-metal}, macchine virtuali, \textit{cloud}, \href{https://7last.github.io/docs/rtb/documentazione-interna/glossario#docker}{Docker\textsubscript{G}}, Kubernetes    & \textit{Bare-metal}, macchine virtuali, \textit{cloud}, \href{https://7last.github.io/docs/rtb/documentazione-interna/glossario#docker}{Docker\textsubscript{G}}, Kubernetes                                                                                                       \\
	\hline
	\textbf{\textit{Managed deploy}}     & Numerosi servizi di terze parti, come Confluent Cloud, AWS MSK...             & Offre 3 opzioni: \textit{cluster} dedicati gestiti da Redpanda, BYOC (\textit{Bring Your Own Cloud}), \textit{cluster serverless} su architettura gestita da Redpanda.           \\
	\hline
	\caption{Tabella comparativa tra \href{https://7last.github.io/docs/rtb/documentazione-interna/glossario#apache-kafka}{Apache Kafka\textsubscript{G}} e Redpanda}
	\label{table:1}
\end{longtable}


