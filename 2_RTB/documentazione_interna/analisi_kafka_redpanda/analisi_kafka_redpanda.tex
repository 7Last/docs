\documentclass[italian,12pt]{article}

%--------------variabili------------------%
\def\Title{Norme di Progetto}
\def\Author{7Last}
\def\Version{v0.2}
%-----------------------------------------%


\usepackage[left=2cm, right=2cm, bottom=3cm, top=3cm]{geometry}
\usepackage{fancyhdr}
\usepackage{graphicx}
\graphicspath{ {../../logo/} }
\usepackage{href-ul}
\usepackage{tikz}
\usepackage{tgadventor}
\usepackage[useregional=numeric,showseconds=true,showzone=false]{datetime2}
\usepackage{caption}
\usepackage{longtable}
\usepackage{xcolor}




\linespread{1.2}
\captionsetup[table]{name=Tabella}
\geometry{headsep=1.5cm}

\renewcommand{\contentsname}{Indice}
\renewcommand\familydefault{\sfdefault}
\renewcommand{\listtablename}{Indice delle tabelle}
\renewcommand\familydefault{\sfdefault}
\renewcommand{\listfigurename}{Indice delle immagini}
\renewcommand\familydefault{\sfdefault}

\begin{document}
\newgeometry{left=2cm,right=2cm,bottom=2.1cm,top=2.1cm}
\begin{titlepage}
	\vspace*{.5cm}

	\vspace{2cm}
	{
		\centering
		{\bfseries\huge \Title\par}
		\bigbreak
		{\bfseries\Large \Subtitle\par}
		\bigbreak
		{\bfseries\large \Author\par}
		\bigbreak
		{\Date\;-\;\Version\par}
		\vfill

		\begin{center}
			\begin{tikzpicture}
				\clip (0,0) circle (2cm) node {\includegraphics[width=4cm]{logo.jpg}};
			\end{tikzpicture}
		\end{center}
	}

	\vfill

\end{titlepage}

\restoregeometry





















\newpage
\pagestyle{fancy}
\fancyhead{}
\lhead{
	\begin{tikzpicture}
		\clip (0,0) circle (0.5cm);
		\node at (0,0) {\includegraphics[width=1cm]{./../logo/logo.png}};
	\end{tikzpicture}%
}
\chead{\vspace{\fill}\Title\vspace{\fill}}
\rhead{\vspace{\fill}\Version\vspace{\fill}}


\begin{table}[!h]
	\caption*{Versioni}
	\begin{center}
		\begin{tabular}{ c c c c p{6.1cm} }
			\hline                                                                                                    \\[-2ex]
			Ver. & Data       & Redattore     & Verificatore      & Descrizione                                       \\
			\\[-2ex] \hline \\[-1.5ex]
			0.2  & 2024-04-22 & Matteo Tiozzo & Antonio Benetazzo & Limitazioni di Apache Kafka, Vantaggi di Redpanda \\
			0.1  & 2024-04-22 & Elena Ferro   & Antonio Benetazzo & Confronto Redpanda vs Apache Kafka                \\
			\\[-1.5ex] \hline
		\end{tabular}
	\end{center}
\end{table}

\newpage
\tableofcontents
\listoftables
\listoffigures
\newpage

\section{Introduzione}
\setcounter{subsection}{0}
\subsection{Scopo del documento}
Il seguente documento si propone di definire la pianificazione e la gestione delle attività richieste per ultimare il progetto. Vengono esaminati in dettaglio elementi cruciali come l’\textit{Analisi dei Rischi}, il \textit{modello di sviluppo adottato}, la \textit{pianificazione delle attività}, la \textit{suddivisione dei ruoli}, oltre a \textit{stime dei costi} e delle \textit{risorse necessarie}.

\subsection{Scopo del prodotto}
Lo scopo principale del prodotto è quello di consentire a \textit{Sync Lab S.r.l.} di valutare la \\\textbf{fattibilità} di investire tempo e risorse nell'implementazione del progetto  \href{https://7last.github.io/docs/rtb/documentazione-interna/glossario\#synccity}{\textit{\textbf{SyncCity} \textsubscript{G}}- A \href{https://7last.github.io/docs/rtb/documentazione-interna/glossario\#smart-city}{smart city\textsubscript{G}} monitoring platform}. Questa soluzione, attraverso l'utilizzo di dispositivi IoT, consente un monitoraggio costante delle città. \href{https://7last.github.io/docs/rtb/documentazione-interna/glossario\#synccity}{SyncCity\textsubscript{G}} avrà lo scopo di monitorare e raccogliere dati da sensori posizionati nelle città, per poi analizzarli e fornire informazioni utili alla gestione della città. Il prodotto finale sarà un prototipo funzionale che consentirà la visualizzazione dei dati raccolti su un cruscotto.

\subsection{Glossario}
Per evitare qualsiasi ambiguità o malinteso sui termini utilizzati nel documento, verrà adottato un \href{https://7last.github.io/docs/rtb/documentazione-interna/glossario\#glossario}{glossario\textsubscript{G}}. Questo \href{https://7last.github.io/docs/rtb/documentazione-interna/glossario\#glossario}{glossario\textsubscript{G}} conterrà varie definizioni. Ogni termine incluso nel \href{https://7last.github.io/docs/rtb/documentazione-interna/glossario\#glossario}{glossario\textsubscript{G}} sarà indicato applicando uno stile specifico:
\begin{itemize}
    \item aggiungendo una "G" al pedice della parola;
    \item fornendo il link al \href{https://7last.github.io/docs/rtb/documentazione-interna/glossario\#glossario}{glossario\textsubscript{G}} online;
\end{itemize}

\subsection{Riferimenti}
    \subsubsection{Normativi}DA SISTEMARE
        \begin{itemize}
            \item \textbf{ISO/IEC 12207:2008} - Systems and software engineering - Software life cycle processes
            \item \textbf{ISO/IEC 31000:2009} - Risk management - Principles and guidelines
        \end{itemize}
    \subsubsection{Informativi}
        \begin{itemize}
            \item \href{https://7last.github.io/docs/rtb/documentazione-interna/glossario\#capitolato}{\textbf{Capitolato \textsubscript{G}}C6 - Sync City}: \textit{A \href{https://7last.github.io/docs/rtb/documentazione-interna/glossario\#smart-city}{smart city\textsubscript{G}} monitoring platform}
            \item \textbf{T2 - Processi di ciclo di vita del software}\\ https://www.math.unipd.it/~tullio/IS-1/2023/Dispense/T2.pdf;
            \item \textbf{T4 - Gestione di progetto}\\ Visibili a questo \uline{\href{https://www.math.unipd.it/~tullio/IS-1/2023/Dispense/T4.pdf}{link}};
            \item \href{https://7last.github.io/docs/rtb/documentazione-interna/glossario\#glossario}{\textbf{Glossario}\textsubscript{G}}\\ Visibile a questo \uline{\href{https://7last.github.io/docs/rtb/documentazione-interna/glossario}{link}};
        \end{itemize}
\subsection{Preventivo iniziale}
Il preventivo iniziale presentato durante la fase di candidatura è disponibile al seguente \uline{\href{https://github.com/7Last/docs/blob/main/1_candidatura/preventivo_costi_assunzione_impegni_v2.0.pdf}{riferimento}}. All'interno di questo documento viene calcolato il preventivo iniziale del progetto, pari a €12.670,00. Inoltre, si specifica che il gruppo \textit{7Last} stima di \textbf{completare} il prodotto entro e non oltre il \textbf{24 Settembre 2024}.

\section{Limitazioni di Apache Kafka}



\section{Vantaggi di Redpanda}
\subsection{Performance}
Redpanda è scritto in C++ e utilizza il \textit{framework} Seastar, offrendo un'architettura \textit{thread-per-core} ad alte prestazioni.
Ciò permette di ottenere un'elevata \textit{throughput} e latenze costantemente basse, evitando cambi di contesto e blocchi.
Inoltre, è progettato per sfruttare l'\textit{hardware} moderno, tra cui unità NVMe, processori \textit{multi-core} e interfacce di rete ad alta velocità.

\subsection{Costi}
Anche per carichi di lavoro ridotti, l'utilizzo di Apache Kafka può essere fino a 3 volte più costoso rispetto a Redpanda. Per carichi di lavoro più complessi, questa differenza può aumentare fino a 5 volte o più (\href{https://redpanda.com/blog/is-redpanda-better-than-kafka-tco-comparison}{fonte dei dati}).

\subsection{Semplicità di configurazione}
Il binario di Redpanda include, oltre al \textit{message broker}, anche un \textit{proxy} HTTP e uno \textit{schema registry}.

\subsection{BYOC (\textit{Bring Your Own Cluster})}
Redpanda offre una terza opzione oltre alla gestione autonoma di un \textit{cluster} di \textit{streaming}
dati e all'utilizzo di un servizio \textit{cloud} completamente gestito: \textit{Bring Your Own Cluster} (BYOC).
Questa alternativa consente agli utenti finali di implementare una soluzione parzialmente gestita dal fornitore nella propria infrastruttura (come il proprio \textit{data center}
o il proprio \textit{VPC cloud}).

\subsection{Compatibilità con API di Kafka}
Redpanda è progettato per essere compatibile con le API di Kafka, consentendo di utilizzare i \textit{client} Kafka esistenti senza modifiche.

\subsection{\textit{Self-healing}}
Redpanda è self-healing e redistribuisce continuamente i dati e la \textit{leadership} tra i nodi per mantenere il \textit{cluster} in uno stato ottimale mentre evolve o quando i nodi falliscono.


\begin{center}
	\includegraphics[width=0.65\textwidth]{imgs/kafka_zookeeper.png}
	\captionof{figure}{\href{https://redpanda.com/blog/kafka-kraft-vs-redpanda-performance-2023}{Architettura di Kafka con ZooKeeper}}
\end{center}














\section{Svantaggi di Redpanda}
\subsection{Maturità}
\subsection{\textit{Community} e supporto}
\subsection{Integrazione con altri servizi}




\section{Vantaggi di Apache Kafka}
\subsection{Maturità}
Redpanda è stato rilasciato per la prima volta nel 2019, mentre Apache Kafka nel 2011.
Quest'ultimo dunque ha potuto svilupparsi e stabilizzarsi nel tempo, raggiungendo
un livello di maturità più elevato rispetto a Redpanda.\\
Ne consegue dunque che Apache Kafka è maggiormente diffuso e utilizzato in ambienti di
produzione.

\subsection{Licenza}
Apache Kafka è rilasciato con la licenza \textit{open source} Apache 2.0, la quale consente di utilizzare, modificare e distribuire il software liberamente.
Al contrario, sia l'edizione \textit{community} che quella \textit{enterprise} di Redpanda hanno licenza Business Source License (BSL), che
nonostante renda il codice sorgente disponibile, impone delle restrizioni sull'utilizzo e la distribuzione del software.


\subsection{Comunità e supporto}
Apache Kafka ha una vasta e attiva comunità di sviluppatori, che forniscono supporto, risorse e strumenti per estendere e migliorare il progetto.
La sua documentazione è molto completa e ben strutturata, con numerosi tutorial, guide e risorse online per imparare ad utilizzarlo.\\
Redpanda al contrario ha una comunità più piccola e meno attiva, con un numero ridotto di risorse disponibili.

\subsection{Integrazione con altri servizi}
Apache Kafka è supportato da una vasta gamma di strumenti e librerie di terze parti che lo integrano con altri sistemi e servizi
(con cui tuttavia \href{https://docs.redpanda.com/current/develop/kafka-clients/}{Redpanda è compatibile}).

\subsection{Scalabilità}
Redpanda dimostra bassa latenza e alto throughput su \textit{workload} semplici. Tuttavia esso è stato studiato per essere ottimizzato per il \textit{random IO}, e non per il \textit{sequential IO} come Apache Kafka.\\
Questo significa che in situazioni con un alto numero di produttori, un utilizzo del disco superiore al 30\%, l'abilitazione delle chiavi dei messaggi, l'abilitazione di TLS o l'esecuzione per più di 24 ore,
le prestazioni di Redpanda possono degradarsi significativamente.\\

\subsection{Protocollo di replicazione}
Il protocollo Raft utilizzato da Redpanda per la replicazione e la scrittura su disco è sincrona. \\
Nei sistemi Linux \textit{fsync} garantisce che i dati siano persistiti in modo sincrono, tuttavia
è un'operazione costosa in termini di prestazioni.\\
Apache Kafka può essere configurato per utilizzare anche un protocollo di replicazione asincrono, che non richiede l'utilizzo di \textit{fsync}.
Nonostante ciò, Redpanda è in grado di garantire prestazioni migliori rispetto ad Apache Kafka, come mostrato nel grafico sottostante.

\begin{center}
	\includegraphics[width=0.75\textwidth]{imgs/fsync.png}
	\captionof{figure}{\href{https://redpanda.com/blog/kafka-kraft-vs-redpanda-performance-2023}{Confronto di latenza tra Kafka e Redpanda con e senza \textit{fsync}.}}
\end{center}






























\section{\textit{Benchmark}}
Seguono i risultati dei \textit{benchmark} effettuati dal team di sviluppo di Redpanda, che confrontano
le prestazioni dei due strumenti.

\begin{center}
	\includegraphics[width=0.85\textwidth]{imgs/latency.png}
	\captionof{figure}{\href{https://redpanda.com/blog/is-redpanda-better-than-kafka-tco-comparison}{Risultati del \textit{benchmark} di latenza.}}
\end{center}

\begin{center}
	\includegraphics[width=0.85\textwidth]{imgs/costs.png}
	\captionof{figure}{\href{https://redpanda.com/blog/is-redpanda-better-than-kafka-tco-comparison}{Costo relativo di esecuzione di Redpanda vs Kafka.}}
\end{center}















\section{Tabella riassuntiva}

\begin{longtable}{|>{\centering\arraybackslash}p{0.30\textwidth}|>{\centering\arraybackslash}p{0.30\textwidth}|>{\centering\arraybackslash}p{0.30\textwidth}|}
	\hline
	\textbf{Paragone}                    & \textbf{Apache Kafka}                                                                                                                                       & \textbf{Redpanda}                                                                                                                                                                \\
	\hline
	\endfirsthead
	\hline
	\textbf{Paragone}                    & \textbf{Apache Kafka}                                                                                                                                       & \textbf{Redpanda}                                                                                                                                                                \\
	\endhead
	\hline
	\textbf{Adozione}                    & Utilizzato da migliaia di compagnie (tra cui LinkedIn, Airbnb, e Netflix)                                                                                   & Non chiaro quante organizzazioni lo usino. Adottato da Cisco e Vodafone.                                                                                                         \\
	\hline
	\textbf{\textit{Community}}          & Migliaia di contributori                                                                                                                                    & \textit{Community} più piccola ed emergente.                                                                                                                                     \\
	\hline
	\textbf{Maturità}                    & Stabile, sviluppato dal 2011                                                                                                                                & Emergente, lanciato nel 2019.                                                                                                                                                    \\
	\hline
	\textbf{Documentazione, risorse}     & Documentazione dettagliata, forum, tutorial, e corsi online                                                                                                 & Documentazione dettagliata, ma non altrettante risorse. Tutorial creati dal team di Redpanda.                                                                                    \\
	\hline
	\textbf{\textit{Client}}             & Ampia varietà di \textit{client} per i principali linguaggi di programmazione                                                                               & Lista di \href{https://docs.redpanda.com/current/develop/kafka-clients/}{client ufficialmente testati}, ma secondo la documentazione qualsiasi client Kafka dovrebbe funzionare. \\
	\hline
	\textbf{CLIs}                        & Include un set di strumenti per gestire i topic, messaggi, cluster...                                                                                       & Include \texttt{rpk} , un'interfaccia per gestire topic, messaggi, debugging, interazione con Redpanda Cloud.                                                                    \\
	\hline
	\textbf{Monitoraggio}                & Richiede configurazioni di sistemi di monitoraggio (JMX, Grafana, Prometheus)                                                                               & Integrato direttamente con Prometheus e Grafana.                                                                                                                                 \\
	\hline
	\textbf{Facilità di utilizzo}        & Complesso da configurare e gestire                                                                                                                          & Facile da installare e configurare, indipendente da Zookeeper                                                                                                                    \\
	\hline
	\textbf{Licenza}                     & Open source, Apache 2.0                                                                                                                                     & Edizioni \textit{Community} e \textit{Enterprise}, BSL (Business Source License).                                                                                                \\
	\hline
	\textbf{\textit{Deploy self-hosted}} & \textit{Bare-metal}, macchine virtuali, \textit{cloud}, \href{https://7last.github.io/docs/rtb/documentazione-interna/glossario#docker}{Docker}, Kubernetes & \textit{Bare-metal}, macchine virtuali, \textit{cloud}, \href{https://7last.github.io/docs/rtb/documentazione-interna/glossario#docker}{Docker}, Kubernetes                      \\
	\hline
	\textbf{\textit{Managed deploy}}     & Numerosi servizi di terze parti, come Confluent Cloud, AWS MSK...                                                                                           & Offre 3 opzioni: \textit{cluster} dedicati gestiti da Redpanda, BYOC (\textit{Bring Your Own Cloud}), \textit{cluster serverless} su architettura gestita da Redpanda.           \\
	\hline
	\caption{Riassunto del confronto tra \textit{Apache Kafka} e \textit{Redpanda}}
	\label{table:2}
\end{longtable}




\section{Sitografia}
\begin{itemize}
	\item \href{https://www.confluent.io/redpanda-vs-kafka-vs-confluent/}{Confronto tra Redpanda, Kafka e Confluent}
	\item \href{https://redpanda.com/blog/kafka-kraft-vs-redpanda-performance-2023}{KRaft}
	\item \href{https://redpanda.com/blog/is-redpanda-better-than-kafka-tco-comparison}{Benchmark}
	\item \href{https://www.kai-waehner.de/blog/2022/11/16/when-to-choose-redpanda-instead-of-apache-kafka/amp/}{Opzioni di deployment}
\end{itemize}









\end{document}







