\documentclass[italian,12pt]{article} %tipo di documento

%--------------variabili------------------%
\def\Title{Norme di Progetto}
\def\Author{7Last}
\def\Version{v0.2}
%-----------------------------------------%


\usepackage[left=2cm, right=2cm, bottom=3cm, top=3cm]{geometry}
\usepackage{fancyhdr}
\usepackage{graphicx}
\graphicspath{ {../../logo/} }
\usepackage{href-ul}
\usepackage{tikz}
\usepackage{tgadventor}
\usepackage[useregional=numeric,showseconds=true,showzone=false]{datetime2}
\usepackage{caption}
\usepackage{longtable}
\usepackage{xcolor}



% Definizione delle nuove classi di titolo
\titleclass{\subsubsubsection}{straight}[\subsection]
\titleclass{\subsubsubsubsection}{straight}[\subsubsubsection]
\titleclass{\subsubsubsubsubsection}{straight}[\subsubsubsubsection] % nuovo livello

% Creazione dei nuovi contatori
\newcounter{subsubsubsection}[subsubsection]
\newcounter{subsubsubsubsection}[subsubsubsection]
\newcounter{subsubsubsubsubsection}[subsubsubsubsection] % nuovo livello

% Rinnovo dei comandi per la formattazione dei numeri delle sezioni
\renewcommand\thesubsubsubsection{\thesubsubsection.\arabic{subsubsubsection}}
\renewcommand\thesubsubsubsubsection{\thesubsubsubsection.\arabic{subsubsubsubsection}}
\renewcommand\thesubsubsubsubsubsection{\thesubsubsubsubsection.\arabic{subsubsubsubsubsection}} % nuovo livello
\renewcommand\theparagraph{\thesubsubsubsubsubsection.\arabic{paragraph}} % opzionale; utile se i paragrafi devono essere numerati

% Formattazione dei titoli delle sezioni
\titleformat{\subsubsubsection}
  {\normalfont\normalsize\bfseries}{\thesubsubsubsection}{1em}{}
\titleformat{\subsubsubsubsection}
  {\normalfont\normalsize\bfseries}{\thesubsubsubsubsection}{1em}{}
\titleformat{\subsubsubsubsubsection} % nuovo livello
  {\normalfont\normalsize\bfseries}{\thesubsubsubsubsubsection}{1em}{} 

% Spaziatura dei titoli delle sezioni
\titlespacing*{\subsubsubsection}
{0pt}{3.25ex plus 1ex minus .2ex}{1.5ex plus .2ex}
\titlespacing*{\subsubsubsubsection}
{0pt}{3.25ex plus 1ex minus .2ex}{1.5ex plus .2ex}
\titlespacing*{\subsubsubsubsubsection} % nuovo livello
{0pt}{3.25ex plus 1ex minus .2ex}{1.5ex plus .2ex}

\makeatletter
% Rinnovo dei comandi per la formattazione dei paragrafi e sottoparagrafi
\renewcommand\paragraph{\@startsection{paragraph}{6}{\z@}%
  {3.25ex \@plus1ex \@minus.2ex}%
  {-1em}%
  {\normalfont\normalsize\bfseries}}
\renewcommand\subparagraph{\@startsection{subparagraph}{7}{\parindent}%
  {3.25ex \@plus1ex \@minus .2ex}%
  {-1em}%
  {\normalfont\normalsize\bfseries}}

% Definizione dei livelli per il Table of Contents
\def\toclevel@subsubsubsection{4}
\def\toclevel@subsubsubsubsection{5}
\def\toclevel@subsubsubsubsubsection{6} % nuovo livello
\def\toclevel@paragraph{7}
\def\toclevel@subparagraph{8}

% Definizione della formattazione per il Table of Contents
\def\l@subsubsubsection{\@dottedtocline{4}{7em}{4em}}
\def\l@subsubsubsubsection{\@dottedtocline{5}{10em}{5em}}
\def\l@subsubsubsubsubsection{\@dottedtocline{6}{14em}{6em}} % nuovo livello
\def\l@paragraph{\@dottedtocline{7}{18em}{7em}}
\def\l@subparagraph{\@dottedtocline{8}{22em}{8em}}
\makeatother

% Impostazione della profondità dei numeri di sezione e del Table of Contents
\setcounter{secnumdepth}{6} % nuovo livello
\setcounter{tocdepth}{6} % nuovo livello


\linespread{1.2}
\captionsetup[table]{labelformat=empty}
\geometry{headsep=1.5cm}

\renewcommand{\contentsname}{Indice}%imposto il nome dell'indice

\makeindex
\newlistof{tabelle}{tabelle}{Elenco delle tabelle}
\newlistof{immagini}{immagini}{Elenco delle figure}

\renewcommand{\cfttabelletitlefont}{\Large\bfseries}
\renewcommand{\cftimmaginititlefont}{\Large\bfseries}
%-------------------INIZIO DOCUMENTO--------------
\begin{document}

\newgeometry{left=2cm,right=2cm,bottom=2.1cm,top=2.1cm}
\begin{titlepage}
	\vspace*{.5cm}

	\vspace{2cm}
	{
		\centering
		{\bfseries\huge \Title\par}
		\bigbreak
		{\bfseries\Large \Subtitle\par}
		\bigbreak
		{\bfseries\large \Author\par}
		\bigbreak
		{\Date\;-\;\Version\par}
		\vfill

		\begin{center}
			\begin{tikzpicture}
				\clip (0,0) circle (2cm) node {\includegraphics[width=4cm]{logo.jpg}};
			\end{tikzpicture}
		\end{center}
	}

	\vfill

\end{titlepage}

\restoregeometry






















\newpage

\pagestyle{fancy}
\fancyhead{}
\lhead{
	\begin{tikzpicture}
		\clip (0,0) circle (0.5cm);
		\node at (0,0) {\includegraphics[width=1cm]{./../logo/logo.png}};
	\end{tikzpicture}%
}
\chead{\vspace{\fill}\Title\vspace{\fill}}
\rhead{\vspace{\fill}\Version\vspace{\fill}}


%-----------tabella revisioni-----------%
\begin{table}[!h]
    \caption{Versioni}
    \begin{center}
        \begin{tabular}{ l l l l p{9cm} }
            \hline                                                                                 \\[-2ex]
            Ver. & Data       & Autore  & Verificatore            & Descrizione                                   \\
            \\[-2ex] \hline \\[-1.5ex]
                                  \\
            0.1  & 28/03/2024 & Matteo Tiozzo  &    & Inizio scrittura documento                    \\
            0.0  & 28/03/2024 & Matteo Tiozzo &	   & Stesura struttura del documento                           \\
            \\[-1.5ex] \hline
        \end{tabular}
    \end{center}
\end{table}
%---------------------------------------%

\newpage

\tableofcontents

\newpage

\listoftabelle

\newpage

\listofimmagini

\newpage

\section{Introduzione}
\setcounter{subsection}{0}
\subsection{Scopo del documento}
Questo documento ha l’obiettivo di delineare la pianificazione e la gestione delle attività necessarie per la realizzazione del progetto.Vengono approfonditi aspetti chiave come l’\textit{Analisi dei Rischi}, il \textit{modello di sviluppo adottato}, la \textit{pianificazione delle attività}, la \textit{suddivisione dei ruoli}, nonché \textit{stime dei costi} e delle \textit{risorse necessarie}.

\subsection{Scopo del prodotto}
Lo scopo principale del prodotto é quello di permettere all’azienda \textit{Sync Lab S.r.l.} di poter valutare se é conveniente investire tempo e risorse per implementare il capitolato \textit{SyncCity - A smart city monitoring platform}. Una soluzione che, tramite l'uso di dispositivi IoT, permette di monitorare costantemente le città. SyncCity infatti servirà a monitorare e raccogliere dati da sensori posti in città, per poi analizzarli e fornire informazioni utili per la gestione della città stessa. Il prodotto finale sarà un prototipo funzionante che permetterà di visualizzare i dati raccolti in una dashboard.

\subsection{Glossario}
Al fine di evitare ambiguità o incomprensioni riguardanti i termini utilizzati nel documento, verrà adottato un glossario in cui saranno presenti le varie definizioni. La presenza di un termine all'interno del glossario verrà indicata applicando questo particolare \textcolor{red}{\uline{\textit{stile}}}.
\subsection{Riferimenti}
    \subsubsection{Normativi}SONO COMPLETAMENTE BUTTATI A CASO
        \begin{itemize}
            \item \textbf{ISO/IEC 12207:2008} - Systems and software engineering - Software life cycle processes
            \item \textbf{ISO/IEC 31000:2009} - Risk management - Principles and guidelines
        \end{itemize}
    \subsubsection{Informativi}
        \begin{itemize}
            \item T2 - Processi di ciclo di vita del software\\ https://www.math.unipd.it/~tullio/IS-1/2023/Dispense/T2.pdf
            \item T4 - Gestione di progetto\\ https://www.math.unipd.it/~tullio/IS-1/2023/Dispense/T4.pdf
            \item Glossario\\ Link al nostro glossario
        \end{itemize}
\subsection{Preventivo iniziale}
Il preventivo iniziale presentato in fase di candidatura è reperibile al seguente \uline{\href{https://github.com/7Last/docs/blob/main/1_candidatura/preventivo_costi_assunzione_impegni_v2.0.pdf}{link}}. All'interno di tale documento viene calcolato il preventivo iniziale del progetto, che equivale a €12.670,00. In aggiunta viene specificato che il gruppo \textit{7Last} stima di terminare il prodotto entro e non oltre la data 24 Settembre 2024.

\section{Analisi dei rischi}
Durante l'avanzamento del progetto è di fondamentale importanza mitigare gli impatti delle difficoltà riscontrate attraverso una corretta \textit{analisi dei rischi}. È stata inclusa questa sezione in questo documento al fine di evitare che eventuali problematiche possano compromettere il corretto svolgimento del progetto. 
Dopo aver elencato i rischi, vengono identificati una serie di passi da intraprendere nel caso in cui uno di essi si verifichi. Secondo lo standard ISO/IEC 31000:2009 \uline{CAMBIARE IL TIPO DI STANDARD}, il processo di gestione dei rischi si articola in 5 fasi:
\begin{itemize}
    \item \textbf{Identificazione dei rischi}: consiste nel riconoscere le possibili cause di \textcolor{red}{\uline{\textit{rischio}}}, le aree di impatto, gli eventi, le cause e le potenziali conseguenze. Questa fase sarà costituita da un'analisi delle attività che permetteranno di dare vita ad un elenco dei rischi basato sugli eventi che potrebbero influenzare il raggiungimento degli obiettivi. 
    \item \textbf{Analisi dei rischi}: questa fase si compone di un processo di valutazione che contribuisce alla valutazione e alle decisioni sul trattamento dei rischi, identificando le strategie più adeguate.
    \item \textbf{Valutazione dei rischi}: il goal di questa fase è quello di prendere decisioni basati sui risultati dell'analisi dei rischi in modo da poter attuare la migliore strategia di trattamento.
    \item \textbf{Trattamento dei rischi}: dopo l'analisi e la valutazione dei rischi, è di fondamentale importanza decidere come trattare questi rischi, al fine di ridurre il loro effetto. 
    \item \textbf{Monitoraggio e revisione dei rischi}: queste attività richiedono di essere integrate nella pianificazione del del processo di gestione del rischio e richiedono un controllo regolare.
\end{itemize}
I fattori fondamentali per identificare i rischi sono:
\begin{itemize}
    \item Tipologia: rappresenta la categoria di rischio, che può essere organizzativa, tecnologica o comunicativa.
    \item Indice: è un valore numerico incrementale che identifica univocamente il rischio per ogni Tipologia. Rischio elevato equivale a 3, rischio medio equivale a 2, mentre rischio basso equivale a 1.
\end{itemize}
Per una rappresentazione schematica dei rischi, si è deciso di attuare la seguente convenzione: R[Tipologia][Indice].

\subsection{Rischi organizzativi}
\begin{table}[!h]
    \hbox{RO1 - Inesperienza del team}
    \vspace{0.3cm}
	\begin{tabular}{|l|p{10cm}|} 
		\hline
		\textbf{Descrizione} & La sottostima/sovrastima dei costi delle attività a causa dell'inesperienza del team può causare ritardi o spreco di tempo \\ 
        \hline
        \textbf{Probabilità} & Media \\
        \hline
        \textbf{Pericolosità} & Alta \\
        \hline
        \textbf{Rilevamento} & Attraverso il cruscotto e confronto periodico con il Piano di Progetto, il Responsabile può monitorare lo stato di avanzamento del progetto \\
        \hline
        \textbf{Piano di contingenza} & In caso di cambiamenti non gravi, si cerca di implementare rapidamente quanto è rimasto aperto. Se significativo, si discute con il proponente per trovare un accordo su come affrontare i cambiamenti. \\
		\hline
	\end{tabular}
    \caption*{Tabella 1: descrizione tabella}
\end{table}

\begin{table}[!h]
    \hbox{RO2 - Imprecisione nella pianificazione delle attività}
    \vspace{0.3cm}
	\begin{tabular}{|l|p{10cm}|} 
		\hline
		\textbf{Descrizione} & La sottostima/sovrastima dei costi delle attività a causa dell'inesperienza del team può causare ritardi o spreco di tempo \\ 
        \hline
        \textbf{Probabilità} & Media \\
        \hline
        \textbf{Pericolosità} & Alta \\
        \hline
        \textbf{Rilevamento} & Attraverso il cruscotto e confronto periodico con il Piano di Progetto, il Responsabile può monitorare lo stato di avanzamento del progetto \\
        \hline
        \textbf{Piano di contingenza} & In caso di cambiamenti non gravi, si cerca di implementare rapidamente quanto è rimasto aperto. Se significativo, si discute con il proponente per trovare un accordo su come affrontare i cambiamenti. \\
		\hline
	\end{tabular}
    \caption*{Tabella 1: descrizione tabella}
\end{table}

\begin{table}[!h]
    \hbox{RO2 - Impegni personali o universitari}
    \vspace{0.3cm}
	\begin{tabular}{|l|p{10cm}|} 
		\hline
		\textbf{Descrizione} & La sottostima/sovrastima dei costi delle attività a causa dell'inesperienza del team può causare ritardi o spreco di tempo \\ 
        \hline
        \textbf{Probabilità} & Media \\
        \hline
        \textbf{Pericolosità} & Alta \\
        \hline
        \textbf{Rilevamento} & Attraverso il cruscotto e confronto periodico con il Piano di Progetto, il Responsabile può monitorare lo stato di avanzamento del progetto \\
        \hline
        \textbf{Piano di contingenza} & In caso di cambiamenti non gravi, si cerca di implementare rapidamente quanto è rimasto aperto. Se significativo, si discute con il proponente per trovare un accordo su come affrontare i cambiamenti. \\
		\hline
	\end{tabular}
    \caption*{Tabella 1: descrizione tabella}
\end{table}

\begin{table}[!h]
    \hbox{RO2 - Ritardi rispetto ai costi previsti}
    \vspace{0.3cm}
	\begin{tabular}{|l|p{10cm}|} 
		\hline
		\textbf{Descrizione} & La sottostima/sovrastima dei costi delle attività a causa dell'inesperienza del team può causare ritardi o spreco di tempo \\ 
        \hline
        \textbf{Probabilità} & Media \\
        \hline
        \textbf{Pericolosità} & Alta \\
        \hline
        \textbf{Rilevamento} & Attraverso il cruscotto e confronto periodico con il Piano di Progetto, il Responsabile può monitorare lo stato di avanzamento del progetto \\
        \hline
        \textbf{Piano di contingenza} & In caso di cambiamenti non gravi, si cerca di implementare rapidamente quanto è rimasto aperto. Se significativo, si discute con il proponente per trovare un accordo su come affrontare i cambiamenti. \\
		\hline
	\end{tabular}
    \caption*{Tabella 1: descrizione tabella}
\end{table}

\begin{table}[!h]
    \hbox{RO2 - Scarsa collaborazione da parte di uno o più membri}
    \vspace{0.3cm}
	\begin{tabular}{|l|p{10cm}|} 
		\hline
		\textbf{Descrizione} & La sottostima/sovrastima dei costi delle attività a causa dell'inesperienza del team può causare ritardi o spreco di tempo \\ 
        \hline
        \textbf{Probabilità} & Media \\
        \hline
        \textbf{Pericolosità} & Alta \\
        \hline
        \textbf{Rilevamento} & Attraverso il cruscotto e confronto periodico con il Piano di Progetto, il Responsabile può monitorare lo stato di avanzamento del progetto \\
        \hline
        \textbf{Piano di contingenza} & In caso di cambiamenti non gravi, si cerca di implementare rapidamente quanto è rimasto aperto. Se significativo, si discute con il proponente per trovare un accordo su come affrontare i cambiamenti. \\
		\hline
	\end{tabular}
    \caption*{Tabella 1: descrizione tabella}
\end{table}

\subsection{Rischi tecnologici}
\hbox{RT2 - Inesperienza nell'uso delle tecnologie adottate}
\begin{table}[!h]
	\begin{tabular}{ c c c c c } 
		\hline
		\textbf{Descrizione} & \textbf{Probabilità} & \textbf{Pericolosità} &\textbf{Rilevamento} & \textbf{Piano di contingenza} \\
		\hline 
        Descrizione esempio & Alta & Elevata & Facile & Piano esempio \\
		\hline
	\end{tabular}
\end{table}

\hbox{RT2 - Perdita di informazioni}
\begin{table}[!h]
	\begin{tabular}{ c c c c c } 
		\hline
		\textbf{Descrizione} & \textbf{Probabilità} & \textbf{Pericolosità} &\textbf{Rilevamento} & \textbf{Piano di contingenza} \\
		\hline 
        Descrizione esempio & Alta & Elevata & Facile & Piano esempio \\
		\hline
	\end{tabular}
\end{table}

\hbox{RT3 - Problemi di compatibilità tra le tecnologie utilizzate}
\begin{table}[!h]
	\begin{tabular}{ c c c c c } 
		\hline
		\textbf{Descrizione} & \textbf{Probabilità} & \textbf{Pericolosità} &\textbf{Rilevamento} & \textbf{Piano di contingenza} \\
		\hline 
        Descrizione esempio & Alta & Elevata & Facile & Piano esempio \\
		\hline
	\end{tabular}
\end{table}

\subsection{Rischi comunicativi}
\hbox{RC1 - Disaccordi all'interno del gruppo }
\begin{table}[!h]
	\begin{tabular}{ c c c c c } 
		\hline
		\textbf{Descrizione} & \textbf{Probabilità} & \textbf{Pericolosità} &\textbf{Rilevamento} & \textbf{Piano di contingenza} \\
		\hline 
        Descrizione esempio & Alta & Elevata & Facile & Piano esempio \\
		\hline
	\end{tabular}
\end{table}

\hbox{RC2 - Problemi di comunicazione }
\begin{table}[!h]
	\begin{tabular}{ c c c c c } 
		\hline
		\textbf{Descrizione} & \textbf{Probabilità} & \textbf{Pericolosità} &\textbf{Rilevamento} & \textbf{Piano di contingenza} \\
		\hline 
        Descrizione esempio & Alta & Elevata & Facile & Piano esempio \\
		\hline
	\end{tabular}
\end{table}

\subsection{Valutazione delle misure mitegative}
VALUTA SE FARE UNA PARTE QUI, OPPURE SE AGGIUNGERE UNA COLONNA ALLA TABELLA SOPRA

\section{Pianificazione}
\subsection{Modello adottato}
METTERE MODELLO UNA VOLTA DECISO QUALE ADOTTARE

\subsection{Periodi}
Per ogni periodo si riportano di seguito le seguenti informazioni:
\begin{itemize}
    \item Data di inizio, data di fine prevista, data di fine attuale ed eventuali giorni di ritardo
    \item Pianificazione delle attività da svolgere al suo interno (avanzamento atteso), con tanto di potenziali rischi
    \item Tempo stimato per poter completare tutte le attività previste (preventivo)
    \item Confronto fra il lavoro svolto (avanzamento conseguito) e quello preventivato, con annessa analisi dei costi;
    \item Rischi effettivamente occorsi, valutandone il loro impatto e la loro mitigazione;
    \item Retrospettiva di periodo per capire cosa e come migliorare in futuro e cosa invece mantenere.
\end{itemize}
I periodi vengono suddivisi in 3 grandi insiemi corrispondenti alle revisioni di avanzamento del progetto:
\begin{itemize}
    \item \textbf{RTB}: \textcolor{red}{\uline{\textit{Requirements and Technology Baseline}}}
    \item \textbf{PB}:  Product \textcolor{red}{\uline{\textit{Baseline}}}
    \item \textbf{CA}:  \textcolor{red}{\uline{\textit{Customer Acceptance}}}
\end{itemize}

\subsection{Requirements and Technology Baseline}
    \subsubsection{Primo sprint:}
        \begin{itemize}
            \item Inizio: 26/03/2024
            \item FIne: 09/04/2024
            \item Fine attuale:
            \item Giorni di ritardo:
        \end{itemize}
        \subsubsubsection{Pianificazione} 
        SCRIVI QUALCOSA A RIGUARDO (PT. 4.1.1.1 DEGLI OVERTURE)
        \subsubsubsubsection{Rischi attesi}
        I rischi attesi per questo periodo sono:
        \begin{itemize}
            \item RT2 - Inesperienza del team
            \item RO2 - Imprecisione nella pianificazione delle attività
            \item RO2 - Elevati costi delle attività
            \item RC1 - Rischio di conflitti interni 
            \item RC1 - Problemi di comunicazione
        \end{itemize}
        Questo è dovuto al fatto che, essendo l'inizio del progetto, non sappiamo ancora come organizzarci per ottimizzare al meglio il tempo e le risorse. La probabilità che si verifichi uno dei rischi elencati è elevata.
        \subsubsubsection{Preventivo}
        Ruoli coinvolti: Amministratore, Responsabile, Verificatore, Analista, Progettista.

        \begin{table}[!h]
            \begin{tabular}{ l c c c c c } 
                \hline
                \textbf{} & \textbf{Amministratore} & \textbf{Responsabile} & \textbf{Verificatore} &\textbf{Analista} & \textbf{Progettista} \\
                \hline 
                Tiozzo      & 0 & 0 & 0 & 0 & 0 \\ 
                Malgarise   & 0 & 0 & 0 & 0 & 0 \\ 
                Ferro       & 0 & 0 & 0 & 0 & 0 \\ 
                Benetazzo   & 0 & 0 & 0 & 0 & 0 \\ 
                Occhinegro  & 0 & 0 & 0 & 0 & 0 \\ 
                Baldo       & 0 & 0 & 0 & 0 & 0 \\ 
                Seganfreddo & 0 & 0 & 0 & 0 & 0 \\
                \hline
            \end{tabular}
            \caption*{Tabella 1: preventivo orario per ruolo di ciascun membro del team durante il primo periodo}
        \end{table}

        AGGIUNGERE GRAFICO INUTILE

        \subsubsubsection{Consuntivo}
        Le attività previste sono state tutte svolte con successo
        \subsubsubsubsection{Prospetto orario}
        \begin{table}[!h]
            \begin{tabular}{ l c c c c c } 
                \hline
                \textbf{} & \textbf{Amministratore} & \textbf{Responsabile} & \textbf{Verificatore} &\textbf{Analista} & \textbf{Progettista} \\
                \hline 
                Tiozzo      & 0 & 0 & 0 & 0 & 0 \\ 
                Malgarise   & 0 & 0 & 0 & 0 & 0 \\ 
                Ferro       & 0 & 0 & 0 & 0 & 0 \\ 
                Benetazzo   & 0 & 0 & 0 & 0 & 0 \\ 
                Occhinegro  & 0 & 0 & 0 & 0 & 0 \\ 
                Baldo       & 0 & 0 & 0 & 0 & 0 \\ 
                Seganfreddo & 0 & 0 & 0 & 0 & 0 \\
                \hline
            \end{tabular}
            \caption*{Tabella 2: impegno effettivo per ruolo di ciascun membro del team durante il primo periodo}
        \end{table}
        \subsubsubsubsection{Prospetto economico}
        \begin{table}[!h]
            \begin{tabular}{ l c c c c c } 
                \hline
                \textbf{} & \textbf{Ruolo} & \textbf{Ore} & \textbf{Costo} &\textbf{Differenza} \\
                \hline  
                 & Responsabile        & 0 & 0 & 0 \\ 
                 & Amministratore      & 0 & 0 & 0 \\ 
                 & Verificatore        & 0 & 0 & 0 \\ 
                 & Analista            & 0 & 0 & 0 \\ 
                 & Progettista         & 0 & 0 & 0 \\ 
                 & Programmatore       & 0 & 0 & 0 \\ 
                 & Seganfreddo         & 0 & 0 & 0 \\
                Totale preventivo & - & 0 & 0 &0 \\
                Totale consuntivo & - & 0 & 0 & 0\\
                \hline
            \end{tabular}
            \caption*{Tabella 12: aggiornamenti economici del progetto al termine del primo periodo, riflettendo le variazioni tra preventivo e ore effettivamente lavorate}
        \end{table}
        \subsubsubsubsection{Rischi effettivamente occorsi e loro mitigazione}
        SCRITTURA DEI RISCHI CHE ABBIAMO TROVATO DURANTE QUESTO PRIMO PERIODO E COME LI ABBIAMO RISOLTI
        \subsubsubsection{Retrospettiva}
        SCRITTURA DI COSA ABBIAMO FATTO BENE E COSA ABBIAMO SBAGIATO, COSÌ DA MIGLIORARCI

    \subsubsection{Secondo sprint:}
    \begin{itemize}
        \item Inizio: 10/04/2024
        \item FIne: 24/04/2024
        \item Fine attuale:
        \item Giorni di ritardo:
    \end{itemize}
    \subsubsubsection{Pianificazione} 
    SCRIVI QUALCOSA A RIGUARDO (PT. 4.1.1.1 DEGLI OVERTURE)
    \subsubsubsubsection{Rischi attesi}
    I rischi attesi per questo periodo sono:
    \begin{itemize}
        \item Inesperienza del team
        \item Imprecisione nella pianificazione delle attività
        \item Elevati costi delle attività
        \item Rischio di conflitti interni 
        \item problemi di comunicazione
        \item problemi di coordinamento
    \end{itemize}
    Questo perchè, essendo all’inizio del progetto, siamo ancora incerti su molti aspetti di quest’ultimo, ci stiamo attualmente organizzando e dobbiamo apprendere ancora molto, dunque la probabilità di incorrere in qualche problema tra quelli riportati è abbastanza elevata.
    \subsubsubsection{Preventivo}
    Ruoli coinvolti: 
    \subsubsubsection{Consuntivo}
    Le attività previste sono state tutte svolte con successo
    \subsubsubsubsection{Prospetto orario}
    \subsubsubsubsection{Prospetto economico}
    \subsubsubsubsection{Rischi effettivamente occorsi e loro mitigazione}
    \subsubsubsection{Retrospettiva}


    \subsubsection{Terzo sprint:}
    \begin{itemize}
        \item Inizio: 25/04/2024
        \item FIne: 09/05/2024
        \item Fine attuale:
        \item Giorni di ritardo:
    \end{itemize}
    \subsubsubsection{Pianificazione} 
    SCRIVI QUALCOSA A RIGUARDO (PT. 4.1.1.1 DEGLI OVERTURE)
    \subsubsubsubsection{Rischi attesi}
    I rischi attesi per questo periodo sono:
    \begin{itemize}
        \item Inesperienza del team
        \item Imprecisione nella pianificazione delle attività
        \item Elevati costi delle attività
        \item Rischio di conflitti interni 
        \item problemi di comunicazione
        \item problemi di coordinamento
    \end{itemize}
    Questo perchè, essendo all’inizio del progetto, siamo ancora incerti su molti aspetti di quest’ultimo, ci stiamo attualmente organizzando e dobbiamo apprendere ancora molto, dunque la probabilità di incorrere in qualche problema tra quelli riportati è abbastanza elevata.
    \subsubsubsection{Preventivo}
    Ruoli coinvolti: 
    \subsubsubsection{Consuntivo}
    Le attività previste sono state tutte svolte con successo
    \subsubsubsubsection{Prospetto orario}
    \subsubsubsubsection{Prospetto economico}
    \subsubsubsubsection{Rischi effettivamente occorsi e loro mitigazione}
    \subsubsubsection{Retrospettiva}

    \subsubsection{Quarto sprint:}
    \begin{itemize}
        \item Inizio: 10/05/2024
        \item FIne: 24/05/2024
        \item Fine attuale:
        \item Giorni di ritardo:
    \end{itemize}
    \subsubsubsection{Pianificazione} 
    SCRIVI QUALCOSA A RIGUARDO (PT. 4.1.1.1 DEGLI OVERTURE)
    \subsubsubsubsection{Rischi attesi}
    I rischi attesi per questo periodo sono:
    \begin{itemize}
        \item Inesperienza del team
        \item Imprecisione nella pianificazione delle attività
        \item Elevati costi delle attività
        \item Rischio di conflitti interni 
        \item problemi di comunicazione
        \item problemi di coordinamento
    \end{itemize}
    Questo perchè, essendo all’inizio del progetto, siamo ancora incerti su molti aspetti di quest’ultimo, ci stiamo attualmente organizzando e dobbiamo apprendere ancora molto, dunque la probabilità di incorrere in qualche problema tra quelli riportati è abbastanza elevata.
    \subsubsubsection{Preventivo}
    Ruoli coinvolti: 
    \subsubsubsection{Consuntivo}
    Le attività previste sono state tutte svolte con successo
    \subsubsubsubsection{Prospetto orario}
    \subsubsubsubsection{Prospetto economico}
    \subsubsubsubsection{Rischi effettivamente occorsi e loro mitigazione}
    \subsubsubsection{Retrospettiva}

    \subsubsection{Quinto sprint:}
    \begin{itemize}
        \item Inizio: 25/05/2024
        \item FIne: 09/06/2024
        \item Fine attuale:
        \item Giorni di ritardo:
    \end{itemize}
    \subsubsubsection{Pianificazione} 
    SCRIVI QUALCOSA A RIGUARDO (PT. 4.1.1.1 DEGLI OVERTURE)
    \subsubsubsubsection{Rischi attesi}
    I rischi attesi per questo periodo sono:
    \begin{itemize}
        \item Inesperienza del team
        \item Imprecisione nella pianificazione delle attività
        \item Elevati costi delle attività
        \item Rischio di conflitti interni 
        \item problemi di comunicazione
        \item problemi di coordinamento
    \end{itemize}
    Questo perchè, essendo all’inizio del progetto, siamo ancora incerti su molti aspetti di quest’ultimo, ci stiamo attualmente organizzando e dobbiamo apprendere ancora molto, dunque la probabilità di incorrere in qualche problema tra quelli riportati è abbastanza elevata.
    \subsubsubsection{Preventivo}
    Ruoli coinvolti: 
    \subsubsubsection{Consuntivo}
    Le attività previste sono state tutte svolte con successo
    \subsubsubsubsection{Prospetto orario}
    \subsubsubsubsection{Prospetto economico}
    \subsubsubsubsection{Rischi effettivamente occorsi e loro mitigazione}
    \subsubsubsection{Retrospettiva}


    \subsubsection{Sesto sprint:}
    \begin{itemize}
        \item Inizio: 10/06/2024
        \item FIne: 24/06/2024
        \item Fine attuale:
        \item Giorni di ritardo:
    \end{itemize}
    \subsubsubsection{Pianificazione} 
    SCRIVI QUALCOSA A RIGUARDO (PT. 4.1.1.1 DEGLI OVERTURE)
    \subsubsubsubsection{Rischi attesi}
    I rischi attesi per questo periodo sono:
    \begin{itemize}
        \item Inesperienza del team
        \item Imprecisione nella pianificazione delle attività
        \item Elevati costi delle attività
        \item Rischio di conflitti interni 
        \item problemi di comunicazione
        \item problemi di coordinamento
    \end{itemize}
    Questo perchè, essendo all’inizio del progetto, siamo ancora incerti su molti aspetti di quest’ultimo, ci stiamo attualmente organizzando e dobbiamo apprendere ancora molto, dunque la probabilità di incorrere in qualche problema tra quelli riportati è abbastanza elevata.
    \subsubsubsection{Preventivo}
    Ruoli coinvolti: 
    \subsubsubsection{Consuntivo}
    Le attività previste sono state tutte svolte con successo
    \subsubsubsubsection{Prospetto orario}
    \subsubsubsubsection{Prospetto economico}
    \subsubsubsubsection{Rischi effettivamente occorsi e loro mitigazione}
    \subsubsubsection{Retrospettiva}

    \subsubsection{Settimo sprint:}
    \begin{itemize}
        \item Inizio: 25/06/2024
        \item FIne: 09/07/2024
        \item Fine attuale:
        \item Giorni di ritardo:
    \end{itemize}
    \subsubsubsection{Pianificazione} 
    SCRIVI QUALCOSA A RIGUARDO (PT. 4.1.1.1 DEGLI OVERTURE)
    \subsubsubsubsection{Rischi attesi}
    I rischi attesi per questo periodo sono:
    \begin{itemize}
        \item Inesperienza del team
        \item Imprecisione nella pianificazione delle attività
        \item Elevati costi delle attività
        \item Rischio di conflitti interni 
        \item problemi di comunicazione
        \item problemi di coordinamento
    \end{itemize}
    Questo perchè, essendo all’inizio del progetto, siamo ancora incerti su molti aspetti di quest’ultimo, ci stiamo attualmente organizzando e dobbiamo apprendere ancora molto, dunque la probabilità di incorrere in qualche problema tra quelli riportati è abbastanza elevata.
    \subsubsubsection{Preventivo}
    Ruoli coinvolti: 
    \subsubsubsection{Consuntivo}
    Le attività previste sono state tutte svolte con successo
    \subsubsubsubsection{Prospetto orario}
    \subsubsubsubsection{Prospetto economico}
    \subsubsubsubsection{Rischi effettivamente occorsi e loro mitigazione}
    \subsubsubsection{Retrospettiva}

    \subsubsection{Sommario finale}
    \subsubsubsection{Riepilogo prospetto orario}
    \subsubsubsubsection{Ore consumate}
    \subsubsubsubsection{Ore rimanenti}
    \subsubsubsection{Riepilogo prospetto economico}
    \subsubsubsection{Costi totali}

    \subsection{Tra RTB e PB}
    \subsubsection{Ottavo periodo}
    \begin{itemize}
        \item Inizio: 10/07/2024
        \item FIne: 24/07/2024
        \item Fine attuale:
        \item Giorni di ritardo:
    \end{itemize}
    \subsubsubsection{Pianificazione} 
    SCRIVI QUALCOSA A RIGUARDO (PT. 4.1.1.1 DEGLI OVERTURE)
    \subsubsubsubsection{Rischi attesi}
    I rischi attesi per questo periodo sono:
    \begin{itemize}
        \item Inesperienza del team
        \item Imprecisione nella pianificazione delle attività
        \item Elevati costi delle attività
        \item Rischio di conflitti interni 
        \item problemi di comunicazione
        \item problemi di coordinamento
    \end{itemize}
    Questo perchè, essendo all’inizio del progetto, siamo ancora incerti su molti aspetti di quest’ultimo, ci stiamo attualmente organizzando e dobbiamo apprendere ancora molto, dunque la probabilità di incorrere in qualche problema tra quelli riportati è abbastanza elevata.
    \subsubsubsection{Preventivo}
    Ruoli coinvolti: 
    \subsubsubsection{Consuntivo}
    \subsubsubsubsection{Prospetto orario}
    \subsubsubsubsection{Prospetto economico}
    \subsubsubsubsection{Rischi effettivamente occorsi e loro mitigazione}
    \subsubsubsection{Retrospettiva}

% \subsection{Product Baseline}


% \section{Preventivo}
% \subsection{Requirements and Technology Baseline}
%     \subsubsection{Primo sprint:}
%     \subsubsection{Secondo sprint:}
%     \subsubsection{Terzo sprint:}
%     \subsubsection{Quarto sprint:}
%     \subsubsection{Quinto sprint:}
%     \subsubsection{Sesto sprint:}
%     \subsubsection{Settimo sprint:}
% \subsection{Preventivo a finire}
%     \subsubsection{Resoconto finale RTB}
%     \subsubsection{Risuddivisione oraria}

% \section{Consuntivo}
% \subsection{Requirements and Technology Baseline}
%     \subsubsection{Primo sprint:}
%     \subsubsection{Secondo sprint:}
%     \subsubsection{Terzo sprint:}
%     \subsubsection{Quarto sprint:}
%     \subsubsection{Quinto sprint:}
%     \subsubsection{Sesto sprint:}
%     \subsubsection{Settimo sprint:}

\newpage

%-----------Secondo indice-----------%

\listoftabelle

\setcounter{section}{0}
\section*{Tabella 1}
\addcontentsline{tabelle}{section}{Tabella dei rischi}

\setcounter{subsection}{0}
\addcontentsline{tabelle}{subsection}{Tabella 1}
\addcontentsline{tabelle}{subsection}{Tabella 2}
\subsection*{Tabella 1}
\subsection*{Tabella 2}

% altre tabelle ecc...

%------------------------------------%
\newpage
%------------Terzo indice------------%

\listofimmagini

\setcounter{section}{0}
\section*{Figura 1}
\addcontentsline{immagini}{section}{Tabella delle immagini}

\setcounter{subsection}{0}
\addcontentsline{immagini}{subsection}{Figura 1}
\addcontentsline{immagini}{subsection}{Figura 2}
\subsection*{Figura 1}
\subsection*{Figura 2}

% altre figure ecc...
%------------------------------------%
\end{document}