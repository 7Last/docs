\section{Casi d'uso}
\subsection{Introduzione}
In questa sezione del documento vengono analizzati nel dettaglio i casi d'uso individuati per il sistema.
nel corso dell'analisi del capitolato e dei colloqui con la proponente.

\subsection{Struttura dei casi d'uso}
In tutto il documento ci si riferirà ai casi d'uso utilizzando la sigla \texttt{UC} seguita dal rispettivo codice nella forma
\begin{center}
	\textbf{UC[identificativo\_caso\_principale].[identificativo\_sotto\_caso]}
\end{center}

il quale permette di utilizzarlo come riferimento in questo e altri documenti.\\
Per ciascun caso d'uso vengono definiti i seguenti elementi:
\begin{itemize}
	\item \textbf{Attore principale}: l'attore primariamente coinvolto nel caso d'uso;
	\item \textbf{Precondizioni}: le condizioni che devono essere verificate affinché il caso d'uso possa essere
	      eseguito;
	\item \textbf{Postcondizioni}: le condizioni che devono essere verificate al termine dell'esecuzione del caso
	\item \textbf{Scenario principale}: la sequenza di passi che descrive il comportamento del sistema durante
	      l'esecuzione del caso d'uso;
	\item \textbf{Sotto-scenari} (opzionale): eventuali scenari alternativi che possono verificarsi durante l'esecuzione del
	      caso d'uso.
\end{itemize}


\subsection{Attori}
I seguenti attori sono coinvolti nei casi d'uso:
\begin{itemize}
	\item Impiegati presso \textbf{autorità locali}: essi possono accedere al sistema per visualizzare i dati di
	      monitoraggio della \textit{Smart City}.
	\item \textbf{Sensori}: sorgente di dati con un determinato dominio di interesse che effettua misurazioni
	      e trasmette i dati al sistema.
\end{itemize}

\subsection{Elenco dei casi d'uso}





