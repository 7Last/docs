\section{Casi d'uso}
\subsection{Introduzione}
In questa sezione del documento vengono analizzati nel dettaglio i casi d'uso individuati per il sistema.
nel corso dell'analisi del \href{https://7last.github.io/docs/rtb/documentazione-interna/glossario\#capitolato}{capitolato\textsubscript{G}} e dei colloqui con la proponente.

\subsection{Struttura dei casi d'uso}
In tutto il documento ci si riferirà ai casi d'uso utilizzando la sigla \texttt{UC} seguita dal rispettivo codice nella forma
\begin{center}
	\textbf{UC-[identificativo\_caso\_principale].[identificativo\_sotto\_caso]}
\end{center}

il quale permette di utilizzarlo come riferimento in questo e altri documenti.\\
Per ciascun caso d'uso vengono definiti i seguenti elementi:
\begin{itemize}
	\item \textbf{Attore principale}: l'attore primariamente coinvolto nel caso d'uso;
	\item \textbf{Precondizioni}: le condizioni che devono essere verificate affinché il caso d'uso possa essere
	      eseguito;
	\item \textbf{Postcondizioni}: le condizioni che devono essere verificate al termine dell'esecuzione del caso
	\item \textbf{Scenario principale}: la sequenza di passi che descrive il comportamento del sistema durante
	      l'esecuzione del caso d'uso;
	\item \href{https://7last.github.io/docs/rtb/documentazione-interna/glossario\#user-story}{\textbf{User story}\textsubscript{G}} (opzionale): una descrizione testuale del caso d'uso;
	\item \textbf{Sotto-scenari} (opzionale): eventuali scenari alternativi che possono verificarsi durante l'esecuzione del
	      caso d'uso.
\end{itemize}


\subsection{Attori}
I seguenti attori sono coinvolti nei casi d'uso:
\begin{itemize}
	\item Impiegati presso \textbf{autorità locali}: essi possono accedere al sistema per visualizzare i dati
	      monitoraggio della \textit{Smart City}.
	\item \textbf{Sensori}: sorgente di dati con un determinato dominio di interesse che effettua misurazioni
	      e trasmette i dati al sistema.
\end{itemize}

\subsection{Elenco dei casi d'uso}
\subsubsection{UC-1: Visualizzazione dashboard}
\subsubsubsection{UC-1.1: Visualizzazione dashboard per tipologia di sensore}
\subsubsubsection{UC-1.2: Visualizzazione mappa sensori}
\subsubsubsection{UC-1.3: Visualizzazione lista sensori}

\subsubsection{UC-2: Visualizzazione dati atmosferici}
\subsubsubsection{UC-2.1: Visualizzazione dati temperatura}
% time series
% tempo reale
% media
% massima
% minima
\subsubsubsection{UC-2.2: Visualizzazione dati umidità}
% time series
% tempo reale
% media
% massima
% minima
\subsubsubsection{UC-2.3: Visualizzazione dati pressione}
% time series
% tempo reale
% media
% massima
% minima
\subsubsubsection{UC-2.4: Visualizzazione dati vento}
% time series velocità
% time series direzione
% tempo reale
% massima velocità
% direzione
\subsubsubsection{UC-2.4: Visualizzazione dati precipitazioni}
% time series
% tempo reale
% quantità totale caduta
% quantità media
\subsubsubsection{UC-2.4: Visualizzazione dati polveri sottili}
% time series
% tempo reale
% giorno con maggiore concentrazione
% giorno con minore concentrazione
% media giornaliera

\subsubsection{UC-3: Visualizzazione dati urbani}
\subsubsubsection{UC-3.1: Visualizzazione dati traffico}
% time series (ultimo giorno)

\subsubsubsection{UC-3.2: Visualizzazione dati lavori in corso}
% attività in corso

\subsubsubsection{UC-3.3: Visualizzazione dati incidenti}
% n incidenti nell'ultimo mese
% n incidenti nell'ultimo anno

\subsubsubsection{UC-3.4: Visualizzazione dati colonnine di ricarica}
% dati guasti
% mappa con colonnine per stato
% n colonnine per stato

\subsubsubsection{UC-3.5: Visualizzazione dati isole ecologiche}
% mappa con isole per stato
% n isole per stato (piene, vuote, mezzo piene)

\subsubsubsection{UC-3.6: Visualizzazione dati parcheggi}
% mappe parcheggi per stato
% n parcheggi per stato 

\subsubsubsection{UC-3.7: Visualizzazione dati livello di acqua}
% time series

\subsubsection{UC-4: Visualizzazione allerte}

\subsubsection{UC-5: Visualizzazione con filtri}

\subsubsection{UC-6: Visualizzazione errori}

\subsubsection{UC-7: Trasmissione dati temperatura}

\subsubsection{UC-8: Trasmissione dati umidità}

\subsubsection{UC-9: Trasmissione dati pressione}

\subsubsection{UC-10: Trasmissione dati vento}

\subsubsection{UC-11: Trasmissione dati precipitazioni}

\subsubsection{UC-12: Trasmissione dati polveri sottili}

\subsubsection{UC-13: Trasmissione dati traffico}

\subsubsection{UC-14: Trasmissione dati lavori in corso}

\subsubsection{UC-15: Trasmissione dati incidenti}

\subsubsection{UC-16: Trasmissione dati colonnine di ricarica}

\subsubsection{UC-17: Trasmissione dati isole ecologiche}

\subsubsection{UC-18: Trasmissione dati parcheggi}

\subsubsection{UC-19: Trasmissione dati livello di acqua}




