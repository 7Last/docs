\section{Requisiti}
\subsection{Definizione di un requisito}
Per ciascun requisito vengono fornite le seguenti informazioni:
\begin{itemize}
	\item \textbf{Codice}: codice identificativo del requisito, meglio specificato nella sezione \ref{sec:codifica_requisiti};
	\item \textbf{Descrizione}: breve descrizione del requisito;
	\item \textbf{Fonte}: provenienza del requisito, meglio specificata nella sezione \ref{sec:fonti_requisiti};
	\item \textbf{Importanza}: indica l'importanza del requisito, meglio specificata nella sezione \ref{sec:importanza_requisiti}.
\end{itemize}

\subsection{Tipologie di requisiti}
I requisiti possono essere di quattro tipologie:
\begin{itemize}
	\item \textbf{Funzionali}: descrivono le funzionalità del sistema;
	\item \textbf{Qualitativi}: descrivono le qualità che il sistema deve avere;
	\item \textbf{Di vincolo}: descrivono i vincoli a cui il sistema deve sottostare;
	\item \textbf{Prestazionali}: descrivono le prestazioni che il sistema deve avere.
\end{itemize}

\subsubsection{Codifica dei requisiti}
\label{sec:codifica_requisiti}
I requisiti sono codificati nel seguente modo:
\begin{center}
	\textbf{R[Tipologia]-[Codice]}
\end{center}
dove \textbf{[Codice]} è un numero progressivo che identifica univocamente il requisito.

\subsubsection{Fonti dei requisiti}
\label{sec:fonti_requisiti}
I requisiti possono avere le seguenti fonti:
\begin{itemize}
	\item \textbf{Capitolato}: requisiti individuati a seguito dell'analisi del capitolato;
	\item \textbf{Interno}: requisiti individuati durante le riunioni interne e da coloro che hanno il ruolo di analista;
	\item \textbf{Esterno}: requisiti aggiuntivi individuati in seguito a incontri con la proponente;
	\item \textbf{Piano di Qualifica}: requisiti necessari per adeguare il prodotto agli standard di qualità definiti nel documento \textit{Piano di Qualifica}.
	\item \textbf{Norme di Progetto}: requisiti necessari per adeguare il prodotto alle norme stabilite nel documento \textit{Norme di Progetto};
	\item \textbf{Caso d'uso}: requisiti individuati da uno o più casi d'uso, di cui si riporta il codice.
\end{itemize}

\subsubsection{Importanza dei requisiti}
\label{sec:importanza_requisiti}
I requisiti possono avere tre livelli di importanza:
\begin{itemize}
	\item \textbf{Obbligatorio}: requisito irrinunciabile per il committente;
	\item \textbf{Desiderabile}: requisito non strettamente necessario, ma che porta valore aggiunto al prodotto;
	\item \textbf{Opzionale}: requisito relativo a funzionalità aggiuntive.
\end{itemize}


\subsection{Requisiti funzionali}
\begin{longtable}{|>{\centering\arraybackslash}m{0.10\textwidth}|>{\centering\arraybackslash}m{0.20\textwidth}|>{\centering\arraybackslash}m{0.20\textwidth}|>{\centering\arraybackslash}m{0.4\textwidth}|}
	\hline
	\textbf{Codice} & \textbf{Importanza} & \textbf{Fonte} & \textbf{Descrizione}                                                                                                                                                                                                                                                                                               \\\hline
	\endfirsthead
	\hline
	\textbf{Codice} & \textbf{Importanza} & \textbf{Fonte} & \textbf{Descrizione}                                                                                                                                                                                                                                                                                               \\\hline
	\endhead
	\hline
	RF-1            & Obbligatorio        & Capitolato     & La parte \textit{IoT} dovrà essere simulata attraverso tool di generazione di informazioni random che tuttavia siano verosimili.                                                                                                                                                                                   \\\hline
	RF-2            & Obbligatorio        & Capitolato     & Il sistema dovrà permettere la visualizzazione dei dati in tempo reale.                                                                                                                                                                                                                                            \\\hline
	RF-3            & Obbligatorio        & Capitolato     & Il sistema dovrà permettere la visualizzazione dei dati storici.                                                                                                                                                                                                                                                   \\\hline
	RF-4            & Obbligatorio        & Capitolato     & L'utente deve poter accedere all'applicativo senza bisogno di autenticazione.                                                                                                                                                                                                                                      \\\hline
	RF-5            & Obbligatorio        & Capitolato     & L'utente dovrà poter visualizzare su una mappa la posizione geografica dei sensori.                                                                                                                                                                                                                                \\\hline
	RF-6            & Obbligatorio        & Capitolato     & I tipi di dati che il sistema dovrà visualizzare sono: temperatura, umidità, polveri sottili dell’aria, traffico, lavori in corso, incidenti, parcheggi, lavori su rete idrica, livelli di acqua, posizione colonne di ricarica, guasti elettrici delle colonnine, ponti e strutture critiche, stato delle strade. \\\hline
	RF-7            & Obbligatorio        & Capitolato     & I dati dovranno essere salvati su un database OLAP.                                                                                                                                                                                                                                                                \\\hline
	RF-8            & Obbligatorio        & Capitolato     & I sensori di temperatura rilevano i dati in Celsius                                                                                                                                                                                                                                                                \\\hline
	RF-9            & Obbligatorio        & Capitolato     & I sensori di polveri sottili rilevano le particelle di polveri nell’aria in $\mu g$/mc.                                                                                                                                                                                                                            \\\hline
	RF-10           & Obbligatorio        & Capitolato     & I sensori di umidità rilevano la percentuale di umidità nell’aria.                                                                                                                                                                                                                                                 \\\hline
	RF-11           & Obbligatorio        & Capitolato     & I sensori livello acqua rilevano il livello di acqua nella zona di installazione                                                                                                                                                                                                                                   \\\hline
	RF-12           & Obbligatorio        & Capitolato     & I sensori che indicano interruzioni della fornitura di energia elettrica in una certa zona inviano un segnale binario, dove 0 indica la mancanza di corrente e 1 la presenza di corrente.                                                                                                                          \\\hline
	RF-13           & Obbligatorio        & Capitolato     & I sensori di soglia rilevano lo stato di riempimento dei vari conferitori nelle isole ecologiche inviando un segnale binario, dove 0 indica che il conferitore è vuoto e 1 che è pieno.                                                                                                                            \\\hline
	RF-14           & Obbligatorio        & Capitolato     & I dati provenienti dai sensori dovranno contenere i seguenti dati: id sensore, data, ora e valore.                                                                                                                                                                                                                 \\\hline
	RF-15           & Desiderabile        & Capitolato     & Sviluppo di componenti quali widget e grafici per la visualizzazione dei dati nelle dashboard.                                                                                                                                                                                                                     \\\hline
	% TODO: aggiunta di requisiti interni obbligatori sulla visualizzazione delle dashboard, quali tipi di dashboard, grafici... per ciascuna categoria di sensori
	% TODO: aggiunta di requisiti esterni desiderabili su quali tipi di notifiche inviare all'utente
	% TODO: aggiunta di requisiti interni obbligatori su quali dati manda ciascun tipo sensore implementato e con quale unità di misura

	\caption{Requisiti funzionali}
	\label{table:1}
\end{longtable}

\subsection{Requisiti qualitativi}
\begin{longtable}{|>{\centering\arraybackslash}m{0.10\textwidth}|>{\centering\arraybackslash}m{0.20\textwidth}|>{\centering\arraybackslash}m{0.20\textwidth}|>{\centering\arraybackslash}m{0.4\textwidth}|}
	\hline
	\textbf{Codice} & \textbf{Importanza} & \textbf{Fonte} & \textbf{Descrizione}                                                                                                                                              \\\hline
	\endfirsthead
	\textbf{Codice} & \textbf{Importanza} & \textbf{Fonte} & \textbf{Descrizione}                                                                                                                                              \\\hline
	\endhead
	RQ-16           & Obbligatorio        & Capitolato     & Sviluppo di test che dimostrino il corretto funzionamento dei servizi e delle funzionalità previste. Viene richiesta una copertura dell'80\% corredata di report. \\\hline
	RQ-17           & Obbligatorio        & Capitolato     & Il progetto deve essere corredato di documentazione riguardo scelte implementative e progettuali effettuate e relative motivazioni.                               \\\hline
	RQ-18           & Obbligatorio        & Capitolato     & Il progetto deve essere corredato di documentazione riguardo problemi aperti e eventuali soluzioni proposte da esplorare.                                         \\\hline
	RQ-19           & Obbligatorio        & Capitolato     & Tutte le componenti del sistema devono essere testate con \textit{test end-to-end}.                                                                               \\\hline
	\caption{Requisiti qualitativi}
	\label{table:2}
\end{longtable}

\subsection{Requisiti di vincolo}
\begin{longtable}{|>{\centering\arraybackslash}m{0.10\textwidth}|>{\centering\arraybackslash}m{0.20\textwidth}|>{\centering\arraybackslash}m{0.20\textwidth}|>{\centering\arraybackslash}m{0.4\textwidth}|}
	\hline
	\textbf{Codice} & \textbf{Importanza} & \textbf{Fonte} & \textbf{Descrizione}                                                                                    \\\hline
	\endfirsthead
	\textbf{Codice} & \textbf{Importanza} & \textbf{Fonte} & \textbf{Descrizione}                                                                                    \\\hline
	\endhead
	RQ-20           & Obbligatorio        & Capitolato     & Deve essere implementato almeno un simulatore di dati.                                                  \\\hline
	RQ-21           & Desiderabile        & Capitolato     & Devono essere implementati più simulatori di dati.                                                      \\\hline
	RQ-22           & Obbligatorio        & Capitolato     & I simulatori devono produrre dei dati verosimili.                                                       \\\hline
	RQ-23           & Obbligatorio        & Capitolato     & Il simulatore di dati deve pubblicare messaggi in una piattaforma di \textit{data streaming}.           \\\hline
	RQ-23           & Obbligatorio        & Capitolato     & La piattaforma di \textit{data streaming} deve essere integrata con un un database OLAP.                \\\hline
	RQ-24           & Obbligatorio        & Capitolato     & Per ciascuna tipologia di sensore dev'essere sviluppata almeno una dashboard.                           \\\hline
	RQ-25           & Opzionale           & Capitolato     & Previsione di dati futuri basati sui dati storici.                                                      \\\hline
	RQ-26           & Desiderabile        & Capitolato     & Deve esistere una dashboard per la visualizzazione della posizione geografica dei sensori su una mappa. \\\hline
	RQ-27           & Opzionale           & Capitolato     & Un sistema di notifiche che allerti l'utente in caso di superamento di soglie prestabilite.             \\\hline
	\caption{Requisiti di vincolo}
	\label{table:3}
\end{longtable}


\subsection{Tracciamento}
\subsubsection{Requisito - Fonte}
% comando bash per generare contenuto tabella:
% cat requisiti.tex | awk -F '&' '/R\w\-[[:digit:]]+.*&.*&.*&.*/ {print $1 "&" $3 "\\\\\\hline"}'
\begin{longtable}{|>{\centering\arraybackslash}m{0.10\textwidth}|>{\centering\arraybackslash}m{0.20\textwidth}|}
	\hline
	\textbf{Requisito} & \textbf{Fonte} \\\hline
	\endfirsthead
	\textbf{Requisito} & \textbf{Fonte} \\\hline
	\endhead
	RF-1               & Capitolato     \\\hline
	RF-2               & Capitolato     \\\hline
	RF-3               & Capitolato     \\\hline
	RF-4               & Capitolato     \\\hline
	RF-5               & Capitolato     \\\hline
	RF-6               & Capitolato     \\\hline
	RF-7               & Capitolato     \\\hline
	RF-8               & Capitolato     \\\hline
	RF-9               & Capitolato     \\\hline
	RF-10              & Capitolato     \\\hline
	RF-11              & Capitolato     \\\hline
	RF-12              & Capitolato     \\\hline
	RF-13              & Capitolato     \\\hline
	RF-14              & Capitolato     \\\hline
	RF-15              & Capitolato     \\\hline
	RQ-16              & Capitolato     \\\hline
	RQ-17              & Capitolato     \\\hline
	RQ-18              & Capitolato     \\\hline
	RQ-19              & Capitolato     \\\hline
	RQ-20              & Capitolato     \\\hline
	RQ-21              & Capitolato     \\\hline
	RQ-22              & Capitolato     \\\hline
	RQ-23              & Capitolato     \\\hline
	RQ-23              & Capitolato     \\\hline
	RQ-24              & Capitolato     \\\hline
	RQ-25              & Capitolato     \\\hline
	RQ-26              & Capitolato     \\\hline
	RQ-27              & Capitolato     \\\hline
	\caption{Tracciamento requisito - fonte}
	\label{table:4}
\end{longtable}

\subsection{Riepilogo}
\begin{longtable}{|>{\centering\arraybackslash}m{0.15\textwidth}|>{\centering\arraybackslash}m{0.15\textwidth}|>{\centering\arraybackslash}m{0.15\textwidth}|>{\centering\arraybackslash}m{0.15\textwidth}|>{\centering\arraybackslash}m{0.2\textwidth}|}
	\hline
	\textbf{Tipologia} & \textbf{Obbligatorio} & \textbf{Desiderabile} & \textbf{Opzionale} & \textbf{Totale} \\\hline
	\endfirsthead
	\textbf{Tipologia} & \textbf{Obbligatorio} & \textbf{Desiderabile} & \textbf{Opzionale} & \textbf{Totale} \\\hline
	\endhead
	Funzionali         & 14                    & 1                     & 0                  & 15              \\\hline
	Qualitativi        & 4                     & 0                     & 0                  & 4               \\\hline
	Di vincolo         & 5                     & 2                     & 2                  & 9               \\\hline
	\caption{Riepilogo}
	\label{table:5}
\end{longtable}




