\section{Requisiti}
\subsection{Definizione di un requisito}
Per ciascun requisito vengono fornite le seguenti informazioni:
\begin{itemize}
	\item \textbf{codice} identificativo del requisito, meglio specificato nella sezione \ref{sec:codifica_requisiti};
	\item \textbf{descrizione} del requisito;
	\item \textbf{fonte}, ovvero la provenienza del requisito, meglio specificata nella sezione \ref{sec:fonti_requisiti};
	\item \textbf{importanza} del requisito, meglio specificata nella sezione \ref{sec:importanza_requisiti}.
\end{itemize}

\subsection{Tipologie di requisiti}
I requisiti possono essere di tre tipologie:
\begin{itemize}
	\item \textbf{funzionali}, descrivono le funzionalità del sistema;
	\item \textbf{qualitativi}, descrivono le qualità che il sistema deve avere;
	\item \textbf{di vincolo}, descrivono i vincoli a cui il sistema deve sottostare.
\end{itemize}

\subsubsection{Codifica dei requisiti}
\label{sec:codifica_requisiti}
I requisiti sono codificati nel seguente modo:
\begin{center}
	\textbf{R[Tipologia]-[Codice]}
\end{center}
dove \textbf{[Codice]} è un numero progressivo che identifica univocamente il requisito
e \textbf{[Tipologia]} è una lettera che identifica la tipologia del requisito:
\begin{itemize}
	\item \textbf{F}: requisito funzionale;
	\item \textbf{Q}: requisito qualitativo;
	\item \textbf{V}: requisito di vincolo;
\end{itemize}

\newpage

\subsubsection{Fonti dei requisiti}
\label{sec:fonti_requisiti}
I requisiti provengono dalle fonti meglio specificate di seguito.
\subsubsubsection*{\href{https://7last.github.io/docs/rtb/documentazione-interna/glossario\#capitolato}{Capitolato\textsubscript{G}}}
Requisiti individuati a seguito dell'analisi dello stesso;
\subsubsubsection*{Interno}
Requisiti individuati durante le riunioni interne e da coloro che hanno il ruolo di \href{https://7last.github.io/docs/rtb/documentazione-interna/glossario\#analista}{analista\textsubscript{G}};
\subsubsubsection*{Esterno}
Requisiti individuati in seguito agli incontri tenuti con la \href{https://7last.github.io/docs/rtb/documentazione-interna/glossario\#proponente}{proponente\textsubscript{G}};
\subsubsubsection*{\href{https://7last.github.io/docs/rtb/documentazione-interna/glossario\#piano-di-qualifica}{Piano di Qualifica\textsubscript{G}}}
Requisiti necessari per adeguare il prodotto agli standard di qualità definiti nel documento \href{https://7last.github.io/docs/rtb/documentazione-interna/glossario\#piano-di-qualifica}{\textit{Piano di Qualifica}\textsubscript{G}};
\subsubsubsection*{\href{https://7last.github.io/docs/rtb/documentazione-interna/glossario\#norme-di-progetto}{Norme di Progetto\textsubscript{G}}}
Requisiti necessari per adeguare il prodotto alle norme stabilite nel documento
 \href{https://7last.github.io/docs/rtb/documentazione-interna/glossario\#norme-di-progetto}{\textit{Norme di Progetto}\textsubscript{G}};


\subsubsection{Importanza dei requisiti}
\label{sec:importanza_requisiti}
I requisiti possono avere tre livelli di importanza:
\begin{itemize}
	\item \textbf{Obbligatorio}, requisito irrinunciabile per il \href{https://7last.github.io/docs/rtb/documentazione-interna/glossario\#committente}{committente\textsubscript{G}};
	\item \textbf{Desiderabile}, requisito non strettamente necessario, ma che porta valore aggiunto al prodotto;
	\item \textbf{Opzionale}, requisito relativo a funzionalità aggiuntive.
\end{itemize}

\pagebreak
\subsection{Requisiti funzionali}
\begin{longtable}{|>{\centering\arraybackslash}m{0.10\textwidth}|>{\centering\arraybackslash}m{0.20\textwidth}|>{\centering\arraybackslash}m{0.20\textwidth}|>{\centering\arraybackslash}m{0.4\textwidth}|}
	\hline
	\textbf{Codice} & \textbf{Importanza} & \textbf{Fonte} & \textbf{Descrizione}\\\hline
	\endfirsthead
	\hline
	\textbf{Codice} & \textbf{Importanza} & \textbf{Fonte} & \textbf{Descrizione}\\\hline
	\endhead
	\hline
	RF-1            & Obbligatorio        & \href{https://7last.github.io/docs/rtb/documentazione-interna/glossario\#capitolato}{Capitolato\textsubscript{G}}     & La parte \textit{IoT} dovrà essere simulata attraverso tool di generazione di dati casuali che tuttavia siano verosimili.
	\\\hline
	RF-2            & Obbligatorio        & \href{https://7last.github.io/docs/rtb/documentazione-interna/glossario\#capitolato}{Capitolato\textsubscript{G}}     & Il sistema dovrà permettere la visualizzazione dei dati in tempo reale.
	\\\hline
	RF-3            & Obbligatorio        & \href{https://7last.github.io/docs/rtb/documentazione-interna/glossario\#capitolato}{Capitolato\textsubscript{G}}     & Il sistema dovrà permettere la visualizzazione dei dati storici.
	\\\hline
	RF-4            & Obbligatorio        & \href{https://7last.github.io/docs/rtb/documentazione-interna/glossario\#capitolato}{Capitolato\textsubscript{G}}     & L'utente deve poter accedere all'applicativo senza bisogno di autenticazione.\\\hline
	RF-5            & Obbligatorio        & \href{https://7last.github.io/docs/rtb/documentazione-interna/glossario\#capitolato}{Capitolato\textsubscript{G}}     & L'utente dovrà poter visualizzare su una mappa la posizione geografica dei sensori.
	\\\hline
	RF-6            & Obbligatorio        & \href{https://7last.github.io/docs/rtb/documentazione-interna/glossario\#capitolato}{Capitolato\textsubscript{G}}     & I tipi di dati che il sistema dovrà visualizzare sono: temperatura, umidità, qualità dell'aria, precipitazioni, traffico, stato delle colonnine di ricarica, stato di occupazione dei parcheggi, stato di riempimento delle isole ecologiche e livello di acqua.
	\\\hline
	RF-7            & Obbligatorio        & \href{https://7last.github.io/docs/rtb/documentazione-interna/glossario\#capitolato}{Capitolato\textsubscript{G}}     & I dati dovranno essere salvati su un database OLAP.
	\\\hline
	RF-8            & Obbligatorio        & \href{https://7last.github.io/docs/rtb/documentazione-interna/glossario\#capitolato}{Capitolato\textsubscript{G}}     & I sensori di temperatura rilevano i dati in gradi Celsius
	\\\hline
	RF-9            & Obbligatorio        & \href{https://7last.github.io/docs/rtb/documentazione-interna/glossario\#capitolato}{Capitolato\textsubscript{G}}     & I sensori di umidità rilevano la percentuale di umidità nell’aria.
	\\\hline
	RF-10           & Obbligatorio        & \href{https://7last.github.io/docs/rtb/documentazione-interna/glossario\#capitolato}{Capitolato\textsubscript{G}}     & I sensori livello acqua rilevano il livello di acqua nella zona di installazione
	\\\hline
	RF-11           & Obbligatorio        & \href{https://7last.github.io/docs/rtb/documentazione-interna/glossario\#capitolato}{Capitolato\textsubscript{G}}     & I dati provenienti dai sensori dovranno contenere i seguenti dati: id \href{https://7last.github.io/docs/rtb/documentazione-interna/glossario\#sensore}{sensore\textsubscript{G}}, data, ora e valore.
	\\\hline
	RF-12           & Obbligatorio        & \href{https://7last.github.io/docs/rtb/documentazione-interna/glossario\#capitolato}{Capitolato\textsubscript{G}}     & Sviluppo di componenti quali \href{https://7last.github.io/docs/rtb/documentazione-interna/glossario\#widget}{widget\textsubscript{G}} e grafici per la visualizzazione dei dati nelle \href{https://7last.github.io/docs/rtb/documentazione-interna/glossario\#dashboard}{dashboard\textsubscript{G}}.
	\\\hline
	RF-13           & Obbligatorio        & Interno        & Il sistema deve permettere di visualizzare una \href{https://7last.github.io/docs/rtb/documentazione-interna/glossario\#dashboard}{dashboard\textsubscript{G}} generale con tutti i dati dei sensori.
	\\\hline
	RF-14           & Obbligatorio        & Interno        & Il sistema deve permettere di visualizzare una \href{https://7last.github.io/docs/rtb/documentazione-interna/glossario\#dashboard}{dashboard\textsubscript{G}} specifica per ciascuna categoria di sensori.
	\\\hline
	RF-15           & Obbligatorio        & Esterno        & Il sistema deve permettere di visualizzare una \href{https://7last.github.io/docs/rtb/documentazione-interna/glossario\#dashboard}{dashboard\textsubscript{G}} con i dati grezzi provenienti da tutti i sensori.
	\\\hline
	RF-16           & Obbligatorio        & Interno        & Nella \href{https://7last.github.io/docs/rtb/documentazione-interna/glossario\#dashboard}{dashboard\textsubscript{G}} dei dati grezzi dovranno essere presenti: una mappa interattiva, un \href{https://7last.github.io/docs/rtb/documentazione-interna/glossario\#widget}{widget\textsubscript{G}} con il conteggio totale dei sensori divisi per tipo, una tabella contente tutti i sensori e la data in cui essi hanno trasmesso l'ultima volta. Inoltre verranno mostrate delle tabelle con i dati filtrabili suddivisi per \href{https://7last.github.io/docs/rtb/documentazione-interna/glossario\#sensore}{sensore\textsubscript{G}} e un grafico \href{https://7last.github.io/docs/rtb/documentazione-interna/glossario\#time-series}{time series\textsubscript{G}} con tutti i dati grezzi. 
	\\\hline
	RF-17           & Obbligatorio        & Interno        & Nella \href{https://7last.github.io/docs/rtb/documentazione-interna/glossario\#dashboard}{dashboard\textsubscript{G}} della temperatura dovranno essere visualizzati: un grafico \href{https://7last.github.io/docs/rtb/documentazione-interna/glossario\#time-series}{time series\textsubscript{G}}, una mappa interattiva, la temperatura media, minima e massima di un certo periodo di tempo, la temperatura in tempo reale e la temperatura media per settimana e mese.
	\\\hline
	RF-18           & Obbligatorio        & Interno        & Nella \href{https://7last.github.io/docs/rtb/documentazione-interna/glossario\#dashboard}{dashboard\textsubscript{G}} dell'umidità dovranno essere visualizzati: un grafico \href{https://7last.github.io/docs/rtb/documentazione-interna/glossario\#time-series}{time series\textsubscript{G}}, una mappa interattiva, l'umidità media, minima e massima di un certo periodo di tempo e l'umidità in tempo reale.
	\\\hline
	RF-19           & Obbligatorio        & Interno        & Nella \href{https://7last.github.io/docs/rtb/documentazione-interna/glossario\#dashboard}{dashboard\textsubscript{G}} della qualità dell'aria dovranno essere visualizzati: un grafico \href{https://7last.github.io/docs/rtb/documentazione-interna/glossario\#time-series}{time series\textsubscript{G}}, una mappa interattiva, la qualità media dell'aria in un certo periodo e in tempo reale, i giorni con la qualità dell'aria migliore e peggiore in un certo periodo di tempo.
	\\\hline
	RF-20           & Obbligatorio        & Interno        & Nella \href{https://7last.github.io/docs/rtb/documentazione-interna/glossario\#dashboard}{dashboard\textsubscript{G}} delle precipitazioni dovranno essere visualizzati: un grafico \href{https://7last.github.io/docs/rtb/documentazione-interna/glossario\#time-series}{time series\textsubscript{G}}, una mappa interattiva, la quantità media di precipitazioni in un certo periodo e in tempo reale, i giorni con la quantità di precipitazioni maggiore e minore in un certo periodo di tempo.
	\\\hline
	RF-21           & Obbligatorio        & Interno        & Nella \href{https://7last.github.io/docs/rtb/documentazione-interna/glossario\#dashboard}{dashboard\textsubscript{G}} del traffico dovranno essere visualizzati: un grafico \href{https://7last.github.io/docs/rtb/documentazione-interna/glossario\#time-series}{time series\textsubscript{G}}, il numero di veicoli e la velocità media in tempo reale, il calcolo dell'ora di punta sulla base del numero di veicoli e velocità media.
	\\\hline
	RF-22           & Obbligatorio        & Interno        & Nella \href{https://7last.github.io/docs/rtb/documentazione-interna/glossario\#dashboard}{dashboard\textsubscript{G}} delle colonnine di ricarica dovranno essere visualizzati: una mappa interattiva contenente anche lo stato e il numero di colonnine di ricarica suddivise per stato in tempo reale.
	\\\hline
	RF-23           & Obbligatorio        & Interno        & Nella \href{https://7last.github.io/docs/rtb/documentazione-interna/glossario\#dashboard}{dashboard\textsubscript{G}} dei parcheggi dovranno essere visualizzati: una mappa interattiva con il rispettivo stato di occupazione e il conteggio di parcheggi suddivisi per stato di occupazione in tempo reale.
	\\\hline
	RF-24           & Obbligatorio        & Interno        & Nella \href{https://7last.github.io/docs/rtb/documentazione-interna/glossario\#dashboard}{dashboard\textsubscript{G}} delle isole ecologiche dovranno essere visualizzati: una mappa interattiva con il rispettivo stato di riempimento e il conteggio di isole ecologiche suddivise per stato di riempimento in tempo reale.
	\\\hline
	RF-25           & Obbligatorio        & Interno        & Nella \href{https://7last.github.io/docs/rtb/documentazione-interna/glossario\#dashboard}{dashboard\textsubscript{G}} del livello di acqua dovranno essere visualizzati: un grafico \href{https://7last.github.io/docs/rtb/documentazione-interna/glossario\#time-series}{time series\textsubscript{G}}, una mappa interattiva, il livello medio di acqua in un certo periodo e in tempo reale.
	\\\hline
	RF-26           & Obbligatorio        & Interno        & Nel caso in cui non ci siano dati visualizzabili, il sistema deve notificare l'utente mostrando un opportuno messaggio.
	\\\hline
	RF-27           & Obbligatorio        & Interno        & I sensori di qualità dell'aria inviano i seguenti dati: \textit{PM10}, \textit{PM2.5}, \textit{NO2}, \textit{CO}, \textit{O3}, \textit{SO2} in $\mu g/m^3$ e la qualità dell'aria in base all'indice \href{https://7last.github.io/docs/rtb/documentazione-interna/glossario\#european-air-quality-index}{\textit{EAQI}\textsubscript{G}}.                                       \\\hline
	RF-28           & Obbligatorio        & Interno        & I sensori di precipitazioni inviano la quantità di pioggia caduta in mm.
	\\\hline
	RF-29           & Obbligatorio        & Interno        & I sensori di traffico inviano il numero di veicoli rilevati e la velocità in km/h.
	\\\hline
	RF-30           & Obbligatorio        & Interno        & Le colonnine di ricarica inviano lo stato di occupazione e il tempo mancante alla fine della ricarica (se occupate) o il tempo passato dalla fine dell'ultima ricarica (se libere).
	\\\hline
	RF-31           & Obbligatorio        & Interno        & I sensori di parcheggio inviano lo stato di occupazione del parcheggio (1 se occupato, 0 se libero) e il timestamp dell'ultimo cambiamento di stato.
	\\\hline
	RF-32           & Obbligatorio        & Interno        & Le isole ecologiche inviano lo stato di riempimento come percentuale.
	\\\hline
	RF-33           & Obbligatorio        & Interno        & I sensori di livello di acqua inviano il livello di acqua in cm.
	\\\hline
	RF-34           & Obbligatorio        & Esterno        & Il sistema deve permettere di filtrare i dati visualizzati in base a un intervallo di tempo.
	\\\hline
	RF-35           & Obbligatorio        & Esterno        & Il sistema deve permettere di filtrare i dati visualizzati in base al \href{https://7last.github.io/docs/rtb/documentazione-interna/glossario\#sensore}{sensore\textsubscript{G}} che li ha generati.
	\\\hline
	RF-36           & Desiderabile        & Esterno        & Devono essere messe in relazione più sorgenti di dati.
	\\\hline
	RQ-37           & Opzionale           & Esterno        & Utilizzo di uno \href{https://7last.github.io/docs/rtb/documentazione-interna/glossario\#schema-registry}{schema registry\textsubscript{G}} per la pubblicazione dei dati sui \href{https://7last.github.io/docs/rtb/documentazione-interna/glossario\#topic}{topic\textsubscript{G}}, per garantire la correttezza dei dati.
	\\\hline
	RQ-38           & Desiderabile        & Esterno        & Devono essere calcolati dei \href{https://7last.github.io/docs/rtb/documentazione-interna/glossario\#key-performance-indicator}{KPI\textsubscript{G}} per ogni categoria di sensori, che rappresentano la qualità di un servizio fornito ai cittadini o delle condizioni della città.
	\\\hline
	RQ-39           & Desiderabile        & Esterno        & Nei grafici \href{https://7last.github.io/docs/rtb/documentazione-interna/glossario\#time-series}{time series\textsubscript{G}} i dati devono essere aggregati calcolando la media di 5 minuti, in modo da risultare più leggibili.
	\\\hline
	\caption{Requisiti funzionali}
\end{longtable}

\subsection{Requisiti qualitativi}
\begin{longtable}{|>{\centering\arraybackslash}m{0.10\textwidth}|>{\centering\arraybackslash}m{0.20\textwidth}|>{\centering\arraybackslash}m{0.20\textwidth}|>{\centering\arraybackslash}m{0.4\textwidth}|}
	\hline
	\textbf{Codice} & \textbf{Importanza} & \textbf{Fonte} & \textbf{Descrizione}
	\\\hline
	\endfirsthead
	RQ-40			& Obbligatorio        & \href{https://7last.github.io/docs/rtb/documentazione-interna/glossario\#capitolato}{Capitolato\textsubscript{G}}, \href{https://7last.github.io/docs/rtb/documentazione-interna/glossario\#piano-di-qualifica}{Piano di Qualifica\textsubscript{G}} & Sviluppo di test che dimostrino il corretto funzionamento dei servizi e delle funzionalità previste. Viene richiesta una copertura dell'80\% corredata di report.
	\\\hline
	RQ-41           & Obbligatorio        & \href{https://7last.github.io/docs/rtb/documentazione-interna/glossario\#capitolato}{Capitolato\textsubscript{G}}, \href{https://7last.github.io/docs/rtb/documentazione-interna/glossario\#piano-di-qualifica}{Piano di Qualifica\textsubscript{G}} & Il progetto deve essere corredato di documentazione riguardo scelte implementative e progettuali effettuate e relative motivazioni.
	\\\hline
	RQ-42           & Obbligatorio        & \href{https://7last.github.io/docs/rtb/documentazione-interna/glossario\#capitolato}{Capitolato\textsubscript{G}}, \href{https://7last.github.io/docs/rtb/documentazione-interna/glossario\#piano-di-qualifica}{Piano di Qualifica\textsubscript{G}} & Il progetto deve essere corredato di documentazione riguardo problemi aperti e eventuali soluzioni proposte da esplorare.
	\\\hline
	RQ-43           & Obbligatorio        & \href{https://7last.github.io/docs/rtb/documentazione-interna/glossario\#capitolato}{Capitolato\textsubscript{G}}, \href{https://7last.github.io/docs/rtb/documentazione-interna/glossario\#piano-di-qualifica}{Piano di Qualifica\textsubscript{G}} & Tutte le componenti del sistema devono essere testate con \href{https://7last.github.io/docs/rtb/documentazione-interna/glossario\#test-end-to-end}{\textit{test end-to-end}\textsubscript{G}}. \\\hline
	\caption{Requisiti qualitativi}
\end{longtable}

\subsection{Requisiti di vincolo}
\begin{longtable}{|>{\centering\arraybackslash}m{0.10\textwidth}|>{\centering\arraybackslash}m{0.20\textwidth}|>{\centering\arraybackslash}m{0.20\textwidth}|>{\centering\arraybackslash}m{0.4\textwidth}|}
	\hline
	\textbf{Codice} & \textbf{Importanza} & \textbf{Fonte} & \textbf{Descrizione}
	\\\hline
	\endfirsthead
	\hline
	\endhead
	RV-44           & Obbligatorio        & \href{https://7last.github.io/docs/rtb/documentazione-interna/glossario\#capitolato}{Capitolato\textsubscript{G}}     & Deve essere implementato almeno un simulatore di dati.
	\\\hline
	RV-45           & Desiderabile        & \href{https://7last.github.io/docs/rtb/documentazione-interna/glossario\#capitolato}{Capitolato\textsubscript{G}}     & Devono essere implementati più simulatori di dati.
	\\\hline
	RV-46           & Obbligatorio        & \href{https://7last.github.io/docs/rtb/documentazione-interna/glossario\#capitolato}{Capitolato\textsubscript{G}}     & I simulatori devono produrre dei dati verosimili.                                                       \\\hline
	RV-47           & Obbligatorio        & \href{https://7last.github.io/docs/rtb/documentazione-interna/glossario\#capitolato}{Capitolato\textsubscript{G}}     & Il simulatore di dati deve pubblicare messaggi in una piattaforma di \textit{data streaming}.
	\\\hline
	RV-48           & Obbligatorio        & \href{https://7last.github.io/docs/rtb/documentazione-interna/glossario\#capitolato}{Capitolato\textsubscript{G}}     & La piattaforma di \textit{data streaming} deve essere integrata con un un database OLAP.
	\\\hline
	RV-49           & Obbligatorio        & \href{https://7last.github.io/docs/rtb/documentazione-interna/glossario\#capitolato}{Capitolato\textsubscript{G}}     & Per ciascuna tipologia di \href{https://7last.github.io/docs/rtb/documentazione-interna/glossario\#sensore}{sensore\textsubscript{G}} dev'essere sviluppata almeno una \href{https://7last.github.io/docs/rtb/documentazione-interna/glossario\#dashboard}{dashboard\textsubscript{G}}.
	\\\hline
	RV-50           & Opzionale           & \href{https://7last.github.io/docs/rtb/documentazione-interna/glossario\#capitolato}{Capitolato\textsubscript{G}}     & Previsione di dati futuri basati sui dati storici.                                                      \\\hline
	RV-51           & Desiderabile        & \href{https://7last.github.io/docs/rtb/documentazione-interna/glossario\#capitolato}{Capitolato\textsubscript{G}}     & Deve esistere una \href{https://7last.github.io/docs/rtb/documentazione-interna/glossario\#dashboard}{dashboard\textsubscript{G}} per la visualizzazione della posizione geografica dei sensori su una mappa
	\\\hline
	RV-52           & Opzionale           & \href{https://7last.github.io/docs/rtb/documentazione-interna/glossario\#capitolato}{Capitolato\textsubscript{G}}     & Un sistema di notifiche che allerti l'utente in caso di superamento di soglie prestabilite.
	\\\hline
	\caption{Requisiti di vincolo}
\end{longtable}

\subsection{Tracciamento}
\subsubsection{Requisito - Fonte}
% comando bash per generare contenuto tabella:
% cat requisiti.tex | awk -F '&' '/R\w\-[[:digit:]]+.*&.*&.*&.*/ {print $1 "&" $3 "\\\\\\hline"}'
\begin{longtable}{|>{\centering\arraybackslash}m{0.30\textwidth}|>{\centering\arraybackslash}m{0.40\textwidth}|}
	\hline
	\textbf{Requisito} & \textbf{Fonte}                 \\\hline
	\endfirsthead
	\hline
	\textbf{Requisito} & \textbf{Fonte}                 \\\hline
	\endhead
	RF-1               & \href{https://7last.github.io/docs/rtb/documentazione-interna/glossario\#capitolato}{Capitolato\textsubscript{G}}                     \\\hline
	RF-2               & \href{https://7last.github.io/docs/rtb/documentazione-interna/glossario\#capitolato}{Capitolato\textsubscript{G}}                     \\\hline
	RF-3               & \href{https://7last.github.io/docs/rtb/documentazione-interna/glossario\#capitolato}{Capitolato\textsubscript{G}}                     \\\hline
	RF-4               & \href{https://7last.github.io/docs/rtb/documentazione-interna/glossario\#capitolato}{Capitolato\textsubscript{G}}                     \\\hline
	RF-5               & \href{https://7last.github.io/docs/rtb/documentazione-interna/glossario\#capitolato}{Capitolato\textsubscript{G}}                     \\\hline
	RF-6               & \href{https://7last.github.io/docs/rtb/documentazione-interna/glossario\#capitolato}{Capitolato\textsubscript{G}}                     \\\hline
	RF-7               & \href{https://7last.github.io/docs/rtb/documentazione-interna/glossario\#capitolato}{Capitolato\textsubscript{G}}                     \\\hline
	RF-8               & \href{https://7last.github.io/docs/rtb/documentazione-interna/glossario\#capitolato}{Capitolato\textsubscript{G}}                     \\\hline
	RF-9               & \href{https://7last.github.io/docs/rtb/documentazione-interna/glossario\#capitolato}{Capitolato\textsubscript{G}}                     \\\hline
	RF-10              & \href{https://7last.github.io/docs/rtb/documentazione-interna/glossario\#capitolato}{Capitolato\textsubscript{G}}                     \\\hline
	RF-11              & \href{https://7last.github.io/docs/rtb/documentazione-interna/glossario\#capitolato}{Capitolato\textsubscript{G}}                     \\\hline
	RF-12              & \href{https://7last.github.io/docs/rtb/documentazione-interna/glossario\#capitolato}{Capitolato\textsubscript{G}}                     \\\hline
	RF-13              & Interno                        \\\hline
	RF-14              & Interno                        \\\hline
	RF-15              & Esterno                        \\\hline
	RF-16              & Interno                        \\\hline
	RF-17              & Interno                        \\\hline
	RF-18              & Interno                        \\\hline
	RF-19              & Interno                        \\\hline
	RF-20              & Interno                        \\\hline
	RF-21              & Interno                        \\\hline
	RF-22              & Interno                        \\\hline
	RF-23              & Interno                        \\\hline
	RF-24              & Interno                        \\\hline
	RF-25              & Interno                        \\\hline
	RF-26              & Interno                        \\\hline
	RF-27              & Interno                        \\\hline
	RF-28              & Interno                        \\\hline
	RF-29              & Interno                        \\\hline
	RF-30              & Interno                        \\\hline
	RF-31              & Interno                        \\\hline
	RF-32              & Interno                        \\\hline
	RF-33              & Interno                        \\\hline
	RF-34              & Esterno                        \\\hline
	RF-35              & Esterno                        \\\hline
	RF-36              & Esterno                        \\\hline
	RQ-37              & Esterno                        \\\hline
	RQ-38              & Esterno                        \\\hline
	RQ-39              & Esterno                        \\\hline
	RQ-40              & \href{https://7last.github.io/docs/rtb/documentazione-interna/glossario\#capitolato}{Capitolato\textsubscript{G}}, \href{https://7last.github.io/docs/rtb/documentazione-interna/glossario\#piano-di-qualifica}{Piano di Qualifica\textsubscript{G}} \\\hline
	RQ-41              & \href{https://7last.github.io/docs/rtb/documentazione-interna/glossario\#capitolato}{Capitolato\textsubscript{G}}, \href{https://7last.github.io/docs/rtb/documentazione-interna/glossario\#piano-di-qualifica}{Piano di Qualifica\textsubscript{G}} \\\hline
	RQ-42              & \href{https://7last.github.io/docs/rtb/documentazione-interna/glossario\#capitolato}{Capitolato\textsubscript{G}}, \href{https://7last.github.io/docs/rtb/documentazione-interna/glossario\#piano-di-qualifica}{Piano di Qualifica\textsubscript{G}} \\\hline
	RQ-43              & \href{https://7last.github.io/docs/rtb/documentazione-interna/glossario\#capitolato}{Capitolato\textsubscript{G}}, \href{https://7last.github.io/docs/rtb/documentazione-interna/glossario\#piano-di-qualifica}{Piano di Qualifica\textsubscript{G}} \\\hline
	RV-44              & \href{https://7last.github.io/docs/rtb/documentazione-interna/glossario\#capitolato}{Capitolato\textsubscript{G}}                     \\\hline
	RV-45              & \href{https://7last.github.io/docs/rtb/documentazione-interna/glossario\#capitolato}{Capitolato\textsubscript{G}}                     \\\hline
	RV-46              & \href{https://7last.github.io/docs/rtb/documentazione-interna/glossario\#capitolato}{Capitolato\textsubscript{G}}                     \\\hline
	RV-47              & \href{https://7last.github.io/docs/rtb/documentazione-interna/glossario\#capitolato}{Capitolato\textsubscript{G}}                     \\\hline
	RV-48              & \href{https://7last.github.io/docs/rtb/documentazione-interna/glossario\#capitolato}{Capitolato\textsubscript{G}}                     \\\hline
	RV-49              & \href{https://7last.github.io/docs/rtb/documentazione-interna/glossario\#capitolato}{Capitolato\textsubscript{G}}                     \\\hline
	RV-50              & \href{https://7last.github.io/docs/rtb/documentazione-interna/glossario\#capitolato}{Capitolato\textsubscript{G}}                     \\\hline
	RV-51              & \href{https://7last.github.io/docs/rtb/documentazione-interna/glossario\#capitolato}{Capitolato\textsubscript{G}}                     \\\hline
	RV-52              & \href{https://7last.github.io/docs/rtb/documentazione-interna/glossario\#capitolato}{Capitolato\textsubscript{G}}                     \\\hline
	\caption{Tracciamento requisito - fonte}
\end{longtable}

\pagebreak
\subsection{Riepilogo}
\begin{longtable}{|>{\centering\arraybackslash}m{0.15\textwidth}|>{\centering\arraybackslash}m{0.15\textwidth}|>{\centering\arraybackslash}m{0.15\textwidth}|>{\centering\arraybackslash}m{0.15\textwidth}|>{\centering\arraybackslash}m{0.2\textwidth}|}
	\hline
	\textbf{Tipologia} & \textbf{Obbligatorio} & \textbf{Desiderabile} & \textbf{Opzionale} & \textbf{Totale} \\\hline
	\endfirsthead
	\textbf{Tipologia} & \textbf{Obbligatorio} & \textbf{Desiderabile} & \textbf{Opzionale} & \textbf{Totale} \\\hline
	\endhead
	Funzionali         & 35                    & 3                     & 1                  & 39              \\\hline
	Qualitativi        & 4                     & 0                     & 0                  & 4               \\\hline
	Di vincolo         & 5                     & 2                     & 2                  & 9               \\\hline
	\caption{Riepilogo}
\end{longtable}

