\section{Requisiti}
\subsection{Definizione di un requisito}
Per ciascun requisito vengono fornite le seguenti informazioni:
\begin{itemize}
	\item \textbf{codice} identificativo del requisito, meglio specificato nella sezione \ref{sec:codifica_requisiti};
	\item \textbf{descrizione} del requisito;
	\item \textbf{fonte}, ovvero la provenienza del requisito, meglio specificata nella sezione \ref{sec:fonti_requisiti};
	\item \textbf{importanza} del requisito, meglio specificata nella sezione \ref{sec:importanza_requisiti}.
\end{itemize}

\subsection{Tipologie di requisiti}
I requisiti possono essere di quattro tipologie:
\begin{itemize}
	\item \textbf{funzionali}, descrivono le funzionalità del sistema;
	\item \textbf{qualitativi}, descrivono le qualità che il sistema deve avere;
	\item \textbf{di vincolo}, descrivono i vincoli a cui il sistema deve sottostare.
	\item \textbf{prestazionali}, descrivono le prestazioni che il sistema deve avere.
\end{itemize}

\subsubsection{Codifica dei requisiti}
\label{sec:codifica_requisiti}
I requisiti sono codificati nel seguente modo:
\begin{center}
	\textbf{R[Tipologia]-[Codice]}
\end{center}
dove \textbf{[Codice]} è un numero progressivo che identifica univocamente il requisito
e \textbf{[Tipologia]} è una lettera che identifica la tipologia del requisito:
\begin{itemize}
	\item \textbf{F}: requisito funzionale;
	\item \textbf{Q}: requisito qualitativo;
	\item \textbf{V}: requisito di vincolo;
	\item \textbf{P}: requisito prestazionale.
\end{itemize}

\newpage

\subsubsection{Fonti dei requisiti}
\label{sec:fonti_requisiti}
I requisiti provengono dalle fonti meglio specificate di seguito.
\subsubsubsection*{\href{https://7last.github.io/docs/rtb/documentazione-interna/glossario\#capitolato}{Capitolato\textsubscript{G}}}
Requisiti individuati a seguito dell'analisi dello stesso;
\subsubsubsection*{Interno}
Requisiti individuati durante le riunioni interne e da coloro che hanno il ruolo di \href{https://7last.github.io/docs/rtb/documentazione-interna/glossario\#analista}{analista\textsubscript{G}};
\subsubsubsection*{Esterno}
Requisiti individuati in seguito agli incontri tenuti con la \href{https://7last.github.io/docs/rtb/documentazione-interna/glossario\#proponente}{proponente\textsubscript{G}};
\subsubsubsection*{\href{https://7last.github.io/docs/rtb/documentazione-interna/glossario\#piano-di-qualifica}{Piano di Qualifica\textsubscript{G}}}
Requisiti necessari per adeguare il prodotto agli standard di qualità definiti nel documento \href{https://7last.github.io/docs/rtb/documentazione-interna/glossario\#piano-di-qualifica}{\textit{Piano di Qualifica}\textsubscript{G}};
\subsubsubsection*{\href{https://7last.github.io/docs/rtb/documentazione-interna/glossario\#norme-di-progetto}{Norme di Progetto\textsubscript{G}}}
Requisiti necessari per adeguare il prodotto alle norme stabilite nel documento
\href{https://7last.github.io/docs/rtb/documentazione-interna/glossario\#norme-di-progetto}{\textit{Norme di Progetto}\textsubscript{G}};


\subsubsection{Importanza dei requisiti}
\label{sec:importanza_requisiti}
I requisiti possono avere tre livelli di importanza:
\begin{itemize}
	\item \textbf{Obbligatorio}, requisito irrinunciabile per il \href{https://7last.github.io/docs/rtb/documentazione-interna/glossario\#committente}{committente\textsubscript{G}};
	\item \textbf{Desiderabile}, requisito non strettamente necessario, ma che porta valore aggiunto al prodotto;
	\item \textbf{Opzionale}, requisito relativo a funzionalità aggiuntive.
\end{itemize}

\pagebreak
\subsection{Requisiti funzionali}
\begin{longtable}{|>{\centering\arraybackslash}m{0.10\textwidth}|>{\centering\arraybackslash}m{0.20\textwidth}|>{\centering\arraybackslash}m{0.20\textwidth}|>{\centering\arraybackslash}m{0.4\textwidth}|}
	\hline
	\textbf{Codice} & \textbf{Importanza} & \textbf{Fonte}                                                                                                    & \textbf{Descrizione}                                                                                                                                                                                                                                                                                                                                                                                                                                                                                                                                                                                                                                                                                                                                                                                  \\\hline
	\endfirsthead
	\hline
	\textbf{Codice} & \textbf{Importanza} & \textbf{Fonte}                                                                                                    & \textbf{Descrizione}                                                                                                                                                                                                                                                                                                                                                                                                                                                                                                                                                                                                                                                                                                                                                                                  \\\hline
	\endhead
	\hline
	RF-1            & Obbligatorio        & \href{https://7last.github.io/docs/rtb/documentazione-interna/glossario\#capitolato}{Capitolato\textsubscript{G}} & La parte \textit{IoT} dovrà essere simulata attraverso tool di generazione di dati casuali che tuttavia siano verosimili.
	\\\hline
	RF-2            & Obbligatorio        & \href{https://7last.github.io/docs/rtb/documentazione-interna/glossario\#capitolato}{Capitolato\textsubscript{G}} & Il sistema dovrà permettere la visualizzazione dei dati in tempo reale.
	\\\hline
	RF-3            & Obbligatorio        & \href{https://7last.github.io/docs/rtb/documentazione-interna/glossario\#capitolato}{Capitolato\textsubscript{G}} & Il sistema dovrà permettere la visualizzazione dei dati storici.
	\\\hline
	RF-4            & Obbligatorio        & \href{https://7last.github.io/docs/rtb/documentazione-interna/glossario\#capitolato}{Capitolato\textsubscript{G}} & L'utente deve poter accedere all'applicativo senza bisogno di autenticazione.
	\\\hline
	RF-5            & Obbligatorio        & \href{https://7last.github.io/docs/rtb/documentazione-interna/glossario\#capitolato}{Capitolato\textsubscript{G}} & L'utente dovrà poter visualizzare su una mappa la posizione geografica dei sensori.
	\\\hline
	RF-6            & Obbligatorio        & \href{https://7last.github.io/docs/rtb/documentazione-interna/glossario\#capitolato}{Capitolato\textsubscript{G}} & I tipi di dati che il sistema dovrà visualizzare sono: temperatura, umidità, qualità dell'aria, precipitazioni, traffico, stato delle colonnine di ricarica, stato di occupazione dei parcheggi, stato di riempimento delle isole ecologiche e livello di acqua.
	\\\hline
	RF-7            & Obbligatorio        & \href{https://7last.github.io/docs/rtb/documentazione-interna/glossario\#capitolato}{Capitolato\textsubscript{G}} & I dati dovranno essere salvati su un database OLAP.
	\\\hline
	RF-8            & Obbligatorio        & \href{https://7last.github.io/docs/rtb/documentazione-interna/glossario\#capitolato}{Capitolato\textsubscript{G}} & I sensori di temperatura rilevano i dati in gradi Celsius
	\\\hline
	RF-9            & Obbligatorio        & \href{https://7last.github.io/docs/rtb/documentazione-interna/glossario\#capitolato}{Capitolato\textsubscript{G}} & I sensori di umidità rilevano la percentuale di umidità nell’aria.
	\\\hline
	RF-10           & Obbligatorio        & \href{https://7last.github.io/docs/rtb/documentazione-interna/glossario\#capitolato}{Capitolato\textsubscript{G}} & I sensori livello acqua rilevano il livello di acqua nella zona di installazione
	\\\hline
	RF-11           & Obbligatorio        & \href{https://7last.github.io/docs/rtb/documentazione-interna/glossario\#capitolato}{Capitolato\textsubscript{G}} & I dati provenienti dai sensori dovranno contenere i seguenti dati: id \href{https://7last.github.io/docs/rtb/documentazione-interna/glossario\#sensore}{sensore\textsubscript{G}}, data, ora e valore.
	\\\hline
	RF-12           & Obbligatorio        & \href{https://7last.github.io/docs/rtb/documentazione-interna/glossario\#capitolato}{Capitolato\textsubscript{G}} & Sviluppo di componenti quali \href{https://7last.github.io/docs/rtb/documentazione-interna/glossario\#widget}{widget\textsubscript{G}} e grafici per la visualizzazione dei dati nelle \href{https://7last.github.io/docs/rtb/documentazione-interna/glossario\#dashboard}{dashboard\textsubscript{G}}.
	\\\hline
	RF-13           & Obbligatorio        & Interno                                                                                                           & Il sistema deve permettere di visualizzare una \href{https://7last.github.io/docs/rtb/documentazione-interna/glossario\#dashboard}{dashboard\textsubscript{G}} generale con tutti i dati dei sensori.
	\\\hline
	RF-14           & Obbligatorio        & Interno                                                                                                           & Il sistema deve permettere di visualizzare una \href{https://7last.github.io/docs/rtb/documentazione-interna/glossario\#dashboard}{dashboard\textsubscript{G}} specifica per ciascuna categoria di sensori.
	\\\hline
	RF-15           & Obbligatorio        & Interno                                                                                                           & Nella \href{https://7last.github.io/docs/rtb/documentazione-interna/glossario\#dashboard}{dashboard\textsubscript{G}} dei dati grezzi dovranno essere presenti: una mappa interattiva, un \href{https://7last.github.io/docs/rtb/documentazione-interna/glossario\#widget}{widget\textsubscript{G}} con il conteggio totale dei sensori divisi per tipo, una tabella contente tutti i sensori e la data in cui essi hanno trasmesso l'ultima volta. Inoltre verranno mostrate delle tabelle con i dati filtrabili suddivisi per \href{https://7last.github.io/docs/rtb/documentazione-interna/glossario\#sensore}{sensore\textsubscript{G}} e un grafico \href{https://7last.github.io/docs/rtb/documentazione-interna/glossario\#time-series}{time series\textsubscript{G}} con tutti i dati grezzi.
	\\\hline
	RF-16           & Obbligatorio        & Interno                                                                                                           & Nella \href{https://7last.github.io/docs/rtb/documentazione-interna/glossario\#dashboard}{dashboard\textsubscript{G}} della temperatura dovranno essere visualizzati: un grafico \href{https://7last.github.io/docs/rtb/documentazione-interna/glossario\#time-series}{time series\textsubscript{G}}, una mappa interattiva, la temperatura media, minima e massima di un certo periodo di tempo, la temperatura in tempo reale e la temperatura media per settimana e mese.
	\\\hline
	RF-17           & Obbligatorio        & Interno                                                                                                           & Nella \href{https://7last.github.io/docs/rtb/documentazione-interna/glossario\#dashboard}{dashboard\textsubscript{G}} dell'umidità dovranno essere visualizzati: un grafico \href{https://7last.github.io/docs/rtb/documentazione-interna/glossario\#time-series}{time series\textsubscript{G}}, una mappa interattiva, l'umidità media, minima e massima di un certo periodo di tempo e l'umidità in tempo reale.
	\\\hline
	RF-18           & Obbligatorio        & Interno                                                                                                           & Nella \href{https://7last.github.io/docs/rtb/documentazione-interna/glossario\#dashboard}{dashboard\textsubscript{G}} della qualità dell'aria dovranno essere visualizzati: un grafico \href{https://7last.github.io/docs/rtb/documentazione-interna/glossario\#time-series}{time series\textsubscript{G}}, una mappa interattiva, la qualità media dell'aria in un certo periodo e in tempo reale, i giorni con la qualità dell'aria migliore e peggiore in un certo periodo di tempo.
	\\\hline
	RF-19           & Obbligatorio        & Interno                                                                                                           & Nella \href{https://7last.github.io/docs/rtb/documentazione-interna/glossario\#dashboard}{dashboard\textsubscript{G}} delle precipitazioni dovranno essere visualizzati: un grafico \href{https://7last.github.io/docs/rtb/documentazione-interna/glossario\#time-series}{time series\textsubscript{G}}, una mappa interattiva, la quantità media di precipitazioni in un certo periodo e in tempo reale, i giorni con la quantità di precipitazioni maggiore e minore in un certo periodo di tempo.
	\\\hline
	RF-20           & Obbligatorio        & Interno                                                                                                           & Nella \href{https://7last.github.io/docs/rtb/documentazione-interna/glossario\#dashboard}{dashboard\textsubscript{G}} del traffico dovranno essere visualizzati: un grafico \href{https://7last.github.io/docs/rtb/documentazione-interna/glossario\#time-series}{time series\textsubscript{G}}, il numero di veicoli e la velocità media in tempo reale, il calcolo dell'ora di punta sulla base del numero di veicoli e velocità media.
	\\\hline
	RF-21           & Obbligatorio        & Interno                                                                                                           & Nella \href{https://7last.github.io/docs/rtb/documentazione-interna/glossario\#dashboard}{dashboard\textsubscript{G}} delle colonnine di ricarica dovranno essere visualizzati: una mappa interattiva contenente anche lo stato e il numero di colonnine di ricarica suddivise per stato in tempo reale.
	\\\hline
	RF-22           & Obbligatorio        & Interno                                                                                                           & Nella \href{https://7last.github.io/docs/rtb/documentazione-interna/glossario\#dashboard}{dashboard\textsubscript{G}} dei parcheggi dovranno essere visualizzati: una mappa interattiva con il rispettivo stato di occupazione e il conteggio di parcheggi suddivisi per stato di occupazione in tempo reale.
	\\\hline
	RF-23           & Obbligatorio        & Interno                                                                                                           & Nella \href{https://7last.github.io/docs/rtb/documentazione-interna/glossario\#dashboard}{dashboard\textsubscript{G}} delle isole ecologiche dovranno essere visualizzati: una mappa interattiva con il rispettivo stato di riempimento e il conteggio di isole ecologiche suddivise per stato di riempimento in tempo reale.
	\\\hline
	RF-24           & Obbligatorio        & Interno                                                                                                           & Nella \href{https://7last.github.io/docs/rtb/documentazione-interna/glossario\#dashboard}{dashboard\textsubscript{G}} del livello di acqua dovranno essere visualizzati: un grafico \href{https://7last.github.io/docs/rtb/documentazione-interna/glossario\#time-series}{time series\textsubscript{G}}, una mappa interattiva, il livello medio di acqua in un certo periodo e in tempo reale.
	\\\hline
	RF-25           & Obbligatorio        & Interno                                                                                                           & Nel caso in cui non ci siano dati visualizzabili, il sistema deve notificare l'utente mostrando un opportuno messaggio.
	\\\hline
	RF-26           & Obbligatorio        & Interno                                                                                                           & I sensori di qualità dell'aria inviano i seguenti dati: \textit{PM10}, \textit{PM2.5}, \textit{NO2}, \textit{CO}, \textit{O3}, \textit{SO2} in $\mu g/m^3$.
	\\\hline
	RF-27           & Obbligatorio        & Interno                                                                                                           & I sensori di precipitazioni inviano la quantità di pioggia caduta in mm.
	\\\hline
	RF-28           & Obbligatorio        & Interno                                                                                                           & I sensori di traffico inviano il numero di veicoli rilevati e la velocità in km/h.
	\\\hline
	RF-29           & Obbligatorio        & Interno                                                                                                           & Le colonnine di ricarica inviano lo stato di occupazione e il tempo mancante alla fine della ricarica (se occupate) o il tempo passato dalla fine dell'ultima ricarica (se libere).
	\\\hline
	RF-30           & Obbligatorio        & Interno                                                                                                           & I sensori di parcheggio inviano lo stato di occupazione del parcheggio (1 se occupato, 0 se libero) e il timestamp dell'ultimo cambiamento di stato.
	\\\hline
	RF-31           & Obbligatorio        & Interno                                                                                                           & Le isole ecologiche inviano lo stato di riempimento come percentuale.
	\\\hline
	RF-32           & Obbligatorio        & Interno                                                                                                           & I sensori di livello di acqua inviano il livello di acqua in cm.
	\\\hline
	RF-33           & Obbligatorio        & Esterno                                                                                                           & Il sistema deve permettere di filtrare i dati visualizzati in base a un intervallo di tempo.
	\\\hline
	RF-34           & Obbligatorio        & Esterno                                                                                                           & Il sistema deve permettere di filtrare i dati visualizzati in base al \href{https://7last.github.io/docs/rtb/documentazione-interna/glossario\#sensore}{sensore\textsubscript{G}} che li ha generati.
	\\\hline
	RF-35           & Desiderabile        & Esterno                                                                                                           & Devono essere messe in relazione più sorgenti di dati.
	\\\hline
	RF-36           & Desiderabile        & Esterno                                                                                                           & Nei grafici \href{https://7last.github.io/docs/rtb/documentazione-interna/glossario\#time-series}{time series\textsubscript{G}} i dati devono essere aggregati calcolando la media di 5 minuti, in modo da risultare più leggibili.
	\\\hline
	RF-37           & Obbligatorio        & \href{https://7last.github.io/docs/rtb/documentazione-interna/glossario\#capitolato}{Capitolato\textsubscript{G}} & Deve essere implementato almeno un simulatore di dati.
	\\\hline
	RF-38           & Desiderabile        & \href{https://7last.github.io/docs/rtb/documentazione-interna/glossario\#capitolato}{Capitolato\textsubscript{G}} & Devono essere implementati più simulatori di dati.
	\\\hline
	RF-39           & Obbligatorio        & \href{https://7last.github.io/docs/rtb/documentazione-interna/glossario\#capitolato}{Capitolato\textsubscript{G}} & I simulatori devono produrre dei dati verosimili.
	\\\hline
	RF-40           & Obbligatorio        & \href{https://7last.github.io/docs/rtb/documentazione-interna/glossario\#capitolato}{Capitolato\textsubscript{G}} & Per ciascuna tipologia di \href{https://7last.github.io/docs/rtb/documentazione-interna/glossario\#sensore}{sensore\textsubscript{G}} dev'essere sviluppata almeno una \href{https://7last.github.io/docs/rtb/documentazione-interna/glossario\#dashboard}{dashboard\textsubscript{G}}.
	\\\hline
	RF-41           & Opzionale           & \href{https://7last.github.io/docs/rtb/documentazione-interna/glossario\#capitolato}{Capitolato\textsubscript{G}} & Deve essere implementata una funzionalità di previsione di dati futuri della temperature, basandosi sui dati dell'anno e della settimana precedente.
	\\\hline
	RF-42           & Desiderabile        & \href{https://7last.github.io/docs/rtb/documentazione-interna/glossario\#capitolato}{Capitolato\textsubscript{G}} & Deve esistere una \href{https://7last.github.io/docs/rtb/documentazione-interna/glossario\#dashboard}{dashboard\textsubscript{G}} per la visualizzazione della posizione geografica dei sensori su una mappa
	\\\hline
	RF-43           & Opzionale           & \href{https://7last.github.io/docs/rtb/documentazione-interna/glossario\#capitolato}{Capitolato\textsubscript{G}} & Deve essere presente un sistema di notifiche che allerti l'utente nel caso in cui la temperatura superi i 40°C per più di 30 minuti.
	\\\hline
	RF-44           & Opzionale           & Interno                                                                                                           & Deve essere presente un sistema di notifiche che allerti l'utente se un'isola ecologica rimane al 100\% di riempimento per più di 24 ore.
	\\\hline
	RF-45           & Opzionale           & Interno                                                                                                           & Deve essere presente un sistema di notifiche che allerti l'utente se la qualità dell'aria supera l'indice 3 dell'EAQI.
	\\\hline
	RF-46           & Opzionale           & Interno                                                                                                           & Deve essere presente un sistema di notifiche che allerti l'utente se la quantità di precipitazioni supera i 10mm in un'ora.
	\\\hline
	RF-47           & Opzionale           & Esterno                                                                                                           & Deve essere implementato il calcolo dell'indice di qualità dell'aria EAQI.
	\\\hline
	RF-48           & Opzionale           & Esterno                                                                                                           & Deve essere implementato il calcolo dell'indice di temperatura percepita Heat Index, combinando i dati provenienti dai sensori di temperatura e umidità.
	\\\hline
	RF-49           & Opzionale           & Esterno                                                                                                           & Devono essere combinati i dati provenienti dalle colonnine di ricarica e dai parcheggi per calcolare quanti parcheggi sono stati utilizzati da veicoli elettrici e se il parcheggio ha fruttato abbastanza per coprire i costi di installazione.
	\\\hline
	RF-50           & Obbligatorio        & Esterno                                                                                                           & Il sistema deve permettere di filtrare i dati visualizzati in base al tipo di sensore che li ha prodotti.
	\\\hline
	\caption{Requisiti funzionali}
\end{longtable}

\newpage

\subsection{Requisiti qualitativi}
\begin{longtable}{|>{\centering\arraybackslash}m{0.10\textwidth}|>{\centering\arraybackslash}m{0.20\textwidth}|>{\centering\arraybackslash}m{0.20\textwidth}|>{\centering\arraybackslash}m{0.4\textwidth}|}
	\hline
	\textbf{Codice} & \textbf{Importanza} & \textbf{Fonte}                                                                                                                                                                                                                                       & \textbf{Descrizione}
	\\\hline
	\endfirsthead
	RQ-51           & Obbligatorio        & \href{https://7last.github.io/docs/rtb/documentazione-interna/glossario\#capitolato}{Capitolato\textsubscript{G}}, \href{https://7last.github.io/docs/rtb/documentazione-interna/glossario\#piano-di-qualifica}{Piano di Qualifica\textsubscript{G}} & Sviluppo di test che dimostrino il corretto funzionamento dei servizi e delle funzionalità previste. Viene richiesta una copertura dell'80\% corredata di report.
	\\\hline
	RQ-52           & Obbligatorio        & \href{https://7last.github.io/docs/rtb/documentazione-interna/glossario\#capitolato}{Capitolato\textsubscript{G}}, \href{https://7last.github.io/docs/rtb/documentazione-interna/glossario\#piano-di-qualifica}{Piano di Qualifica\textsubscript{G}} & Il progetto deve essere corredato di documentazione riguardo scelte implementative e progettuali effettuate e relative motivazioni.
	\\\hline
	RQ-53           & Obbligatorio        & \href{https://7last.github.io/docs/rtb/documentazione-interna/glossario\#capitolato}{Capitolato\textsubscript{G}}, \href{https://7last.github.io/docs/rtb/documentazione-interna/glossario\#piano-di-qualifica}{Piano di Qualifica\textsubscript{G}} & Il progetto deve essere corredato di documentazione riguardo problemi aperti e eventuali soluzioni proposte da esplorare.
	\\\hline
	RQ-54           & Obbligatorio        & \href{https://7last.github.io/docs/rtb/documentazione-interna/glossario\#capitolato}{Capitolato\textsubscript{G}}, \href{https://7last.github.io/docs/rtb/documentazione-interna/glossario\#piano-di-qualifica}{Piano di Qualifica\textsubscript{G}} & Tutte le componenti del sistema devono essere testate con \href{https://7last.github.io/docs/rtb/documentazione-interna/glossario\#test-end-to-end}{\textit{test end-to-end}\textsubscript{G}}.
	\\\hline
	RQ-55           & Obbligatorio        & Interno                                                                                                                                                                                                                                              & Il sistema sarà corredato di un Manuale Utente che spieghi le funzionalità del sistema e come utilizzarle.
	\\\hline
	RQ-56           & Obbligatorio        & Interno                                                                                                                                                                                                                                              & Il sistema sarà corredato di un documento di Specifica Tecnica che spieghi le scelte progettuali effettuate.
	\\\hline
	\caption{Requisiti qualitativi}
\end{longtable}

\newpage

\subsection{Requisiti di vincolo}
\begin{longtable}{|>{\centering\arraybackslash}m{0.10\textwidth}|>{\centering\arraybackslash}m{0.20\textwidth}|>{\centering\arraybackslash}m{0.20\textwidth}|>{\centering\arraybackslash}m{0.4\textwidth}|}
	\hline
	\textbf{Codice} & \textbf{Importanza} & \textbf{Fonte}                                                                                                    & \textbf{Descrizione}
	\\\hline
	\endfirsthead
	\hline
	\endhead
	RV-57           & Obbligatorio        & \href{https://7last.github.io/docs/rtb/documentazione-interna/glossario\#capitolato}{Capitolato\textsubscript{G}} & Il simulatore di dati deve pubblicare messaggi in una piattaforma di \textit{data streaming}.
	\\\hline
	RV-58           & Obbligatorio        & Interno                                                                                                           & La piattaforma di \textit{data streaming} utilizzata è \href{https://7last.github.io/docs/rtb/documentazione-interna/glossario\#redpanda}{\textit{Redpanda}\textsubscript{G}}.
	\\\hline
	RV-59           & Obbligatorio        & \href{https://7last.github.io/docs/rtb/documentazione-interna/glossario\#capitolato}{Capitolato\textsubscript{G}} & I dati pubblicati nella piattaforma di \textit{data streaming} devono essere salvati in un database OLAP.
	\\\hline
	RV-60           & Obbligatorio        & \href{https://7last.github.io/docs/rtb/documentazione-interna/glossario\#capitolato}{Capitolato\textsubscript{G}} & I dati devono poter essere visualizzati dall'utente finale in delle \href{https://7last.github.io/docs/rtb/documentazione-interna/glossario\#dashboard}{\textit{dashboard}\textsubscript{G}}, sviluppate con un \textit{tool} apposito, ad esempio \href{https://7last.github.io/docs/rtb/documentazione-interna/glossario\#grafana}{\textit{Grafana}\textsubscript{G}}.
	\\\hline
	RV-61           & Opzionale           & Esterno                                                                                                           & I dati pubblicati nei \href{https://7last.github.io/docs/rtb/documentazione-interna/glossario\#topic}{\textit{topic}\textsubscript{G}} di \href{https://7last.github.io/docs/rtb/documentazione-interna/glossario\#redpanda}{\textit{Redpanda}\textsubscript{G}} sono serializzati in formato \href{https://docs.confluent.io/platform/current/schema-registry/fundamentals/serdes-develop/serdes-avro.html}{\underline{Confluent Avro}}.
	\\\hline
	RV-62           & Obbligatorio        & Esterno                                                                                                           & Il sistema deve essere sviluppato con \href{https://7last.github.io/docs/rtb/documentazione-interna/glossario\#docker-compose}{\href{https://7last.github.io/docs/rtb/documentazione-interna/glossario\#docker}{\textit{Docker}\textsubscript{G}}\textit{ Compose}\textsubscript{G}}, utilizzando la versione 3.8 della specifica.
	\\\hline
	RV-63           & Obbligatorio        & \href{https://7last.github.io/docs/rtb/documentazione-interna/glossario\#capitolato}{Capitolato\textsubscript{G}} & Il sistema deve poter essere usufruito dalle versioni più recenti dei browser web più diffusi. Al momento della stesura del presente documento, le versioni supportate sono: \textit{Google Chrome} v124, \textit{Safari} v17.4, \textit{Microsoft Edge} v123, \textit{Firefox} v125.
	\\\hline
	RV-64           & Obbligatorio        & Interno                                                                                                           & Il sistema deve poter funzionare su sistema operativo \textit{Linux}, con CPU a 64 bit, almeno 4GB di RAM e una delle seguenti distribuzioni e versioni minime: \textit{Ubuntu} 22.04, \textit{Debian} 12, \textit{Fedora} 38, \textit{Red Hat Enterprise Linux} 8.
	\\\hline
	RV-65           & Obbligatorio        & Interno                                                                                                           & Il sistema deve poter funzionare su sistema operativo \textit{Windows} con versione 10 o 11, CPU a 64 bit, almeno 4GB di RAM e la funzionalità WSL2 abilitata.
	\\\hline
	RV-66           & Obbligatorio        & Interno                                                                                                           & Il sistema deve poter funzionare su sistema operativo \textit{MacOs} con versione 12 o superiore, CPU \textit{Intel} o \textit{Apple Silicon} a 64bit e almeno 4GB di RAM.
	\\\hline
	\caption{Requisiti di vincolo}
\end{longtable}

\subsection{Requisiti prestazionali}
\begin{longtable}{|>{\centering\arraybackslash}m{0.10\textwidth}|>{\centering\arraybackslash}m{0.20\textwidth}|>{\centering\arraybackslash}m{0.20\textwidth}|>{\centering\arraybackslash}m{0.4\textwidth}|}
	\hline
	\textbf{Codice} & \textbf{Importanza} & \textbf{Fonte} & \textbf{Descrizione}
	\\\hline
	\endhead
	RP-67           & Obbligatorio        & Interno        & Il sistema deve garantire che la visualizzazione dei dati in tempo reale avvenga entro 5 secondi dalla ricezione dei dati.
	\\\hline
	\caption{Requisiti prestazionali}
\end{longtable}

\newpage

\subsection{Tracciamento}
\subsubsection{Requisito - Fonte}
% comando bash per generare contenuto tabella:
% cat requisiti.tex | awk -F '&' '/R\w\-[[:digit:]]+.*&.*&.*&.*/ {print $1 "&" $3 "\\\\\\hline"}'
\begin{longtable}{|>{\centering\arraybackslash}m{0.30\textwidth}|>{\centering\arraybackslash}m{0.40\textwidth}|}
	\hline
	\textbf{Requisito} & \textbf{Fonte}                                                                                                                                                                                                                                       \\\hline
	\endfirsthead
	\hline
	\textbf{Requisito} & \textbf{Fonte}                                                                                                                                                                                                                                       \\\hline
	\endhead
	RF-1               & \href{https://7last.github.io/docs/rtb/documentazione-interna/glossario\#capitolato}{Capitolato\textsubscript{G}}                                                                                                                                    \\\hline
	RF-2               & \href{https://7last.github.io/docs/rtb/documentazione-interna/glossario\#capitolato}{Capitolato\textsubscript{G}}                                                                                                                                    \\\hline
	RF-3               & \href{https://7last.github.io/docs/rtb/documentazione-interna/glossario\#capitolato}{Capitolato\textsubscript{G}}                                                                                                                                    \\\hline
	RF-4               & \href{https://7last.github.io/docs/rtb/documentazione-interna/glossario\#capitolato}{Capitolato\textsubscript{G}}                                                                                                                                    \\\hline
	RF-5               & \href{https://7last.github.io/docs/rtb/documentazione-interna/glossario\#capitolato}{Capitolato\textsubscript{G}}                                                                                                                                    \\\hline
	RF-6               & \href{https://7last.github.io/docs/rtb/documentazione-interna/glossario\#capitolato}{Capitolato\textsubscript{G}}                                                                                                                                    \\\hline
	RF-7               & \href{https://7last.github.io/docs/rtb/documentazione-interna/glossario\#capitolato}{Capitolato\textsubscript{G}}                                                                                                                                    \\\hline
	RF-8               & \href{https://7last.github.io/docs/rtb/documentazione-interna/glossario\#capitolato}{Capitolato\textsubscript{G}}                                                                                                                                    \\\hline
	RF-9               & \href{https://7last.github.io/docs/rtb/documentazione-interna/glossario\#capitolato}{Capitolato\textsubscript{G}}                                                                                                                                    \\\hline
	RF-10              & \href{https://7last.github.io/docs/rtb/documentazione-interna/glossario\#capitolato}{Capitolato\textsubscript{G}}                                                                                                                                    \\\hline
	RF-11              & \href{https://7last.github.io/docs/rtb/documentazione-interna/glossario\#capitolato}{Capitolato\textsubscript{G}}                                                                                                                                    \\\hline
	RF-12              & \href{https://7last.github.io/docs/rtb/documentazione-interna/glossario\#capitolato}{Capitolato\textsubscript{G}}                                                                                                                                    \\\hline
	RF-13              & Interno                                                                                                                                                                                                                                              \\\hline
	RF-14              & Interno                                                                                                                                                                                                                                              \\\hline
	RF-15              & Interno                                                                                                                                                                                                                                              \\\hline
	RF-16              & Interno                                                                                                                                                                                                                                              \\\hline
	RF-17              & Interno                                                                                                                                                                                                                                              \\\hline
	RF-18              & Interno                                                                                                                                                                                                                                              \\\hline
	RF-19              & Interno                                                                                                                                                                                                                                              \\\hline
	RF-20              & Interno                                                                                                                                                                                                                                              \\\hline
	RF-21              & Interno                                                                                                                                                                                                                                              \\\hline
	RF-22              & Interno                                                                                                                                                                                                                                              \\\hline
	RF-23              & Interno                                                                                                                                                                                                                                              \\\hline
	RF-24              & Interno                                                                                                                                                                                                                                              \\\hline
	RF-25              & Interno                                                                                                                                                                                                                                              \\\hline
	RF-26              & Interno                                                                                                                                                                                                                                              \\\hline
	RF-27              & Interno                                                                                                                                                                                                                                              \\\hline
	RF-28              & Interno                                                                                                                                                                                                                                              \\\hline
	RF-29              & Interno                                                                                                                                                                                                                                              \\\hline
	RF-30              & Interno                                                                                                                                                                                                                                              \\\hline
	RF-31              & Interno                                                                                                                                                                                                                                              \\\hline
	RF-32              & Interno                                                                                                                                                                                                                                              \\\hline
	RF-33              & Esterno                                                                                                                                                                                                                                              \\\hline
	RF-34              & Esterno                                                                                                                                                                                                                                              \\\hline
	RF-35              & Esterno                                                                                                                                                                                                                                              \\\hline
	RF-36              & Esterno                                                                                                                                                                                                                                              \\\hline
	RF-37              & \href{https://7last.github.io/docs/rtb/documentazione-interna/glossario\#capitolato}{Capitolato\textsubscript{G}}                                                                                                                                    \\\hline
	RF-38              & \href{https://7last.github.io/docs/rtb/documentazione-interna/glossario\#capitolato}{Capitolato\textsubscript{G}}                                                                                                                                    \\\hline
	RF-39              & \href{https://7last.github.io/docs/rtb/documentazione-interna/glossario\#capitolato}{Capitolato\textsubscript{G}}                                                                                                                                    \\\hline
	RF-40              & \href{https://7last.github.io/docs/rtb/documentazione-interna/glossario\#capitolato}{Capitolato\textsubscript{G}}                                                                                                                                    \\\hline
	RF-41              & \href{https://7last.github.io/docs/rtb/documentazione-interna/glossario\#capitolato}{Capitolato\textsubscript{G}}                                                                                                                                    \\\hline
	RF-42              & \href{https://7last.github.io/docs/rtb/documentazione-interna/glossario\#capitolato}{Capitolato\textsubscript{G}}                                                                                                                                    \\\hline
	RF-43              & \href{https://7last.github.io/docs/rtb/documentazione-interna/glossario\#capitolato}{Capitolato\textsubscript{G}}                                                                                                                                    \\\hline
	RF-44              & Interno                                                                                                                                                                                                                                              \\\hline
	RF-45              & Interno                                                                                                                                                                                                                                              \\\hline
	RF-46              & Interno                                                                                                                                                                                                                                              \\\hline
	RF-47              & Esterno                                                                                                                                                                                                                                              \\\hline
	RF-48              & Esterno                                                                                                                                                                                                                                              \\\hline
	RF-49              & Esterno                                                                                                                                                                                                                                              \\\hline
	RF-50              & Esterno                                                                                                                                                                                                                                              \\\hline
	RQ-51              & \href{https://7last.github.io/docs/rtb/documentazione-interna/glossario\#capitolato}{Capitolato\textsubscript{G}}, \href{https://7last.github.io/docs/rtb/documentazione-interna/glossario\#piano-di-qualifica}{Piano di Qualifica\textsubscript{G}} \\\hline
	RQ-52              & \href{https://7last.github.io/docs/rtb/documentazione-interna/glossario\#capitolato}{Capitolato\textsubscript{G}}, \href{https://7last.github.io/docs/rtb/documentazione-interna/glossario\#piano-di-qualifica}{Piano di Qualifica\textsubscript{G}} \\\hline
	RQ-53              & \href{https://7last.github.io/docs/rtb/documentazione-interna/glossario\#capitolato}{Capitolato\textsubscript{G}}, \href{https://7last.github.io/docs/rtb/documentazione-interna/glossario\#piano-di-qualifica}{Piano di Qualifica\textsubscript{G}} \\\hline
	RQ-54              & \href{https://7last.github.io/docs/rtb/documentazione-interna/glossario\#capitolato}{Capitolato\textsubscript{G}}, \href{https://7last.github.io/docs/rtb/documentazione-interna/glossario\#piano-di-qualifica}{Piano di Qualifica\textsubscript{G}} \\\hline
	RQ-55              & Interno                                                                                                                                                                                                                                              \\\hline
	RQ-56              & Interno                                                                                                                                                                                                                                              \\\hline
	RV-57              & \href{https://7last.github.io/docs/rtb/documentazione-interna/glossario\#capitolato}{Capitolato\textsubscript{G}}                                                                                                                                    \\\hline
	RV-58              & Interno                                                                                                                                                                                                                                              \\\hline
	RV-59              & \href{https://7last.github.io/docs/rtb/documentazione-interna/glossario\#capitolato}{Capitolato\textsubscript{G}}                                                                                                                                    \\\hline
	RV-60              & \href{https://7last.github.io/docs/rtb/documentazione-interna/glossario\#capitolato}{Capitolato\textsubscript{G}}                                                                                                                                    \\\hline
	RV-61              & Esterno                                                                                                                                                                                                                                              \\\hline
	RV-62              & Esterno                                                                                                                                                                                                                                              \\\hline
	RV-63              & \href{https://7last.github.io/docs/rtb/documentazione-interna/glossario\#capitolato}{Capitolato\textsubscript{G}}                                                                                                                                    \\\hline
	RV-64              & Interno                                                                                                                                                                                                                                              \\\hline
	RV-65              & Interno                                                                                                                                                                                                                                              \\\hline
	RV-66              & Interno                                                                                                                                                                                                                                              \\\hline
	RP-67              & Interno                                                                                                                                                                                                                                              \\\hline
	\caption{Tracciamento requisito - fonte}
\end{longtable}

\subsubsection{Caso d'uso - Requisito}
\begin{longtable}{|>{\centering\arraybackslash}m{0.30\textwidth}|>{\centering\arraybackslash}m{0.40\textwidth}|}
	\hline
	\textbf{Caso d'uso} & \textbf{Requisito}      \\\hline
	\endfirsthead
	\hline
	\textbf{Caso d'uso} & \textbf{Requisito}      \\\hline
	\endhead
	UC-1                & RF-13                   \\\hline
	UC-2                & RF-15                   \\\hline
	UC-2.1              & RF-15                   \\\hline
	UC-2.2              & RF-15                   \\\hline
	UC-2.3              & RF-15                   \\\hline
	UC-2.4              & RF-15                   \\\hline
	UC-2.5              & RF-15                   \\\hline
	UC-2.6              & RF-15                   \\\hline
	UC-2.7              & RF-15                   \\\hline
	UC-2.8              & RF-15                   \\\hline
	UC-2.9              & RF-15                   \\\hline
	UC-2.10             & RF-15                   \\\hline
	UC-2.11             & RF-15                   \\\hline
	UC-2.12             & RF-15                   \\\hline
	UC-2.13             & RF-15                   \\\hline
	UC-2.14             & RF-15                   \\\hline
	UC-2.15             & RF-15                   \\\hline
	UC-2.16             & RF-15                   \\\hline
	UC-2.17             & RF-15                   \\\hline
	UC-2.18             & RF-15                   \\\hline
	UC-2.19             & RF-15                   \\\hline
	UC-2.20             & RF-15                   \\\hline
	UC-3                & RF-16                   \\\hline
	UC-3.1              & RF-16                   \\\hline
	UC-3.2              & RF-16                   \\\hline
	UC-3.3              & RF-16                   \\\hline
	UC-3.4              & RF-16                   \\\hline
	UC-3.5              & RF-16                   \\\hline
	UC-3.6              & RF-16                   \\\hline
	UC-4                & RF-17                   \\\hline
	UC-4.1              & RF-17                   \\\hline
	UC-4.2              & RF-17                   \\\hline
	UC-4.3              & RF-17                   \\\hline
	UC-4.4              & RF-17                   \\\hline
	UC-4.5              & RF-17                   \\\hline
	UC-4.6              & RF-17                   \\\hline
	UC-5                & RF-18                   \\\hline
	UC-5.1              & RF-18                   \\\hline
	UC-5.2              & RF-18                   \\\hline
	UC-5.3              & RF-18                   \\\hline
	UC-5.4              & RF-18                   \\\hline
	UC-5.5              & RF-18                   \\\hline
	UC-5.6              & RF-18                   \\\hline
	UC-6                & RF-19                   \\\hline
	UC-6.1              & RF-19                   \\\hline
	UC-6.2              & RF-19                   \\\hline
	UC-6.3              & RF-19                   \\\hline
	UC-6.4              & RF-19                   \\\hline
	UC-6.5              & RF-19                   \\\hline
	UC-6.6              & RF-19                   \\\hline
	UC-7                & RF-20                   \\\hline
	UC-7.1              & RF-20                   \\\hline
	UC-7.2              & RF-20                   \\\hline
	UC-7.3              & RF-20                   \\\hline
	UC-7.4              & RF-20                   \\\hline
	UC-7.5              & RF-20                   \\\hline
	UC-8                & RF-21                   \\\hline
	UC-8.1              & RF-21                   \\\hline
	UC-8.2              & RF-21                   \\\hline
	UC-9                & RF-22                   \\\hline
	UC-9.1              & RF-22                   \\\hline
	UC-9.2              & RF-22                   \\\hline
	UC-10               & RF-23                   \\\hline
	UC-10.1             & RF-23                   \\\hline
	UC-10.2             & RF-23                   \\\hline
	UC-10.3             & RF-23                   \\\hline
	UC-10.4             & RF-23                   \\\hline
	UC-10.5             & RF-23                   \\\hline
	UC-10.6             & RF-23                   \\\hline
	UC-11               & RF-24                   \\\hline
	UC-11.1             & RF-24                   \\\hline
	UC-11.2             & RF-24                   \\\hline
	UC-11.3             & RF-24                   \\\hline
	UC-11.4             & RF-24                   \\\hline
	UC-12               & RF-25                   \\\hline
	UC-13               & RF-11                   \\\hline
	UC-13.1             & RF-8                    \\\hline
	UC-13.2             & RF-9                    \\\hline
	UC-13.3             & RF-26                   \\\hline
	UC-13.4             & RF-27                   \\\hline
	UC-13.5             & RF-28                   \\\hline
	UC-13.6             & RF-29                   \\\hline
	UC-13.7             & RF-30                   \\\hline
	UC-13.8             & RF-31                   \\\hline
	UC-13.9             & RF-32                   \\\hline
	UC-14               & RF-33,RF-34,RF-50       \\\hline
	UC-14.1             & RF-50                   \\\hline
	UC-14.2             & RF-33                   \\\hline
	UC-14.3             & RF-34                   \\\hline
	UC-15               & RF-43,RF-44,RF-45,RF-46 \\\hline
	UC-15.1             & RF-43                   \\\hline
	UC-15.2             & RF-44                   \\\hline
	\caption{Tracciamento caso d'uso - requisito}
\end{longtable}

\pagebreak
\subsection{Riepilogo}
\begin{longtable}{|>{\centering\arraybackslash}m{0.15\textwidth}|>{\centering\arraybackslash}m{0.15\textwidth}|>{\centering\arraybackslash}m{0.15\textwidth}|>{\centering\arraybackslash}m{0.15\textwidth}|>{\centering\arraybackslash}m{0.2\textwidth}|}
	\hline
	\textbf{Tipologia} & \textbf{Obbligatorio} & \textbf{Desiderabile} & \textbf{Opzionale} & \textbf{Totale} \\\hline
	\endfirsthead
	\textbf{Tipologia} & \textbf{Obbligatorio} & \textbf{Desiderabile} & \textbf{Opzionale} & \textbf{Totale} \\\hline
	\endhead
	Funzionali         & 38                    & 4                     & 8                  & 50              \\\hline
	Qualitativi        & 6                     & 0                     & 0                  & 6               \\\hline
	Di vincolo         & 9                     & 0                     & 1                  & 10              \\\hline
	Prestazionali      & 1                     & 0                     & 0                  & 1               \\\hline
	\caption{Riepilogo}
\end{longtable}
	\caption{Riepilogo}
\end{longtable}
