\section{Introduzione}
\setcounter{subsection}{0}
\subsection{Scopo del documento}
Questo documento ha lo scopo di illustrare i casi d'uso e i requisiti del \href{https://7last.github.io/docs/rtb/documentazione-interna/glossario\#capitolato}{capitolato\textsubscript{G}}
proposto da \textit{Sync Lab S.r.l.}, a seguito di un'analisi da parte del gruppo
e di un confronto tenuto con l'azienda.\\
Vengono presentate le funzionalità che il progetto dovrà offrire, suddivise in requisiti obbligatori,
desiderabili e opzionali, in accordo con le richieste del \href{https://7last.github.io/docs/rtb/documentazione-interna/glossario\#proponente}{proponente\textsubscript{G}}.

\subsection{Glossario}
Per evitare qualsiasi ambiguità o malinteso sui termini utilizzati nel seguente documento,
è stato aggiunto un \href{https://7last.github.io/docs/rtb/documentazione-interna/glossario\#glossario}{glossario\textsubscript{G}},  contenente le definizioni necessarie. È possibile individuare ogni termine presente
nel \href{https://7last.github.io/docs/rtb/documentazione-interna/glossario\#glossario}{glossario\textsubscript{G}}
grazie ad uno stile specifico:
\begin{itemize}
	\item Ad ogni parola presente sarà aggiunta una "G" al pedice della stessa.
	\item Verrà fornito il link al
	      \href{https://7last.github.io/docs/rtb/documentazione-interna/glossario\#glossario}{glossario\textsubscript{G}} online (v.1.0) per ciascuna parola.

\end{itemize}

\subsection{Riferimenti}
\subsubsection{Normativi}
\begin{itemize}
	\item \href{https://7last.github.io/docs/rtb/documentazione-interna/glossario\#capitolato}{Capitolato\textsubscript{G}} C6 - \href{https://7last.github.io/docs/rtb/documentazione-interna/glossario\#synccity}{SyncCity\textsubscript{G}}: \href{https://7last.github.io/docs/rtb/documentazione-interna/glossario\#smart-city}{Smart city\textsubscript{G}} monitoring platform\\
	      \url{https://www.math.unipd.it/~tullio/IS-1/2023/Progetto/C6.pdf}
	\item Regolamento di progetto didattico\\
	      \url{https://www.math.unipd.it/~tullio/IS-1/2023/Dispense/PD2.pdf}
	\item \href{https://7last.github.io/docs/rtb/documentazione-interna/glossario\#norme-di-progetto}{Norme di progetto\textsubscript{G}}:\\
	      \url{https://7last.github.io/docs/rtb/documentazione-interna/norme-di-progetto}
\end{itemize}

\subsubsection{Interni}
Durante la fase di Analisi del \href{https://7last.github.io/docs/rtb/documentazione-interna/glossario\#capitolato}{capitolato\textsubscript{G}} il gruppo ha proposto all'azienda
l'utilizzo di \href{https://7last.github.io/docs/rtb/documentazione-interna/glossario\#Redpanda}{Redpanda\textsubscript{G}} come piattaforma di \textit{streaming} alternativa ad \href{https://7last.github.io/docs/rtb/documentazione-interna/glossario\#apache-kafka}{Apache Kafka\textsubscript{G}}.
A seguito di un confronto con l'azienda, è stato deciso di utilizzare \href{https://7last.github.io/docs/rtb/documentazione-interna/glossario\#Redpanda}{Redpanda\textsubscript{G}}.\\
Come richiesto dalla \href{https://7last.github.io/docs/rtb/documentazione-interna/glossario\#proponente}{proponente\textsubscript{G}}, il gruppo ha prodotto un documento aggiuntivo di
confronto tra le due tecnologie, disponibile al seguente link:\\
\url{https://7last.github.io/docs/rtb/documentazione-interna/analisi-kafka-redpanda}
