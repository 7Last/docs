\section{Introduzione}
\setcounter{subsection}{0}
\subsection{Scopo del documento}
Questo documento ha lo scopo di illustrare i casi d'uso e i requisiti del capitolato proposto da \textit{Sync Lab S.r.l.}, a seguito di un'attenta analisi del capitolato stesso e di un confronto tenuto con l'azienda.
\subsection{Glossario}
Per evitare qualsiasi ambiguità o malinteso sui termini utilizzati nel seguente documento, è stato adottato un \href{https://7last.github.io/docs/rtb/documentazione-interna/glossario#glossario}{glossario\textsubscript{G}}, contenente le definizioni necessarie. È possibile individuare ogni termine presente nel \href{https://7last.github.io/docs/rtb/documentazione-interna/glossario#glossario}{glossario\textsubscript{G}} grazie ad uno stile specifico:
\begin{itemize}
    \item Ad ogni parola presente sarà aggiunta una "G" al pedice della stessa.
    \item Verrà fornito il link al \href{https://7last.github.io/docs/rtb/documentazione-interna/glossario#glossario}{glossario\textsubscript{G}} online (v.1.0) per ciascuna parola.
\end{itemize}

\subsection{Riferimenti} BYTEOPS METTONO SOLO RIFERIMENTI NORMATIVI
    \subsubsection{Normativi}
    \begin{itemize}
        \item Capitolato C6 - SyncCity: Smart city monitoring platform.\\
        LINK AL CAPITOLATO
        \item Regolamento di progetto didattico.\\
        LINK AL REGOLAMENTO
        \item \href{https://7last.github.io/docs/rtb/documentazione-interna/glossario#norme-di-progetto}{Norme di progetto\textsubscript{G}}.\\
        LINK ALLE \href{https://7last.github.io/docs/rtb/documentazione-interna/glossario#norme-di-progetto}{NORME DI PROGETTO\textsubscript{G}}
    \end{itemize}
        
    % \subsubsection{Informativi}
       
