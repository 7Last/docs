\section{Introduzione}
\subsection{Obiettivo del documento}
Il presente documento ha lo scopo di definire le strategie di verifica e validazione utilizzate per assicurare il corretto funzionamento dello strumento sviluppato e delle
attività che lo accompagnano.  Sarà sottoposto a revisioni continue, così da prevedere situazioni precedentemente non occorse e da seguire l'evoluzione del progetto.
\subsection{Glossario}
Il \href{https://7last.github.io/docs/rtb/documentazione-interna/glossario\#glossario}{glossario\textsubscript{G}} è uno strumento utilizzato per risolvere eventuali dubbi riguardanti
alcuni termini specifici utilizzati nella redazione del documento.
Esso conterrà la definizione dei termini evidenziati e sarà consultabile al seguente \href{https://7last.github.io/docs/rtb/documentazione-interna/glossario}{link}. I termini presenti in tale documento saranno evidenziati da una 'G' a pedice.
\subsection{Riferimenti}
\subsubsection{Riferimenti normativi}
\begin{itemize}
    \item \href{https://7last.github.io/docs/rtb/documentazione-interna/glossario\#norme-di-progetto}{Norme di progetto\textsubscript{G}} (aggiungere versione e/o link al documento);
    \item Regolamento del progetto:\\
		  \url{https://www.math.unipd.it/~tullio/IS-1/2023/Dispense/PD2.pdf}.
\end{itemize}
\subsubsection{Riferimenti informativi}
\begin{itemize}
    \item \href{https://7last.github.io/docs/rtb/documentazione-interna/glossario#capitolato}{Capitolato\textsubscript{G}} d'appalto C6: SyncCity – A smart city monitoring platform\\
    \url{https://www.math.unipd.it/~tullio/IS-1/2023/Progetto/C6.pdf};
    \item \href{https://it.wikipedia.org/wiki/ISO/IEC_9126}{Standard ISO/IEC 9126};
    \item \href{https://iso25000.com/index.php/en/iso-25000-standards/iso-25010}{Standard ISO/IEC 25010};
    \item \href{ https://en.wikipedia.org/wiki/ISO/IEC_12207}{Standard ISO/IEC 12207:1995};
    \item \href{URL}{\textit{Verbali esterni}};
    \item \href{URL}{\textit{Verbali interni}};
    \item \href{URL}{\href{https://7last.github.io/docs/rtb/documentazione-interna/glossario#analisi-dei-requisiti}{\textit{Analisi dei requisiti}\textsubscript{G}}};
    \item AGGIUNGERE LINK
\end{itemize}