\documentclass[italian,12pt]{article} %tipo di documento

%--------------variabili------------------%
\def\Title{Norme di Progetto}
\def\Author{7Last}
\def\Version{v0.2}
%-----------------------------------------%


\usepackage[left=2cm, right=2cm, bottom=3cm, top=3cm]{geometry}
\usepackage{fancyhdr}
\usepackage{graphicx}
\graphicspath{ {../../logo/} }
\usepackage{href-ul}
\usepackage{tikz}
\usepackage{tgadventor}
\usepackage[useregional=numeric,showseconds=true,showzone=false]{datetime2}
\usepackage{caption}
\usepackage{longtable}
\usepackage{xcolor}



\input{subX3_section.tex}

\linespread{1.2}
\captionsetup[table]{labelformat=empty}
\geometry{headsep=1.5cm}

\renewcommand{\contentsname}{Indice}%imposto il nome dell'indice
\renewcommand\familydefault{\sfdefault}

\renewcommand{\listtablename}{Indice delle tabelle}%imposto il nome della lista tabelle
\renewcommand\familydefault{\sfdefault}

\renewcommand{\listfigurename}{Indice delle immagini}%imposto il nome della lista immagini
\renewcommand\familydefault{\sfdefault}

%-------------------INIZIO DOCUMENTO--------------
\begin{document}

\newgeometry{left=2cm,right=2cm,bottom=2.1cm,top=2.1cm}
\begin{titlepage}
	\vspace*{.5cm}

	\vspace{2cm}
	{
		\centering
		{\bfseries\huge \Title\par}
		\bigbreak
		{\bfseries\Large \Subtitle\par}
		\bigbreak
		{\bfseries\large \Author\par}
		\bigbreak
		{\Date\;-\;\Version\par}
		\vfill

		\begin{center}
			\begin{tikzpicture}
				\clip (0,0) circle (2cm) node {\includegraphics[width=4cm]{logo.jpg}};
			\end{tikzpicture}
		\end{center}
	}

	\vfill

\end{titlepage}

\restoregeometry






















\newpage

\pagestyle{fancy}
\fancyhead{}
\lhead{
	\begin{tikzpicture}
		\clip (0,0) circle (0.5cm);
		\node at (0,0) {\includegraphics[width=1cm]{./../logo/logo.png}};
	\end{tikzpicture}%
}
\chead{\vspace{\fill}\Title\vspace{\fill}}
\rhead{\vspace{\fill}\Version\vspace{\fill}}


%-----------tabella revisioni-----------%
\captionsetup[table]{list=no}

\begin{table}[!h]
	\begin{center}
		Versioni\\
		\vspace{0.5cm}
		\begin{tabular}{ c c c p{9cm} }
			\hline                                                                                 \\[-2ex]
			Ver. & Data       & Autore             & Descrizione                                   \\
			\\[-2ex] \hline \\[-1.5ex]
			0.1  & 28/03/2024 & Valerio Occhinegro & Prima redazione          					   \\
			\\[-1.5ex] \hline
		\end{tabular}
	\end{center}
\end{table}
\captionsetup[table]{list=yes}
%---------------------------------------%

\newpage

\tableofcontents

\newpage

\listoftables

\listoffigures

\newpage

\section{Introduzione}

\subsection{Obiettivo del documento}
Il documento contiene tutti i metodi di verifica e validazione utilizzati
per assicurare il corretto funzionamento dello strumento sviluppato e delle
attività che lo accompagnano.
Il documento sarà sottoposto a continui aggiornamenti, per fare in modo che 
rispecchi le eventuali evoluzioni del progetto.
In aggiunta verrano registrati gli esiti delle verifiche svolte, in maniera tale 
da risolvere rapidamente gli errori risultanti.


\subsection{Obiettivo del prodotto}
Il prodotto deve sviluppare una piattaforma di smart city monitoring 
che riesca a simulare dati provenienti da varie tipoologie di sensori e che sia in grado di 
rappresentarli in una serie di dashboard.
L'azienda SyncLab si è già occupata di gestire la funzione di smart-parking (monitoraggio e gestione parcheggi) 
tramite una tecnologia affine e attualmente sta sviluppando un sistema di sincronizzazione semaforica.
Sarà dunque di loro interesse verificare la fattibilità di realizzazione di una piattaforma 
che unisca in un unico luogo il monitoraggio di una città.

\subsection{Glossario}
Il glossario è uno strumento utilizzato per risolvere eventuali dubbi riguardanti 
alcuni termini specifici utilizzati nella redazione del documento.
Esso conterrà la definizione dei termini evidenziati (inserire Metodo con cui viene fatto l'highlight dei termini)
e altre disambiguazioni 
\subsection{Riferimenti}

\subsubsection{Riferimenti normativi}
\begin{itemize}
    \item Norme di progetto (aggiungere versione e/o link al documento)
    \item Capitolato d'appalto C6: SyncCity – A smart city monitoring platform\\
		  \url{https://www.math.unipd.it/~tullio/IS-1/2023/Progetto/C6.pdf}
    \item Regolamento del progetto:\\
		  \url{https://www.math.unipd.it/~tullio/IS-1/2023/Dispense/PD2.pdf}
\end{itemize}

\subsubsection{Riferimenti informativi}
\begin{itemize}
    \item alcuni mettono le dispense del prof altri la documentazione iso del 1995 
			sempre fornita dal prof
    \item Glossario
\end{itemize}

\section{Fini metrici di qualità}
Tutti i processi che compongono il progetto sono valutati tramite l'utilizzo 
di metriche specifiche, che (sono ben definite dalle sezioni Metriche di qualitá del processo e Metriche di qualitá del prodotto del docu-
mento Norme di Progetto v1.0.0. ).
Il contenuto di questa sezione è necessario per valutare le metriche e 
dunque suddividerle in accettabili o ottime.

\subsection{Qualità di processo}
La qualità di processo è un criterio fondamentale ed è alla base di ogni prodotto
che rispecchi lo stato dell'arte. Per raggiungere tale obiettivo è necessario 
sfruttare delle pratiche rigorose che consentano lo svolgimento di ogni attività
in maniera ottimale.

\subsubsection{Processi primari}

\subsubsubsection{Fornitura}

\rowcolors{1}{white}{lightgray}
\begin{table}[h!]
	\centering
	\begin{tabular}{|c|c|c|c|} 
	 \hline
	 Metrica & Nome & Valore ammissibile & Valore ottimo \\  
	 \hline
	 MPC01 & Earned Value (EV) & $\geq 0$ & 7560 \\
	 \hline
	 MPC02 & Planned Value (PV) & $\geq 0$ & 6344 \\ 
	 \hline
	 MPC03 & Actual Cost (AC) & $\geq 0$ & 6344 \\ 
	 \hline
	 MPC04 & Cost Variance (CV) & $\geq -7.5\%$ & 6344 \\ 
	 \hline
	 MPC05 & Schedule Variance (SV) & $\geq 0$ & 6344 \\ 
	 \hline
	 MPC06 & Estimated at Completion (EAC) & 788 & 6344 \\ 
	 \hline
	 MPC07 & Estimate to Complete (ETC)  & $\geq 0$ & 6344 \\ 
	 \hline
	\end{tabular}
	\caption{Tabella dei valori per il processo di Fornitura.}
	\label{table:1}
	\end{table}

\subsubsubsection{Sviluppo}

\subsubsection{Processi di supporto}

\subsubsubsection{Documentazione}

\subsubsubsection{Verifica}

\subsubsubsection{Gestione della qualità}

\subsubsection{Processi organizzativi}

\subsubsubsection{Gestione dei processi}

\subsection{Qualità di prodotto}

\subsubsection{Funzionalità}

\subsubsection{Affidabilità}

\subsubsection{Usabilità}

\subsubsection{Efficienza}

\subsubsection{Manutenibilità}

\section{Metodologie di testing}

\subsection{Codice dei test}

\subsection{Test di unità}

\subsection{Test di integrazione}

\subsection{Test di sistema}

\newpage

\section{Cruscotto di valutazione della qualità}

\subsection{MPC06 - Estimated at Completion(EAC)}

\subsection{MPC01 - Earned Value (EV) e MPC02 - Planned Value (PV)}

\subsection{MPC03 - Actual Cost (AC) e MPC07 - Estimate to Complete (ETC)}

\subsection{MPC04 - Cost Variance (CV) e MPC05 - Schedule Variance (SV)}

\subsection{MPC08 - Requirements stability index (RSI)}

\subsection{MPC11 - Indice Gulpease}

\subsection{MPC12 - Correttezza Ortografica}

\subsection{MPC15 - Quality Metrics Satisfied}

\subsection{MPC16 - Non-Calculated Risk}

\subsection{MPC17 - Efficienza Temporale}

\section{Iniziative di automiglioramento per la qualità}

\subsection{Introduzione}

\subsection{Problemi leagati all’organizzazione generale}

\subsection{Valutazione sui ruoli}

\subsection*{Valutazione sugli strumenti}

\subsection{Considerazioni finali sul miglioramento}

\subsubsection{Analisi della pratiche seguite}

\subsubsection{Valutazioni generali sui miglioramenti conseguiti}

\subsubsection{Valutazioni specifiche sui miglioramenti nei processi}

\subsubsubsection{Gestione delle comunicazioni e degli incontri}

\subsubsubsection{Pianificazione}













\end{document}
