\documentclass[italian,12pt]{article} %tipo di documento

%--------------variabili------------------%
\def\Title{Norme di Progetto}
\def\Author{7Last}
\def\Version{v0.2}
%-----------------------------------------%


\usepackage[left=2cm, right=2cm, bottom=3cm, top=3cm]{geometry}
\usepackage{fancyhdr}
\usepackage{graphicx}
\graphicspath{ {../../logo/} }
\usepackage{href-ul}
\usepackage{tikz}
\usepackage{tgadventor}
\usepackage[useregional=numeric,showseconds=true,showzone=false]{datetime2}
\usepackage{caption}
\usepackage{longtable}
\usepackage{xcolor}



\input{subX3_section.tex}

\linespread{1.2}

\captionsetup[table]{name=Tabella}
\geometry{headsep=1.5cm}

\renewcommand{\contentsname}{Indice}%imposto il nome dell'indice
\renewcommand\familydefault{\sfdefault}

\renewcommand{\listtablename}{Indice delle tabelle}%imposto il nome della lista tabelle
\renewcommand\familydefault{\sfdefault}

\renewcommand{\listfigurename}{Indice delle immagini}%imposto il nome della lista immagini
\renewcommand\familydefault{\sfdefault}

%-------------------INIZIO DOCUMENTO--------------
\begin{document}

\newgeometry{left=2cm,right=2cm,bottom=2.1cm,top=2.1cm}
\begin{titlepage}
	\vspace*{.5cm}

	\vspace{2cm}
	{
		\centering
		{\bfseries\huge \Title\par}
		\bigbreak
		{\bfseries\Large \Subtitle\par}
		\bigbreak
		{\bfseries\large \Author\par}
		\bigbreak
		{\Date\;-\;\Version\par}
		\vfill

		\begin{center}
			\begin{tikzpicture}
				\clip (0,0) circle (2cm) node {\includegraphics[width=4cm]{logo.jpg}};
			\end{tikzpicture}
		\end{center}
	}

	\vfill

\end{titlepage}

\restoregeometry






















\newpage

\pagestyle{fancy}
\fancyhead{}
\lhead{
	\begin{tikzpicture}
		\clip (0,0) circle (0.5cm);
		\node at (0,0) {\includegraphics[width=1cm]{./../logo/logo.png}};
	\end{tikzpicture}%
}
\chead{\vspace{\fill}\Title\vspace{\fill}}
\rhead{\vspace{\fill}\Version\vspace{\fill}}


%-----------tabella revisioni-----------%
\captionsetup[table]{list=no}

\begin{table}[!h]
	\begin{center}
		Versioni\\
		\vspace{0.5cm}
		\begin{tabular}{ l l l l l }
			\hline                                                                          		\\[-2ex]
			Ver. 	& Data			& Autore			 & Verificatore		& Descrizione\\
			\\[-2ex] \hline \\[-1.5ex]

			0.3 	 & 07/04/2024 	& Valerio Occhinegro & Matteo Tiozzo	& Stesura documento	\\
			0.2	     & 29/03/2024 	& Matteo Tiozzo 	 & 					& Modificato tabella versioni	\\
			0.1 	 & 28/03/2024 	& Valerio Occhinegro & Matteo Tiozzo 	& Prima redazione				\\
			\\[-1.5ex] \hline
		\end{tabular}
	\end{center}
\end{table}
\captionsetup[table]{list=yes}
%---------------------------------------%

\newpage
\tableofcontents
\listoftables
\listoffigures
\newpage

\section{Introduzione}
\setcounter{subsection}{0}
\subsection{Scopo del documento}
Il seguente documento si propone di definire la pianificazione e la gestione delle attività richieste per ultimare il progetto. Vengono esaminati in dettaglio elementi cruciali come l’\textit{Analisi dei Rischi}, il \textit{modello di sviluppo adottato}, la \textit{pianificazione delle attività}, la \textit{suddivisione dei ruoli}, oltre a \textit{stime dei costi} e delle \textit{risorse necessarie}.

\subsection{Scopo del prodotto}
Lo scopo principale del prodotto è quello di consentire a \textit{Sync Lab S.r.l.} di valutare la \\\textbf{fattibilità} di investire tempo e risorse nell'implementazione del progetto  \href{https://7last.github.io/docs/rtb/documentazione-interna/glossario\#synccity}{\textit{\textbf{SyncCity} \textsubscript{G}}- A \href{https://7last.github.io/docs/rtb/documentazione-interna/glossario\#smart-city}{smart city\textsubscript{G}} monitoring platform}. Questa soluzione, attraverso l'utilizzo di dispositivi IoT, consente un monitoraggio costante delle città. \href{https://7last.github.io/docs/rtb/documentazione-interna/glossario\#synccity}{SyncCity\textsubscript{G}} avrà lo scopo di monitorare e raccogliere dati da sensori posizionati nelle città, per poi analizzarli e fornire informazioni utili alla gestione della città. Il prodotto finale sarà un prototipo funzionale che consentirà la visualizzazione dei dati raccolti su un cruscotto.

\subsection{Glossario}
Per evitare qualsiasi ambiguità o malinteso sui termini utilizzati nel documento, verrà adottato un \href{https://7last.github.io/docs/rtb/documentazione-interna/glossario\#glossario}{glossario\textsubscript{G}}. Questo \href{https://7last.github.io/docs/rtb/documentazione-interna/glossario\#glossario}{glossario\textsubscript{G}} conterrà varie definizioni. Ogni termine incluso nel \href{https://7last.github.io/docs/rtb/documentazione-interna/glossario\#glossario}{glossario\textsubscript{G}} sarà indicato applicando uno stile specifico:
\begin{itemize}
    \item aggiungendo una "G" al pedice della parola;
    \item fornendo il link al \href{https://7last.github.io/docs/rtb/documentazione-interna/glossario\#glossario}{glossario\textsubscript{G}} online;
\end{itemize}

\subsection{Riferimenti}
    \subsubsection{Normativi}DA SISTEMARE
        \begin{itemize}
            \item \textbf{ISO/IEC 12207:2008} - Systems and software engineering - Software life cycle processes
            \item \textbf{ISO/IEC 31000:2009} - Risk management - Principles and guidelines
        \end{itemize}
    \subsubsection{Informativi}
        \begin{itemize}
            \item \href{https://7last.github.io/docs/rtb/documentazione-interna/glossario\#capitolato}{\textbf{Capitolato \textsubscript{G}}C6 - Sync City}: \textit{A \href{https://7last.github.io/docs/rtb/documentazione-interna/glossario\#smart-city}{smart city\textsubscript{G}} monitoring platform}
            \item \textbf{T2 - Processi di ciclo di vita del software}\\ https://www.math.unipd.it/~tullio/IS-1/2023/Dispense/T2.pdf;
            \item \textbf{T4 - Gestione di progetto}\\ Visibili a questo \uline{\href{https://www.math.unipd.it/~tullio/IS-1/2023/Dispense/T4.pdf}{link}};
            \item \href{https://7last.github.io/docs/rtb/documentazione-interna/glossario\#glossario}{\textbf{Glossario}\textsubscript{G}}\\ Visibile a questo \uline{\href{https://7last.github.io/docs/rtb/documentazione-interna/glossario}{link}};
        \end{itemize}
\subsection{Preventivo iniziale}
Il preventivo iniziale presentato durante la fase di candidatura è disponibile al seguente \uline{\href{https://github.com/7Last/docs/blob/main/1_candidatura/preventivo_costi_assunzione_impegni_v2.0.pdf}{riferimento}}. All'interno di questo documento viene calcolato il preventivo iniziale del progetto, pari a €12.670,00. Inoltre, si specifica che il gruppo \textit{7Last} stima di \textbf{completare} il prodotto entro e non oltre il \textbf{24 Settembre 2024}.

\section{Metriche di qualità}
\subsection{Metriche per la qualità di processo}
\subsection{Metriche per la qualità di prodotto}



\section{Metodologie di Testing}
In questa sezione verranno illustrate le metodologie di \textit{testing} adottate per garantire il rispetto dei vincoli individuati
nella sezione \textit{Requisiti} del documento \href{https://7last.github.io/docs/rtb/documentazione-esterna/analisi-dei-requisiti}{\href{https://7last.github.io/docs/rtb/documentazione-interna/glossario\#analisi-dei-requisiti}{\textit{Analisi dei Requisiti}\textsubscript{G}}}.
I test verranno suddivisi in cinque categorie:
\begin{itemize}
    \item test di Unità;
    \item test di Integrazione;
    \item test di Sistema;
    \item test di Regressione;
    \item test di Accettazione.
\end{itemize}

Verranno elencate le varie tipologie di test eseguite, indicando il codice del test, una breve descrizione di ciò che viene verificato e lo stato di avanzamento del test, espresso come segue:
\begin{itemize}
	\item \textbf{S}: test superato;
	\item \textbf{NS}: test non superato;
	\item \textbf{NI}: test non implementato.
\end{itemize}

\subsection{Test di Unità}
I test di unità verificano il corretto funzionamento delle singole unità di codice, ovvero le più piccole parti di un programma,
per assicurarsi che ognuna funzioni correttamente e che sia in grado di eseguire le operazioni richieste. \\
\begin{longtable}{|>{\raggedright\arraybackslash}m{0.1\textwidth}|>{\raggedright\arraybackslash}m{0.6\textwidth}|>{\raggedright\arraybackslash}m{0.1\textwidth}|}
	\hline
	\textbf{Codice} & \textbf{Descrizione}                                                                                                                                              & \textbf{Stato} \\
	\hline
	\endfirsthead
	\hline
	\textbf{Codice} & \textbf{Descrizione}                                                                                                                                              & \textbf{Stato} \\
	\endhead
	\hline
	\textbf{1T-U}   & Verificare che la classe \texttt{TemperatureRawData} venga creata correttamente.                                                                                  & NI             \\
	\hline
	\textbf{2T-U}   & Verificare che il metodo \texttt{topic()} di \texttt{TemperatureRawData} restituisca \texttt{"temperature"}.                                                      & NI             \\
	\hline
	\textbf{3T-U}   & Verificare che il metodo \texttt{subject()} di \texttt{TemperatureRawData} restituisca \texttt{"temperature-value"}.                                              & NI             \\
	\hline
	\textbf{4T-U}   & Verificare che la classe \texttt{TrafficRawData} venga creata correttamente.                                                                                      & NI             \\
	\hline
	\textbf{5T-U}   & Verificare che il metodo \texttt{topic()} di \texttt{TrafficRawData} restituisca \texttt{"traffic"}.                                                              & NI             \\
	\hline
	\textbf{6T-U}   & Verificare che il metodo \texttt{subject()} di \texttt{TrafficRawData} restituisca \texttt{"traffic-value"}.                                                      & NI             \\
	\hline
	\textbf{7T-U}   & Verificare che la classe \texttt{RecyclingPointRawData} venga creata correttamente.                                                                               & NI             \\
	\hline
	\textbf{8T-U}   & Verificare che il metodo \texttt{topic()} di \texttt{RecyclingPointRawData} restituisca \texttt{"recycling\_point"}.                                              & NI             \\
	\hline
	\textbf{9T-U}   & Verificare che il metodo \texttt{subject()} di \texttt{RecyclingPointRawData} restituisca \texttt{"recycling\_point-value"}.                                      & NI             \\
	\hline
	\textbf{10T-U}  & Verificare che la classe \texttt{HumidityRawData} venga creata correttamente.                                                                                     & NI             \\
	\hline
	\textbf{11T-U}  & Verificare che il metodo \texttt{topic()} di \texttt{HumidityRawData} restituisca \texttt{"humidity"}.                                                            & NI             \\
	\hline
	\textbf{12T-U}  & Verificare che il metodo \texttt{subject()} di \texttt{HumidityRawData} restituisca \texttt{"humidity-value"}.                                                    & NI             \\
	\hline
	\textbf{13T-U}  & Verificare che la classe \texttt{AirQualityRawData} venga creata correttamente.                                                                                   & NI             \\
	\hline
	\textbf{14T-U}  & Verificare che il metodo \texttt{topic()} di \texttt{AirQualityRawData} restituisca \texttt{"air\_quality"}.                                                      & NI             \\
	\hline
	\textbf{15T-U}  & Verificare che il metodo \texttt{subject()} di \texttt{AirQualityRawData} restituisca \texttt{"air\_quality-value"}.                                              & NI             \\
	\hline
	\textbf{16T-U}  & Verificare che la classe \texttt{RainRawData} venga creata correttamente.                                                                                         & NI             \\
	\hline
	\textbf{17T-U}  & Verificare che il metodo \texttt{topic()} di \texttt{RainRawData} restituisca \texttt{"rain"}.                                                                    & NI             \\
	\hline
	\textbf{18T-U}  & Verificare che il metodo \texttt{subject()} di \texttt{RainRawData} restituisca \texttt{"rain-value"}.                                                            & NI             \\
	\hline
	\textbf{19T-U}  & Verificare che la classe \texttt{ChargingStationRawData} venga creata correttamente.                                                                              & NI             \\
	\hline
	\textbf{20T-U}  & Verificare che il metodo \texttt{topic()} di \texttt{ChargingStationRawData} restituisca \texttt{"charging\_station"}.                                            & NI             \\
	\hline
	\textbf{21T-U}  & Verificare che il metodo \texttt{subject()} di \texttt{ChargingStationRawData} restituisca \texttt{"charging\_station-value"}.                                    & NI             \\
	\hline
	\textbf{22T-U}  & Verificare che la classe \texttt{ParkingLotRawData} venga creata correttamente.                                                                                   & NI             \\
	\hline
	\textbf{23T-U}  & Verificare che il metodo \texttt{topic()} di \texttt{ParkingLotRawData} restituisca \texttt{"parking\_lot"}.                                                      & NI             \\
	\hline
	\textbf{24T-U}  & Verificare che il metodo \texttt{subject()} di \texttt{ParkingLotRawData} restituisca \texttt{"parking\_lot-value"}.                                              & NI             \\
	\hline
	\textbf{25T-U}  & Verificare che la classe \texttt{WaterLevelRawData} venga creata correttamente.                                                                                   & NI             \\
	\hline
	\textbf{26T-U}  & Verificare che il metodo \texttt{topic()} di \texttt{WaterLevelRawData} restituisca \texttt{"water\_level"}.                                                      & NI             \\
	\hline
	\textbf{27T-U}  & Verificare che il metodo \texttt{subject()} di \texttt{WaterLevelRawData} restituisca \texttt{"water\_level-value"}.                                              & NI             \\
	\hline
	\textbf{28-U}   & Verificare che il metodo \texttt{from\_str()} di \texttt{SensorType} effettui il parsing correttamente.                                                           & NI             \\
	\hline
	\textbf{29-U}   & Verificare che la classe \texttt{EnvConfig} venga creata correttamente se tutte le variabili d'ambiente sono impostate.                                           & NI             \\
	\hline
	\textbf{30-U}   & Verificare che la classe \texttt{EnvConfig} venga creata correttamente se la variabile d'ambiente \texttt{MAX\_BLOCK\_MS} non è impostata.                        & NI             \\
	\hline
	\textbf{31-U}   & Verificare che la creazione della classe \texttt{EnvConfig} fallisca con un'eccezione se le variabili d'ambiente non sono impostate.                              & NI             \\
	\hline
	\textbf{32-U}   & Verificare che il metodo \texttt{bootstrap\_server} della classe \texttt{EnvConfig} ritorni correttamente il valore dell'host concatenato alla porta con ':'..    & NI             \\
	\hline
	\textbf{33-U}   & Verificare che la classe \texttt{SensorConfig} sia creata correttamente.                                                                                          & NI             \\
	\hline
	\textbf{34-U}   & Verificare che la creazione della classe \texttt{SensorConfig} fallisca con un'eccezione se il tipo di sensore fornito non esiste.                                & NI             \\
	\hline
	\textbf{35-U}   & Verificare che la creazione della classe \texttt{SensorConfig} fallisca con un'eccezione se il tipo di sensore non è fornito.                                     & NI             \\
	\hline
	\textbf{36-U}   & Verificare che la creazione della classe \texttt{SensorConfig} fallisca con un'eccezione se il campo \texttt{generation\_delay} non rispetta lo standard ISO8601. & NI             \\
	\hline
	\textbf{37-U}   & Verificare che la creazione della classe \texttt{SensorConfig} fallisca con un'eccezione se il campo \texttt{points\_spacing} non rispetta lo standard ISO8601.   & NI             \\
	\hline
	\textbf{38-U}   & Verificare che la funzione \texttt{simulator\_generator()} crei correttamente i \texttt{Simulator} a partire da una lista di \texttt{SensorConfig}.               & NI             \\
	\hline
	\textbf{39-U}   & Verificare che il metodo \texttt{serialize\_temperature\_raw\_data()} serializzi correttamente \texttt{TemperatureRawData}.                                       & NI             \\
	\hline
	\textbf{40-U}   & Verificare che il metodo \texttt{serialize\_traffic\_raw\_data()} serializzi correttamente \texttt{TrafficRawData}.                                               & NI             \\
	\hline
	\textbf{41-U}   & Verificare che il metodo \texttt{serialize\_recycling\_point\_raw\_data()} serializzi correttamente \texttt{RecyclingPointRawData}.                               & NI             \\
	\hline
	\textbf{42-U}   & Verificare che il metodo \texttt{serialize\_humidity\_raw\_data()} serializzi correttamente \texttt{HumidityRawData}.                                             & NI             \\
	\hline
	\textbf{43-U}   & Verificare che il metodo \texttt{serialize\_air\_quality\_raw\_data()} serializzi correttamente \texttt{AirQualityRawData}.                                       & NI             \\
	\hline
	\textbf{44-U}   & Verificare che il metodo \texttt{serialize\_humidity\_raw\_data()} serializzi correttamente \texttt{HumidityRawData}.                                             & NI             \\
	\hline
	\textbf{46-U}   & Verificare che il metodo \texttt{serialize\_rain\_raw\_data()} serializzi correttamente \texttt{RainRawData}.                                                     & NI             \\
	\hline
	\textbf{47-U}   & Verificare che il metodo \texttt{serialize\_charging\_station\_raw\_data()} serializzi correttamente \texttt{ChargingStationRawData}.                             & NI             \\
	\hline
	\textbf{48-U}   & Verificare che il metodo \texttt{serialize\_parking\_lot\_raw\_data()} serializzi correttamente \texttt{ParkingLotRawData}.                                       & NI             \\
	\hline
	\textbf{49-U}   & Verificare che il metodo \texttt{serialize\_water\_level\_raw\_data()} serializzi correttamente \texttt{WaterLevelRawData}.                                       & NI             \\
	\hline
	\textbf{50-U}   & Verificare che il metodo \texttt{run()} della classe \texttt{Runner} esegua correttamente i simulatori.                                                           & NI             \\
	\hline
	\textbf{51-U}   & Verificare che il metodo \texttt{stream()} della classe \texttt{TemperatureSimulator} generi correttamente i dati casuali.                                        & NI             \\
	\hline
	\textbf{52-U}   & Verificare che il metodo \texttt{stream()} della classe \texttt{TrafficSimulator} generi correttamente i dati casuali.                                            & NI             \\
	\hline
	\textbf{53-U}   & Verificare che il metodo \texttt{stream()} della classe \texttt{RecyclingPointSimulator} generi correttamente i dati casuali.                                     & NI             \\
	\hline
	\textbf{54-U}   & Verificare che il metodo \texttt{stream()} della classe \texttt{HumiditySimulator} generi correttamente i dati casuali.                                           & NI             \\
	\hline
	\textbf{55-U}   & Verificare che il metodo \texttt{stream()} della classe \texttt{AirQualitySimulator} generi correttamente i dati casuali.                                         & NI             \\
	\hline
	\textbf{56-U}   & Verificare che il metodo \texttt{stream()} della classe \texttt{RainSimulator} generi correttamente i dati casuali.                                               & NI             \\
	\hline
	\textbf{57-U}   & Verificare che il metodo \texttt{stream()} della classe \texttt{ChargingStationSimulator} generi correttamente i dati casuali.                                    & NI             \\
	\hline
	\textbf{58-U}   & Verificare che il metodo \texttt{stream()} della classe \texttt{ParkingLotSimulator} generi correttamente i dati casuali.                                         & NI             \\
	\hline
	\textbf{59-U}   & Verificare che il metodo \texttt{stream()} della classe \texttt{WaterLevelSimulator} generi correttamente i dati casuali.                                         & NI             \\
	\hline
	\caption{Test di Unità}
\end{longtable}

\subsection{Test di Integrazione}
I test di integrazione verificano il corretto funzionamento delle interfacce tra le varie unità di codice,
assicurandosi che esse interagiscano correttamente tra di loro e che siano in grado di comunicare e scambiarsi i dati necessari. \\
\begin{longtable}{|>{\raggedright\arraybackslash}m{0.1\textwidth}|>{\raggedright\arraybackslash}m{0.6\textwidth}|>{\raggedright\arraybackslash}m{0.1\textwidth}|}
	\hline
	\textbf{Codice} & \textbf{Descrizione}                                                                                                                 & \textbf{Stato} \\
	\hline
	\endfirsthead
	\hline
	\textbf{Codice} & \textbf{Descrizione}                                                                                                                 & \textbf{Stato} \\
	\endhead
	\textbf{1T-I}   & Verificare che i dati generati dal sensore di temperatura siano pubblicati correttamente nel rispettivo topic Redpanda.              & NI             \\
	\hline
	\textbf{2T-I}   & Verificare che i dati generati dal sensore di traffico siano pubblicati correttamente nel rispettivo topic Redpanda.                 & NI             \\
	\hline
	\textbf{3T-I}   & Verificare che i dati generati dal sensore di isola ecologica siano pubblicati correttamente nel rispettivo topic Redpanda.          & NI             \\
	\hline
	\textbf{4T-I}   & Verificare che i dati generati dal sensore di umidità siano pubblicati correttamente nel rispettivo topic Redpanda.                  & NI             \\
	\hline
	\textbf{5T-I}   & Verificare che i dati generati dal sensore di qualità dell'aria siano pubblicati correttamente nel rispettivo topic Redpanda.        & NI             \\
	\hline
	\textbf{6T-I}   & Verificare che i dati generati dal sensore di precipitazioni siano pubblicati correttamente nel rispettivo topic Redpanda.           & NI             \\
	\hline
	\textbf{7T-I}   & Verificare che i dati generati dalle colonnine di ricarica siano pubblicati correttamente nel rispettivo topic Redpanda.             & NI             \\
	\hline
	\textbf{8T-I}   & Verificare che i dati generati dai sensori di occupazione di parcheggi siano pubblicati correttamente nel rispettivo topic Redpanda. & NI             \\
	\hline
	\textbf{9T-I}   & Verificare che i dati generati dai sensori di livello dell'acqua siano pubblicati correttamente nel rispettivo topic Redpanda.       & NI             \\
	\hline
	\textbf{10T-I}  & Verificare che i dati generati dal sensore di temperatura siano memorizzati correttamente nel database.                              & NI             \\
	\hline
	\textbf{11T-I}  & Verificare che i dati generati dal sensore di temperatura aggregati per 5 minuti siano memorizzati correttamente nel database.       & NI             \\
	\hline
	\textbf{12T-I}  & Verificare che i dati generati dal sensore di temperatura aggregati per settimana siano memorizzati correttamente nel database.      & NI             \\
	\hline
	\textbf{13T-I}  & Verificare che i dati generati dal sensore di temperatura aggregati per giorno siano memorizzati correttamente nel database.         & NI             \\
	\hline
	\textbf{14T-I}  & Verificare che i dati generati dal sensore di traffico siano memorizzati correttamente nel database.                                 & NI             \\
	\hline
	\textbf{15T-I}  & Verificare che i dati generati dal sensore di traffico aggregati per 5 minuti siano memorizzati correttamente nel database.          & NI             \\
	\hline
	\textbf{16T-I}  & Verificare che i dati generati dal sensore di traffico aggregati per ora siano memorizzati correttamente nel database.               & NI             \\
	\hline
	\textbf{17T-I}  & Verificare che i dati generati dal sensore di isola ecologica siano memorizzati correttamente nel database.                          & NI             \\
	\hline
	\textbf{18T-I}  & Verificare che i dati generati dal sensore di isola ecologica aggregati per 5 minuti siano memorizzati correttamente nel database.   & NI             \\
	\hline
	\textbf{19T-I}  & Verificare che i dati generati dal sensore di umidità siano memorizzati correttamente nel database.                                  & NI             \\
	\hline
	\textbf{20T-I}  & Verificare che i dati generati dal sensore di qualità dell'aria siano memorizzati correttamente nel database.                        & NI             \\
	\hline
	\textbf{21T-I}  & Verificare che i dati generati dal sensore di precipitazioni siano memorizzati correttamente nel database.                           & NI             \\
	\hline
	\textbf{22T-I}  & Verificare che i dati generati dalle colonnine di ricarica siano memorizzati correttamente nel database.                             & NI             \\
	\hline
	\textbf{23T-I}  & Verificare che i dati generati dai sensori di occupazione di parcheggi siano memorizzati correttamente nel database.                 & NI             \\
	\hline
	\textbf{24T-I}  & Verificare che i dati generati dai sensori di livello dell'acqua siano memorizzati correttamente nel database.                       & NI             \\
	\hline
	\textbf{25T-I}  & Verificare che i dati salvati su Clickhouse siano correttamente accessibili da Grafana.                                              & NI             \\
	\hline
	\caption{Test di Integrazione}
\end{longtable}

\newpage
\subsection{Test di Sistema}
I test di sistema sono finalizzati alla verifica del soddisfacimento dei requisiti richiesti ed evidenziati nel documento \href{https://7last.github.io/docs/rtb/documentazione-esterna/analisi-dei-requisiti}{\href{https://7last.github.io/docs/rtb/documentazione-interna/glossario\#analisi-dei-requisiti}{\textit{Analisi dei Requisiti}\textsubscript{G}}}. Questi test vengono effettuati sul sistema nel suo complesso, per verificare che il software funzioni correttamente e che sia in grado di eseguire le operazioni richieste. \\
\begin{longtable}{|>{\raggedright\arraybackslash}m{0.1\textwidth}|>{\raggedright\arraybackslash}m{0.6\textwidth}|>{\raggedright\arraybackslash}m{0.1\textwidth}|}
	\hline
	\textbf{Codice} & \textbf{Descrizione}                                                                                                                                                                             & \textbf{Stato} \\
	\hline
	\endfirsthead
	\hline
	\textbf{Codice} & \textbf{Descrizione}                                                                                                                                                                             & \textbf{Stato} \\
	\endhead
	\textbf{1T-S}   & Verificare che l'accesso al sistema non richieda alcuna procedura di login e che sia direttamente accessibile dall'utente.                                                                       & NI             \\
	\hline
	\textbf{2T-S}   & Verificare che il prodotto non abbia alcuna sezione o funzionalità di amministrazione o gestione riservata.                                                                                      & NI             \\
	\hline
	\textbf{3T-S}   & Verificare che i sensori integrati producano una misurazione coerente con il tipo di sensore simulato.                                                                                           & NI             \\
	\hline
	\textbf{4T-S}   & Verificare che ogni misurazione inviata dal simulatore contenga l’identificativo del sensore, le misurazioni d'interesse e il timestamp.                                                         & NI             \\
	\hline
	\textbf{5T-S}   & Verificare che il sistema sia in grado di ricevere e memorizzare correttamente le misurazioni inviate dai sensori.                                                                               & NI             \\
	\hline
	\textbf{6T-S}   & Verificare che il sistema sia in grado di simulare almeno un sensore per rilevare la temperatura.                                                                                                & NI             \\
	\hline
	\textbf{7T-S}   & Verificare che il sistema sia in grado di simulare almeno un sensore per rilevare il traffico.                                                                                                   & NI             \\
	\hline
	\textbf{8T-S}   & Verificare che il sistema sia in grado di simulare almeno un sensore per rilevare il riempimento delle isole ecologiche.                                                                         & NI             \\
	\hline
	\textbf{9T-S}   & Verificare che il sistema sia in grado di simulare almeno un sensore per rilevare l'umidità.                                                                                                     & NI             \\
	\hline
	\textbf{10T-S}  & Verificare che il sistema sia in grado di simulare almeno un sensore per rilevare la qualità dell'aria.                                                                                          & NI             \\
	\hline
	\textbf{11T-S}  & Verificare che il sistema sia in grado di simulare almeno un sensore per rilevare le precipitazioni.                                                                                             & NI             \\
	\hline
	\textbf{12T-S}  & Verificare che il sistema sia in grado di simulare almeno un sensore per rilevare le colonnine di ricarica.                                                                                      & NI             \\
	\hline
	\textbf{13T-S}  & Verificare che il sistema sia in grado di simulare almeno un sensore per rilevare l'occupazione dei parcheggi.                                                                                   & NI             \\
	\hline
	\textbf{14T-S}  & Verificare che il sistema sia in grado di simulare almeno un sensore per rilevare il livello dell'acqua.                                                                                         & NI             \\
	\hline
	\textbf{14T-S}  & Verificare che ogni dato generato dai simulatori dei sensori sia strettamente correlato al dato successivo, garantendo una transizione realistica tra le misurazioni.                            & NI             \\
	\hline
	\textbf{15T-S}  & Verificare la facilità di comprensione e l'intuitività dell'interfaccia grafica, garantendo un'esperienza utente piacevole e soddisfacente.                                                      & NI             \\
	\hline
	\textbf{16T-S}  & Verificare che le dashboard si aggiornino quasi istantaneamente per riflettere i dati provenienti dai sensori entro un massimo di 15 secondi.                                                    & NI             \\
	\hline
	\textbf{17T-S}  & Verificare che la dashboard del traffico contenga almeno un \textit{panel} con un grafico time-series.                                                                                           & NI             \\
	\hline
	\textbf{18T-S}  & Verificare che la dashboard della temperatura contenga almeno un \textit{panel} con un grafico time-series.                                                                                      & NI             \\
	\hline
	\textbf{19T-S}  & Verificare che la dashboard delle isole ecologiche contenga almeno un \textit{panel} con un grafico time-series.                                                                                 & NI             \\
	\hline
	\textbf{20T-S}  & Verificare che la dashboard dell'umidità contenga almeno un \textit{panel} con un grafico time-series.                                                                                           & NI             \\
	\hline
	\textbf{21T-S}  & Verificare che la dashboard della qualità dell'aria contenga almeno un \textit{panel} con un grafico time-series.                                                                                & NI             \\
	\hline
	\textbf{22T-S}  & Verificare che la dashboard delle precipitazioni contenga almeno un \textit{panel} con un grafico time-series.                                                                                   & NI             \\
	\hline
	\textbf{23T-S}  & Verificare che la dashboard dei parcheggi contenga almeno un \textit{panel} con un grafico time-series.                                                                                          & NI             \\
	\hline
	\textbf{24T-S}  & Verificare che la dashboard delle colonnine di ricarica contenga almeno un \textit{panel} con un grafico time-series.                                                                            & NI             \\
	\hline
	\textbf{25T-S}  & Verificare che la dashboard del livello di acqua contenga almeno un \textit{panel} con un grafico time-series.                                                                                   & NI             \\
	\hline
	\textbf{26T-S}  & Verificare che la dashboard delle isole ecologiche contenga almeno un \textit{panel} con un grafico time-series.                                                                                 & NI             \\
	\hline
	\textbf{27T-S}  & Verificare che i sensori presenti sulla mappa siano distinguibili in modo chiaro ed inequivocabile, permettendo il riconoscimento della loro tipologia.                                          & NI             \\
	\hline
	\textbf{28T-S}  & Verificare che in ciascuna dashboard l’utente possa filtrare la visualizzazione delle misurazioni di uno specifico sensore.                                                                      & NI             \\
	\hline
	\textbf{29T-S}  & Verificare che nella dashboard dei dati grezzi l’utente possa visualizzare la lista delle misurazioni in un formato tabellare, divise per tipo di sensore.                                       & NI             \\ %da controllare
	\hline
	\textbf{30T-S}  & Verificare che l’utente riceva notifiche quando i sensori superano determinate soglie di sicurezza.                                                                                          & NI             \\
	\hline
	\textbf{31T-S}  & Verificare che l’utente possa visualizzare correttamente le coordinate dei sensori, con un numero congruo di cifre decimali.                                                                     & NI             \\
	\hline
	\textbf{32T-S}  & Verificare che l’utente possa visualizzare correttamente l’unità di misura associata a ciascuna misurazione.                                                                                     & NI             \\
	\hline
	\textbf{33T-S}  & Verificare che nella dashboard dei dati grezzi l'utente possa visualizzare una tabella contente l'indentificativo del sensore, la sua tipologia e la data dell'ultimo messaggio da esso inviato. & NI             \\
	\hline
	\caption{Test di Sistema}
\end{longtable}

% \subsection{Test di Regressione}
% I test di regressione sono test che vengono effettuati per verificare che le modifiche apportate al software non abbiano introdotto nuovi errori o problemi di funzionamento e che il software continui a funzionare correttamente anche dopo le modifiche fatte. \\
% \begin{longtable}{|>{\raggedright\arraybackslash}m{0.1\textwidth}|>{\raggedright\arraybackslash}m{0.6\textwidth}|>{\raggedright\arraybackslash}m{0.1\textwidth}|}
% 	\hline
% 	\textbf{Tipologia di test} & \textbf{Codice} & \textbf{Stato} \\
% 	\hline
% 	\endfirsthead
% 	\hline
% 	\textbf{Tipologia di test} & \textbf{Codice} & \textbf{Stato} \\
% 	\endhead
% 	\textbf{Test di unità}   		& Codice del test che andremo ad effettuare     & NI\\
% 	\hline
% 	\textbf{Test di sistema}  		& Codice del test che andremo ad effettuare     & NI\\
% 	\hline
% 	\textbf{Test di integrazione}   & Codice del test che andremo ad effettuare     & NI\\
% 	\hline
% 	\caption{Test di Regressione} % TODO aggiungere test effettivamente progettati
% 	\label{table:15}
% \end{longtable}
\subsection{Test di Accettazione}
I test di accettazione vengono effettuati per verificare che il software soddisfi i requisiti richiesti e consentono di ultimare il processo di validazione del prodotto finale.
Essi verranno eseguiti sia dal gruppo di sviluppo \textit{7Last} che dall'azienda \href{https://7last.github.io/docs/rtb/documentazione-interna/glossario\#proponente}{proponente\textsubscript{G}} \textit{SyncLab S.r.l.}. \\
\begin{longtable}{|>{\raggedright\arraybackslash}m{0.1\textwidth}|>{\raggedright\arraybackslash}m{0.6\textwidth}|>{\raggedright\arraybackslash}m{0.1\textwidth}|}
	\hline
	\textbf{Codice} & \textbf{Descrizione}                                                                                                   & \textbf{Stato} \\
	\hline
	\endfirsthead
	\hline
	\textbf{Codice} & \textbf{Descrizione}                                                                                                   & \textbf{Stato} \\
	\endhead
	\textbf{1T-A}   & Verificare che tutti i widget relativi alle diverse tipologie di sensori siano visibili sulla dashboard.               & NI             \\
	\hline
	\textbf{2T-A}   & Verificare che la mappa dei sensori si carichi correttamente e permetta interazioni fluide.                            & NI             \\
	\hline
	\textbf{3T-A}   & Verifica della gestione corretta degli errori nel caso in cui i dati dei sensori non siano disponibili.                & NI             \\
	\hline
	\textbf{4T-A}   & Verifica della corretta visualizzazione delle misurazioni effettuate nel tempo dai sensori.                            & NI             \\
	\hline
	\textbf{6T-A}   & Verificare che sia possibile visualizzare correttamente la dashboard dei sensori di temperatura.                       & NI             \\
	\hline
	\textbf{7T-A}   & Verificare che sia possibile visualizzare correttamente la dashboard dei sensori di traffico.                          & NI             \\
	\hline
	\textbf{8T-A}   & Verificare che sia possibile visualizzare correttamente la dashboard dei sensori di isola ecologica.                   & NI             \\
	\hline
	\textbf{9T-A}   & Verificare che sia possibile visualizzare correttamente la dashboard dei sensori di umidità.                           & NI             \\
	\hline
	\textbf{10T-A}  & Verificare che sia possibile visualizzare correttamente la dashboard dei sensori di qualità dell'aria.                 & NI             \\
	\hline
	\textbf{11T-A}  & Verificare che sia possibile visualizzare correttamente la dashboard dei sensori di precipitazioni.                    & NI             \\
	\hline
	\textbf{12T-A}  & Verificare che sia possibile visualizzare correttamente la dashboard dei sensori di colonnine di ricarica.             & NI             \\
	\hline
	\textbf{13T-A}  & Verificare che sia possibile visualizzare correttamente la dashboard dei sensori di occupazione di parcheggi.          & NI             \\
	\hline
	\textbf{14T-A}  & Verificare che sia possibile visualizzare correttamente la dashboard dei sensori di livello dell'acqua.                & NI             \\
	\hline
	\textbf{15T-A}  & Verificare che sia possibile visualizzare correttamente la dashboard dei dati grezzi                                   & NI             \\
	\hline
	\textbf{16T-A}  & Verificare si possa filtrare correttamente la visualizzazione delle misurazioni in base al sensore che le ha prodotte. & NI             \\
	\hline
	\textbf{17T-A}  & Verificare che si possa rimuovere correttamente i filtri attivi per visualizzazione delle misurazioni dei sensori.     & NI             \\
	\hline
	\textbf{18T-A}  & Verificare che si riceva correttamente una notifica in caso di superamento delle soglie impostate per le misurazioni.  & NI             \\
	\hline
	\caption{Test di Accettazione}
\end{longtable}


\definecolor{opt}{HTML}{51B824}
\definecolor{amm}{HTML}{CC0000}

\section{Cruscotto di valutazione della qualità}
\subsection{Qualità del processo di Fornitura}
\subsubsection{1M-PV - Planned Value e 2M-EV - Earned Value}
%--------- GRAFICO -----------%
\begin{figure*}[!h]
    \centering
    \begin{tikzpicture}
        \begin{axis}[
            width  = 0.85*\textwidth,
            height = 8cm,
            ymajorgrids = true,
            symbolic x coords={Sprint 1, Sprint 2, Sprint 3, Sprint 4},
            xtick = data,
            ymin=0,
            axis lines*=left,
            legend cell align=left,
            legend style={
                at={(0.5,1.15)},
                anchor=south,
                column sep=0.1ex,
                legend columns=3
            },      
            xticklabel style={rotate=45, anchor=north east, yshift=0ex, xshift=0ex},
            scaled y ticks = false,
            yticklabel style={/pgf/number format/fixed}
            ]
            \addplot[color=opt, style={dashed, line width=3pt}, mark=none] coordinates {(Sprint 1, 12567.5) (Sprint 2, 12605) (Sprint 3, 12735) (Sprint 4, 12650)}; % TODO mettere valori corretti EAC
            \addplot[color=amm, style={dashed, line width=3pt}, mark=none] coordinates {(Sprint 1, 1090) (Sprint 2, 2107.5) (Sprint 3, 2737.5) (Sprint 4, 3442.5)}; % TODO mettere valori corretti PV
            \addplot[color=blue, style={line width=1pt}, mark=none] coordinates {(Sprint 1, 977.5) (Sprint 2, 2032.5) (Sprint 3, 2792.5) (Sprint 4, 3412.5)}; % TODO mettere valori corretti EV
            \legend{EAC, PV, EV}
        \end{axis}
    \end{tikzpicture}
    \caption{Proiezione del PV e dell'EV}
\end{figure*}
%--------- FINE GRAFICO -----------%
%\href{https://7last.github.io/docs/rtb/documentazione-interna/glossario\#requirements-and-technology-baseline}{\textbf{RTB}\textsubscript{G}} \\
% TODO considerazioni finali per \href{https://7last.github.io/docs/rtb/documentazione-interna/glossario\#requirements-and-technology-baseline}{RTB\textsubscript{G}} \\
%\href{https://7last.github.io/docs/rtb/documentazione-interna/glossario\#product-baseline}{\textbf{PB}\textsubscript{G}} \\
% TODO considerazioni finali per \href{https://7last.github.io/docs/rtb/documentazione-interna/glossario\#product-baseline}{PB\textsubscript{G}}

\newpage
\subsubsection{3M-AC - Actual Cost  e 9M-ETC - Estimate to Complete}
%--------- GRAFICO -----------%
\begin{figure*}[!h]
    \centering
    \begin{tikzpicture}
        \begin{axis}[
            width  = 0.85*\textwidth,
            height = 8cm,
            ymajorgrids = true,
            symbolic x coords={Sprint 1, Sprint 2, Sprint 3, Sprint 4},
            xtick = data,
            ymin=0,
            axis lines*=left,
            legend cell align=left,
            legend style={
                at={(0.5,1.15)},
                anchor=south,
                column sep=0.1ex,
                legend columns=3
            },      
            xticklabel style={rotate=45, anchor=north east, yshift=0ex, xshift=0ex},
            scaled y ticks = false,
            yticklabel style={/pgf/number format/fixed}
            ]
            \addplot[color=opt, style={dashed, line width=3pt}, mark=none] coordinates {(Sprint 1, 12567.5) (Sprint 2, 12605) (Sprint 3, 12735) (Sprint 4, 12650)}; 
            \addplot[color=amm, style={dashed, line width=3pt}, mark=none] coordinates {(Sprint 1, 11590) (Sprint 2, 11550) (Sprint 3, 11975) (Sprint 4, 12030)};
            \addplot[color=blue, style={line width=1pt}, mark=none] coordinates {(Sprint 1, 977.5) (Sprint 2, 2032.5) (Sprint 3, 2792.5) (Sprint 4, 3412.5)};
            \legend{EAC, ETC, AC}
        \end{axis}
    \end{tikzpicture}
    \caption{Proiezione dell'AC e dell'ETC}
\end{figure*}
%--------- FINE GRAFICO -----------%
%\href{https://7last.github.io/docs/rtb/documentazione-interna/glossario\#requirements-and-technology-baseline}{\textbf{RTB}\textsubscript{G}} \\
% TODO considerazioni finali per \href{https://7last.github.io/docs/rtb/documentazione-interna/glossario\#requirements-and-technology-baseline}{RTB\textsubscript{G}} \\
%\href{https://7last.github.io/docs/rtb/documentazione-interna/glossario\#product-baseline}{\textbf{PB}\textsubscript{G}} \\
% TODO considerazioni finali per \href{https://7last.github.io/docs/rtb/documentazione-interna/glossario\#product-baseline}{PB\textsubscript{G}}

\newpage
\subsubsection{4M-SV - Schedule Variance e 5M-CV - Cost Variance}
%--------- GRAFICO -----------%
\begin{figure*}[!h]
    \centering
    \begin{tikzpicture}
        \begin{axis}[
            width  = 0.85*\textwidth,
            height = 8cm,
            ymajorgrids = true,
            symbolic x coords={Sprint 1, Sprint 2, Sprint 3, Sprint 4},
            xtick = data,
            axis lines*=left,
            legend cell align=left,
            legend style={
                at={(0.5,1.15)},
                anchor=south,
                column sep=0.1ex,
                legend columns=3
            },      
            xticklabel style={rotate=45, anchor=north east, yshift=0ex, xshift=0ex}
            ]
            \addplot[color=opt, style={dashed, line width=3pt}, mark=none] coordinates {(Sprint 1, 0) (Sprint 2, 0) (Sprint 3, 0) (Sprint 4, 0)};
            \addplot[color=amm, style={dashed, line width=3pt}, mark=none] coordinates {(Sprint 1, -10) (Sprint 2, -10) (Sprint 3, -10) (Sprint 4, -10)};
            \addplot[color=blue, style={line width=1pt}, mark=none] coordinates {(Sprint 1, 25) (Sprint 2, -9.09) (Sprint 3, 7.69) (Sprint 4, 0)};
            \addplot[color=yellow, style={line width=1pt}, mark=none] coordinates {(Sprint 1, 51.82) (Sprint 2, 27.14) (Sprint 3, 15.30) (Sprint 4, 10.29)};
            \legend{Valore ottimo, Valore ammissibile, SV, CV}
        \end{axis}
    \end{tikzpicture}
    \caption{Andamento percentuale di SV e CV}
\end{figure*}
%--------- FINE GRAFICO -----------%
%\href{https://7last.github.io/docs/rtb/documentazione-interna/glossario\#requirements-and-technology-baseline}{\textbf{RTB}\textsubscript{G}} \\
% TODO considerazioni finali per \href{https://7last.github.io/docs/rtb/documentazione-interna/glossario\#requirements-and-technology-baseline}{RTB\textsubscript{G}} \\
%\href{https://7last.github.io/docs/rtb/documentazione-interna/glossario\#product-baseline}{\textbf{PB}\textsubscript{G}} \\
% TODO considerazioni finali per \href{https://7last.github.io/docs/rtb/documentazione-interna/glossario\#product-baseline}{PB\textsubscript{G}}

\newpage
\subsubsection{8M-EAC - Estimated at Completion}
%--------- GRAFICO -----------%
\begin{figure*}[!h]
    \centering
    \begin{tikzpicture}
        \begin{axis}[
            width  = 0.85*\textwidth,
            height = 8cm,
            ymajorgrids = true,
            symbolic x coords={Sprint 1, Sprint 2, Sprint 3, Sprint 4},
            xtick = data,
            ytick = {10000, 11000, 12000, 13000, 14000, 15000},
            ymin=10000, ymax=15000,
            axis lines*=left,
            legend cell align=left,
            legend style={
                at={(0.5,1.15)},
                anchor=south,
                column sep=0.1ex,
                legend columns=3
            },      
            xticklabel style={rotate=45, anchor=north east, yshift=0ex, xshift=0ex},
            scaled y ticks = false,
            yticklabel style={/pgf/number format/fixed}
            ]
            \addplot[color=opt, style={dashed, line width=3pt}, mark=none] coordinates {(Sprint 1, 12680) (Sprint 2, 12670) (Sprint 3, 12680) (Sprint 4, 12680) }; 
            \addplot[color=amm, style={dashed, line width=3pt}, mark=none] coordinates {(Sprint 1, 13314) (Sprint 2, 13314) (Sprint 3, 13314) (Sprint 4, 13314)};
            \addplot[color=blue, style={line width=1pt}, mark=none] coordinates {(Sprint 1, 12567.5) (Sprint 2, 12605) (Sprint 3, 12735) (Sprint 4, 12650)};
            \legend{Valore ottimo (BAC), Valore ammissibile, Valore effettivo}
        \end{axis}
    \end{tikzpicture}
    \caption{Proiezione dell'EAC}
\end{figure*}
%--------- FINE GRAFICO -----------%
%\href{https://7last.github.io/docs/rtb/documentazione-interna/glossario\#requirements-and-technology-baseline}{\textbf{RTB}\textsubscript{G}} \\
% TODO considerazioni finali per \href{https://7last.github.io/docs/rtb/documentazione-interna/glossario\#requirements-and-technology-baseline}{RTB\textsubscript{G}} \\
%\href{https://7last.github.io/docs/rtb/documentazione-interna/glossario\#product-baseline}{\textbf{PB}\textsubscript{G}} \\
% TODO considerazioni finali per \href{https://7last.github.io/docs/rtb/documentazione-interna/glossario\#product-baseline}{PB\textsubscript{G}}

\newpage

\subsection{Qualità del processo di Analisi dei requisiti}
\subsubsection{11M-PRO - Percentuale Requisiti Obbligatori}
%--------- GRAFICO -----------%
\begin{figure*}[!h]
    \centering
    \begin{tikzpicture}
        \begin{axis}[
            width  = 0.85*\textwidth,
            height = 8cm,
            ymajorgrids = true,
            symbolic x coords={Sprint 1, Sprint 2, Sprint 3, Sprint 4},
            xtick = data,
            ytick = {0, 25, 50, 75, 100},
            ymin=0, ymax=100,
            axis lines*=left,
            legend cell align=left,
            legend style={
                at={(0.5,1.15)},
                anchor=south,
                column sep=0.1ex,
                legend columns=3
            },      
            xticklabel style={rotate=45, anchor=north east, yshift=0ex, xshift=0ex}
            ]
            \addplot[color=opt, style={dashed, line width=3pt}, mark=none] coordinates {(Sprint 1, 100) (Sprint 2, 100) (Sprint 3, 100) (Sprint 4, 100)};
            \addplot[color=blue, style={line width=1pt}, mark=none] coordinates {(Sprint 1, 25) (Sprint 2, 48) (Sprint 3, 52) (Sprint 4, 64)};
            \legend{Valore ottimo ed ammissibile, Valore effettivo}
        \end{axis}
    \end{tikzpicture}
    %44 requisiti obbligatori, 11 implementati nel primo sprint, 21 nel secondo e 24 nel terzo, 28 nel quarto
    \caption{Percentuale di copertura dei requisiti obbligatori}
\end{figure*}
%--------- FINE GRAFICO -----------%

%\href{https://7last.github.io/docs/rtb/documentazione-interna/glossario\#requirements-and-technology-baseline}{\textbf{RTB}\textsubscript{G}} \\
% TODO considerazioni finali per \href{https://7last.github.io/docs/rtb/documentazione-interna/glossario\#requirements-and-technology-baseline}{RTB\textsubscript{G}} \\
%\href{https://7last.github.io/docs/rtb/documentazione-interna/glossario\#product-baseline}{\textbf{PB}\textsubscript{G}} \\
% TODO considerazioni finali per \href{https://7last.github.io/docs/rtb/documentazione-interna/glossario\#product-baseline}{PB\textsubscript{G}}

\newpage
\subsubsection{12M-PRD - Percentuale Requisiti Desiderabili}
%--------- GRAFICO -----------%
\begin{figure*}[!h]
    \centering
    \begin{tikzpicture}
        \begin{axis}[
            width  = 0.85*\textwidth,
            height = 8cm,
            ymajorgrids = true,
            symbolic x coords={Sprint 1, Sprint 2, Sprint 3, Sprint 4},
            xtick = data,
            ytick = {0, 25, 50, 75, 100},
            ymin=0, ymax=100,
            axis lines*=left,
            legend cell align=left,
            legend style={
                at={(0.5,1.15)},
                anchor=south,
                column sep=0.1ex,
                legend columns=3
            },      
            xticklabel style={rotate=45, anchor=north east, yshift=0ex, xshift=0ex}
            ]
            \addplot[color=opt, style={dashed, line width=3pt}, mark=none] coordinates {(Sprint 1, 100) (Sprint 2, 100) (Sprint 3, 100) (Sprint 4, 100)};
            \addplot[color=amm, style={dashed, line width=3pt}, mark=none] coordinates {(Sprint 1, 35) (Sprint 2, 35) (Sprint 3, 35) (Sprint 4, 35)};
            \addplot[color=blue, style={line width=1pt}, mark=none] coordinates {(Sprint 1, 0) (Sprint 2, 20) (Sprint 3, 40) (Sprint 4, 60)};
            \legend{Valore ottimo, Valore ammissibile, Valore effettivo}
        \end{axis}
    \end{tikzpicture}
    \caption{Percentuale di copertura dei requisiti desiderabili}
\end{figure*}
%5 requisiti desiderabili; 0 nel primo sprint, 1 nel secondo,1 nel terzo, 1 nel quarto
%--------- FINE GRAFICO -----------%
%\href{https://7last.github.io/docs/rtb/documentazione-interna/glossario\#requirements-and-technology-baseline}{\textbf{RTB}\textsubscript{G}} \\
% TODO considerazioni finali per \href{https://7last.github.io/docs/rtb/documentazione-interna/glossario\#requirements-and-technology-baseline}{RTB\textsubscript{G}} \\
%\href{https://7last.github.io/docs/rtb/documentazione-interna/glossario\#product-baseline}{\textbf{PB}\textsubscript{G}} \\
% TODO considerazioni finali per \href{https://7last.github.io/docs/rtb/documentazione-interna/glossario\#product-baseline}{PB\textsubscript{G}}

\newpage
\subsubsection{13M-PRO - Percentuale Requisiti Opzionali}
%--------- GRAFICO -----------%
\begin{figure*}[!h]
    \centering
    \begin{tikzpicture}
        \begin{axis}[
            width  = 0.85*\textwidth,
            height = 8cm,
            ymajorgrids = true,
            symbolic x coords={Sprint 1, Sprint 2, Sprint 3, Sprint 4},
            xtick = data,
            ytick = {0, 25, 50, 75, 100},
            ymin=0, ymax=100,
            axis lines*=left,
            legend cell align=left,
            legend style={
                at={(0.5,1.15)},
                anchor=south,
                column sep=0.1ex,
                legend columns=3
            },      
            xticklabel style={rotate=45, anchor=north east, yshift=0ex, xshift=0ex}
            ]
            \addplot[color=opt, style={dashed, line width=3pt}, mark=none] coordinates {(Sprint 1, 100) (Sprint 2, 100) (Sprint 3, 100) (Sprint 4, 100)};
            \addplot[color=amm, style={dashed, line width=3pt}, mark=none] coordinates {(Sprint 1, 0) (Sprint 2, 0) (Sprint 3, 0) (Sprint 4, 0)};
            \addplot[color=blue, style={line width=1pt}, mark=none] coordinates {(Sprint 1, 0) (Sprint 2, 0) (Sprint 3, 0) (Sprint 4, 33)};
            \legend{Valore ottimo, Valore ammissibile, Valore effettivo}
        \end{axis}
    \end{tikzpicture}
    \caption{Percentuale di copertura dei requisiti opzionali} 
    %3 requisiti opzionali, 0 primo sprint, 0 secondo sprint, 0 terzo sprint 1 quarto sprint
\end{figure*}
%--------- FINE GRAFICO -----------%
%\href{https://7last.github.io/docs/rtb/documentazione-interna/glossario\#requirements-and-technology-baseline}{\textbf{RTB}\textsubscript{G}} \\
% TODO considerazioni finali per \href{https://7last.github.io/docs/rtb/documentazione-interna/glossario\#requirements-and-technology-baseline}{RTB\textsubscript{G}} \\
%\href{https://7last.github.io/docs/rtb/documentazione-interna/glossario\#product-baseline}{\textbf{PB}\textsubscript{G}} \\
% TODO considerazioni finali per \href{https://7last.github.io/docs/rtb/documentazione-interna/glossario\#product-baseline}{PB\textsubscript{G}}



\newpage
\subsection{Qualità del processo di Documentazione}
\subsubsection{19M-IG - Indice Gulpease}
%--------- GRAFICO -----------%
\begin{figure*}[!h]
    \centering
    \begin{tikzpicture}
        \begin{axis}[
            width  = 0.85*\textwidth,
            height = 8cm,
            ymajorgrids = true,
            symbolic x coords={Sprint 1, Sprint 2, Sprint 3, Sprint 4},
            xtick = data,
            ytick = {0, 25, 50, 75, 100},
            ymin=0, ymax=100,
            axis lines*=left,
            legend cell align=left,
            legend style={
                at={(0.5,1.15)},
                anchor=south,
                column sep=0.1ex,
                legend columns=3
            },           
            xticklabel style={rotate=45, anchor=north east, yshift=0ex, xshift=0ex}
            ]
            \addplot[color=opt, style={dashed, line width=3pt}, mark=none] coordinates {(Sprint 1, 75) (Sprint 2, 75) (Sprint 3, 75) (Sprint 4, 75)};
            \addplot[color=amm, style={dashed, line width=3pt}, mark=none] coordinates {(Sprint 1, 50) (Sprint 2, 50) (Sprint 3, 50) (Sprint 4, 50)};
            \addplot[color=cyan, style={line width=1pt}, mark=none] coordinates {(Sprint 1, 45) (Sprint 2, 55) (Sprint 3, 65) (Sprint 4, 60)}; 
            \addplot[color=magenta, style={line width=1pt}, mark=none] coordinates {(Sprint 1, 70) (Sprint 2, 40) (Sprint 3, 60) (Sprint 4, 50)}; 
            \addplot[color=lime, style={line width=1pt}, mark=none] coordinates {(Sprint 1, 40) (Sprint 2, 35) (Sprint 3, 45) (Sprint 4, 55)}; 
            \addplot[color=brown, style={line width=1pt}, mark=none] coordinates {(Sprint 1, 65) (Sprint 2, 45) (Sprint 3, 40) (Sprint 4, 60)}; 
            \addplot[color=pink, style={line width=1pt}, mark=none] coordinates {(Sprint 1, 75) (Sprint 2, 60) (Sprint 3, 55) (Sprint 4, 63)}; 
            \addplot[color=violet, style={line width=1pt}, mark=none] coordinates {(Sprint 1, 0) (Sprint 2, 0) (Sprint 3, 0) (Sprint 4, 0)}; 
            \addplot[color=yellow, style={line width=1pt}, mark=none] coordinates {(Sprint 1, 0) (Sprint 2, 0) (Sprint 3, 0) (Sprint 4, 0)}; 
            \legend{Valore ottimo, Valore ammissibile, NdP, PdP, PdQ, Glossario, AdR, MU, ST}
        \end{axis}
    \end{tikzpicture}
    \caption{Andamento indice di Gulpease per ciascun documento}
\end{figure*}
%--------- FINE GRAFICO -----------%
%\href{https://7last.github.io/docs/rtb/documentazione-interna/glossario\#requirements-and-technology-baseline}{\textbf{RTB}\textsubscript{G}} \\
% TODO considerazioni finali per \href{https://7last.github.io/docs/rtb/documentazione-interna/glossario\#requirements-and-technology-baseline}{RTB\textsubscript{G}} \\
%\href{https://7last.github.io/docs/rtb/documentazione-interna/glossario\#product-baseline}{\textbf{PB}\textsubscript{G}} \\
% TODO considerazioni finali per \href{https://7last.github.io/docs/rtb/documentazione-interna/glossario\#product-baseline}{PB\textsubscript{G}}

\newpage
\subsubsection{20M-CO - Correttezza Ortografica}
%--------- GRAFICO -----------%
\begin{figure*}[!h]
    \centering
    \begin{tikzpicture}
        \begin{axis}[
            width  = 0.85*\textwidth,
            height = 8cm,
            ymajorgrids = true,
            symbolic x coords={Sprint 1, Sprint 2, Sprint 3, Sprint 4},
            xtick = data,
            ymin=0, ymax=10,
            axis lines*=left,
            legend cell align=left,
            legend style={
                at={(0.5,1.15)},
                anchor=south,
                column sep=0.1ex,
                legend columns=3
            },           
            xticklabel style={rotate=45, anchor=north east, yshift=0ex, xshift=0ex}
            ]
            \addplot[color=opt, style={dashed, line width=3pt}, mark=none] coordinates {(Sprint 1, 0) (Sprint 2, 0) (Sprint 3, 0) (Sprint 4, 0)};
            \addplot[color=cyan, style={line width=1pt}, mark=none] coordinates {(Sprint 1, 9) (Sprint 2, 5) (Sprint 3, 5) (Sprint 4, 0)}; 
            \addplot[color=magenta, style={line width=1pt}, mark=none] coordinates {(Sprint 1, 8) (Sprint 2, 7) (Sprint 3, 4) (Sprint 4, 0)}; 
            \addplot[color=lime, style={line width=1pt}, mark=none] coordinates {(Sprint 1, 9) (Sprint 2, 3) (Sprint 3, 3) (Sprint 4,0)}; 
            \addplot[color=brown, style={line width=1pt}, mark=none] coordinates {(Sprint 1, 0) (Sprint 2, 0) (Sprint 3, 0) (Sprint 4, 0)}; 
            \addplot[color=pink, style={line width=1pt}, mark=none] coordinates {(Sprint 1, 0) (Sprint 2, 6) (Sprint 3, 3) (Sprint 4, 0)}; 
            \addplot[color=violet, style={line width=1pt}, mark=none] coordinates {(Sprint 1, 0) (Sprint 2, 0) (Sprint 3, 0) (Sprint 4, 0)}; 
            \addplot[color=yellow, style={line width=1pt}, mark=none] coordinates {(Sprint 1, 0) (Sprint 2, 0) (Sprint 3, 0) (Sprint 4, 0)}; 
            \legend{Valore ottimo ed ammissibile, NdP, PdP, PdQ, Glossario, AdR, MU, ST}
        \end{axis}
    \end{tikzpicture}
    \caption{Errori ortografici per ciascun documento}
\end{figure*}
%--------- FINE GRAFICO -----------%
%\href{https://7last.github.io/docs/rtb/documentazione-interna/glossario\#requirements-and-technology-baseline}{\textbf{RTB}\textsubscript{G}} \\
% TODO considerazioni finali per \href{https://7last.github.io/docs/rtb/documentazione-interna/glossario\#requirements-and-technology-baseline}{RTB\textsubscript{G}} \\
%\href{https://7last.github.io/docs/rtb/documentazione-interna/glossario\#product-baseline}{\textbf{PB}\textsubscript{G}} \\
% TODO considerazioni finali per \href{https://7last.github.io/docs/rtb/documentazione-interna/glossario\#product-baseline}{PB\textsubscript{G}}


\newpage
\subsection{Qualità del processo di Gestione della qualità}
\subsubsection{25M-QMS - Metriche di Qualità Soddisfatte}
%--------- GRAFICO -----------%
\begin{figure*}[!h]
    \centering
    \begin{tikzpicture}
        \begin{axis}[
            width  = 0.85*\textwidth,
            height = 8cm,
            ymajorgrids = true,
            symbolic x coords={Sprint 1, Sprint 2, Sprint 3, Sprint 4},
            xtick = data,
            ytick = {0, 25, 50, 75, 100},
            ymin=0, ymax=100,
            axis lines*=left,
            legend cell align=left,
            legend style={
                at={(0.5,1.15)},
                anchor=south,
                column sep=0.1ex,
                legend columns=3
            },      
            xticklabel style={rotate=45, anchor=north east, yshift=0ex, xshift=0ex}
            ]
            \addplot[color=opt, style={dashed, line width=3pt}, mark=none] coordinates {(Sprint 1, 100) (Sprint 2, 100) (Sprint 3, 100) (Sprint 4, 100)};
            \addplot[color=amm, style={dashed, line width=3pt}, mark=none] coordinates {(Sprint 1, 90) (Sprint 2, 90) (Sprint 3, 90) (Sprint 4, 90)};
            \addplot[color=blue, style={line width=1pt}, mark=none] coordinates {(Sprint 1, 100) (Sprint 2, 85) (Sprint 3, 100) (Sprint 4, 100)};
            \legend{Valore ottimo, Valore ammissibile, Valore effettivo}
        \end{axis}
    \end{tikzpicture}
    \caption{Percentuale di metriche di qualità soddisfatte}
\end{figure*}
%--------- FINE GRAFICO -----------%
%\href{https://7last.github.io/docs/rtb/documentazione-interna/glossario\#requirements-and-technology-baseline}{\textbf{RTB}\textsubscript{G}} \\
% TODO considerazioni finali per \href{https://7last.github.io/docs/rtb/documentazione-interna/glossario\#requirements-and-technology-baseline}{RTB\textsubscript{G}} \\
%\href{https://7last.github.io/docs/rtb/documentazione-interna/glossario\#product-baseline}{\textbf{PB}\textsubscript{G}} \\
% TODO considerazioni finali per \href{https://7last.github.io/docs/rtb/documentazione-interna/glossario\#product-baseline}{PB\textsubscript{G}}


\newpage
\subsection{Qualità del processo di Verifica}
\subsubsection{26M-CC - Code coverage}
%Da qui in poi byteops fanno da "periodo 6"
%--------- GRAFICO -----------%
\begin{figure*}[!h]
    \centering
    \begin{tikzpicture}
        \begin{axis}[
            width  = 0.85*\textwidth,
            height = 8cm,
            ymajorgrids = true,
            symbolic x coords={Sprint 1, Sprint 2, Sprint 3, Sprint 4},
            xtick = data,
            ytick = {0, 25, 50, 75, 100},
            ymin=0, ymax=100,
            axis lines*=left,
            legend cell align=left,
            legend style={
                at={(0.5,1.15)},
                anchor=south,
                column sep=0.1ex,
                legend columns=-1
            },           
            xticklabel style={rotate=45, anchor=north east, yshift=0ex, xshift=0ex}
            ]
            \addplot[color=opt, style={dashed, line width=3pt}, mark=none] coordinates {(Sprint 1, 100) (Sprint 2, 100) (Sprint 3, 100) (Sprint 4, 100)};
            \addplot[color=amm, style={dashed, line width=3pt}, mark=none] coordinates {(Sprint 1, 80) (Sprint 2, 80) (Sprint 3, 80) (Sprint 4, 80)};
            \addplot[color=blue, style={line width=1pt}, mark=none] coordinates {(Sprint 1, 0) (Sprint 2, 0) (Sprint 3, 0) (Sprint 4, 0)};
            \legend{Valore ottimo, Valore ammissibile, Valore effettivo}
        \end{axis}
    \end{tikzpicture}
    \caption{Percentuale di code coverage dei test implementati}
\end{figure*}
%--------- FINE GRAFICO -----------%
%\href{https://7last.github.io/docs/rtb/documentazione-interna/glossario\#requirements-and-technology-baseline}{\textbf{RTB}\textsubscript{G}} \\
% TODO considerazioni finali per \href{https://7last.github.io/docs/rtb/documentazione-interna/glossario\#requirements-and-technology-baseline}{RTB\textsubscript{G}} \\
%\href{https://7last.github.io/docs/rtb/documentazione-interna/glossario\#product-baseline}{\textbf{PB}\textsubscript{G}} \\
% TODO considerazioni finali per \href{https://7last.github.io/docs/rtb/documentazione-interna/glossario\#product-baseline}{PB\textsubscript{G}}

\newpage
\subsubsection{27M-BC - Branch coverage}
%--------- GRAFICO -----------%
\begin{figure*}[!h]
    \centering
    \begin{tikzpicture}
        \begin{axis}[
            width  = 0.85*\textwidth,
            height = 8cm,
            ymajorgrids = true,
            symbolic x coords={Sprint 1, Sprint 2, Sprint 3, Sprint 4},
            xtick = data,
            ytick = {0, 25, 50, 75, 100},
            ymin=0, ymax=100,
            axis lines*=left,
            legend cell align=left,
            legend style={
                at={(0.5,1.15)},
                anchor=south,
                column sep=0.1ex,
                legend columns=3
            },      
            xticklabel style={rotate=45, anchor=north east, yshift=0ex, xshift=0ex}
            ]
            \addplot[color=opt, style={dashed, line width=3pt}, mark=none] coordinates {(Sprint 1, 100) (Sprint 2, 100) (Sprint 3, 100) (Sprint 4, 100)};
            \addplot[color=amm, style={dashed, line width=3pt}, mark=none] coordinates {(Sprint 1, 80) (Sprint 2, 80) (Sprint 3, 80) (Sprint 4, 80)};
            \addplot[color=blue, style={line width=1pt}, mark=none] coordinates {(Sprint 1, 0) (Sprint 2, 0) (Sprint 3, 0) (Sprint 4, 0)};
            \legend{Valore ottimo, Valore ammissibile, Valore effettivo}
        \end{axis}
    \end{tikzpicture}
    \caption{Percentuale di branch coverage dei test implementati}
\end{figure*}
%--------- FINE GRAFICO -----------%
%\href{https://7last.github.io/docs/rtb/documentazione-interna/glossario\#requirements-and-technology-baseline}{\textbf{RTB}\textsubscript{G}} \\
% TODO considerazioni finali per \href{https://7last.github.io/docs/rtb/documentazione-interna/glossario\#requirements-and-technology-baseline}{RTB\textsubscript{G}} \\
%\href{https://7last.github.io/docs/rtb/documentazione-interna/glossario\#product-baseline}{\textbf{PB}\textsubscript{G}} \\
% TODO considerazioni finali per \href{https://7last.github.io/docs/rtb/documentazione-interna/glossario\#product-baseline}{PB\textsubscript{G}}

\newpage
\subsubsection{28M-SC - Statement coverage}
%--------- GRAFICO -----------%
\begin{figure*}[!h]
    \centering
    \begin{tikzpicture}
        \begin{axis}[
            width  = 0.85*\textwidth,
            height = 8cm,
            ymajorgrids = true,
            symbolic x coords={Sprint 1, Sprint 2, Sprint 3, Sprint 4},
            xtick = data,
            ytick = {0, 25, 50, 75, 100},
            ymin=0, ymax=100,
            axis lines*=left,
            legend cell align=left,
            legend style={
                at={(0.5,1.15)},
                anchor=south,
                column sep=0.1ex,
                legend columns=3
            },      
            xticklabel style={rotate=45, anchor=north east, yshift=0ex, xshift=0ex}
            ]
            \addplot[color=opt, style={dashed, line width=3pt}, mark=none] coordinates {(Sprint 1, 100) (Sprint 2, 100) (Sprint 3, 100) (Sprint 4, 100)};
            \addplot[color=amm, style={dashed, line width=3pt}, mark=none] coordinates {(Sprint 1, 80) (Sprint 2, 80) (Sprint 3, 80) (Sprint 4, 80)};
            \addplot[color=blue, style={line width=1pt}, mark=none] coordinates {(Sprint 1, 0) (Sprint 2, 0) (Sprint 3, 0)  (Sprint 4, 0)};
            \legend{Valore ottimo, Valore ammissibile, Valore effettivo}
        \end{axis}
    \end{tikzpicture}
    \caption{Percentuale di statement coverage dei test implementati}
\end{figure*}
%--------- FINE GRAFICO -----------%
%\href{https://7last.github.io/docs/rtb/documentazione-interna/glossario\#requirements-and-technology-baseline}{\textbf{RTB}\textsubscript{G}} \\
% TODO considerazioni finali per \href{https://7last.github.io/docs/rtb/documentazione-interna/glossario\#requirements-and-technology-baseline}{RTB\textsubscript{G}} \\
%\href{https://7last.github.io/docs/rtb/documentazione-interna/glossario\#product-baseline}{\textbf{PB}\textsubscript{G}} \\
% TODO considerazioni finali per \href{https://7last.github.io/docs/rtb/documentazione-interna/glossario\#product-baseline}{PB\textsubscript{G}}

\newpage
\subsubsection{29M-FD - Failure density}
%--------- GRAFICO -----------%
\begin{figure*}[!h]
    \centering
    \begin{tikzpicture}
        \begin{axis}[
            width  = 0.85*\textwidth,
            height = 8cm,
            ymajorgrids = true,
            symbolic x coords={Sprint 1, Sprint 2, Sprint 3, Sprint 4},
            xtick = data,
            ytick = {0, 25, 50, 75, 100},
            ymin=0, ymax=100,
            axis lines*=left,
            legend cell align=left,
            legend style={
                at={(0.5,1.15)},
                anchor=south,
                column sep=0.1ex,
                legend columns=3
            },      
            xticklabel style={rotate=45, anchor=north east, yshift=0ex, xshift=0ex}
            ]
            \addplot[color=opt, style={dashed, line width=3pt}, mark=none] coordinates {(Sprint 1, 0) (Sprint 2, 0) (Sprint 3, 0) (Sprint 4, 0)};
            \addplot[color=amm, style={dashed, line width=3pt}, mark=none] coordinates {(Sprint 1, 15) (Sprint 2, 15) (Sprint 3, 15) (Sprint 4, 15)};
            \addplot[color=blue, style={line width=1pt}, mark=none] coordinates {(Sprint 1, 0) (Sprint 2, 0) (Sprint 3, 0) (Sprint 4, 0)};
            \legend{Valore ottimo, Valore ammissibile, Valore effettivo}
        \end{axis}
    \end{tikzpicture}
    \caption{Percentuale di failure density}
\end{figure*}
%--------- FINE GRAFICO -----------%
%\href{https://7last.github.io/docs/rtb/documentazione-interna/glossario\#requirements-and-technology-baseline}{\textbf{RTB}\textsubscript{G}} \\
% TODO considerazioni finali per \href{https://7last.github.io/docs/rtb/documentazione-interna/glossario\#requirements-and-technology-baseline}{RTB\textsubscript{G}} \\
%\href{https://7last.github.io/docs/rtb/documentazione-interna/glossario\#product-baseline}{\textbf{PB}\textsubscript{G}} \\
% TODO considerazioni finali per \href{https://7last.github.io/docs/rtb/documentazione-interna/glossario\#product-baseline}{PB\textsubscript{G}}

\newpage
\subsubsection{30M-PTCP - Passed Test Case Percentage}
%--------- GRAFICO -----------%
\begin{figure*}[!h]
    \centering
    \begin{tikzpicture}
        \begin{axis}[
            width  = 0.85*\textwidth,
            height = 8cm,
            ymajorgrids = true,
            symbolic x coords={Sprint 1, Sprint 2, Sprint 3, Sprint 4},
            xtick = data,
            ytick = {0, 25, 50, 75, 100},
            ymin=0, ymax=100,
            axis lines*=left,
            legend cell align=left,
            legend style={
                at={(0.5,1.15)},
                anchor=south,
                column sep=0.1ex,
                legend columns=3
            },      
            xticklabel style={rotate=45, anchor=north east, yshift=0ex, xshift=0ex}
            ]
            \addplot[color=opt, style={dashed, line width=3pt}, mark=none] coordinates {(Sprint 1, 100) (Sprint 2, 100) (Sprint 3, 100) (Sprint 4, 100)};
            \addplot[color=amm, style={dashed, line width=3pt}, mark=none] coordinates {(Sprint 1, 90) (Sprint 2, 90) (Sprint 3, 90) (Sprint 4, 90)};
            \addplot[color=blue, style={line width=1pt}, mark=none] coordinates {(Sprint 1, 0) (Sprint 2, 0) (Sprint 3, 0) (Sprint 4, 0)};
            \legend{Valore ottimo, Valore ammissibile, Valore effettivo}
        \end{axis}
    \end{tikzpicture}
    \caption{Percentuale di casi di test superati}
\end{figure*}
%--------- FINE GRAFICO -----------%
%\href{https://7last.github.io/docs/rtb/documentazione-interna/glossario\#requirements-and-technology-baseline}{\textbf{RTB}\textsubscript{G}} \\
% TODO considerazioni finali per \href{https://7last.github.io/docs/rtb/documentazione-interna/glossario\#requirements-and-technology-baseline}{RTB\textsubscript{G}} \\
%\href{https://7last.github.io/docs/rtb/documentazione-interna/glossario\#product-baseline}{\textbf{PB}\textsubscript{G}} \\
% TODO considerazioni finali per \href{https://7last.github.io/docs/rtb/documentazione-interna/glossario\#product-baseline}{PB\textsubscript{G}}

\newpage
\subsection{Qualità del processo di Gestione dei rischi}
\subsubsection{32M-NCR - Rischi Non Calcolati}
%--------- GRAFICO -----------%
\begin{figure*}[!h]
    \centering
    \begin{tikzpicture}
        \begin{axis}[
            width  = 0.85*\textwidth,
            height = 8cm,
            ymajorgrids = true,
            symbolic x coords={Sprint 1, Sprint 2, Sprint 3, Sprint 4},
            xtick = data,
            ymin=0, ymax=5,
            axis lines*=left,
            legend cell align=left,
            legend style={
                at={(0.5,1.15)},
                anchor=south,
                column sep=0.1ex,
                legend columns=3
            },      
            xticklabel style={rotate=45, anchor=north east, yshift=0ex, xshift=0ex}
            ]
            \addplot[color=opt, style={dashed, line width=3pt}, mark=none] coordinates {(Sprint 1, 0) (Sprint 2, 0) (Sprint 3, 0) (Sprint 4, 0)};
            \addplot[color=amm, style={dashed, line width=3pt}, mark=none] coordinates {(Sprint 1, 3) (Sprint 2, 3) (Sprint 3, 3) (Sprint 4, 3)};
            \addplot[color=blue, style={line width=1pt}, mark=none] coordinates {(Sprint 1, 1) (Sprint 2, 1) (Sprint 3, 0) (Sprint 4, 0)};
            \legend{Valore ottimo, Valore ammissibile, Valore effettivo}
        \end{axis}
    \end{tikzpicture}
    \caption{Rischi non calcolati occorsi durante il progetto}
\end{figure*}
%--------- FINE GRAFICO -----------%
%\href{https://7last.github.io/docs/rtb/documentazione-interna/glossario\#requirements-and-technology-baseline}{\textbf{RTB}\textsubscript{G}} \\
% TODO considerazioni finali per \href{https://7last.github.io/docs/rtb/documentazione-interna/glossario\#requirements-and-technology-baseline}{RTB\textsubscript{G}} \\
%\href{https://7last.github.io/docs/rtb/documentazione-interna/glossario\#product-baseline}{\textbf{PB}\textsubscript{G}} \\
% TODO considerazioni finali per \href{https://7last.github.io/docs/rtb/documentazione-interna/glossario\#product-baseline}{PB\textsubscript{G}}


\newpage
\subsection{Qualità del processo di Pianificazione}
\subsubsection{33M-RSI - Requirements Stability Index}
%--------- GRAFICO -----------%
\begin{figure*}[!h]
    \centering
    \begin{tikzpicture}
        \begin{axis}[
            width  = 0.85*\textwidth,
            height = 8cm,
            ymajorgrids = true,
            symbolic x coords={Sprint 1, Sprint 2, Sprint 3, Sprint 4},
            xtick = data,
            ytick = {0, 25, 50, 75, 100},
            ymin=0, ymax=100,
            axis lines*=left,
            legend cell align=left,
            legend style={
                at={(0.5,1.15)},
                anchor=south,
                column sep=0.1ex,
                legend columns=3
            },      
            xticklabel style={rotate=45, anchor=north east, yshift=0ex, xshift=0ex}
            ]
            \addplot[color=opt, style={dashed, line width=3pt}, mark=none] coordinates {(Sprint 1, 100) (Sprint 2, 100) (Sprint 3, 100) (Sprint 4, 100)};
            \addplot[color=amm, style={dashed, line width=3pt}, mark=none] coordinates {(Sprint 1, 75) (Sprint 2, 75) (Sprint 3, 75) (Sprint 4, 75)};
            \addplot[color=blue, style={line width=1pt}, mark=none] coordinates {(Sprint 1, 0) (Sprint 2, 0) (Sprint 3, 60) (Sprint 4, 85)};
            \legend{Valore ottimo, Valore ammissibile, Valore effettivo}
        \end{axis}
    \end{tikzpicture}
    \caption{Percentuale di stabilità dei requisiti}
\end{figure*}
%--------- FINE GRAFICO -----------%
%\href{https://7last.github.io/docs/rtb/documentazione-interna/glossario\#requirements-and-technology-baseline}{\textbf{RTB}\textsubscript{G}} \\
% TODO considerazioni finali per \href{https://7last.github.io/docs/rtb/documentazione-interna/glossario\#requirements-and-technology-baseline}{RTB\textsubscript{G}} \\
%\href{https://7last.github.io/docs/rtb/documentazione-interna/glossario\#product-baseline}{\textbf{PB}\textsubscript{G}} \\
% TODO considerazioni finali per \href{https://7last.github.io/docs/rtb/documentazione-interna/glossario\#product-baseline}{PB\textsubscript{G}}


\section{Iniziative di automiglioramento per la qualità}
\subsection{Introduzione}
In questa sezione verranno riportate le iniziative di automiglioramentoche il nostro gruppo ha deciso di adottare per migliorare la qualità del prodotto e dei processi. Queste iniziative sono state individuate grazie all'esperienza acquisita durante lo svolgimento del progetto e grazie alle valutazioni effettuate sulle attività svolte. \\
Trattandosi per tutti noi della prima volta in cui ci siamo trovati a dover affrontare un progetto di questa portata, abbiamo dovuto fare molte prove ed errori per capire come organizzarci al meglio e come svolgere al meglio le attività. Questo ci ha permesso di capire quali sono stati i punti di forza e i punti deboli del nostro lavoro e di individuare le aree in cui è possibile migliorare. \\
Per ciascuna delle difficoltà riscontrate verranno indicate:
\begin{itemize}
    \item fase del progetto in cui si è verificato il problema;
    \item descrizione del problema;
    \item contromisura adottata per risolvere il problema evidenziato.
\end{itemize}

\subsection{Problemi rilevati ed iniziative adottate}
\begin{itemize}
    \item \textbf{Organizzazione delle riunioni}
    \begin{itemize}
        \item \textbf{Fase del progetto}: iniziale;
        \item \textbf{Descrizione}: nelle prime settimane di lavoro, a partire dalla formazione dei gruppi sino ai primi Diari di bordo, si è riscontrata una certa difficoltà nell'organizzazione delle riunioni a causa dei vari impegni di ciascun membro (lezioni diverse in orari diversi, lavoro per alcuni, impegni personali) e soprattutto a causa delle diverse riunioni che si andavano a sommare (SAL con l'azienda prima e Diari di bordo poi) e questo ha portato a una certa confusione e a un rallentamento delle attività;
        \item \textbf{Contromisura}: abbiamo deciso di effettuare le riunioni a distanza tramite la piattaforma \textit{Discord} e di fissare un giorno e un orario durante la settimana per ciascun tipo di riunione in cui tutti i membri del gruppo sono solitamente disponibili, così da poterci organizzare al meglio e da poter svolgere le attività in modo più efficiente; qualora qualcuno, per impegni di natura eccezionale, non abbia modo di essere presente potrà successivamente informarsi sui contenuti di tale riunione attraverso i verbali che verranno redatti e messi a disposizione di tutti.
    \end{itemize}
    \item \textbf{Suddivisione compiti}
    \begin{itemize}
        \item \textbf{Fase del progetto}: iniziale;
        \item \textbf{Descrizione}: all'inizio del progetto si è riscontrata una certa difficoltà nella suddivisione dei compiti a causa della mancanza di esperienza e della poca conoscenza delle competenze di ciascun membro del gruppo, è quindi capitato più volte che alcuni membri completassero lo svolgimento dei compiti loro assegnati in anticipo rispetto agli altri membri e che quindi si ritrovassero senza nulla da fare, mentre altri membri si ritrovavano con troppo lavoro da svolgere e non riuscivano a completare i compiti assegnati entro i tempi prestabiliti;
        \item \textbf{Contromisura}: abbiamo quindi deciso, come suggerito anche dal professor Vardanega al primo Diario di bordo, di non assegnare preventivamente tutti i compiti da svolgere a ciascun membro, ma piuttosto di metterli in un contenitore condiviso (abbiamo deciso di usare le annotazioni di \textit{ClickUp}) e di permettere a ciascun membro di prendere in autonomia i compiti da svolgere, così che chiunque finisca in anticipo possa prenderne altri; in questo modo siamo riusciti a svolgere le attività in modo più equo e a completare i compiti entro i tempi prestabiliti.
    \end{itemize}
\end{itemize}

\subsection{Considerazioni finali}
Fin da subito il nostro gruppo si è posto come obiettivo principale quello di dotarsi di un \textit{Way of Working} preciso e ben definito, di pianificare ogni singola attività e di prevedere tutte le possibili difficoltà che avremmo potuto incontrare durante lo svolgimento del progetto. Questo per evitare di dover affrontare eventuali problemi solamente dopo che si fossero presentati, ma piuttosto per cercare di prevenirli e di risolverli prima che potessero avvenire o di fornire delle contromisure per affrontarli nel caso in cui si presentassero. \\
Questo inizialmente ci ha messo in difficoltà, poichè nessuno di noi aveva mai affrontato un progetto con queste modalità e quindi non sapevamo bene come organizzarci e come pianificare le varie attività. Tuttavia, grazie all'esperienza acquisita durante lo svolgimento del progetto e grazie ai consigli e ai suggerimenti che ci sono stati forniti dai professori e dall'azienda proponente, siamo riusciti a individuare i problemi e a mettere in atto delle contromisure per risolverli. \\
Questo ci ha permesso di migliorare notevolmente la qualità del nostro lavoro e di svolgere le attività in modo più efficiente e più equo. Nonostante ciò siamo anche consapevoli che ci sono ancora molti aspetti su cui possiamo migliorare e che ci sono ancora molte iniziative di automiglioramento che possiamo adottare. Tuttavia, siamo convinti che, se continueremo a lavorare con lo stesso impegno e la stessa determinazione che abbiamo dimostrato finora, saremo in grado di ottenere ottimi risultati e di migliorare sempre di più la qualità del nostro lavoro.

\end{document}
