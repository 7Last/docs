\documentclass[italian,12pt]{article} %tipo di documento

%--------------variabili------------------%
\def\Title{Norme di Progetto}
\def\Author{7Last}
\def\Version{v0.2}
%-----------------------------------------%


\usepackage[left=2cm, right=2cm, bottom=3cm, top=3cm]{geometry}
\usepackage{fancyhdr}
\usepackage{graphicx}
\graphicspath{ {../../logo/} }
\usepackage{href-ul}
\usepackage{tikz}
\usepackage{tgadventor}
\usepackage[useregional=numeric,showseconds=true,showzone=false]{datetime2}
\usepackage{caption}
\usepackage{longtable}
\usepackage{xcolor}



\input{subX3_section.tex}

\linespread{1.2}

\captionsetup[table]{name=Tabella}
\geometry{headsep=1.5cm}

\renewcommand{\contentsname}{Indice}%imposto il nome dell'indice
\renewcommand\familydefault{\sfdefault}

\renewcommand{\listtablename}{Indice delle tabelle}%imposto il nome della lista tabelle
\renewcommand\familydefault{\sfdefault}

\renewcommand{\listfigurename}{Indice delle immagini}%imposto il nome della lista immagini
\renewcommand\familydefault{\sfdefault}

%-------------------INIZIO DOCUMENTO--------------
\begin{document}

\newgeometry{left=2cm,right=2cm,bottom=2.1cm,top=2.1cm}
\begin{titlepage}
	\vspace*{.5cm}

	\vspace{2cm}
	{
		\centering
		{\bfseries\huge \Title\par}
		\bigbreak
		{\bfseries\Large \Subtitle\par}
		\bigbreak
		{\bfseries\large \Author\par}
		\bigbreak
		{\Date\;-\;\Version\par}
		\vfill

		\begin{center}
			\begin{tikzpicture}
				\clip (0,0) circle (2cm) node {\includegraphics[width=4cm]{logo.jpg}};
			\end{tikzpicture}
		\end{center}
	}

	\vfill

\end{titlepage}

\restoregeometry






















\newpage

\pagestyle{fancy}
\fancyhead{}
\lhead{
	\begin{tikzpicture}
		\clip (0,0) circle (0.5cm);
		\node at (0,0) {\includegraphics[width=1cm]{./../logo/logo.png}};
	\end{tikzpicture}%
}
\chead{\vspace{\fill}\Title\vspace{\fill}}
\rhead{\vspace{\fill}\Version\vspace{\fill}}


%-----------tabella revisioni-----------%
\captionsetup[table]{list=no}

\begin{table}[!h]
	\begin{center}
		Versioni\\
		\vspace{0.5cm}
		\begin{tabular}{ l l l l l }
			\hline                                                                          		\\[-2ex]
			Ver. 	& Data			& Autore			 & Verificatore		& Descrizione\\
			\\[-2ex] \hline \\[-1.5ex]

			0.3 	 & 07/04/2024 	& Valerio Occhinegro & Matteo Tiozzo	& Stesura documento	\\
			0.2	     & 29/03/2024 	& Matteo Tiozzo 	 & 					& Modificato tabella versioni	\\
			0.1 	 & 28/03/2024 	& Valerio Occhinegro & Matteo Tiozzo 	& Prima redazione				\\
			\\[-1.5ex] \hline
		\end{tabular}
	\end{center}
\end{table}
\captionsetup[table]{list=yes}
%---------------------------------------%

\newpage

\tableofcontents

\newpage

\listoftables

\listoffigures

\newpage

\section{Introduzione}

\subsection{Obiettivo del documento}
Il presente documento ha lo scopo di definire le strategie di verifica e validazione utilizzate per assicurare il corretto funzionamento dello strumento sviluppato e delle
attività che lo accompagnano.  Sarà sottoposto a revisioni continue, così da prevedere situazioni precedentemente non occorse e da seguire l'evoluzione del progetto.

\subsection{Glossario}
Il glossario è uno strumento utilizzato per risolvere eventuali dubbi riguardanti 
alcuni termini specifici utilizzati nella redazione del documento.
Esso conterrà la definizione dei termini evidenziati e sarà consultabile al seguente \href{https://7last.github.io/docs/rtb/documentazione-interna/glossario}{link}. I termini presenti in tale documento saranno evidenziati da una 'G' a pedice.

\subsection{Riferimenti}

\subsubsection{Riferimenti normativi}
\begin{itemize}
    \item Norme di progetto (aggiungere versione e/o link al documento);
    \item Capitolato d'appalto C6: SyncCity – A smart city monitoring platform\\
		  \url{https://www.math.unipd.it/~tullio/IS-1/2023/Progetto/C6.pdf};
    \item Regolamento del progetto:\\
		  \url{https://www.math.unipd.it/~tullio/IS-1/2023/Dispense/PD2.pdf}.
\end{itemize}

\subsubsection{Riferimenti informativi}
\begin{itemize}
    \item Alcuni mettono le dispense del prof altri la documentazione iso del 1995 
			sempre fornita dal prof;
    \item Glossario.
\end{itemize}

\section{Fini metrici di qualità}
Al fine di valutare nel miglior modo possibile la qualità del prodotto e l'efficacia dei processi, sono state definite delle metriche, meglio specificate nel documento Norme di Progetto v1.0. METTERE LINK NORME DI PROGETTO. Il contenuto di questa sezione è necessario per identificare i parametri che le metriche devono rispettare per essere considerate accettabili o ottime. 

\subsection{Qualità di processo}
La qualità di processo è un criterio fondamentale ed è alla base di ogni prodotto
che rispecchi lo stato dell'arte. Per raggiungere tale obiettivo è necessario 
sfruttare delle pratiche rigorose che consentano lo svolgimento di ogni attività
in maniera ottimale.

\subsubsection{Processi primari}

\subsubsubsection{Fornitura}

	
\begin{table}[h!]
    \centering
    \begin{tabularx}{\textwidth}{|X|X|X|X|X|} 
		\hline
		\textbf{Metrica} 	& \textbf{Nome} & \textbf{Valore ammissibile} & \textbf{Valore ottimo}& \textbf{Descrizione}\\  	 
		\hline
		\textbf{MPC01} 	& Earned Value (EV) 			& $\geq 0$ 						& $\leq$ EAC &  \\
		\hline
		\textbf{MPC02} 	& Planned Value (PV) 			& $\geq 0$ 						& $\leq$ Budget At Completion (BAC) &  \\ 
		\hline
		\textbf{MPC03} 	& Actual Cost (AC) 				& $\geq 0$ 						& $\leq$ EAC &	Misura i costi effettivamente sostenuti dall'inizio del progetto fino al presente momento.\\ 
		\hline
		\textbf{MPC04} 	& Cost Variance (CV) 			& $\geq -7.5\%$ 				& $\geq 0\%$ & \\ 
		\hline
		\textbf{MPC05} 	& Schedule Variance (SV) 		& $\geq -7.5\%$ 				& $\geq 0\%$ &Indica in percentuale quanto si è in anticipo o in ritardo con le attività pianificate.\\ 
		\hline
		\textbf{MPC06} 	& Estimated at Completion (EAC) & Errore del $\pm 3\%$ rispetto al BAC  & Equivalente al BAC & Misura il costo realizzativo stimato per terminare il progetto.\\ 
		\hline
		\textbf{MPC07} 	& Estimate to Complete (ETC)  	& $\geq 0$						 & $\leq$ EAC & Stima dei costi realizzativi fino alla fine del progetto.\\ 
		\hline
    \end{tabularx}
    \caption{ Valori delle metriche inerenti al processo di Fornitura}
    \label{table:1}
\end{table}

\newpage
\subsubsubsection{Sviluppo}

% \rowcolors{1}{white}{lightgray}
\begin{table}[h!]
	\centering
	\begin{tabularx}{\textwidth}{|X|X|X|X|X|} 	 
		\hline
		\textbf{Metrica} 	& \textbf{Nome} & \textbf{Valore ammissibile} & \textbf{Valore ottimo} & \textbf{Descrizione}\\  	 
		\hline
		\textbf{MPC08} 		& Requirements Stability Index (RSI) 	& $\geq 75\% $ 			& 100\% & \\
		\hline
		\textbf{MPC09} 		& Structural Fan-In (SFIN) 				&  						& Da massimizzare & \\ 
		\hline
		\textbf{MPC10} 		& Structural Fan-Out (SFOUT) 			&  						& Da minimizzare & \\ 
		\hline
	\end{tabularx}
	\caption{ Valori delle metriche inerenti al processo di Sviluppo}
	\label{table:2}
\end{table}

\subsubsection{Processi di supporto}

\subsubsubsection{Documentazione}

% \rowcolors{1}{white}{lightgray}
\begin{table}[h!]
	\centering
	\begin{tabularx}{\textwidth}{|X|X|X|X|X|} 	 
		\hline
		\textbf{Metrica} 	& \textbf{Nome} & \textbf{Valore ammissibile} & \textbf{Valore ottimo}& \textbf{Descrizione}\\  	 
		\hline
		\textbf{MPC11} & Indice Gulpease & $\geq 60\% $ & 80\% & Misura la leggibilità di un testo in base alla lunghezza delle parole e delle frasi.\\
		\hline
		\textbf{MPC12} & Correttezza Ortografica & 0 errori & 0 errori &Misura la presenza di errori ortografici nei documenti.\\ 
		\hline
	\end{tabularx}
	\caption{ Valori delle metriche inerenti al processo di Documentazione}
	\label{table:3}
\end{table}

\subsubsubsection{Verifica}
% \rowcolors{1}{white}{lightgray}
\begin{table}[h!]
	\centering
	\begin{tabularx}{\textwidth}{|X|X|X|X|X|} 	 
		\hline
		\textbf{Metrica} 	& \textbf{Nome} & \textbf{Valore ammissibile} & \textbf{Valore ottimo}& \textbf{Descrizione}\\  	 
		\hline
		\textbf{MPC13} & Code Coverage & $\geq 90\% $ & 100\% &\\
		\hline
		\textbf{MPC14} & Passed Test Cases Percentage & 100\% & 100\% &\\ 
		\hline
	\end{tabularx}
	\caption{ Valori delle metriche inerenti al processo di Verifica}
	\label{table:4}
\end{table}

\subsubsubsection{Gestione della qualità}

% \rowcolors{1}{white}{lightgray}
\begin{table}[h!]
	\centering
	\begin{tabularx}{\textwidth}{|X|X|X|X|X|} 	 
		\hline
		\textbf{Metrica} 	& \textbf{Nome} & \textbf{Valore ammissibile} & \textbf{Valore ottimo} & \textbf{Descrizione}\\  	 
		\hline
		\textbf{MPC15} & Quality Metrics Satisfied & $\geq 85\% $ & 100\% &\\
		\hline
	\end{tabularx}
	\caption{ Valori delle metriche inerenti al processo di Verifica}
	\label{table:5}
\end{table}

\subsubsection{Processi organizzativi}

\subsubsubsection{Gestione dei processi}

% \rowcolors{1}{white}{lightgray}
\begin{table}[h!]
	\centering
	\begin{tabularx}{\textwidth}{|X|X|X|X|X|} 	 
		\hline
		\textbf{Metrica} 	& \textbf{Nome} & \textbf{Valore ammissibile} & \textbf{Valore ottimo} & \textbf{Descrizione}\\  	 
		\hline
		\textbf{MPC16} & Non Calculated Risk & $\leq 3 $ & 0 &\\
		\hline
		\textbf{MPC17} & Time Efficiency & $\leq 3 $ & $\leq 1 $ &\\
		\hline
	\end{tabularx}
	\caption{ Valori delle metriche inerenti al processo di Gestione dei processi}
	\label{table:6}
\end{table}


\subsection{Qualità di prodotto}

Per qualità di prodotto si intende la capacità del software di rispettare 
le caratteristiche richieste dal cliente e quelle dettate dallo standard.
Più il risultato si avvicina a quello atteso, più al qualità del prodotto
sarà elevata. 

\subsubsection{Funzionalità}

% \rowcolors{1}{white}{lightgray}
\begin{table}[h!]
	\centering
	\begin{tabularx}{\textwidth}{|X|X|X|X|X|} 	 
		\hline
		\textbf{Metrica} 	& \textbf{Nome} & \textbf{Valore ammissibile} & \textbf{Valore ottimo} & \textbf{Descrizione}\\  	 
		\hline
		\textbf{MPD01} & Copertura dei requisiti obbligatori & 100\%  & 100\% &\\
		\hline
		\textbf{MPD02} & Copertura dei requisiti desiderabili & $\geq 50\% $  & 100\% &\\ 
		\hline
		\textbf{MPD03} & Copertura dei requisiti opzionali & $\geq 0\% $ & $\geq 50\% $ &\\ 
		\hline
	\end{tabularx}
	\caption{ Valori delle metriche inerenti alla Funzionalità del prodotto}
	\label{table:7}
\end{table}

\subsubsection{Affidabilità}

% \rowcolors{1}{white}{lightgray}
\begin{table}[h!]
	\centering
	\begin{tabularx}{\textwidth}{|X|X|X|X|X|} 	 
		\hline
		\textbf{Metrica} 	& \textbf{Nome} & \textbf{Valore ammissibile} & \textbf{Valore ottimo} & \textbf{Descrizione}\\  	 
		\hline
		\textbf{MPD04} & Code Coverage & $\geq 80\% $  & 100\% &\\
		\hline
		\textbf{MPD05} & Branch Coverage & $\geq 50\% $  & $\geq 80\% $ &\\ 
		\hline
		\textbf{MPD06} & Statement Coverage & $\geq 60\% $ & $\geq 80\% $ &\\ 
		\hline
		\textbf{MPD07} & Failure Density & 100\%  & 100\%  &\\ 
		\hline
	\end{tabularx}
	\caption{ Valori delle metriche inerenti all'Affidabilità del prodotto}
	\label{table:8}
\end{table}


\subsubsection{Usabilità}

% \rowcolors{1}{white}{lightgray}
\begin{table}[h!]
	\centering
	\begin{tabularx}{\textwidth}{|X|X|X|X|X|} 	 
		\hline
		\textbf{Metrica} 	& \textbf{Nome} & \textbf{Valore ammissibile} & \textbf{Valore ottimo} & \textbf{Descrizione}\\  	 
		\hline
		\textbf{MPD08} & Facilità di utilizzo & $\leq 3 $ errori di utilizzo & 0 errori di utilizzo &\\
		\hline
		\textbf{MPD09} & Tempo di apprendimento & $\leq 15 $ minuti  & $\leq 5 $ minuti & \\ 
		\hline
	\end{tabularx}
	\caption{ Valori delle metriche inerenti all'Usabilità del prodotto}
	\label{table:9}
\end{table}

\subsubsection{Efficienza}

% \rowcolors{1}{white}{lightgray}
\begin{table}[h!]
	\centering
	\begin{tabularx}{\textwidth}{|X|X|X|X|X|} 	 
		\hline
		\textbf{Metrica} 	& \textbf{Nome} & \textbf{Valore ammissibile} & \textbf{Valore ottimo} & \textbf{Descrizione}\\  	 
		\hline
		\textbf{MPD10} & Utilizzo risorse & $\geq 75\% $  & 100\% & \\
		\hline
	\end{tabularx}
	\caption{ Valori delle metriche inerenti all'Efficienza del prodotto}
	\label{table:10}
	\end{table}

\subsubsection{Manutenibilità}

% \rowcolors{1}{white}{lightgray}
\begin{table}[h!]
	\centering
	\begin{tabularx}{\textwidth}{|X|X|X|X|X|} 	 
		\hline
		\textbf{Metrica} 	& \textbf{Nome} & \textbf{Valore ammissibile} & \textbf{Valore ottimo} & \textbf{Descrizione}\\  	 
		\hline
		\textbf{MPD11} & Complessità ciclomatica & 1-10 & 11-20 & \\
		\hline
		\textbf{MPD12} & Code Smell & 0 & 0 & \\ 
		\hline
		\textbf{MPD13} & Coefficient of Coupling (COC) & $\leq 30\% $ & $\leq 10\% $ & \\ 
		\hline
	\end{tabularx}
	\caption{ Valori delle metriche inerenti alla Manutenibilità del prodotto}
	\label{table:11}
\end{table}

\section{Metodologie di testing}

\subsection{Codice dei test}

\subsection{Test di unità}

\subsection{Test di integrazione}

\subsection{Test di sistema}

\newpage

\section{Cruscotto di valutazione della qualità}

\subsection{MPC06 - Estimated at Completion(EAC)}

\subsection{MPC01 - Earned Value (EV) e MPC02 - Planned Value (PV)}

\subsection{MPC03 - Actual Cost (AC) e MPC07 - Estimate to Complete (ETC)}

\subsection{MPC04 - Cost Variance (CV) e MPC05 - Schedule Variance (SV)}

\subsection{MPC08 - Requirements stability index (RSI)}

\subsection{MPC11 - Indice Gulpease}

\subsection{MPC12 - Correttezza Ortografica}

\subsection{MPC15 - Quality Metrics Satisfied}

\subsection{MPC16 - Non-Calculated Risk}

\subsection{MPC17 - Efficienza Temporale}

\section{Iniziative di automiglioramento per la qualità}

\subsection{Introduzione}

\subsection{Problemi leagati all’organizzazione generale}

\subsection{Valutazione sui ruoli}

\subsection*{Valutazione sugli strumenti}

\subsection{Considerazioni finali sul miglioramento}

\subsubsection{Analisi della pratiche seguite}

\subsubsection{Valutazioni generali sui miglioramenti conseguiti}

\subsubsection{Valutazioni specifiche sui miglioramenti nei processi}

\subsubsubsection{Gestione delle comunicazioni e degli incontri}

\subsubsubsection{Pianificazione}

\end{document}
