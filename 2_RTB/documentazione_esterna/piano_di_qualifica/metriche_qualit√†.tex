\section{Fini metrici di qualità}
Al fine di valutare nel miglior modo possibile la qualità del prodotto e l'efficacia dei processi, sono state definite delle metriche, meglio specificate nel documento Norme di Progetto v1.0. METTERE LINK NORME DI PROGETTO. Il contenuto di questa sezione è necessario per identificare i parametri che le metriche devono rispettare per essere considerate accettabili o ottime. 
\subsection{Qualità di processo}
La qualità di processo è un criterio fondamentale ed è alla base di ogni prodotto
che rispecchi lo stato dell'arte. Per raggiungere tale obiettivo è necessario 
sfruttare delle pratiche rigorose che consentano lo svolgimento di ogni attività
in maniera ottimale.
\subsubsection{Processi primari}
\subsubsubsection{Fornitura}

\begin{longtable}{|>{\centering\arraybackslash}p{0.18\textwidth}|>{\centering\arraybackslash}p{0.18\textwidth}|>{\centering\arraybackslash}p{0.18\textwidth}|>{\centering\arraybackslash}p{0.18\textwidth}|>{\centering\arraybackslash}p{0.18\textwidth}|}
    \hline
    \textbf{Metrica} & \textbf{Nome} & \textbf{Valore ammissibile} & \textbf{Valore ottimo}& \textbf{Descrizione}\\
	\hline
    \endfirsthead
    \hline
    \textbf{Metrica} & \textbf{Nome} & \textbf{Valore ammissibile} & \textbf{Valore ottimo}& \textbf{Descrizione}\\
    \endhead
    \textbf{1M-EV} & Earned Value (EV) & $\geq 0$ & $\leq$ EAC & Valore del lavoro effettivamente svolto fino al determinato periodo\\
    \hline
	\textbf{2M-PV} 	& Planned Value (PV) 			& $\geq 0$ 						& $\leq$ Budget At Completion (BAC) &  Stima la somma dei costi realizzativi delle attività imminenti periodo per periodo\\ 
	\hline
	\textbf{3M-AC} 	& Actual Cost (AC) 				& $\geq 0$ 						& $\leq$ EAC &	Misura i costi effettivamente sostenuti dall'inizio del progetto fino al presente momento.\\ 
	\hline
	\textbf{4M-CV} 	& Cost Variance (CV) 			& $\geq -7.5\%$ 				& $\geq 0\%$ & Misura la differenza percentuale di budget tra quanto previsto nella pianificazione di un periodo e l'effettiva realizzazione. \textbf{CONTROLLARE}\\ 
	\hline
	\textbf{5M-SV} 	& Schedule Variance (SV) 		& $\geq -7.5\%$ 				& $\geq 0\%$ &Indica in percentuale quanto si è in anticipo o in ritardo con le attività pianificate.\\ 
	\hline
	\textbf{6M-EAC} 	& Estimated at Completion (EAC) & Errore del $\pm 3\%$ rispetto al BAC  & Equivalente al BAC & Misura il costo realizzativo stimato per terminare il progetto.\\ 
	\hline
	\textbf{7M-ETC} 	& Estimate to Complete (ETC)  	& $\geq 0$						 & $\leq$ EAC & Stima dei costi realizzativi fino alla fine del progetto.\\ 
	\hline
    \caption{Valori delle metriche inerenti al processo di Fornitura}
    \label{table:1}
\end{longtable}

\subsubsubsection{Sviluppo}
\begin{longtable}{|>{\centering\arraybackslash}p{0.18\textwidth}|>{\centering\arraybackslash}p{0.18\textwidth}|>{\centering\arraybackslash}p{0.18\textwidth}|>{\centering\arraybackslash}p{0.18\textwidth}|>{\centering\arraybackslash}p{0.18\textwidth}|}
    \hline
    \textbf{Metrica} & \textbf{Nome} & \textbf{Valore ammissibile} & \textbf{Valore ottimo}& \textbf{Descrizione}\\
	\hline
    \endfirsthead
    \hline
    \textbf{Metrica} & \textbf{Nome} & \textbf{Valore ammissibile} & \textbf{Valore ottimo}& \textbf{Descrizione}\\
    \endhead
	\textbf{8M-RSI} 		& Requirements Stability Index (RSI) 	& $\geq 75\% $ 			& 100\% & \\
	\hline
	\textbf{9M-SFIN} 		& Structural Fan-In (SFIN) 				&  						& Da massimizzare & \\ 
	\hline
	\textbf{10M-SFOUT} 		& Structural Fan-Out (SFOUT) 			&  						& Da minimizzare & \\ 
	\hline
	\caption{ Valori delle metriche inerenti al processo di Sviluppo}
	\label{table:2}
\end{longtable}
\subsubsection{Processi di supporto}
\subsubsubsection{Documentazione}
\begin{longtable}{|>{\centering\arraybackslash}p{0.18\textwidth}|>{\centering\arraybackslash}p{0.18\textwidth}|>{\centering\arraybackslash}p{0.18\textwidth}|>{\centering\arraybackslash}p{0.18\textwidth}|>{\centering\arraybackslash}p{0.18\textwidth}|}
    \hline
    \textbf{Metrica} & \textbf{Nome} & \textbf{Valore ammissibile} & \textbf{Valore ottimo}& \textbf{Descrizione}\\
	\hline
    \endfirsthead
    \hline
    \textbf{Metrica} & \textbf{Nome} & \textbf{Valore ammissibile} & \textbf{Valore ottimo}& \textbf{Descrizione}\\
    \endhead
	\textbf{11M-IG} & Indice Gulpease & $\geq 60\% $ & 80\% & Misura la leggibilità di un testo in base alla lunghezza delle parole e delle frasi.\\
	\hline
	\textbf{12M-CO} & Correttezza Ortografica & 0 errori & 0 errori &Misura la presenza di errori ortografici nei documenti.\\ 
	\hline
	\caption{ Valori delle metriche inerenti al processo di Documentazione}
	\label{table:3}
\end{longtable}
\subsubsubsection{Verifica}
\begin{longtable}{|>{\centering\arraybackslash}p{0.18\textwidth}|>{\centering\arraybackslash}p{0.18\textwidth}|>{\centering\arraybackslash}p{0.18\textwidth}|>{\centering\arraybackslash}p{0.18\textwidth}|>{\centering\arraybackslash}p{0.18\textwidth}|}
    \hline
    \textbf{Metrica} & \textbf{Nome} & \textbf{Valore ammissibile} & \textbf{Valore ottimo}& \textbf{Descrizione}\\
	\hline
    \endfirsthead
    \hline
    \textbf{Metrica} & \textbf{Nome} & \textbf{Valore ammissibile} & \textbf{Valore ottimo}& \textbf{Descrizione}\\
    \endhead
	\textbf{13M-CC} & Code Coverage & $\geq 90\% $ & 100\% &\\
	\hline
	\textbf{14M-PTCP} & Passed Test Cases Percentage & 100\% & 100\% & Percentuale di casi di test superati.\\ 
	\hline
	\caption{ Valori delle metriche inerenti al processo di Verifica}
	\label{table:4}
\end{longtable}
\subsubsubsection{Gestione della qualità}
\begin{longtable}{|>{\centering\arraybackslash}p{0.18\textwidth}|>{\centering\arraybackslash}p{0.18\textwidth}|>{\centering\arraybackslash}p{0.18\textwidth}|>{\centering\arraybackslash}p{0.18\textwidth}|>{\centering\arraybackslash}p{0.18\textwidth}|}
    \hline
    \textbf{Metrica} & \textbf{Nome} & \textbf{Valore ammissibile} & \textbf{Valore ottimo}& \textbf{Descrizione}\\
	\hline
    \endfirsthead
    \hline
    \textbf{Metrica} & \textbf{Nome} & \textbf{Valore ammissibile} & \textbf{Valore ottimo}& \textbf{Descrizione}\\
    \endhead
	\textbf{15M-QMS} & Quality Metrics Satisfied & $\geq 85\% $ & 100\% & Misura che valuta quante metriche, tra quelle definite, sono state implementate e soddisfatte.\\
	\hline
	\caption{ Valori delle metriche inerenti al processo di Verifica}
	\label{table:5}
\end{longtable}
\subsubsection{Processi organizzativi}
\subsubsubsection{Gestione dei processi}
\begin{longtable}{|>{\centering\arraybackslash}p{0.18\textwidth}|>{\centering\arraybackslash}p{0.18\textwidth}|>{\centering\arraybackslash}p{0.18\textwidth}|>{\centering\arraybackslash}p{0.18\textwidth}|>{\centering\arraybackslash}p{0.18\textwidth}|}
    \hline
    \textbf{Metrica} & \textbf{Nome} & \textbf{Valore ammissibile} & \textbf{Valore ottimo}& \textbf{Descrizione}\\
	\hline
    \endfirsthead
    \hline
    \textbf{Metrica} & \textbf{Nome} & \textbf{Valore ammissibile} & \textbf{Valore ottimo}& \textbf{Descrizione}\\
    \endhead
	\hline
	\textbf{16M-NCR} & Non Calculated Risk & $\leq 3 $ & 0 &\\
	\hline
	\textbf{17M-TE} & Time Efficiency & $\leq 3 $ & $\leq 1 $ &\\
	\hline

	\caption{ Valori delle metriche inerenti al processo di Gestione dei processi}
	\label{table:6}
\end{longtable}
\subsection{Qualità di prodotto}
Per qualità di prodotto si intende la capacità del software di rispettare 
le caratteristiche richieste dal cliente e quelle dettate dallo standard.
Più il risultato si avvicina a quello atteso, più la qualità del prodotto
sarà elevata. 
\subsubsection{Funzionalità}
\begin{longtable}{|>{\centering\arraybackslash}p{0.18\textwidth}|>{\centering\arraybackslash}p{0.18\textwidth}|>{\centering\arraybackslash}p{0.18\textwidth}|>{\centering\arraybackslash}p{0.18\textwidth}|>{\centering\arraybackslash}p{0.18\textwidth}|}
    \hline
    \textbf{Metrica} & \textbf{Nome} & \textbf{Valore ammissibile} & \textbf{Valore ottimo}& \textbf{Descrizione}\\
	\hline
    \endfirsthead
    \hline
    \textbf{Metrica} & \textbf{Nome} & \textbf{Valore ammissibile} & \textbf{Valore ottimo}& \textbf{Descrizione}\\
    \endhead
	\textbf{18M-CRO} & Copertura dei requisiti obbligatori & 100\%  & 100\% & Metrica che valuta quanto del lavoro svolto durante lo sviluppo corrisponda ai requisiti essenziali o obbligatori definiti in fase di analisi dei requisiti.\\
	\hline
	\textbf{19M-CRD} & Copertura dei requisiti desiderabili & $\geq 50\% $  & 100\% & Metrica usata per valutare quanti di quei requisiti, che se integrati arricchirebbero l'esperienza dell'utente o fornirebbero vantaggi aggiuntivi non strettamente necessari, sono stati implementati o soddisfatti nel prodotto.\\ 
	\hline
	\textbf{20M-CROP} & Copertura dei requisiti opzionali & $\geq 0\% $ & $\geq 50\% $ & Metrica per valutare quanti dei requisiti aggiuntivi, non essenziali o di bassa priorità, sono stati implementati o soddisfatti nel prodotto.\\ 
	\hline
	\caption{ Valori delle metriche inerenti alla Funzionalità del prodotto}
	\label{table:7}
\end{longtable}
\subsubsection{Affidabilità}
\begin{longtable}{|>{\centering\arraybackslash}p{0.18\textwidth}|>{\centering\arraybackslash}p{0.18\textwidth}|>{\centering\arraybackslash}p{0.18\textwidth}|>{\centering\arraybackslash}p{0.18\textwidth}|>{\centering\arraybackslash}p{0.18\textwidth}|}
    \hline
    \textbf{Metrica} & \textbf{Nome} & \textbf{Valore ammissibile} & \textbf{Valore ottimo}& \textbf{Descrizione}\\
	\hline
    \endfirsthead
    \hline
    \textbf{Metrica} & \textbf{Nome} & \textbf{Valore ammissibile} & \textbf{Valore ottimo}& \textbf{Descrizione}\\
    \endhead
	\hline
	\textbf{21M-CC} & Code Coverage & $\geq 80\% $  & 100\% &\\
	\hline
	\textbf{22M-BC} & Branch Coverage & $\geq 50\% $  & $\geq 80\% $ & Metrica di copertura del codice che indica la percentuale dei rami decisione del codice coperti dai test.\\ 
	\hline
	\textbf{23M-SC} & Statement Coverage & $\geq 60\% $ & $\geq 80\% $ & Metrica di copertura del codice che indica la percentuale degli statement del codice coperti dai test.\\ 
	\hline
	\textbf{24M-FD} & Failure Density & 100\%  & 100\%  &\\ 
	\hline
	\caption{ Valori delle metriche inerenti all'Affidabilità del prodotto}
	\label{table:8}
\end{longtable}
\subsubsection{Usabilità}
\begin{longtable}{|>{\centering\arraybackslash}p{0.18\textwidth}|>{\centering\arraybackslash}p{0.18\textwidth}|>{\centering\arraybackslash}p{0.18\textwidth}|>{\centering\arraybackslash}p{0.18\textwidth}|>{\centering\arraybackslash}p{0.18\textwidth}|}
    \hline
    \textbf{Metrica} & \textbf{Nome} & \textbf{Valore ammissibile} & \textbf{Valore ottimo}& \textbf{Descrizione}\\
	\hline
    \endfirsthead
    \hline
    \textbf{Metrica} & \textbf{Nome} & \textbf{Valore ammissibile} & \textbf{Valore ottimo}& \textbf{Descrizione}\\
    \endhead
	\textbf{25M-FU} & Facilità di utilizzo & $\leq 3 $ errori di utilizzo & 0 errori di utilizzo & Metrica che misura l'usabilità di un sistema software.\\
	\hline
	\textbf{26M-TA} & Tempo di apprendimento & $\leq 15 $ minuti  & $\leq 5 $ minuti & Misura il tempo massimo richiesto per apprendere l'utilizzo del prodotto. \\ 
	\hline

	\caption{ Valori delle metriche inerenti all'Usabilità del prodotto}
	\label{table:9}
\end{longtable}
\subsubsection{Efficienza}
	\begin{longtable}{|>{\centering\arraybackslash}p{0.18\textwidth}|>{\centering\arraybackslash}p{0.18\textwidth}|>{\centering\arraybackslash}p{0.18\textwidth}|>{\centering\arraybackslash}p{0.18\textwidth}|>{\centering\arraybackslash}p{0.18\textwidth}|}
    \hline
    \textbf{Metrica} & \textbf{Nome} & \textbf{Valore ammissibile} & \textbf{Valore ottimo}& \textbf{Descrizione}\\
	\hline
    \endfirsthead
    \hline
    \textbf{Metrica} & \textbf{Nome} & \textbf{Valore ammissibile} & \textbf{Valore ottimo}& \textbf{Descrizione}\\
    \endhead
	\textbf{27M-UR} & Utilizzo risorse & $\geq 75\% $  & 100\% & \\
	\hline
	\caption{ Valori delle metriche inerenti all'Efficienza del prodotto}
	\label{table:10}
	\end{longtable}
\subsubsection{Manutenibilità}
\begin{longtable}{|>{\centering\arraybackslash}p{0.18\textwidth}|>{\centering\arraybackslash}p{0.18\textwidth}|>{\centering\arraybackslash}p{0.18\textwidth}|>{\centering\arraybackslash}p{0.18\textwidth}|>{\centering\arraybackslash}p{0.18\textwidth}|}
    \hline
    \textbf{Metrica} & \textbf{Nome} & \textbf{Valore ammissibile} & \textbf{Valore ottimo}& \textbf{Descrizione}\\
	\hline
    \endfirsthead
    \hline
    \textbf{Metrica} & \textbf{Nome} & \textbf{Valore ammissibile} & \textbf{Valore ottimo}& \textbf{Descrizione}\\
    \endhead
	\textbf{28M-CCM} & Complessità ciclomatica & 1-10 & 11-20 & Rappresenta la complessità di un metodo in base ai percorsi possibili. \textbf{CONTROLLARE}\\
	\hline
	\textbf{29M-CSM} & Code Smell & 0 & 0 & \\ 
	\hline
	\textbf{30M-COC} & Coefficient of Coupling (COC) & $\leq 30\% $ & $\leq 10\% $ & \\ 
	\hline
	\caption{ Valori delle metriche inerenti alla Manutenibilità del prodotto}
	\label{table:11}
\end{longtable}