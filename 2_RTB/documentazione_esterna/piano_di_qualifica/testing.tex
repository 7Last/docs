\section{Metodologie di Testing}
La fase di testing è un'attività fondamentale per garantire la qualità del prodotto software. Permette di verificare che il software sia conforme ai requisiti e alle specifiche richieste e di individuare tempestivamente eventuali bug o problemi di funzionamento, così da poterli correggere prima del rilascio del prodotto; garantisce inoltre che gli stessi errori non si ripetano in futuro. \\
In questa sezione verranno descritte le metodologie di testing adottate per garantire il rispetto dei vincoli individuati nella sezione \textit{Requisiti} del documento \href{https://7last.github.io/docs/rtb/documentazione-esterna/analisi-dei-requisiti}{\href{https://7last.github.io/docs/rtb/documentazione-interna/glossario\#analisi-dei-requisiti}{\textit{Analisi dei Requisiti}\textsubscript{G}}}. \\ % TODO aggiungere versione documento
Nelle successive sottosezioni verranno descritte le tipologie di test effettuati con l'indicazione del codice del test, una breve descrizione di ciò che viene verificato e lo stato di superamento del test, espresso nel seguente modo: \\
\begin{itemize}
    \item \textbf{S}: test superato;
    \item \textbf{NS}: test non superato;
    \item \textbf{NI}: test non implementato.
\end{itemize}

\subsection{Test di Unità}
I test di unità sono test che verificano il corretto funzionamento delle singole unità di codice, ovvero le più piccole parti di un programma. Questi test vengono effettuati per verificare che ogni unità funzioni correttamente e che sia in grado di eseguire le operazioni richieste. \\
\begin{longtable}{|>{\raggedright\arraybackslash}m{0.1\textwidth}|>{\raggedright\arraybackslash}m{0.6\textwidth}|>{\raggedright\arraybackslash}m{0.1\textwidth}|}
	\hline
	\textbf{Codice} & \textbf{Descrizione} & \textbf{Stato} \\
	\hline
	\endfirsthead
	\hline
	\textbf{Codice} & \textbf{Descrizione} & \textbf{Stato} \\
	\endhead
	\textbf{1T-U}   & Verificare che la classe \textit{temperature} venga creata correttamente. & NI\\
	\hline
	\textbf{2T-U}   & Verificare che il livello di salute di una città si modifichi ad ogni nuova misurazione. & NI\\
	\hline
	\textbf{3T-U}   & Verificare che il metodo "GENERA\_MISURA" di una classe generi correttamente la misurazione. & NI\\
	\hline
	\caption{Test di Unità}
	\label{table:12}
\end{longtable}

\subsection{Test di Integrazione}
I test di integrazione sono test che verificano il corretto funzionamento delle interfacce tra le varie unità di codice. Questi test vengono effettuati per verificare che le varie unità di codice e i vari moduli interagiscano correttamente tra di loro e che siano in grado di comunicare e scambiarsi i dati necessari. \\
\begin{longtable}{|>{\raggedright\arraybackslash}m{0.1\textwidth}|>{\raggedright\arraybackslash}m{0.6\textwidth}|>{\raggedright\arraybackslash}m{0.1\textwidth}|}
	\hline
	\textbf{Codice} & \textbf{Descrizione} & \textbf{Stato} \\
	\hline
	\endfirsthead
	\hline
	\textbf{Codice} & \textbf{Descrizione} & \textbf{Stato} \\
	\endhead
	\textbf{1T-I}   & Verificare che i dati generati dal sensore di temperatura siano memorizzati correttamente nel database. & NI\\
	\hline
	\textbf{2T-I}   & Verificare che i dati generati dal sensore di traffico siano memorizzati correttamente nel database. & NI\\
	\hline
	\textbf{3T-I}   & Verificare che i dati della temperatura media siano memorizzati correttamente nel database. & NI\\
	\hline
	\textbf{4T-I}   & Verificare che i dati della velocità media dei veicoli siano memorizzati correttamente nel database. & NI\\
	\hline
	\textbf{5T-I}   & Verificare che i dati dei veicoli transitati siano memorizzati correttamente. & NI\\
	\hline
	\caption{Test di Integrazione} % TODO mettere sensore implementato da raul e davide
	\label{table:13}
\end{longtable}

\subsection{Test di Sistema}
I test di sistema sono finalizzati alla verifica del soddisfacimento dei requisiti richiesti ed evidenziati nel documento \href{https://7last.github.io/docs/rtb/documentazione-esterna/analisi-dei-requisiti}{\href{https://7last.github.io/docs/rtb/documentazione-interna/glossario\#analisi-dei-requisiti}{\textit{Analisi dei Requisiti}\textsubscript{G}}}. Questi test vengono effettuati sul sistema nel suo complesso, per verificare che il software funzioni correttamente e che sia in grado di eseguire le operazioni richieste. \\
\begin{longtable}{|>{\raggedright\arraybackslash}m{0.1\textwidth}|>{\raggedright\arraybackslash}m{0.6\textwidth}|>{\raggedright\arraybackslash}m{0.1\textwidth}|}
	\hline
	\textbf{Codice} & \textbf{Descrizione} & \textbf{Stato} \\
	\hline
	\endfirsthead
	\hline
	\textbf{Codice} & \textbf{Descrizione} & \textbf{Stato} \\
	\endhead
	\textbf{1T-S}   & Verificare che l'accesso al sistema non richieda alcuna procedura di login e che sia direttamente accessibile dall'utente.     & NI             \\
	\hline
	\textbf{2T-S}   & Verificare che il prodotto non abbia alcuna sezione o funzionalità di amministrazione o gestione riservata.     & NI\\
	\hline
	\textbf{3T-S}   & Verificare che i sensori integrati producano una misurazione coerente con il tipo di sensore simulato.     & NI\\
	\hline
	\textbf{4T-S}   & Verificare che ogni misurazione inviata dal simulatore contenga l’identificativo del sensore, la misurazione d'interesse e il timestamp. & NI\\
	\hline
	\textbf{5T-S}   & Verificare che il sistema sia in grado di ricevere e memorizzare correttamente le misurazioni inviate dai sensori. & NI\\
	\hline
	\textbf{6T-S}   & Verificare che il sistema sia in grado di simulare almeno un sensore per rilevare la temperatura. & NI\\
	\hline
	\textbf{7T-S}   & Verificare che ogni dato generato dai simulatori dei sensori sia strettamente correlato al dato successivo, garantendo una transizione realistica tra le misurazioni. & NI\\
	\hline
	\textbf{8T-S}   & Verificare che il prodotto di visualizzazione supporti la rappresentazione di dati provenienti da diversi tipi di sensori, permettendo una simulazione quanto più possibile reale. & NI\\
	\hline
	\textbf{9T-S}   & Verificare che l'utente possa vedere una dashboard completo delle condizioni attuali della città tramite l'uso di appositi widget rappresentanti le misurazioni dei sensori. & NI\\
	\hline
	\textbf{10T-S}   & Verificare la facilità di comprensione e l'intuitività dell'interfaccia grafica, garantendo un'esperienza utente piacevole e soddisfacente. & NI\\
	\hline
	\textbf{11T-S}   & Verificare che il sistema sia in grado di inviare notifiche all'utente in caso di superamento di soglie di temperatura predefinite. & NI\\
	\hline
	\textbf{12T-S}   & Verificare che l’utente possa vedere le misurazioni all’interno dei widget dedicati alla rappresentazione delle rilevazioni dei sensori in un formato testuale. & NI\\
	\hline
	\textbf{13T-S}   & Verificare che la dashboard si aggiorni quasi istantaneamente per riflettere i dati provenienti dai sensori entro un massimo di 15 secondi. & NI\\
	\hline
	\textbf{14T-S}   & Verificare che ogni widget che visualizza le misurazioni includa informazioni sull’identificativo dei sensori che hanno contribuito a quelle misurazioni. & NI\\
	\hline
	\textbf{15T-S}   & Verificare che la dashboard contenga almeno un widget dedicato alle misurazioni dei sensori di temperatura. & NI\\
	\hline
	\textbf{16T-S}   & Verificare che la dashboard contenga almeno un widget dedicato alle misurazioni dei sensori di traffico. & NI\\
	\hline
	\textbf{17T-S}   & Verificare che i sensori presenti sulla mappa siano distinguibili in modo chiaro ed inequivocabile, permettendo il riconoscimento della loro tipologia. & NI\\
	\hline
	\textbf{18T-S}   & Verificare che l’utente possa filtrare la visualizzazione delle misurazioni di una specifica tipologia di sensori. & NI\\
	\hline
	\textbf{19T-S}   & Verificare che il sistema verifichi la validità del sensore inserito dall’utente. & NI\\
	\hline
	\textbf{20T-S}   & Verificare che, in caso di inserimento di un sensore non valido, il sistema generi una notifica di errore. & NI\\
	\hline
	\textbf{21T-S}   & Verificare che la notifica di errore relativa all’inserimento di un sensore non valido richieda all’utente di inserire nuovamente il sensore. & NI\\
	\hline
	\textbf{22T-S}   & Verificare che la notifica di errore relativa all’inserimento di un sensore non valido sia chiara e informativa, indicando il motivo specifico dell’invalidità del sensore. & NI\\
	\hline
	\textbf{23T-S}   & Verificare che l’utente possa visualizzare la lista delle mis- urazioni rilevanti. & NI\\
	\hline
	\textbf{24T-S}   & Verificare che ogni misurazione nella lista dei più importanti fornisca correttamente l’identificativo del sensore. & NI\\
	\hline
	\textbf{25T-S}   & Verificare che ogni misurazione nella lista dei più importanti fornisca correttamente la tipologia del sensore. & NI\\
	\hline
	\textbf{26T-S}   & Verificare che ogni misurazione nella lista dei più importanti fornisca correttamente il valore della misurazione. & NI\\
	\hline
	\textbf{27T-S}   & Verificare che ogni misurazione nella lista dei più importanti fornisca correttamente il timestamp della misurazione. & NI\\
	\hline
	\textbf{28T-S}   & Verificare che l’utente possa visualizzare la lista delle misurazioni rilevanti in un formato tabellare. & NI\\ %da controllare
	\hline
	\textbf{29T-S}   & Verificare che l’utente possa rimuovere una misurazione specifica dalla lista delle misurazioni rilevanti. & NI\\
	\hline
	\textbf{30T-S}   & Verificare che l’utente possa rimuovere tutte le misurazioni dalla lista delle misurazioni rilevanti. & NI\\
	\hline
	\textbf{31T-S}   & Verificare che l’utente riceva notifiche quando i sensori superano pre-determinate soglie di sicurezza. & NI\\
	\hline
	\textbf{32T-S}   & Verificare che l’utente possa visualizzare le notifiche relative ai sensori che superano le soglie di sicurezza. & NI\\
	\hline
	\textbf{33T-S}   & Verificare che l’utente possa visualizzare correttamente le informazioni richieste per i sensori. & NI\\
	\hline
	\textbf{34T-S}   & Verificare che l’utente possa visualizzare correttamente la posizione in coordinate dei sensori. & NI\\
	\hline
	\textbf{35T-S}   & Verificare che l’utente possa visualizzare correttamente la data di installazione del sensore. & NI\\
	\hline
	\textbf{36T-S}   & Verificare che l’utente possa visualizzare correttamente la data di ultima manutenzione del sensore. & NI\\
	\hline
	\textbf{37T-S}   & Verificare che l’utente possa visualizzare correttamente l’unità di misura associata al sensore. & NI\\
	\hline
	\caption{Test di Sistema} % TODO aggiungere test effettivamente progettati
	\label{table:14}
\end{longtable}

% \subsection{Test di Regressione}
% I test di regressione sono test che vengono effettuati per verificare che le modifiche apportate al software non abbiano introdotto nuovi errori o problemi di funzionamento e che il software continui a funzionare correttamente anche dopo le modifiche fatte. \\
% \begin{longtable}{|>{\raggedright\arraybackslash}m{0.1\textwidth}|>{\raggedright\arraybackslash}m{0.6\textwidth}|>{\raggedright\arraybackslash}m{0.1\textwidth}|}
% 	\hline
% 	\textbf{Tipologia di test} & \textbf{Codice} & \textbf{Stato} \\
% 	\hline
% 	\endfirsthead
% 	\hline
% 	\textbf{Tipologia di test} & \textbf{Codice} & \textbf{Stato} \\
% 	\endhead
% 	\textbf{Test di unità}   		& Codice del test che andremo ad effettuare     & NI\\
% 	\hline
% 	\textbf{Test di sistema}  		& Codice del test che andremo ad effettuare     & NI\\
% 	\hline
% 	\textbf{Test di integrazione}   & Codice del test che andremo ad effettuare     & NI\\
% 	\hline
% 	\caption{Test di Regressione} % TODO aggiungere test effettivamente progettati
% 	\label{table:15}
% \end{longtable}
\newpage
\subsection{Test di Accettazione}
I test di accettazione sono test che vengono effettuati per verificare che il software soddisfi i requisiti richiesti. Consentono di ultimare il processo di validazione del prodotto finale. Questi test verranno eseguiti sia dal gruppo di sviluppo \textit{7Last} che dall'azienda \href{https://7last.github.io/docs/rtb/documentazione-interna/glossario\#proponente}{proponente\textsubscript{G}} \textit{SyncLab S.r.l.}. \\
\begin{longtable}{|>{\raggedright\arraybackslash}m{0.1\textwidth}|>{\raggedright\arraybackslash}m{0.6\textwidth}|>{\raggedright\arraybackslash}m{0.1\textwidth}|}
	\hline
	\textbf{Codice} & \textbf{Descrizione} & \textbf{Stato} \\
	\hline
	\endfirsthead
	\hline
	\textbf{Codice} & \textbf{Descrizione} & \textbf{Stato} \\
	\endhead
	\textbf{1T-A}   & Verificare che tutti i widget relativi alle diverse tipologie di sensori siano visibili sulla dashboard. & NI\\
	\hline
	\textbf{2T-A}   & Verificare che la mappa dei sensori si carichi correttamente e permetta interazioni fluide. & NI\\
	\hline
	\textbf{3T-A}   & Verificare che le informazioni di un sensore specifico siano visualizzate correttamente quando selezionate dalla dashboard. & NI\\
	\hline
	\textbf{4T-A}   & Verifica della gestione corretta degli errori nel caso in cui i dati dei sensori non siano disponibili o siano incompleti all’interno della visualizzazione testuale. & NI\\
	\hline
	\textbf{5T-A}   & Verifica della corretta visualizzazione delle misurazioni effettuate nel tempo dai sensori. & NI\\
	\hline
	\textbf{6T-A}   & Verificare l’accuratezza e la completezza delle opzioni di interazione offerte dall’interfaccia del widget per esaminare i dati di temperatura. & NI\\
	\hline
	\textbf{7T-A}   & Verificare ci sia l’opportunità di visualizzare correttamente il widget contenente le misurazioni dei sensori di temperatura. 	& NI\\
	\hline
	\textbf{8T-A}   & Verificare si possa filtrare correttamente la visualizzazione delle misurazioni in base ad una specifica selezione di sensori. & NI\\
	\hline
	\textbf{9T-A}   & Verificare che si possa rimuovere correttamente i filtri attivi per visualizzazione delle misurazioni dei sensori. & NI\\
	\hline
	\textbf{10T-A}   & Verificare che si riceva correttamente una notifica in caso di superamento delle soglie impostate per le misurazioni. & NI\\
	\hline
	\textbf{11T-A}   & Verificare che si possa visualizzare correttamente la lista delle misurazioni rilevanti. & NI\\
	\hline
	\textbf{12T-A}   & Verificare che si possa rimuovere correttamente una misurazione specifica dalla lista delle misurazioni rilevanti. & NI\\
	\hline
	\textbf{13T-A}   & Verificare che si possa rimuovere correttamente tutte le misurazioni dalla lista delle misurazioni rilevanti. & NI\\
	\hline
	\textbf{14T-A}   & Verificare che si possa inserire correttamente una misurazione nella lsita delle misurazioni rilevanti. & NI\\
	\hline
	\caption{Test di Accettazione} % TODO aggiungere test effettivamente progettati
	\label{table:16}
\end{longtable}
