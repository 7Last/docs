\section{Metodologie di Testing}
La fase di testing è un'attività fondamentale per garantire la qualità del prodotto software. Permette di verificare che il software sia conforme ai requisiti e alle specifiche richieste e di individuare tempestivamente eventuali bug o problemi di funzionamento, così da poterli correggere prima del rilascio del prodotto; garantisce inoltre che gli stessi errori non si ripetano in futuro. \\
In questa sezione verranno descritte le metodologie di testing adottate per garantire il rispetto dei vincoli individuati nella sezione \textit{Requisiti} del documento \href{https://7last.github.io/docs/rtb/documentazione-esterna/analisi-dei-requisiti}{\href{https://7last.github.io/docs/rtb/documentazione-interna/glossario\#analisi-dei-requisiti}{\textit{Analisi dei Requisiti}\textsubscript{G}}}. \\ % TODO aggiungere versione documento
Nelle successive sottosezioni verranno descritte le tipologie di test effettuati con l'indicazione del codice del test, una breve descrizione di ciò che viene verificato e lo stato di superamento del test, espresso nel seguente modo: \\
\begin{itemize}
    \item \textbf{S}: test superato;
    \item \textbf{NS}: test non superato;
    \item \textbf{NI}: test non implementato.
\end{itemize}

\subsection{Test di Unità}
I test di unità sono test che verificano il corretto funzionamento delle singole unità di codice, ovvero le più piccole parti di un programma. Questi test vengono effettuati per verificare che ogni unità funzioni correttamente e che sia in grado di eseguire le operazioni richieste. \\
\begin{longtable}{|>{\raggedright\arraybackslash}m{0.1\textwidth}|>{\raggedright\arraybackslash}m{0.6\textwidth}|>{\raggedright\arraybackslash}m{0.1\textwidth}|}
	\hline
	\textbf{Codice} & \textbf{Descrizione} & \textbf{Stato} \\
	\hline
	\endfirsthead
	\hline
	\textbf{Codice} & \textbf{Descrizione} & \textbf{Stato} \\
	\endhead
	\textbf{1T-U}   & Descrizione test     & NI             \\
	\hline
	\textbf{2T-U}   & Descrizione test     & NI             \\
	\hline
	\caption{Test di Unità} % TODO aggiungere test effettivamente progettati
	\label{table:12}
\end{longtable}

\subsection{Test di Integrazione}
I test di integrazione sono test che verificano il corretto funzionamento delle interfacce tra le varie unità di codice. Questi test vengono effettuati per verificare che le varie unità di codice e i vari moduli interagiscano correttamente tra di loro e che siano in grado di comunicare e scambiarsi i dati necessari. \\
\begin{longtable}{|>{\raggedright\arraybackslash}m{0.1\textwidth}|>{\raggedright\arraybackslash}m{0.6\textwidth}|>{\raggedright\arraybackslash}m{0.1\textwidth}|}
	\hline
	\textbf{Codice} & \textbf{Descrizione} & \textbf{Stato} \\
	\hline
	\endfirsthead
	\hline
	\textbf{Codice} & \textbf{Descrizione} & \textbf{Stato} \\
	\endhead
	\textbf{3T-I}   & Descrizione test     & NI             \\
	\hline
	\textbf{4T-I}   & Descrizione test     & NI             \\
	\hline
	\caption{Test di Integrazione} % TODO aggiungere test effettivamente progettati
	\label{table:13}
\end{longtable}

\subsection{Test di Sistema}
I test di sistema sono finalizzati alla verifica del soddisfacimento dei requisiti richiesti ed evidenziati nel documento \href{https://7last.github.io/docs/rtb/documentazione-esterna/analisi-dei-requisiti}{\href{https://7last.github.io/docs/rtb/documentazione-interna/glossario\#analisi-dei-requisiti}{\textit{Analisi dei Requisiti}\textsubscript{G}}}. Questi test vengono effettuati sul sistema nel suo complesso, per verificare che il software funzioni correttamente e che sia in grado di eseguire le operazioni richieste. \\
\begin{longtable}{|>{\raggedright\arraybackslash}m{0.1\textwidth}|>{\raggedright\arraybackslash}m{0.6\textwidth}|>{\raggedright\arraybackslash}m{0.1\textwidth}|}
	\hline
	\textbf{Codice} & \textbf{Descrizione} & \textbf{Stato} \\
	\hline
	\endfirsthead
	\hline
	\textbf{Codice} & \textbf{Descrizione} & \textbf{Stato} \\
	\endhead
	\textbf{5T-S}   & Descrizione test     & NI             \\
	\hline
	\textbf{6T-S}   & Descrizione test     & NI             \\
	\hline
	\caption{Test di Sistema} % TODO aggiungere test effettivamente progettati
	\label{table:14}
\end{longtable}

\subsection{Test di Regressione}
I test di regressione sono test che vengono effettuati per verificare che le modifiche apportate al software non abbiano introdotto nuovi errori o problemi di funzionamento e che il software continui a funzionare correttamente anche dopo le modifiche effettuate. \\
\begin{longtable}{|>{\raggedright\arraybackslash}m{0.1\textwidth}|>{\raggedright\arraybackslash}m{0.6\textwidth}|>{\raggedright\arraybackslash}m{0.1\textwidth}|}
	\hline
	\textbf{Codice} & \textbf{Descrizione} & \textbf{Stato} \\
	\hline
	\endfirsthead
	\hline
	\textbf{Codice} & \textbf{Descrizione} & \textbf{Stato} \\
	\endhead
	\textbf{7T-R}   & Descrizione test     & NI             \\
	\hline
	\textbf{8T-R}   & Descrizione test     & NI             \\
	\hline
	\caption{Test di Regressione} % TODO aggiungere test effettivamente progettati
	\label{table:15}
\end{longtable}

\subsection{Test di Accettazione}
I test di accettazione sono test che vengono effettuati per verificare che il software soddisfi i requisiti richiesti. Consentono di ultimare il processo di validazione del prodotto finale. Questi test verranno eseguiti sia dal gruppo di sviluppo \textit{7Last} che dall'azienda \href{https://7last.github.io/docs/rtb/documentazione-interna/glossario\#proponente}{proponente\textsubscript{G}} \textit{SyncLab S.r.l.}. \\
\begin{longtable}{|>{\raggedright\arraybackslash}m{0.1\textwidth}|>{\raggedright\arraybackslash}m{0.6\textwidth}|>{\raggedright\arraybackslash}m{0.1\textwidth}|}
	\hline
	\textbf{Codice} & \textbf{Descrizione} & \textbf{Stato} \\
	\hline
	\endfirsthead
	\hline
	\textbf{Codice} & \textbf{Descrizione} & \textbf{Stato} \\
	\endhead
	\textbf{9T-A}   & Descrizione test     & NI             \\
	\hline
	\textbf{10T-A}   & Descrizione test     & NI             \\
	\hline
	\caption{Test di Accettazione} % TODO aggiungere test effettivamente progettati
	\label{table:16}
\end{longtable}
