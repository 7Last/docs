\section{Iniziative di automiglioramento per la qualità}
\subsection{Introduzione}
In questa sezione verranno riportate le iniziative di automiglioramento che il nostro gruppo ha deciso di adottare per 
aumentare la qualità del prodotto e dei processi. Queste iniziative sono state individuate grazie all'esperienza acquisita 
durante lo svolgimento del progetto e grazie alle valutazioni effettuate sulle attività svolte. \\
Trattandosi per tutti noi della prima esperienza con un progetto di questa portata, 
è stato necessario un grande numero di tentativi per comprendere al meglio come organizzarci e come svolgere le attività.
Questo ci ha permesso di capire quali sono stati i punti di forza e i punti deboli del nostro lavoro e di individuare le aree 
in cui è possibile migliorare. \\
Per ciascuna delle difficoltà riscontrate verranno indicate:
\begin{itemize}
    \item fase del progetto in cui si è verificato il problema;
    \item descrizione del problema;
    \item contromisura adottata per risolvere il problema evidenziato.
\end{itemize}

\subsection{Problemi rilevati ed iniziative adottate}
\begin{itemize}
    \item \textbf{Organizzazione delle riunioni}
    \begin{itemize}
        \item \textbf{Fase del progetto}: iniziale;
        \item \textbf{Descrizione}: nelle prime settimane di lavoro, a partire dalla formazione dei gruppi sino ai primi Diari di bordo, 
        si è riscontrata una certa difficoltà nell'organizzazione delle riunioni causata dai vari impegni di ciascun membro 
        (lezioni diverse in orari diversi, lavoro per alcuni, impegni personali) e soprattutto alimentata dalle diverse riunioni che si 
        accumulavano (\href{https://7last.github.io/docs/rtb/documentazione-interna/glossario\#stato-avanzamento-lavori}{SAL\textsubscript{G}} con l'azienda prima e Diari di bordo poi) portando a una certa confusione e a un 
        rallentamento delle attività;
        \item \textbf{Contromisura}: abbiamo deciso di effettuare le riunioni a distanza tramite la piattaforma \textit{Discord} 
        e di fissare un giorno e un orario durante la settimana per ciascuna tipologia di incontro in maniera tale da
        rispettare le disponibilità di ogni membro;
        qualora qualcuno, per impegni di natura eccezionale, non abbia modo di essere presente potrà successivamente informarsi sui 
        contenuti trattati attraverso i verbali che verranno redatti e messi a disposizione di tutti.
    \end{itemize}
    \item \textbf{Suddivisione compiti}
    \begin{itemize}
        \item \textbf{Fase del progetto}: iniziale;
        \item \textbf{Descrizione}: all'inizio del progetto si è riscontrata una certa difficoltà nella suddivisione dei 
        compiti a causa della mancanza di esperienza e della poca conoscenza delle competenze possedute da ciascuno. È risultato dunque 
        difficile il bilanciamento delle mansioni e si sono verificati più volte casi in cui alcuni membri sono stati in grado di completare
        le attività a loro assegnate in anticipo, e casi opposti in cui il lavoro da svolgere è risultato eccessivo
        e difficilmente completabile entro i tempi prestabiliti;
        \item \textbf{Contromisura}: abbiamo quindi deciso, come suggerito anche dal professor Vardanega al primo Diario di bordo, 
        di non assegnare preventivamente tutti i compiti da svolgere a ciascun membro, ma piuttosto di metterli in un contenitore 
        condiviso (abbiamo deciso di usare le annotazioni di \href{https://7last.github.io/docs/rtb/documentazione-interna/glossario\#clickup}{\textit{ClickUp}\textsubscript{G}}) e di permettere a ciascun membro di prendere in 
        autonomia i compiti da svolgere, così che chiunque finisca in anticipo possa prenderne altri; in questo modo siamo riusciti 
        a svolgere le attività in modo più equo e a completare i compiti entro i tempi prestabiliti.
    \end{itemize}
\end{itemize}

\subsection{Considerazioni finali}
Fin da subito il nostro gruppo si è posto come obiettivo principale quello di dotarsi di un \textit{Way of Working} preciso 
e ben definito, di pianificare ogni singola attività e di prevedere tutte le possibili difficoltà che avremmo potuto incontrare
durante lo svolgimento del progetto. Questo per cercare di prevenire i problemi prima che potessero avvenire o di fornire delle 
contromisure per affrontarli nel caso in cui si fossero presentati. \\
Inizialmente si sono palesate delle difficoltà, causate dall'inesperienza del gruppo in ambito organizzativo. Tuttavia, grazie alla 
familiarizzazione ottenuta tramite lo svolgimento del progetto e grazie ai consigli e ai suggerimenti che ci 
sono stati forniti dai professori e dall'azienda \href{https://7last.github.io/docs/rtb/documentazione-interna/glossario\#proponente}{proponente\textsubscript{G}}, 
siamo riusciti a individuare i problemi e a mettere in atto delle contromisure per risolverli. \\
Questo ci ha permesso di migliorare notevolmente la qualità del nostro lavoro e di svolgere le attività in modo più efficiente e più equo. 
Nonostante ciò siamo anche consapevoli che ci sono ancora molti aspetti su cui possiamo progredire e che ci sono ancora molte 
iniziative di automiglioramento che possiamo adottare. Tuttavia, siamo convinti che, se continueremo a lavorare con lo stesso impegno e 
la stessa determinazione che abbiamo dimostrato finora, saremo in grado di ottenere ottimi risultati di qualità superiore.