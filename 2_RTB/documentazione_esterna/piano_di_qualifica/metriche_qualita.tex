\section{Metriche di qualità per obiettivo}
La qualità di processo è un criterio fondamentale ed è alla base di ogni prodotto che rispecchi lo stato dell'arte. Per raggiungere tale obiettivo è necessario sfruttare delle pratiche rigorose che consentano lo svolgimento di ogni attività in maniera ottimale. Dunque, al fine di valutare nel miglior modo possibile la qualità del prodotto e l'efficacia dei processi, sono state definite delle metriche, meglio specificate nel documento \href{https://7last.github.io/docs/rtb/documentazione-interna/glossario#norme-di-progetto}{Norme di Progetto\textsubscript{G}}. Il contenuto di questa sezione è necessario per identificare i parametri che le metriche devono rispettare per essere considerate accettabili o ottime. Esse sono state suddivise utilizzando lo \textbf{standard ISO/IEC 12207:1995}, il quale separa i processi di ciclo di vita del software, in tre categorie: % TODO aggiungere versione documento
\begin{itemize}
	\item Processi di base e/o primari;
	\item Processi di supporto;
	\item Processi organizzativi.
\end{itemize}

\subsection{Processi di base e/o primari}
\subsubsection{Analisi dei requisiti}
Questa fase consiste nell'esaminare le richieste del \href{https://7last.github.io/docs/rtb/documentazione-interna/glossario\#proponente}{proponente\textsubscript{G}} e nel definire i requisiti che il prodotto dovrà soddisfare. Per valutare la qualità di tale processo, sono state definite le seguenti metriche:
\begin{longtable}{|>{\raggedright\arraybackslash}m{0.13\textwidth}|>{\raggedright\arraybackslash}m{0.17\textwidth}|>{\raggedright\arraybackslash}m{0.14\textwidth}|>{\raggedright\arraybackslash}m{0.14\textwidth}|>{\raggedright\arraybackslash}m{0.29\textwidth}|}
	\hline
	\textbf{Metrica} & \textbf{Nome}                        & \textbf{Valore ammissibile} & \textbf{Valore ottimo} & \textbf{Descrizione}                                                                                                                                                                                      \\
	\hline
	\endfirsthead
	\hline
	\textbf{Metrica} & \textbf{Nome}                        & \textbf{Valore ammissibile} & \textbf{Valore ottimo} & \textbf{Descrizione}                                                                                                                                                                                      \\
	\endhead
	\textbf{1M-CRO}  & Copertura dei requisiti obbligatori  & 100\%                       & 100\%                  & Descrive quanto del lavoro svolto durante lo sviluppo corrisponde ai requisiti essenziali o obbligatori definiti in fase di \href{https://7last.github.io/docs/rtb/documentazione-interna/glossario\#analisi-dei-requisiti}{Analisi dei Requisiti\textsubscript{G}}.                                                        \\
	\hline
	\textbf{2M-CRD}  & Copertura dei requisiti desiderabili & $\geq 50\% $                & 100\%                  & Rileva la percentuale di requisiti (i quali, una volta integrati arricchiscono l'esperienza dell'utente o forniscono vantaggi aggiuntivi non strettamente necessari) che sono stati implementati o soddisfatti nel prodotto. \\
	\hline
	\textbf{3M-CROP} & Copertura dei requisiti opzionali    & $\geq 0\% $                 & $\geq 50\% $           & Stima la percentuale di requisiti aggiuntivi (non essenziali o di bassa priorità) che sono stati implementati o soddisfatti nel prodotto.                                                                            \\
	\hline
	\caption{Metriche di \href{https://7last.github.io/docs/rtb/documentazione-interna/glossario\#analisi-dei-requisiti}{Analisi dei Requisiti\textsubscript{G}}}
	\label{table:1}
\end{longtable}

\subsubsection{Progettazione}
In questa fase si definiscono le specifiche del prodotto, quali ad esempio dettagli tecnici e design architetturale del sistema.
Per valutare la qualità di tale processo, sono state definite le seguenti metriche:
\subsubsubsection{Usabilità}
\begin{longtable}{|>{\raggedright\arraybackslash}m{0.13\textwidth}|>{\raggedright\arraybackslash}m{0.17\textwidth}|>{\raggedright\arraybackslash}m{0.14\textwidth}|>{\raggedright\arraybackslash}m{0.14\textwidth}|>{\raggedright\arraybackslash}m{0.29\textwidth}|}
	\hline
	\textbf{Metrica} & \textbf{Nome}          & \textbf{Valore ammissibile}  & \textbf{Valore ottimo} & \textbf{Descrizione}                                                      \\
	\hline
	\endfirsthead
	\hline
	\textbf{Metrica} & \textbf{Nome}          & \textbf{Valore ammissibile}  & \textbf{Valore ottimo} & \textbf{Descrizione}                                                      \\
	\endhead
	\textbf{4M-FU}   & Facilità di utilizzo   & $\leq 3 $ errori di utilizzo & 0 errori di utilizzo   & Rappresenta l'usabilità di un sistema software.                           \\
	\hline
	\textbf{5M-TA}   & Tempo di apprendimento & $\leq 12 $ minuti            & $\leq 8 $ minuti       & Indica il tempo massimo richiesto per apprendere l'utilizzo del prodotto. \\
	\hline
	\caption{Metriche di Progettazione - Usabilità}
	\label{table:2}
\end{longtable}

\subsubsubsection{Manutenibilità}
\begin{longtable}{|>{\raggedright\arraybackslash}m{0.13\textwidth}|>{\raggedright\arraybackslash}m{0.17\textwidth}|>{\raggedright\arraybackslash}m{0.14\textwidth}|>{\raggedright\arraybackslash}m{0.14\textwidth}|>{\raggedright\arraybackslash}m{0.29\textwidth}|}
	\hline
	\textbf{Metrica}  & \textbf{Nome}                 & \textbf{Valore ammissibile} & \textbf{Valore ottimo} & \textbf{Descrizione}                                                                                                  \\
	\hline
	\endfirsthead
	\hline
	\textbf{6M-COC}   & Coefficient of Coupling (COC) & $\leq 30\% $                & $\leq 10\% $           & Rappresenta il grado di dipendenza tra diversi moduli o componenti di un sistema software.                            \\
	\hline
	\textbf{7M-SFIN}  & Structural Fan-In (SFIN)      & $\leq$ 7                    & $\le$ 5                & Riferita ad una classe che è progettata in modo tale che un gran numero di altre classi possa facilmente utilizzarla. \\
	\hline
	\textbf{8M-SFOUT} & Structural Fan-Out (SFOUT)    & $\leq$ 7                    & $\le$ 5                & Rappresenta il numero dei moduli subordinati immediati di un metodo.                                                  \\
	\hline
	\caption{Metriche di Progettazione - Manutenibilità}
	\label{table:3}
\end{longtable}

\subsubsection{Fornitura}
Nella fase di fornitura si definiscono le procedure e le risorse (economiche e temporali) necessarie per la consegna del prodotto.
Per valutare la qualità di tale processo, sono state definite le seguenti metriche:
\begin{longtable}{|>{\raggedright\arraybackslash}m{0.13\textwidth}|>{\raggedright\arraybackslash}m{0.17\textwidth}|>{\raggedright\arraybackslash}m{0.14\textwidth}|>{\raggedright\arraybackslash}m{0.14\textwidth}|>{\raggedright\arraybackslash}m{0.29\textwidth}|}
	\hline
	\textbf{Metrica}  & \textbf{Nome}                & \textbf{Valore ammissibile} & \textbf{Valore ottimo} & \textbf{Descrizione}\\
	\hline
	\endhead
	\textbf{9M-EV}   & Earned Value (EV)             & $\geq 0$                                                    & $\leq$ EAC (Estimated At Completion)      & Valore del lavoro effettivamente svolto fino al periodo in analisi.                                                            \\
	\hline
	\textbf{10M-PV}   & Planned Value (PV)            & $\geq 0$                                                    & $\leq$ BAC (Budget At Completion)         & Consente di stimare i costi realizzativi delle attività imminenti periodo per periodo.                                         \\
	\hline
	\textbf{11M-AC}  & Actual Cost (AC)              & $\geq 0$                                                    & $\leq$ EAC (Estimated At Completion)      & Misura i costi effettivamente sostenuti dall'inizio del progetto fino al presente.                                             \\
	\hline
	\textbf{12M-CV}  & Cost Variance (CV)            & $\geq -7.5\%$                                               & $\geq 0\%$                                & Valuta la differenza percentuale di budget tra quanto previsto nella pianificazione di un periodo e l'effettiva realizzazione. \\
	\hline
	\textbf{13M-EAC} & Estimated at Completion (EAC) & Errore del $\pm 4\%$ rispetto al BAC (Budget At Completion) & Equivalente al BAC (Budget At Completion) & Calcola il costo realizzativo stimato per terminare il progetto.                                                               \\
	\hline
	\textbf{14M-ETC} & Estimate to Complete (ETC)    & $\geq 0$                                                    & $\leq$ EAC (Estimated At Completion)      & Previsione dei costi realizzativi fino alla fine del progetto.                                                                 \\
	\hline
	\textbf{15M-CPI} & Cost Performance Index (CPI)  & $\pm 13\%$                                                  & 0                                         & Indica il rapporto tra il valore del lavoro effettivamente svolto e i costi sostenuti.                                         \\
	\hline
	\caption{Metriche di Fornitura}
	\label{table:4}
\end{longtable}

\subsubsection{Sviluppo}
Nella fase di sviluppo si realizza il prodotto software, seguendo le specifiche definite in fase di progettazione.
Per valutare la qualità di tale processo, sono state definite le seguenti metriche:
\subsubsubsection{Complessità e struttura del codice}
\begin{longtable}{|>{\raggedright\arraybackslash}m{0.13\textwidth}|>{\raggedright\arraybackslash}m{0.17\textwidth}|>{\raggedright\arraybackslash}m{0.14\textwidth}|>{\raggedright\arraybackslash}m{0.14\textwidth}|>{\raggedright\arraybackslash}m{0.29\textwidth}|}
	\hline
	\textbf{Metrica}  & \textbf{Nome}              & \textbf{Valore ammissibile} & \textbf{Valore ottimo} & \textbf{Descrizione}                                                                                \\
	\hline
	\endhead
	\textbf{16M-CCM}  & Complessità ciclomatica    & $\leq 3$                    & $\leq 6$               & Indica il numero di cammini linearmente indipendenti attraverso il codice sorgente di un programma. \\
	\hline
	\textbf{17M-PPM}  & Parametri per metodo       & $\leq 7$                    & $\leq 5$               & Indica il numero di parametri per metodo.                                                           \\
	\hline
	\textbf{18M-CPC}  & Campi per classe           & $\leq 10$                   & $\leq 7$               & Indica il numero di parametri per classe.                                                           \\
	\hline
	\textbf{19M-LCPM} & Linee di codice per metodo & $\leq 30$                   & $\leq 20$              & Indica il numero di linee di codice per metodo.                                                     \\
	\hline
	\caption{Metriche di Sviluppo - Complessità e struttura del codice}
	\label{table:5}
\end{longtable}

\subsubsubsection{Efficienza}
\begin{longtable}{|>{\raggedright\arraybackslash}m{0.13\textwidth}|>{\raggedright\arraybackslash}m{0.17\textwidth}|>{\raggedright\arraybackslash}m{0.14\textwidth}|>{\raggedright\arraybackslash}m{0.14\textwidth}|>{\raggedright\arraybackslash}m{0.29\textwidth}|}
	\hline
	\textbf{Metrica}  & \textbf{Nome}                & \textbf{Valore ammissibile} & \textbf{Valore ottimo} & \textbf{Descrizione}\\
	\hline
	\endhead
	\textbf{20M-TR}  & Tempo di risposta (interfaccia utente)  & $\leq 1.5$ s                 & $\leq 1$s              & Indica il tempo massimo di risposta del sistema.                                       \\
	\hline
	\textbf{21M-TE}  & Tempo di elaborazione di un dato grezzo & $\leq 1.5$ s                 & $\leq 1$s              & Indica il tempo massimo di elaborazione di un dato grezzo fino alla sua presentazione. \\
	\hline
	\caption{Metriche di Sviluppo - Efficienza}
	\label{table:6}
\end{longtable}


\subsection{Processi di supporto}
I processi di supporto si affiancano ai processi primari per garantire il corretto svolgimento delle attività.

\subsubsection{Documentazione}
La documentazione è un aspetto fondamentale per la comprensione del prodotto e per la sua manutenibilità. Consiste,
a livello pratico, nella redazione di manuali e documenti tecnici che descrivano il funzionamento del prodotto e le
scelte progettuali adottate.
Per valutare la qualità di tale processo, sono state definite le seguenti metriche:
\begin{longtable}{|>{\raggedright\arraybackslash}m{0.13\textwidth}|>{\raggedright\arraybackslash}m{0.17\textwidth}|>{\raggedright\arraybackslash}m{0.14\textwidth}|>{\raggedright\arraybackslash}m{0.14\textwidth}|>{\raggedright\arraybackslash}m{0.29\textwidth}|}
	\hline
	\textbf{Metrica} & \textbf{Nome}           & \textbf{Valore ammissibile} & \textbf{Valore ottimo} & \textbf{Descrizione}                                                                 \\
	\hline
	\endfirsthead
	\textbf{22M-IG}  & Indice Gulpease         & $\geq 60\% $                & $\geq 90\% $           & Misura la leggibilità di un testo in base alla lunghezza delle parole e delle frasi. \\
	\hline
	\textbf{23M-CO}  & Correttezza Ortografica & 0 errori                    & 0 errori               & Presenza di errori ortografici nei documenti.                                        \\
	\hline
	\caption{Metriche di Documentazione}
	\label{table:7}
\end{longtable}

\subsubsection{Verifica}
La verifica è un processo che si occupa di controllare che il prodotto soddisfi i requisiti stabiliti
e sia pienamente funzionante. Per valutare la qualità di tale processo, sono state definite le seguenti metriche:
\begin{longtable}{|>{\raggedright\arraybackslash}m{0.13\textwidth}|>{\raggedright\arraybackslash}m{0.17\textwidth}|>{\raggedright\arraybackslash}m{0.14\textwidth}|>{\raggedright\arraybackslash}m{0.14\textwidth}|>{\raggedright\arraybackslash}m{0.29\textwidth}|}
	\hline
	\textbf{Metrica}  & \textbf{Nome}                & \textbf{Valore ammissibile} & \textbf{Valore ottimo} & \textbf{Descrizione}\\
	\hline
	\endhead
	\textbf{24M-CC}   & Code Coverage                & $\geq 80\% $                & 100\%                  & Fornisce una misura quantitativa del grado o della percentuale di codice eseguito durante i test.                  \\
	\hline
	\textbf{25M-BC}   & Branch Coverage              & $\geq 80\% $                & 100\%                  & Metrica di copertura del codice che indica la percentuale dei rami decisione del codice coperti dai test.          \\
	\hline
	\textbf{26M-SC}   & Statement Coverage           & $\geq 80\% $                & 100\%                  & Metrica di copertura del codice che indica la percentuale degli statement del codice coperti dai test.             \\
	\hline
	\textbf{27M-FD}   & Failure Density              & 100\%                       & 100\%                  & Misura che indica il numero di difetti trovati in un software o in una parte di esso durante il ciclo di sviluppo. \\
	\hline
	\textbf{28M-PTCP} & Passed Test Cases Percentage & $\geq 80\%$                 & 100\%                  & Percentuale di casi di test superati.                                                                              \\
	\hline
	\caption{Metriche di Verifica}
	\label{table:8}
\end{longtable}

\subsubsection{Gestione dei rischi}
La gestione dei rischi è un processo che si occupa di identificare, analizzare e gestire i rischi che possono
insorgere durante lo svolgimento del progetto. Per valutare la qualità di tale processo, sono state definite le seguenti metriche:
\begin{longtable}{|>{\raggedright\arraybackslash}m{0.13\textwidth}|>{\raggedright\arraybackslash}m{0.17\textwidth}|>{\raggedright\arraybackslash}m{0.14\textwidth}|>{\raggedright\arraybackslash}m{0.14\textwidth}|>{\raggedright\arraybackslash}m{0.29\textwidth}|}
	\hline
	\textbf{Metrica} & \textbf{Nome}       & \textbf{Valore ammissibile} & \textbf{Valore ottimo} & \textbf{Descrizione}                                                                     \\
	\hline
	\endfirsthead
	\hline
	\textbf{Metrica} & \textbf{Nome}       & \textbf{Valore ammissibile} & \textbf{Valore ottimo} & \textbf{Descrizione}                                                                     \\
	\endhead
	\textbf{29M-NCR} & Non Calculated Risk & $\leq 3 $                   & 0                      & Indica un rischio che è stato trascurato o non considerato durante l’Analisi dei Rischi. \\
	\hline
	\hline
	\caption{Metriche di Gestione dei processi}
	\label{table:9}
\end{longtable}

\subsubsection{Gestione della Qualità}
La gestione della qualità è un processo che si occupa di definire una metodologia per garantire la qualità del prodotto.
Per valutare la qualità di tale processo, sono state definite le seguenti metriche:
\begin{longtable}{|>{\raggedright\arraybackslash}m{0.13\textwidth}|>{\raggedright\arraybackslash}m{0.17\textwidth}|>{\raggedright\arraybackslash}m{0.14\textwidth}|>{\raggedright\arraybackslash}m{0.14\textwidth}|>{\raggedright\arraybackslash}m{0.29\textwidth}|}
	\hline
	\textbf{Metrica} & \textbf{Nome}             & \textbf{Valore ammissibile} & \textbf{Valore ottimo} & \textbf{Descrizione}                                                          \\
	\hline
	\endfirsthead
	\hline
	\textbf{Metrica} & \textbf{Nome}             & \textbf{Valore ammissibile} & \textbf{Valore ottimo} & \textbf{Descrizione}                                                          \\
	\endhead
	\textbf{30M-QMS} & Quality Metrics Satisfied & $\geq 85\% $                & 100\%                  & Indica il numero di metriche implementate e soddisfatte, tra quelle definite. \\
	\hline
	\textbf{31M-TE}  & Time Efficiency           & $\leq 3 $                   & $\leq 1 $              & Livello di efficienza del team nello sviluppo di codice di alta qualità.      \\
	\hline
	\caption{Metriche di Gestione della Qualità}
	\label{table:10}
\end{longtable}

\subsection{Processi organizzativi}
I processi organizzativi sono processi che si occupano di definire le linee guida e le procedure da seguire per garantire
un'efficace gestione e coordinazione del progetto.

\subsubsection{Pianificazione}
La pianificazione è un processo che si occupa di definire le attività da svolgere e le risorse temporali e umane necessarie
per il loro svolgimento. Per valutare la qualità di tale processo, sono state definite le seguenti metriche:
\begin{longtable}{|>{\raggedright\arraybackslash}m{0.13\textwidth}|>{\raggedright\arraybackslash}m{0.17\textwidth}|>{\raggedright\arraybackslash}m{0.14\textwidth}|>{\raggedright\arraybackslash}m{0.14\textwidth}|>{\raggedright\arraybackslash}m{0.29\textwidth}|}
	\hline
	\textbf{Metrica} & \textbf{Nome}                      & \textbf{Valore ammissibile} & \textbf{Valore ottimo} & \textbf{Descrizione}                                                                                  \\
	\hline
	\endfirsthead
	\hline
	\textbf{32M-RSI} & Requirements Stability Index (RSI) & $\geq 75\% $                & 100\%                  & Misura utilizzata per quantificare il grado di cambiamento dei requisiti in un progetto. \\
	\hline
	\textbf{33M-SV}  & Schedule Variance (SV)             & $\geq -7.5\%$               & $\geq 0\%$             & Indica in percentuale il livello di anticipo (+) o ritardo (-) rispetto le attività pianificate.              \\
	\hline
	\textbf{34M-BV}  & Budget Variance (BV)               & $\geq -7.5\%$               & $\geq 0\%$             & Indica in percentuale il livello di eccedenze (+) o risparmi (-) rispetto al budget pianificato.                \\
	\hline
	\caption{Metriche di Pianificazione}
	\label{table:11}
\end{longtable}
