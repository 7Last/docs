\section{Introduzione}
\setcounter{subsection}{0}
\subsection{Scopo del documento}
Questo documento ha l’obiettivo di delineare la pianificazione e la gestione delle attività necessarie per la realizzazione del progetto.Vengono approfonditi aspetti chiave come l’\textit{Analisi dei Rischi}, il \textit{modello di sviluppo adottato}, la \textit{pianificazione delle attività}, la \textit{suddivisione dei ruoli}, nonché \textit{stime dei costi} e delle \textit{risorse necessarie}.

\subsection{Scopo del prodotto}
Lo scopo principale del prodotto é quello di permettere all’azienda \textit{Sync Lab S.r.l.} di poter valutare se é conveniente investire tempo e risorse per implementare il capitolato \textit{SyncCity - A smart city monitoring platform}. Una soluzione che, tramite l'uso di dispositivi IoT, permette di monitorare costantemente le città. SyncCity infatti servirà a monitorare e raccogliere dati da sensori posti in città, per poi analizzarli e fornire informazioni utili per la gestione della città stessa. Il prodotto finale sarà un prototipo funzionante che permetterà di visualizzare i dati raccolti in una dashboard.

\subsection{Glossario}
Al fine di evitare ambiguità o incomprensioni riguardanti i termini utilizzati nel documento, verrà adottato un \href{https://7last.github.io/docusaurus/docs/rtb/glossario#glossario-1}{glossario} in cui saranno presenti le varie definizioni. La presenza di un termine all'interno del glossario verrà indicata applicando questo particolare \textcolor{red}{\uline{\textit{stile}}}.
\subsection{Riferimenti}
    \subsubsection{Normativi}SONO COMPLETAMENTE BUTTATI A CASO
        \begin{itemize}
            \item \textbf{ISO/IEC 12207:2008} - Systems and software engineering - Software life cycle processes
            \item \textbf{ISO/IEC 31000:2009} - Risk management - Principles and guidelines
        \end{itemize}
    \subsubsection{Informativi}
        \begin{itemize}
            \item T2 - Processi di ciclo di vita del software\\ https://www.math.unipd.it/~tullio/IS-1/2023/Dispense/T2.pdf
            \item T4 - Gestione di progetto\\ https://www.math.unipd.it/~tullio/IS-1/2023/Dispense/T4.pdf
            \item \href{https://7last.github.io/docusaurus/docs/rtb/glossario#glossario-1}{glossario}
        \end{itemize}
\subsection{Preventivo iniziale}
Il preventivo iniziale presentato in fase di candidatura è reperibile al seguente \uline{\href{https://github.com/7Last/docs/blob/main/1_candidatura/preventivo_costi_assunzione_impegni_v2.0.pdf}{link}}. All'interno di tale documento viene calcolato il preventivo iniziale del progetto, che equivale a €12.670,00. In aggiunta viene specificato che il gruppo \textit{7Last} stima di terminare il prodotto entro e non oltre la data 24 Settembre 2024.
