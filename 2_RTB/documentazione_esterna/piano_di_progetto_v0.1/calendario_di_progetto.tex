\section{Calendario di progetto}
\subsection{Introduzione}
Il calendario di progetto illustra le date previste per le revisioni del progetto alla luce di quanto analizzato nelle sezioni:
\begin{itemize}
    \item Analisi dei rischi  
    \item Pianificazione
\end{itemize}

\subsection{Prima stesura 28/03/2024}
\textit{7Last} si pone come ovviettivo temporale delle revisioni il seguente calendario:
\begin{table}[!h]

    \begin{center}
        \begin{tabular}{ l l l l p{9cm} }
            \hline                                                                                                          \\[-2ex]
            Revisione & Data\\
            \\[-2ex] \hline \\[-1.5ex]                                                                                      \\
            Requirements and Technology Baseline & 09/04/2024 \\
            Product Baseline & 07/05/2024 \\
            Customer Acceptance & 24/09/2024 \\
            \\[-1.5ex] \hline
        \end{tabular}
    \end{center}
    \caption{Calendario di progetto}
    \label{tab:1}
\end{table}
\newpage
\subsection{Seconda stesura DATA DA DEFINIRE}
\textit{7Last} si pone come ovviettivo temporale delle revisioni il seguente calendario:

\begin{table}[!h]
    \begin{center}
        \begin{tabular}{ l l l l p{9cm} }
            \hline                                                                                                          \\[-2ex]
            Revisione & Data\\
            \\[-2ex] \hline \\[-1.5ex]                                                                                      \\
            Requirements and Technology Baseline & 09/04/2024 \\
            Product Baseline & 07/05/2024 \\
            Customer Acceptance & 24/09/2024 \\
            \\[-1.5ex] \hline
        \end{tabular}
    \end{center}
    \caption{Calendario di progetto}
    \label{tab:2}
\end{table}
