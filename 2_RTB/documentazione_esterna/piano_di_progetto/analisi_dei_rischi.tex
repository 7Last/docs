\section{Analisi dei rischi} % TODO: verificare riferimento allo standard adottato
È fondamentale mitigare l'impatto delle difficoltà incontrate durante lo svolgimento del progetto attraverso un'adeguata \textit{analisi dei rischi}. Questa sezione è stata inserita nel documento per evitare che potenziali problemi compromettano il successo del progetto.
Dopo aver elencato i rischi, viene identificata una serie di passi da compiere nel caso in cui uno di essi si verifichi. Secondo lo standard \textit{ISO/IEC 31000:2009} il processo di gestione del rischio consiste in 5 fasi di seguito elencate.
\begin{itemize}
    \item \textbf{Identificazione del rischio} - Consiste nel riconoscere le possibili cause del rischio, le aree di impatto, gli eventi, le cause e le potenziali conseguenze. Questa fase comporta un'analisi delle attività per creare un elenco di rischi basato sugli eventi che potrebbero influenzare il raggiungimento degli obiettivi.
    \item \textbf{Analisi del rischio} - Questa fase prevede un processo di valutazione che contribuisce alla valutazione e al processo decisionale sul trattamento del rischio, identificando le strategie più adatte.
    \item \textbf{Valutazione del rischio} - L'obiettivo di questa fase è prendere decisioni basate sui risultati dell'analisi del rischio per attuare la migliore strategia di trattamento.
    \item \textbf{Trattamento del rischio} - Dopo l'analisi e la valutazione dei rischi, è fondamentale decidere come trattarli per ridurne l'impatto.
    \item \textbf{Monitoraggio e revisione del rischio} - Queste attività devono essere integrate nella pianificazione del processo di gestione dei rischi e richiedono un monitoraggio regolare.
\end{itemize}
I fattori chiave per l'identificazione dei rischi sono:
\begin{itemize}
    \item la \textbf{tipologia} che rappresenta la categoria di rischio, la quale può essere organizzativa, tecnologica o comunicativa;
    \item l'\textbf{indice}, un valore numerico incrementale che identifica univocamente il rischio per ogni Tipologia: un rischio elevato equivale a 3, un rischio medio equivale a 2, mentre un rischio basso equivale a 1.
\end{itemize}
Per una rappresentazione schematica dei rischi, si è deciso di attuare la seguente convenzione: R [ Tipologia ] [ Indice ].

\clearpage

%---------SEZIONE DEI RISCHI ORGANIZZATIVI-----------%

\subsection{Rischi organizzativi}
\begin{longtable}{ | l | p{10cm} | }
    \hline
    \textbf{Descrizione} & Durante il periodo iniziale, la pianificazione delle attività può non essere ottimale a causa della mancanza di esperienza del team, della scarsa conoscenza dei requisiti e della sovrastima/sottostima delle risorse/tempi necessari. \\
    \hline
    \textbf{Probabilità} & Alta. \\
    \hline
    \textbf{Pericolosità} & Alta. \\
    \hline
    \textbf{Rilevamento} & Monitorazione continua di GitHub e con il \href{https://7last.github.io/docs/rtb/documentazione-interna/glossario#piano-di-progetto}{\textit{Piano di progetto}\textsubscript{G}} \\
    \hline
    \textbf{Piano di contingenza} & In caso di difficoltà o ritardi, il \href{https://7last.github.io/docs/rtb/documentazione-interna/glossario#piano-di-progetto}{\textit{piano di progetto}\textsubscript{G}} viene rivisto per allineare le attività ai progressi. Se un membro segnala difficoltà nel rispettare una scadenza, al responsabile il compito di assegnare più risorse o, in casi più gravi, spostare la scadenza.\\
    \hline
    \caption{RO1 - Inesperienza dei membri del team nella pianificazione delle attivitià}
    \label{table:1}
\end{longtable}

\newpage
\begin{longtable}{ | l | p{10cm} | }
    \hline
    \textbf{Descrizione} & Gli impegni personali e/o universitari possono limitare la disponibilità di uno o più membri del gruppo. \\
    \hline
    \textbf{Probabilità} & Media. \\
    \hline
    \textbf{Pericolosità} & Bassa. \\
    \hline
    \textbf{Rilevamento} & Condividendo i propri impegni e indicando la disponibilità, i membri possono concordare momenti della settimana per tenere le riunioni e comprendere lo stato di sviluppo del progetto da parte di ciascun membro. \\
    \hline
    \textbf{Piano di contingenza} & Il compito del responsabile è quello di rivedere la suddivisione dei ruoli e compiti in base agli impegni di ciascun membro. In casi gravi, le scadenze devono essere spostate e la pianificazione deve essere rivista se non tiene conto di questi inconvenienti.\\
    \hline
    \caption{RO2 - Impegni personali o universitari}
    \label{table:2}
\end{longtable}

\begin{longtable}{ | l | p{10cm} | }
    \hline
    \textbf{Descrizione} & La sottostima/sovrastima dei costi orari delle attività, dovuta alla mancanza di esperienza del team, può causare ritardi, perdite di tempo e di risorse. \\
    \hline
    \textbf{Probabilità} & Bassa. \\
    \hline
    \textbf{Pericolosità} & Alta. \\
    \hline
    \textbf{Rilevamento} & Attraverso il cruscotto e confronto periodico con il \href{https://7last.github.io/docs/rtb/documentazione-interna/glossario#piano-di-progetto}{Piano di Progetto\textsubscript{G}}, il Responsabile può monitorare lo stato di avanzamento del progetto \\
    \hline
    \textbf{Piano di contingenza} & In caso di modifiche non gravi, cerchiamo di implementare rapidamente ciò che viene lasciato in sospeso. Se sono significative, discutiamo con il proponente per trovare un accordo su come gestire le modifiche e affrontare i cambiamenti.\\
    \hline
    \caption{RO3 - Ritardi rispetto alle tempistiche previste}
    \label{table:3}
\end{longtable}

\begin{longtable}{ | l | p{10cm} | }
    \hline
    \textbf{Descrizione} & La possibilità che uno o più membri del gruppo non collaborino attivamente allo sviluppo del progetto. \\
    \hline
    \textbf{Probabilità} & Media. \\
    \hline
    \textbf{Pericolosità} & Alta. \\
    \hline
    \textbf{Rilevamento} & Contando le volte che un membro è assente, dopo la quinta volta viene attivato un rapporto interno al team.\\
    \hline
    \textbf{Piano di contingenza} & È compito dell'amministratore comunicare la situazione alla persona interessata e invitarla a partecipare attivamente allo sviluppo. In caso di esito negativo, il compito del manager è quello di assegnare maggiori risorse o, nei casi più gravi, di posticipare la scadenza.\\
    \hline
    \caption{RO4 - Scarsa collaborazione da parte di uno o più membri}
    \label{table:4}
\end{longtable}

%--------------------------------------------------%

\newpage

%---------SEZIONE DEI RISCHI TECNOLOGICI-----------%

\subsection{Rischi tecnologici}
\begin{longtable}{ | l | p{10cm} | }
    \hline
    \textbf{Descrizione} & Dato il livello di esperienza che il capitolato richiede,  alcuni membri del gruppo potrebbero dover acquisire le competenze necessarie. Questo potrebbe causare ritardi sia nella fase di progettazione che in quella di sviluppo.  \\
    \hline
    \textbf{Probabilità} & Alta. \\
    \hline
    \textbf{Pericolosità} & Alta. \\
    \hline
    \textbf{Rilevamento} & Dopo aver compreso le competenze di ciascun membro del team, il responsabile deve assegnare i compiti in modo che non siano troppo facili, ma nemmeno troppo difficili per ciascun membro. \\
    \hline
    \textbf{Piano di contingenza} & Se i membri del gruppo incontrano difficoltà nello svolgimento di un'attività, saranno assistiti da un membro con maggiore esperienza in quell'ambito. \\
    \hline
    \caption{RT1 - Inesperienza nell'uso delle tecnologie adottate}
    \label{table:5}
\end{longtable}

\begin{longtable}{ | l | p{10cm} | }
    \hline
    \textbf{Descrizione} & La perdita di informazioni rappresenta un rischio di impatto importante per il progetto. Può verificarsi in caso di guasti hardware, errori umani o malfunzionamenti dei sistemi utilizzati. \\
    \hline
    \textbf{Probabilità} & Media. \\
    \hline
    \textbf{Pericolosità} & Alta. \\
    \hline
    \textbf{Rilevamento} & Attraverso la monitorazione continua dei sistemi utilizzati. \\
    \hline
    \textbf{Piano di contingenza} & In caso perdita di informazioni, è necessario poter reperire quelle di riserva, tramite un backup. \\
    \hline
    \caption{RT2 - Perdita di informazioni}
    \label{table:6}
\end{longtable}

\begin{longtable}{ | l | p{10cm} | }
    \hline
    \textbf{Descrizione} & Per lo sviluppo del progetto è necessario utilizzare diverse tecnologie. I malfunzionamenti di queste tecnologie non dipendono dal gruppo e la loro risoluzione può richiedere tempo e risorse, incidendo così sulla velocità e sui costi del progetto. \\
    \hline
    \textbf{Probabilità} & Alta. \\
    \hline
    \textbf{Pericolosità} & Alta. \\
    \hline
    \textbf{Rilevamento} & Solo al momento dell'utilizzo di queste tecnologie il team potrà scoprire se si verificheranno malfunzionamenti o no.  \\
    \hline
    \textbf{Piano di contingenza} & In caso di malfunzionamenti, è responsabilità del project manager allocare le risorse necessarie per la loro risoluzione nel più breve tempo possibile.\\
    \hline
    \caption{RT3 - Problemi di compatibilità tra le tecnologie utilizzate}
    \label{table:7}
\end{longtable}

%--------------------------------------------------%

\newpage

%---------SEZIONE DEI RISCHI COMUNICATIVI-----------%

\subsection{Rischi comunicativi}
\begin{longtable}{ | l | p{10cm} | }
    \hline
    \textbf{Descrizione} & Le differenze all'interno del gruppo possono derivare da ideologie e opinioni diverse tra i suoi membri. \\
    \hline
    \textbf{Probabilità} & Media. \\
    \hline
    \textbf{Pericolosità} & Alta. \\
    \hline
    \textbf{Rilevamento} & Possono essere identificate attraverso le loro opinioni espresse o osservando le dinamiche del gruppo. \\
    \hline
    \textbf{Piano di contingenza} & In caso di disaccordo, si procederà a una votazione democratica e si attuerà l'opzione con il maggior numero di voti. \\
    \hline
    \caption{RC1 - Disaccordi all'interno del gruppo}
    \label{table:8}
\end{longtable}

\begin{longtable}{ | l | p{10cm} | }
    \hline
    \textbf{Descrizione} & Una comunicazione inefficace può causare ritardi, stress, e disaccordo interno al gruppo. \\
    \hline
    \textbf{Probabilità} & Media. \\
    \hline
    \textbf{Pericolosità} & Alta. \\
    \hline
    \textbf{Rilevamento} & Questo può essere identificato attraverso sondaggi, feedback e comportamenti da parte dei membri del gruppo durante le riunioni o comunicazioni via messaggio. \\
    \hline
    \textbf{Piano di contingenza} & Il responsabile ha il compito di promuovere una comunicazione attiva, organizzare riunioni regolari, indagare sulle cause del disaccordo e ricercare soluzioni. \\
    \hline
    \caption{RC2 - Problemi di comunicazione}
    \label{table:9}
\end{longtable}