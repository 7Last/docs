\section{Calendario di progetto}
\subsection{Introduzione}
Il calendario di progetto presenta le date previste per le revisioni del capitolato alla luce di quanto analizzato nelle sezioni:
\begin{itemize}
    \item Analisi dei rischi.
    \item Pianificazione.
\end{itemize}

\subsection{Prima stesura 2024-03-28}
\textit{7Last} si pone come obiettivo temporale delle revisioni il seguente calendario:
\begin{table}[!h]
    \begin{center}
        \begin{tabular}{ | l | l | l | l | l | }
            \hline
            Revisione                               & Data       \\ \hline
            Requirements and Technology Baseline    & 2024-05-09 \\
            Product Baseline                        & 2024-08-09 \\
            Customer Acceptance                     & 2024-09-24 \\
            \hline
        \end{tabular}
    \end{center}
    \caption{Calendario di progetto}
    \label{tab:10}
\end{table}
\newpage
% ------------------- TODO: da dicedere e completare --------------
% \subsection{Seconda stesura DATA DA DEFINIRE}
% \textit{7Last} si pone come ovviettivo temporale delle revisioni il seguente calendario:
%
% \begin{table}[!h]
%     \begin{center}
%         \begin{tabular}{ | l | l | l | l | l | }
%             \hline
%             Revisione                               & Data       \\ \hline
%             Requirements and Technology Baseline    & 2024-04-09 \\
%             Product Baseline                        & 2024-05-07 \\
%             Customer Acceptance                     & 2024-09-24 \\
%             \hline
%         \end{tabular}
%     \end{center}
%     \caption{Calendario di progetto}
%     \label{tab:11}
% \end{table}
