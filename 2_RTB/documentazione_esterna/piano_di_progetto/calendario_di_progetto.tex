\section{Calendario di progetto}
\subsection{Introduzione}
Il calendario di progetto contiene le date previste per le revisioni del capitolato, tenendo conto delle osservazioni fatte nelle sezioni:
\begin{itemize}
    \item \textbf{Analisi dei rischi};
    \item \textbf{Pianificazione}.
\end{itemize}

\subsection{Prima stesura 2024-03-28}
\textit{7Last} si pone come obiettivo temporale delle revisioni il seguente calendario:
\begin{table}[!h]
    \begin{center}
        \begin{tabular}{ | l | l | l | l | l | }
            \hline
            Revisione                               & Data       \\ \hline
            Requirements and Technology Baseline    & 2024-05-09 \\
            Product Baseline                        & 2024-08-09 \\
            Customer Acceptance                     & 2024-09-24 \\
            \hline
        \end{tabular}
    \end{center}
    \caption{Calendario di progetto}
\end{table}

\subsection{Seconda stesura 2024-05-15}
Dopo aver analizzato l'andamento del progetto nei primi sprint, \textit{7Last} ha appurato che gli obiettivi temporali definiti in sede di candidatura erano mal calibrati. Di conseguenza, il calendario di progetto è stato pianificato nuovamente come segue:
\begin{table}[!h]
    \begin{center}
        \begin{tabular}{ | l | l | l | l | l | }
            \hline
            Revisione                               & Data       \\ \hline
            Requirements and Technology Baseline    & 2024-05-28 \\
            Product Baseline                        & 2024-07-31 \\
            Customer Acceptance                     & 2024-08-28 \\
            \hline
        \end{tabular}
    \end{center}
    \caption{Calendario di progetto}
\end{table}