\section{Introduzione}
\setcounter{subsection}{0}
\subsection{Scopo del documento}
Il documento in questione ha lo scopo di fornire una guida dettagliata e strutturata su come il progetto verrò eseguito e gestito. In particolare, verrano trattati i seguenti argomenti:
\begin{itemize}
	\item \textbf{suddivisione dei ruoli};
	\item \textbf{stime dei costi};
	\item \textbf{risorse necessarie};
	\item \textbf{modello di sviluppo adottato};
	\item \textbf{analisi dei rischi}.
\end{itemize}

\subsection{Scopo del prodotto}
Lo scopo principale del prodotto è quello di consentire a \textit{Sync Lab S.r.l.} di valutare la \textbf{fattibilità} prima di investire tempo e risorse nell'implementazione del progetto  \textit{\textbf{SyncCity} - A smart city monitoring platform}. Questa soluzione consente un monitoraggio costante delle città, attraverso l'utilizzo di dispositivi IoT. SyncCity sarà in grado di raccogliere dati da appositi sensori, per poi analizzarli e fornire informazioni utili alla gestione della città. Il prodotto finale sarà un prototipo funzionale che consentirà la visualizzazione dei dati raccolti su delle {dashboard\textsubscript{G}}.

\subsection{Glossario}
Per evitare qualsiasi ambiguità o malinteso sui termini utilizzati nel documento, verrà adottato un \href{https://7last.github.io/docs/rtb/documentazione-interna/glossario#glossario}{Glossario\textsubscript{G}}. Questo \href{https://7last.github.io/docs/rtb/documentazione-interna/glossario#glossario}{Glossario\textsubscript{G}} conterrà varie definizioni. Ogni termine incluso nel \href{https://7last.github.io/docs/rtb/documentazione-interna/glossario#glossario}{glossario\textsubscript{G}} sarà indicato applicando uno stile specifico:
\begin{itemize}
	\item aggiungendo una "G" al pedice della parola;
	\item fornendo il link al \href{https://7last.github.io/docs/rtb/documentazione-interna/glossario\#glossario}{glossario\textsubscript{G}} online;
\end{itemize}

\subsection{Riferimenti}
\subsubsection{Normativi} % TODO: verificare versioni standard con altri documenti ed aggiungere link
\begin{itemize}
	\item \textbf{ISO/IEC 12207:2008}: Systems and software engineering - Software life cycle processes
	\item \textbf{ISO/IEC 31000:2009}: Risk management - Principles and guidelines
\end{itemize}
\subsubsection{Informativi}
\begin{itemize}
	\item \textbf{Processi di ciclo di vita del software} \\
	      \url{https://www.math.unipd.it/~tullio/IS-1/2023/Dispense/T2.pdf}
	\item \textbf{Gestione di progetto} \\
	      \url{https://www.math.unipd.it/~tullio/IS-1/2023/Dispense/T4.pdf}
	\item \textbf{Capitolato d'appalto C6}: SyncCity – A smart city monitoring platform\\
	      \url{https://www.math.unipd.it/~tullio/IS-1/2023/Progetto/C6.pdf}
	\item \textbf{Glossario} \\
	      \url{https://7last.github.io/docs/rtb/documentazione-interna/glossario}
\end{itemize}

\subsection{Preventivo iniziale}
Il preventivo presentato durante la fase di candidatura è disponibile al seguente \uline{\href{https://7last.github.io/docs/candidatura/preventivo-costi-assunzione-impegni}{riferimento}}.
All'interno di questo documento viene calcolato il preventivo iniziale del progetto, pari a €12.680,00.
Inoltre, si specifica che il gruppo \textit{7Last} stima di \textbf{completare} il prodotto entro e non oltre il \textbf{24 Settembre 2024}.
