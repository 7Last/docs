\section{Introduzione}
\setcounter{subsection}{0}
\subsection{Scopo del documento}
Il seguente documento si propone di definire la pianificazione e la gestione delle attività richieste per ultimare il progetto. Vengono esaminati in dettaglio elementi cruciali come l’\textit{Analisi dei Rischi}, il \textit{modello di sviluppo adottato}, la \textit{pianificazione delle attività}, la \textit{suddivisione dei ruoli}, oltre a \textit{stime dei costi} e delle \textit{risorse necessarie}.

\subsection{Scopo del prodotto}
Lo scopo principale del prodotto è quello di consentire a \textit{Sync Lab S.r.l.} di valutare la \textbf{fattibilità} di investire tempo e risorse nell'implementazione del progetto  \textit{\textbf{SyncCity} - A smart city monitoring platform}. Questa soluzione, attraverso l'utilizzo di dispositivi IoT, consente un monitoraggio costante delle città. SyncCity avrà lo scopo di monitorare e raccogliere dati da sensori posizionati nelle città, per poi analizzarli e fornire informazioni utili alla gestione della città. Il prodotto finale sarà un prototipo funzionale che consentirà la visualizzazione dei dati raccolti su un cruscotto.

\subsection{Glossario}
Per evitare qualsiasi ambiguità o malinteso sui termini utilizzati nel documento, verrà adottato un \href{https://7last.github.io/docs/rtb/documentazione-interna/glossario#glossario}{Glossario\textsubscript{G}}. Questo \href{https://7last.github.io/docs/rtb/documentazione-interna/glossario#glossario}{Glossario\textsubscript{G}} conterrà varie definizioni. Ogni termine incluso nel \href{https://7last.github.io/docs/rtb/documentazione-interna/glossario#glossario}{glossario\textsubscript{G}} sarà indicato applicando uno stile specifico:
\begin{itemize}
    \item aggiungendo una "G" al pedice della parola;
    \item fornendo il link al \href{https://7last.github.io/docs/rtb/documentazione-interna/glossario#glossario}{glossario\textsubscript{G}} online;
\end{itemize}

\subsection{Riferimenti}
    \subsubsection{Normativi} % TODO: verificare versioni standard con altri documenti ed aggiungere link
        \begin{itemize}
            \item \textbf{ISO/IEC 12207:2008}: Systems and software engineering - Software life cycle processes
            \item \textbf{ISO/IEC 31000:2009}: Risk management - Principles and guidelines
        \end{itemize}
    \subsubsection{Informativi}
        \begin{itemize}
            \item \textbf{Processi di ciclo di vita del software} \\
                \url{https://www.math.unipd.it/~tullio/IS-1/2023/Dispense/T2.pdf}
            \item \textbf{Gestione di progetto} \\ 
                \url{https://www.math.unipd.it/~tullio/IS-1/2023/Dispense/T4.pdf}
            \item \textbf{Capitolato d'appalto C6}: SyncCity – A smart city monitoring platform\\
                \url{https://www.math.unipd.it/~tullio/IS-1/2023/Progetto/C6.pdf}
            \item \textbf{Glossario} \\
                \url{https://7last.github.io/docs/rtb/documentazione-interna/glossario}
            \end{itemize}

\subsection{Preventivo iniziale}
Il preventivo iniziale presentato durante la fase di candidatura è disponibile al seguente \uline{\href{https://github.com/7Last/docs/blob/main/1_candidatura/preventivo_costi_assunzione_impegni_v2.0.pdf}{riferimento}}. All'interno di questo documento viene calcolato il preventivo iniziale del progetto, pari a €12.670,00. Inoltre, si specifica che il gruppo \textit{7Last} stima di \textbf{completare} il prodotto entro e non oltre il \textbf{24 Settembre 2024}.