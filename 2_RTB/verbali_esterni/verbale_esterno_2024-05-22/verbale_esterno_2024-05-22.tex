\documentclass[italian,12pt]{article}

%--------------variabili------------------%
\def\Title{Norme di Progetto}
\def\Author{7Last}
\def\Version{v0.2}
%-----------------------------------------%


\usepackage[left=2cm, right=2cm, bottom=3cm, top=3cm]{geometry}
\usepackage{fancyhdr}
\usepackage{graphicx}
\graphicspath{ {../../logo/} }
\usepackage{href-ul}
\usepackage{tikz}
\usepackage{tgadventor}
\usepackage[useregional=numeric,showseconds=true,showzone=false]{datetime2}
\usepackage{caption}
\usepackage{longtable}
\usepackage{xcolor}




\linespread{1.2}
\captionsetup[table]{labelformat=empty}
\geometry{headsep=1.5cm}

\renewcommand{\contentsname}{Indice}
\renewcommand\familydefault{\sfdefault}

\begin{document}

\newgeometry{left=2cm,right=2cm,bottom=2.1cm,top=2.1cm}
\begin{titlepage}
	\vspace*{.5cm}

	\vspace{2cm}
	{
		\centering
		{\bfseries\huge \Title\par}
		\bigbreak
		{\bfseries\Large \Subtitle\par}
		\bigbreak
		{\bfseries\large \Author\par}
		\bigbreak
		{\Date\;-\;\Version\par}
		\vfill

		\begin{center}
			\begin{tikzpicture}
				\clip (0,0) circle (2cm) node {\includegraphics[width=4cm]{logo.jpg}};
			\end{tikzpicture}
		\end{center}
	}

	\vfill

\end{titlepage}

\restoregeometry






















\newpage

\pagestyle{fancy}
\fancyhead{}
\lhead{
	\begin{tikzpicture}
		\clip (0,0) circle (0.5cm);
		\node at (0,0) {\includegraphics[width=1cm]{./../logo/logo.png}};
	\end{tikzpicture}%
}
\chead{\vspace{\fill}\Title\vspace{\fill}}
\rhead{\vspace{\fill}\Version\vspace{\fill}}


\begin{table}[!h]
	\caption{Versioni}
	\footnotesize
	\begin{center}
		\begin{tabular}{ l l l l p{6cm} }
			\hline                                                                              \\[-2ex]
			Ver. & Data       & Redattore          & Verificatore       & Descrizione           \\
			\\[-2ex] \hline \\[-1.5ex]
			1.0  & 2024-03-16 & Raul Seganfredo    & Davide Malgarise & Stesura verbale \\
			\\[-1.5ex] \hline
		\end{tabular}
	\end{center}
\end{table}

\newpage

\tableofcontents

\newpage

\section{Dettagli della riunione}


\textbf{Sede della riunione}: Google Meet\\
\textbf{Orario di inizio}: 15:30\\
\textbf{Orario di fine}: 16:30\\

\begin{flushleft}
	\begin{table}[!h]
	\begin{tabular}{ |l|l|l| } 
		\hline
		\textbf{Partecipante} & \textbf{Ruolo}       & \textbf{Presenza} \\
		\hline 
		Antonio Benetazzo     &                      & Presente          \\
		Davide Malgarise      & Redattore            & Presente          \\
		Elena Ferro           & Verificatore         & Presente          \\
		Leonardo Baldo        &                      & Presente          \\
		Matteo Tiozzo         &                      & Presente          \\
		Raul Seganfreddo      & Redattore			 & Presente          \\
		Valerio Occhinegro    &                      & Presente          \\
		\hline
	\end{tabular}
	\end{table}
	\textbf{Partecipanti esterni}: Andrea Dorigo, Daniele Zorzi, Fabio Pallaro.\\
\end{flushleft}

\section{Ordine del giorno}
\begin{itemize}
	\item Stato di sviluppo del prodotto
	\item Discussione su eventuali modifiche
	\item Discussione su eventuale candidatura per revisione RTB
\end{itemize}

\newpage

\section{Verbale}

\subsection{Stato di sviluppo del prodotto}
Come prima cosa, il responsabile di questo sprint, Valerio Occhinegro, ha presentato lo stato di sviluppo del prodotto,
mostrando le varie modifiche fatte. Come richiesto nel precedente incontro, sono stati mostrati i vari miglioramenti
ai vari grafici delle relative dashboard, i nuovi sensori aggiunti e i nuovi grafici creati.\\
Sono inoltre stati mostrati alcune modifiche fatte ai sensori delle isole ecologiche, come la modifica della
simulazione dei dati che permette alle isole ecologiche di rimanere piene per un certo periodo di tempo e di non essere
svuotate completamente. Inoltre è stato illustrato il sistema di allerting implementato per il sensore della
temperatura, che manda delle segnalazioni in caso di temperature troppo alte.\\

\subsection{Discussione su eventuali modifiche}
Successivamente alla presentazione, la proponente ha consigliato alcuni miglioramenti da fare ai grafici, come ad
esempio migliorare la leggibilità dei dati in alcuni grafici e cambiare la visualizzazione della mappa a satellitare.\\

\subsection{Discussione su eventuale candidatura per revisione RTB}
Infine, è stato discusso con la proponente la possibilità di candidarsi per la revisione \textbf{RTB}, discutendo lo
stato di avanzamento del prodotto e le modifiche da fare per poter essere pronti per la revisione. La proponente ha
dichiarato che il prodotto è pronto per la revisione al netto di alcune modifiche minori consigliate in precedenza.\\

\subsection{Decisioni prese e conclusioni}
In comune accordo con la proponente, si è deciso di rimandare la scelta della data per il prossimo incontro, in quanto
sarebbe meglio farlo dopo la revisione \textbf{RTB}.\\


\newpage
\begin{table}[b]
	\begin{tabular}{@{}p{5cm}p{10cm}@{}}
		Data:  & \hrulefill \\
		       &            \\
		       &            \\
		Firma della proponente: & \hrulefill \\
	\end{tabular}
\end{table}

\end{document}
