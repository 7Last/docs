\documentclass[italian,12pt]{article}

%--------------variabili------------------%
\def\Title{Norme di Progetto}
\def\Author{7Last}
\def\Version{v0.2}
%-----------------------------------------%


\usepackage[left=2cm, right=2cm, bottom=3cm, top=3cm]{geometry}
\usepackage{fancyhdr}
\usepackage{graphicx}
\graphicspath{ {../../logo/} }
\usepackage{href-ul}
\usepackage{tikz}
\usepackage{tgadventor}
\usepackage[useregional=numeric,showseconds=true,showzone=false]{datetime2}
\usepackage{caption}
\usepackage{longtable}
\usepackage{xcolor}




\linespread{1.2}
\captionsetup[table]{labelformat=empty}
\geometry{headsep=1.5cm}

\renewcommand{\contentsname}{Indice}
\renewcommand\familydefault{\sfdefault}

\begin{document}

\newgeometry{left=2cm,right=2cm,bottom=2.1cm,top=2.1cm}
\begin{titlepage}
	\vspace*{.5cm}

	\vspace{2cm}
	{
		\centering
		{\bfseries\huge \Title\par}
		\bigbreak
		{\bfseries\Large \Subtitle\par}
		\bigbreak
		{\bfseries\large \Author\par}
		\bigbreak
		{\Date\;-\;\Version\par}
		\vfill

		\begin{center}
			\begin{tikzpicture}
				\clip (0,0) circle (2cm) node {\includegraphics[width=4cm]{logo.jpg}};
			\end{tikzpicture}
		\end{center}
	}

	\vfill

\end{titlepage}

\restoregeometry






















\newpage

\pagestyle{fancy}
\fancyhead{}
\lhead{
	\begin{tikzpicture}
		\clip (0,0) circle (0.5cm);
		\node at (0,0) {\includegraphics[width=1cm]{./../logo/logo.png}};
	\end{tikzpicture}%
}
\chead{\vspace{\fill}\Title\vspace{\fill}}
\rhead{\vspace{\fill}\Version\vspace{\fill}}


\begin{table}[!h]
	\caption{Versioni}
	\footnotesize
	\begin{center}
		\begin{tabular}{ l l l l l }
			\hline \\[-2ex]
			Ver. & Data       & Redattore          & Verificatore       & Descrizione  \\
			\\[-2ex] \hline \\[-1.5ex]
			1.0  & 23/04/2024 & Leonardo Baldo  & Antonio Benetazzo & Stesura verbale \\
			\\[-1.5ex] \hline
		\end{tabular}
	\end{center}
\end{table}

\newpage

\tableofcontents

\newpage

\section{Dettagli della riunione}


\textbf{Sede della riunione}: Google Meet\\
\textbf{Orario di inizio}: 15:30\\
\textbf{Orario di fine}: 16:30\\


\begin{flushleft}
	\begin{table}[!h]
	\begin{tabular}{ |l|l|l| } 
		\hline
		\textbf{Partecipante} & \textbf{Ruolo}       & \textbf{Presenza} \\
		\hline 
		Antonio Benetazzo     & Verificatore         & Presente          \\
		Davide Malgarise      & Amministratore       & Presente          \\
		Elena Ferro           &                      & Presente          \\
		Leonardo Baldo        & Redattore	         & Presente          \\
		Matteo Tiozzo         &                      & Assente           \\
		Raul Seganfreddo      &                      & Presente          \\
		Valerio Occhinegro    &                      & Presente          \\
		\hline
	\end{tabular}
	\end{table}
	\textbf{Partecipanti esterni}: Daniele Zorzi.
\end{flushleft}
\section{Ordine del giorno}
\begin{itemize}
	\item Lavoro svolto
	\item Apache Kafka vs Redpanda
	\item Obiettivi secondo sprint
	\item Consigli su suddivisione e turnazione ruoli
	\item Decisioni prese e conclusioni
\end{itemize}
\newpage
\section{Verbale}
\subsection{Lavoro svolto}
L'incontro inizia con una spiegazione del codice sviluppato fino al presente, seguito da una visualizzazione dei risultati ottenuti.
L'azienda considera il lavoro svolto finora ben fatto. In particolare, è stata apprezzata l'integrazione con Kafka e lo sviluppo del simulatore per un sensore aggiuntivo, entrambi ritenuti opzionali per questo sprint.
\subsection{Apache Kafka vs Redpanda}
Durante l'integrazione del codice simulatore di sensori con Kafka, è emersa un'ulteriore alternativa, non ancora valutata, rispetto \textit{Apache Kafka}: \textbf{\textit{Redpanda}}. Dopo un'attenta comparazione e analisi dei due strumenti, il team ha deciso di proporre all'azienda l'utilizzo di quest'ultimo. La prononente richiede di produrre un documento che illustri i vantaggi e gli svantaggi di entrambi gli strumenti, per poter valutare le due opzioni. Le conclusioni verranno discusse durante il prossimo SAL. 
\subsection{Obiettivi secondo sprint}
Per il secondo sprint sono stati definiti i seguenti obiettivi:
\begin{itemize}
	\item continuazione programma \textit{Python} per la generazione dei dati di test;
	\item aggiungere alcune tabelle di aggregazione dati su \textit{ClickHouse};
	\item visualizzazione su \textit{Grafana} di almeno un sensore;
	\item continuazione della documentazione;
	\item stesura \textit{Analisi dei Requisiti}.
\end{itemize}
\newpage
\subsection{Decisioni prese e conclusioni}
Discutiamo della durata degli sprint, concordando con l'azienda di mantenere una durata di 2 settimane, riservandoci la possibilità di modificarla in futuro.
La riunione si conclude dandoci appuntamento per il prossimo SAL, fissato in data 2024-05-06.
\begin{table}[b]
	\begin{tabular}{@{}p{.5in}p{4in}@{}}
		Data:  & \hrulefill \\
			   &     		\\
			   &     		\\
		Firma: & \hrulefill \\
	\end{tabular}
	\end{table}

\end{document}
