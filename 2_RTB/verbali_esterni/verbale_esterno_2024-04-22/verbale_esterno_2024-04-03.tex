\documentclass[italian,12pt]{article}

%--------------variabili------------------%
\def\Title{Norme di Progetto}
\def\Author{7Last}
\def\Version{v0.2}
%-----------------------------------------%


\usepackage[left=2cm, right=2cm, bottom=3cm, top=3cm]{geometry}
\usepackage{fancyhdr}
\usepackage{graphicx}
\graphicspath{ {../../logo/} }
\usepackage{href-ul}
\usepackage{tikz}
\usepackage{tgadventor}
\usepackage[useregional=numeric,showseconds=true,showzone=false]{datetime2}
\usepackage{caption}
\usepackage{longtable}
\usepackage{xcolor}




\linespread{1.2}
\captionsetup[table]{labelformat=empty}
\geometry{headsep=1.5cm}

\renewcommand{\contentsname}{Indice}
\renewcommand\familydefault{\sfdefault}

\begin{document}

\newgeometry{left=2cm,right=2cm,bottom=2.1cm,top=2.1cm}
\begin{titlepage}
	\vspace*{.5cm}

	\vspace{2cm}
	{
		\centering
		{\bfseries\huge \Title\par}
		\bigbreak
		{\bfseries\Large \Subtitle\par}
		\bigbreak
		{\bfseries\large \Author\par}
		\bigbreak
		{\Date\;-\;\Version\par}
		\vfill

		\begin{center}
			\begin{tikzpicture}
				\clip (0,0) circle (2cm) node {\includegraphics[width=4cm]{logo.jpg}};
			\end{tikzpicture}
		\end{center}
	}

	\vfill

\end{titlepage}

\restoregeometry






















\newpage

\pagestyle{fancy}
\fancyhead{}
\lhead{
	\begin{tikzpicture}
		\clip (0,0) circle (0.5cm);
		\node at (0,0) {\includegraphics[width=1cm]{./../logo/logo.png}};
	\end{tikzpicture}%
}
\chead{\vspace{\fill}\Title\vspace{\fill}}
\rhead{\vspace{\fill}\Version\vspace{\fill}}


\begin{table}[!h]
	\caption{Versioni}
	\footnotesize
	\begin{center}
		\begin{tabular}{ l l l l l }
			\hline \\[-2ex]
			Ver. & Data       & Redattore          & Verificatore       & Descrizione  \\
			\\[-2ex] \hline \\[-1.5ex]
			1.0  & 23/04/2024 & Leonardo Baldo  & Antonio Benetazzo & Stesura verbale \\
			\\[-1.5ex] \hline
		\end{tabular}
	\end{center}
\end{table}

\newpage

\tableofcontents

\newpage

\section{Dettagli della riunione}


\textbf{Sede della riunione}: Google Meet\\
\textbf{Orario di inizio}: 15:30\\
\textbf{Orario di fine}: 16:30\\


\begin{flushleft}
	\begin{table}[!h]
	\begin{tabular}{ |l|l|l| } 
		\hline
		\textbf{Partecipante} & \textbf{Ruolo}       & \textbf{Presenza} \\
		\hline 
		Antonio Benetazzo     & Verificatore         & Presente          \\
		Davide Malgarise      & Redattore            & Presente          \\
		Elena Ferro           &                      & Presente          \\
		Leonardo Baldo        & Amministratore       & Presente          \\
		Matteo Tiozzo         &                      & Assente           \\
		Raul Seganfreddo      &                      & Presente          \\
		Valerio Occhinegro    &                      & Presente          \\
		\hline
	\end{tabular}
	\end{table}
	\textbf{Partecipanti esterni}: Daniele Zorzi.
\end{flushleft}

\section{Ordine del giorno}
\begin{itemize}
	\item Lavoro svolto
	\item Apache Kafka vs RedPanda
	\item Obiettivi secondo sprint
	\item Consigli su suddivisione e turnazione ruoli
	\item Decisioni prese e conclusioni
\end{itemize}

\newpage

\section{Verbale}

\subsection{Lavoro svolto}
L'incontro è iniziato con una spiegazione dettagliata del codice sviluppato fino a quel momento, seguito da una visualizzazione dei risultati ottenuti, presentati con alcuni esempi.
L'azienda considera il lavoro svolto finora ben fatto.

\subsection{Apache Kafka vs RedPanda}
Durante il primo sprint, abbiamo confrontato 2 strumenti da utilizzare nel progetto: \textit{Apache Kafka} (consigliato dall'azienda) e \textit{RedPanda} (alternativa da considerare).
Abbiamo posto il dubbio sulla scelta, spiegando all'azienda i vantaggi dell'utilizzo di \textit{RedPanda}.
Per consolidare meglio la scelta da prendere, la proponente ha chiesto di mandare una mail, con scritte le nostre opinioni sui 2 strumenti e i vantaggi e svantaggi di ciascuno.

\subsection{Obiettivi secondo sprint}
Per il primo sprint ci siamo prefissati i seguenti obiettivi:
\begin{itemize}
	\item continuazione programma Python per la generazione dei dati di test;
	\item consolidamento database;
	\item migliorare persistenza su CLickHouse;
	\item visualizzazione su Grafana di almeno un sensore;
	\item continuo documentazione.
\end{itemize}

\subsection{Consigli su suddivisione e turnazione ruoli}
Durante il primo sprint, abbiamo adottato un sistema di rotazione dei ruoli ogni due settimane per ottenere una visione più chiara e completa delle attività svolte da ciascun membro del team. L'azienda ha sottolineato l'importanza del cambio di ruoli per consentire a tutti di svolgerli e acquisire nuove competenze.
È stato evidenziato che, nonostante le differenze di livello tra i membri, è fondamentale alternare i ruoli. In particolare, si è posta un'attenzione speciale sul ruolo del programmatore, considerato cruciale per il successo del progetto.

\subsection{Decisioni prese e conclusioni}
Infine discutiamo la durata degli sprint, decidendo insieme all'azienda, che per ora teniamo una durata di 2 settimane, ma in futuro potrà essere modificata.
La riunione si conclude dandoci appuntamento con l'azienda per il primo SAL fissato in data 06 maggio, entro il quale dovremo presentare il lavoro svolto durante il secondo sprint secondo gli obiettivi prefissati.

\begin{table}[b]
	\begin{tabular}{@{}p{.5in}p{4in}@{}}
		Data:  & \hrulefill \\
			   &     		\\
			   &     		\\
		Firma: & \hrulefill \\
	\end{tabular}
	\end{table}

\end{document}
