\documentclass[italian,12pt]{article}

%--------------variabili------------------%
\def\Title{Norme di Progetto}
\def\Author{7Last}
\def\Version{v0.2}
%-----------------------------------------%


\usepackage[left=2cm, right=2cm, bottom=3cm, top=3cm]{geometry}
\usepackage{fancyhdr}
\usepackage{graphicx}
\graphicspath{ {../../logo/} }
\usepackage{href-ul}
\usepackage{tikz}
\usepackage{tgadventor}
\usepackage[useregional=numeric,showseconds=true,showzone=false]{datetime2}
\usepackage{caption}
\usepackage{longtable}
\usepackage{xcolor}




\linespread{1.2}
\captionsetup[table]{labelformat=empty}
\geometry{headsep=1.5cm}

\renewcommand{\contentsname}{Indice}
\renewcommand\familydefault{\sfdefault}

\begin{document}

\newgeometry{left=2cm,right=2cm,bottom=2.1cm,top=2.1cm}
\begin{titlepage}
	\vspace*{.5cm}

	\vspace{2cm}
	{
		\centering
		{\bfseries\huge \Title\par}
		\bigbreak
		{\bfseries\Large \Subtitle\par}
		\bigbreak
		{\bfseries\large \Author\par}
		\bigbreak
		{\Date\;-\;\Version\par}
		\vfill

		\begin{center}
			\begin{tikzpicture}
				\clip (0,0) circle (2cm) node {\includegraphics[width=4cm]{logo.jpg}};
			\end{tikzpicture}
		\end{center}
	}

	\vfill

\end{titlepage}

\restoregeometry






















\newpage

\pagestyle{fancy}
\fancyhead{}
\lhead{
	\begin{tikzpicture}
		\clip (0,0) circle (0.5cm);
		\node at (0,0) {\includegraphics[width=1cm]{./../logo/logo.png}};
	\end{tikzpicture}%
}
\chead{\vspace{\fill}\Title\vspace{\fill}}
\rhead{\vspace{\fill}\Version\vspace{\fill}}


\begin{table}[!h]
	\caption{Versioni}
	\footnotesize
	\begin{center}
		\begin{tabular}{ l l l l p{6cm} }
			\hline                                                             \\[-2ex]
			Ver. & Data       & Redattore   & Verificatore   & Descrizione     \\
			\\[-2ex] \hline \\[-1.5ex]
			1.0  & 2024-03-16 & Elena Ferro & Leonardo Baldo & Stesura verbale \\
			\\[-1.5ex] \hline
		\end{tabular}
	\end{center}
\end{table}

\newpage

\tableofcontents

\newpage

\section{Dettagli della riunione}


\textbf{Sede della riunione}: Google Meet\\
\textbf{Orario di inizio}: 15:30\\
\textbf{Orario di fine}: 16:30\\

\begin{flushleft}
	\begin{table}[!h]
		\begin{tabular}{ |l|l|l| }
			\hline
			\textbf{Partecipante} & \textbf{Ruolo} & \textbf{Presenza} \\
			\hline
			Antonio Benetazzo     &                & Presente          \\
			Davide Malgarise      &                & Presente          \\
			Elena Ferro           & Redattore      & Presente          \\
			Leonardo Baldo        & Verificatore   & Presente          \\
			Matteo Tiozzo         &                & Assente           \\
			Raul Seganfreddo      & Amministratore & Presente          \\
			Valerio Occhinegro    &                & Presente          \\
			\hline
		\end{tabular}
	\end{table}
	\textbf{Partecipanti esterni}: Andrea Dorigo, Daniele Zorzi, Fabio Pallaro.\\
\end{flushleft}

\section{Ordine del giorno}
\begin{itemize}
	\item Stato di avanzamento della documentazione
	\item Stato di avanzamento dello sviluppo
	\item Considerazioni su tecnologie alternative a ClickHouse
	\item Obiettivi per il terzo sprint
	\item Decisioni prese
\end{itemize}

\newpage

\section{Verbale}

\subsection{Stato di avanzamento della documentazione}
L'incontro è iniziato con un riassunto da parte del responsabile, Davide Malgarise, di quanto svolto nel terzo sprint.
Per quanto riguarda la documentazione, il gruppo si è occupato di portare a termine l'\textit{Analisi dei Requisiti}
ed effettuare una prima stesura delle metriche che costituiscono il cruscotto di valutazione della qualità, all'interno del \textit{Piano di Qualifica}.
La proponente si ritiene soddisfatta delle metriche proposte e dell'impostazione dei grafici relativi.
I contenuti di tale documento verranno ulteriormente discussi quando il gruppo ne porterà a termine la stesura.

\subsection{Stato di avanzamento dello sviluppo}
Il gruppo ha poi proceduto a mostrare una dashboard realizzata con Grafana, la quale mostra i dati relativi a
un sensore salvati su ClickHouse.\\
La proponente ha suggerito di raffinare la visualizzazione delle dashboard, in particolare:
\begin{itemize}
	\label{item:consigli}
	\item Assicurarsi che nei marker rappresentanti i sensori nelle mappe siano presenti più di una cifra decimale per la latitudine e la longitudine;
	\item Fornire all'utente la possibilità di filtrare i dati visualizzati in base a determinati parametri (ad esempio il sensore a cui si riferiscono);
	\item Rendere più intuitivo il grafico \textit{daily mean temperature}, aggiungendo una \textit{label} sull'asse x che rappresenti il giorno (al momento presente solo l'orario);
	\item Rimuovere il grafico \textit{average temperature per minute} perché poco significativo. Convertirlo piuttosto in un grafico che mostri la temperatura in tempo reale di un sensore;
\end{itemize}

\subsection{Sviluppo di \textit{unit test} e \textit{integration test}}
Il gruppo ha inoltre posto una domanda riguardante la necessità di sviluppare \textit{unit test} e \textit{integration test} per il codice prodotto.
La proponente afferma che per il \textit{Proof of Concept} non è obbligatorio, in quanto l'architettura del sistema è ancora in fase di definizione
e potrebbe subire variazioni; eventuali test potrebbero essere sviluppati su componenti che si ritengono stabili e si prevede di continuare a mantenere.

\subsection{Considerazioni su tecnologie alternative a ClickHouse}
Si è in seguito brevemente discusso sulle motivazioni per cui ClickHouse è stato proposto da SyncLab per il progetto. La proponente ha spiegato che
rispetto a concorrenti come InfluxDB e TimescaleDB, ClickHouse consente di effettuare più efficientemente query analitiche complesse.
I punti di forza di InfluxDB e TimescaleDB sono invece una maggiore efficenza e minore richiesta di risorse, tuttavia
hanno una limitata capacità di effettuare query complesse.\\
È stata consigliata al gruppo la lettura di un articolo di confronto tra ClickHouse e TimescaleDB disponibile a \underline{\href{https://www.timescale.com/blog/what-is-clickhouse-how-does-it-compare-to-postgresql-and-timescaledb-and-how-does-it-perform-for-time-series-data/}{questo link}}.

\subsection{Obiettivi per il terzo sprint}
Sono stati definiti gli obiettivi per il terzo sprint:
\begin{itemize}
	\item Rifinire le dashboard seguendo i consigli di cui discusso al punto \ref{item:consigli};
	\item Aggiungere un'ulteriore tipologia di sensore;
	\item Portare a termine la definizione delle metriche all'interno del \textit{Piano di Qualifica};
	\item Completare definitivamente l'\textit{Analisi dei Requisiti};
	\item Stendere la parte di \textit{Piano di Progetto} relativa al terzo sprint.
\end{itemize}

\subsection{Decisioni prese}
\begin{itemize}
	\item Il gruppo si impegnerà a comunicare alla proponente i documenti modificati e i loro contenuti con anticipo di qualche giorno rispetto ai \textit{meeting}
	      di Stato Avanzamento Lavori, in modo da poterli più efficacemente discutere;
	\item Il terzo sprint avrà una durata di una settimana.
\end{itemize}

\newpage
\begin{table}[b]
	\begin{tabular}{@{}p{.5in}p{4in}@{}}
		Data:  & \hrulefill \\
		       &            \\
		       &            \\
		Firma: & \hrulefill \\
	\end{tabular}
\end{table}

\end{document}
