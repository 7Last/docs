\documentclass[italian,12pt]{article}

%--------------variabili------------------%
\def\Title{Norme di Progetto}
\def\Author{7Last}
\def\Version{v0.2}
%-----------------------------------------%


\usepackage[left=2cm, right=2cm, bottom=3cm, top=3cm]{geometry}
\usepackage{fancyhdr}
\usepackage{graphicx}
\graphicspath{ {../../logo/} }
\usepackage{href-ul}
\usepackage{tikz}
\usepackage{tgadventor}
\usepackage[useregional=numeric,showseconds=true,showzone=false]{datetime2}
\usepackage{caption}
\usepackage{longtable}
\usepackage{xcolor}




\linespread{1.2}
\captionsetup[table]{labelformat=empty}
\geometry{headsep=1.5cm}

\renewcommand{\contentsname}{Indice}
\renewcommand\familydefault{\sfdefault}

\begin{document}

\newgeometry{left=2cm,right=2cm,bottom=2.1cm,top=2.1cm}
\begin{titlepage}
	\vspace*{.5cm}

	\vspace{2cm}
	{
		\centering
		{\bfseries\huge \Title\par}
		\bigbreak
		{\bfseries\Large \Subtitle\par}
		\bigbreak
		{\bfseries\large \Author\par}
		\bigbreak
		{\Date\;-\;\Version\par}
		\vfill

		\begin{center}
			\begin{tikzpicture}
				\clip (0,0) circle (2cm) node {\includegraphics[width=4cm]{logo.jpg}};
			\end{tikzpicture}
		\end{center}
	}

	\vfill

\end{titlepage}

\restoregeometry






















\newpage

\pagestyle{fancy}
\fancyhead{}
\lhead{
	\begin{tikzpicture}
		\clip (0,0) circle (0.5cm);
		\node at (0,0) {\includegraphics[width=1cm]{./../logo/logo.png}};
	\end{tikzpicture}%
}
\chead{\vspace{\fill}\Title\vspace{\fill}}
\rhead{\vspace{\fill}\Version\vspace{\fill}}


\begin{table}[!h]
	\caption{Versioni}
	\footnotesize
	\begin{center}
		\begin{tabular}{ l l l l p{6cm} }
			\hline                                                                              \\[-2ex]
			Ver. & Data       & Redattore          & Verificatore       & Descrizione           \\
			\\[-2ex] \hline \\[-1.5ex]
			1.0  & 2024-06-08 & Davide Malgarise    & Elena Ferro & Stesura verbale \\
			\\[-1.5ex] \hline
		\end{tabular}
	\end{center}
\end{table}

\newpage

\tableofcontents

\newpage

\section{Dettagli della riunione}


\textbf{Sede della riunione}: Google Meet\\
\textbf{Orario di inizio}: 16:00\\
\textbf{Orario di fine}: 16:45\\

\begin{flushleft}
	\begin{table}[!h]
	\begin{tabular}{ |l|l|l| } 
		\hline
		\textbf{Partecipante} & \textbf{Ruolo}       & \textbf{Presenza} \\
		\hline 
		Antonio Benetazzo     &                      & Presente          \\
		Davide Malgarise      & Redattore            & Presente          \\
		Elena Ferro           & Verificatore         & Presente          \\
		Leonardo Baldo        &                      & Presente          \\
		Matteo Tiozzo         &                      & Presente          \\
		Raul Seganfreddo      &  					 & Presente          \\
		Valerio Occhinegro    & Amministratore       & Presente          \\
		\hline
	\end{tabular}
	\end{table}
	\textbf{Partecipanti esterni}: Andrea Dorigo, Daniele Zorzi.\\
\end{flushleft}

\section{Ordine del giorno}
\begin{itemize}
	\item Stato di sviluppo del prodotto
	\item Discussione sull'architettura del prodotto
	\item Obiettivi per il settimo sprint
	\item Decisioni prese e conclusioni
\end{itemize}

\newpage

\section{Verbale}

\subsection{Stato di sviluppo del prodotto}
Come prima cosa, il responsabile del periodo appena concluso, Elena Ferro, riferisce al prononente l'esito della prima parte di valutazione RTB,
avvenuta in data 29 maggio in presenza del professore Riccardo Cardin. Il gruppo ha ricevuto un feedback positivo e può procedere con la seconda
parte di valutazione anche se sono state richieste alcune migliorie. Di conseguenza, assieme al proponente il team discute e analizza
i requisiti funzionali, qualitativi e di vincolo. Ne risulta che il grado di specificità è adeguato e soddisfacente, è però consigliata l'aggiunta 
di requisiti di performance e tempistica, tenendo in considerazione casi di latenza e di carico.

\subsection{Discussione sull'architettura del prodotto}
L'incontro prosegue con la presentazione dell'architettura progettata, che viene valutata positivamente e in linea con il progetto.
Il gruppo espone due principali dubbi

\subsection{Obiettivi per il settimo sprint}
Sono stati definiti i seguenti obiettivi per il settimo sprint:
\begin{itemize}
	\item Implementare due nuovi sensori;
	\item Iniziare la fase di test;
	\item Effettuare prove di carico e registrare i risultati;
	\item Aggiungere requisiti di performance.
\end{itemize}

\subsection{Decisioni prese e conclusioni}
In accordo con il proponente, si mantiene la durata del periodo di lavoro di una settimana e dunque si fissa il prossimo incontro in data
12 giugno 2024. La riunione si conclude con una breve discussione sulle possibile tempistiche di completamento e consegna del progetto,
auspicabile per la prima metà di luglio.

\newpage
\begin{table}[b]
	\begin{tabular}{@{}p{5cm}p{10cm}@{}}
		Data:  & \hrulefill \\
		       &            \\
		       &            \\
		Firma del proponente: & \hrulefill \\
	\end{tabular}
\end{table}

\end{document}
