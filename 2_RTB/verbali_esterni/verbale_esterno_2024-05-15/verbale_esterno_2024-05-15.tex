\documentclass[italian,12pt]{article}

%--------------variabili------------------%
\def\Title{Norme di Progetto}
\def\Author{7Last}
\def\Version{v0.2}
%-----------------------------------------%


\usepackage[left=2cm, right=2cm, bottom=3cm, top=3cm]{geometry}
\usepackage{fancyhdr}
\usepackage{graphicx}
\graphicspath{ {../../logo/} }
\usepackage{href-ul}
\usepackage{tikz}
\usepackage{tgadventor}
\usepackage[useregional=numeric,showseconds=true,showzone=false]{datetime2}
\usepackage{caption}
\usepackage{longtable}
\usepackage{xcolor}




\linespread{1.2}
\captionsetup[table]{labelformat=empty}
\geometry{headsep=1.5cm}

\renewcommand{\contentsname}{Indice}
\renewcommand\familydefault{\sfdefault}

\begin{document}

\newgeometry{left=2cm,right=2cm,bottom=2.1cm,top=2.1cm}
\begin{titlepage}
	\vspace*{.5cm}

	\vspace{2cm}
	{
		\centering
		{\bfseries\huge \Title\par}
		\bigbreak
		{\bfseries\Large \Subtitle\par}
		\bigbreak
		{\bfseries\large \Author\par}
		\bigbreak
		{\Date\;-\;\Version\par}
		\vfill

		\begin{center}
			\begin{tikzpicture}
				\clip (0,0) circle (2cm) node {\includegraphics[width=4cm]{logo.jpg}};
			\end{tikzpicture}
		\end{center}
	}

	\vfill

\end{titlepage}

\restoregeometry






















\newpage

\pagestyle{fancy}
\fancyhead{}
\lhead{
	\begin{tikzpicture}
		\clip (0,0) circle (0.5cm);
		\node at (0,0) {\includegraphics[width=1cm]{./../logo/logo.png}};
	\end{tikzpicture}%
}
\chead{\vspace{\fill}\Title\vspace{\fill}}
\rhead{\vspace{\fill}\Version\vspace{\fill}}


\begin{table}[!h]
	\caption{Versioni}
	\footnotesize
	\begin{center}
		\begin{tabular}{ l l l l p{6cm} }
			\hline                                                             \\[-2ex]
			Ver. & Data       & Redattore   & Verificatore   & Descrizione     \\
			\\[-2ex] \hline \\[-1.5ex]
			1.0  & 2024-05-16 & Davide Malgarise & Raul Seganfreddo & Stesura verbale \\
			\\[-1.5ex] \hline
		\end{tabular}
	\end{center}
\end{table}

\newpage

\tableofcontents

\newpage

\section{Dettagli della riunione}


\textbf{Sede della riunione}: Google Meet\\
\textbf{Orario di inizio}: 15:30\\
\textbf{Orario di fine}: 16:30\\

\begin{flushleft}
	\begin{table}[!h]
		\begin{tabular}{ |l|l|l| }
			\hline
			\textbf{Partecipante} & \textbf{Ruolo}  & \textbf{Presenza} \\
			\hline
			Antonio Benetazzo     &                 & Presente          \\
			Davide Malgarise      & Redattore       & Presente          \\
			Elena Ferro           &                 & Presente          \\
			Leonardo Baldo        & 			    & Presente          \\
			Matteo Tiozzo         & Amministratore  & Presente          \\
			Raul Seganfreddo      & Verificatore    & Presente          \\
			Valerio Occhinegro    &                 & Presente          \\
			\hline
		\end{tabular}
	\end{table}
	\textbf{Partecipanti esterni}: Andrea Dorigo, Daniele Zorzi, Fabio Pallaro.\\
\end{flushleft}


\section{Ordine del giorno}
\begin{itemize}
	\item Stato di avanzamento della documentazione
	\item Stato di avanzamento dello sviluppo
	\item Stream processing
	\item Obiettivi per il quarto sprint
	\item Decisioni prese
\end{itemize}

\newpage

\section{Verbale}

\subsection{Stato di avanzamento della documentazione}
Come prima cosa, il responsabile del terzo sprint, Antonio Benetazzo, ha riassunto quanto è stato svolto nel corso del periodo di lavoro corrente.
Riguardo la documentazione, il gruppo ha aggiunto i casi d'uso mancanti al documento \textit{Analisi dei Requisiti}, 
ha provveduto a popolare i grafici nel cruscotto di valutazione della qualità all'interno del \textit{Piano di Qualifica} e
ha aggiornato il documento \textit{Piano di Progetto} con i dati relativi al terzo sprint. Nel frattempo, sono stati apportati aggiustamenti
ortografici nelle \textit{Norme di Progetto} e sono stati aggiunti nuovi termini al \textit{Glossario}.

\subsection{Stato di avanzamento dello sviluppo}
In seguito il team ha proseguito con la presentazione delle modifiche apportate alle dashboard in \textit{Grafana}. Come richiesto nel precedente
incontro, sono state aggiunte le cifre decimali alle variabili di latitudine e longitudine nelle mappe e le unità di misura nei grafici,
è stato implementato un filtro in base al sensore desiderato e una lista delle dashboard disponibili, che aiuta l'utente nella navigazione.
Inoltre sono stati introdotti nuovi grafici, sia nella dashboard riguardante la temperatura, sia in quella riguardante il traffico, e infine
è stato implementato un nuovo sensore che gestisce il riempimento delle isole ecologiche. Nel corso della presentazione la proponente ha suggerito
alcune modifiche per raffinare la visualizzazione delle dashboard, come ad esempio:
\begin{itemize}
	\item aggiunta di altri sensori dislocati in modo significativo;
	\item utilizzo della funzione \textit{zoom to data} per una maggiore immediatezza nella visualizzazione dei dati;
	\item distinzione tra dashboard di monitoraggio e dashboard di analisi;
	\item aggiunta di query di aggregazione su più livelli
	(ad esempio nella dashboard riguardante il traffico, distinguere le varie tipologie di veicoli monitorati).
\end{itemize}

\newpage
\subsection{Stream processing}

\subsection{Obiettivi per il quarto sprint}
Sono stati definiti gli obiettivi per il quarto sprint:
\begin{itemize}
	\item popolamento dei grafici del documento \textit{Piano di Qualifica};
	\item revisione dei documenti \textit{Piano di Progetto} e \textit{Piano di Qualifica};
	\item formazione e analisi dei business case e relativi strumenti;
	\item raffinamento delle dashboard attraverso le modifiche elencate precedentemente.
\end{itemize}

\subsection{Decisioni prese}

\newpage
\begin{table}[b]
	\begin{tabular}{@{}p{.5in}p{4in}@{}}
		Data:  & \hrulefill \\
		       &            \\
		       &            \\
		Firma: & \hrulefill \\
	\end{tabular}
\end{table}

\end{document}
