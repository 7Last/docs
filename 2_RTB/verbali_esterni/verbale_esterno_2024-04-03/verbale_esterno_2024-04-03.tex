\documentclass[italian,12pt]{article}

%--------------variabili------------------%
\def\Title{Norme di Progetto}
\def\Author{7Last}
\def\Version{v0.2}
%-----------------------------------------%


\usepackage[left=2cm, right=2cm, bottom=3cm, top=3cm]{geometry}
\usepackage{fancyhdr}
\usepackage{graphicx}
\graphicspath{ {../../logo/} }
\usepackage{href-ul}
\usepackage{tikz}
\usepackage{tgadventor}
\usepackage[useregional=numeric,showseconds=true,showzone=false]{datetime2}
\usepackage{caption}
\usepackage{longtable}
\usepackage{xcolor}




\linespread{1.2}
\captionsetup[table]{labelformat=empty}
\geometry{headsep=1.5cm}

\renewcommand{\contentsname}{Indice}
\renewcommand\familydefault{\sfdefault}

\begin{document}

\newgeometry{left=2cm,right=2cm,bottom=2.1cm,top=2.1cm}
\begin{titlepage}
	\vspace*{.5cm}

	\vspace{2cm}
	{
		\centering
		{\bfseries\huge \Title\par}
		\bigbreak
		{\bfseries\Large \Subtitle\par}
		\bigbreak
		{\bfseries\large \Author\par}
		\bigbreak
		{\Date\;-\;\Version\par}
		\vfill

		\begin{center}
			\begin{tikzpicture}
				\clip (0,0) circle (2cm) node {\includegraphics[width=4cm]{logo.jpg}};
			\end{tikzpicture}
		\end{center}
	}

	\vfill

\end{titlepage}

\restoregeometry






















\newpage

\pagestyle{fancy}
\fancyhead{}
\lhead{
	\begin{tikzpicture}
		\clip (0,0) circle (0.5cm);
		\node at (0,0) {\includegraphics[width=1cm]{./../logo/logo.png}};
	\end{tikzpicture}%
}
\chead{\vspace{\fill}\Title\vspace{\fill}}
\rhead{\vspace{\fill}\Version\vspace{\fill}}


\begin{table}[!h]
	\caption{Versioni}
	\footnotesize
	\begin{center}
		\begin{tabular}{ c c c c p{6.1cm} }
			\hline                                                                                                      \\[-2ex]
			Ver. & Data       & Redattore          & Verificatore       & Descrizione                                   \\
			\\[-2ex] \hline \\[-1.5ex]
			1.0  & 09/04/2024 & Antonio Benetazzo  & Valerio Occhinegro & Stesura verbale                               \\
			\\[-1.5ex] \hline
		\end{tabular}
	\end{center}
\end{table}

\newpage

\tableofcontents

\newpage

\section{Dettagli della riunione}


\textbf{Sede della riunione}: Piattaforma Discord\\
\textbf{Orario di inizio}: 17:00\\
\textbf{Orario di fine}: 18:00\\


\begin{flushleft}
	\begin{table}[!h]
	\begin{tabular}{ |l|l|l| } 
		\hline
		\textbf{Partecipante} & \textbf{Ruolo}       & \textbf{Presenza} \\
		\hline 
		Antonio Benetazzo     & Redattore            & Presente          \\
		Davide Malgarise      & Partecipante interno & Presente          \\
		Elena Ferro           & Partecipante interno & Presente          \\
		Leonardo Baldo        & Partecipante interno & Presente          \\
		Matteo Tiozzo         & Amministratore       & Presente          \\
		Raul Seganfreddo      & Partecipante interno & Presente          \\
		Valerio Occhinegro    & Verificatore         & Presente          \\
		Andrea Dorigo         & Partecipante esterno & Presente          \\
		Daniele Zorzi         & Partecipante esterno & Presente          \\
		Fabio Pallaro         & Partecipante esterno & Presente          \\
		\hline
	\end{tabular}
	\end{table}
\end{flushleft}

\section{Ordine del giorno}
\subsection{Approfondire conoscenza reciproca}
\subsection{Modalità di comunicazione e organizzazione del lavoro}
\subsection{Obiettivi primo sprint}
\subsection{Consigli su suddivisione e turnazione Ruoli}
\subsection{Varie ed eventuali}

\newpage

\section{Verbale}

\subsection{Approfondire conoscenza reciproca}
L'incontro è iniziato con l'ulteriore comunicazione della conferma dell'assegnazione 
della candidatura da parte del professor Vardanega (precedentemente comunicato via e-mail). 
Successivamente ci siamo presentati e abbiamo discusso brevemente delle motivazioni 
che ci hanno portato alla scelta di questo capitolato. Abbiamo inoltre ritagliato un breve spazio 
per discutere dei nostri hobby e delle nostre passioni, al fine di creare un ambiente di lavoro 
più informale e amichevole.

\subsection{Modalità di comunicazione e organizzazione del lavoro}
Ci viene indicata la figura di Andrea Zorzi come principale figura di riferimento per 
supporto sulle tecnologie che verranno adottate. Abbiamo deciso di utilizzare principalmente 
la piattaforma Discord per la comunicazione, già utilizzata anche da parte dell'azienda per 
le comunicazioni interne. Con l'azienda si è concordato di partire con sprint della durata 
di 2 settimnane, questo per poter avere un feedback rapido e poter correggere eventuali errori 
tempestivamente, ma al tempo stesso avere tempo a sufficienza per poter prendere confidenza 
con il ruolo assegnato a ciascun membro del gruppo; una durata inferiore, in particolare 
in questa fase iniziale del progetto, potrebbe infatti portare a confusione e disorganizzazione.

\subsection{Obiettivi primo sprint}
Per il primo sprint ci siamo prefissati i seguenti obiettivi:
\begin{itemize}
	\item Studio delle tecnologie \textit{Docker Compose} e \textit{Kafka} (ci riserviamo di valutare eventuali alternative);
	\item Prima configurazione di questi strumenti;
	\item Creazione di un semplice programma in python per la generazione dei dati di test (per ora anche di un solo tipo di dato);
	\item (OPZIONALE ma considerato fattibile) Inserimento di questi dati direttamente in \textit{Kafka}, senza l'utilizzo di scambio di file;
	\item Deploy su \textit{Docker Compose};
	\item Prima documentazione.
\end{itemize}

\subsection{Consigli su suddivisione e turnazione Ruoli}
Chiediamo ai rappresentanti dell'azienda un consiglio su come suddividere i ruoli e come 
organizzare la turnazione degli stessi, vista la loro pregressa esperienza nella gestione 
di progetti simili e vista la precedente esperienza coi nostri colleghi del primo lotto. 
Ci viene consigliato di mantenere una turnazione dei ruoli ogni due settimane, in linea con 
le tempistiche degli sprint, in modo da poter avere una visione più chiara e completa di 
come si svolge il lavoro di ciascun membro del gruppo.

\subsection{Decisioni prese e conclusioni}
La riunione si conclude dandoci appuntamento con l'azienda per il primo SAL fissato 
in data 19 aprile, entro il quale dovremo presentare il lavoro svolto durante il primo sprint 
secondo gli obiettivi prefissati.

\newpage
\begin{table}[b]
	\begin{tabular}{@{}p{.5in}p{4in}@{}}
		Data:  & \hrulefill \\
			   &     		\\
			   &     		\\
		Firma: & \hrulefill \\
	\end{tabular}
	\end{table}

\end{document}
