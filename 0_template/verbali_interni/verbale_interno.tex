\documentclass[italian,12pt]{article} %tipo di documento

%--------------variabili------------------%
\def\Title{Norme di Progetto}
\def\Author{7Last}
\def\Version{v0.2}
%-----------------------------------------%


\usepackage[left=2cm, right=2cm, bottom=3cm, top=3cm]{geometry}
\usepackage{fancyhdr}
\usepackage{graphicx}
\graphicspath{ {../../logo/} }
\usepackage{href-ul}
\usepackage{tikz}
\usepackage{tgadventor}
\usepackage[useregional=numeric,showseconds=true,showzone=false]{datetime2}
\usepackage{caption}
\usepackage{longtable}
\usepackage{xcolor}




\linespread{1.2}
\captionsetup[table]{labelformat=empty}

\renewcommand{\contentsname}{Indice}%imposto il nome dell'indice
\renewcommand\familydefault{\sfdefault}

%-------------------INIZIO DOCUMENTO--------------
\begin{document}

\newgeometry{left=2cm,right=2cm,bottom=2.1cm,top=2.1cm}
\begin{titlepage}
	\vspace*{.5cm}

	\vspace{2cm}
	{
		\centering
		{\bfseries\huge \Title\par}
		\bigbreak
		{\bfseries\Large \Subtitle\par}
		\bigbreak
		{\bfseries\large \Author\par}
		\bigbreak
		{\Date\;-\;\Version\par}
		\vfill

		\begin{center}
			\begin{tikzpicture}
				\clip (0,0) circle (2cm) node {\includegraphics[width=4cm]{logo.jpg}};
			\end{tikzpicture}
		\end{center}
	}

	\vfill

\end{titlepage}

\restoregeometry






















\newpage

\pagestyle{fancy}
\fancyhead{}
\lhead{
	\begin{tikzpicture}
		\clip (0,0) circle (0.5cm);
		\node at (0,0) {\includegraphics[width=1cm]{./../logo/logo.png}};
	\end{tikzpicture}%
}
\chead{\vspace{\fill}\Title\vspace{\fill}}
\rhead{\vspace{\fill}\Version\vspace{\fill}}


%-----------tabella revisioni-----------%
\begin{table}[!h]
	\caption{Versioni}
	\begin{center}
		\begin{tabular}{ c c c p{9cm} }
			\hline                                                                                 \\[-2ex]
			Ver. & Data       & Autore             & Descrizione                                   \\
			\\[-2ex] \hline \\[-1.5ex]
			1.1  & 22/03/2024 & Antonio Benetazzo  & Riordinamento copertina e intestazione pagine \\
			1.0  & 16/03/2024 & Raul Seganfredo    & Approvazione template                         \\
			0.3  & 16/03/2024 & Matteo Tiozzo      & Modificato layout grafico                     \\
			0.2  & 13/03/2024 & Elena Ferro        & Modificato font e logo                        \\
			0.1  & 12/03/2024 & Valerio Occhinegro & Creazione template                            \\
			\\[-1.5ex] \hline
		\end{tabular}
	\end{center}
\end{table}
%---------------------------------------%

\newpage

\tableofcontents

\newpage

\section{Sezione 1}
Qui va il testo della sezione 1.

\paragraph{Paragrafo 1}
Qui va il testo del paragrafo 1.

\paragraph{Paragrafo 2}
Qui va il testo del paragrafo 2, con esempio di elenco puntato.
\begin{itemize}
	\itemsep0em
	\item \textbf{1}: Elemento 1
	\item \textbf{2}: Elemento 2
	\item \textbf{3}: Elemento 3
\end{itemize}



\section{Sezione 2}
Qui va il testo della sezione 2.
\subsection{Sottosezione 1}
Qui va il testo della sottosezione 1.


\end{document}
