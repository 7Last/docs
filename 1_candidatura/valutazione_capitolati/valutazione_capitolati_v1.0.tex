\documentclass[italian,12pt]{article} %tipo di documento

%--------------variabili------------------%
\def\Title{Norme di Progetto}
\def\Author{7Last}
\def\Version{v0.2}
%-----------------------------------------%


\usepackage[left=2cm, right=2cm, bottom=3cm, top=3cm]{geometry}
\usepackage{fancyhdr}
\usepackage{graphicx}
\graphicspath{ {../../logo/} }
\usepackage{href-ul}
\usepackage{tikz}
\usepackage{tgadventor}
\usepackage[useregional=numeric,showseconds=true,showzone=false]{datetime2}
\usepackage{caption}
\usepackage{longtable}
\usepackage{xcolor}




\linespread{1.2}
\captionsetup[table]{labelformat=empty}

\renewcommand{\contentsname}{Indice}%imposto il nome dell'indice
\renewcommand\familydefault{\sfdefault}

%-------------------INIZIO DOCUMENTO--------------
\begin{document}

\newgeometry{left=2cm,right=2cm,bottom=2.1cm,top=2.1cm}
\begin{titlepage}
	\vspace*{.5cm}

	\vspace{2cm}
	{
		\centering
		{\bfseries\huge \Title\par}
		\bigbreak
		{\bfseries\Large \Subtitle\par}
		\bigbreak
		{\bfseries\large \Author\par}
		\bigbreak
		{\Date\;-\;\Version\par}
		\vfill

		\begin{center}
			\begin{tikzpicture}
				\clip (0,0) circle (2cm) node {\includegraphics[width=4cm]{logo.jpg}};
			\end{tikzpicture}
		\end{center}
	}

	\vfill

\end{titlepage}

\restoregeometry






















\newpage

\pagestyle{fancy}
\fancyhead{}
\lhead{
	\begin{tikzpicture}
		\clip (0,0) circle (0.5cm);
		\node at (0,0) {\includegraphics[width=1cm]{./../logo/logo.png}};
	\end{tikzpicture}%
}
\chead{\vspace{\fill}\Title\vspace{\fill}}
\rhead{\vspace{\fill}\Version\vspace{\fill}}


%-----------tabella revisioni-----------%
\begin{table}[!h]
	\caption{Versioni}
	\begin{center}
		\begin{tabular}{ c c c c c c }
			\hline                                                                              \\[-2ex]
			Ver. & Data       & Autore         & Descrizione                                    \\
			\\[-2ex] \hline \\[-1.5ex]
			1.0  & 19/03/2024 & Elena Ferro    & Revisione finale                               \\
			0.3  & 18/03/2024 & Elena Ferro    & Aggiunta di pro e contro                       \\
			0.2  & 18/03/2024 & Leonardo Baldo & Aggiunti domini applicativi e tecnologici      \\
			0.1  & 17/03/2024 & Leonardo Baldo & Aggiunta Introduzione e descrizione Capitolati \\
			\\[-1.5ex] \hline
		\end{tabular}
	\end{center}
\end{table}
%---------------------------------------%

\newpage

\tableofcontents

\newpage

\section{Introduzione}
Lo scopo del presente documento è quello di effettuare un'analisi dei Capitolati esposti
dalle aziende proponenti, consultabili presso

\url{https://www.math.unipd.it/~tullio/IS-1/2023/Progetto/Capitolati.html}\\
Verranno analizzate le motivazioni della scelta del Capitolato C6
e gli aspetti positivi e negativi che hanno spinto il gruppo a scartare le altre proposte.

\section{Valutazione del Capitolato Selezionato}

\subsection{Capitolato C6 - SyncCity}

\subsubsection{Descrizione}
\begin{itemize}
	\item \textbf{Nome}: SyncCity: Smart city monitoring platform
	\item \textbf{Proponente}: {\it Sync Lab S.r.l.}
	\item \textbf{Committenti}: {\it Prof. Tullio Vardanega, Prof. Riccardo Cardin}
	\item \textbf{Obiettivo}: Questo progetto mira a creare una piattaforma
	      con dashboard associata, che permetta di monitorare dati provenienti da sensori IoT
	      distribuiti nelle città, al fine di adottare decisioni tempestive
	      e di analizzarne gli effetti.
\end{itemize}

\subsubsection{Domini}
\paragraph{Dominio applicativo}\mbox{}\\
Le tecnologie che la proponente suggerisce di utilizzare (quali \textit{Apache Kafka} e \textit{ClickHouse}) offrono la possibilità di elaborare,
aggregare e archiviare i \textit{raw data} raccolti dai sensori.
Per semplificarne la realizzazione, si richiede che lo \textit{stream} di dati sia simulato in modo realistico.
Possibili esempi forniti dalla proponente di sensori sono i seguenti:
\begin{itemize}\itemsep0em
	\item temperatura, umidità, polveri sottili dell’aria
	\item il traffico, i lavori in corso, gli incidenti e i parcheggi nella cella
	\item i lavori sulla rete idrica, i livelli dell'acqua
	\item la posizione delle colonnine di ricarica ed eventuali guasti elettrici
	\item lo stato di riempimento delle isole ecologiche
\end{itemize}

\paragraph{Dominio tecnologico}\mbox{}\\
La proponente consiglia fortemente l'utilizzo delle seguenti tecnologie per lo svolgimento del progetto:
\begin{itemize}
	\item\textbf{Python}: per la simulazione dei dati quanto più possibile realistica attraverso script, ed eventualmente librerie di generazione dati (\textit{Faker}).
	\item\textbf{Apache Kafka}: un broker, strumento standard per separare e gestire il flusso di informazioni proveniente da diverse fonti, come i simulatori.
	\item\textbf{ClickHouse}: è un database OLAP colonnare che svolgerà il ruolo di persistere grandi quantità di dati. Inoltre è agevolmente integrabile con \textit{Kafka}.
	\item\textbf{Grafana}: piattaforma di \textit{data visualization} delle informazioni, che consente di creare delle \textit{dashboard} altamente personalizzabili. Questa componente rappresenta il \textit{frontend} dell'utente e consentirà il monitoraggio della città.
	\item\textbf{Docker}: per lo sviluppo e l'esecuzione dell'applicazione in un ambiente isolato.
\end{itemize}

\subsubsection{Valutazione}
\paragraph{Aspetti positivi}
\begin{itemize}
	\item L'ampia \textbf{diffusione delle tecnologie} utilizzate consentirebbe al gruppo, almeno inizialmente, di accedere a una vasta gamma di risorse online, come documentazione e supporto, prima di dover ricorrere alla proponente in caso di eventuali necessità.
	\item L'utilizzo di tali tecnologie è molto richiesto nel \textbf{mondo del lavoro} e consentirebbe dunque ai componenti del gruppo di arricchire il proprio bagaglio di preziose conoscenze.
	\item L'azienda proponente si è dichiarata disponibile ad offrire il \textbf{supporto tecnico} necessario.
	\item La presentazione del capitolato e il successivo incontro conoscitivo hanno lasciato un'\textbf{ottima impressione}. Il documento era chiaro e ben strutturato, inoltre i rappresentanti dell'azienda si sono dimostrati competenti e disponibili durante la discussione.
	\item L'azienda proponente si è dimostrata propensa ad \textbf{investire sui giovani} e sulle nuove tecnologie.
\end{itemize}

\paragraph{Aspetti negativi}\mbox{}\\
Durante le discussioni non sono emersi aspetti negativi legati a questo capitolato.

\subsubsection{Conclusioni}
Tutti i membri del gruppo hanno espresso all'unisono una preferenza per questo capitolato.
La decisione di adottare questo progetto è supportata dalla chiarezza del documento
presentato e dall'alto livello di competenza e disponibilità dimostrato dai membri dell'azienda.

\section{Valutazione degli altri Capitolati}

\subsection{Capitolato C9 - ChatSQL}

\subsubsection{Descrizione}
\begin{itemize}
	\item \textbf{Nome}: ChatSQL: creare frasi SQL da linguaggio naturale
	\item \textbf{Proponente}: {\it Zucchetti S.p.A.}
	\item \textbf{Committenti}: {\it Prof. Tullio Vardanega, Prof. Riccardo Cardin}
	\item \textbf{Obiettivo}: Nel capitolato si propone la realizzazione di un chatbot per la generazione di query SQL a partire da una frase in linguaggio naturale e dalla struttura del database.
\end{itemize}

\subsubsection{Domini}
\paragraph{Dominio applicativo}\mbox{}\\
Il proposito di questo capitolato è semplificare l'interazione con i LLM (\textit{Large Language Model}) al fine di facilitare la generazione di query SQL da descrizioni in
linguaggio naturale. Rispetto all'interrogazione diretta dei modelli di intelligenza artificiale, questo approccio elimina la necessità di specificare la struttura
del database ogni volta. Inoltre, consente una generazione più efficiente e mirata delle query, poiché nel \textit{prompt} vengono incluse solo le tabelle coinvolte.
\\\linebreak\pagebreak\linebreak
L'azienda chiede di sviluppare un'applicazione che svolga i seguenti compiti:
\begin{itemize}
	\item Archiviazione della descrizione strutturale di un database, possibilmente commentata in tutte le sue parti.
	\item Maschera di richiesta di una frase di interrogazione del database in linguaggio naturale.
	\item Procedura che combina la richiesta di interrogazione con le informazioni della struttura del database creando un \textit{prompt}, che
	      dato in input ad un LLM fornisce l’interrogazione equivalente al linguaggio naturale in linguaggio SQL.
	\item Tutte queste funzionalità dovranno essere integrate in un unico sistema.
\end{itemize}
\paragraph{Dominio tecnologico}\mbox{}\\
Non sono imposti vincoli sulle tecnologie da utilizzare, ma la proponente consiglia le seguenti:
\begin{itemize}
	\item \textbf{Python} che trova facile applicazione in questo ambito
	\item \textbf{HTML}, \textbf{JavaScript}, \textbf{CSS}
\end{itemize}

\subsubsection{Valutazione}
\paragraph{Aspetti positivi}
\begin{itemize}
	\item \textbf{Libertà sulle tecnologie} utilizzabili.
	\item La proponente si dichiara disponibile a \textbf{fornire il supporto} necessario al gruppo.
	\item Il capitolato prevede di lavorare a contatto con una \textbf{tecnologia emergente} come gli LLM.
\end{itemize}

\paragraph{Aspetti negativi}
\begin{itemize}
	\item Il \textbf{documento} di presentazione è in alcune sue parti \textbf{poco chiaro}, ad esempio sui vantaggi nell'utilizzare tale progetto rispetto ad interagire
	      direttamente con LLM come ChatGPT.
	\item Molti LLM, tra cui ChatGPT, offrono delle \textbf{API key} solo \textbf{a pagamento}, cosa che limita il funzionamento del progetto nella
	      sua interezza. Verrebbe infatti ritornato un prompt da dare in input manualmente ad un modello di AI generativa.

\end{itemize}

\subsubsection{Conclusioni}
La maggioranza dei membri del gruppo (4 membri su 7) ha votato questo capitolato come seconda scelta. Nonostante si riconosca
che l'applicazione di questo progetto possa portare notevoli vantaggi nel velocizzare l'interrogazione di \textit{database},
non risulta sufficientemente stimolante per poter essere posto come prima scelta.

\subsection{Capitolato C3 - EasyMeal}

\subsubsection{Descrizione}
\begin{itemize}
	\item \textbf{Nome}: EasyMeal
	\item \textbf{Proponente}: {\it Imola informatica S.p.A.}
	\item \textbf{Committenti}: {\it Prof. Tullio Vardanega, Prof. Riccardo Cardin}
	\item \textbf{Obiettivo}: Il capitolato intende semplificare il processo di prenotazione nei ristoranti,
	      permettendo ai clienti di riservare un tavolo, ordinare in modo rapido e consentendo
	      loro di interagire direttamente con il personale, tramite una \textit{web app}.
\end{itemize}

\subsubsection{Domini}
\paragraph{Dominio applicativo}\mbox{}\\
Oltre a facilitare la prenotazione dei pasti, \textit{EasyMeal} porta anche ad una diminuzione degli sprechi alimentari, un tema molto importante al giorno d'oggi. \\
Nello specifico, in questo progetto si prendono in considerazione le seguenti operazioni che vengono effettuate dai clienti o dai ristoranti:
\begin{itemize}\itemsep0em
	\item Registrazione di un nuovo utente
	\item Prenotazione di un tavolo
	\item Ordinazione collaborativa dei pasti
	\item Interazione con lo staff del ristorante
	\item Divisione del conto
	\item Consultazione delle prenotazioni da parte di un amministratore del ristorante
	\item Inserimento di feedback e recensioni
\end{itemize}
\paragraph{Dominio tecnologico}\mbox{}\\
L'azienda proponente richiede l'implementazione di un'applicazione \textit{web responsive} (browser, IOS e Android) e rispettivo servizio \textit{backend}.
Viene lasciata totale libertà implementativa, tuttavia sono state consigliate alcune tecnologie:
\begin{itemize}
	\item \textbf{HTML}, \textbf{CSS}, \textbf{JavaScript}, \textbf{PHP}.
	\item Framework per applicazioni web: \textbf{ReactJS} e \textbf{Angular}.
\end{itemize}

\subsubsection{Valutazione}
\paragraph{Aspetti positivi}
\begin{itemize}
	\item I rappresentanti della proponente si sono mostrati \textbf{competenti e disponibili} a fornire il supporto tecnico necessario.
	      Inoltre si propongono di offrire aiuto nella gestione dei ruoli nel team.
	\item Essendo che gli sviluppatori di \textit{web app responsive} sono molto \textbf{richiesti} nel \textbf{mondo del lavoro}, \textit{EasyMeal} permetterebbe
	      di migliorare o consolidare le \textit{skill} del gruppo in questo ambito.
\end{itemize}
\paragraph{Aspetti negativi}
\begin{itemize}
	\item Più di un componente del gruppo ha \textbf{esperienza} nello sviluppo di \textbf{progetti simili}, dunque risulta poco accattivante.
	\item Il focus di questo capitolato sarebbe la scrittura di codice piuttosto che l'aspetto progettuale.
\end{itemize}

\subsubsection{Conclusioni}
Nonostante questo capitolato abbia degli aspetti allettanti, non coincide con il percorso di crescita
che \textit{7Last} vuole intraprendere in questo progetto didattico.

\end{document}


























































































































































































































































