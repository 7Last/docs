\documentclass[italian,12pt]{article} %tipo di documento

%--------------variabili------------------%
\def\Title{Norme di Progetto}
\def\Author{7Last}
\def\Version{v0.2}
%-----------------------------------------%


\usepackage[left=2cm, right=2cm, bottom=3cm, top=3cm]{geometry}
\usepackage{fancyhdr}
\usepackage{graphicx}
\graphicspath{ {../../logo/} }
\usepackage{href-ul}
\usepackage{tikz}
\usepackage{tgadventor}
\usepackage[useregional=numeric,showseconds=true,showzone=false]{datetime2}
\usepackage{caption}
\usepackage{longtable}
\usepackage{xcolor}




\linespread{1.2}
\captionsetup[table]{labelformat=empty}

\renewcommand{\contentsname}{Indice}%imposto il nome dell'indice
\renewcommand\familydefault{\sfdefault}
%-------------------INIZIO DOCUMENTO--------------
\begin{document}

\newgeometry{left=2cm,right=2cm,bottom=2.1cm,top=2.1cm}
\begin{titlepage}
	\vspace*{.5cm}

	\vspace{2cm}
	{
		\centering
		{\bfseries\huge \Title\par}
		\bigbreak
		{\bfseries\Large \Subtitle\par}
		\bigbreak
		{\bfseries\large \Author\par}
		\bigbreak
		{\Date\;-\;\Version\par}
		\vfill

		\begin{center}
			\begin{tikzpicture}
				\clip (0,0) circle (2cm) node {\includegraphics[width=4cm]{logo.jpg}};
			\end{tikzpicture}
		\end{center}
	}

	\vfill

\end{titlepage}

\restoregeometry






















\newpage

\pagestyle{fancy}
\fancyhead{}
\lhead{
	\begin{tikzpicture}
		\clip (0,0) circle (0.5cm);
		\node at (0,0) {\includegraphics[width=1cm]{./../logo/logo.png}};
	\end{tikzpicture}%
}
\chead{\vspace{\fill}\Title\vspace{\fill}}
\rhead{\vspace{\fill}\Version\vspace{\fill}}




%-----------tabella versioni-----------%
\begin{table}[!h]
	\caption{Versioni}
	\begin{center}
		\begin{tabular}{ c c c c c c }
			\hline \\[-2ex]
			Ver. & Data & Autore & Descrizione \\
			\\[-2ex] \hline \\[-1.5ex]
			1.3 & 18/03/2024 & Elena Ferro & Revisione finale \\
			1.2 & 18/03/2024 & Matteo Tiozzo & Prima revisione \\
			1.1 & 16/03/2024 & Valerio Occhinegro& Sistemazione ortografica e di linguaggio \\
			1.0 & 15/03/2024 & Leonardo Baldo& Prima stesura del documento \\
			\\[-1.5ex] \hline
		\end{tabular}
	\end{center}
\end{table}
%---------------------------------------%
\newpage

\tableofcontents

\newpage

\section{Partecipanti}


\textbf{Sede della riunione}: piattaforma Teams\\
\textbf{Orario di inizio}: 14:30\\
\textbf{Orario di fine}: 15:00\\


\paragraph{Componenti di 7Last}

\begin{flushleft}
\begin{table}[!h]
\begin{tabular}{ |c|c|c| } 
	\hline
	\textbf{Partecipanti} & \textbf{Presenza} \\
	\hline 
	Leonardo Baldo 		 & 0:30 h \\ 
	Antonio Benetazzo 	 & 0:30 h \\
	Elena Ferro 		 & 0:30 h \\
	Davide Malgarise 	 & 0:30 h \\
	Valerio Occhinegro 	 & 0:30 h \\
	Raul Seganfreddo 	 & 0:30 h \\
	Matteo Tiozzo 		 & 0:30 h \\ 
	\hline
\end{tabular}
\end{table}
\end{flushleft}

\paragraph{Componenti di Imola Informatica}

\begin{itemize}
	\item Federico Bernacca
	\item Stefan Glamocak
\end{itemize}

\newpage

\begin{flushleft}
\section{Verbale dell'incontro}

\subsection{Specifiche del progetto}
	\textbf{Domanda}: è possibile avere maggior dettaglio sulle funzionalità obbligatorie dell'applicazione?\\
	L'applicazione dovrà permettere la prenotazione di un tavolo, l'ordinazione di un pasto (anche per soggetti non registrati) e la creazione di una admin page per i ristoranti, il tutto tramite web-app.\\
\subsection{Tecnologie utilizzate}
	\textbf{Domanda}: che tecnologie consiglia l'azienda per affrontare il progetto?\\
	L'azienda lascia libera scelta sulle tecnologie per lo sviluppo del progetto, tuttavia consiglia l'utilizzo di AngularJS e ReactJS. \\
\subsection{Struttura del backend}
	\textbf{Domanda}: verrà fornita una struttura per il backend o dovrà essere sviluppata?\\
	L'azienda richiede di occuparsi anche della progettazione e sviluppo del backend e della creazione di un mock per il processo di pagamento.\\

\subsection{Disponibilità e organizzazione Imola Informatica}
	\textbf{Domanda}: quale supporto fornirà l'azienda? Come saranno organizzate le riunioni e gli incontri?\\
	L’azienda offre un incontro settimanale dedicato alla formazione e all’affiancamento pratico del progetto nel caso in cui vi 
	siano dubbi o emergano problematiche; inoltre mette a disposizione un canale Telegram per gestire le comunicazioni, velocizzare le tempistiche e rendere più semplici gli scambi. Imola Informatica offre un supporto per 
	l’organizzazione e l’dentificazione dei ruoli necessari per la riuscita del lavoro e la frammentazione dello stesso; 
	sarà dunque utile fornire delle preferenze riguardanti il ruolo da investire, anche in base alle conoscenze personali 
	relative alla padronanza e conoscenza dei linguaggi.
	
\end{flushleft}


\vspace*{4cm}
\begin{table}[b]
\begin{tabular}{@{}p{.5in}p{4in}@{}}
	Data:  & \hrulefill \\
		   &     		\\
		   &     		\\
	Firma: & \hrulefill \\
\end{tabular}
\end{table}
	

\end{document}

























