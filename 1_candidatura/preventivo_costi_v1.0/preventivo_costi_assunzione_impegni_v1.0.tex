\documentclass[italian,12pt]{article} %tipo di documento

%--------------variabili------------------%
\def\Title{Norme di Progetto}
\def\Author{7Last}
\def\Version{v0.2}
%-----------------------------------------%


\usepackage[left=2cm, right=2cm, bottom=3cm, top=3cm]{geometry}
\usepackage{fancyhdr}
\usepackage{graphicx}
\graphicspath{ {../../logo/} }
\usepackage{href-ul}
\usepackage{tikz}
\usepackage{tgadventor}
\usepackage[useregional=numeric,showseconds=true,showzone=false]{datetime2}
\usepackage{caption}
\usepackage{longtable}
\usepackage{xcolor}




\linespread{1.2}
\captionsetup[table]{labelformat=empty}

\renewcommand{\contentsname}{Indice}%imposto il nome dell'indice
\renewcommand\familydefault{\sfdefault}

%-------------------INIZIO DOCUMENTO--------------
\begin{document}

\newgeometry{left=2cm,right=2cm,bottom=2.1cm,top=2.1cm}
\begin{titlepage}
	\vspace*{.5cm}

	\vspace{2cm}
	{
		\centering
		{\bfseries\huge \Title\par}
		\bigbreak
		{\bfseries\Large \Subtitle\par}
		\bigbreak
		{\bfseries\large \Author\par}
		\bigbreak
		{\Date\;-\;\Version\par}
		\vfill

		\begin{center}
			\begin{tikzpicture}
				\clip (0,0) circle (2cm) node {\includegraphics[width=4cm]{logo.jpg}};
			\end{tikzpicture}
		\end{center}
	}

	\vfill

\end{titlepage}

\restoregeometry






















\newpage

\pagestyle{fancy}
\fancyhead{}
\lhead{
	\begin{tikzpicture}
		\clip (0,0) circle (0.5cm);
		\node at (0,0) {\includegraphics[width=1cm]{./../logo/logo.png}};
	\end{tikzpicture}%
}
\chead{\vspace{\fill}\Title\vspace{\fill}}
\rhead{\vspace{\fill}\Version\vspace{\fill}}


%-----------tabella revisioni-----------%
\begin{table}[!h]
	\caption{Versioni}
	\begin{center}
		\begin{tabular}{ c c c p{9cm} }
			\hline                                                                                      \\[-2ex]
			Ver. & Data       & Autore           & Descrizione                                          \\
			\\[-2ex] \hline \\[-1.5ex]
			1.1  & 25/03/2024 & Antonio Benetazzo & Aggiunta pianificazione mancante                    \\
			1.0  & 20/03/2024 & Davide Malgarise  & Approvazione documento                              \\
			0.5  & 19/03/2024 & Matteo Tiozzo     & Prima revisione                                     \\
			0.4  & 19/03/2024 & Raul Seganfreddo  & Aggiunto Costo orario e Ripartizione ore per membro \\
			0.3  & 19/03/2024 & Elena Ferro       & Aggiunta analisi dei rischi                         \\
			0.2  & 19/03/2024 & Raul Seganfreddo  & Correzioni sintattiche                              \\
			0.1  & 18/03/2024 & Raul Seganfreddo  & Creazione Documento                                 \\
			\\[-1.5ex] \hline
		\end{tabular}
	\end{center}
\end{table}
%---------------------------------------%

\newpage

\tableofcontents

\newpage

\section{Analisi dei rischi}
In questa sezione verranno elencati i rischi che potrebbero verificarsi durante lo svolgimento del progetto e le relative contromisure.
A ciascuno verrà assegnato un \textbf{indice di gravità} e \textbf{probabilità} che si verifichi, in modo da poter valutare la criticità di ciascuno di essi.

\paragraph{Indice di gravità} Il valore dell'indice di gravità è definito come segue:
\begin{itemize}
	\itemsep0em
	\item \textbf{1}: Basso
	\item \textbf{2}: Medio
	\item \textbf{3}: Alto
\end{itemize}

\paragraph{Probabilità} Il valore della probabilità che si verifichi un rischio è definito come segue:
\begin{itemize}
	\itemsep0em
	\item \textbf{1}: Bassa
	\item \textbf{2}: Media
	\item \textbf{3}: Alta
\end{itemize}

\subsection{Difficoltà nell'apprendimento delle tecnologie}
% descrizione
Possibilità di incontrare difficoltà nell'apprendimento delle tecnologie utilizzate dalla proponente.
\begin{itemize}
	\item \textbf{Indice di gravità}: 2
	\item \textbf{Probabilità}: 2
	\item \textbf{Contromisure}: il gruppo si impegnerà a studiare le tecnologie in anticipo rispetto alle attività previste.
	      Inoltre, verranno organizzati dei momenti di formazione interna per condividere le conoscenze acquisite.
	      Nel caso in cui le difficoltà persistano, la proponente verrà contattata per ricevere supporto.
\end{itemize}

\subsection{Scarsa collaborazione da parte di uno o più membri}
\begin{itemize}
	\item \textbf{Indice di gravità}: 3
	\item \textbf{Probabilità}: 1
	\item \textbf{Contromisure}: comunicazione costante e trasparente sulle
	      difficoltà incontrate e segnalazione tempestiva di eventuali problemi.
\end{itemize}

\subsection{Impegni personali o universitari}
\begin{itemize}
	\item \textbf{Indice di gravità}: 1
	\item \textbf{Probabilità}: 2
	\item \textbf{Contromisure}: Comunicare con anticipo eventuali impegni personali che potrebbero interferire con le attività del gruppo.
	      Di conseguenza, pianificare i \textit{task} in modo da evitare sovrapposizioni.
\end{itemize}

\subsection{Disaccordi all'interno del gruppo}
\begin{itemize}
	\item \textbf{Indice di gravità}: 3
	\item \textbf{Probabilità}: 2
	\item \textbf{Contromisure}:
	      Cercare di mantenere un clima di collaborazione e rispetto reciproco. Nel caso di conflitti,
	      ciascun membro del gruppo potrà proporre una soluzione, evidenziandone i punti di forza.
	      Si utilizzerà poi un sistema di voto basato sulla maggioranza.
\end{itemize}

\subsection{Deviazione rispetto ai tempi e costi previsti}
\begin{itemize}
	\item \textbf{Indice di gravità}: 3
	\item \textbf{Probabilità}: 1
	\item \textbf{Contromisure}: Monitorare costantemente il progresso delle attività, svolgere frequenti riunioni per valutare lo stato di avanzamento del progetto.
	      Nella pianificazione delle attività, verranno inseriti dei \textit{buffer} temporali per far fronte a eventuali ritardi.

\end{itemize}

\section{Analisi dei ruoli}
\subsection{Responsabile}
Il responsabile ha il compito di coordinare il gruppo di lavoro, pianificare e controllare le attività e gesire le risorse.
In poche parole, è colui che si occupa di garantire che il progetto venga portato a termine nel rispetto dei tempi e delle risorse disponibili.\\
Secondo la nostra analisi di progetto, il responsabile necessita di un numero di ore inferiore alla media rispetto agli altri ruoli, in quanto utilizza meno tempo per completare le attività assegnate e inoltre,
il costo orario del responsabile è dispendioso.

\subsection{Amministratore}
L'amministratore è colui che si occupa della gestione delle risorse e delle infrastrutture, inclusa la configurazione degli strumenti di supporto alla produzione del software.
Inoltre, si occupa della gestione del corretto utilizzo delle procedure andando a garantire un'elevata efficienza e produttività del gruppo di lavoro.\\
Come per il responsabile l'amministratore necessita di un numero di ore inferiore alla media rispetto agli altri ruoli, in quanto utilizza meno tempo per completare le attività assegnate.

\subsection{Analista}
L'analista è un ruolo fondamentale sopratutto nella fase iniziale del progetto.
Egli ha il compito di analizzare quella che dovrebbe essere la funzionalità del software, andando a definire i requisiti e i casi d'uso.\\
Il numero di ore di lavoro dell'analista è inferiore rispetto agli altri ruoli, in quanto la sua figura sarà necessaria solo nella fase iniziale del progetto.

\subsection{Progettista}
Il progettista è colui che si occupa di definire l'architettura del software, andando a definire le componenti e le relazioni tra di esse, basandosi sui requisiti definiti dall'analista.\\
Il numero di ore di lavoro del progettista è superiore rispetto alla media degli altri ruoli, in quanto comporta un elevato dispendio temporale.

\subsection{Programmatore}
Il programmatore è colui che si occupa di scrivere il codice del software, basandosi sulle specifiche definite dal progettista.\\
Riguardo al numero di ore necessarie per completare le attività assegnate, il programmatore necessita di un numero di ore superiore alla media rispetto agli altri ruoli,
in quanto utilizza più tempo per completare le attività assegnate rispetto agli altri ruoli.

\subsection{Verificatore}
Il verificatore è colui che si occupa di verificare che il software prodotto e la documentazione siano conformi alle norme e alle specifiche definite.\\
Proprio per questo motivo, il verificatore necessita di un numero di ore superiore alla media rispetto agli altri ruoli, in quanto la sua figura sarà necessaria per tutta la durata del progetto.

\newpage
\section{Impegni orari}
Ogni componente del gruppo 7Last si impegna a dedicare \textbf{92 ore} allo svolgiemento del capitolato \textbf{Sync City} e a ricoprire ciascun ruolo per minimo di ore in modo da garantire un'equa distribuzione del lavoro.

\subsection{Costo orario}
Il costo orario di ciascun ruolo è definito come segue:

\begin{table}[!h]
	\begin{center}
		\begin{tabular}{ |c|c|c|c| }
			\hline
			Ruolo          & Costo orario & Ore per ruolo & Ore per membro \\
			\hline
			Responsabile   & 30           & 56            & 8              \\
			Amministratore & 20           & 56            & 8              \\
			Analista       & 25           & 77            & 11             \\
			Progettista    & 25           & 112           & 16             \\
			Programmatore  & 15           & 168           & 24             \\
			Verificatore   & 15           & 175           & 25             \\
			\hline
			Totale         & €12670       & 644           & 92             \\
			\hline
		\end{tabular}
	\end{center}
\end{table}

\newpage
\subsection{Ripatrizione ore per membro}
Di seguito è riportata la ripartizione delle ore per ciascun membro del gruppo.

\begin{table}[!h]
	\begin{center}
		\begin{tabular}{ |c|c|c|c|c|c|c|c| }
			\hline
			\textbf{Membro}    & \textbf{Re} & \textbf{Am} & \textbf{An} & \textbf{Pj} & \textbf{Pg} & \textbf{Ve} & \textbf{Totale} \\
			\hline
			Leonardo Baldo     & 8           & 8           & 11          & 18          & 22          & 25          & 92              \\
			Antonio Benetazzo  & 8           & 9           & 10          & 17          & 25          & 23          & 92              \\
			Elena Ferro        & 9           & 7           & 11          & 15          & 24          & 26          & 92              \\
			Davide Malgarise   & 8           & 10          & 10          & 14          & 26          & 24          & 92              \\
			Valerio Occhinegro & 7           & 8           & 12          & 16          & 25          & 24          & 92              \\
			Raul Seganfreddo   & 9           & 6           & 12          & 16          & 25          & 24          & 92              \\
			Matteo Tiozzo      & 7           & 8           & 12          & 16          & 21          & 28          & 92              \\
			\hline
		\end{tabular}
	\end{center}
\end{table}

Dove i ruoli sono abbreviati come segue:
\begin{itemize}
	\item \textbf{Re}: Responsabile
	\item \textbf{Am}: Amministratore
	\item \textbf{An}: Analista
	\item \textbf{Pj}: Progettista
	\item \textbf{Pg}: Programmatore
	\item \textbf{Ve}: Verificatore
\end{itemize}

\section{Costo totale preventivato}
Il costo totale del progetto, visto quanto è stato definito in precedenza, è di \textbf{€12.670,00}.

\newpage

\section{Scadenza di consegna}
\textit{7Last} si impegna a consegnare il progetto entro il \textbf{24 Settembre 2024}. \\
Sulla base dell'analisi fatta nel presente documento, il tempo totale preventivato di 24 settimane prevediamo verrà suddiviso come segue:
\begin{itemize}
    \item \textbf{6 settimane per lo sviluppo del \textit{PoC (Proof of Concept)}:} durante questa fase provvederemo alla ricerca di tecnologie che, assieme a quelle proposte dall'azienda, potranno esserci utili allo sviluppo di quanto richiesto dal capitolato scelto; una volta individuate provvederemo al loro studio per poterle utilizzare al meglio.
    \item \textbf{18 settimane per lo sviluppo del \textit{MVP (Minimum Viable Product)}:} questa fase sarà dedicata allo sviluppo di un prodotto funzionante e che rispetti i requisiti minimi che verranno individuati nel documento \textit{Analisi dei requisiti} richiesto dal \textit{PoC}.
\end{itemize}

\end{document}
