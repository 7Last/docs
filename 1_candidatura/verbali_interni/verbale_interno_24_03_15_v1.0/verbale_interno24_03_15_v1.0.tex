\documentclass[italian,12pt]{article} %tipo di documento

%--------------variabili------------------%
\def\Title{Norme di Progetto}
\def\Author{7Last}
\def\Version{v0.2}
%-----------------------------------------%


\usepackage[left=2cm, right=2cm, bottom=3cm, top=3cm]{geometry}
\usepackage{fancyhdr}
\usepackage{graphicx}
\graphicspath{ {../../logo/} }
\usepackage{href-ul}
\usepackage{tikz}
\usepackage{tgadventor}
\usepackage[useregional=numeric,showseconds=true,showzone=false]{datetime2}
\usepackage{caption}
\usepackage{longtable}
\usepackage{xcolor}




\linespread{1.2}
\captionsetup[table]{labelformat=empty}

\renewcommand{\contentsname}{Indice}%imposto il nome dell'indice
\renewcommand\familydefault{\sfdefault}
%-------------------INIZIO DOCUMENTO--------------
\begin{document}

\newgeometry{left=2cm,right=2cm,bottom=2.1cm,top=2.1cm}
\begin{titlepage}
	\vspace*{.5cm}

	\vspace{2cm}
	{
		\centering
		{\bfseries\huge \Title\par}
		\bigbreak
		{\bfseries\Large \Subtitle\par}
		\bigbreak
		{\bfseries\large \Author\par}
		\bigbreak
		{\Date\;-\;\Version\par}
		\vfill

		\begin{center}
			\begin{tikzpicture}
				\clip (0,0) circle (2cm) node {\includegraphics[width=4cm]{logo.jpg}};
			\end{tikzpicture}
		\end{center}
	}

	\vfill

\end{titlepage}

\restoregeometry






















\newpage

\pagestyle{fancy}
\fancyhead{}
\lhead{
	\begin{tikzpicture}
		\clip (0,0) circle (0.5cm);
		\node at (0,0) {\includegraphics[width=1cm]{./../logo/logo.png}};
	\end{tikzpicture}%
}
\chead{\vspace{\fill}\Title\vspace{\fill}}
\rhead{\vspace{\fill}\Version\vspace{\fill}}




%-----------tabella versioni-----------%
\begin{table}[!h]
	\caption{Versioni}
	\begin{center}
		\begin{tabular}{ c c c p{9cm}}
			\hline \\[-2ex]
			Ver. & Data & Autore & Descrizione \\
			\\[-2ex] \hline \\[-1.5ex]
			1.0 & 20/03/2024 & Antonio Benetazzo & Approvazione documento \\
			0.3 & 19/03/2024 & Matteo Tiozzo & Prima revisione \\
			0.2 & 16/03/2024 & Valerio Occhinegro& Controllo errori grammaticali e di sintassi  \\
			0.1 & 15/03/2024 & Leonardo Baldo& Prima stesura del documento \\
			\\[-1.5ex] \hline
		\end{tabular}
	\end{center}
\end{table}
%---------------------------------------%
\newpage

\tableofcontents

\newpage

\section{Registro presenze}


\textbf{Sede della riunione}: Piattaforma Discord\\
\textbf{Orario di inizio}: 15:00\\
\textbf{Orario di fine}: 16:00\\

\begin{flushleft}
\begin{table}[!h]
\begin{tabular}{ |c|c|c| } 
	\hline
	\textbf{Componente} & \textbf{Ruolo} & \textbf{Presenza} \\
	\hline 
	Leonardo Baldo 		& Redattore & Presente \\ 
	Antonio Benetazzo 	& Responsabile & Presente \\
	Elena Ferro 		& Verificatore & Presente \\
	Valerio Occhinegro 	& Redattore & Presente \\
	Raul Seganfreddo 	& Amministratore & Presente \\
	Matteo Tiozzo 		& Verificatore & Presente \\ 
	Davide Malgarise 	& Redattore & Assente \\
	\hline
\end{tabular}
\end{table}
Solo per questa riunione, il ruolo di Davide Malgarise è stato assunto da Antonio Benetazzo.
\end{flushleft}

\section{Ordine del giorno}
\subsection{Discussione sul way of working}
\begin{flushleft}
	\begin{itemize}
		\item Scelta finale del capitolato
		\item Organizzazione del repository
		\item Workflow
		\item Organizzazione interna
		\item Adozione template LaTex
	\end{itemize}
\end{flushleft}


\newpage

\section{Verbale}
\subsection{Scelta finale del capitolato}
\begin{flushleft}
	Il gruppo sceglie definitivamente il capitolato proposto da Sync Lab S.r.l. e inizia la preparazione della candidatura per tale capitolato.
\end{flushleft}
\subsection{Organizzazione del repository}
\begin{flushleft}
	Antonio Benetazzo crea il repository \href{https://github.com/7Last/docs}{ \textbf{\textit{docs}}} su GitHub.
	Il gruppo decide di suddividere quest'ultima in tre cartelle principali: 
	\begin{itemize}
		\item \textbf{Candidatura}: contenente tutta la documentazione relativa alla candidatura per il capitolato scelto e, al suo interno, le cartelle:
		\begin{itemize}
			\item \textbf{verbali\ interni}: contenente i verbali delle riunioni interne (tra i membri del gruppo)
			\item \textbf{verbali\ esterni}: contenente i verbali delle riunioni esterne (con le aziende proponenti)
			\end{itemize}
		\item \textbf{RTB}: che verrà sviluppata durante lo svolgimento del progetto e conterrà la documentazione relativa a questa fase
		\item \textbf{PB}: che verrà sviluppata durante lo svolgimento del progetto e conterrà la documentazione relativa a questa fase
	\end{itemize}
\end{flushleft}
\subsection{Workflow}
\begin{flushleft}
	Viene scelto di utilizzare il modello di branching GitFlow per la gestione del workflow del progetto.
\end{flushleft}
\subsection{Organizzazione interna}
\begin{flushleft}
	Vengono decisi i metodi di comunicazione interna del gruppo:
	\begin{itemize}
		\item \textbf{Telegram}: per la comunicazione tramite messaggi
		\item \textbf{Discord}: per la comunicazione tramite chiamate
	\end{itemize}
\end{flushleft}
\subsection{Adozione template LaTex}
\begin{flushleft}
	Viene scelto di modificare il template LaTex per avere una migliore personalizzazione dei documenti. Il componente Matteo Tiozzo si assume l'incarico di progettare il nuovo template. 
\end{flushleft}
\subsection{Creazione pagina web per documentazione}
\begin{flushleft}
	Per facilitare la visione dei documenti del repository viene deciso di creare una pagina web. Il componente Elena Ferro si assume l'incarico di progettare e realizzare la pagina web. Tramite GitHub Actions verrà automatizzato l'aggiornamento dei contenuti della pagina.
\end{flushleft}

\end{document}

























