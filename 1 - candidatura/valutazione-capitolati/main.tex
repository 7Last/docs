\documentclass[italian,12pt]{article} %tipo di documento

%--------------variabili------------------%
\def\Title{Norme di Progetto}
\def\Author{7Last}
\def\Version{v0.2}
%-----------------------------------------%


\usepackage[left=2cm, right=2cm, bottom=3cm, top=3cm]{geometry}
\usepackage{fancyhdr}
\usepackage{graphicx}
\graphicspath{ {../../logo/} }
\usepackage{href-ul}
\usepackage{tikz}
\usepackage{tgadventor}
\usepackage[useregional=numeric,showseconds=true,showzone=false]{datetime2}
\usepackage{caption}
\usepackage{longtable}
\usepackage{xcolor}




\linespread{1.2}
\captionsetup[table]{labelformat=empty}

\renewcommand{\contentsname}{Indice}%imposto il nome dell'indice
\renewcommand\familydefault{\sfdefault}

%-------------------INIZIO DOCUMENTO--------------
\begin{document}

\newgeometry{left=2cm,right=2cm,bottom=2.1cm,top=2.1cm}
\begin{titlepage}
	\vspace*{.5cm}

	\vspace{2cm}
	{
		\centering
		{\bfseries\huge \Title\par}
		\bigbreak
		{\bfseries\Large \Subtitle\par}
		\bigbreak
		{\bfseries\large \Author\par}
		\bigbreak
		{\Date\;-\;\Version\par}
		\vfill

		\begin{center}
			\begin{tikzpicture}
				\clip (0,0) circle (2cm) node {\includegraphics[width=4cm]{logo.jpg}};
			\end{tikzpicture}
		\end{center}
	}

	\vfill

\end{titlepage}

\restoregeometry






















\newpage

\pagestyle{fancy}
\fancyhead{}
\lhead{
	\begin{tikzpicture}
		\clip (0,0) circle (0.5cm);
		\node at (0,0) {\includegraphics[width=1cm]{./../logo/logo.png}};
	\end{tikzpicture}%
}
\chead{\vspace{\fill}\Title\vspace{\fill}}
\rhead{\vspace{\fill}\Version\vspace{\fill}}


%-----------tabella revisioni-----------%
\begin{table}[!h]
	\caption{Versioni}
	\begin{center}
		\begin{tabular}{ c c c c c c }
			\hline \\[-2ex]
			Ver. & Data & Autore & Descrizione \\
			\\[-2ex] \hline \\[-1.5ex]
			1.0  & 17/03/2024 & Leonardo & Creazione documento \\
			1.1 & 17/03/2024 & Leonardo & Aggiunta Introduzione e descrizione Capitolati \\

			\\[-1.5ex] \hline
		\end{tabular}
	\end{center}
\end{table}
%---------------------------------------%

\newpage

\tableofcontents

\newpage

\section{Introduzione}
Lo scopo del presente documento è quello di rappresentare un'analisi dei Capitolati esposti dalle aziende proponenti. Verranna analizzate le motivazioni della scelta del Capitolato C6, e gli aspetti positivi e negativi che stanno alla base della scelta di scartare le altre proposte.



\section{Valutazione del Capitolato Selezionato}

\subsection{Capitolato C6 - SyncCity}

\subsubsection{Descrizione}
\begin{itemize}
	\item Nome: SyncCity: Smart city monitoring platform
	\item Proponente: Synclab
	\item Committenti: {\it Prof. Tullio Vardanega, Prof. Riccardo Cardin}
	\item Obiettivo: Lo scopo di questo progetto è di realizzare una piattaforma e relativa dashboard, che consenta a chi la usa di avere sotto controllo una serie di informazioni sullo stato di salute della città, in modo da prendere decisioni veloci, efficaci ed analizzare poi gli effetti conseguenti.
\end{itemize}

\subsubsection{Domini}
\paragraph{Dominio applicativo}
...
\paragraph{Dominio tecnologico}
...

\subsubsection{Valutazione}
\paragraph{Aspetti positivi}
...
\paragraph{Aspetti negativi}
...

\subsubsection{Conclusioni}
...



\section{Valutazione degli altri Capitolati}

\subsection{Capitolato C9 - ChatSQL}

\subsubsection{Descrizione}
\begin{itemize}
	\item Nome: ChatSQL: creare frasi SQL da linguaggio naturale
	\item Proponente: Zucchetti
	\item Committenti: {\it Prof. Tullio Vardanega, Prof. Riccardo Cardin}
	\item Obiettivo: Nel capitolato si propone la realizzazione di un chatbot per la generazione di query SQL a partire da una frase in linguaggio naturale e la struttura del database.
\end{itemize}

\subsubsection{Domini}
\paragraph{Dominio applicativo}
...
\paragraph{Dominio tecnologico}
...

\subsubsection{Valutazione}
\paragraph{Aspetti positivi}
...
\paragraph{Aspetti negativi}
...

\subsubsection{Conclusioni}
...



\subsection{Capitolato C3 - Easy meal}

\subsubsection{Descrizione}
\begin{itemize}
	\item Nome: Easy meal
	\item Proponente: Imola informatica
	\item ommittenti: {\it Prof. Tullio Vardanega, Prof. Riccardo Cardin}
	\item Obiettivo: II capitolato intende semplificare il processo di prenotazione, permettendo ai clienti di riservare un tavolo in Inodo rapido e intuitivo, consentendo ai clienti di interagire direttamente con il personale del ristorante in modo da migliorare la comodità, la personalizzazione e l'efficienza, delle prenotazioni nei ristoranti.
\end{itemize}

\subsubsection{Domini}
\paragraph{Dominio applicativo}
...
\paragraph{Dominio tecnologico}
...

\subsubsection{Valutazione}
\paragraph{Aspetti positivi}
...
\paragraph{Aspetti negativi}
...

\subsubsection{Conclusioni}
...

\end{document}

























