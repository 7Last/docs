\documentclass[italian,12pt]{article} %tipo di documento

%--------------variabili------------------%
\def\Title{Norme di Progetto}
\def\Author{7Last}
\def\Version{v0.2}
%-----------------------------------------%


\usepackage[left=2cm, right=2cm, bottom=3cm, top=3cm]{geometry}
\usepackage{fancyhdr}
\usepackage{graphicx}
\graphicspath{ {../../logo/} }
\usepackage{href-ul}
\usepackage{tikz}
\usepackage{tgadventor}
\usepackage[useregional=numeric,showseconds=true,showzone=false]{datetime2}
\usepackage{caption}
\usepackage{longtable}
\usepackage{xcolor}




\linespread{1.2}
\captionsetup[table]{labelformat=empty}

\renewcommand{\contentsname}{Indice}%imposto il nome dell'indice
\renewcommand\familydefault{\sfdefault}

%-------------------INIZIO DOCUMENTO--------------
\begin{document}

\newgeometry{left=2cm,right=2cm,bottom=2.1cm,top=2.1cm}
\begin{titlepage}
	\vspace*{.5cm}

	\vspace{2cm}
	{
		\centering
		{\bfseries\huge \Title\par}
		\bigbreak
		{\bfseries\Large \Subtitle\par}
		\bigbreak
		{\bfseries\large \Author\par}
		\bigbreak
		{\Date\;-\;\Version\par}
		\vfill

		\begin{center}
			\begin{tikzpicture}
				\clip (0,0) circle (2cm) node {\includegraphics[width=4cm]{logo.jpg}};
			\end{tikzpicture}
		\end{center}
	}

	\vfill

\end{titlepage}

\restoregeometry






















\newpage

\pagestyle{fancy}
\fancyhead{}
\lhead{
	\begin{tikzpicture}
		\clip (0,0) circle (0.5cm);
		\node at (0,0) {\includegraphics[width=1cm]{./../logo/logo.png}};
	\end{tikzpicture}%
}
\chead{\vspace{\fill}\Title\vspace{\fill}}
\rhead{\vspace{\fill}\Version\vspace{\fill}}


%-----------tabella revisioni-----------%
\begin{table}[!h]
	\caption{Versioni}
	\begin{center}
		\begin{tabular}{ c c c c c c }
			\hline \\[-2ex]
			Ver. & Data & Autore & Descrizione \\
			\\[-2ex] \hline \\[-1.5ex]
			1.0  & 17/03/2024 & Leonardo Baldo & Creazione documento \\
			1.1 & 17/03/2024 & Leonardo Baldo & Aggiunta Introduzione e descrizione Capitolati \\
			1.2 & 18/03/2024 & Leonardo Baldo & Aggiunti domini \\

			\\[-1.5ex] \hline
		\end{tabular}
	\end{center}
\end{table}
%---------------------------------------%

\newpage

\tableofcontents

\newpage

\section{Introduzione}
Lo scopo del presente documento è quello di rappresentare un'analisi dei Capitolati esposti dalle aziende proponenti. Verranna analizzate le motivazioni della scelta del Capitolato C6, e gli aspetti positivi e negativi che stanno alla base della scelta di scartare le altre proposte.



\section{Valutazione del Capitolato Selezionato}

\subsection{Capitolato C6 - SyncCity}

\subsubsection{Descrizione}
\begin{itemize}
	\item Nome: SyncCity: Smart city monitoring platform
	\item Proponente: Synclab
	\item Committenti: {\it Prof. Tullio Vardanega, Prof. Riccardo Cardin}
	\item Obiettivo: Lo scopo di questo progetto è di realizzare una piattaforma e relativa dashboard, che consenta a chi la usa di avere sotto controllo una serie di informazioni sullo stato di salute della città, in modo da prendere decisioni veloci, efficaci ed analizzare poi gli effetti conseguenti.
\end{itemize}

\subsubsection{Domini}
\paragraph{Dominio applicativo}
Queste tecnologie possiedono la capacità di elaborare, migliorare e archiviare i dati grezzi raccolti dai sensori. Il focus principale di questa area è garantire un ambiente di simulazione affidabile e una chiara rappresentazione dei dati sensoriali, permettendo agli utenti di monitorare e analizzare tali informazioni in modo efficace per diagnosi, decisioni e informazioni.
\paragraph{Dominio tecnologico}
La proponente consiglia fortemente l'utilizzo delle seguenti tecnologie per lo svolgimento del progetto:
\begin{itemize}
	\item Python: per la simulazione dei dati quanto più possibile realistica attraverso script, ed eventualmente librerie di generazione dati (faker).
	\item Apache Kafka: broker per disaccoppiare lo stream di informazioni provenienti dai simulatori, ormai definito come tool standard per gestire il gathering dei dati da più fonti.
	\item ClickHouse: database OLAP colonnari, questa componente avrà il compito di mantenere la persistenza di dati con numerosità elevata. La sua integrazione con Kafka ne facilità l'utilizzo nel progetto.
	\item Grafana: piattaforma di data visualization delle informazioni, questa componente rappresenta il front-end dell'utente, la finestra che consentirà il monitoraggio della città.
\end{itemize}

\subsubsection{Valutazione}
\paragraph{Aspetti positivi}
...
\paragraph{Aspetti negativi}
...

\subsubsection{Conclusioni}
...



\section{Valutazione degli altri Capitolati}

\subsection{Capitolato C9 - ChatSQL}

\subsubsection{Descrizione}
\begin{itemize}
	\item Nome: ChatSQL: creare frasi SQL da linguaggio naturale
	\item Proponente: Zucchetti
	\item Committenti: {\it Prof. Tullio Vardanega, Prof. Riccardo Cardin}
	\item Obiettivo: Nel capitolato si propone la realizzazione di un chatbot per la generazione di query SQL a partire da una frase in linguaggio naturale e la struttura del database.
\end{itemize}

\subsubsection{Domini}
\paragraph{Dominio applicativo}
Il progetto propone di affrontare un aspetto cruciale nel contesto dei colloqui di lavoro: fornire agli intervistatori uno strumento avanzato per semplificare il processo di valutazione delle competenze dei candidati.
L'obiettivo di questa iniziativa è creare un ambiente di intervista più strutturato ed efficiente, dove gli intervistatori possano fare affidamento su un assistente virtuale per generare domande rilevanti e valutare le risposte in modo obiettivo. Tale sistema mira a ridurre il rischio di errori umani nelle valutazioni, assicurando una valutazione più equa e basata su dati oggettivi.
L'azienda chiede di sviluppare una applicazione che svolga i seguenti compiti:
\begin{itemize}
	\item Archiviazione della descrizione della struttura di un database, possibilmente commentata in tutte le sue parti.
	\item Maschera di richiesta di una frase di interrogazione del database in linguaggio naturale.
	\item Procedura che combina la richiesta di interrogazione con le informazioni della struttura del database creando un “prompt” che sottoposto ad un sistema di AI fornisce l’interrogazione equivalente al linguaggio naturale in linguaggio SQL.
	\item Il tutto integrato in un unico sistema che permetta di utilizzarli in modo integrato.
\end{itemize}
\paragraph{Dominio tecnologico}
Non sono presetni tecnologie obbligatorie, ma la proponente consiglia le seguenti:
\begin{itemize}
	\item Python
	\item interfaccia HTML, JS, CSS
\end{itemize}

\subsubsection{Valutazione}
\paragraph{Aspetti positivi}
...
\paragraph{Aspetti negativi}
...

\subsubsection{Conclusioni}
...



\subsection{Capitolato C3 - Easy meal}

\subsubsection{Descrizione}
\begin{itemize}
	\item Nome: Easy meal
	\item Proponente: Imola informatica
	\item ommittenti: {\it Prof. Tullio Vardanega, Prof. Riccardo Cardin}
	\item Obiettivo: II capitolato intende semplificare il processo di prenotazione, permettendo ai clienti di riservare un tavolo in Inodo rapido e intuitivo, consentendo ai clienti di interagire direttamente con il personale del ristorante in modo da migliorare la comodità, la personalizzazione e l'efficienza, delle prenotazioni nei ristoranti.
\end{itemize}

\subsubsection{Domini}
\paragraph{Dominio applicativo}
L'obiettivo principale di questo progetto è eendere più semplice i porccesso dalla prenotazione al pagamento di un pasto, tramite un'applicazione web responsive. Questo porta ache ad una diminuzione gli sprechi alimentari, tema molto importante al giorno d'oggi. \\
Nello specifico, in questo progetto si prendono in considerazione i seguenti casi di studio per individuare le operazioni che vengono effettuate dai clienti o dai ristoranti:
\begin{itemize}
	\item Registrazione di nuovo utente
	\item Prenotazione di un tavolo
	\item Ordinazione collaborativa dei pasti
	\item Interazione con lo staff del ristorante
	\item Divisione del conto
	\item Consultazione delle prenotazioni da parte di un amministratore del ristorante
	\item Inserimento di feedback e recensioni
\end{itemize}
\paragraph{Dominio tecnologico}
L'azienda proponente richiede l'implementazione di un applicazione \textit{web responsive} (PC, IOS e Android).
Ci viene lasciata totale libertà implementativa, tuttavia sono stati consigliati alcune tecnologie:
\begin{itemize}
	\item HTML, CSS, JavaScript, PHP
    \item Framework per applicazioni web: React, Angular.
\end{itemize}

\subsubsection{Valutazione}
\paragraph{Aspetti positivi}
...
\paragraph{Aspetti negativi}
...

\subsubsection{Conclusioni}
...

\end{document}

























