\documentclass[italian,12pt]{article}

%--------------variabili------------------%
\def\Title{Norme di Progetto}
\def\Author{7Last}
\def\Version{v0.2}
%-----------------------------------------%


\usepackage[left=2cm, right=2cm, bottom=3cm, top=3cm]{geometry}
\usepackage{fancyhdr}
\usepackage{graphicx}
\graphicspath{ {../../logo/} }
\usepackage{href-ul}
\usepackage{tikz}
\usepackage{tgadventor}
\usepackage[useregional=numeric,showseconds=true,showzone=false]{datetime2}
\usepackage{caption}
\usepackage{longtable}
\usepackage{xcolor}




\linespread{1.2}
\captionsetup[table]{labelformat=empty}
\geometry{headsep=1.5cm}

\renewcommand{\contentsname}{Indice}
\renewcommand\familydefault{\sfdefault}

\let\oldthepage\thepage
\renewcommand{\thepage}{\sffamily\oldthepage}


\begin{document}

\newcommand{\mySkip}[1][]{#1}

\newgeometry{left=2cm,right=2cm,bottom=2.1cm,top=2.1cm}
\begin{titlepage}
	\vspace*{.5cm}

	\vspace{2cm}
	{
		\centering
		{\bfseries\huge \Title\par}
		\bigbreak
		{\bfseries\Large \Subtitle\par}
		\bigbreak
		{\bfseries\large \Author\par}
		\bigbreak
		{\Date\;-\;\Version\par}
		\vfill

		\begin{center}
			\begin{tikzpicture}
				\clip (0,0) circle (2cm) node {\includegraphics[width=4cm]{logo.jpg}};
			\end{tikzpicture}
		\end{center}
	}

	\vfill

\end{titlepage}

\restoregeometry






















\newpage

\pagestyle{fancy}
\fancyhead{}
\lhead{
	\begin{tikzpicture}
		\clip (0,0) circle (0.5cm);
		\node at (0,0) {\includegraphics[width=1cm]{./../logo/logo.png}};
	\end{tikzpicture}%
}
\chead{\vspace{\fill}\Title\vspace{\fill}}
\rhead{\vspace{\fill}\Version\vspace{\fill}}


\begin{table}[!h]
	\caption{Versioni}
	\footnotesize
	\begin{center}
		\begin{tabular}{ l l l l p{6cm} }
			\hline                                                                           \\[-2ex]
			Ver. & Data       & Redattore      & Verificatore & Descrizione     \\
			\\[-2ex] \hline \\[-1.5ex]
			1.0  & 2024-06-09 & Leonardo Baldo & Elena Ferro  & Stesura verbale \\
			\\[-1.5ex] \hline
		\end{tabular}
	\end{center}
\end{table}
\newpage

\tableofcontents

\newpage

\section{Dettagli della riunione}


\textbf{Sede della riunione}: Piattaforma Discord\\
\textbf{Orario di inizio}: 17:00\\
\textbf{Orario di fine}: 18:00\\


\begin{flushleft}
	\begin{table}[!h]
		\begin{tabular}{ |l|l|l| }
			\hline
			\textbf{Partecipante} & \textbf{Ruolo} & \textbf{Presenza} \\
			\hline
			Antonio Benetazzo     &                & Presente          \\
			Davide Malgarise      & Amministratore & Presente          \\
			Elena Ferro           & Verificatore   & Presente          \\
			Leonardo Baldo        & Redattore      & Presente          \\
			Matteo Tiozzo         &                & Presente          \\
			Raul Seganfreddo      &                & Presente          \\
			Valerio Occhinegro    &                & Presente          \\
			\hline
		\end{tabular}
	\end{table}
\end{flushleft}

\raggedright
\textbf{Ordine del giorno}:
\begin{itemize}
	\item Suddivisione dei ruoli per il settimo sprint;
	\item Cosa è stato fatto fino ad ora;
	\item Attività da svolgere per il settimo sprint;
	\item Varie ed eventuali;
	\item Decisioni prese e conclusioni.
\end{itemize}

\newpage

\section{Verbale}
\subsection{Suddivisione dei ruoli per il settimo sprint}
Il gruppo stabilisce innanzitutto la suddivisione dei ruoli per il settimo periodo, assegnando a ciascun membro un compito non ancora svolto. La suddivisione risultante è la seguente:
\begin{itemize}
	\item \textbf{Responsabile}: Leonardo Baldo;
	\item \textbf{Amministratore}: Davide Malgarise;
	\item \textbf{Analista}: Raul Seganfreddo;
	\item \textbf{Progettista}: Matteo Tiozzo, Davide Malgarise, Valerio Occhinegro;
	\item \textbf{Programmatore}: Antonio Benetazzo;
	\item \textbf{Verificatore}: Elena Ferro.
\end{itemize}
Successivamente vengono definiti i ruoli per ogni riunione interna o esterna:
\begin{itemize}
	\item \textbf{Amministratore}: Davide Malgarise;
	\item \textbf{Redattore}: Leonardo Baldo;
	\item \textbf{Verificatore}: Elena Ferro.
\end{itemize}

\subsection{Cosa è stato fatto fino ad ora}
Viene fatta una panoramica sul lavoro svolto fino ad ora:
\begin{itemize}
	\item \textbf{Verbali} \\
		  Redatto, revisionato e approvato il verbale della riunione interna del 2024-05-29.
	\item \textbf{Stato della documentazione} \\
		  Continua la stesura dei documenti \textit{Piano di Qualifica}, aggiornando i grafici con i dati dello sprint e \textit{Piano di Progetto}, aggiungendo il consuntivo.
	\item \textbf{Studio} \\
		  Inoltre come prossimo passo, inizia lo studio dell'architettura di progetto.
\end{itemize}

\subsection{Attività da svolgere per il settimo sprint}
Vengono definite le seguenti attività da svolgere per il settimo periodo:
\begin{table}[!h]
	\centering
	\begin{tabular}{ |l||p{7cm}|l|l| }
		\hline
		\textbf{Codice}          & \textbf{Nome attività}                           & \textbf{Assegnatario} & \textbf{Scadenza} \\
		\hline
		\mySkip[86bz36wet]       & Glo - Eventuale aggiunta nuovi termini           & Davide Malgarise      & 2024-06-12        \\
		\mySkip[86bz36ugd]       & ST - Prima stesura documento                     & Matteo Tiozzo         & 2024-06-12        \\
		\mySkip[86bz36uee]       & MU - Prima stesura documento                     & Antonio Benetazzo     & 2024-06-12        \\
		\mySkip[86bz36u3g]       & PdQ - Aggiornare grafici 6 sprint                & Elena Ferro           & 2024-06-12        \\
		\mySkip[86bz36tuh]       & PdP - Consuntivo 6 sprint e preventivo 7 sprint  & Leonardo Baldo        & 2024-06-12        \\
		\mySkip[86bz36tj8]       & PdQ - Aggiunta test di integrazione e unità      & Elena Ferro           & 2024-06-12        \\
		\mySkip[86bz36p9v]       & Implementazione primi test                       & Antonio Benetazzo     & 2024-06-12        \\
		\mySkip[86bz36jf7]       & AdR - Aggiornamento                              & Raul Seganfreddo      & 2024-06-12        \\
		\mySkip[86bz36hj6]       & Aggiungere tipi di sensori                       & Antonio Benetazzo     & 2024-06-12        \\
		\mySkip[86bz36hgq]       & Verbale esterno 2024-05-05                       & Valerio Occhinegro    & 2024-06-12        \\
		\mySkip[86bz36h8r]       & Verbale interno 2024-05-05                       & Valerio Occhinegro    & 2024-06-12        \\
		\hline
	\end{tabular}
	\caption{Attività da svolgere per il settimo sprint}
\end{table}

\subsection{Varie ed eventuali}
Sono sorti alcuni dubbi e difficoltà riguardanti lo studio architetturale. Soprattutto nella individuazione di una architettura di progetto che meglio si adatti al nostro scopo. \\
Procediamo a continuarne lo studio per analizzare il problema e ad individuare una soluzione.

\subsection{Decisioni prese e conclusioni}
Non essendo emerse ulteriori problematiche o dubbi, il gruppo concorda la prossima riunione per il 2024-06-12.
\end{document}