\documentclass[italian,12pt]{article}

%--------------variabili------------------%
\def\Title{Norme di Progetto}
\def\Author{7Last}
\def\Version{v0.2}
%-----------------------------------------%


\usepackage[left=2cm, right=2cm, bottom=3cm, top=3cm]{geometry}
\usepackage{fancyhdr}
\usepackage{graphicx}
\graphicspath{ {../../logo/} }
\usepackage{href-ul}
\usepackage{tikz}
\usepackage{tgadventor}
\usepackage[useregional=numeric,showseconds=true,showzone=false]{datetime2}
\usepackage{caption}
\usepackage{longtable}
\usepackage{xcolor}




\linespread{1.2}
\captionsetup[table]{labelformat=empty}
\geometry{headsep=1.5cm}

\renewcommand{\contentsname}{Indice}
\renewcommand\familydefault{\sfdefault}

\let\oldthepage\thepage
\renewcommand{\thepage}{\sffamily \oldthepage}

\begin{document}

\newcommand{\mySkip}[1][]{#1}

\newgeometry{left=2cm,right=2cm,bottom=2.1cm,top=2.1cm}
\begin{titlepage}
	\vspace*{.5cm}

	\vspace{2cm}
	{
		\centering
		{\bfseries\huge \Title\par}
		\bigbreak
		{\bfseries\Large \Subtitle\par}
		\bigbreak
		{\bfseries\large \Author\par}
		\bigbreak
		{\Date\;-\;\Version\par}
		\vfill

		\begin{center}
			\begin{tikzpicture}
				\clip (0,0) circle (2cm) node {\includegraphics[width=4cm]{logo.jpg}};
			\end{tikzpicture}
		\end{center}
	}

	\vfill

\end{titlepage}

\restoregeometry






















\newpage

\pagestyle{fancy}
\fancyhead{}
\lhead{
	\begin{tikzpicture}
		\clip (0,0) circle (0.5cm);
		\node at (0,0) {\includegraphics[width=1cm]{./../logo/logo.png}};
	\end{tikzpicture}%
}
\chead{\vspace{\fill}\Title\vspace{\fill}}
\rhead{\vspace{\fill}\Version\vspace{\fill}}


\begin{table}[!h]
	\caption{Versioni}
	\footnotesize
	\begin{center}
		\begin{tabular}{ l l l l p{6cm} }
			\hline                                                                      \\[-2ex]
			Ver. & Data       & Redattore        & Verificatore       & Descrizione     \\
			\\[-2ex] \hline \\[-1.5ex]
			1.0  & 2024-06-24 & Leonardo Baldo   & Raul Seganfreddo   & Stesura verbale \\
			\\[-1.5ex] \hline
		\end{tabular}
	\end{center}
\end{table}

\newpage

\tableofcontents

\newpage

\section{Dettagli della riunione}

\textbf{Sede della riunione}: Piattaforma Discord\\
\textbf{Orario di inizio}: 16:45\\
\textbf{Orario di fine}: 17:15\\

\begin{flushleft}
	\begin{table}[!h]
		\begin{tabular}{ |l|l|l| }
			\hline
			\textbf{Partecipante} & \textbf{Ruolo} & \textbf{Presenza} \\
			\hline
			Antonio Benetazzo     & Amministratore & Presente          \\
			Davide Malgarise      &                & Assente           \\
			Elena Ferro           &                & Presente          \\
			Leonardo Baldo        &                & Presente          \\
			Matteo Tiozzo         & Redattore      & Presente          \\
			Raul Seganfreddo      & Verificatore   & Presente          \\
			Valerio Occhinegro    &                & Presente          \\
			\hline
		\end{tabular}
	\end{table}
\end{flushleft}

\subsection*{Ordine del giorno:}
\begin{itemize}
	\item retrospettiva sullo sprint appena concluso;
	\item suddivisione dei ruoli per il nono sprint;
	\item attività da svolgere per il nono sprint;
	\item decisioni prese e conclusioni.
\end{itemize}


\newpage

\section{Verbale}

\subsection{Retrospettiva sullo sprint appena concluso}
Il gruppo comincia la riunione discutendo della valutazione ricevuta in seguito alla revisione \textit{RTB}. In generale viene considerata una valutazione positiva, con pochi errori e mancanze da correggere. In particolare, come già emerso in precedenza, viene sottolineata la necessità di migliorare la qualità espositiva, oltre che la precisione nei dettagli. Vengono inoltre fatte ulteriori osservazioni riguardo ad alcuni particolari da rivedere nella documentazione, che saranno oggetto di correzione nei prossimi sprint. \\
Viene inoltre fatta una retrospettiva sul periodo di lavoro appena concluso, in cui si evidenzia una lenta ripresa delle attività, dovuta principalmente ad impegni universitari, complice la sessione estiva e gli esami che impegnavo diversi membri del gruppo. Si decide di pianificare con attenzione le attività per il prossimo sprint, cercando però di non abbassare il ritmo di lavoro, questo per non incorrere in ritardi che potrebbero compromettere la consegna del prodotto entro i tempi e costi stabiliti.

\subsection{Suddivisione dei ruoli per il nono sprint}
Il gruppo stabilisce la suddivisione dei ruoli per il prossimo periodo di lavoro:
\begin{itemize}
	\item \textbf{Responsabile}: Elena Ferro;
	\item \textbf{Amministratore}: Valerio Occhinegro;
	\item \textbf{Analista}: Matteo Tiozzo;
	\item \textbf{Progettista}: Leonardo Baldo;
	\item \textbf{Programmatore}: Davide Malgarise, Antonio Benetazzo;
	\item \textbf{Verificatore}: Raul Seganfreddo.
\end{itemize}
Inoltre si decide di usare i seguenti ruoli per ogni riunione interna o esterna:
\begin{itemize}
	\item \textbf{Amministratore}: Valerio Occhinegro;
	\item \textbf{Redattore}: Leonardo Baldo;
	\item \textbf{Verificatore}: Raul Seganfreddo.
\end{itemize}

\subsection{Attività da svolgere per il nono sprint}
Vengono definite le seguenti attività da svolgere per il nono periodo:
\begin{table}[!h]
	\centering
	\begin{tabular}{ |l||p{7cm}|l|l| }
		\hline
		\textbf{Codice}          & \textbf{Nome attività}                           & \textbf{Assegnatario} & \textbf{Scadenza} \\
		\hline
		\mySkip[86bz8e28m]       & NdP - Aggiunta dei test previsti ed implementati & Elena Ferro           & 2024-06-26        \\
		\mySkip[86bz8e6cp]       & PdP - Saldo 9 sprint + preventivo 10 sprint      & Valerio Occhinegro    & 2024-06-26        \\
		\mySkip[86bz8e4ac]       & PdQ - Aggiornare grafici 9 sprint                & Raul Seganfreddo      & 2024-06-26        \\
		\mySkip[86bz8e8g5]       & Glo - Eventuale aggiunta nuovi termini           & Leonardo Baldo        & 2024-06-26        \\
		\mySkip[86bz8e13d]       & AdR - Aggiornamento                              & Matteo Tiozzo         & 2024-06-26        \\
		\mySkip[86bz8e2ja]       & MU - Prosecuzione stesura documento              & Antonio Benetazzo     & 2024-06-26        \\
		\mySkip[86bz8e3xn]       & ST - Prosecuzione stesura documento              & Leonardo Baldo        & 2024-06-26        \\
		\mySkip[86bz9gj8r]       & Verbale esterno 19 giugno                        & Elena Ferro           & 2024-06-26        \\
		\mySkip[86bz9gjd9]       & Verbale interno 19 giugno                        & Raul Seganfreddo      & 2024-06-26        \\
		\mySkip[86bz8dzy0]       & Business case parcheggio colonnine               & Davide Malgarise      & 2024-06-26        \\
		\mySkip[86bz8dzz7]       & Progettazione dashboard                          & Antonio Benetazzo     & 2024-06-26        \\
		\mySkip[86bz8dzzg]       & Implementazione test                             & Antonio Benetazzo     & 2024-06-26        \\
		\mySkip[86bz8e010]       & Implementazione alert                            & Davide Malgarise      & 2024-06-26        \\
		\hline
	\end{tabular}
	\caption{Attività da svolgere per il nono sprint}
\end{table}

\subsection{Decisioni prese e conclusioni}
Non essendo emersi problemi o dubbi degni di nota durante la riunione, il gruppo decide di pianificare la prossima riunione per il giorno 26 giugno indicativamente per le ore 16:30, successivamente al SAL con il proponente.

\end{document}
