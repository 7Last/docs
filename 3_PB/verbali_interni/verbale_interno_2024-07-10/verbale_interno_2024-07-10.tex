\documentclass[italian,12pt]{article}

%--------------variabili------------------%
\def\Title{Norme di Progetto}
\def\Author{7Last}
\def\Version{v0.2}
%-----------------------------------------%


\usepackage[left=2cm, right=2cm, bottom=3cm, top=3cm]{geometry}
\usepackage{fancyhdr}
\usepackage{graphicx}
\graphicspath{ {../../logo/} }
\usepackage{href-ul}
\usepackage{tikz}
\usepackage{tgadventor}
\usepackage[useregional=numeric,showseconds=true,showzone=false]{datetime2}
\usepackage{caption}
\usepackage{longtable}
\usepackage{xcolor}




\linespread{1.2}
\captionsetup[table]{labelformat=empty}
\geometry{headsep=1.5cm}

\renewcommand{\contentsname}{Indice}
\renewcommand\familydefault{\sfdefault}

\let\oldthepage\thepage
\renewcommand{\thepage}{\sffamily \oldthepage}

\begin{document}

\newcommand{\mySkip}[1][]{#1}

\newgeometry{left=2cm,right=2cm,bottom=2.1cm,top=2.1cm}
\begin{titlepage}
	\vspace*{.5cm}

	\vspace{2cm}
	{
		\centering
		{\bfseries\huge \Title\par}
		\bigbreak
		{\bfseries\Large \Subtitle\par}
		\bigbreak
		{\bfseries\large \Author\par}
		\bigbreak
		{\Date\;-\;\Version\par}
		\vfill

		\begin{center}
			\begin{tikzpicture}
				\clip (0,0) circle (2cm) node {\includegraphics[width=4cm]{logo.jpg}};
			\end{tikzpicture}
		\end{center}
	}

	\vfill

\end{titlepage}

\restoregeometry






















\newpage

\pagestyle{fancy}
\fancyhead{}
\lhead{
	\begin{tikzpicture}
		\clip (0,0) circle (0.5cm);
		\node at (0,0) {\includegraphics[width=1cm]{./../logo/logo.png}};
	\end{tikzpicture}%
}
\chead{\vspace{\fill}\Title\vspace{\fill}}
\rhead{\vspace{\fill}\Version\vspace{\fill}}


\begin{table}[!h]
	\caption{Versioni}
	\footnotesize
	\begin{center}
		\begin{tabular}{ l l l l p{6cm} }
			\hline                                                                       \\[-2ex]
			Ver. & Data       & Redattore         & Verificatore       & Descrizione     \\
			\\[-2ex] \hline \\[-1.5ex]
			1.0  & 2024-07-12 & Antonio Benetazzo & Valerio Occhinegro & Stesura verbale \\
			\\[-1.5ex] \hline
		\end{tabular}
	\end{center}
\end{table}

\newpage

\tableofcontents

\newpage

\section{Dettagli della riunione}

\textbf{Sede della riunione}: Piattaforma Discord\\
\textbf{Orario di inizio}: 17:00\\
\textbf{Orario di fine}: 17:30\\

\begin{flushleft}
	\begin{table}[!h]
		\begin{tabular}{ |l|l|l| }
			\hline
			\textbf{Partecipante} & \textbf{Ruolo} & \textbf{Presenza} \\
			\hline
			Antonio Benetazzo     & Amministratore & Presente          \\
			Davide Malgarise      & Redattore      & Presente          \\
			Elena Ferro           &                & Presente          \\
			Leonardo Baldo        &                & Presente          \\
			Matteo Tiozzo         & Verificatore   & Presente          \\
			Raul Seganfreddo      &                & Presente          \\
			Valerio Occhinegro    &                & Presente          \\
			\hline
		\end{tabular}
	\end{table}
\end{flushleft}

\subsection*{Ordine del giorno:}
\begin{itemize}
	\item retrospettiva sullo sprint appena concluso;
	\item suddivisione dei ruoli per il dodicesimo sprint;
	\item attività da svolgere per il dodicesimo sprint;
	\item decisioni prese e conclusioni.
\end{itemize}


\newpage

\section{Verbale}

\subsection{Retrospettiva sullo sprint appena concluso}
La riunione inizia con un confronto sull'andamento del periodo appena terminato e sullo stato di completamento delle attività previste. Gli obiettivi che ci eravamo posti ad inizio sprint sono stati raggiunti, in particolare siamo riusciti ad ottenere alcuni grafici che mostrano l'efficienza delle colonnine di ricarica, dato che viene reso disponibile mediante l'uso di \textit{Apache Flink} che ci ha permesso di combinare i dati provenienti da sensori diversi, quelli delle colonnine di ricarica e quelli dei parcheggi. \\
Successivamente abbiamo discusso sulla possibile data per il collaudo finale. Nonostante l'andamento positivo dell'ultimo sprint, ci sono ancora diversi aspetti da migliorare nel prodotto, oltre alla documentazione che risulta ancora molto carente, in particolare nei documenti \textit{Specifica Tecnica} e \textit{Manuale Utente}. Riteniamo quindi opportuno rimandare il collaudo finale di un'ulteriore settimana, in modo da avere il tempo di completare la documentazione e di migliorare il prodotto e poter così presentare al proponente un prodotto più maturo e completo.

\subsection{Suddivisione dei ruoli per il dodicesimo sprint}
Il gruppo stabilisce la suddivisione dei ruoli per il prossimo periodo di lavoro:
\begin{itemize}
	\item \textbf{Responsabile}: Antonio Benetazzo;
	\item \textbf{Amministratore}: Davide Malgarise;
	\item \textbf{Progettista}: Leonardo Baldo;
	\item \textbf{Programmatore}: Raul Seganfreddo, Elena Ferro;
	\item \textbf{Verificatore}: Matteo Tiozzo, Valerio Occhinegro.
\end{itemize}
\newblock
Inoltre si decide di usare i seguenti ruoli per ogni riunione interna o esterna:
\begin{itemize}
	\item \textbf{Amministratore}: Elena Ferro;
	\item \textbf{Redattore}: Antonio Benetazzo;
	\item \textbf{Verificatore}: Valerio Occhinegro.
\end{itemize}

\newpage

\subsection{Attività da svolgere per il dodicesimo sprint} % TODO: sistemare le attività e i ruoli di tutti i verbali interni
Vengono definite le seguenti attività da svolgere per il prossimo periodo:
\begin{table}[!h]
	\centering
	\begin{tabular}{ |l||p{7cm}|l|l| }
		\hline
		\textbf{Codice}          & \textbf{Nome attività}                           & \textbf{Assegnatario} & \textbf{Scadenza} \\
		\hline
		\mySkip[86bz8e29k]       & NdP - Aggiunta dei test previsti ed implementati & Antonio Benetazzo     & 2024-07-03        \\
		\mySkip[86bz8e6dg]       & PdP - Saldo 10 sprint               				& Davide Malgarise      & 2024-07-03        \\
		\hline
	\end{tabular}
	\caption{Attività da svolgere per il dodicesimo sprint}
\end{table}

\subsection{Decisioni prese e conclusioni}
Sulla base di quanto emerso, il gruppo pianifica la prossima riunione per il giorno 17 luglio alle ore 16:00, in modo da poter verificare che siano state svolte tutte le attività previste per il collaudo finale con l'azienda, che viene fissato per il giorno 19 luglio.

\end{document}
