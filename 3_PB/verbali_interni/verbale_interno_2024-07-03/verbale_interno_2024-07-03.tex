\documentclass[italian,12pt]{article}

%--------------variabili------------------%
\def\Title{Norme di Progetto}
\def\Author{7Last}
\def\Version{v0.2}
%-----------------------------------------%


\usepackage[left=2cm, right=2cm, bottom=3cm, top=3cm]{geometry}
\usepackage{fancyhdr}
\usepackage{graphicx}
\graphicspath{ {../../logo/} }
\usepackage{href-ul}
\usepackage{tikz}
\usepackage{tgadventor}
\usepackage[useregional=numeric,showseconds=true,showzone=false]{datetime2}
\usepackage{caption}
\usepackage{longtable}
\usepackage{xcolor}




\linespread{1.2}
\captionsetup[table]{labelformat=empty}
\geometry{headsep=1.5cm}

\renewcommand{\contentsname}{Indice}
\renewcommand\familydefault{\sfdefault}

\let\oldthepage\thepage
\renewcommand{\thepage}{\sffamily \oldthepage}

\begin{document}

\newcommand{\mySkip}[1][]{#1}

\newgeometry{left=2cm,right=2cm,bottom=2.1cm,top=2.1cm}
\begin{titlepage}
	\vspace*{.5cm}

	\vspace{2cm}
	{
		\centering
		{\bfseries\huge \Title\par}
		\bigbreak
		{\bfseries\Large \Subtitle\par}
		\bigbreak
		{\bfseries\large \Author\par}
		\bigbreak
		{\Date\;-\;\Version\par}
		\vfill

		\begin{center}
			\begin{tikzpicture}
				\clip (0,0) circle (2cm) node {\includegraphics[width=4cm]{logo.jpg}};
			\end{tikzpicture}
		\end{center}
	}

	\vfill

\end{titlepage}

\restoregeometry






















\newpage

\pagestyle{fancy}
\fancyhead{}
\lhead{
	\begin{tikzpicture}
		\clip (0,0) circle (0.5cm);
		\node at (0,0) {\includegraphics[width=1cm]{./../logo/logo.png}};
	\end{tikzpicture}%
}
\chead{\vspace{\fill}\Title\vspace{\fill}}
\rhead{\vspace{\fill}\Version\vspace{\fill}}


\begin{table}[!h]
	\caption{Versioni}
	\footnotesize
	\begin{center}
		\begin{tabular}{ l l l l p{6cm} }
			\hline                                                                      \\[-2ex]
			Ver. & Data       & Redattore        & Verificatore       & Descrizione     \\
			\\[-2ex] \hline \\[-1.5ex]
			1.0  & 2024-07-08 & Leonardo Baldo   & Antonio Benetazzo  & Stesura verbale \\
			\\[-1.5ex] \hline
		\end{tabular}
	\end{center}
\end{table}

\newpage

\tableofcontents

\newpage

\section{Dettagli della riunione}

\textbf{Sede della riunione}: Piattaforma Discord\\
\textbf{Orario di inizio}: 17:00\\
\textbf{Orario di fine}: 18:00\\

\begin{flushleft}
	\begin{table}[!h]
		\begin{tabular}{ |l|l|l| }
			\hline
			\textbf{Partecipante} & \textbf{Ruolo} & \textbf{Presenza} \\
			\hline
			Antonio Benetazzo     & Verificatore   & Presente          \\
			Davide Malgarise      &                & Presente          \\
			Elena Ferro           &                & Assente           \\
			Leonardo Baldo        & Redattore      & Presente          \\
			Matteo Tiozzo         & Amministratore & Presente          \\
			Raul Seganfreddo      &                & Presente          \\
			Valerio Occhinegro    &                & Presente          \\
			\hline
		\end{tabular}
	\end{table}
\end{flushleft}

\subsection*{Ordine del giorno:}
\begin{itemize}
	\item retrospettiva sullo sprint appena concluso;
	\item suddivisione dei ruoli per l'undicesimo sprint;
	\item attività da svolgere per l'undicesimo sprint;
	\item decisioni prese e conclusioni.
\end{itemize}


\newpage

\section{Verbale}

\subsection{Retrospettiva sullo sprint appena concluso}
Il gruppo comincia la riunione discutendo della proposta dell'azienda di sostenere un collaudo finale, direttamente nella loro sede, così da presentare il prodotto e conoscere il proponete anche fisicamente. \\
Inoltre viene fatta un'ultima verifica e modifica degli ultimi verbali, prima di inviarli all'azienda per farli firmare. \\
...

\subsection{Suddivisione dei ruoli per l'undicesimo sprint}
Il gruppo stabilisce la suddivisione dei ruoli per il prossimo periodo di lavoro:
\begin{itemize}
	\item \textbf{Responsabile}: Davide Malgarise, Raul Seganfreddo;
	\item \textbf{Amministratore}: Matteo Tiozzo;
	\item \textbf{Programmatore}: Davide Malgarise, Valerio Occhinegro;
	\item \textbf{Verificatore}: Antonio Benetazzo, Elena Ferro, Leonardo Baldo.
\end{itemize}
Inoltre si decide di usare i seguenti ruoli per ogni riunione interna o esterna:
\begin{itemize}
	\item \textbf{Amministratore}: Matteo Tiozzo;
	\item \textbf{Redattore}: Leonardo Baldo;
	\item \textbf{Verificatore}: Antonio Benetazzo.
\end{itemize}

\newpage
\subsection{Attività da svolgere per l'undicesimo sprint}
Vengono definite le seguenti attività da svolgere per l'undicesimo periodo:
\begin{table}[!h]
	\centering
	\begin{tabular}{ |l||p{7cm}|l|l| }
		\hline
		\textbf{Codice}          & \textbf{Nome attività}                           & \textbf{Assegnatario} & \textbf{Scadenza} \\
		\hline
		\mySkip[]       & NdP - Aggiunta dei test previsti ed implementati &  & 2024-07-10        \\
		\mySkip[]       & PdP - Saldo 11 sprint                            &  & 2024-07-10        \\
		\mySkip[]       & PdQ - Aggiornare grafici 11 sprint               &  & 2024-07-10        \\
		\mySkip[]       & Glo - Eventuale aggiunta nuovi termini           &  & 2024-07-10        \\
		\mySkip[]       & AdR - Aggiornamento                              &  & 2024-07-10        \\
		\mySkip[]       & MU - Prosecuzione stesura documento              &  & 2024-07-10        \\
		\mySkip[]       & ST - Prosecuzione stesura documento              &  & 2024-07-10        \\
		\mySkip[]       & Verbale esterno 2024-07-03                       &  & 2024-07-10        \\
		\mySkip[]       & Verbale interno 2024-07-03                       &  & 2024-07-10        \\
		\mySkip[]       & Modifiche alle ashboard                          &  & 2024-07-10        \\
		\mySkip[]       & Implementazione integration test                 &  & 2024-07-10        \\
		\mySkip[]       & ...                                              &  & 2024-07-10        \\
		\hline
	\end{tabular}
	\caption{Attività da svolgere per l'undicesimo sprint}
\end{table}

\subsection{Decisioni prese e conclusioni}
Dopo aver discusso tra i membri del gruppo, decidiamo di pianificare il collaudo finale con il proponente nel pomeriggio del 2024-07-19.
Non essendo emersi problemi o dubbi degni di nota durante la riunione, il gruppo decide di pianificare la prossima riunione per il giorno 2024-07-10 indicativamente per le ore 17:00, successivamente al SAL con il proponente.

\end{document}
