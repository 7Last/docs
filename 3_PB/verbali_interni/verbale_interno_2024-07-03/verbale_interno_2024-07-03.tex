\documentclass[italian,12pt]{article}

%--------------variabili------------------%
\def\Title{Norme di Progetto}
\def\Author{7Last}
\def\Version{v0.2}
%-----------------------------------------%


\usepackage[left=2cm, right=2cm, bottom=3cm, top=3cm]{geometry}
\usepackage{fancyhdr}
\usepackage{graphicx}
\graphicspath{ {../../logo/} }
\usepackage{href-ul}
\usepackage{tikz}
\usepackage{tgadventor}
\usepackage[useregional=numeric,showseconds=true,showzone=false]{datetime2}
\usepackage{caption}
\usepackage{longtable}
\usepackage{xcolor}




\linespread{1.2}
\captionsetup[table]{labelformat=empty}
\geometry{headsep=1.5cm}

\renewcommand{\contentsname}{Indice}
\renewcommand\familydefault{\sfdefault}

\let\oldthepage\thepage
\renewcommand{\thepage}{\sffamily \oldthepage}

\begin{document}

\newcommand{\mySkip}[1][]{#1}

\newgeometry{left=2cm,right=2cm,bottom=2.1cm,top=2.1cm}
\begin{titlepage}
	\vspace*{.5cm}

	\vspace{2cm}
	{
		\centering
		{\bfseries\huge \Title\par}
		\bigbreak
		{\bfseries\Large \Subtitle\par}
		\bigbreak
		{\bfseries\large \Author\par}
		\bigbreak
		{\Date\;-\;\Version\par}
		\vfill

		\begin{center}
			\begin{tikzpicture}
				\clip (0,0) circle (2cm) node {\includegraphics[width=4cm]{logo.jpg}};
			\end{tikzpicture}
		\end{center}
	}

	\vfill

\end{titlepage}

\restoregeometry






















\newpage

\pagestyle{fancy}
\fancyhead{}
\lhead{
	\begin{tikzpicture}
		\clip (0,0) circle (0.5cm);
		\node at (0,0) {\includegraphics[width=1cm]{./../logo/logo.png}};
	\end{tikzpicture}%
}
\chead{\vspace{\fill}\Title\vspace{\fill}}
\rhead{\vspace{\fill}\Version\vspace{\fill}}


\begin{table}[!h]
	\caption{Versioni}
	\footnotesize
	\begin{center}
		\begin{tabular}{ l l l l p{6cm} }
			\hline                                                                      \\[-2ex]
			Ver. & Data       & Redattore        & Verificatore       & Descrizione     \\
			\\[-2ex] \hline \\[-1.5ex]
			1.0  & 2024-07-08 & Leonardo Baldo   & Antonio Benetazzo  & Stesura verbale \\
			\\[-1.5ex] \hline
		\end{tabular}
	\end{center}
\end{table}

\newpage

\tableofcontents

\newpage

\section{Dettagli della riunione}

\textbf{Sede della riunione}: Piattaforma Discord\\
\textbf{Orario di inizio}: 17:00\\
\textbf{Orario di fine}: 18:00\\

\begin{flushleft}
	\begin{table}[!h]
		\begin{tabular}{ |l|l|l| }
			\hline
			\textbf{Partecipante} & \textbf{Ruolo} & \textbf{Presenza} \\
			\hline
			Antonio Benetazzo     & Verificatore   & Presente          \\
			Davide Malgarise      &                & Presente          \\
			Elena Ferro           &                & Assente           \\
			Leonardo Baldo        & Redattore      & Presente          \\
			Matteo Tiozzo         & Amministratore & Presente          \\
			Raul Seganfreddo      &                & Presente          \\
			Valerio Occhinegro    &                & Presente          \\
			\hline
		\end{tabular}
	\end{table}
\end{flushleft}

\subsection*{Ordine del giorno:}
\begin{itemize}
	\item retrospettiva sullo sprint appena concluso;
	\item suddivisione dei ruoli per l'undicesimo sprint;
	\item attività da svolgere per l'undicesimo sprint;
	\item decisioni prese e conclusioni.
\end{itemize}


\newpage

\section{Verbale}

\subsection{Retrospettiva sullo sprint appena concluso}
Il gruppo si ritiene soddisfatto dell'andamento dello sprint appena concluso. Nonostante il potenziale rischio di ritardi dovuto alle numerose attività da svolgere e al periodo di esami che sta impegnando diversi membri, siamo riusciti a raggiungere gli obiettivi prefissati. Anche il proponente ha espresso apprezzamento per il lavoro svolto e ha proposto di fissare un appuntamento presso la loro sede con lo scopo di presentare il prodotto ed effettuare un collaudo finale per l'MVP e con l'occasione conoscersi di persona. L'azienda ha proposto come data utile il 19 luglio, data ritenuta da alcuni membri troppo ravvicinata, viste le numerose attività ancora da svolgere.

\subsection{Suddivisione dei ruoli per l'undicesimo sprint}
Il gruppo stabilisce la suddivisione dei ruoli per il prossimo periodo di lavoro:
\begin{itemize}
	\item \textbf{Responsabile}: Raul Seganfreddo;
	\item \textbf{Amministratore}: Matteo Tiozzo;
	\item \textbf{Programmatore}: Davide Malgarise, Valerio Occhinegro;
	\item \textbf{Verificatore}: Antonio Benetazzo, Elena Ferro, Leonardo Baldo.
\end{itemize}
\newblock
Inoltre si decide di usare i seguenti ruoli per ogni riunione interna o esterna:
\begin{itemize}
	\item \textbf{Amministratore}: Matteo Tiozzo;
	\item \textbf{Redattore}: Leonardo Baldo;
	\item \textbf{Verificatore}: Antonio Benetazzo.
\end{itemize}

\newpage
\subsection{Attività da svolgere per l'undicesimo sprint}
Vengono definite le seguenti attività da svolgere per l'undicesimo periodo:
\begin{table}[!h]
	\centering
	\begin{tabular}{ |l||p{7cm}|l|l| }
		\hline
		\textbf{Codice}    & \textbf{Nome attività}                           & \textbf{Assegnatario} & \textbf{Scadenza} \\
		\hline
		\mySkip[86bzjkp80] & PdP - Saldo 11 sprint + preventivo 12 sprint     & Matteo Tiozzo         & 2024-07-10 \\
		\mySkip[86bzjkpg8] & PdQ - Aggiornare grafici 11 sprint               & Antonio Benetazzo     & 2024-07-10 \\
		\mySkip[86bzjkpx9] & Verbale esterno 2024-07-03                       & Leonardo Baldo        & 2024-07-10 \\
		\mySkip[86bzjkpy4] & Verbale interno 2024-07-03                       & Davide Malgarise      & 2024-07-10 \\
		\hline
	\end{tabular}
	\caption{Attività da svolgere per l'undicesimo sprint}
\end{table}

\subsection{Decisioni prese e conclusioni}
Dopo un breve confronto tra i vari membri decidiamo di tenere in sospeso la decisione sulla data del collaudo finale con il proponente, in modo da poter valutare meglio la situazione e concordare una data congrua in modo tale da permetterci di completare le attività rimanenti e presentarci all'appuntamento con un prodotto il più completo possibile. \\
Non essendo emerso altro durante la riunione, il gruppo decide di pianificare il prossimo incontro interno per il giorno 10 luglio indicativamente per le ore 17:00.

\end{document}
