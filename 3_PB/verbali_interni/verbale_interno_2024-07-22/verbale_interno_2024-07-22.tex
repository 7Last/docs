\documentclass[italian,12pt]{article}

%--------------variabili------------------%
\def\Title{Norme di Progetto}
\def\Author{7Last}
\def\Version{v0.2}
%-----------------------------------------%


\usepackage[left=2cm, right=2cm, bottom=3cm, top=3cm]{geometry}
\usepackage{fancyhdr}
\usepackage{graphicx}
\graphicspath{ {../../logo/} }
\usepackage{href-ul}
\usepackage{tikz}
\usepackage{tgadventor}
\usepackage[useregional=numeric,showseconds=true,showzone=false]{datetime2}
\usepackage{caption}
\usepackage{longtable}
\usepackage{xcolor}




\linespread{1.2}
\captionsetup[table]{labelformat=empty}
\geometry{headsep=1.5cm}

\renewcommand{\contentsname}{Indice}
\renewcommand\familydefault{\sfdefault}

\let\oldthepage\thepage
\renewcommand{\thepage}{\sffamily \oldthepage}

\begin{document}

\newcommand{\mySkip}[1][]{#1}

\newgeometry{left=2cm,right=2cm,bottom=2.1cm,top=2.1cm}
\begin{titlepage}
	\vspace*{.5cm}

	\vspace{2cm}
	{
		\centering
		{\bfseries\huge \Title\par}
		\bigbreak
		{\bfseries\Large \Subtitle\par}
		\bigbreak
		{\bfseries\large \Author\par}
		\bigbreak
		{\Date\;-\;\Version\par}
		\vfill

		\begin{center}
			\begin{tikzpicture}
				\clip (0,0) circle (2cm) node {\includegraphics[width=4cm]{logo.jpg}};
			\end{tikzpicture}
		\end{center}
	}

	\vfill

\end{titlepage}

\restoregeometry






















\newpage

\pagestyle{fancy}
\fancyhead{}
\lhead{
	\begin{tikzpicture}
		\clip (0,0) circle (0.5cm);
		\node at (0,0) {\includegraphics[width=1cm]{./../logo/logo.png}};
	\end{tikzpicture}%
}
\chead{\vspace{\fill}\Title\vspace{\fill}}
\rhead{\vspace{\fill}\Version\vspace{\fill}}


\begin{table}[!h]
	\caption{Versioni}
	\footnotesize
	\begin{center}
		\begin{tabular}{ l l l l p{6cm} }
			\hline                                                                       \\[-2ex]
			Ver. & Data       & Redattore         & Verificatore       & Descrizione     \\
			\\[-2ex] \hline \\[-1.5ex]
			1.0  & 2024-07-22 & Raul Seganfreddo  & Elena Ferro        & Stesura verbale \\
			\\[-1.5ex] \hline
		\end{tabular}
	\end{center}
\end{table}

\newpage

\tableofcontents

\newpage

\section{Dettagli della riunione}

\textbf{Sede della riunione}: Piattaforma Discord\\
\textbf{Orario di inizio}: 15:00\\
\textbf{Orario di fine}: 15:30\\

\begin{flushleft}
	\begin{table}[!h]
		\begin{tabular}{ |l|l|l| }
			\hline
			\textbf{Partecipante} & \textbf{Ruolo} & \textbf{Presenza} \\
			\hline
			Antonio Benetazzo     & Redattore      & Presente          \\
			Davide Malgarise      &                & Presente          \\
			Elena Ferro           & Amministratore & Presente          \\
			Leonardo Baldo        &                & Presente          \\
			Matteo Tiozzo         & Verificatore   & Presente          \\
			Raul Seganfreddo      &                & Presente          \\
			Valerio Occhinegro    &                & Presente          \\
			\hline
		\end{tabular}
	\end{table}
\end{flushleft}

\subsection*{Ordine del giorno:}
\begin{itemize}
	\item retrospettiva sullo sprint appena concluso;
	\item candidatura per revisione PB;
	\item decisioni prese e conclusioni.
\end{itemize}


\newpage

\section{Verbale}

\subsection{Retrospettiva sullo sprint appena concluso}
Ci confrontiamo inizialmente sull'andamento dello sprint appena concluso, in particolare sul collaudo finale avuto con il proponente. Tutti i membri del gruppo concordano sul fatto che il collaudo sia andato bene, come confermato anche dall'azienda, e che quindi il prodotto sia sufficiente maturo per essere considerato un \textit{Minimum Viable Product}. Il fatto di non dover apportare modifiche al prodotto ha permesso al gruppo di concentrarsi immediatamente sulle attività di verifica della documentazione, che sono state effettuate e terminate con successo, permettendoci di raggiungere gli obiettivi prefissati con due giorni di anticipo.

\subsection{Candidatura per revisione PB}
Sulla base di quanto emerso dalla retrospettiva, il gruppo decide di procedere con la candidatura per la revisione PB. Il responsabile si occuperà di contattare il professor Cardin per ufficializzare la candidatura e chiedere un appuntamento per la prima parte della revisione, nella speranza di ottenere un riscontro positivo e poter procedere con la seconda e ultima parte della revisione con il professor Vardanega.

\subsection{Decisioni prese e conclusioni}
Il gruppo si congratula per il lavoro svolto e per il raggiungimento degli obiettivi prefissati, e si prepara alla revisione PB.

\end{document}
