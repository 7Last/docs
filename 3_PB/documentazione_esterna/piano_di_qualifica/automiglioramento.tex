\section{Iniziative di automiglioramento per la qualità}
\subsection{Introduzione}
In questa sezione verranno riportate le iniziative di automiglioramento che il nostro gruppo ha deciso di adottare per aumentare la qualità del prodotto e dei processi. Queste iniziative sono state individuate grazie all'esperienza acquisita durante lo svolgimento del progetto e grazie alle valutazioni effettuate sulle attività svolte. \\
Trattandosi per tutti noi della prima esperienza con un progetto di questa portata, è stato necessario un grande numero di tentativi per comprendere al meglio come organizzarci e come svolgere le attività. Questo ci ha permesso di capire quali sono stati i punti di forza e i punti deboli del nostro lavoro e di individuare le aree in cui è possibile migliorare. \\
Per ciascuna delle difficoltà riscontrate verranno indicate:
\begin{itemize}
    \item fase del progetto in cui si è verificato il problema;
    \item descrizione del problema;
    \item contromisura adottata per risolvere il problema evidenziato.
\end{itemize}

\subsection{Problemi rilevati ed iniziative adottate}
\begin{itemize}
    \item \textbf{Organizzazione delle riunioni}
    \begin{itemize}
        \item \textbf{Fase del progetto}: iniziale;
        \item \textbf{Descrizione}: nelle prime settimane di lavoro, a partire dalla formazione dei gruppi sino ai primi Diari di bordo, si è riscontrata una certa difficoltà nell'organizzazione delle riunioni causata dai vari impegni di ciascun membro (lezioni diverse in orari diversi, lavoro per alcuni, impegni personali) e soprattutto alimentata dalle diverse riunioni che si accumulavano (\href{https://7last.github.io/docs/pb/documentazione-interna/glossario\#stato-avanzamento-lavori}{SAL\textsubscript{G}} con l'azienda prima e Diari di bordo poi) portando a una certa confusione e a un rallentamento delle attività;
        \item \textbf{Contromisura}: abbiamo deciso di effettuare le riunioni a distanza tramite la piattaforma \textit{Discord} e di fissare un giorno e un orario durante la settimana per ciascuna tipologia di incontro in maniera tale da rispettare le disponibilità di ogni membro; qualora qualcuno, per impegni di natura eccezionale, non abbia modo di essere presente potrà successivamente informarsi sui contenuti trattati attraverso i verbali che verranno redatti e messi a disposizione di tutti.
    \end{itemize}
    \item \textbf{Suddivisione compiti}
    \begin{itemize}
        \item \textbf{Fase del progetto}: iniziale;
        \item \textbf{Descrizione}: all'inizio del progetto si è riscontrata una certa difficoltà nella suddivisione dei compiti a causa della mancanza di esperienza e della poca conoscenza delle competenze possedute da ciascuno. È risultato dunque difficile il bilanciamento delle mansioni e si sono verificati più volte casi in cui alcuni membri sono stati in grado di completare le attività a loro assegnate in anticipo, e casi opposti in cui il lavoro da svolgere è risultato eccessivo e difficilmente completabile entro i tempi prestabiliti;
        \item \textbf{Contromisura}: abbiamo quindi deciso, come suggerito anche dal professor Vardanega al primo Diario di bordo, di non assegnare preventivamente tutti i compiti da svolgere a ciascun membro, ma piuttosto di metterli in un contenitore condiviso (abbiamo deciso di usare le annotazioni di \href{https://7last.github.io/docs/pb/documentazione-interna/glossario\#clickup}{\textit{ClickUp}\textsubscript{G}}) e di permettere a ciascun membro di prendere in autonomia i compiti da svolgere, così che chiunque finisca in anticipo possa prenderne altri; in questo modo siamo riusciti a svolgere le attività in modo più equo e a completare i compiti entro i tempi prestabiliti.
    \end{itemize}
    \item \textbf{Familiarità con le tecnologie}
    \begin{itemize}
        \item \textbf{Fase del progetto}: intermedia;
        \item \textbf{Descrizione}: durante lo svolgimento del progetto ci siamo resi conto che la mancanza di familiarità con le tecnologie utilizzate (in particolare con \href{https://7last.github.io/docs/pb/documentazione-interna/glossario\#docker}{\textit{Docker}\textsubscript{G}}, \href{https://7last.github.io/docs/pb/documentazione-interna/glossario\#grafana}{\textit{Grafana}\textsubscript{G}} e \href{https://7last.github.io/docs/pb/documentazione-interna/glossario\#clickhouse}{\textit{Clickhouse}\textsubscript{G}}) ha rallentato inizialmente l'attività di sviluppo e ha portato a un aumento del carico di lavoro per alcuni membri del gruppo;
        \item \textbf{Contromisura}: abbiamo deciso di organizzare un incontro di formazione in cui i membri più esperti hanno spiegato ai meno esperti il funzionamento di \href{https://7last.github.io/docs/pb/documentazione-interna/glossario\#docker}{\textit{Docker}\textsubscript{G}} e le modalità di utilizzo. Inoltre, abbiamo deciso di utilizzare la funzionalità di \textit{pair programming} per permettere ai membri meno esperti di lavorare a stretto contatto con quelli più esperti e di apprendere da loro.
    \end{itemize}
\end{itemize}

\subsection{Considerazioni finali}
Fin da subito il nostro gruppo si è posto come obiettivo principale quello di dotarsi di un \textit{Way of Working} preciso e ben definito, di pianificare ogni singola attività e di prevedere tutte le possibili difficoltà incontrabili durante lo svolgimento del progetto. Questo per cercare di prevenire i problemi e di fornire delle contromisure efficaci per affrontarli. \\
Inizialmente si sono presentate delle difficoltà dovute all'inesperienza del gruppo in ambito organizzativo. Tuttavia, grazie alla familiarizzazione ottenuta tramite lo svolgimento del progetto e grazie ai consigli e suggerimenti che ci sono stati forniti dai professori e dall'azienda \href{https://7last.github.io/docs/pb/documentazione-interna/glossario\#proponente}{proponente\textsubscript{G}}, siamo riusciti a individuare i problemi e a mettere in atto delle contromisure per risolverli. \\
Questo ci ha permesso di migliorare notevolmente la qualità del nostro lavoro e di svolgere le attività in modo più efficiente e più equo. Nonostante ciò siamo anche consapevoli che ci sono ancora molti aspetti su cui possiamo progredire e che ci sono ancora molte iniziative di automiglioramento che possiamo adottare. Siamo convinti che, se continueremo a lavorare con lo stesso impegno e la stessa determinazione che abbiamo dimostrato finora, saremo in grado di ottenere risultati di qualità superiore.