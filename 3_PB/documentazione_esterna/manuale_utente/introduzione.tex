\section{Introduzione}
\setcounter{subsection}{0}
\subsection{Scopo del manuale} %TODO decidere se mettere anche parte di installazione
Il presente manuale ha lo scopo di fornire tutte le informazioni necessarie per l’utilizzo del prodotto SyncCity. È rivolto agli utenti che ne faranno uso, offrendo istruzioni dettagliate per l’installazione e l’utilizzo del sistema. L’obiettivo è guidare l’utente attraverso le funzionalità offerte, assicurando un’esperienza ottimale.

\subsection{Scopo del prodotto}
Lo scopo principale del prodotto è consentire a \textit{Sync Lab S.r.l.} di valutare la fattibilità di investire tempo e risorse nell’implementazione del progetto \textit{\textbf{SyncCity} - A smart city monitoring platform}. Questa soluzione, grazie all’utilizzo di dispositivi IoT, permette un monitoraggio costante delle città. SyncCity avrà l’obiettivo di raccogliere e analizzare dati provenienti da sensori posizionati nelle città, fornendo informazioni utili per la gestione urbana. Il prodotto finale sarà un prototipo funzionale che consentirà la visualizzazione dei dati raccolti su un cruscotto.

% \subsection{Accesso alla piattaforma} %TODO se avremo le credenziali possiamo pensare di inserirla

\subsection{Glossario}
Per evitare qualsiasi ambiguità o malinteso sui termini utilizzati nel documento, verrà adottato un glossario. Questo glossario conterrà varie definizioni. Ogni termine incluso nel glossario sarà indicato applicando uno stile specifico:
\begin{itemize}
    \item aggiungendo una "G" al pedice della parola;
    \item fornendo il link al glossario online;
\end{itemize}

\subsection{Riferimenti}
    \subsubsection{Normativi} %DA SISTEMARE
        \begin{itemize}
            \item 
            \item 
        \end{itemize}
    \subsubsection{Informativi}
        \begin{itemize}%TODO aggiungere link sync city
            \item 
            \item 
        \end{itemize}
    