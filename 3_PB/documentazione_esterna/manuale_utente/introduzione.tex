\section{Introduzione}
\setcounter{subsection}{0}
\subsection{Scopo del manuale} %TODO decidere se mettere anche parte di installazione
Il presente manuale ha lo scopo di fornire tutte le informazioni necessarie per l’utilizzo del prodotto SyncCity. È rivolto agli utenti che ne faranno uso, offrendo istruzioni dettagliate per l’utilizzo del sistema. L’obiettivo è guidare l’utente attraverso le funzionalità offerte, assicurando un’esperienza ottimale.

\subsection{Scopo del progetto}
Lo scopo principale del progetto è consentire a \textit{Sync Lab S.r.l.} di valutare la fattibilità di investire tempo e risorse nell’implementazione del progetto \textit{\textbf{SyncCity} - A smart city monitoring platform}. Questa soluzione, grazie all’utilizzo di dispositivi IoT, permette un monitoraggio costante delle città. SyncCity avrà l’obiettivo di raccogliere e analizzare dati provenienti da sensori posizionati nelle città, fornendo informazioni utili per la gestione urbana e monitoraggio delle condizioni ambientali. Il prodotto finale sarà un prototipo funzionante che consentirà la visualizzazione dei dati raccolti su un cruscotto.

\subsection{Glossario}
Per evitare qualsiasi ambiguità o malinteso sui termini utilizzati nel documento, verrà adottato un glossario. Questo glossario conterrà varie definizioni. Ogni termine incluso nel glossario sarà indicato applicando uno stile specifico:
\begin{itemize}
    \item aggiungendo una "G" al pedice della parola;
    \item fornendo il link al glossario online;
\end{itemize}

\subsection{Riferimenti}
    \subsubsection{Normativi}
        \begin{itemize}
            \item \textbf{Capitolato d'appalto C6}: \textit{SyncCity\textsubscript{G} } – A smart city monitoring platform\\
            \url{https://www.math.unipd.it/~tullio/IS-1/2023/Progetto/C6.pdf}
            \item \textbf{Norme di Progetto v.2.0}\\
            \url{https://7last.github.io/docs/pb/documentazione-interna/norme-di-progetto}
            \item \textbf{Regole del progetto didattico}\\
            \url{https://www.math.unipd.it/~tullio/IS-1/2023/Dispense/PD2.pdf}
        \end{itemize}
    \subsubsection{Informativi}
        \begin{itemize}
            \item \textbf{Analisi dei Requisiti v.2.0}\\
            \url{https://7last.github.io/docs/pb/documentazione-esterna/analisi-dei-requisiti}
            \item \textbf{Specifica Tecnica v.1.0}\\
            \url{https://7last.github.io/docs/pb/documentazione-esterna/specifica-tecnica}
            \item \textbf{Docker} [Ultima consultazione: 2024-07-05]\\ \url{https://docs.docker.com/} 
            \item \textbf{Grafana} [Ultima consultazione: 2024-07-05]\\ \url{https://grafana.com/docs/grafana/latest/}
        \end{itemize}
    