\section{Retrospettiva generale}
Analizzando l'andamento complessivo del progetto, si può affermare che il team ha lavorato in modo efficiente e produttivo, soddisfacendo le aspettative del proponente. La comunicazione tra i membri del team è stata sempre chiara e trasparente, evitando incomprensioni o contrasti. La collaborazione tra i membri del team ha permesso di condividere conoscenze e competenze, migliorando la qualità del prodotto finale e la velocità di sviluppo. La previsione e gestione dei rischi ha permesso di individuarli e mitigarli in modo tempestivo, quasi sempre evitando ritardi e costi aggiuntivi. \\
Siamo riusciti a completare il progetto rispettando il budget preventivato, ottenendo un avanzo di 300 €. Abbiamo anche rispettato le scadenze previste, anticipando notevolmente la scadenza prevista inizialmente in fase di candidatura, nonostante il piccolo ritardo rispetto alla calendarizzazione intermedia, successivamente ripianificata e rispettata. \\
La ripartizione delle ore e la turnazione dei ruoli è stata abbastanza equa e bilanciata, permettendo a tutti i membri del team di partecipare attivamente allo sviluppo del progetto sperimentando ruoli diversi e acquisendo nuove competenze. Dalla suddivisione delle ore emerge una minima differenza nel monte ore complessivo tra i vari membri del gruppo, questo è dovuto ad una maggiore disponibilità di alcuni membri del team rispetto ad altri, in particolare per impegni universitari. Per affrontare queste difficoltà ed evitare ulteriori ritardi, il team ha deciso di rivedere la pianificazione e di ridistribuire le ore tra i vari membri del team, portando a questa differenza.

\begin{table}[!h]
    \centering
    \begin{tabular}{ | l | c | c | c | c | c | c || c | c | }
        \hline
        \textbf{} & \textbf{Re} & \textbf{Am} &\textbf{An} & \textbf{Pj} & \textbf{Pg} & \textbf{Ve} & \textbf{Ore} & \textbf{Costi} \\
        \hline
        Baldo        &    7   &   10   &   10   &   19,5 &   20   &   13,5 &   80   &  1650,00 € \\
        Benetazzo    &    8,5 &   10   &   15   &    6   &   24   &   30,5 &   94   &  1797,50 € \\
        Ferro        &    8   &    7   &   10   &   16   &   25   &   29   &   95   &  1840,00 € \\
        Malgarise    &    9   &    9   &   15,5 &   12,5 &   15   &   21   &   82   &  1690,00 € \\
        Occhinegro   &    7   &    8   &   15,5 &   16   &   23   &   25   &   94,5 &  1877,00 € \\
        Seganfreddo  &    8   &    6,5 &   13   &   20   &   14   &   24   &   85,5 &  1765,00 € \\
        Tiozzo       &    6,5 &    8,5 &    7   &   14   &   30   &   28   &   94   &  1760,00 € \\
        \hline
        Totale ore   &   54   &   59   &   86   &  104   &  151   &  171   &  625   &     -      \\
        \hline
        Totale costi & 1620 € & 1180 € & 2150 € & 2600 € & 2265 € & 2565 € &    -   & 12380,00 € \\
        \hline
    \end{tabular}
    \caption{Consuntivo generale delle ore e dei costi}
\end{table}

\subsection{Aspetti positivi}
\begin{itemize}
    \item \textbf{Automazione}: l'automazione dei processi ha permesso di ridurre i tempi di sviluppo e di test;
    \item \textbf{riunioni frequenti}: le riunioni frequenti hanno permesso di mantenere un alto livello di comunicazione tra i membri del team e il proponente, allineando la visione dello stesso con il prodotto sviluppato;
    \item \textbf{comunicazione}: la comunicazione tra i membri del team è stata sempre chiara e trasparente, evitando incomprensioni e ritardi;
    \item \textbf{collaborazione e formazione}: la collaborazione tra i membri del team ha permesso di condividere conoscenze e competenze, migliorando la qualità del prodotto finale e la velocità di sviluppo;
    \item \textbf{rispetto delle scadenze}: il rispetto delle scadenze ha permesso di mantenere un alto livello di produttività e di soddisfare le aspettative del proponente;
    \item \textbf{gestione dei rischi}: la previsione e gestione dei rischi ha permesso di individuarli e mitigarli in modo tempestivo, evitando ritardi e costi aggiuntivi;
    \item \textbf{autoformazione}: l'autoformazione dei membri del team ha permesso di acquisire nuove competenze e di migliorare la qualità del prodotto finale;
    \item \textbf{continuous integration}: l'utilizzo della continuous integration ha permesso di ridurre i tempi di sviluppo e di test, migliorando la qualità del prodotto finale;  Integrando regolarmente il codice sorgente e sottoponendolo a test
    automatici, sono stati ridotti i conflitti e individuati tempestivamente gli errori,
    garantendo un flusso di sviluppo fluido ed efficiente."
\end{itemize}

\subsection{Aspetti negativi}
\begin{itemize}
    \item \textbf{Compilazione del Piano di Qualifica}
\end{itemize}