\section{Retrospettiva generale}
In questa sezione indicheremo le principali osservazioni e criticità riscontrate durante lo svolgimento del progetto. 

\subsection{RTB}

\subsection{PB}

\subsection{Complessivo}

\subsection{Aspetti positivi}
\begin{itemize}
    \item \textbf{Automazione}: il grado di automazione dei processi raggiunto dal gruppo ha permesso di ridurre le attività ripetitive da dover effettuare manualmente e di conseguenza ha ridotto il rischio di errori. Tra queste attività rientrano la verifica del codice, l'inserimento della "G" al pedice e del link per i termini presenti nel \textit{Glossario}, il controllo ortografico e grammaticale nei documenti, le notifiche per gli eventi che accadono nel repository, il calcolo dell'indice di Gulpease e il merge automatico nel branch principale dei file pdf rimuovendo il codice sorgente;
    \item \textbf{riunioni frequenti e comunicazione}: le riunioni frequenti hanno permesso di mantenere un alto livello di comunicazione, sia interna che con il proponente, allineando la visione dello stesso con il prodotto sviluppato. Inoltre, eccetto alcuni casi, hanno permesso di evitare ritardi nello svolgimento del progetto;
    \item \textbf{collaborazione e formazione}: nei periodi in cui si sono verificati ritardi, la collaborazione tra i membri del team ha permesso di portare a termine nel minor tempo possibile le attività rimanenti. Inoltre, è capitato che alcuni membri del team fossero più preparati su determinati argomenti, potendo così fornire supporto e formazione ai membri meno esperti, diminuendo così i tempi di apprendimento, sviluppo e verifica del prodotto;
    \item \textbf{rispetto delle scadenze e gestione dei rischi}: nonostante alcuni ritardi, il gruppo è riuscito a rispettare la maggior parte delle scadenze prefissate, dimostrando così una buona capacità di pianificazione delle scadenze. Inoltre, la buona previsione e conseguente gestione dei rischi ha permesso di individuarli e mitigarli in modo tempestivo, evitando ulteriori ritardi e costi aggiuntivi indesiderati;
    \item \textbf{autoformazione}: il livello di conoscenze che questo progetto richiede ha indotto i membri del team a compiere un processo di autoformazione per poter avere padronanza sugli strumenti e tecnologie adottate e ha permesso di acquisire nuove competenze che potranno essere utili in futuro. Inoltre, la formazione continua ha permesso di consolidare le conoscenze acquisite e di migliorare il livello di competenza del team, contribuendo così a fornire un prodotto di qualità e avendo un impatto positivo sulle tempistiche di sviluppo;
    \item \textbf{continuous integration}: rilasciando regolarmente delle integrazioni e sottoponendole immediatamente a test, è stato possibile ridurre i tempi di sviluppo e individuare tempestivamente gli errori, garantendo un flusso di sviluppo fluido ed efficace.
\end{itemize}

\subsection{Aspetti negativi}
\begin{itemize}
    \item \textbf{Compilazione del Piano di Qualifica}: non essendo riusciti ad automatizzare il completo calcolo delle metriche presenti nel \textit{Piano di Qualifica}, si è reso necessario un lavoro manuale per il completamento, che ha richiesto tempo e risorse, nonché la possibilità di commettere errori.
\end{itemize}