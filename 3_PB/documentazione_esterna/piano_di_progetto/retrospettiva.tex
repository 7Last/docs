\section{Retrospettiva generale}
Analizzando l'andamento complessivo del progetto, si può affermare che il team ha lavorato in modo efficiente e produttivo, soddisfacendo le aspettative del \href{https://7last.github.io/docs/pb/documentazione-interna/glossario\#proponente}{proponente\textsubscript{G}}. La comunicazione tra i membri del team è stata sempre chiara e trasparente, evitando incomprensioni o contrasti. La collaborazione tra gli stessi ha permesso di condividere conoscenze e competenze, migliorando la qualità del prodotto finale e la velocità di sviluppo. La previsione e gestione dei rischi ha permesso di individuarli e mitigarli in modo tempestivo, quasi sempre evitando ritardi e costi aggiuntivi. \\
Siamo riusciti a completare il progetto rispettando il budget preventivato, ottenendo un avanzo di € 300,00. Inoltre abbiamo rispettato le scadenze previste, anticipando notevolmente la scadenza prevista inizialmente in fase di candidatura, nonostante il piccolo ritardo rispetto alla calendarizzazione intermedia, successivamente ripianificata e rispettata. \\
La ripartizione delle ore e la turnazione dei ruoli è stata abbastanza equa e bilanciata, permettendo a tutti i membri del team di partecipare attivamente allo sviluppo del progetto sperimentando ruoli diversi e acquisendo nuove competenze. Dalla suddivisione delle ore emerge una differenza nel monte ore complessivo tra i vari membri del gruppo, questo è dovuto ad una maggiore disponibilità di alcuni rispetto ad altri, in particolare per impegni universitari. Per affrontare queste difficoltà ed evitare ulteriori ritardi, il team ha deciso di rivedere la pianificazione e di ridistribuire le ore, portando a questa differenza.

\begin{table}[!h]
    \centering
    \begin{tabular}{ | l | c | c | c | c | c | c || c | c | }
        \hline
        \textbf{} & \textbf{Re} & \textbf{Am} &\textbf{An} & \textbf{Pj} & \textbf{Pg} & \textbf{Ve} & \textbf{Ore} & \textbf{Costi} \\
        \hline
        Baldo        &    7   &   10   &   10   &   19,5 &   20   &   13,5 &   80   &  1650,00 € \\
        Benetazzo    &    8,5 &   10   &   15   &    6   &   24   &   30,5 &   94   &  1797,50 € \\
        Ferro        &    8   &    7   &   10   &   16   &   25   &   29   &   95   &  1840,00 € \\
        Malgarise    &    9   &    9   &   15,5 &   12,5 &   15   &   21   &   82   &  1690,00 € \\
        Occhinegro   &    7   &    8   &   15,5 &   16   &   23   &   25   &   94,5 &  1877,00 € \\
        Seganfreddo  &    8   &    6,5 &   13   &   20   &   14   &   24   &   85,5 &  1765,00 € \\
        Tiozzo       &    6,5 &    8,5 &    7   &   14   &   30   &   28   &   94   &  1760,00 € \\
        \hline
        Totale ore   &   54   &   59   &   86   &  104   &  151   &  171   &  625   &     -      \\
        \hline
        Totale costi & 1620 € & 1180 € & 2150 € & 2600 € & 2265 € & 2565 € &    -   & 12380,00 € \\
        \hline
    \end{tabular}
    \caption{Consuntivo generale delle ore e dei costi}
\end{table}

\newpage

%---------1_ISTOGRAMMA-----------%
\begin{figure}[!h]
    \centering
    \begin{tikzpicture}
        \begin{axis}[
            width  = 0.85*\textwidth,
            height = 8cm,
            ybar stacked,
            bar width=14pt,
            ymajorgrids = true,
            symbolic x coords={Baldo, Benetazzo, Ferro, Malgarise, Occhinegro, Seganfreddo, Tiozzo},
            xtick = data,
            scaled y ticks = false,
            enlarge x limits=0.2,
            ymin=0,
            legend cell align=left,
            legend style={
                at={(0.5,1.15)},
                anchor=south,
                column sep=1ex,
                legend columns=-1
            },
            xticklabel style={rotate=45, anchor=north east, yshift=0ex, xshift=0ex},
            ]
            \addplot+[ybar, resp, fill=resp, mark=none] plot coordinates {
                (Baldo,        7.0)
                (Benetazzo,    8.5)
                (Ferro,        8.0)
                (Malgarise,    9.0)
                (Occhinegro,   7.0)
                (Seganfreddo,  8.0)
                (Tiozzo,       6.5)
            };
            \addplot+[ybar, amm, fill=amm, mark=none] plot coordinates {
                (Baldo,       10.0)
                (Benetazzo,   10.0)
                (Ferro,        7.0)
                (Malgarise,    9.0)
                (Occhinegro,   8.0)
                (Seganfreddo,  6.5)
                (Tiozzo,       8.5)
            };
            \addplot+[ybar, an, fill=an, mark=none] plot coordinates {
                (Baldo,       10.0)
                (Benetazzo,   15.0)
                (Ferro,       10.0)
                (Malgarise,   15.5)
                (Occhinegro,  15.5)
                (Seganfreddo, 13.0)
                (Tiozzo,       7.0)
            };
            \addplot+[ybar, pj, fill=pj, mark=none] plot coordinates {
                (Baldo,       19.5)
                (Benetazzo,    6.0)
                (Ferro,       16.0)
                (Malgarise,   12.5)
                (Occhinegro,  16.0)
                (Seganfreddo, 20.0)
                (Tiozzo,      14.0)
            };
            \addplot+[ybar, pg, fill=pg, mark=none] plot coordinates {
                (Baldo,       20.0)
                (Benetazzo,   24.0)
                (Ferro,       25.0)
                (Malgarise,   15.0)
                (Occhinegro,  23.0)
                (Seganfreddo, 14.0)
                (Tiozzo,      30.0)
            };
            \addplot+[ybar, ver, fill=ver, mark=none] plot coordinates {
                (Baldo,       13.5)
                (Benetazzo,   30.5)
                (Ferro,       29.0)
                (Malgarise,   21.0)
                (Occhinegro,  25.0)
                (Seganfreddo, 24.0)
                (Tiozzo,      28.0)
            };
            \legend{Re, Am, An, Pj, Pg, Ve}
        \end{axis}
    \end{tikzpicture}
    \caption{Ripartizione delle ore totali per componente}
    
\end{figure}

%---------1_GRAFICO A TORTA-----------%
\begin{figure}[!h]
    \centering
    \begin{tikzpicture}
        \def\printonlypositive#1{\ifdim#1pt>0pt
        #1
        \fi}
        \pie[pos={8,0},radius=3.5,text=legend,
        before number=\printonlypositive,color={resp,amm, an, pj,pg,ver}] {
             8.6/\href{https://7last.github.io/docs/pb/documentazione-interna/glossario\#responsabile}{Responsabile\textsubscript{G}},
             9.4/\href{https://7last.github.io/docs/pb/documentazione-interna/glossario\#amministratore}{Amministratore\textsubscript{G}},
            13.8/\href{https://7last.github.io/docs/pb/documentazione-interna/glossario\#analista}{Analista\textsubscript{G}},
            16.6/\href{https://7last.github.io/docs/pb/documentazione-interna/glossario\#progettista}{Progettista\textsubscript{G}},
            24.2/\href{https://7last.github.io/docs/pb/documentazione-interna/glossario\#programmatore}{Programmatore\textsubscript{G}},
            27.4/\href{https://7last.github.io/docs/pb/documentazione-interna/glossario\#verificatore}{Verificatore\textsubscript{G}}
        }
        \end{tikzpicture}
    \caption{Ripartizione delle ore totali per ruolo}
\end{figure}

\newpage

%---------1_ISTOGRAMMA-----------%
\begin{figure}[!h]
    \centering
    \begin{tikzpicture}
        \begin{axis}[
            width  = 0.85*\textwidth,
            height = 8cm,
            ybar stacked,
            bar width=14pt,
            ymajorgrids = true,
            symbolic x coords={Baldo, Benetazzo, Ferro, Malgarise, Occhinegro, Seganfreddo, Tiozzo},
            xtick = data,
            scaled y ticks = false,
            enlarge x limits=0.2,
            ymin=0,
            legend cell align=left,
            legend style={
                at={(0.5,1.15)},
                anchor=south,
                column sep=1ex,
                legend columns=-1
            },
            xticklabel style={rotate=45, anchor=north east, yshift=0ex, xshift=0ex},
            ]
            \addplot+[ybar, resp, fill=resp, mark=none] plot coordinates {
                (Baldo,       210.0)
                (Benetazzo,   255.0)
                (Ferro,       240.0)
                (Malgarise,   270.0)
                (Occhinegro,  210.0)
                (Seganfreddo, 240.0)
                (Tiozzo,      195.0)
            };
            \addplot+[ybar, amm, fill=amm, mark=none] plot coordinates {
                (Baldo,       200.0)
                (Benetazzo,   200.0)
                (Ferro,       140.0)
                (Malgarise,   180.0)
                (Occhinegro,  160.0)
                (Seganfreddo, 130.0)
                (Tiozzo,      170.0)
            };
            \addplot+[ybar, an, fill=an, mark=none] plot coordinates {
                (Baldo,       250.0)
                (Benetazzo,   375.0)
                (Ferro,       250.0)
                (Malgarise,   387.5)
                (Occhinegro,  387.5)
                (Seganfreddo, 325.0)
                (Tiozzo,      175.0)
            };
            \addplot+[ybar, pj, fill=pj, mark=none] plot coordinates {
                (Baldo,       487.5)
                (Benetazzo,   150.0)
                (Ferro,       400.0)
                (Malgarise,   312.5)
                (Occhinegro,  400.0)
                (Seganfreddo, 500.0)
                (Tiozzo,      350.0)
            };
            \addplot+[ybar, pg, fill=pg, mark=none] plot coordinates {
                (Baldo,       300.0)
                (Benetazzo,   360.0)
                (Ferro,       375.0)
                (Malgarise,   225.0)
                (Occhinegro,  345.0)
                (Seganfreddo, 210.0)
                (Tiozzo,      450.0)
            };
            \addplot+[ybar, ver, fill=ver, mark=none] plot coordinates {
                (Baldo,       202.5)
                (Benetazzo,   457.5)
                (Ferro,       435.0)
                (Malgarise,   315.0)
                (Occhinegro,  375.0)
                (Seganfreddo, 360.0)
                (Tiozzo,      420.0)
            };
            \legend{Re, Am, An, Pj, Pg, Ve}
        \end{axis}
    \end{tikzpicture}
    \caption{Ripartizione dei costi totali per componente}
    
\end{figure}

%---------1_GRAFICO A TORTA-----------%
\begin{figure}[!h]
    \centering
    \begin{tikzpicture}
        \def\printonlypositive#1{\ifdim#1pt>0pt
        #1
        \fi}
        \pie[pos={8,0},radius=3.5,text=legend,
        before number=\printonlypositive,color={resp,amm, an, pj,pg,ver}] {
            13.1/\href{https://7last.github.io/docs/pb/documentazione-interna/glossario\#responsabile}{Responsabile\textsubscript{G}},
             9.5/\href{https://7last.github.io/docs/pb/documentazione-interna/glossario\#amministratore}{Amministratore\textsubscript{G}},
            17.4/\href{https://7last.github.io/docs/pb/documentazione-interna/glossario\#analista}{Analista\textsubscript{G}},
            21.0/\href{https://7last.github.io/docs/pb/documentazione-interna/glossario\#progettista}{Progettista\textsubscript{G}},
            18.3/\href{https://7last.github.io/docs/pb/documentazione-interna/glossario\#programmatore}{Programmatore\textsubscript{G}},
            20.7/\href{https://7last.github.io/docs/pb/documentazione-interna/glossario\#verificatore}{Verificatore\textsubscript{G}}
        }
        \end{tikzpicture}
    \caption{Ripartizione dei costi totali per ruolo}
\end{figure}

\subsection{Aspetti positivi}
\begin{itemize}
    \item \textbf{Automazione}: il grado di automazione dei processi raggiunto dal gruppo ha permesso di ridurre le attività ripetitive da dover effettuare manualmente e di conseguenza ha ridotto il rischio di errori. Tra queste attività rientrano la verifica del codice, l'inserimento della "G" al pedice e del link per i termini presenti nel \href{https://7last.github.io/docs/pb/documentazione-interna/glossario\#glossario}{\textit{Glossario}\textsubscript{G}}, il controllo ortografico e grammaticale nei documenti, le notifiche per gli eventi che accadono nel repository, il calcolo dell'indice di Gulpease e il merge automatico nel branch principale dei file pdf rimuovendo il codice sorgente;
    \item \textbf{riunioni frequenti e comunicazione}: le riunioni frequenti hanno permesso di mantenere un alto livello di comunicazione, sia interna che con il \href{https://7last.github.io/docs/pb/documentazione-interna/glossario\#proponente}{proponente\textsubscript{G}}, allineando la visione dello stesso con il prodotto sviluppato. Inoltre, eccetto alcuni casi, hanno permesso di evitare ritardi nello svolgimento del progetto;
    \item \textbf{collaborazione e formazione}: nei periodi in cui si sono verificati ritardi, la collaborazione tra i membri del team ha permesso di portare a termine nel minor tempo possibile le attività rimanenti. Inoltre, è capitato che alcuni membri del team fossero più preparati su determinati argomenti, potendo così fornire supporto e formazione ai membri meno esperti, diminuendo così i tempi di apprendimento, sviluppo e verifica del prodotto;
    \item \textbf{rispetto delle scadenze e gestione dei rischi}: nonostante alcuni ritardi, il gruppo è riuscito a rispettare la maggior parte delle scadenze prefissate, dimostrando così una buona capacità di pianificazione delle scadenze. Inoltre, la buona previsione e conseguente gestione dei rischi ha permesso di individuarli e mitigarli in modo tempestivo, evitando ulteriori ritardi e costi aggiuntivi indesiderati;
    \item \textbf{autoformazione}: il livello di conoscenze che questo progetto richiede ha indotto i membri del team a compiere un processo di autoformazione per poter avere padronanza sugli strumenti e tecnologie adottate e ha permesso di acquisire nuove competenze che potranno essere utili in futuro. Inoltre, la formazione continua ha permesso di consolidare le conoscenze acquisite e di migliorare il livello di competenza del team, contribuendo così a fornire un prodotto di qualità e avendo un impatto positivo sulle tempistiche di sviluppo;
    \item \textbf{continuous integration}: rilasciando regolarmente delle integrazioni e sottoponendole immediatamente a test, è stato possibile ridurre i tempi di sviluppo e individuare tempestivamente gli errori, garantendo un flusso di sviluppo fluido ed efficace.
\end{itemize}

\subsection{Aspetti negativi}
\begin{itemize}
    \item \textbf{Compilazione del \href{https://7last.github.io/docs/pb/documentazione-interna/glossario\#piano-di-qualifica}{Piano di Qualifica\textsubscript{G}}}: non essendo riusciti ad automatizzare il completo calcolo delle metriche presenti nel \href{https://7last.github.io/docs/pb/documentazione-interna/glossario\#piano-di-qualifica}{\textit{Piano di Qualifica}\textsubscript{G}}, si è reso necessario un lavoro manuale per il completamento, che ha richiesto tempo e risorse, nonché la possibilità di commettere errori.
\end{itemize}