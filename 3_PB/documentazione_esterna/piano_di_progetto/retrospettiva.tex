\section{Retrospettiva generale}
In questa sezione indicheremo le principali osservazioni e criticità riscontrate durante lo svolgimento del progetto. 

\subsection{RTB}

\subsection{PB}

\subsection{Complessivo}

\subsection{Aspetti positivi}
\begin{itemize}
    \item \textbf{Automazione}: l'automazione dei processi ha permesso di ridurre i tempi di sviluppo e di test;
    \item \textbf{riunioni frequenti}: le riunioni frequenti hanno permesso di mantenere un alto livello di comunicazione tra i membri del team e il proponente, allineando la visione dello stesso con il prodotto sviluppato;
    \item \textbf{comunicazione}: la comunicazione tra i membri del team è stata sempre chiara e trasparente, evitando incomprensioni e ritardi;
    \item \textbf{collaborazione e formazione}: la collaborazione tra i membri del team ha permesso di condividere conoscenze e competenze, migliorando la qualità del prodotto finale e la velocità di sviluppo;
    \item \textbf{rispetto delle scadenze}: il rispetto delle scadenze ha permesso di mantenere un alto livello di produttività e di soddisfare le aspettative del proponente;
    \item \textbf{gestione dei rischi}: la previsione e gestione dei rischi ha permesso di individuarli e mitigarli in modo tempestivo, evitando ritardi e costi aggiuntivi;
    \item \textbf{autoformazione}: l'autoformazione dei membri del team ha permesso di acquisire nuove competenze e di migliorare la qualità del prodotto finale;
    \item \textbf{continuous integration}: l'utilizzo della continuous integration ha permesso di ridurre i tempi di sviluppo e di test, migliorando la qualità del prodotto finale;  Integrando regolarmente il codice sorgente e sottoponendolo a test
    automatici, sono stati ridotti i conflitti e individuati tempestivamente gli errori,
    garantendo un flusso di sviluppo fluido ed efficiente."
\end{itemize}

\subsection{Aspetti negativi}
\begin{itemize}
    \item \textbf{Compilazione del Piano di Qualifica}
\end{itemize}