\section{Analisi dei rischi}
È essenziale ridurre l'impatto delle difficoltà che possono sorgere durante il progetto mediante un'adeguata \textit{analisi dei rischi}. o
Questa sezione è stata inclusa nel documento per prevenire che eventuali problemi possano compromettere il successo del progetto. Dopo aver elencato i rischi, viene definita una serie di azioni da intraprendere qualora uno di essi si manifesti. Secondo lo standard \textit{ISO/IEC 31000:2009}, il processo di gestione del rischio si articola in cinque fasi, di seguito elencate:
\begin{itemize}
	\item \textbf{valutazione del rischio}, composta da:
	      \begin{itemize}
		      \item \textbf{identificazione del rischio}: identificare rischi specifici del progetto software, come errori di codifica, problemi di integrazione, fallimenti nelle specifiche, scadenze non rispettate, cambiamenti nei requisiti, vulnerabilità di sicurezza, ecc;
		      \item \textbf{analisi del rischio}: valutare la probabilità e l'impatto di ciascun rischio identificato;
		      \item \textbf{valutazione del rischio}: confrontare i rischi analizzati con i criteri di accettabilità stabiliti per determinare quali rischi necessitano di interventi;
	      \end{itemize}
	\item \textbf{trattamento del rischio}: definire piani d'azione per affrontare i rischi, come l'adozione di pratiche di codifica sicura, l'implementazione di test approfonditi, la revisione del codice, l'utilizzo di strumenti di gestione dei requisiti e la formazione del personale;
	\item \textbf{monitoraggio del rischio}: queste attività richiedono di monitorare continuamente i rischi e l'efficacia delle misure di trattamento implementate;
	\item \textbf{comunicazione del rischio}: è importante comunicare i rischi identificati e le misure di trattamento adottate a tutte le parti interessate.
\end{itemize}
I fattori chiave per l'identificazione dei rischi sono:
\begin{itemize}
	\item la \textbf{tipologia} che rappresenta la categoria di rischio, la quale può essere organizzativa, tecnologica o comunicativa;
	\item l'\textbf{indice}, un valore numerico incrementale che identifica univocamente il rischio per ogni tipologia.
	\item la \textbf{probabilità} che rappresenta la possibilità che un rischio si verifichi, la quale può essere bassa, media o alta;
	\item la \textbf{pericolosità} che rappresenta l'impatto che un rischio può avere sul progetto, la quale può essere bassa, media o alta;
	\item il \textbf{rilevamento} che rappresenta il modo in cui il rischio può essere identificato;
	\item la \textbf{mitigazione} che rappresenta le azioni da intraprendere qualora il rischio si manifesti.
\end{itemize}
Per una rappresentazione schematica dei rischi, si è deciso di attuare la seguente convenzione: \textbf{R[tipologia]-[indice]}.

%---------SEZIONE DEI RISCHI ORGANIZZATIVI-----------%

\subsection{Rischi organizzativi}

\subsubsubsection*{RO-1 - Inesperienza dei membri del team nella pianificazione delle attività}

\begin{longtable}{ | l | p{12cm} | }
	\hline
	\textbf{Descrizione}  & L'inesperienza del team nella pianificazione delle attività rappresenta un rischio significativo che può influenzare negativamente il successo del progetto. Questo rischio emerge quando il team manca di esperienza o competenza nella gestione dei processi di pianificazione, portando a possibili conseguenze come l'allocazione inefficace delle risorse, la scarsa definizione dei requisiti, valutazioni inaccurate dei tempi e la mancanza di piani di contingenza. \\
	\hline
	\textbf{Probabilità}  & Alta.                                                                                                                                                                                                                                                                                                                                                                                                                                                                        \\
	\hline
	\textbf{Pericolosità} & Alta.                                                                                                                                                                                                                                                                                                                                                                                                                                                                        \\
	\hline
	\textbf{Rilevamento}  & Monitoraggio continua di \href{https://7last.github.io/docs/pb/documentazione-interna/glossario\#clickup}{ClickUp\textsubscript{G}} e del \href{https://7last.github.io/docs/pb/documentazione-interna/glossario#piano-di-progetto}{\textit{Piano di progetto}\textsubscript{G}}.                                                                                                                                                                                                                                                                                              \\
	\hline
	\textbf{Mitigazione}  & In caso di difficoltà o ritardi, il \href{https://7last.github.io/docs/pb/documentazione-interna/glossario#piano-di-progetto}{\textit{piano di progetto}\textsubscript{G}} viene rivisto per allineare le attività ai progressi. Se un membro segnala difficoltà nel rispettare una scadenza, al \href{https://7last.github.io/docs/pb/documentazione-interna/glossario\#responsabile}{responsabile\textsubscript{G}} il compito di assegnare più risorse o, in casi più gravi, spostare la scadenza.                                                                               \\
	\hline
	\caption{RO-1 - Inesperienza dei membri del team nella pianificazione delle attività}
\end{longtable}

\newpage

\subsubsubsection*{RO-2 - Impegni personali o universitari}

\begin{longtable}{ | l | p{12cm} | }
	\hline
	\textbf{Descrizione}  & Gli impegni personali e/o universitari possono limitare la disponibilità di uno o più membri del gruppo.                                                                                                                                                           \\
	\hline
	\textbf{Probabilità}  & Media.                                                                                                                                                                                                                                                             \\
	\hline
	\textbf{Pericolosità} & Bassa.                                                                                                                                                                                                                                                             \\
	\hline
	\textbf{Rilevamento}  & Condividendo i propri impegni e indicando la disponibilità, i membri possono concordare momenti della settimana per tenere le riunioni e comprendere lo stato di sviluppo del progetto da parte di ciascun membro.                                                 \\
	\hline
	\textbf{Mitigazione}  & Il compito del \href{https://7last.github.io/docs/pb/documentazione-interna/glossario\#responsabile}{responsabile\textsubscript{G}} è quello di rivedere la suddivisione dei ruoli e compiti in base agli impegni di ciascun membro. In casi gravi, le scadenze devono essere spostate e la pianificazione deve essere rivista se non tiene conto di questi inconvenienti. \\
	\hline
	\caption{RO-2 - Impegni personali o universitari}
\end{longtable}

\subsubsubsection*{RO-3 - Ritardi rispetto alle tempistiche previste}

\begin{longtable}{ | l | p{12cm} | }
	\hline
	\textbf{Descrizione}  & La sottostima/sovrastima dei costi orari delle attività, dovuta alla mancanza di esperienza del team, può causare ritardi, perdite di tempo e di risorse.                                                                                                          \\
	\hline
	\textbf{Probabilità}  & Bassa.                                                                                                                                                                                                                                                             \\
	\hline
	\textbf{Pericolosità} & Alta.                                                                                                                                                                                                                                                              \\
	\hline
	\textbf{Rilevamento}  & Attraverso il \href{https://7last.github.io/docs/pb/documentazione-interna/glossario\#cruscotto}{cruscotto\textsubscript{G}} e confronto periodico con il \href{https://7last.github.io/docs/pb/documentazione-interna/glossario#piano-di-progetto}{\textit{Piano di Progetto}\textsubscript{G}}, il \href{https://7last.github.io/docs/pb/documentazione-interna/glossario\#responsabile}{Responsabile\textsubscript{G}} può monitorare lo stato di avanzamento del progetto. \\
	\hline
	\textbf{Mitigazione}  & In caso di modifiche non gravi, cerchiamo di implementare rapidamente ciò che viene lasciato in sospeso. Se sono significative, discutiamo con il \href{https://7last.github.io/docs/pb/documentazione-interna/glossario\#proponente}{proponente\textsubscript{G}} per trovare un accordo su come gestire le modifiche e affrontare i cambiamenti.                       \\
	\hline
	\caption{RO-3 - Ritardi rispetto alle tempistiche previste}
\end{longtable}

\newpage

\subsubsubsection*{RO-4 - Scarsa collaborazione da parte di uno o più membri}

\begin{longtable}{ | l | p{12cm} | }
	\hline
	\textbf{Descrizione}  & La possibilità che uno o più membri del gruppo non collaborino attivamente allo sviluppo del progetto.                                                                                                                                                                           \\
	\hline
	\textbf{Probabilità}  & Media.                                                                                                                                                                                                                                                                           \\
	\hline
	\textbf{Pericolosità} & Alta.                                                                                                                                                                                                                                                                            \\
	\hline
	\textbf{Rilevamento}  & Contando le volte che un membro è assente, dopo la quinta volta viene attivato un rapporto interno al team.                                                                                                                                                                      \\
	\hline
	\textbf{Mitigazione}  & È compito dell'\href{https://7last.github.io/docs/pb/documentazione-interna/glossario\#amministratore}{amministratore\textsubscript{G}} comunicare la situazione alla persona interessata e invitarla a partecipare attivamente allo sviluppo. In caso di esito negativo, il compito del manager è quello di assegnare maggiori risorse o, nei casi più gravi, di posticipare la scadenza. \\
	\hline
	\caption{RO-4 - Scarsa collaborazione da parte di uno o più membri}
\end{longtable}

%--------------------------------------------------%

%---------SEZIONE DEI RISCHI TECNOLOGICI-----------%

\subsection{Rischi tecnologici}

\subsubsubsection*{RT-1 - Inesperienza nell'uso delle tecnologie adottate}

\begin{longtable}{ | l | p{12cm} | }
	\hline
	\textbf{Descrizione}  & Alcuni membri del gruppo potrebbero dover acquisire le competenze necessarie allo sviluppo del progetto. Questo potrebbe causare ritardi sia nella fase di progettazione che in quella di sviluppo. \\
	\hline
	\textbf{Probabilità}  & Alta.                                                                                                                                                                                               \\
	\hline
	\textbf{Pericolosità} & Alta.                                                                                                                                                                                               \\
	\hline
	\textbf{Rilevamento}  & Dopo aver compreso le competenze di ciascun membro del team, il \href{https://7last.github.io/docs/pb/documentazione-interna/glossario\#responsabile}{responsabile\textsubscript{G}} deve assegnare i compiti in modo che non siano troppo facili, ma nemmeno troppo difficili per ciascun membro.          \\
	\hline
	\textbf{Mitigazione}  & Se i membri del gruppo incontrano difficoltà nello svolgimento di un'attività, saranno assistiti da un membro con maggiore esperienza in quell'ambito.                                              \\
	\hline
	\caption{RT-1 - Inesperienza nell'uso delle tecnologie adottate}
\end{longtable}

\newpage

\subsubsubsection*{RT-2 - Perdita di informazioni}

\begin{longtable}{ | l | p{12cm} | }
	\hline
	\textbf{Descrizione}  & La perdita di informazioni rappresenta un rischio di impatto importante per il progetto. Può verificarsi in caso di guasti hardware, errori umani o malfunzionamenti dei sistemi utilizzati. \\
	\hline
	\textbf{Probabilità}  & Media.                                                                                                                                                                                       \\
	\hline
	\textbf{Pericolosità} & Alta.                                                                                                                                                                                        \\
	\hline
	\textbf{Rilevamento}  & Attraverso il monitoraggio continua dei sistemi utilizzati.                                                                                                                                  \\
	\hline
	\textbf{Mitigazione}  & In caso perdita di informazioni, è necessario poter reperire quelle di riserva, tramite un backup.                                                                                           \\
	\hline
	\caption{RT-2 - Perdita di informazioni}
\end{longtable}

\subsubsubsection*{RT-3 - Problemi di compatibilità tra le tecnologie utilizzate}

\begin{longtable}{ | l | p{12cm} | }
	\hline
	\textbf{Descrizione}  & Le tecnologie adottate (framework, librerie, strumenti) potrebbero non essere completamente compatibili tra loro, causando errori di integrazione, malfunzionamenti e inefficienze. \\
	\hline
	\textbf{Probabilità}  & Alta.\\
	\hline
	\textbf{Pericolosità} & Alta.\\
	\hline
	\textbf{Rilevamento}  & Solo al momento dell'utilizzo di queste tecnologie il team potrà scoprire se si verificheranno problemi di compatibilità o meno.\\
	\hline
	\textbf{Mitigazione}  & In caso di malfunzionamenti, il \href{https://7last.github.io/docs/pb/documentazione-interna/glossario\#responsabile}{responsabile\textsubscript{G}} deve allocare le risorse necessarie per risolverli nel più breve tempo possibile.\\
	\hline
	\caption{RT-3 - Problemi di compatibilità tra le tecnologie utilizzate}
\end{longtable}

%--------------------------------------------------%

\newpage

%---------SEZIONE DEI RISCHI COMUNICATIVI-----------%

\subsection{Rischi comunicativi}

\subsubsubsection*{RC-1 - Disaccordi all'interno del gruppo}

\begin{longtable}{ | l | p{12cm} | }
	\hline
	\textbf{Descrizione}  & Le differenze all'interno del gruppo possono derivare da ideologie e opinioni diverse tra i suoi membri.              \\
	\hline
	\textbf{Probabilità}  & Media.                                                                                                                \\
	\hline
	\textbf{Pericolosità} & Alta.                                                                                                                 \\
	\hline
	\textbf{Rilevamento}  & Possono essere identificate attraverso le loro opinioni espresse o osservando le dinamiche del gruppo.                \\
	\hline
	\textbf{Mitigazione}  & In caso di disaccordo, si procederà a una votazione democratica e si attuerà l'opzione con il maggior numero di voti. \\
	\hline
	\caption{RC-1 - Disaccordi all'interno del gruppo}
\end{longtable}

\subsubsubsection*{RC-2 - Problemi di comunicazione}

\begin{longtable}{ | l | p{12cm} | }
	\hline
	\textbf{Descrizione}  & Una comunicazione inefficace può causare ritardi, stress e disaccordo interno al gruppo.                                                                        \\
	\hline
	\textbf{Probabilità}  & Media.                                                                                                                                                          \\
	\hline
	\textbf{Pericolosità} & Alta.                                                                                                                                                           \\
	\hline
	\textbf{Rilevamento}  & Questo può essere identificato attraverso sondaggi, feedback e comportamenti da parte dei membri del gruppo durante le riunioni o comunicazioni via messaggio.  \\
	\hline
	\textbf{Mitigazione}  & Il \href{https://7last.github.io/docs/pb/documentazione-interna/glossario\#responsabile}{responsabile\textsubscript{G}} ha il compito di promuovere una comunicazione attiva, organizzare riunioni regolari, indagare sulle cause del disaccordo e ricercare soluzioni. \\
	\hline
	\caption{RC-2 - Problemi di comunicazione}
\end{longtable}
