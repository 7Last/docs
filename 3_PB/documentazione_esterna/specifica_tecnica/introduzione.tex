\section{Introduzione}
\setcounter{subsection}{0}
\subsection{Scopo della specifica tecnica}
Questo documento è rivolto a tutti gli \textit{stakeholder} coinvolti nel progetto \textit{\textbf{SyncCity} - A smart city monitoring platform}.
Esso ha lo scopo di fornire una visione dettagliata riguardo l'architettura del sistema, i \textit{design pattern} utilizzati, le tecnologie adottate e le scelte progettuali effettuate.
Inoltre, contiene diagrammi UML delle classi e delle attività.

\subsection{Scopo del prodotto}
Lo scopo del prodotto è realizzare un prototipo di una piattaforma di monitoraggio per una \textit{Smart City}, la quale permetta di raccogliere e analizzare dati provenienti da sensori IoT posizionati nelle città.
Questi dati, una volta elaborati, devono essere visualizzati in maniera chiara e intuitiva, tramite grafici e mappe, per permettere alle autorità locali della città di prendere decisioni
tempestive e mirate per migliorare la qualità della vita dei cittadini.

\subsection{Glossario}
Per evitare qualsiasi ambiguità o malinteso sui termini utilizzati nel documento, verrà adottato un glossario. Questo glossario conterrà varie definizioni. Ogni termine incluso nel glossario sarà indicato applicando uno stile specifico:
\begin{itemize}
	\item aggiungendo una "G" al pedice della parola;
	\item fornendo il link al glossario online.
\end{itemize}

\subsection{Riferimenti}
\subsubsection{Normativi} %TODO: da sistemare
\begin{itemize}
	\item
	\item
\end{itemize}
\subsubsection{Informativi}
\begin{itemize} %TODO da sistemare
	\item
	\item
\end{itemize}
