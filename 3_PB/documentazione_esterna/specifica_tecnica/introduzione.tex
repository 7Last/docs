\section{Introduzione}
\setcounter{subsection}{0}
\subsection{Scopo della specifica tecnica} 
Fornisce una descrizione dettagliata dei requisiti tecnici e delle funzionalità del sistema software in fase di sviluppo. È un punto di riferimento per gli sviluppatori e gli stakeholder, delineando le specifiche necessarie per la realizzazione del prodotto finale. Include una descrizione delle funzionalità del software, i requisiti hardware e software, le specifiche dei componenti, i criteri di performance e sicurezza, e le linee guida per la verifica e la validazione del sistema.

\subsection{Scopo del prodotto}
Lo scopo principale del prodotto è consentire a \textit{Sync Lab S.r.l.} di valutare la fattibilità di investire tempo e risorse nell’implementazione del progetto \textit{\textbf{SyncCity} - A smart city monitoring platform}. Questa soluzione, grazie all’utilizzo di dispositivi IoT, permette un monitoraggio costante delle città. SyncCity avrà l’obiettivo di raccogliere e analizzare dati provenienti da sensori posizionati nelle città, fornendo informazioni utili per la gestione urbana. Il prodotto finale sarà un prototipo funzionale che consentirà la visualizzazione dei dati raccolti su un cruscotto.

\subsection{Glossario}
Per evitare qualsiasi ambiguità o malinteso sui termini utilizzati nel documento, verrà adottato un glossario. Questo glossario conterrà varie definizioni. Ogni termine incluso nel glossario sarà indicato applicando uno stile specifico:
\begin{itemize}
    \item aggiungendo una "G" al pedice della parola;
    \item fornendo il link al glossario online.
\end{itemize}

\subsection{Riferimenti}
\subsubsection{Normativi} %TODO: da sistemare
\begin{itemize}
    \item 
    \item 
\end{itemize}
\subsubsection{Informativi}
\begin{itemize} %TODO da sistemare
    \item 
    \item 
\end{itemize}
