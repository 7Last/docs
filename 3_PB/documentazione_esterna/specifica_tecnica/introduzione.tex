\section{Introduzione}
\setcounter{subsection}{0}
\subsection{Scopo della specifica tecnica}
Questo documento è rivolto a tutti gli \textit{stakeholder} coinvolti nel progetto \textit{\textbf{SyncCity} - A smart city monitoring platform}.
Esso ha lo scopo di fornire una visione dettagliata riguardo l'architettura del sistema, i \textit{design pattern} utilizzati, le tecnologie adottate e le scelte progettuali effettuate.
Inoltre, contiene diagrammi UML delle classi e delle attività.

\subsection{Scopo del prodotto}
Lo scopo del prodotto è realizzare un prototipo di una piattaforma di monitoraggio per una \textit{Smart City}, la quale permetta di raccogliere e analizzare dati provenienti da sensori IoT posizionati nelle città.
Questi dati, una volta elaborati, devono essere visualizzati in maniera chiara e intuitiva, tramite grafici e mappe, per permettere alle autorità locali della città di prendere decisioni
tempestive e mirate per migliorare la qualità della vita dei cittadini.

\subsection{Glossario}
Per evitare qualsiasi ambiguità o malinteso sui termini utilizzati nel documento, verrà adottato un glossario. Questo glossario conterrà varie definizioni. Ogni termine incluso nel glossario sarà indicato applicando uno stile specifico:
\begin{itemize}
	\item aggiungendo una "G" al pedice della parola;
	\item fornendo il link al glossario online.
\end{itemize}

\subsection{Riferimenti}
\subsubsection{Normativi} %TODO: da sistemare
\begin{itemize}
	\item Standard \textbf{ISO 8601}: \url{https://www.iso.org/iso-8601-date-and-time-format.html};
	\item \href{https://7last.github.io/docs/rtb/documentazione-interna/glossario\#capitolato}{\textbf{Capitolato\textsubscript{G} d'appalto C6}}: \href{https://7last.github.io/docs/rtb/documentazione-interna/glossario\#synccity}{\textit{SyncCity\textsubscript{G} } – A \href{https://7last.github.io/docs/rtb/documentazione-interna/glossario\#smart-city}{smart city\textsubscript{G}} monitoring platform}\\
	      \url{https://www.math.unipd.it/~tullio/IS-1/2023/Progetto/C6.pdf}
	\item \textbf{Regolamento di progetto didattico}\\
	      \url{https://www.math.unipd.it/~tullio/IS-1/2023/Dispense/PD2.pdf}
	\item \href{https://7last.github.io/docs/rtb/documentazione-interna/glossario\#norme-di-progetto}{\textbf{Norme di Progetto\textsubscript{G}}} v2.0:\\
	      \url{https://7last.github.io/docs/pb/documentazione-interna/norme-di-progetto}
\end{itemize}
\subsubsection{Informativi}
\begin{itemize}
	\item \textbf{Architettura Kappa e Lambda}: \\\url{https://nexocode.com/blog/posts/lambda-vs-kappa-architecture/} [Ultima consultazione: 2024-07-18];
	\item \textbf{Gestione degli eventi temporali con Apache Flink}: \url{https://nightlies.apache.org/flink/flink-docs-release-1.18/docs/concepts/time/} [Ultima consultazione: 2024-07-18];
	\item \textbf{Glossario} v2.0: \url{https://7last.github.io/docs/pb/documentazione-interna/glossario}
\end{itemize}
