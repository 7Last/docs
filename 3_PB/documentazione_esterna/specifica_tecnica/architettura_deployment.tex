\section{Architettura di deployment}
Per implementare ed eseguire l'intero stack tecnologico e i livelli del modello architetturale, viene creato un ambiente \textit{Docker} che riproduce la suddivisione e la distribuzione dei servizi. In particolare, per l'ambiente di produzione, sono stati creati i seguenti container:
\begin{itemize}
    \item \textbf{Data feed}
        \begin{itemize}
            \item Container: \textbf{Simulator};
            \item Descrizione: simula la generazione di dati;
        \end{itemize}
    \item \textbf{Streaming layer}
        \begin{itemize}
            \item Container: \textbf{Redpanda};
            \item Descrizione: definisce il flusso di dati in tempo reale;
            \item Componenti di supporto: schema registry;
            \item Porta: 18082.
        \end{itemize}
    \item \textbf{Processing Layer}
        \begin{itemize}
            \item Container: \textbf{Flink};
            \item Descrizione: pianifica, assegna e coordina l'esecuzione dei task di elaborazione dei dati su un cluster di nodi, garantendo prestazioni elevate, scalabilità e affidabilità nell'elaborazione dei dati.
        \end{itemize}
    \item \textbf{Storage Layer}
        \begin{itemize}
            \item Container: \textbf{Clickhouse};
            \item Descrizione: memorizza i dati;
            \item Porta: 8123.
        \end{itemize}
    \item \textbf{Data Visualization Layer}
        \begin{itemize}
            \item Container: \textbf{Grafana};
            \item Descrizione: visualizza i dati;
            \item Porta: 3000.
        \end{itemize}
\end{itemize}

