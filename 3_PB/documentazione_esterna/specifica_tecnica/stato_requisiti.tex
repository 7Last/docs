
\section{Requisiti}
\subsection{Requisiti funzionali}
\begin{longtable}{|>{\centering\arraybackslash}m{0.10\textwidth}|>{\centering\arraybackslash}m{0.20\textwidth}|>{\centering\arraybackslash}m{0.20\textwidth}|>{\centering\arraybackslash}m{0.4\textwidth}|}
	\hline
	\textbf{Codice} & \textbf{Importanza} & \textbf{Stato}& \textbf{Descrizione}\\\hline
	\endfirsthead
	\hline
	\textbf{Codice} & \textbf{Importanza} & \textbf{Stato}& \textbf{Descrizione}\\\hline
	\endhead
	\hline
	RF-1            & Obbligatorio        & soddisfatto & La parte \textit{IoT} dovrà essere simulata attraverso tool di generazione di dati casuali che tuttavia siano verosimili.
	\\\hline
	RF-2            & Obbligatorio        & soddisfatto & Il sistema dovrà permettere la visualizzazione dei dati in tempo reale.
	\\\hline
	RF-3            & Obbligatorio        & soddisfatto & Il sistema dovrà permettere la visualizzazione dei dati storici.
	\\\hline
	RF-4            & Obbligatorio        & soddisfatto & L'utente deve poter accedere all'applicativo senza bisogno di autenticazione.
	\\\hline
	RF-5            & Obbligatorio        & soddisfatto & L'utente dovrà poter visualizzare su una mappa la posizione geografica dei sensori.
	\\\hline
	RF-6            & Obbligatorio        & soddisfatto & I tipi di dati che il sistema dovrà visualizzare sono: temperatura, umidità, qualità dell'aria, precipitazioni, traffico, stato delle colonnine di ricarica, stato di occupazione dei parcheggi, stato di riempimento delle isole ecologiche e livello di acqua.
	\\\hline
	RF-7            & Obbligatorio        & soddisfatto & I dati dovranno essere salvati su un database OLAP.
	\\\hline
	RF-8            & Obbligatorio        & soddisfatto & I sensori di temperatura rilevano i dati in gradi Celsius
	\\\hline
	RF-9            & Obbligatorio        & soddisfatto & I sensori di umidità rilevano la percentuale di umidità nell’aria.
	\\\hline
	RF-10           & Obbligatorio        & soddisfatto & I sensori livello acqua rilevano il livello di acqua nella zona di installazione
	\\\hline
	RF-11           & Obbligatorio        & soddisfatto & I dati provenienti dai sensori dovranno contenere i seguenti dati: id \href{https://7last.github.io/docs/pb/documentazione-interna/glossario\#sensore}{sensore\textsubscript{G}}, data, ora e valore.
	\\\hline
	RF-12           & Obbligatorio        & soddisfatto & Sviluppo di componenti quali \href{https://7last.github.io/docs/pb/documentazione-interna/glossario\#widget}{widget\textsubscript{G}} e grafici per la visualizzazione dei dati nelle \href{https://7last.github.io/docs/pb/documentazione-interna/glossario\#dashboard}{dashboard\textsubscript{G}}.
	\\\hline
	RF-13           & Obbligatorio        & soddisfatto                                                                                                          & Il sistema deve permettere di visualizzare una \href{https://7last.github.io/docs/pb/documentazione-interna/glossario\#dashboard}{dashboard\textsubscript{G}} generale con tutti i dati dei sensori.
	\\\hline
	RF-14           & Obbligatorio        & soddisfatto                                                                                                           & Il sistema deve permettere di visualizzare una \href{https://7last.github.io/docs/pb/documentazione-interna/glossario\#dashboard}{dashboard\textsubscript{G}} contenente tutti i dati dei sensori che monitorano l'ambiente.
	\\\hline
	RF-15           & Obbligatorio        & soddisfatto                                                                                                           & Il sistema deve permettere di visualizzare una \href{https://7last.github.io/docs/pb/documentazione-interna/glossario\#dashboard}{dashboard\textsubscript{G}} contenente tutti i dati dei sensori che monitorano gli aspetti urbani.
	\\\hline
	RF-16           & Obbligatorio        & soddisfatto                                                                                                           & Il sistema deve permettere di visualizzare una sezione specifica per ciascuna categoria di sensori.
	\\\hline
	RF-17           & Obbligatorio        & soddisfatto                                                                                                           & Nella \href{https://7last.github.io/docs/pb/documentazione-interna/glossario\#dashboard}{dashboard\textsubscript{G}} dei dati grezzi dovranno essere presenti: una mappa interattiva, un \href{https://7last.github.io/docs/pb/documentazione-interna/glossario\#widget}{widget\textsubscript{G}} con il conteggio totale dei sensori divisi per tipo, una tabella contente tutti i sensori e la data in cui essi hanno trasmesso l'ultima volta. Inoltre verranno mostrate delle tabelle con i dati filtrabili suddivisi per \href{https://7last.github.io/docs/pb/documentazione-interna/glossario\#sensore}{sensore\textsubscript{G}} e un grafico \href{https://7last.github.io/docs/pb/documentazione-interna/glossario\#time-series}{time series\textsubscript{G}} con tutti i dati grezzi.
	\\\hline
	RF-18           & Obbligatorio        & soddisfatto                                                                                                           & Nella \href{https://7last.github.io/docs/pb/documentazione-interna/glossario\#dashboard}{dashboard\textsubscript{G}} dei dati ambientali dovranno essere presenti delle sezioni contenenti i panel relativi ai sensori di temperatura, umidità, precipitazioni, livello dei fiumi e qualità dell'aria.
	\\\hline
	RF-19           & Obbligatorio        & soddisfatto                                                                                                           & Nella \href{https://7last.github.io/docs/pb/documentazione-interna/glossario\#dashboard}{dashboard\textsubscript{G}} dei dati legati agli aspetti urbani dovranno essere presenti delle sezioni contenenti i panel relativi ai sensori di parcheggio, traffico, isole ecologiche e colonnine di ricarica.
	\\\hline
	RF-20           & Obbligatorio        & soddisfatto                                                                                                           & Nella sezione della temperatura dovranno essere visualizzati: un grafico \href{https://7last.github.io/docs/pb/documentazione-interna/glossario\#time-series}{time series\textsubscript{G}}, una mappa interattiva, la temperatura media, minima e massima di un certo periodo di tempo, la temperatura in tempo reale e la temperatura media per settimana e mese.
	\\\hline
	RF-21           & Obbligatorio        & soddisfatto                                                                                                           & Nella sezione dell'umidità dovranno essere visualizzati: un grafico \href{https://7last.github.io/docs/pb/documentazione-interna/glossario\#time-series}{time series\textsubscript{G}}, una mappa interattiva, l'umidità media, minima e massima di un certo periodo di tempo e l'umidità in tempo reale.
	\\\hline
	RF-22           & Obbligatorio        & soddisfatto                                                                                                           & Nella sezione della qualità dell'aria dovranno essere visualizzati: un grafico \href{https://7last.github.io/docs/pb/documentazione-interna/glossario\#time-series}{time series\textsubscript{G}}, una mappa interattiva, la qualità media dell'aria in un certo periodo e in tempo reale, i giorni con la qualità dell'aria migliore e peggiore in un certo periodo di tempo.
	\\\hline
	RF-23           & Obbligatorio        & soddisfatto                                                                                                           & Nella sezione delle precipitazioni dovranno essere visualizzati: un grafico \href{https://7last.github.io/docs/pb/documentazione-interna/glossario\#time-series}{time series\textsubscript{G}}, una mappa interattiva, la quantità media di precipitazioni in un certo periodo e in tempo reale, i giorni con la quantità di precipitazioni maggiore e minore in un certo periodo di tempo.
	\\\hline
	RF-24           & Obbligatorio        & soddisfatto                                                                                                           & Nella sezione del livello di acqua dovranno essere visualizzati: un grafico \href{https://7last.github.io/docs/pb/documentazione-interna/glossario\#time-series}{time series\textsubscript{G}}, una mappa interattiva, il livello medio di acqua in un certo periodo e in tempo reale.
	\\\hline
	RF-25           & Obbligatorio        & soddisfatto                                                                                                           & Nella sezione delle isole ecologiche dovranno essere visualizzati: una mappa interattiva con il rispettivo stato di riempimento e il conteggio di isole ecologiche suddivise per stato di riempimento in tempo reale.
	\\\hline
	RF-26           & Obbligatorio        & soddisfatto                                                                                                           & Nella sezione dei parcheggi dovranno essere visualizzati: una mappa interattiva con il rispettivo stato di occupazione e il conteggio di parcheggi suddivisi per stato di occupazione in tempo reale.
	\\\hline
	RF-27           & Obbligatorio        & soddisfatto                                                                                                           & Nella sezione delle colonnine di ricarica dovranno essere visualizzati: una mappa interattiva contenente anche lo stato e il numero di colonnine di ricarica suddivise per stato in tempo reale.
	\\\hline
	RF-28           & Obbligatorio        & soddisfatto                                                                                                           & Nella sezione del traffico dovranno essere visualizzati: un grafico \href{https://7last.github.io/docs/pb/documentazione-interna/glossario\#time-series}{time series\textsubscript{G}}, il numero di veicoli e la velocità media in tempo reale, il calcolo dell'ora di punta sulla base del numero di veicoli e velocità media.
	\\\hline
	RF-29           & Obbligatorio        & soddisfatto                                                                                                           & Nel caso in cui non ci siano dati visualizzabili, il sistema deve notificare l'utente mostrando un opportuno messaggio.
	\\\hline
	RF-30           & Obbligatorio        & soddisfatto                                                                                                           & I sensori di qualità dell'aria inviano i seguenti dati: \textit{PM10}, \textit{PM2.5}, \textit{NO2}, \textit{CO}, \textit{O3}, \textit{SO2} in $\mu g/m^3$.
	\\\hline
	RF-31           & Obbligatorio        & soddisfatto                                                                                                           & I sensori di precipitazioni inviano la quantità di pioggia caduta in mm.
	\\\hline
	RF-32           & Obbligatorio        & soddisfatto                                                                                                           & I sensori di traffico inviano il numero di veicoli rilevati e la velocità in km/h.
	\\\hline
	RF-33           & Obbligatorio        & soddisfatto                                                                                                           & Le colonnine di ricarica inviano lo stato di occupazione e il tempo mancante alla fine della ricarica (se occupate) o il tempo passato dalla fine dell'ultima ricarica (se libere).
	\\\hline
	RF-34           & Obbligatorio        & soddisfatto                                                                                                           & I sensori di parcheggio inviano lo stato di occupazione del parcheggio (1 se occupato, 0 se libero) e il timestamp dell'ultimo cambiamento di stato.
	\\\hline
	RF-35           & Obbligatorio        & soddisfatto                                                                                                           & Le isole ecologiche inviano lo stato di riempimento come percentuale.
	\\\hline
	RF-36           & Obbligatorio        & soddisfatto                                                                                                           & I sensori di livello di acqua inviano il livello di acqua in cm.
	\\\hline
	RF-37           & Obbligatorio        & soddisfatto                                                                                                           & Il sistema deve permettere di filtrare i dati visualizzati in base a un intervallo di tempo.
	\\\hline
	RF-38           & Obbligatorio        & soddisfatto                                                                                                           & Il sistema deve permettere di filtrare i dati visualizzati in base al \href{https://7last.github.io/docs/pb/documentazione-interna/glossario\#sensore}{sensore\textsubscript{G}} che li ha generati.
	\\\hline
	RF-39           & Desiderabile        & soddisfatto                                                                                                           & Devono essere messe in relazione più sorgenti di dati.
	\\\hline
	RF-40           & Desiderabile        & soddisfatto                                                                                                           & Nei grafici \href{https://7last.github.io/docs/pb/documentazione-interna/glossario\#time-series}{time series\textsubscript{G}} i dati devono essere aggregati calcolando la media di 5 minuti, in modo da risultare più leggibili.
	\\\hline
	RF-41           & Obbligatorio        & soddisfatto & Deve essere implementato almeno un simulatore di dati.
	\\\hline
	RF-42           & Desiderabile        & soddisfatto & Devono essere implementati più simulatori di dati.
	\\\hline
	RF-43           & Obbligatorio        & soddisfatto & I simulatori devono produrre dei dati verosimili.
	\\\hline
	RF-44           & Obbligatorio        & soddisfatto & Per ciascuna tipologia di \href{https://7last.github.io/docs/pb/documentazione-interna/glossario\#sensore}{sensore\textsubscript{G}} dev'essere sviluppata almeno una sezione.
	\\\hline
	RF-45           & Opzionale           & non soddisfatto & Deve essere implementata una funzionalità di previsione di dati futuri della temperature, basandosi sui dati dell'anno e della settimana precedente.
	\\\hline
	RF-46           & Desiderabile        & soddisfatto & Deve esistere una \href{https://7last.github.io/docs/pb/documentazione-interna/glossario\#dashboard}{dashboard\textsubscript{G}} per la visualizzazione della posizione geografica dei sensori su una mappa.
	\\\hline
	RF-47           & Opzionale           & soddisfatto & Deve essere presente un sistema di notifiche che allerti l'utente nel caso in cui la temperatura superi i 40°C per più di 30 minuti.
	\\\hline
	RF-48           & Opzionale           & soddisfatto                                                                                                           & Deve essere presente un sistema di notifiche che allerti l'utente se un'isola ecologica rimane al 100\% di riempimento per più di 24 ore.
	\\\hline
	RF-49           & Opzionale           & soddisfatto                                                                                                           & Deve essere presente un sistema di notifiche che allerti l'utente se la qualità dell'aria supera l'indice 3 dell'EAQI.
	\\\hline
	RF-50           & Opzionale           & soddisfatto                                                                                                           & Deve essere presente un sistema di notifiche che allerti l'utente se la quantità di precipitazioni supera i 10mm in un'ora.
	\\\hline
	RF-51           & Opzionale           & soddisfatto                                                                                                           & Deve essere implementato il calcolo dell'indice di qualità dell'aria EAQI.
	\\\hline
	RF-52           & Opzionale           & soddisfatto                                                                                                           & Deve essere implementato il calcolo dell'indice di temperatura percepita Heat Index, combinando i dati provenienti dai sensori di temperatura e umidità.
	\\\hline
	RF-53           & Opzionale           & soddisfatto                                                                                                           & Devono essere combinati i dati provenienti dalle colonnine di ricarica e dai parcheggi per calcolare quanti parcheggi sono stati utilizzati da veicoli elettrici e se il parcheggio ha fruttato abbastanza per coprire i costi di installazione.
	\\\hline
	RF-54           & Obbligatorio        & soddisfatto                                                                                                           & Il sistema deve permettere di filtrare i dati visualizzati in base al tipo di sensore che li ha prodotti.
	\\\hline
	\caption{Requisiti funzionali}
\end{longtable}

%--------------GRAFICO PERCENTUALE REQUISITI FUNZIONALI OBBLIGATORI----------------%
\begin{figure}[!h]
	\centering
	\begin{tikzpicture}
		\def\printonlypositive#1{\ifdim#1pt>0pt#1\%\else\fi}
		\pie[pos={8,0},radius=3.5,text=legend,
			before number=\printonlypositive, after number=, color={orange, cyan}] {
            100.0/Completati,
			0.54/Non completati
		}
	\end{tikzpicture}
	\caption{Percentuale di soddisfacimento dei requisiti funzionali}
\end{figure}
%----------------------------------------------------------------------------------%

\subsection{Requisiti qualitativi}
\begin{longtable}{|>{\centering\arraybackslash}m{0.10\textwidth}|>{\centering\arraybackslash}m{0.20\textwidth}|>{\centering\arraybackslash}m{0.20\textwidth}|>{\centering\arraybackslash}m{0.4\textwidth}|}
	\hline
	\textbf{Codice} & \textbf{Importanza} & \textbf{Stato}& \textbf{Descrizione}\\\hline
	\endfirsthead
	\hline
	\textbf{Codice} & \textbf{Importanza} & \textbf{Stato}& \textbf{Descrizione}\\\hline
	\endhead
	\hline
	RQ-55           & Obbligatorio        & soddisfatto & Sviluppo di test che dimostrino il corretto funzionamento dei servizi e delle funzionalità previste. Viene richiesta una copertura dell'80\% corredata di report.
	\\\hline
	RQ-56           & Obbligatorio        & soddisfatto & Il progetto deve essere corredato di documentazione riguardo scelte implementative e progettuali effettuate e relative motivazioni.
	\\\hline
	RQ-57           & Obbligatorio        & soddisfatto & Il progetto deve essere corredato di documentazione riguardo problemi aperti e eventuali soluzioni proposte da esplorare.
	\\\hline
	RQ-58           & Obbligatorio        & soddisfatto & Tutte le componenti del sistema devono essere testate con \href{https://7last.github.io/docs/pb/documentazione-interna/glossario\#test-end-to-end}{\textit{test end-to-end}\textsubscript{G}}.
	\\\hline
	RQ-59           & Obbligatorio        & soddisfatto                                                                                                                                                                                                                                              & Il sistema sarà corredato di un Manuale Utente che spieghi le funzionalità del sistema e come utilizzarle.
	\\\hline
	RQ-60           & Obbligatorio        & soddisfatto                                                                                                                                                                                                                                              & Il sistema sarà corredato di un documento di Specifica Tecnica che spieghi le scelte progettuali effettuate.
	\\\hline
	\caption{Requisiti qualitativi}
\end{longtable}

%--------------GRAFICO PERCENTUALE REQUISITI QUALITATIVI----------------%
\begin{figure}[!h]
	\centering
	\begin{tikzpicture}
		\def\printonlypositive#1{\ifdim#1pt>0pt#1\%\else\fi}
		\pie[pos={8,0},radius=3.5,text=legend,
			before number=\printonlypositive, after number=, color={orange}] {
            100.0/Completati
		}
	\end{tikzpicture}
	\caption{Percentuale di soddisfacimento dei requisiti qualitativi}
\end{figure}
%----------------------------------------------------------------------------------%

\subsection{Requisiti di vincolo}
\begin{longtable}{|>{\centering\arraybackslash}m{0.10\textwidth}|>{\centering\arraybackslash}m{0.20\textwidth}|>{\centering\arraybackslash}m{0.20\textwidth}|>{\centering\arraybackslash}m{0.4\textwidth}|}
	\hline
	\textbf{Codice} & \textbf{Importanza} & \textbf{Stato}& \textbf{Descrizione}\\\hline
	\endfirsthead
	\hline
	\textbf{Codice} & \textbf{Importanza} & \textbf{Stato}& \textbf{Descrizione}\\\hline
	\endhead
	\hline
	RV-61           & Obbligatorio        & soddisfatto & Il simulatore di dati deve pubblicare messaggi in una piattaforma di \textit{data streaming}.
	\\\hline
	RV-62           & Obbligatorio        & soddisfatto                                                                                                           & La piattaforma di \textit{data streaming} utilizzata è \href{https://7last.github.io/docs/pb/documentazione-interna/glossario\#redpanda}{\textit{Redpanda}\textsubscript{G}}.
	\\\hline
	RV-63           & Obbligatorio        & soddisfatto & I dati pubblicati nella piattaforma di \textit{data streaming} devono essere salvati in un database OLAP.
	\\\hline
	RV-64           & Obbligatorio        & soddisfatto & I dati devono poter essere visualizzati dall'utente finale in delle \href{https://7last.github.io/docs/pb/documentazione-interna/glossario\#dashboard}{\textit{dashboard}\textsubscript{G}}, sviluppate con un \textit{tool} apposito, ad esempio \href{https://7last.github.io/docs/pb/documentazione-interna/glossario\#grafana}{\textit{Grafana}\textsubscript{G}}.
	\\\hline
	RV-65           & Opzionale           & soddisfatto                                                                                                           & I dati pubblicati nei \href{https://7last.github.io/docs/pb/documentazione-interna/glossario\#topic}{\textit{topic}\textsubscript{G}} di \href{https://7last.github.io/docs/pb/documentazione-interna/glossario\#redpanda}{\textit{Redpanda}\textsubscript{G}} sono serializzati in formato \href{https://docs.confluent.io/platform/current/schema-registry/fundamentals/serdes-develop/serdes-avro.html}{\underline{Confluent Avro}}.
	\\\hline
	RV-66           & Obbligatorio        & soddisfatto                                                                                                           & Il sistema deve essere sviluppato con \href{https://7last.github.io/docs/pb/documentazione-interna/glossario\#docker-compose}{\href{https://7last.github.io/docs/pb/documentazione-interna/glossario\#docker}{\textit{Docker}\textsubscript{G}}\textit{ Compose}\textsubscript{G}}, utilizzando la versione 3.8 della specifica.
	\\\hline
	RV-67           & Obbligatorio        & soddisfatto & Il sistema deve poter essere usufruito dalle versioni più recenti dei browser web più diffusi. Al momento della stesura del presente documento, le versioni supportate sono: \textit{Google Chrome} v124, \textit{Safari} v17.4, \textit{Microsoft Edge} v123, \textit{Firefox} v125.
	\\\hline
	RV-68           & Obbligatorio        & soddisfatto                                                                                                           & Il sistema deve poter funzionare su sistema operativo \textit{Linux}, con CPU a 64 bit, almeno 4GB di RAM e una delle seguenti distribuzioni e versioni minime: \textit{Ubuntu} 22.04, \textit{Debian} 12, \textit{Fedora} 38, \textit{Red Hat Enterprise Linux} 8.
	\\\hline
	RV-69           & Obbligatorio        & soddisfatto                                                                                                           & Il sistema deve poter funzionare su sistema operativo \textit{Windows} con versione 10 o 11, CPU a 64 bit, almeno 4GB di RAM e la funzionalità WSL2 abilitata.
	\\\hline
	RV-70           & Obbligatorio        & Interno                                                                                                           & Il sistema deve poter funzionare su sistema operativo \textit{MacOs} con versione 12 o superiore, CPU \textit{Intel} o \textit{Apple Silicon} a 64bit e almeno 4GB di RAM.
	\\\hline
	\caption{Requisiti di vincolo}
\end{longtable}

%--------------GRAFICO PERCENTUALE REQUISITI DI VINCOLO----------------%
\begin{figure}[!h]
	\centering
	\begin{tikzpicture}
		\def\printonlypositive#1{\ifdim#1pt>0pt#1\%\else\fi}
		\pie[pos={8,0},radius=3.5,text=legend,
			before number=\printonlypositive, after number=, color={orange}] {
            100.0/Completati
		}
	\end{tikzpicture}
	\caption{Percentuale di soddisfacimento dei requisiti di vincolo}
\end{figure}
%----------------------------------------------------------------------------------%


\subsection{Requisiti prestazionali}
\begin{longtable}{|>{\centering\arraybackslash}m{0.10\textwidth}|>{\centering\arraybackslash}m{0.20\textwidth}|>{\centering\arraybackslash}m{0.20\textwidth}|>{\centering\arraybackslash}m{0.4\textwidth}|}
	\hline
	\textbf{Codice} & \textbf{Importanza} & \textbf{Stato}& \textbf{Descrizione}\\\hline
	\endfirsthead
	\hline
	\textbf{Codice} & \textbf{Importanza} & \textbf{Stato}& \textbf{Descrizione}\\\hline
	\endhead
	\hline
	RP-71           & Obbligatorio        & soddisfatto       & Il sistema deve garantire che la visualizzazione dei dati in tempo reale avvenga entro 5 secondi dalla ricezione dei dati.
	\\\hline
	\caption{Requisiti prestazionali}
\end{longtable}

%--------------GRAFICO PERCENTUALE REQUISITI PRESTAZIONALI----------------%
\begin{figure}[!h]
	\centering
	\begin{tikzpicture}
		\def\printonlypositive#1{\ifdim#1pt>0pt#1\%\else\fi}
		\pie[pos={8,0},radius=3.5,text=legend,
			before number=\printonlypositive, after number=, color={orange}] {
            100.0/Completati
		}
	\end{tikzpicture}
	\caption{Percentuale di soddisfacimento dei requisiti prestazionali}
\end{figure}
%----------------------------------------------------------------------------------%


%------------------------GRAFICO PERCENTUALE REQUISITI TOTALI----------------------%
\begin{figure}[!h]
	\centering
	\begin{tikzpicture}
		\def\printonlypositive#1{\ifdim#1pt>0pt#1\%\else\fi}
		\pie[pos={8,0},radius=3.5,text=legend,
			before number=\printonlypositive, after number=, color={orange, cyan}] {
            99.29/Completati,
            0.71/Non Completati
		}
	\end{tikzpicture}
	\caption{Percentuale di soddisfacimento dei requisiti totale}
\end{figure}
%----------------------------------------------------------------------------------%