\documentclass[italian,12pt]{article} %tipo di documento

%--------------variabili------------------%
\def\Title{Norme di Progetto}
\def\Author{7Last}
\def\Version{v0.2}
%-----------------------------------------%


\usepackage[left=2cm, right=2cm, bottom=3cm, top=3cm]{geometry}
\usepackage{fancyhdr}
\usepackage{graphicx}
\graphicspath{ {../../logo/} }
\usepackage{href-ul}
\usepackage{tikz}
\usepackage{tgadventor}
\usepackage[useregional=numeric,showseconds=true,showzone=false]{datetime2}
\usepackage{caption}
\usepackage{longtable}
\usepackage{xcolor}



% Definizione delle nuove classi di titolo
\titleclass{\subsubsubsection}{straight}[\subsection]
\titleclass{\subsubsubsubsection}{straight}[\subsubsubsection]
\titleclass{\subsubsubsubsubsection}{straight}[\subsubsubsubsection] % nuovo livello

% Creazione dei nuovi contatori
\newcounter{subsubsubsection}[subsubsection]
\newcounter{subsubsubsubsection}[subsubsubsection]
\newcounter{subsubsubsubsubsection}[subsubsubsubsection] % nuovo livello

% Rinnovo dei comandi per la formattazione dei numeri delle sezioni
\renewcommand\thesubsubsubsection{\thesubsubsection.\arabic{subsubsubsection}}
\renewcommand\thesubsubsubsubsection{\thesubsubsubsection.\arabic{subsubsubsubsection}}
\renewcommand\thesubsubsubsubsubsection{\thesubsubsubsubsection.\arabic{subsubsubsubsubsection}} % nuovo livello
\renewcommand\theparagraph{\thesubsubsubsubsubsection.\arabic{paragraph}} % opzionale; utile se i paragrafi devono essere numerati

% Formattazione dei titoli delle sezioni
\titleformat{\subsubsubsection}
  {\normalfont\normalsize\bfseries}{\thesubsubsubsection}{1em}{}
\titleformat{\subsubsubsubsection}
  {\normalfont\normalsize\bfseries}{\thesubsubsubsubsection}{1em}{}
\titleformat{\subsubsubsubsubsection} % nuovo livello
  {\normalfont\normalsize\bfseries}{\thesubsubsubsubsubsection}{1em}{} 

% Spaziatura dei titoli delle sezioni
\titlespacing*{\subsubsubsection}
{0pt}{3.25ex plus 1ex minus .2ex}{1.5ex plus .2ex}
\titlespacing*{\subsubsubsubsection}
{0pt}{3.25ex plus 1ex minus .2ex}{1.5ex plus .2ex}
\titlespacing*{\subsubsubsubsubsection} % nuovo livello
{0pt}{3.25ex plus 1ex minus .2ex}{1.5ex plus .2ex}

\makeatletter
% Rinnovo dei comandi per la formattazione dei paragrafi e sottoparagrafi
\renewcommand\paragraph{\@startsection{paragraph}{6}{\z@}%
  {3.25ex \@plus1ex \@minus.2ex}%
  {-1em}%
  {\normalfont\normalsize\bfseries}}
\renewcommand\subparagraph{\@startsection{subparagraph}{7}{\parindent}%
  {3.25ex \@plus1ex \@minus .2ex}%
  {-1em}%
  {\normalfont\normalsize\bfseries}}

% Definizione dei livelli per il Table of Contents
\def\toclevel@subsubsubsection{4}
\def\toclevel@subsubsubsubsection{5}
\def\toclevel@subsubsubsubsubsection{6} % nuovo livello
\def\toclevel@paragraph{7}
\def\toclevel@subparagraph{8}

% Definizione della formattazione per il Table of Contents
\def\l@subsubsubsection{\@dottedtocline{4}{7em}{4em}}
\def\l@subsubsubsubsection{\@dottedtocline{5}{10em}{5em}}
\def\l@subsubsubsubsubsection{\@dottedtocline{6}{14em}{6em}} % nuovo livello
\def\l@paragraph{\@dottedtocline{7}{18em}{7em}}
\def\l@subparagraph{\@dottedtocline{8}{22em}{8em}}
\makeatother

% Impostazione della profondità dei numeri di sezione e del Table of Contents
\setcounter{secnumdepth}{6} % nuovo livello
\setcounter{tocdepth}{6} % nuovo livello


\linespread{1.2}
\captionsetup[table]{name=Tabella}
\geometry{headsep=1.5cm}

\renewcommand{\contentsname}{Indice}%imposto il nome dell'indice
\renewcommand\familydefault{\sfdefault}

\renewcommand{\listtablename}{Indice delle tabelle}%imposto il nome della lista tabelle
\renewcommand\familydefault{\sfdefault}

\renewcommand{\listfigurename}{Indice delle immagini}%imposto il nome della lista immagini
\renewcommand\familydefault{\sfdefault}

\let\oldthepage\thepage
\renewcommand{\thepage}{\sffamily\oldthepage}

%-------------------INIZIO DOCUMENTO--------------
\begin{document}

\newgeometry{left=2cm,right=2cm,bottom=2.1cm,top=2.1cm}
\begin{titlepage}
	\vspace*{.5cm}

	\vspace{2cm}
	{
		\centering
		{\bfseries\huge \Title\par}
		\bigbreak
		{\bfseries\Large \Subtitle\par}
		\bigbreak
		{\bfseries\large \Author\par}
		\bigbreak
		{\Date\;-\;\Version\par}
		\vfill

		\begin{center}
			\begin{tikzpicture}
				\clip (0,0) circle (2cm) node {\includegraphics[width=4cm]{logo.jpg}};
			\end{tikzpicture}
		\end{center}
	}

	\vfill

\end{titlepage}

\restoregeometry






















\newpage

\pagestyle{fancy}
\fancyhead{}
\lhead{
	\begin{tikzpicture}
		\clip (0,0) circle (0.5cm);
		\node at (0,0) {\includegraphics[width=1cm]{./../logo/logo.png}};
	\end{tikzpicture}%
}
\chead{\vspace{\fill}\Title\vspace{\fill}}
\rhead{\vspace{\fill}\Version\vspace{\fill}}


%-----------tabella revisioni-----------%
\begin{table}[!h]
	\caption*{Versioni}
	\begin{center}
		\begin{tabular}{ l l l l l }
			\hline                                                                             \\[-2ex]
			Ver. & Data       & Autore        & Verificatore & Descrizione                     \\
			\hline                                                                             \\[-1.5ex]
			0.1  & 02/06/2024 & Matteo Tiozzo &              & Stesura struttura del documento \\
			\\[-1.5ex] \hline
		\end{tabular}
	\end{center}
\end{table}
%---------------------------------------%

\newpage

\tableofcontents

\newpage

\listoftables

\listoffigures

\newpage


\section{Introduzione}
\setcounter{subsection}{0}
\subsection{Scopo del documento}
Il seguente documento si propone di definire la pianificazione e la gestione delle attività richieste per ultimare il progetto. Vengono esaminati in dettaglio elementi cruciali come l’\textit{Analisi dei Rischi}, il \textit{modello di sviluppo adottato}, la \textit{pianificazione delle attività}, la \textit{suddivisione dei ruoli}, oltre a \textit{stime dei costi} e delle \textit{risorse necessarie}.

\subsection{Scopo del prodotto}
Lo scopo principale del prodotto è quello di consentire a \textit{Sync Lab S.r.l.} di valutare la \\\textbf{fattibilità} di investire tempo e risorse nell'implementazione del progetto  \href{https://7last.github.io/docs/rtb/documentazione-interna/glossario\#synccity}{\textit{\textbf{SyncCity} \textsubscript{G}}- A \href{https://7last.github.io/docs/rtb/documentazione-interna/glossario\#smart-city}{smart city\textsubscript{G}} monitoring platform}. Questa soluzione, attraverso l'utilizzo di dispositivi IoT, consente un monitoraggio costante delle città. \href{https://7last.github.io/docs/rtb/documentazione-interna/glossario\#synccity}{SyncCity\textsubscript{G}} avrà lo scopo di monitorare e raccogliere dati da sensori posizionati nelle città, per poi analizzarli e fornire informazioni utili alla gestione della città. Il prodotto finale sarà un prototipo funzionale che consentirà la visualizzazione dei dati raccolti su un cruscotto.

\subsection{Glossario}
Per evitare qualsiasi ambiguità o malinteso sui termini utilizzati nel documento, verrà adottato un \href{https://7last.github.io/docs/rtb/documentazione-interna/glossario\#glossario}{glossario\textsubscript{G}}. Questo \href{https://7last.github.io/docs/rtb/documentazione-interna/glossario\#glossario}{glossario\textsubscript{G}} conterrà varie definizioni. Ogni termine incluso nel \href{https://7last.github.io/docs/rtb/documentazione-interna/glossario\#glossario}{glossario\textsubscript{G}} sarà indicato applicando uno stile specifico:
\begin{itemize}
    \item aggiungendo una "G" al pedice della parola;
    \item fornendo il link al \href{https://7last.github.io/docs/rtb/documentazione-interna/glossario\#glossario}{glossario\textsubscript{G}} online;
\end{itemize}

\subsection{Riferimenti}
    \subsubsection{Normativi}DA SISTEMARE
        \begin{itemize}
            \item \textbf{ISO/IEC 12207:2008} - Systems and software engineering - Software life cycle processes
            \item \textbf{ISO/IEC 31000:2009} - Risk management - Principles and guidelines
        \end{itemize}
    \subsubsection{Informativi}
        \begin{itemize}
            \item \href{https://7last.github.io/docs/rtb/documentazione-interna/glossario\#capitolato}{\textbf{Capitolato \textsubscript{G}}C6 - Sync City}: \textit{A \href{https://7last.github.io/docs/rtb/documentazione-interna/glossario\#smart-city}{smart city\textsubscript{G}} monitoring platform}
            \item \textbf{T2 - Processi di ciclo di vita del software}\\ https://www.math.unipd.it/~tullio/IS-1/2023/Dispense/T2.pdf;
            \item \textbf{T4 - Gestione di progetto}\\ Visibili a questo \uline{\href{https://www.math.unipd.it/~tullio/IS-1/2023/Dispense/T4.pdf}{link}};
            \item \href{https://7last.github.io/docs/rtb/documentazione-interna/glossario\#glossario}{\textbf{Glossario}\textsubscript{G}}\\ Visibile a questo \uline{\href{https://7last.github.io/docs/rtb/documentazione-interna/glossario}{link}};
        \end{itemize}
\subsection{Preventivo iniziale}
Il preventivo iniziale presentato durante la fase di candidatura è disponibile al seguente \uline{\href{https://github.com/7Last/docs/blob/main/1_candidatura/preventivo_costi_assunzione_impegni_v2.0.pdf}{riferimento}}. All'interno di questo documento viene calcolato il preventivo iniziale del progetto, pari a €12.670,00. Inoltre, si specifica che il gruppo \textit{7Last} stima di \textbf{completare} il prodotto entro e non oltre il \textbf{24 Settembre 2024}.

\newpage

\section{Tecnologie}
Questa sezione si occupa di fornire una panoramica delle tecnologie utilizzate per implementare il sistema software.
In particolare, delinea le piattaforme, gli strumenti, i linguaggi di programmazione, i framework e altre risorse tecnologiche che sono state impiegate durante lo sviluppo.

\subsection{Docker}
È una piattaforma di virtualizzazione leggera che semplifica lo sviluppo, il testing e il rilascio delle applicazioni fornendo un ambiente isolato e riproducibile.
È utilizzato per creare ambienti di sviluppo standardizzati, facilitare la scalabilità delle applicazioni e semplificare la gestione delle risorse.

\subsubsection{Ambienti}
Per lo sviluppo di questo progetto sono stati ipotizzati i due seguenti scenari di esecuzione, separati grazie all'utilizzo di profili diversi di \href{https://7last.github.io/docs/pb/documentazione-interna/glossario\#docker-compose}{Docker Compose\textsubscript{G}}:
\begin{itemize}
	\item \texttt{local}: utilizzato dagli sviluppatori per testare e sviluppare le funzionalità dell'applicazione sui propri computer.
	      Questo ambiente permette di eseguire tutti i componenti del sistema all'interno di un container \href{https://7last.github.io/docs/pb/documentazione-interna/glossario\#docker}{Docker\textsubscript{G}}, ad eccezione del simulatore \href{https://7last.github.io/docs/pb/documentazione-interna/glossario\#python}{Python\textsubscript{G}}.
	      Esso viene eseguito direttamente sul sistema operativo dell'utente, in modo da facilitare il debugging e il testing delle funzionalità,
	      senza dover necessariamente eseguire la \textit{build} dell'immagine \href{https://7last.github.io/docs/pb/documentazione-interna/glossario\#docker}{Docker\textsubscript{G}} ad ogni modifica del codice;
	\item \texttt{release}: utilizzato quando si desidera simulare un ipotetico ambiente di produzione o non è necessario modificare il codice \href{https://7last.github.io/docs/pb/documentazione-interna/glossario\#python}{Python\textsubscript{G}}. Consente di non dover manualmente
	      installare le dipendenze o configurare l'ambiente di esecuzione. In questo caso, tutti i componenti del sistema vengono eseguiti all'interno di container \href{https://7last.github.io/docs/pb/documentazione-interna/glossario\#docker}{Docker\textsubscript{G}}.

\end{itemize}

\subsubsection{Immagini Docker}
Nello sviluppo di questo progetto abbiamo utilizzato diverse immagini \href{https://7last.github.io/docs/pb/documentazione-interna/glossario\#docker}{Docker\textsubscript{G}}, di seguito elencate.
Le porte sono indicate nel formato \texttt{<porta\_host>:<porta\_container>}, dove \texttt{<porta\_host>} è la porta del sistema host a cui si mappa la porta del container \texttt{<porta\_container>}.
\begin{itemize}
	\item \textbf{Simulator - \href{https://7last.github.io/docs/pb/documentazione-interna/glossario\#python}{Python\textsubscript{G}}}
	      \begin{itemize}
		      \item \textbf{Immagine}: \href{https://7last.github.io/docs/pb/documentazione-interna/glossario\#python}{python\textsubscript{G}}:3.11.9-alpine;
		      \item \textbf{Riferimento}: \underline{\href{https://hub.docker.com/_/python}{\href{https://7last.github.io/docs/pb/documentazione-interna/glossario\#python}{Python\textsubscript{G}} \href{https://7last.github.io/docs/pb/documentazione-interna/glossario\#docker}{Docker\textsubscript{G}} Image}} [Ultima consultazione: 2024-06-02].
		      \item \textbf{Ambiente}: \texttt{release};
	      \end{itemize}

	\item \href{https://7last.github.io/docs/pb/documentazione-interna/glossario\#redpanda}{\textbf{Redpanda}\textsubscript{G}}\textbf{ \mySkip{Init}}:
	      l'immagine di \texttt{alpine} viene utilizzata per creare un container che si occupa di inizializzare il \href{https://7last.github.io/docs/pb/documentazione-interna/glossario\#broker}{broker\textsubscript{G}} \href{https://7last.github.io/docs/pb/documentazione-interna/glossario\#redpanda}{Redpanda\textsubscript{G}}.
	      \begin{itemize}
		      \item \textbf{Immagine}: alpine:3.20.1;
		      \item \textbf{Riferimento}: \underline{\href{https://hub.docker.com/_/alpine}{Alpine}} [Ultima consultazione: 2024-06-25].
		      \item \textbf{Ambiente}: \texttt{local}, \texttt{release}.
	      \end{itemize}

	\item \href{https://7last.github.io/docs/pb/documentazione-interna/glossario\#redpanda}{\textbf{Redpanda}\textsubscript{G}}
	      \begin{itemize}
		      \item \textbf{Immagine}: \href{https://7last.github.io/docs/pb/documentazione-interna/glossario\#docker}{docker\textsubscript{G}}.\href{https://7last.github.io/docs/pb/documentazione-interna/glossario\#redpanda}{redpanda\textsubscript{G}}.com/redpandadata/\href{https://7last.github.io/docs/pb/documentazione-interna/glossario\#redpanda}{redpanda\textsubscript{G}}:v23.3.11;
		      \item \textbf{Riferimento}: \underline{\href{https://hub.docker.com/r/redpandadata/redpanda}{\href{https://7last.github.io/docs/pb/documentazione-interna/glossario\#redpanda}{Redpanda\textsubscript{G}} \href{https://7last.github.io/docs/pb/documentazione-interna/glossario\#docker}{Docker\textsubscript{G}} Image}} [Ultima consultazione: 2024-06-02].
		      \item \textbf{Ambiente}: \texttt{local}, \texttt{release};
		      \item \textbf{Porte}:
		            \begin{itemize}
			            \item 18081:18081: \href{https://7last.github.io/docs/pb/documentazione-interna/glossario\#schema-registry}{\textit{schema registry}\textsubscript{G}};
			            \item 18082:18082: \textit{proxy};
			            \item 19092:19092: \href{https://7last.github.io/docs/pb/documentazione-interna/glossario\#broker}{\textit{broker}\textsubscript{G}}.
		            \end{itemize}
	      \end{itemize}

	\item \href{https://7last.github.io/docs/pb/documentazione-interna/glossario\#redpanda}{\textbf{Redpanda}\textsubscript{G}}\textbf{ console}
	      \begin{itemize}
		      \item \textbf{Immagine}: \href{https://7last.github.io/docs/pb/documentazione-interna/glossario\#docker}{docker\textsubscript{G}}.\href{https://7last.github.io/docs/pb/documentazione-interna/glossario\#redpanda}{redpanda\textsubscript{G}}.com/redpandadata/console:v2.4.6;
		      \item \textbf{Riferimento}: \underline{\href{https://hub.docker.com/r/redpandadata/redpanda}{\href{https://7last.github.io/docs/pb/documentazione-interna/glossario\#redpanda}{Redpanda\textsubscript{G}} Console \href{https://7last.github.io/docs/pb/documentazione-interna/glossario\#docker}{Docker\textsubscript{G}} Image}} [Ultima consultazione: 2024-06-02].
		      \item \textbf{Ambiente}: \texttt{local}, \texttt{release};
		      \item \textbf{Porte}:
		            \begin{itemize}
			            \item 8080:8080.
		            \end{itemize}
	      \end{itemize}

	\item \textbf{Connectors}
	      \begin{itemize}
		      \item \textbf{Immagine}: \href{https://7last.github.io/docs/pb/documentazione-interna/glossario\#docker}{docker\textsubscript{G}}.\href{https://7last.github.io/docs/pb/documentazione-interna/glossario\#redpanda}{redpanda\textsubscript{G}}.com/redpandadata/connectors:v1.0.27;
		      \item \textbf{Riferimento}: \underline{\href{https://hub.docker.com/r/redpandadata/connectors}{\href{https://7last.github.io/docs/pb/documentazione-interna/glossario\#redpanda}{Redpanda\textsubscript{G}} Connectors \href{https://7last.github.io/docs/pb/documentazione-interna/glossario\#docker}{Docker\textsubscript{G}} Image}} [Ultima consultazione: 2024-06-02].
		      \item \textbf{Ambiente}: \texttt{local}, \texttt{release};
		      \item \textbf{Porte}:
		            \begin{itemize}
			            \item 8083:8083.
		            \end{itemize}
	      \end{itemize}

	\item \href{https://7last.github.io/docs/pb/documentazione-interna/glossario\#clickhouse}{\textbf{ClickHouse}\textsubscript{G}}
	      \begin{itemize}
		      \item \textbf{Immagine}: \href{https://7last.github.io/docs/pb/documentazione-interna/glossario\#clickhouse}{clickhouse\textsubscript{G}}/\href{https://7last.github.io/docs/pb/documentazione-interna/glossario\#clickhouse}{clickhouse\textsubscript{G}}-server:24-alpine;
		      \item \textbf{Riferimento}: \underline{\href{https://hub.docker.com/r/clickhouse/clickhouse-server}{\href{https://7last.github.io/docs/pb/documentazione-interna/glossario\#clickhouse}{ClickHouse\textsubscript{G}} \href{https://7last.github.io/docs/pb/documentazione-interna/glossario\#docker}{Docker\textsubscript{G}} Image}} [Ultima consultazione: 2024-06-02].
		      \item \textbf{Ambiente}: \texttt{local}, \texttt{release}.
		      \item \textbf{Porte}:
		            \begin{itemize}
			            \item 8123:8123.
		            \end{itemize}
	      \end{itemize}

	\item \textbf{Apache Flink}
	      \begin{itemize}
		      \item \textbf{Immagine}: flink:1.18.1-java17;
		      \item \textbf{Riferimento}: \underline{\href{https://hub.docker.com/_/flink}{Flink \href{https://7last.github.io/docs/pb/documentazione-interna/glossario\#docker}{Docker\textsubscript{G}} Image}} [Ultima consultazione: 2024-06-02].
		      \item \textbf{Ambiente}: \texttt{local}, \texttt{release}.
		      \item \textbf{Porte}:
		            \begin{itemize}
			            \item servizio \texttt{jobmanager}: 9001:8081 interfaccia \textit{web};
			            \item servizio \href{https://7last.github.io/docs/pb/documentazione-interna/glossario\#taskmanager}{\texttt{taskmanager}\textsubscript{G}}: 6121:6121.
		            \end{itemize}
	      \end{itemize}

	\item \href{https://7last.github.io/docs/pb/documentazione-interna/glossario\#grafana}{\textbf{Grafana}\textsubscript{G}}
	      \begin{itemize}
		      \item \textbf{Immagine}: \href{https://7last.github.io/docs/pb/documentazione-interna/glossario\#grafana}{grafana\textsubscript{G}}/\href{https://7last.github.io/docs/pb/documentazione-interna/glossario\#grafana}{grafana\textsubscript{G}}-oss:10.3.0;
		      \item \textbf{Riferimento}: \underline{\href{https://hub.docker.com/r/grafana/grafana-oss}{\href{https://7last.github.io/docs/pb/documentazione-interna/glossario\#grafana}{Grafana\textsubscript{G}} \href{https://7last.github.io/docs/pb/documentazione-interna/glossario\#docker}{Docker\textsubscript{G}} Image}} [Ultima consultazione: 2024-06-02].
		      \item \textbf{Ambiente}: \texttt{local}, \texttt{release}.
		      \item \textbf{Porte}:
		            \begin{itemize}
			            \item 3000:3000.
		            \end{itemize}
	      \end{itemize}
\end{itemize}

\subsection{Linguaggi e formato dati}
\begin{longtable}{|>{\centering\arraybackslash}m{0.10\textwidth}|>{\centering\arraybackslash}m{0.10\textwidth}|>{\centering\arraybackslash}m{0.35\textwidth}|>{\centering\arraybackslash}m{0.35\textwidth}|}
	\hline
	\textbf{Nome} & \textbf{Versione} & \textbf{Descrizione}                                                                                                                                 & \textbf{Impiego}                                                                                                             \\\hline
	\endfirsthead
	\hline
	\textbf{Nome} & \textbf{Versione} & \textbf{Descrizione}                                                                                                                                 & \textbf{Impiego}                                                                                                             \\\hline
	\endhead
	\href{https://7last.github.io/docs/pb/documentazione-interna/glossario\#python}{Python\textsubscript{G}}        & 3.11.9            & Linguaggio di programmazione ad alto livello, interpretato e multiparadigma.                                                                         & Simulatore di sensori, \textit{testing}, \textit{script} per automatizzare il \textit{deployment} dei \href{https://7last.github.io/docs/pb/documentazione-interna/glossario\#job}{\textit{job}\textsubscript{G}} di Flink. \\\hline
	\href{https://7last.github.io/docs/pb/documentazione-interna/glossario\#javascript-object-notation}{JSON\textsubscript{G}}          & -                 & Formato di dati semplice da interpretare e generare, ampiamente utilizzato per lo scambio di dati tra applicazioni.                                  & Configurazione \href{https://7last.github.io/docs/pb/documentazione-interna/glossario\#dashboard}{\textit{dashboard}\textsubscript{G}} \href{https://7last.github.io/docs/pb/documentazione-interna/glossario\#grafana}{Grafana\textsubscript{G}}, definizione di schemi Avro.                                                       \\\hline
	\href{https://7last.github.io/docs/pb/documentazione-interna/glossario\#yet-another-markup-language}{YAML\textsubscript{G}}          & -                 & Linguaggio di serializzazione dei dati leggibile sia per gli esseri umani sia per le macchine.                                                       & \href{https://7last.github.io/docs/pb/documentazione-interna/glossario\#docker-compose}{Docker Compose\textsubscript{G}}, provisioning \href{https://7last.github.io/docs/pb/documentazione-interna/glossario\#grafana}{Grafana\textsubscript{G}} e configurazione \textit{alert}, file di \textit{workflow} per le \href{https://7last.github.io/docs/pb/documentazione-interna/glossario\#github}{GitHub\textsubscript{G}} Actions.       \\\hline
	SQL           & Ansi SQL          & Linguaggio di programmazione specificamente progettato per la gestione e la manipolazione di dati all'interno di sistemi di gestione di database.    & \textit{Query} e gestione database \href{https://7last.github.io/docs/pb/documentazione-interna/glossario\#clickhouse}{ClickHouse\textsubscript{G}}.                                                                               \\\hline
	\href{https://7last.github.io/docs/pb/documentazione-interna/glossario\#tom's-obvious-minimal-language}{TOML\textsubscript{G}}          & 1.0.0             & Linguaggio di \textit{markup} progettato per essere più leggibile e facile da scrivere rispetto ad altri formati di configurazione come \href{https://7last.github.io/docs/pb/documentazione-interna/glossario\#javascript-object-notation}{JSON\textsubscript{G}} e \href{https://7last.github.io/docs/pb/documentazione-interna/glossario\#yet-another-markup-language}{YAML\textsubscript{G}}. & Configurazione e gestione dei sensori simulati.                                                                              \\\hline
	Java          & 17                & Linguaggio di programmazione ad alto livello, orientato agli oggetti.                                                                                & Creazione di \href{https://7last.github.io/docs/pb/documentazione-interna/glossario\#job}{\textit{job}\textsubscript{G}} per le aggregazioni dei dati in Apache Flink.                                                      \\\hline
	\caption{Linguaggi e formato dati}
\end{longtable}

\subsection{Librerie}
\begin{longtable}{|>{\centering\arraybackslash}m{0.35\textwidth}|>{\centering\arraybackslash}m{0.10\textwidth}|>{\centering\arraybackslash}m{0.45\textwidth}|}
	\hline
	\multicolumn{3}{|c|}{\href{https://7last.github.io/docs/pb/documentazione-interna/glossario\#python}{\textbf{Python}\textsubscript{G}}}                                                                                                          \\
	\hline
	\textbf{Nome}                          & \textbf{Versione} & \textbf{Impiego}                                                                  \\\hline
	\endfirsthead
	\hline
	\textbf{Nome}                          & \textbf{Versione} & \textbf{Impiego}                                                                  \\
	\endhead
	% \texttt{avro} & 1.11.3            & Libreria per la serializzazione dei dati in formato Avro. & Serializzazione dei dati in formato Avro. \\\hline
	\texttt{confluent\_avro}               & 1.8.0             & Serializzazione dei dati in formato Confluent Avro.                                         \\\hline
	\texttt{coverage}                      & 7.5.1             & Strumento per misurare la percentuale di linee di codice e rami coperti dai test. \\\hline
	\texttt{\mySkip{isodate}}                       & 0.6.1             & Libreria per la manipolazione delle date e delle ore in formato ISO8601.          \\\hline
	\texttt{kafka-\href{https://7last.github.io/docs/pb/documentazione-interna/glossario\#python}{python\textsubscript{G}}-\mySkip{ng}}               & 2.2.2             & Client Kafka per \href{https://7last.github.io/docs/pb/documentazione-interna/glossario\#python}{Python\textsubscript{G}}.                                                          \\\hline
	% \texttt{requests}          & 2.32.3            & Libreria per effettuare richieste HTTP.                                           \\\hline
	\texttt{ruff}                          & 0.3.5             & Libreria per l'analisi statica del codice.                                        \\\hline
	\href{https://7last.github.io/docs/pb/documentazione-interna/glossario\#tom's-obvious-minimal-language}{\texttt{toml}\textsubscript{G}}                          & 0.10.2            & Libreria per effettuare il parsing dei file di configurazione in formato \href{https://7last.github.io/docs/pb/documentazione-interna/glossario\#tom's-obvious-minimal-language}{TOML\textsubscript{G}}.    \\\hline

	\multicolumn{3}{|c|}{\textbf{Java}}                                                                                                            \\\hline
	% TODO: controllare quali effettivamente necessarie/utilizzate
	\texttt{flink-streaming-java}          & 1.18.0            & Utilizzo di DataStream API di Flink.                                              \\\hline
	\texttt{flink-connector-kafka}         & 3.1.0-1.18        & Connessione di Flink a Kafka.                                                     \\\hline
	\texttt{flink-clients}                 & 1.18.0            & Creazione di \href{https://7last.github.io/docs/pb/documentazione-interna/glossario\#job}{\textit{job}\textsubscript{G}} di Flink.                                               \\\hline
	\texttt{flink-java}                    & 1.18.0            & Creazione di \href{https://7last.github.io/docs/pb/documentazione-interna/glossario\#job}{\textit{job}\textsubscript{G}} di Flink.                                               \\\hline
	\texttt{flink-avro-confluent-registry} & 1.18.0            & Connessione di Flink a uno \href{https://7last.github.io/docs/pb/documentazione-interna/glossario\#schema-registry}{\textit{schema registry}\textsubscript{G}} che utilizza Avro.            \\\hline
	% \texttt{flink-avro}            & 1.18.0            & Libreria per la connessione di Flink ad Avro.                                      \\\hline
	\texttt{flink-shaded-guava}            & 31.1-jre-17.0     & Gestione delle dipendenze di Flink.                                               \\\hline
	\texttt{\mySkip{slf}4j-simple}                  & 1.7.36            & Implementazione di \mySkip{SLF4J}.                                                         \\\hline
	\href{https://7last.github.io/docs/pb/documentazione-interna/glossario\#lombok}{\texttt{lombok}\textsubscript{G}}                        & 1.18.32           & Libreria per la generazione di codice \textit{boilerplate}.                       \\\hline
	\texttt{maven-assembly-plugin}         & 3.7.1             & Plugin Maven per la creazione di un \textit{fat jar}.                             \\\hline
	\caption{Librerie utilizzate}
\end{longtable}

\subsection{Servizi}
\subsubsection{Redpanda}
\href{https://7last.github.io/docs/pb/documentazione-interna/glossario\#redpanda}{Redpanda\textsubscript{G}} è una piattaforma di streaming sviluppata in C++. Il suo obiettivo è fornire una soluzione leggera, semplice e performante, pensata per essere un'alternativa ad \href{https://7last.github.io/docs/pb/documentazione-interna/glossario\#apache-kafka}{Apache Kafka\textsubscript{G}}. Viene utilizzato per disaccoppiare i dati provenienti dal simulatore.

\begin{itemize}
	\item \textbf{Versione}: v23.3.11;
	\item \textbf{documentazione}: \url{https://docs.redpanda.com/current/home/} [Ultima consultazione: 2024-06-02].
\end{itemize}

\subsubsubsection{Vantaggi}
I vantaggi nell'utilizzo di questo strumento consistono in:
\begin{itemize}
	\item \textbf{\textit{performance}}: è scritto in C++ e utilizza il \textit{framework} Seastar, offrendo un'architettura \textit{thread-per-core} ad alte prestazioni.
	      Ciò permette di ottenere un'elevata \textit{throughput} e latenze costantemente basse, evitando cambi di contesto e blocchi.
	      Inoltre, è progettato per sfruttare l'\textit{hardware} moderno, tra cui unità NVMe, processori \textit{multi-core} e interfacce di rete ad alta velocità;
	\item \textbf{semplicità di configurazione}: oltre al \textit{message \href{https://7last.github.io/docs/pb/documentazione-interna/glossario\#broker}{broker\textsubscript{G}}}, contiene anche un \textit{proxy} HTTP e uno \href{https://7last.github.io/docs/pb/documentazione-interna/glossario\#schema-registry}{\textit{schema registry}\textsubscript{G}};
	\item \textbf{minore richiesta di risorse}: rispetto ad \href{https://7last.github.io/docs/pb/documentazione-interna/glossario\#apache-kafka}{Apache Kafka\textsubscript{G}}, richiede meno risorse per l'esecuzione in locale, rendendolo più adatto per l'esecuzione su \textit{hardware} meno potente;
	\item \textbf{compatibilità con le API di Kafka}: è compatibile con le API di \href{https://7last.github.io/docs/pb/documentazione-interna/glossario\#apache-kafka}{Apache Kafka\textsubscript{G}}, consentendo di utilizzare le librerie e gli strumenti esistenti;
\end{itemize}
\subsubsubsection{Casi d'uso}
Tra i casi d'uso di \href{https://7last.github.io/docs/pb/documentazione-interna/glossario\#redpanda}{Redpanda\textsubscript{G}} si possono citare:
\begin{itemize}
	\item \textbf{\textit{streaming} di eventi}, permettendo la gestione e l'elaborazione di flussi di dati in tempo reale;
	\item \textbf{\textit{data integration}}, agisce come un intermediario flessibile e robusto per l'integrazione dei dati, consentendo la raccolta, il trasporto e la trasformazione dei dati provenienti da diverse sorgenti verso varie destinazioni;
	\item \textbf{elaborazione di \textit{big data}}, permette di gestire e processare enormi volumi di dati in modo efficiente e scalabile;
	\item \textbf{messaggistica \textit{real time}}, supporta la messaggistica in tempo reale tra applicazioni e sistemi distribuiti.
\end{itemize}

\subsubsubsection{Impiego nel progetto}
Il \href{https://7last.github.io/docs/pb/documentazione-interna/glossario\#broker}{\textbf{broker}\textsubscript{G}} \href{https://7last.github.io/docs/pb/documentazione-interna/glossario\#redpanda}{Redpanda\textsubscript{G}} gestisce i dati provenienti dai simulatori e li rende disponibili per i due consumatori. Inoltre,
con lo \href{https://7last.github.io/docs/pb/documentazione-interna/glossario\#schema-registry}{\textit{schema registry}\textsubscript{G}} integrato è possibile garantire la compatibilità tra i dati prodotti dai simulatori e i consumatori.
I consumatori sono:
\begin{itemize}
	\item Il \textbf{connector sink \href{https://7last.github.io/docs/pb/documentazione-interna/glossario\#clickhouse}{ClickHouse\textsubscript{G}}}, che salva i dati nelle tabelle di \href{https://7last.github.io/docs/pb/documentazione-interna/glossario\#clickhouse}{ClickHouse\textsubscript{G}};
	\item \textbf{Apache Flink}, che elabora i dati in tempo reale.
\end{itemize}

\subsubsection{ClickHouse}
\href{https://7last.github.io/docs/pb/documentazione-interna/glossario\#clickhouse}{ClickHouse\textsubscript{G}} è un sistema di gestione di database colonnare \textit{open-source} progettato per l'analisi dei dati in tempo reale e l'elaborazione di grandi volumi di dati.
\begin{itemize}
	\item \textbf{Versione}: v24-alpine;
	\item \textbf{documentazione}: \url{https://clickhouse.com/docs/en/intro} [Ultima consultazione: 2024-06-02].
\end{itemize}

\subsubsubsection{Vantaggi}
I vantaggi nell'utilizzo di questo strumento consistono in:
\begin{itemize}
	\item \textbf{alte prestazioni}, è progettato per eseguire query analitiche complesse in modo estremamente rapido;
	\item \textbf{scalabilità orizzontale}, può essere scalato orizzontalmente su più nodi, permettendo di gestire grandi volumi di dati;
	\item \textbf{elaborazione in tempo reale}, è in grado di gestire l'ingestione e l'elaborazione dei dati in tempo reale, rendendolo ideale per applicazioni che richiedono l'analisi immediata dei dati appena arrivano;
	\item \textbf{compressione efficiente}, utilizza algoritmi di compressione avanzati per ridurre lo spazio di archiviazione e migliorare l'efficienza I/O;
	\item \textbf{facilità di integrazione}, si integra facilmente con molti strumenti di visualizzazione dei dati e piattaforme di business intelligence come \href{https://7last.github.io/docs/pb/documentazione-interna/glossario\#grafana}{Grafana\textsubscript{G}};
	\item \textbf{partizionamento e indici}, supporta il partizionamento dei dati e l'uso di indici per ottimizzare le query;
\end{itemize}
\subsubsubsection{Casi d'uso}
\href{https://7last.github.io/docs/pb/documentazione-interna/glossario\#clickhouse}{ClickHouse\textsubscript{G}} è utilizzato in una varietà di casi d'uso, tra cui:
\begin{itemize}
	\item \textbf{analisi dei \textit{log} e monitoraggio}, utilizzato per l'analisi e il monitoraggio dei log in tempo reale;
	\item \textbf{\textit{business intelligence}}, impiegato in applicazioni di BI per eseguire analisi approfondite dei dati aziendali, supportando la presa di decisioni basata sui dati;
	\item \textbf{\textit{data warehousing}}, funziona come data warehouse per memorizzare e analizzare grandi volumi di dati.
\end{itemize}
\subsubsubsection{Impiego nel progetto}
\href{https://7last.github.io/docs/pb/documentazione-interna/glossario\#clickhouse}{ClickHouse\textsubscript{G}} viene utilizzato per memorizzare i dati grezzi provenienti dai simulatori; attraverso il \textit{connector sink} di \href{https://7last.github.io/docs/pb/documentazione-interna/glossario\#redpanda}{Redpanda\textsubscript{G}}, i \textit{record} pubblicati nei \href{https://7last.github.io/docs/pb/documentazione-interna/glossario\#topic}{\textit{topic}\textsubscript{G}} vengono
salvati in tabelle di \href{https://7last.github.io/docs/pb/documentazione-interna/glossario\#clickhouse}{ClickHouse\textsubscript{G}}.
Inoltre, tramite l'utilizzo di \href{https://7last.github.io/docs/pb/documentazione-interna/glossario\#materialized-view}{Materialized Views\textsubscript{G}}, vengono effettuate delle semplici aggregazioni sui dati, come ad esempio la media oraria o giornaliera, le quali sono poi memorizzate in apposite tabelle,
in modo da poterne monitorare l'andamento nel tempo.\\
Le aggregazioni più complesse che coinvolgono dati provenienti da sensori differenti sono invece effettuate utilizzando Apache Flink, come meglio descritto nella sezione \ref{sec:flink}.\\
\href{https://7last.github.io/docs/pb/documentazione-interna/glossario\#clickhouse}{ClickHouse\textsubscript{G}} si integra semplicemente con \href{https://7last.github.io/docs/pb/documentazione-interna/glossario\#grafana}{Grafana\textsubscript{G}}, attraverso l'utilizzo del plugin \\
\texttt{datasource-\href{https://7last.github.io/docs/pb/documentazione-interna/glossario\#clickhouse}{clickhouse\textsubscript{G}}}, fornito da \href{https://7last.github.io/docs/pb/documentazione-interna/glossario\#grafana}{Grafana\textsubscript{G}} Labs.

\subsubsection{Apache Flink} \label{sec:flink}
Apache Flink è un framework open-source per l'elaborazione dei dati in tempo reale e in \textit{batch}.
Sviluppato in Java e Scala, è progettato per gestire \textit{data stream} in modo efficiente, consentendo l'elaborazione di grandi volumi di dati in tempo reale.
Flink si distingue per la sua capacità di fornire elaborazione a bassa latenza, esecuzione \textit{fault-tolerant} e scalabilità orizzontale.
Come sistema di \textit{build} e gestione delle dipendenze, è stato utilizzato Maven.

\begin{itemize}
	\item \textbf{Versione}: v1.18.1;
	\item \textbf{documentazione}: \url{https://flink.apache.org} [Ultima consultazione: 2024-06-25].
\end{itemize}

\subsubsubsection{Vantaggi}
\begin{itemize}
	\item \textbf{elaborazione a bassa latenza}: Flink è progettato per elaborare i dati in tempo reale con latenza estremamente bassa, rendendolo ideale per applicazioni che richiedono risposte rapide ai cambiamenti dei dati.
	\item \textbf{\textit{fault tolerance}}: Flink utilizza una tecnologia chiamata \textit{Stateful Stream Processing} che garantisce che lo stato dell'applicazione venga memorizzato in modo sicuro e possa essere recuperato in caso di guasti. Questo consente un'elaborazione affidabile e continua anche in presenza di errori hardware o software.
	\item \textbf{scalabilità}: Flink può scalare orizzontalmente su cluster di grandi dimensioni, distribuendo il carico di lavoro tra molteplici nodi per gestire volumi di dati crescenti senza compromettere le prestazioni.
	\item \textbf{modello di programmazione flessibile}: Flink offre due tipologie di API per l'elaborazione dei dati: DataStream API, utilizzato per l'elaborazione di flussi di dati non strutturati in tempo reale, e Table API, un'astrazione di livello superiore per manipolare dati strutturati come tabelle, facilitando l'uso di operazioni simili a SQL;
	\item \textbf{supporto per analisi complesse}: Flink fornisce potenti funzionalità di analisi come aggregazioni, join e \textit{windowing}, che consentono di realizzare analisi complesse sui flussi di dati.
\end{itemize}

\subsubsubsection{Casi d'uso}
Tra i principali casi d'uso di Apache Flink si trovano:
\begin{itemize}
	\item \textbf{applicazioni \textit{event-driven}}: applicazioni \textit{stateful} che elaborano eventi provenienti da uno o più flussi di eventi e reagiscono agli eventi in ingresso attivando calcoli, aggiornamenti di stato o azioni esterne;
	\item \textbf{\textit{data analytics}}: estrazione di informazioni e \textit{insight} a partire dall'elaborazione dei dati grezzi, sia in \textit{real time} che in modalità \textit{batch};
	\item \textbf{\textit{data pipeline}}: ad esempio per la costruzione di \textit{Extract-transform-load} (\href{https://7last.github.io/docs/pb/documentazione-interna/glossario\#extract-transform-load}{ETL\textsubscript{G}}) o integrazione di dati provenienti da sorgenti differenti.
\end{itemize}

\subsubsubsection{Impiego nel progetto}
Per ciascuno degli indici in seguito elencati è stato sviluppato un \href{https://7last.github.io/docs/pb/documentazione-interna/glossario\#job}{\textit{job}\textsubscript{G}} in Apache Flink; i dettagli implementativi di ciascuno di essi sono meglio discussi nella sezione 3.5. %TODO: numero corretto
\begin{itemize}
	\item \href{https://7last.github.io/docs/pb/documentazione-interna/glossario\#heat-index}{\textbf{\textit{Heat Index}}\textsubscript{G}}: una misura che combina la temperatura dell'aria e l'umidità relativa per determinare la temperatura percepita dall'uomo. Questa misura riflette meglio il livello di disagio che una persona potrebbe sperimentare rispetto alla sola temperatura dell'aria;
	\item \textbf{\textit{Charging Efficiency}}: combinando i dati provenienti dai sensori di occupazione dei parcheggi e all'utilizzo delle colonnine di ricarica (relative
	      allo stesso parcheggio), si calcola l'efficienza di utilizzo di queste ultime; viene calcolato sia un indice di utilizzo (\texttt{utilization\_rate}) nel periodo
	      di tempo considerato, sia un indice di efficienza (\texttt{efficiency\_rate}) che rappresenta quanto tempo la colonnina è stata effettivamente utilizzata rispetto
	      al tempo in cui il parcheggio è stato occupato.
\end{itemize}

\subsubsection{Grafana}
È una potente piattaforma di visualizzazione dei dati progettata per creare, esplorare e condividere \href{https://7last.github.io/docs/pb/documentazione-interna/glossario\#dashboard}{\textit{dashboard}\textsubscript{G}}  interattive che visualizzano metriche, \textit{log} e altri dati di monitoraggio in tempo reale.

\begin{itemize}
	\item \textbf{Versione}: v10.3.0;
	\item \textbf{documentazione}: \url{https://grafana.com/docs/grafana/v10.4/} [Ultima consultazione: 2024-06-02].
\end{itemize}

\subsubsubsection{Vantaggi}
\begin{itemize}
	\item \textbf{Facilità d'uso}: possiede un'interfaccia intuitiva che rende facile la creazione e la gestione delle \href{https://7last.github.io/docs/pb/documentazione-interna/glossario\#dashboard}{dashboard\textsubscript{G}};
	\item \textbf{flessibilità}: La capacità di integrarsi con molteplici sorgenti dati e l'ampia gamma di plugin disponibili la rendono estremamente flessibile;
	\item \textbf{personalizzazione}: permette una personalizzazione completa delle \href{https://7last.github.io/docs/pb/documentazione-interna/glossario\#dashboard}{dashboard\textsubscript{G}}, soddisfando ogni possibile necessità di visualizzazione dei dati;
	\item \textbf{gestione degli accessi}: offre funzionalità avanzate di gestione degli accessi e delle autorizzazioni, consentendo di controllare chi può accedere alle \href{https://7last.github.io/docs/pb/documentazione-interna/glossario\#dashboard}{\textit{dashboard}\textsubscript{G}} e quali azioni possono eseguire.
\end{itemize}
\subsubsubsection{Casi d'uso}
\begin{itemize}
	\item \textbf{Monitoraggio delle infrastrutture}: utilizzato per monitorare le prestazioni e la disponibilità delle infrastrutture IT, inclusi server, database, servizi cloud e altro;
	\item \textbf{analisi delle performance delle applicazioni}: utilizzato per monitorare le prestazioni delle applicazioni e identificare eventuali problemi di prestazioni;
	\item \textbf{analisi delle serie temporali}: utilizzato per visualizzare e analizzare dati di serie temporali, come metriche di monitoraggio, log e dati di sensori;
	\item \textbf{business intelligence}: utilizzato per creare \href{https://7last.github.io/docs/pb/documentazione-interna/glossario\#dashboard}{\textit{dashboard}\textsubscript{G}} personalizzate per l'analisi dei dati aziendali e la visualizzazione delle metriche chiave.
\end{itemize}
\subsubsubsection{Impiego nel progetto}
\begin{itemize}
	\item \textbf{Visualizzazione dei dati}: creazione \href{https://7last.github.io/docs/pb/documentazione-interna/glossario\#dashboard}{\textit{dashboard}\textsubscript{G}} interattive che visualizzano i dati salvati su \href{https://7last.github.io/docs/pb/documentazione-interna/glossario\#clickhouse}{ClickHouse\textsubscript{G}};
	\item \textbf{notifiche superamento soglie}: invio di notifiche nel caso in cui vengano superate delle soglie prestabilite, che rappresentano situazioni di eventuale pericolo, forte disagio o disservizio per i cittadini.
\end{itemize}


\newpage

\section{Architettura di sistema}
\subsection{Modello architetturale}
Il prodotto necessita di un'architettura in grado di gestire in tempo reale un vasto flusso di dati provenienti da sensori, offrendo contemporaneamente strumenti di visualizzazione efficaci per rendere comprensibili e utili queste informazioni. A tal fine è stata scelta la \textit{k-architecture}.
\subsubsection{k-architecture}
Introdotta come alternativa alla \textit{Lambda Architecture}, la \textit{Kappa Architecture} semplifica questo modello. In questa architettura i dati vengono acquisiti, elaborati e analizzati in tempo reale senza la necessità di separare il flusso di dati in due percorsi distinti per il batch processing e il processing in tempo reale. 
\subsubsubsection{Vantaggi}
\begin{itemize}
    \item \textbf{Semplicità}: semplifica notevolmente il processo di sviluppo e manutenzione dei sistemi di data processing in tempo reale, in quanto elimina la necessità di gestire due percorsi separati per il batch processing e il processing in tempo reale;
    \item \textbf{costi ridotti}: la gestione di un'unica pipeline di dati riduce i costi di sviluppo e manutenzione;
    \item \textbf{reattività}: grazie alla capacità di elaborare i dati in arrivo immediatamente senza attendere il completamento di finestre di tempo predefinite, consente di ottenere risposte e feedback in tempo reale sui dati;
    \item \textbf{scalabilità}: i sistemi basati su quest'architettura possono essere facilmente scalati per gestire grandi volumi di dati e picchi di carico improvvisi, poiché è progettata per essere distribuita e può essere facilmente adattata alle esigenze di crescita del sistema;
    \item \textbf{flessibilità}: è altamente flessibile e può essere implementata utilizzando una varietà di strumenti e tecnologie.
\end{itemize}
\subsubsection{Componenti di sistema}
%TODO inserire immagine dei componenti di sistema + descrizione dei vari componenti
\subsubsubsection*{Sorgenti di dati}
Costituite dai simulatori di sensori IoT distribuiti per la città.
\subsubsubsection*{Streaming layer}
Gestisce il flusso di dati in tempo reale provenienti dai sensori. È composto da \textit{Redpanda} e \textit{Schema Registry}.
\subsubsubsection*{Processing layer}
Elabora i dati in tempo reale per calcolare i KPI richiesti. È composto da \textit{Flink}.
\subsubsubsection*{Storage layer}
Memorizza i dati elaborati per l'analisi e la visualizzazione. È composto da \textit{ClickHouse}.
\subsubsubsection*{Data visualization layer}
Visualizza i dati elaborati in modo chiaro e intuitivo. È composto da \textit{Grafana}. 

%-------------------CHIEDERE QUESTA PARTE----------------------%
\subsection{Flusso di dati}
%TODO guardare disegno presentato a sal
\subsection{Architettura dei simulatori}
\subsubsection{Modulo simulatori sensori}
%TODO inserire immagine modulo dei sensori
\subsubsection{Modulo Writers}
\subsubsection{Modulo Threading/Scheduling}
\subsubsection{Progettazione - Panoramica UML}
%--------------------------------------------------------------%
\subsection{Redpanda}
\subsubsection{Kafka topic}
I Kafka topic sono categorie o canali di messaggi all'interno di \textit{Redpanda}. Un topic in Kafka è come una cassetta postale virtuale o una categoria di messaggi in cui i dati vengono pubblicati dai produttori e letti dai consumatori.
\subsubsection{Formato messaggi}
\textit{7Last} ha scelto di adottare lo standard Avro per i messaggi scambiati tra i produttori e i consumatori. Avro è un sistema di serializzazione dati che offre un modo efficiente per rappresentare dati complessi in un formato binario, rendendoli adatti per il trasporto su rete o per la persistenza su disco. Uno dei principali vantaggi di Avro è la possibilità di definire uno schema per i dati, che viene incluso nei dati stessi. Ecco una panoramica del formato dei messaggi Avro:
\begin{itemize}
    \item \textbf{schema}: definito in formato JSON e descrive la struttura dei dati. Include informazioni come il tipo di ogni campo e la loro posizione all'interno della struttura dei dati;
    \item \textbf{serializzazione binaria}: Avro serializza i dati in un formato binario compatto, che rende efficiente il trasporto e la memorizzazione dei dati. Utilizza un'organizzazione binaria che incorpora lo schema dei dati insieme ai dati stessi. Questo significa che non c'è bisogno di includere esplicitamente il tipo di dato per ogni campo, poiché lo schema fornisce questa informazione;
    \item \textbf{compatibilità}: è progettato per supportare l'evoluzione dei dati nel tempo. Puoi aggiungere nuovi campi, rimuovere campi esistenti o modificare il tipo di dati di un campo mantenendo la compatibilità con le versioni precedenti degli schemi;
    \item \textbf{condivisione dello schema}:supporta la condivisione dello schema tramite un registro dello schema (Schema Registry). Questo consente di registrare gli schemi Avro utilizzati nel sistema in un registro centralizzato, in modo che i produttori e i consumatori possano recuperare gli schemi necessari quando ne hanno bisogno.
\end{itemize}

% TODO finire questa parte simulatore
\subsubsection{Modulo producers}
\subsubsection{Modulo serializers}
\subsubsection{Modulo simulators}

\subsection{Flink - Processing Layer}
\subsubsection{Introduzione}
È un framework di elaborazione dati distribuito e open-source che si distingue per la sua capacità di gestire sia dati di flusso in tempo reale che dati batch. Una delle sue caratteristiche principali è la capacità di gestire dati in tempo reale con latenze molto basse. Ciò significa che può elaborare i dati man mano che arrivano, consentendo alle applicazioni di reagire istantaneamente ai cambiamenti nell'input. Questa caratteristica è particolarmente importante per le applicazioni che richiedono analisi in tempo reale, come il monitoraggio di sensori, il rilevamento di anomalie o la personalizzazione di contenuti.
\subsubsection{Componenti Flink \& Processing Layer}
È costituito da diverse componenti fondamentali che lavorano insieme per consentire l'elaborazione efficiente e scalabile dei dati in tempo reale e batch. Queste componenti formano il cuore del sistema Flink e forniscono le basi per la sua potente capacità di elaborazione dei dati. In questa sezione, esamineremo le principali componenti di Flink e il loro ruolo nella creazione di un'infrastruttura robusta per l'analisi dei dati.
\begin{itemize}
    \item \textbf{JobManager}: è il componente centrale di Flink responsabile della pianificazione e del coordinamento dei job di elaborazione dei dati. Gestisce il flusso di lavoro complessivo, assegnando i task ai TaskManager per l'esecuzione;
    \item \textbf{TaskManager}: è responsabile dell'esecuzione effettiva delle operazioni di elaborazione dei dati, eseguendo i task assegnati loro dal JobManager e gestendo il caricamento, l'elaborazione e la distribuzione dei dati all'interno del cluster;
    \item \textbf{Processing layer}: è responsabile dell'esecuzione delle operazioni di elaborazione dei dati all'interno del cluster distribuito. Questa layer sfrutta le risorse di calcolo e memorizzazione disponibili nei nodi del cluster per eseguire le operazioni di trasformazione, aggregazione, filtraggio e altro ancora sui dati in ingresso. Utilizzando un modello di esecuzione distribuita, la Processing Layer di Flink è in grado di scalare orizzontalmente per gestire grandi volumi di dati e carichi di lavoro ad alta intensità computazionale.
\end{itemize}
\subsubsection{Processing layer data-flow}
%TODO mettere foto del processing layer data-flow di flink
\begin{enumerate}
    \item \textbf{Acquisizione dei dati};
    \item \textbf{partizione e distribuzione}: i dati vengono divisi in parti più piccole e distribuiti tra i nodi del cluster per massimizzare l'utilizzo delle risorse;
    \item \textbf{pianificazione dei task}: il JobManager assegna compiti di elaborazione ai TaskManager basandosi sullo stato del cluster e sull'ottimizzazione delle prestazioni;
    \item \textbf{esecuzione dei task}: i TaskManager eseguono i compiti assegnati in parallelo, elaborando i dati in base alla logica definita nell'applicazione Flink;
    \item \textbf{scambio e movimento dei dati}: i dati possono essere scambiati e spostati tra i nodi del cluster per supportare operazioni complesse come il join o l'aggregazione;
    \item \textbf{persistenza e output}: una volta elaborati, i risultati vengono eventualmente salvati o inviati ad altre destinazioni per l'analisi o l'utilizzo successivo.
\end{enumerate}
\subsubsection{Modello per il calcolo dei KPI}
%TODO inserire foto del modello per il calcolo del punteggio di salute

% KPI che useremo :european air quality index, heat index, charging station efficiency index, comfort heat zone index 


\subsection{Database ClickHouse}
Come detto in precedenza, il database adottato è \textit{ClickHouse}. Per ogni sensore è stata creata una tabella \textit{MergeTree}, che permette di memorizzare i dati in modo efficiente e di eseguire query complesse in modo veloce.

\subsubsection{Funzionalità utilizzate}
\subsubsubsection{Materialized View}
Sono una potente funzionalità di \textit{ClickHouse} per migliorare le prestazioni delle query e semplificare l'analisi dei dati. In sostanza, una materialized view è una vista pre-calcolata o una copia di una query, memorizzata fisicamente su disco in forma tabellare. Ciò consente di evitare il calcolo ripetuto dei risultati della query ogni volta che viene eseguita.
\subsubsubsection*{Documentazione}
\url{https://clickhouse.com/docs/en/guides/developer/cascading-materialized-views} (Consultato il 2024-06-05)
\subsubsubsection*{Utilizzi}
\begin{itemize}
    \item \textbf{Aggregazioni pre-calcoltate}: le materialized view possono essere utilizzate per memorizzare i risultati di aggregazioni complesse, come somme, medie, conteggi, ecc., in modo che non debbano essere calcolati ogni volta che viene eseguita una query;
    \item \textbf{rapporti pre-calcolati}: possono essere utilizzate per memorizzare i risultati di query complesse o di rapporti, in modo che i risultati siano immediatamente disponibili senza dover eseguire la query ogni volta;
    \item \textbf{join ottimizzati}: possono essere utilizzate per memorizzare i risultati di join complessi tra più tabelle, in modo che i risultati siano immediatamente disponibili senza dover eseguire il join ogni volta;
    \item \textbf{filtraggio e selezione efficiente}: possono essere utilizzate per filtrare e selezionare dati in base a criteri specifici, migliorando le prestazioni delle query che richiedono l'accesso solo a una parte dei dati.
\end{itemize}
\subsubsubsection{MergeTree}
MergeTree è uno dei principali motori di archiviazione di ClickHouse, progettato per gestire grandi volumi di dati e fornire elevate prestazioni di lettura e scrittura. È particolarmente adatto per applicazioni in cui i dati vengono aggiunti in modo incrementale e le query vengono eseguite su intervalli di tempo specifici.
Le caratteristiche principali sono:
\begin{itemize}
    \item \textbf{partizionamento}, in cui i dati vengono partizionati in base a una colonna di data o di tempo, in modo che i dati più recenti siano memorizzati in partizioni separate e possano essere facilmente eliminati o archiviati;
    \item \textbf{ordine dei dati}, dove i dati vengono ordinati in base a una colonna di ordinamento, in modo che i dati siano memorizzati in modo sequenziale e possano essere letti in modo efficiente;
    \item \textbf{indice primario}, tramite il quale i dati vengono indicizzati in base a una colonna di chiave primaria, in modo che le query di ricerca e di join siano veloci ed efficienti;
    \item \textbf{merging dei dati}, in questo modo i dati vengono uniti in modo incrementale in background, in modo che le query di aggregazione e di analisi siano veloci ed efficienti;
    \item \textbf{compressione}, i dati vengono compressi in modo efficiente per ridurre lo spazio di archiviazione e migliorare le prestazioni di lettura e scrittura;
    \item \textbf{replica e distribuzione}, i dati possono essere replicati e distribuiti su più nodi per garantire l'affidabilità e la disponibilità del sistema.
\end{itemize}
\subsubsubsection*{Documentazione}
\url{https://clickhouse.com/docs/en/engines/table-engines/mergetree-family/mergetree} \\(Consultato il 2024-06-05)

\subsubsubsection*{Utilizzi}
\begin{itemize}
    \item \textbf{Analisi dei dati storici}: i dati storici vengono memorizzati in tabelle MergeTree per consentire l'analisi e l'elaborazione dei dati storici;
    \item \textbf{applicazioni di business intelligence}: i dati vengono memorizzati in tabelle MergeTree per consentire l'analisi e la generazione di report per le applicazioni di business intelligence;
    \item \textbf{log e monitoraggio}: i dati di log e di monitoraggio vengono memorizzati in tabelle MergeTree per consentire l'analisi e il monitoraggio delle attività di sistema.
\end{itemize}


\subsubsection{Trasferimento dati tramite Materialized View}
Le Materialized View in ClickHouse sono viste che memorizzano fisicamente i risultati di una query specifica in modo da permettere un accesso rapido e efficiente ai dati pre-elaborati. Quando vengono create, le Materialized View eseguono la query definita e archiviano i risultati in una struttura di dati ottimizzata per l'accesso veloce. Questo consente di evitare il calcolo ripetuto dei risultati della query ogni volta che viene eseguita, migliorando notevolmente le prestazioni complessive del sistema. I vantaggi derivanti da questo approccio sono molteplici, tra questi troviamo:
\begin{itemize}
    \item \textbf{prestazioni ottimizzate}: grazie alla memorizzazione fisica dei risultati delle query, le Materialized View consentono un accesso rapido ai dati pre-elaborati, riducendo i tempi di risposta delle query complesse;
    \item \textbf{riduzione del carico di lavoro}: trasferendo i dati pre-elaborati in Materialized View, si riduce il carico di lavoro sul sistema sorgente, consentendo una maggiore scalabilità e riducendo il rischio di sovraccarico del sistema durante le operazioni di estrazione dei dati;
    \item \textbf{sempre aggiornate}: possono essere progettate per aggiornarsi automaticamente in risposta alle modifiche nei dati sottostanti, garantendo che i risultati siano sempre aggiornati e coerenti con lo stato attuale dei dati;
    \item \textbf{semplificazione dell'architettura}: è possibile semplificare l'architettura complessiva del sistema eliminando la necessità di eseguire query complesse e costose ogni volta che si accede ai dati.
\end{itemize}
%TODO prendere le query che creano le materialized view
\subsubsection{Misurazioni isole ecologiche}
Di seguito viene riportata la configurazione della tabella per le misurazioni delle isole ecologiche. Le misurazioni includono:
\begin{itemize}
    \item \textbf{sensor\_uuid}: identificativo univoco del sensore (formato UUID);
    \item \textbf{sensor\_name}: nome del sensore (formato String);
    \item \textbf{timestamp}: data e ora della misurazione (formato DateTime64);
    \item \textbf{latitude}: latitudine del sensore (formato Float64);
    \item \textbf{longitude}: longitudine del sensore (formato Float64);
    \item \textbf{filling\_value}: percentuale di riempimento dell'isola ecologica (formato Float32).
\end{itemize}
%TODO mettere foto della tabella isole ecologiche
\subsubsection{Misurazioni temperatura}
Di seguito viene riportata la configurazione della tabella per le misurazioni della temperatura. Le misurazioni includono:
\begin{itemize}
    \item \textbf{sensor\_uuid}: identificativo univoco del sensore (formato UUID);
    \item \textbf{sensor\_name}: nome del sensore (formato String);
    \item \textbf{timestamp}: data e ora della misurazione (formato DateTime64);
    \item \textbf{value}: valore della temperatura rilevata (formato Float32);
    \item \textbf{latitude}: latitudine del sensore (formato Float64);
    \item \textbf{longitude}: longitudine del sensore (formato Float64);
\end{itemize}
%TODO mettere foto della tabella temperatura
\subsubsection{Misurazioni traffico}
Di seguito viene riportata la configurazione della tabella per le misurazioni della traffico. Le misurazioni includono:
\begin{itemize}
    \item \textbf{sensor\_uuid}: identificativo univoco del sensore (formato UUID);
    \item \textbf{sensor\_name}: nome del sensore (formato String);
    \item \textbf{timestamp}: data e ora della misurazione (formato DateTime64);
    \item \textbf{latitude}: latitudine del sensore (formato Float64);
    \item \textbf{longitude}: longitudine del sensore (formato Float64);
    \item \textbf{vehicles}: numero di veicoli rilevati (formato Int32);
    \item \textbf{avg\_speed}: velocità media del traffico (formato Float32).
\end{itemize}
%TODO mettere foto della tabella traffico
\subsection{Grafana}
Grafana è uno strumento di analisi e monitoraggio che permette di visualizzare dati provenienti da una varietà di fonti. È sviluppato principalmente in Go e Typescript ed è noto per la sua capacità di creare dashboard personalizzabili e intuitive.
\subsubsection{Dashboard}
%metti stessa cosa del manuale utente riguardo le dashboard. 

\subsubsection{ClickHouse datasource plugin}
Il plugin ClickHouse per Grafana è un'implementazione che consente di utilizzare ClickHouse come fonte di dati per Grafana. Questo plugin facilita la connessione e l'interrogazione dei dati archiviati in ClickHouse direttamente da Grafana, permettendo di creare dashboard dinamiche e interattive.
\subsubsubsection*{Documentazione}
\url{https://grafana.com/grafana/plugins/grafana-clickhouse-datasource/}
\subsubsubsection{Configurazione del Datasource}
La configurazione  grafana/provisioning/datasources/default.yaml


\subsubsection{Variabili Grafana}
\subsubsubsection{Documentazione}
\url{https://grafana.com/docs/grafana/latest/dashboards/variables/} (Consultato il 2024-06-05)
\subsubsubsection*{Variabili nella dashboard principale}
Le variabili presenti nella dashboard principale sono:
\begin{itemize}
    \item \textbf{tipo sensore}: permette di selezionare il tipo di sensore da visualizzare (temperatura, traffico, isola ecologica);
    \item \textbf{nome sensore}: permette di selezionare il nome del sensore da visualizzare (es. sensore1, sensore2, ecc.);
\end{itemize}
%TODO mettere foto codice query
\subsubsubsection*{Variabili nella dashboard dettagliata}
Le variabili presenti nelle dashboard dettagliate sono:
\begin{itemize}
    \item \textbf{nome sensore}: permette di selezionare il nome del sensore da visualizzare (es. sensore1, sensore2, ecc.);
\end{itemize}

\subsubsection{Grafana Alerts}
Sono una funzionalità che permettono di definire, configurare e gestire avvisi basati su condizioni specifiche rilevate nei dati monitorati. Questi avvisi consentono agli utenti di essere informati tempestivamente su eventuali problemi o cambiamenti critici nei loro sistemi, applicazioni o infrastrutture.
\subsubsubsection*{Documentazione}
\url{https://grafana.com/docs/grafana/latest/alerting/} (Consultato il 2024-06-05)
\subsubsubsection{Configurazione delle regole di alert}
Definiscono le condizioni che devono essere soddisfatte per attivare un alert. Gli eventi che generano un alert sono:
\begin{itemize}
    \item temperatura maggiore di 40°C per più di 30 minuti;
    \item isola ecologica piena al 100\% per più di 24 ore;
    \item superamento dell'indice 3 dell'EAQI (indice di qualità dell'aria);
    \item livello di precipitazioni superiore a 10 mm in 1 ora.
\end{itemize}
Gli alert possono possedere tre diversi tipi di stati:
\begin{itemize}
    \item \textbf{normal}, indica che l'alert non è attivo perché le condizioni definite per l'attivazione dell'avviso non sono soddisfatte;
    \item \textbf{pending}, indica che le metriche monitorate stanno iniziando a deviare dalle condizioni normali ma non hanno ancora soddisfatto completamente le condizioni per attivare l'alert;
    \item \textbf{firing}, significa che le condizioni definite per l'avviso sono state soddisfatte e l'alert è attivo.
\end{itemize}
\subsubsubsection{Configurazione canale di notifica}
Per configurare un canale di notifica è necessario:
\begin{enumerate}
    \item nel menù di sinistra, cliccare sull'icona "Alerting";
    \item selezionare la voce "Notification channels";  
    \item cliccare sul pulsante "Add channel" per aggiungere un nuovo canale di notifica;
    \item selezionare il tipo di canale di notifica desiderato tra quelli disponibili;
    \item configurare le impostazioni del canale di notifica in base alle proprie esigenze;
    \item cliccare sul pulsante "Save" per salvare le impostazioni del canale di notifica.
\end{enumerate}
\textit{7Last} ha deciso di rendere risponibile il server \textit{Discord} configurato a questo scopo e raggiungibile a questo link:
\begin{center}
    \url{https://discord.com/channels/1214553333113556992/1241974479345942568}
\end{center}
\subsubsection{Altri plugin}
\subsubsubsection{Orchestra Cities Map plugin}
Progettato per facilitare la visualizzazione e l'analisi dei dati geospaziali all'interno di piattaforme di pianificazione urbana e sviluppo territoriale.\\
Le principali funzionalità offerte da questo plugin sono:
\begin{itemize}
    \item \textbf{visualizzazione dei dati geospaziali}: consente agli utenti di visualizzare dati geografici, come mappe, strati di dati GIS (Geographic Information System), punti di interesse e altre informazioni territoriali;
    \item \textbf{interfaccia interattiva}: offre un'interfaccia utente intuitiva e interattiva che consente agli utenti di esplorare e interagire con i dati geospaziali in modo dinamico;
    \item \textbf{personalizzazione}: offre opzioni di personalizzazione per adattarsi alle esigenze specifiche dell'utente o dell'applicazione;
    \item \textbf{analisi dei dati}: oltre alla semplice visualizzazione dei dati geospaziali, il plugin può anche supportare funzionalità avanzate di analisi dei dati, come l'identificazione di cluster, la creazione di heatmap e l'esecuzione di analisi spaziali per identificare tendenze o pattern significativi nei dati territoriali;
    \item \textbf{integrazione}: è progettato per integrarsi facilmente con altre componenti dell'ecosistema Orchestra Cities e con altre piattaforme software di pianificazione urbana e sviluppo territoriale.
\end{itemize}

\subsubsubsection*{Documentazione}
\url{https://grafana.com/grafana/plugins/orchestracities-map-panel/?tab=installation} (Consultato il 2024-06-05)


\newpage

\section{Architettura di deployment}
Per implementare ed eseguire l'intero stack tecnologico e i livelli del modello architetturale, viene creato un ambiente \textit{Docker} che riproduce la suddivisione e la distribuzione dei servizi. In particolare, per l'ambiente di produzione, sono stati creati i seguenti container:
\begin{itemize}
    \item \textbf{Data feed}
        \begin{itemize}
            \item Container: \textbf{Simulator};
            \item Descrizione: simula la generazione di dati;
        \end{itemize}
    \item \textbf{Streaming layer}
        \begin{itemize}
            \item Container: \textbf{Redpanda};
            \item Descrizione: definisce il flusso di dati in tempo reale;
            \item Componenti di supporto: schema registry;
            \item Porta: 18082.
        \end{itemize}
    \item \textbf{Processing Layer}
        \begin{itemize}
            \item Container: \textbf{Flink};
            \item Descrizione: pianifica, assegna e coordina l'esecuzione dei task di elaborazione dei dati su un cluster di nodi, garantendo prestazioni elevate, scalabilità e affidabilità nell'elaborazione dei dati.
        \end{itemize}
    \item \textbf{Storage Layer}
        \begin{itemize}
            \item Container: \textbf{Clickhouse};
            \item Descrizione: memorizza i dati;
            \item Porta: 8123.
        \end{itemize}
    \item \textbf{Data Visualization Layer}
        \begin{itemize}
            \item Container: \textbf{Grafana};
            \item Descrizione: visualizza i dati;
            \item Porta: 3000.
        \end{itemize}
\end{itemize}



\newpage


\section{Requisiti}
\subsection{Requisiti funzionali}
\begin{longtable}{|>{\centering\arraybackslash}m{0.10\textwidth}|>{\centering\arraybackslash}m{0.20\textwidth}|>{\centering\arraybackslash}m{0.20\textwidth}|>{\centering\arraybackslash}m{0.4\textwidth}|}
	\hline
	\textbf{Codice} & \textbf{Importanza} & \textbf{Stato}& \textbf{Descrizione}\\\hline
	\endfirsthead
	\hline
	\textbf{Codice} & \textbf{Importanza} & \textbf{Stato}& \textbf{Descrizione}\\\hline
	\endhead
	\hline
	RF-1            & Obbligatorio        & Soddisfatto & La parte \textit{IoT} dovrà essere simulata attraverso tool di generazione di dati casuali che tuttavia siano verosimili.
	\\\hline
	RF-2            & Obbligatorio        & Soddisfatto & Il sistema dovrà permettere la visualizzazione dei dati in tempo reale.
	\\\hline
	RF-3            & Obbligatorio        & Soddisfatto & Il sistema dovrà permettere la visualizzazione dei dati storici.
	\\\hline
	RF-4            & Obbligatorio        & Soddisfatto & L'utente deve poter accedere all'applicativo senza bisogno di autenticazione.
	\\\hline
	RF-5            & Obbligatorio        & Soddisfatto & L'utente dovrà poter visualizzare su una mappa la posizione geografica dei sensori.
	\\\hline
	RF-6            & Obbligatorio        & Soddisfatto & I tipi di dati che il sistema dovrà visualizzare sono: temperatura, umidità, qualità dell'aria, precipitazioni, traffico, stato delle colonnine di ricarica, stato di occupazione dei parcheggi, stato di riempimento delle isole ecologiche e livello di acqua.
	\\\hline
	RF-7            & Obbligatorio        & Soddisfatto & I dati dovranno essere salvati su un database OLAP.
	\\\hline
	RF-8            & Obbligatorio        & Soddisfatto & I sensori di temperatura rilevano i dati in gradi Celsius
	\\\hline
	RF-9            & Obbligatorio        & Soddisfatto & I sensori di umidità rilevano la percentuale di umidità nell’aria.
	\\\hline
	RF-10           & Obbligatorio        & Soddisfatto & I sensori livello acqua rilevano il livello di acqua nella zona di installazione
	\\\hline
	RF-11           & Obbligatorio        & Soddisfatto & I dati provenienti dai sensori dovranno contenere i seguenti dati: id \href{https://7last.github.io/docs/pb/documentazione-interna/glossario\#sensore}{sensore\textsubscript{G}}, data, ora e valore.
	\\\hline
	RF-12           & Obbligatorio        & Soddisfatto & Sviluppo di componenti quali \href{https://7last.github.io/docs/pb/documentazione-interna/glossario\#widget}{widget\textsubscript{G}} e grafici per la visualizzazione dei dati nelle \href{https://7last.github.io/docs/pb/documentazione-interna/glossario\#dashboard}{dashboard\textsubscript{G}}.
	\\\hline
	RF-13           & Obbligatorio        & Soddisfatto & Il sistema deve permettere di visualizzare una \href{https://7last.github.io/docs/pb/documentazione-interna/glossario\#dashboard}{dashboard\textsubscript{G}} generale con tutti i dati dei sensori.
	\\\hline
	RF-14           & Obbligatorio        & Soddisfatto & Il sistema deve permettere di visualizzare una \href{https://7last.github.io/docs/pb/documentazione-interna/glossario\#dashboard}{dashboard\textsubscript{G}} contenente tutti i dati dei sensori che monitorano l'ambiente.
	\\\hline
	RF-15           & Obbligatorio        & Soddisfatto & Il sistema deve permettere di visualizzare una \href{https://7last.github.io/docs/pb/documentazione-interna/glossario\#dashboard}{dashboard\textsubscript{G}} contenente tutti i dati dei sensori che monitorano gli aspetti urbani.
	\\\hline
	RF-16           & Obbligatorio        & Soddisfatto & Il sistema deve permettere di visualizzare una sezione specifica per ciascuna categoria di sensori.
	\\\hline
	RF-17           & Obbligatorio        & Soddisfatto & Nella \href{https://7last.github.io/docs/pb/documentazione-interna/glossario\#dashboard}{dashboard\textsubscript{G}} dei dati grezzi dovranno essere presenti: una mappa interattiva, un \href{https://7last.github.io/docs/pb/documentazione-interna/glossario\#widget}{widget\textsubscript{G}} con il conteggio totale dei sensori divisi per tipo, una tabella contente tutti i sensori e la data in cui essi hanno trasmesso l'ultima volta. Inoltre verranno mostrate delle tabelle con i dati filtrabili suddivisi per \href{https://7last.github.io/docs/pb/documentazione-interna/glossario\#sensore}{sensore\textsubscript{G}} e un grafico \href{https://7last.github.io/docs/pb/documentazione-interna/glossario\#time-series}{time series\textsubscript{G}} con tutti i dati grezzi.
	\\\hline
	RF-18           & Obbligatorio        & Soddisfatto & Nella \href{https://7last.github.io/docs/pb/documentazione-interna/glossario\#dashboard}{dashboard\textsubscript{G}} dei dati ambientali dovranno essere presenti delle sezioni contenenti i \href{https://7last.github.io/docs/pb/documentazione-interna/glossario\#panel}{panel\textsubscript{G}} relativi ai sensori di temperatura, umidità, precipitazioni, livello dei fiumi e qualità dell'aria.
	\\\hline
	RF-19           & Obbligatorio        & Soddisfatto & Nella \href{https://7last.github.io/docs/pb/documentazione-interna/glossario\#dashboard}{dashboard\textsubscript{G}} dei dati legati agli aspetti urbani dovranno essere presenti delle sezioni contenenti i \href{https://7last.github.io/docs/pb/documentazione-interna/glossario\#panel}{panel\textsubscript{G}} relativi ai sensori di parcheggio, traffico, isole ecologiche e colonnine di ricarica.
	\\\hline
	RF-20           & Obbligatorio        & Soddisfatto & Nella sezione della temperatura dovranno essere visualizzati: un grafico \href{https://7last.github.io/docs/pb/documentazione-interna/glossario\#time-series}{time series\textsubscript{G}}, una mappa interattiva, la temperatura media, minima e massima di un certo periodo di tempo, la temperatura in tempo reale e la temperatura media per settimana e mese.
	\\\hline
	RF-21           & Obbligatorio        & Soddisfatto & Nella sezione dell'umidità dovranno essere visualizzati: un grafico \href{https://7last.github.io/docs/pb/documentazione-interna/glossario\#time-series}{time series\textsubscript{G}}, una mappa interattiva, l'umidità media, minima e massima di un certo periodo di tempo e l'umidità in tempo reale.
	\\\hline
	RF-22           & Obbligatorio        & Soddisfatto & Nella sezione della qualità dell'aria dovranno essere visualizzati: un grafico \href{https://7last.github.io/docs/pb/documentazione-interna/glossario\#time-series}{time series\textsubscript{G}}, una mappa interattiva, la qualità media dell'aria in un certo periodo e in tempo reale, i giorni con la qualità dell'aria migliore e peggiore in un certo periodo di tempo.
	\\\hline
	RF-23           & Obbligatorio        & Soddisfatto & Nella sezione delle precipitazioni dovranno essere visualizzati: un grafico \href{https://7last.github.io/docs/pb/documentazione-interna/glossario\#time-series}{time series\textsubscript{G}}, una mappa interattiva, la quantità media di precipitazioni in un certo periodo e in tempo reale, i giorni con la quantità di precipitazioni maggiore e minore in un certo periodo di tempo.
	\\\hline
	RF-24           & Obbligatorio        & Soddisfatto & Nella sezione del livello di acqua dovranno essere visualizzati: un grafico \href{https://7last.github.io/docs/pb/documentazione-interna/glossario\#time-series}{time series\textsubscript{G}}, una mappa interattiva, il livello medio di acqua in un certo periodo e in tempo reale.
	\\\hline
	RF-25           & Obbligatorio        & Soddisfatto & Nella sezione delle isole ecologiche dovranno essere visualizzati: una mappa interattiva con il rispettivo stato di riempimento e il conteggio di isole ecologiche suddivise per stato di riempimento in tempo reale.
	\\\hline
	RF-26           & Obbligatorio        & Soddisfatto & Nella sezione dei parcheggi dovranno essere visualizzati: una mappa interattiva con il rispettivo stato di occupazione e il conteggio di parcheggi suddivisi per stato di occupazione in tempo reale.
	\\\hline
	RF-27           & Obbligatorio        & Soddisfatto & Nella sezione delle colonnine di ricarica dovranno essere visualizzati: una mappa interattiva contenente anche lo stato e il numero di colonnine di ricarica suddivise per stato in tempo reale.
	\\\hline
	RF-28           & Obbligatorio        & Soddisfatto & Nella sezione del traffico dovranno essere visualizzati: un grafico \href{https://7last.github.io/docs/pb/documentazione-interna/glossario\#time-series}{time series\textsubscript{G}}, il numero di veicoli e la velocità media in tempo reale, il calcolo dell'ora di punta sulla base del numero di veicoli e velocità media.
	\\\hline
	RF-29           & Obbligatorio        & Soddisfatto & Nel caso in cui non ci siano dati visualizzabili, il sistema deve notificare l'utente mostrando un opportuno messaggio.
	\\\hline
	RF-30           & Obbligatorio        & Soddisfatto & I sensori di qualità dell'aria inviano i seguenti dati: \textit{PM10}, \textit{PM2.5}, \textit{NO2}, \textit{CO}, \textit{O3}, \textit{SO2} in $\mu g/m^3$.
	\\\hline
	RF-31           & Obbligatorio        & Soddisfatto & I sensori di precipitazioni inviano la quantità di pioggia caduta in mm.
	\\\hline
	RF-32           & Obbligatorio        & Soddisfatto & I sensori di traffico inviano il numero di veicoli rilevati e la velocità in km/h.
	\\\hline
	RF-33           & Obbligatorio        & Soddisfatto & Le colonnine di ricarica inviano lo stato di occupazione e il tempo mancante alla fine della ricarica (se occupate) o il tempo passato dalla fine dell'ultima ricarica (se libere).
	\\\hline
	RF-34           & Obbligatorio        & Soddisfatto & I sensori di parcheggio inviano lo stato di occupazione del parcheggio (1 se occupato, 0 se libero) e il timestamp dell'ultimo cambiamento di stato.
	\\\hline
	RF-35           & Obbligatorio        & Soddisfatto & Le isole ecologiche inviano lo stato di riempimento come percentuale.
	\\\hline
	RF-36           & Obbligatorio        & Soddisfatto & I sensori di livello di acqua inviano il livello di acqua in cm.
	\\\hline
	RF-37           & Obbligatorio        & Soddisfatto & Il sistema deve permettere di filtrare i dati visualizzati in base a un intervallo di tempo.
	\\\hline
	RF-38           & Obbligatorio        & Soddisfatto & Il sistema deve permettere di filtrare i dati visualizzati in base al \href{https://7last.github.io/docs/pb/documentazione-interna/glossario\#sensore}{sensore\textsubscript{G}} che li ha generati.
	\\\hline
	RF-39           & Desiderabile        & Soddisfatto & Devono essere messe in relazione più sorgenti di dati.
	\\\hline
	RF-40           & Desiderabile        & Soddisfatto & Nei grafici \href{https://7last.github.io/docs/pb/documentazione-interna/glossario\#time-series}{time series\textsubscript{G}} i dati devono essere aggregati calcolando la media di 5 minuti, in modo da risultare più leggibili.
	\\\hline
	RF-41           & Obbligatorio        & Soddisfatto & Deve essere implementato almeno un simulatore di dati.
	\\\hline
	RF-42           & Desiderabile        & Soddisfatto & Devono essere implementati più simulatori di dati.
	\\\hline
	RF-43           & Obbligatorio        & Soddisfatto & I simulatori devono produrre dei dati verosimili.
	\\\hline
	RF-44           & Obbligatorio        & Soddisfatto & Per ciascuna tipologia di \href{https://7last.github.io/docs/pb/documentazione-interna/glossario\#sensore}{sensore\textsubscript{G}} dev'essere sviluppata almeno una sezione.
	\\\hline
	RF-45           & Opzionale           & Non soddisfatto & Deve essere implementata una funzionalità di previsione di dati futuri della temperature, basandosi sui dati dell'anno e della settimana precedente.
	\\\hline
	RF-46           & Desiderabile        & Soddisfatto & Deve esistere una \href{https://7last.github.io/docs/pb/documentazione-interna/glossario\#dashboard}{dashboard\textsubscript{G}} per la visualizzazione della posizione geografica dei sensori su una mappa.
	\\\hline
	RF-47           & Opzionale           & Soddisfatto & Deve essere presente un sistema di notifiche che allerti l'utente nel caso in cui la temperatura superi i 40°C per più di 30 minuti.
	\\\hline
	RF-48           & Opzionale           & Soddisfatto                                                                                                           & Deve essere presente un sistema di notifiche che allerti l'utente se un'isola ecologica rimane al 100\% di riempimento per più di 24 ore.
	\\\hline
	RF-49           & Opzionale           & Soddisfatto                                                                                                           & Deve essere presente un sistema di notifiche che allerti l'utente se la qualità dell'aria supera l'indice 3 dell'\href{https://7last.github.io/docs/pb/documentazione-interna/glossario\#european-air-quality-index}{EAQI\textsubscript{G}}.
	\\\hline
	RF-50           & Opzionale           & Soddisfatto                                                                                                           & Deve essere presente un sistema di notifiche che allerti l'utente se la quantità di precipitazioni supera i 10mm in un'ora.
	\\\hline
	RF-51           & Opzionale           & Soddisfatto                                                                                                           & Deve essere implementato il calcolo dell'indice di qualità dell'aria \href{https://7last.github.io/docs/pb/documentazione-interna/glossario\#european-air-quality-index}{EAQI\textsubscript{G}}.
	\\\hline
	RF-52           & Opzionale           & Soddisfatto                                                                                                           & Deve essere implementato il calcolo dell'indice di temperatura percepita \href{https://7last.github.io/docs/pb/documentazione-interna/glossario\#heat-index}{Heat Index\textsubscript{G}}, combinando i dati provenienti dai sensori di temperatura e umidità.
	\\\hline
	RF-53           & Opzionale           & Soddisfatto & Devono essere combinati i dati provenienti dalle colonnine di ricarica e dai parcheggi per calcolare quanti parcheggi sono stati utilizzati da veicoli elettrici e se il parcheggio ha fruttato abbastanza per coprire i costi di installazione.
	\\\hline
	RF-54           & Obbligatorio        & Soddisfatto & Il sistema deve permettere di filtrare i dati visualizzati in base al tipo di \href{https://7last.github.io/docs/pb/documentazione-interna/glossario\#sensore}{sensore\textsubscript{G}} che li ha prodotti.
	\\\hline
	\caption{Requisiti funzionali}
\end{longtable}


\begin{figure}[!h]
	\centering
	\begin{tikzpicture}
		\def\WCtest#1#2{%
		\pgfmathparse{\WCpercentage>10?"#1":"#2"}%
		\pgfmathresult%
		}
		\wheelchart[
		anchor xsep=35,
		anchor ysep=55,
		counterclockwise,
		data=\WCtest{}{\WCperc},
		legend row={\tikz\fill[\WCvarB,draw=black] (0,0) rectangle (0.3,0.3); & \WCvarC},
		legend={
			\node[anchor=west] at (4,0) {%
			\begin{tabular}{l@{ }l}%
			\WClegend%
			\end{tabular}%
			};
		},
		perc precision=2,
		radius={0}{2.5},
		slices style={\WCvarB,draw=black},
		start angle=90,
		wheel data=\WCtest{\WCperc}{}
		]{
		100.0/orange!90/Soddisfatti,
		0.54/cyan/Non soddisfatti
		}
		\end{tikzpicture}
	\caption{Percentuale di soddisfacimento dei requisiti funzionali}
\end{figure}


\newpage 
\subsection{Requisiti qualitativi}
\begin{longtable}{|>{\centering\arraybackslash}m{0.10\textwidth}|>{\centering\arraybackslash}m{0.20\textwidth}|>{\centering\arraybackslash}m{0.20\textwidth}|>{\centering\arraybackslash}m{0.4\textwidth}|}
	\hline
	\textbf{Codice} & \textbf{Importanza} & \textbf{Stato}& \textbf{Descrizione}\\\hline
	\endfirsthead
	\hline
	\textbf{Codice} & \textbf{Importanza} & \textbf{Stato}& \textbf{Descrizione}\\\hline
	\endhead
	\hline
	RQ-55           & Obbligatorio        & Soddisfatto & Sviluppo di test che dimostrino il corretto funzionamento dei servizi e delle funzionalità previste. Viene richiesta una copertura dell'80\% corredata di report.
	\\\hline
	RQ-56           & Obbligatorio        & Soddisfatto & Il progetto deve essere corredato di documentazione riguardo scelte implementative e progettuali effettuate e relative motivazioni.
	\\\hline
	RQ-57           & Obbligatorio        & Soddisfatto & Il progetto deve essere corredato di documentazione riguardo problemi aperti e eventuali soluzioni proposte da esplorare.
	\\\hline
	RQ-58           & Obbligatorio        & Soddisfatto & Tutte le componenti del sistema devono essere testate con \href{https://7last.github.io/docs/pb/documentazione-interna/glossario\#test-end-to-end}{\textit{test end-to-end}\textsubscript{G}}.
	\\\hline
	RQ-59           & Obbligatorio        & Soddisfatto & Il sistema sarà corredato di un Manuale Utente che spieghi le funzionalità del sistema e come utilizzarle.
	\\\hline
	RQ-60           & Obbligatorio        & Soddisfatto & Il sistema sarà corredato di un documento di Specifica Tecnica che spieghi le scelte progettuali effettuate.
	\\\hline
	\caption{Requisiti qualitativi}
\end{longtable}

%--------------GRAFICO PERCENTUALE REQUISITI QUALITATIVI----------------%
\begin{figure}[!h]
	\centering
	\begin{tikzpicture}
		\def\WCtest#1#2{%
		\pgfmathparse{\WCpercentage>10?"#1":"#2"}%
		\pgfmathresult%
		}
		\wheelchart[
		anchor xsep=35,
		anchor ysep=55,
		counterclockwise,
		data=\WCtest{}{\WCperc},
		legend row={\tikz\fill[\WCvarB,draw=black] (0,0) rectangle (0.3,0.3); & \WCvarC},
		legend={
			\node[anchor=west] at (4,0) {%
			\begin{tabular}{l@{ }l}%
			\WClegend%
			\end{tabular}%
			};
		},
		perc precision=2,
		radius={0}{2.5},
		slices style={\WCvarB,draw=black},
		start angle=90,
		wheel data=\WCtest{\WCperc}{}
		]{
		100.0/orange!90/Soddisfatti
		}
		\end{tikzpicture}
	\caption{Percentuale di soddisfacimento dei requisiti qualitativi}
\end{figure}

%----------------------------------------------------------------------------------%
\newpage
\subsection{Requisiti di vincolo}
\begin{longtable}{|>{\centering\arraybackslash}m{0.10\textwidth}|>{\centering\arraybackslash}m{0.20\textwidth}|>{\centering\arraybackslash}m{0.20\textwidth}|>{\centering\arraybackslash}m{0.4\textwidth}|}
	\hline
	\textbf{Codice} & \textbf{Importanza} & \textbf{Stato}& \textbf{Descrizione}\\\hline
	\endfirsthead
	\hline
	\textbf{Codice} & \textbf{Importanza} & \textbf{Stato}& \textbf{Descrizione}\\\hline
	\endhead
	\hline
	RV-61           & Obbligatorio        & Soddisfatto & Il simulatore di dati deve pubblicare messaggi in una piattaforma di \textit{data streaming}.
	\\\hline
	RV-62           & Obbligatorio        & Soddisfatto                                                                                                           & La piattaforma di \textit{data streaming} utilizzata è \href{https://7last.github.io/docs/pb/documentazione-interna/glossario\#redpanda}{\textit{Redpanda}\textsubscript{G}}.
	\\\hline
	RV-63           & Obbligatorio        & Soddisfatto & I dati pubblicati nella piattaforma di \textit{data streaming} devono essere salvati in un database OLAP.
	\\\hline
	RV-64           & Obbligatorio        & Soddisfatto & I dati devono poter essere visualizzati dall'utente finale in delle \href{https://7last.github.io/docs/pb/documentazione-interna/glossario\#dashboard}{\textit{dashboard}\textsubscript{G}}, sviluppate con un \textit{tool} apposito, ad esempio \href{https://7last.github.io/docs/pb/documentazione-interna/glossario\#grafana}{\textit{Grafana}\textsubscript{G}}.
	\\\hline
	RV-65           & Opzionale           & Soddisfatto                                                                                                           & I dati pubblicati nei \href{https://7last.github.io/docs/pb/documentazione-interna/glossario\#topic}{\textit{topic}\textsubscript{G}} di \href{https://7last.github.io/docs/pb/documentazione-interna/glossario\#redpanda}{\textit{Redpanda}\textsubscript{G}} sono serializzati in formato \href{https://docs.confluent.io/platform/current/schema-registry/fundamentals/serdes-develop/serdes-avro.html}{\underline{Confluent Avro}}.
	\\\hline
	RV-66           & Obbligatorio        & Soddisfatto                                                                                                           & Il sistema deve essere sviluppato con \href{https://7last.github.io/docs/pb/documentazione-interna/glossario\#docker-compose}{\href{https://7last.github.io/docs/pb/documentazione-interna/glossario\#docker}{\textit{Docker}\textsubscript{G}}\textit{ Compose}\textsubscript{G}}, utilizzando la versione 3.8 della specifica.
	\\\hline
	RV-67           & Obbligatorio        & Soddisfatto & Il sistema deve poter essere usufruito dalle versioni più recenti dei browser web più diffusi. Al momento della stesura del presente documento, le versioni supportate sono: \textit{Google Chrome} v124, \textit{Safari} v17.4, \textit{Microsoft Edge} v123, \textit{Firefox} v125.
	\\\hline
	RV-68           & Obbligatorio        & Soddisfatto                                                                                                           & Il sistema deve poter funzionare su sistema operativo \textit{Linux}, con CPU a 64 bit, almeno 4GB di RAM e una delle seguenti distribuzioni e versioni minime: \textit{Ubuntu} 22.04, \textit{Debian} 12, \textit{Fedora} 38, \textit{Red Hat Enterprise Linux} 8.
	\\\hline
	RV-69           & Obbligatorio        & Soddisfatto                                                                                                           & Il sistema deve poter funzionare su sistema operativo \textit{Windows} con versione 10 o 11, CPU a 64 bit, almeno 4GB di RAM e la funzionalità WSL2 abilitata.
	\\\hline
	RV-70           & Obbligatorio        & Soddisfatto                                                                                                           & Il sistema deve poter funzionare su sistema operativo \textit{MacOs} con versione 12 o superiore, CPU \textit{Intel} o \textit{Apple Silicon} a 64bit e almeno 4GB di RAM.
	\\\hline
	\caption{Requisiti di vincolo}
\end{longtable}

%--------------GRAFICO PERCENTUALE REQUISITI DI VINCOLO----------------%
\begin{figure}[!h]
	\centering
	\begin{tikzpicture}
		\def\WCtest#1#2{%
		\pgfmathparse{\WCpercentage>10?"#1":"#2"}%
		\pgfmathresult%
		}
		\wheelchart[
		anchor xsep=35,
		anchor ysep=55,
		counterclockwise,
		data=\WCtest{}{\WCperc},
		legend row={\tikz\fill[\WCvarB,draw=black] (0,0) rectangle (0.3,0.3); & \WCvarC},
		legend={
			\node[anchor=west] at (4,0) {%
			\begin{tabular}{l@{ }l}%
			\WClegend%
			\end{tabular}%
			};
		},
		perc precision=2,
		radius={0}{2.5},
		slices style={\WCvarB,draw=black},
		start angle=90,
		wheel data=\WCtest{\WCperc}{}
		]{
		100.0/orange!90/Soddisfatti
		}
		\end{tikzpicture}
	\caption{Percentuale di soddisfacimento dei requisiti di vincolo}
\end{figure}
%----------------------------------------------------------------------------------%

\newpage
\subsection{Requisiti prestazionali}
\begin{longtable}{|>{\centering\arraybackslash}m{0.10\textwidth}|>{\centering\arraybackslash}m{0.20\textwidth}|>{\centering\arraybackslash}m{0.20\textwidth}|>{\centering\arraybackslash}m{0.4\textwidth}|}
	\hline
	\textbf{Codice} & \textbf{Importanza} & \textbf{Stato}& \textbf{Descrizione}\\\hline
	\endfirsthead
	\hline
	\textbf{Codice} & \textbf{Importanza} & \textbf{Stato}& \textbf{Descrizione}\\\hline
	\endhead
	\hline
	RP-71           & Obbligatorio        & Soddisfatto       & Il sistema deve garantire che la visualizzazione dei dati in tempo reale avvenga entro 5 secondi dalla ricezione dei dati.
	\\\hline
	\caption{Requisiti prestazionali}
\end{longtable}

%--------------GRAFICO PERCENTUALE REQUISITI PRESTAZIONALI----------------%
\begin{figure}[!h]
	\centering
	\begin{tikzpicture}
		\def\WCtest#1#2{%
		\pgfmathparse{\WCpercentage>10?"#1":"#2"}%
		\pgfmathresult%
		}
		\wheelchart[
		anchor xsep=35,
		anchor ysep=55,
		counterclockwise,
		data=\WCtest{}{\WCperc},
		legend row={\tikz\fill[\WCvarB,draw=black] (0,0) rectangle (0.3,0.3); & \WCvarC},
		legend={
			\node[anchor=west] at (4,0) {%
			\begin{tabular}{l@{ }l}%
			\WClegend%
			\end{tabular}%
			};
		},
		perc precision=2,
		radius={0}{2.5},
		slices style={\WCvarB,draw=black},
		start angle=90,
		wheel data=\WCtest{\WCperc}{}
		]{
		100.0/orange!90/Soddisfatti
		}
		\end{tikzpicture}
	\caption{Percentuale di soddisfacimento dei requisiti prestazionali}
\end{figure}
%----------------------------------------------------------------------------------%

\newpage
%--------------GRAFICO PERCENTUALE REQUISITI OBBLIGATORI----------------%
\begin{figure}[!h]
	\centering
	\begin{tikzpicture}
		\def\WCtest#1#2{%
		\pgfmathparse{\WCpercentage>10?"#1":"#2"}%
		\pgfmathresult%
		}
		\wheelchart[
		anchor xsep=35,
		anchor ysep=55,
		counterclockwise,
		data=\WCtest{}{\WCperc},
		legend row={\tikz\fill[\WCvarB,draw=black] (0,0) rectangle (0.3,0.3); & \WCvarC},
		legend={
			\node[anchor=west] at (4,0) {%
			\begin{tabular}{l@{ }l}%
			\WClegend%
			\end{tabular}%
			};
		},
		perc precision=2,
		radius={0}{2.5},
		slices style={\WCvarB,draw=black},
		start angle=90,
		wheel data=\WCtest{\WCperc}{}
		]{
		100.0/orange!90/Soddisfatti
		}
		\end{tikzpicture}
	\caption{Percentuale di soddisfacimento dei requisiti obbligatori}
\end{figure}
%----------------------------------------------------------------------------------%

%--------------GRAFICO PERCENTUALE REQUISITI DESIDERABILI----------------%
\begin{figure}[!h]
	\centering
	\begin{tikzpicture}
		\def\WCtest#1#2{%
		\pgfmathparse{\WCpercentage>10?"#1":"#2"}%
		\pgfmathresult%
		}
		\wheelchart[
		anchor xsep=35,
		anchor ysep=55,
		counterclockwise,
		data=\WCtest{}{\WCperc},
		legend row={\tikz\fill[\WCvarB,draw=black] (0,0) rectangle (0.3,0.3); & \WCvarC},
		legend={
			\node[anchor=west] at (4,0) {%
			\begin{tabular}{l@{ }l}%
			\WClegend%
			\end{tabular}%
			};
		},
		perc precision=2,
		radius={0}{2.5},
		slices style={\WCvarB,draw=black},
		start angle=90,
		wheel data=\WCtest{\WCperc}{}
		]{
		100.0/orange!90/Soddisfatti
		}
		\end{tikzpicture}
	\caption{Percentuale di soddisfacimento dei requisiti desiderabili}
\end{figure}
%----------------------------------------------------------------------------------%


%--------------GRAFICO PERCENTUALE REQUISITI OPZIONALI----------------%
\begin{figure}[!h]
	\centering
	\begin{tikzpicture}
		\def\WCtest#1#2{%
		\pgfmathparse{\WCpercentage>10?"#1":"#2"}%
		\pgfmathresult%
		}
		\wheelchart[
		anchor xsep=35,
		anchor ysep=55,
		counterclockwise,
		data=\WCtest{}{\WCperc},
		legend row={\tikz\fill[\WCvarB,draw=black] (0,0) rectangle (0.3,0.3); & \WCvarC},
		legend={
			\node[anchor=west] at (4,0) {%
			\begin{tabular}{l@{ }l}%
			\WClegend%
			\end{tabular}%
			};
		},
		perc precision=2,
		radius={0}{2.5},
		slices style={\WCvarB,draw=black},
		start angle=90,
		wheel data=\WCtest{\WCperc}{}
		]{
		89.0/orange!90/Soddisfatti,
		11.0/cyan/Non soddisfatti
		}
		\end{tikzpicture}
	\caption{Percentuale di soddisfacimento dei requisiti opzionali}
\end{figure}
%----------------------------------------------------------------------------------%

\newpage
%------------------------GRAFICO PERCENTUALE REQUISITI TOTALI----------------------%
\begin{figure}[!h]
	\centering
	\begin{tikzpicture}
		\def\WCtest#1#2{%
		\pgfmathparse{\WCpercentage>10?"#1":"#2"}%
		\pgfmathresult%
		}
		\wheelchart[
		anchor xsep=35,
		anchor ysep=55,
		counterclockwise,
		data=\WCtest{}{\WCperc},
		legend row={\tikz\fill[\WCvarB,draw=black] (0,0) rectangle (0.3,0.3); & \WCvarC},
		legend={
			\node[anchor=west] at (4,0) {%
			\begin{tabular}{l@{ }l}%
			\WClegend%
			\end{tabular}%
			};
		},
		perc precision=2,
		radius={0}{2.5},
		slices style={\WCvarB,draw=black},
		start angle=90,
		wheel data=\WCtest{\WCperc}{}
		]{
		98.6/orange!90/Soddisfatti,
		1.4/cyan/Non soddisfatti
		}
		\end{tikzpicture}
	\caption{Percentuale di soddisfacimento dei requisiti totale}
\end{figure}
%----------------------------------------------------------------------------------%

\end{document}
