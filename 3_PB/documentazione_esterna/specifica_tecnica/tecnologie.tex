\section{Tecnologie}
Questa sezione si occupa di fornire una panoramica delle tecnologie utilizzate per implementare il sistema software. In particolare delinea le piattaforme, gli strumenti, i linguaggi di programmazione, i framework e altre risorse tecnologiche che sono state impiegate durante lo sviluppo.

\subsection{Docker}
È una piattaforma di virtualizzazione leggera che semplifica lo sviluppo, il testing e il deployment delle applicazioni fornendo un ambiente isolato e riproducibile. È utilizzato per creare ambienti di sviluppo standardizzati, facilitare la scalabilità delle applicazioni e semplificare la gestione delle risorse. 
\subsubsection{Ambienti}
\subsubsubsection*{Ambiente di sviluppo locale}
\begin{itemize}
    \item Ambiente in cui gli sviluppatori lavorano sui propri computer;
    \item permette di creare e gestire i container localmente;
    \item utilizzato per testare le modifiche e sviluppare le funzionalità senza influenzare gli altri membri del gruppo.
\end{itemize}
\subsubsubsection*{Ambiente di sviluppo condiviso}
\begin{itemize}
    \item Ambiente utilizzato quando più membri del team devono collaborare su un progetto;
    \item è possibile utilizzare strumenti di orchestrazione come Docker Compose per definire un set di container che costituiscono l'applicazione in sviluppo;
    \item tutti i membri del team possono utilizzare lo stesso set di container per sviluppare e testare le loro modifiche.
\end{itemize}
\subsubsubsection*{Ambiente di test}
\begin{itemize}
    \item Questo ambiente viene utilizzato per testare l'applicazione in un ambiente simile a quello di produzione prima di rilasciarla;
    \item può includere una replica dell'infrastruttura di produzione, con configurazioni simili e dati di test;
    \item gli ambienti di test possono essere automaticamente aggiornati con le ultime versioni dell'applicazione dai rami di sviluppo tramite integrazione continua (CI) o integrazione continua/produzione (CI/CD).
\end{itemize}
\subsubsubsection*{Ambiente di produzione}
\begin{itemize}
    \item Questo è l'ambiente in cui viene eseguita effettivamente l'applicazione per gli utenti finali;
    \item gli ambienti di produzione sono solitamente più rigorosamente controllati e configurati rispetto agli ambienti di sviluppo e di test;
    \item le applicazioni vengono distribuite qui dopo essere state testate e validate in ambienti di test.
\end{itemize}
\subsubsection{Docker images}
Nello sviluppo di questo progetto \textit{7Last} ha utilizzato diverse immagini Docker di seguito elencate.
\begin{itemize}
    \item \textbf{Simulatore - Python}
    \begin{itemize}
        \item Immagine: 3.11.9-alpine;
        \item Riferimento: \href{https://hub.docker.com/_/python}{Python Official Docker Image} (Consultato 2024-06-02).
        \item Ambiente: 
        \begin{itemize}
            \item sviluppo;
            \item produzione.
        \end{itemize}
    \end{itemize}
    \item \textbf{Broker - Redpanda}
    \begin{itemize}
        \item Immagine: docker.redpanda.com/redpandadata/redpanda:v23.3.11;
        \item Riferimento: \href{https://hub.docker.com/r/redpandadata/redpanda}{Redpanda Official Docker Image} (Consultato 2024-06-02).
        \item Ambiente: 
        \begin{itemize}
            \item sviluppo;
            \item produzione;
            \item test.
        \end{itemize}
    \end{itemize}
    \item \textbf{Redpanda console}
    \begin{itemize}
        \item Immagine: docker.redpanda.com/redpandadata/console:v2.4.6;
        \item Riferimento: \href{https://hub.docker.com/r/redpandadata/redpanda}{Redpanda Official Docker Image}(Consultato 2024-06-02).
        \item Ambiente: 
        \begin{itemize}
            \item sviluppo;
            \item produzione;
            \item test.
        \end{itemize}
    \end{itemize}
    \item \textbf{Database - ClickHouse}
    \begin{itemize}
        \item Immagine: clickhouse/clickhouse-server:24-alpine;
        \item Riferimento: \href{https://hub.docker.com/r/clickhouse/clickhouse-server}{ClickHouse Official Docker Image} (Consultato 2024-06-02).
        \item Ambiente: 
        \begin{itemize}
            \item sviluppo;
            \item produzione;
            \item test.
        \end{itemize}
    \end{itemize}
    \item \textbf{Grafana}
    \begin{itemize}
        \item Immagine: grafana/grafana-oss:10.3.0;
        \item Riferimento: \href{https://hub.docker.com/r/grafana/grafana-oss}{Grafana Official Docker Image} (Consultato 2024-06-02).
        \item Ambiente: 
        \begin{itemize}
            \item sviluppo;
            \item produzione.
        \end{itemize}
    \end{itemize}
\end{itemize}
\subsection{Linguaggi e formato dati}
\subsubsection{Python}
Python è un linguaggio di programmazione versatile, noto per la sua semplicità, leggibilità e vasta gamma di applicazioni.
\subsubsubsection{Versione}
\begin{itemize}
    \item Versione utilizzata: 3.11.9
\end{itemize}
\subsubsubsection{Documentazione}
\begin{center}
    \url{https://docs.python.org/3.11/} (Consultato 2024-06-02).
\end{center}
\subsubsubsection{Utilizzo nel progetto}
\begin{itemize}
    \item Sviluppo dei simulatori di sensori;
    \item sviluppo di moduli di elaborazione dei dati;
    \item automazione e testing.
\end{itemize}
\subsubsubsection{Librerie}
\begin{itemize}
    \item \textbf{Astroid}:
    \begin{itemize}
        \item Versione: 3.1.0;
        \item Documentazione: \url{https://pypi.org/project/astroid/} (Consultato 2024-06-02);
        \item Descrizione: utilizzata per rappresentare e manipolare l'Abstract Syntax Tree (AST) del codice Python. È ampiamente utilizzata da strumenti di analisi del codice come pylint per effettuare analisi statiche del codice Python. Astroid fornisce una rappresentazione astratta del codice sorgente che permette di esplorare e navigare la struttura del codice, compresi moduli, classi, funzioni e variabili, in modo dettagliato e programmabile.
    \end{itemize}
    \item \textbf{Avro}:
    \begin{itemize}
        \item Versione: 1.11.3;
        \item Documentazione: \url{https://avro.apache.org/docs/1.11.1/getting-started-python/} (Consultato 2024-06-02);
        \item Descrizione: è progettata per essere utilizzata da applicazioni che necessitano di una rappresentazione compatta, veloce e binaria dei dati, con un robusto supporto per l'evoluzione degli schemi dei dati. Avro è particolarmente utilizzata in ambienti di big data e di elaborazione distribuita.
    \end{itemize}
    \item \textbf{Certifi}:
    \begin{itemize}
        \item Versione: 2024.2.2;
        \item Documentazione: \url{https://certifi.io/} (Consultato 2024-06-02);
        \item Descrizione: è una libreria Python che fornisce una versione aggiornata del pacchetto di certificati CA (Certificate Authority) per la verifica dei certificati SSL/TLS.
    \end{itemize}
    \item \textbf{Charset-normalizer}:
    \begin{itemize}
        \item Versione: 3.3.2;
        \item Documentazione: \url{https://charset-normalizer.readthedocs.io/en/stable/} (Consultato 2024-06-02);
        \item Descrizione: è progettata per rilevare e normalizzare automaticamente la codifica del testo. È particolarmente utile quando si lavora con dati di origine sconosciuta o con file di testo che potrebbero avere diverse codifiche.
    \end{itemize}
    \item \textbf{Dill}:
    \begin{itemize}
        \item Versione: 0.3.8;
        \item Documentazione: \url{https://dill.readthedocs.io/en/latest/} (Consultato 2024-06-02);
        \item Descrizione: estende le funzionalità del modulo standard pickle, permettendo di serializzare e deserializzare una gamma più ampia di oggetti.
    \end{itemize}
    \item \textbf{Idna}:
    \begin{itemize}
        \item Versione: 3.7;
        \item Documentazione: \url{https://pypi.org/project/idna/} (Consultato 2024-06-02);
        \item Descrizione: implementa l'Internationalized Domain Names in Applications (IDNA) standard. Questo standard consente l'uso di caratteri Unicode nei nomi di dominio, permettendo l'inclusione di caratteri non-ASCII come quelli utilizzati in molte lingue del mondo.
    \end{itemize}
    \item \textbf{Isolate}:
    \begin{itemize}
        \item Versione: 0.6.1;
        \item Documentazione: \url{https://docs.python.org/3/howto/isolating-extensions.html} (Consultato 2024-06-02);
        \item Descrizione: progettata per eseguire codice Python in un ambiente isolato, separato dal contesto globale dell'applicazione.
    \end{itemize}
    \item \textbf{Isort}:
    \begin{itemize}
        \item Versione: 5.13.2;
        \item Documentazione: \url{https://pycqa.github.io/isort/} (Consultato 2024-06-02);
        \item Descrizione: utilizzata per ordinare automaticamente le dichiarazioni di import in un file Python in base a determinati criteri. Mantenere un ordine coerente e ben strutturato delle importazioni può rendere il codice più leggibile e facilitare la manutenzione. 
    \end{itemize}
    \item \textbf{Kafka-python}:
    \begin{itemize}
        \item Versione: 2.2.2;
        \item Documentazione: \url{https://kafka-python.readthedocs.io/en/master/} (Consultato 2024-06-02);
        \item Descrizione: libreria Python ampiamente utilizzata per interagire con Apache Kafka, una piattaforma di streaming distribuita. 
    \end{itemize}
    \item \textbf{McCabe}:
    \begin{itemize}
        \item Versione: 0.7.0;
        \item Documentazione: \url{https://here-be-pythons.readthedocs.io/en/latest/python/mccabe.html} (Consultato 2024-06-02);
        \item Descrizione: fornisce strumenti per misurare la complessità ciclomatica del codice sorgente Python.
    \end{itemize}
    \item \textbf{Pip-autoremove}:
    \begin{itemize}
        \item Versione: 0.10.0;
        \item Documentazione: \url{https://pypi.org/project/pip-autoremove/} (Consultato 2024-06-02);
        \item Descrizione: è uno strumento aggiuntivo per pip che consente di rimuovere automaticamente i pacchetti Python non utilizzati da un ambiente virtuale.
    \end{itemize}
    \item \textbf{Platformdirs}:
    \begin{itemize}
        \item Versione: 4.2.0;
        \item Documentazione: \url{https://platformdirs.readthedocs.io/en/latest/} (Consultato 2024-06-02);
        \item Descrizione: fornisce un'interfaccia semplice e multi-piattaforma per ottenere i percorsi di directory standard per l'archiviazione di dati specifici dell'applicazione su diverse piattaforme.
    \end{itemize}
    \item \textbf{Python-dotenv}:
    \begin{itemize}
        \item Versione: 1.0.1;
        \item Documentazione: \url{https://pypi.org/project/python-dotenv/} (Consultato 2024-06-02);
        \item Descrizione: carica le variabili d'ambiente da file .env nel sistema operativo. Questo è particolarmente utile durante lo sviluppo per mantenere segrete e configurabili le variabili d'ambiente utilizzate nel progetto.
    \end{itemize}
    \item \textbf{Requests}:
    \begin{itemize}
        \item Versione: 2.31.0;
        \item Documentazione: \url{https://requests.readthedocs.io/en/latest/} (Consultato 2024-06-02);
        \item Descrizione: semplifica notevolmente l'invio di richieste HTTP e la gestione delle risposte, fornendo un'API semplice e intuitiva per interagire con servizi web.
    \end{itemize}
    \item \textbf{Ruff}:
    \begin{itemize}
        \item Versione: 0.3.5;
        \item Documentazione: \url{https://docs.astral.sh/ruff/} (Consultato 2024-06-02);
        \item Descrizione: impiegato per l'analisi statica del codice sorgente.
    \end{itemize}
    \item \textbf{Setuptools}:
    \begin{itemize}
        \item Versione: 69.5.1;
        \item Documentazione: \url{https://setuptools.pypa.io/en/latest/} (Consultato 2024-06-02);
        \item Descrizione: fornisce strumenti per la creazione, la distribuzione e l'installazione di pacchetti Python.
    \end{itemize}
    \item \textbf{Six}:
    \begin{itemize}
        \item Versione: 1.16.0;
        \item Documentazione: \url{https://six.readthedocs.io/} (Consultato 2024-06-02);
        \item Descrizione: progettata per semplificare la scrittura di codice Python che deve essere compatibile con entrambe le versioni di Python 2 e 3.
    \end{itemize}
    \item \textbf{Toml}:
    \begin{itemize}
        \item Versione: 0.10.2;
        \item Documentazione: \url{https://pypi.org/project/toml/} (Consultato 2024-06-02);
        \item Descrizione: utilizzata per la lettura e la scrittura di file di configurazione nel formato TOML.
    \end{itemize}
    \item \textbf{Urllib3}:
    \begin{itemize}
        \item Versione: 2.2.1;
        \item Documentazione: \url{https://urllib3.readthedocs.io/en/2.2.1/} (Consultato 2024-06-02);
        \item Descrizione: è una libreria Python che fornisce funzionalità avanzate per effettuare richieste HTTP in modo sicuro e affidabile.
    \end{itemize}
\end{itemize}
\subsubsection{JSON}
JSON (JavaScript Object Notation) è un formato di scambio dati leggero e facile da leggere e scrivere sia per gli esseri umani che per le macchine. È ampiamente utilizzato per trasmettere dati tra un server e un client, in particolare nelle applicazioni web. JSON è basato su un sottoinsieme del linguaggio di programmazione JavaScript, ma è indipendente dal linguaggio, il che significa che può essere utilizzato con quasi tutti i linguaggi di programmazione. Questo linguaggio è composto da due strutture fondamentali, \textbf{oggetti} e \textbf{array}.
\begin{itemize}
    \item \textbf{Oggetti}:
	\begin{itemize}
		\item racchiusi tra parentesi graffe;
		\item contengono coppie chiave-valore, dove le chiavi sono le stringhe e i valori possono essere di vari tipi;
		\item le coppia chiave-valore sono separate dal simbolo ":" e ogni coppia è separata dal simbolo ",".
	\end{itemize}
   \item \textbf{Array}:
	\begin{itemize}
		\item racchiusi tra parentesi quadre;
		\item contengono una lista ordinata di valori separati da virgole.
	\end{itemize}
\end{itemize}
\subsubsubsection{Impiego nel progetto}
\begin{itemize}
	\item Configurazione dashboard Grafana;
	\item struttura dei dati inviati dai simulatori dei sensori al broker Kafka.
\end{itemize}

\subsubsection{YAML}
YAML (YAML Ain't Markup Language) è un linguaggio di serializzazione dei dati, concepito per essere leggibile sia per gli esseri umani sia per le macchine. YAML viene spesso utilizzato per configurare file e scambiare dati tra applicazioni. 
\subsubsubsection{Impiego nel progetto}
\begin{itemize}
	\item Configurazione Docker Compose per l'avvio del progetto;
	\item configurazione provisioning Grafana.
\end{itemize}
\subsubsection{SQL}
SQL (Structured Query Language) è un linguaggio di programmazione specificamente progettato per la gestione e la manipolazione di dati all'interno di sistemi di gestione di database.
\subsubsubsection{Impiego nel progetto}
\begin{itemize}
        \item Configurazione e gestione database ClickHouse.
\end{itemize}
\subsubsection{Toml} 
TOML (Tom's Obvious, Minimal Language) è un linguaggio di serializzazione configurabile utilizzato principalmente per i file di configurazione.
\subsubsubsection{Impiego nel progetto}
\begin{itemize}
    \item Configurazione e gestione dei sensori simulati.
\end{itemize}

\subsubsection{Java} 
È un linguaggio di programmazione ad alto livello, orientato agli oggetti e progettato per avere il minor numero possibile di dipendenze di implementazione.
\subsubsubsection{Versione}
\begin{itemize}
    \item Versione utilizzata: 21
\end{itemize}
\subsubsubsection{Documentazione}
\begin{center}
    \url{https://docs.oracle.com/en/java/} (Consultato 2024-06-02).
\end{center}
\subsubsubsection{Impiego nel progetto}
\begin{itemize}
    \item Creazione di job per le aggregazioni dei dati di Flink.
\end{itemize}

\subsection{Database e servizi}
\subsubsection{Redpanda}
Redpanda è una piattaforma di streaming sviluppata in C++. Il suo obiettivo è fornire una soluzione leggera, semplice e performante, pensata per essere un'alternativa ad Apache Kafka. Viene utilizzato per disaccoppiare i dati provenienti dal simulatore. Ciascun tipo di dato viene inviato su un topic specifico, in modo da poter essere elaborato in modo indipendente.
\subsubsubsection{Versione}
La versione impiegata al momento dello sviluppo del progetto è la seguente: v23.3.11.
\subsubsubsection{Documentazione}
\url{https://docs.redpanda.com/current/home/} (Consultato 2024-06-02).
\subsubsubsection{Vantaggi}
I vantaggi nell'utilizzo di questo strumento consistono in:
\begin{itemize}
	\item \textbf{performance}: è scritto in C++ e utilizza il \textit{framework} Seastar, offrendo un'architettura \textit{thread-per-core} ad alte prestazioni.
    Ciò permette di ottenere un'elevata \textit{throughput} e latenze costantemente basse, evitando cambi di contesto e blocchi.
    Inoltre, è progettato per sfruttare l'\textit{hardware} moderno, tra cui unità NVMe, processori \textit{multi-core} e interfacce di rete ad alta velocità;
    \item \textbf{costi}: rispetto ad altre tecnologie il carico di lavoro può essere ridotto fino a 5 volte in meno;
    \item \textbf{semplicità di configurazione}: oltre al \textit{message broker}, contiene anche un \textit{proxy} HTTP e uno \textit{schema registry};
    \item \textbf{BYOC (Bring Your Own Cluster)}: consente agli utenti finali di implementare una soluzione parzialmente gestita dal fornitore nella propria infrastruttura (come il proprio \textit{data center} o il proprio \textit{VPC cloud});
    \item \textbf{compatibilità con le API di Kafka}: è compatibile con le API di Apache Kafka, consentendo di utilizzare le librerie e gli strumenti esistenti;
    \item \textbf{self-healing}: redistribuisce continuamente i dati e la \textit{leadership} tra i nodi per mantenere il \textit{cluster} in uno stato ottimale mentre evolve o quando i nodi falliscono.
\end{itemize}
\subsubsubsection{Casi d'Uso}
Esistono molteplici casi d'uso associati all'uso di \textit{Redpanda}, tra cui:
\begin{itemize}
	\item \textbf{streaming di eventi}, permettendo la gestione e l'elaborazione di flussi di dati in tempo reale; 
	\item \textbf{data integration}, agisce come un intermediario flessibile e robusto per l'integrazione dei dati, consentendo la raccolta, il trasporto e la trasformazione dei dati provenienti da diverse sorgenti verso varie destinazioni;
	\item \textbf{elaborazione di big data}, permette di gestire e processare enormi volumi di dati in modo efficiente e scalabile;
	\item \textbf{messaggistica real time}, supporta la messaggistica in tempo reale tra applicazioni e sistemi distribuiti.
\end{itemize}
\subsubsubsection{Impiego nel progetto}
\begin{itemize}
	\item \textbf{Broker}: riceve i dati prodotti dal simulatore e li rende disponibili ai consumatori. 
\end{itemize}
All'interno di questo progetto i dati vengono trasmessi in formato JSON.
In questo momento il consumatore è rappresentato dal database \textit{CLickHouse}, il quale salva i dati resi disponibili da \textit{Redpanda}. 


\subsubsection{ClickHouse}
ClickHouse è un sistema di gestione di database colonnare open-source progettato per l'analisi dei dati in tempo reale e l'elaborazione di grandi volumi di dati.
\subsubsubsection{Versione}
La versione impiegata al momento dello sviluppo del progetto è la seguente: 24.3.2.23.
\subsubsubsection{Documentazione}
\url{https://clickhouse.com/docs/en/intro} (Consultato 2024-06-02).
\subsubsubsection{Vantaggi}
I vantaggi nell'utilizzo di questo strumento consistono in:
\begin{itemize}
    \item \textbf{alte prestazioni}, è progettato per eseguire query analitiche complesse in modo estremamente rapido;
	\item \textbf{scalabilità orizzontale}, può essere scalato orizzontalmente su più nodi, permettendo di gestire petabyte di dati;
	\item \textbf{elaborazione in tempo reale}, è in grado di gestire l'ingestione e l'elaborazione dei dati in tempo reale, rendendolo ideale per applicazioni che richiedono l'analisi immediata dei dati appena arrivano;
	\item \textbf{replica ad alta disponibilità}, supporta la replica dei dati tra diversi nodi, offrendo tolleranza ai guasti e alta disponibilità;
	\item \textbf{compressione efficiente}, utilizza algoritmi di compressione avanzati per ridurre lo spazio di archiviazione e migliorare l'efficienza I/O;
	\item \textbf{supporto SQL avanzato}, supporta un dialetto SQL ricco di funzionalità, permettendo agli utenti di eseguire query complesse e di sfruttare funzioni avanzate per l'analisi dei dati;
	\item \textbf{facilità di integrazione}, si integra facilmente con molti strumenti di visualizzazione dei dati e piattaforme di business intelligence come Grafana;
	\item \textbf{partizionamento e indici}, supporta il partizionamento dei dati e l'uso di indici per ottimizzare le query;
	\item \textbf{costo efficacia}, essendo open source non ha costi di licenza, il che lo rende una soluzione economica.
\end{itemize}
\subsubsubsection{Casi d'Uso}
Esistono molteplici casi d'uso associati all'uso di \textit{Redpanda}, tra cui:
\begin{itemize}
        \item \textbf{analisi dei log e monitoraggio}, utilizzato per l'analisi e il monitoraggio dei log in tempo reale;
        \item \textbf{business intelligence}, impiegato in applicazioni di BI per eseguire analisi approfondite dei dati aziendali, supportando la presa di decisioni basata sui dati;
        \item \textbf{data warehousing}, funziona come data warehouse per memorizzare e analizzare grandi volumi di dati.
\end{itemize}
\subsubsubsection{Impiego nel progetto}
\begin{itemize}
    \item \textbf{Organizzazione efficiente dei dati}: il tipo di architettura di \textit{ClickHouse} permette di comprimere i dati in modo più efficace e di leggere solo le colonne necessarie durante l'esecuzione delle query. 
    \item \textbf{Integrazione con Redpanda}: può essere utilizzato in sinergia con Redpanda, una piattaforma di streaming dati compatibile con Apache Kafka. Questa integrazione permette di ingestire, elaborare e analizzare flussi di dati in tempo reale. 
	\item \textbf{Aggregazione rapida dei dati}: è progettato per eseguire aggregazioni di dati in modo estremamente veloce. Grazie alle sue capacità di elaborazione colonnare, alle tecniche avanzate di compressione e all'uso di indici, ClickHouse può eseguire calcoli aggregati su grandi dataset in tempi molto ridotti. 
	\item \textbf{Integrazione con Grafana}: si integra facilmente con Grafana, una delle piattaforme di visualizzazione dei dati più popolari. Questa integrazione permette di creare dashboard interattivi e personalizzati che visualizzano i dati in tempo reale provenienti da ClickHouse. 
\end{itemize}

\subsubsection{Grafana}
È una potente piattaforma di visualizzazione dei dati progettata per creare, esplorare e condividere dashboard interattive che visualizzano metriche, log e altri dati di monitoraggio in tempo reale.
\subsubsubsection{Versione}
La versione impiegata al momento dello sviluppo del progetto è la seguente: 10.3.0.
\subsubsubsection{Documentazione}
\url{https://grafana.com/docs/grafana/v10.4/} (Consultato 2024-06-02).

\subsubsubsection{Vantaggi}
\begin{itemize}
	\item \textbf{Facilità d'uso}: possiede un'interfaccia intuitiva che rende facile la creazione e la gestione delle dashboard;
	\item \textbf{flessibilità}: La capacità di integrarsi con molteplici sorgenti dati e l'ampia gamma di plugin disponibili la rendono estremamente flessibile;
	\item \textbf{personalizzazione}: permette una personalizzazione completa delle dashboard, soddisfando ogni possibile necessità di visualizzazione dei dati;
	\item \textbf{gestione degli accessi}: offre funzionalità avanzate di gestione degli accessi e delle autorizzazioni, consentendo di controllare chi può accedere alle dashboard e quali azioni possono eseguire.
\end{itemize}
\subsubsubsection{Casi d'Uso}
\begin{itemize}
    \item \textbf{Monitoraggio delle infrastrutture}: utilizzato per monitorare le prestazioni e la disponibilità delle infrastrutture IT, inclusi server, database, servizi cloud e altro;
    \item \textbf{analisi delle performance delle applicazioni}: utilizzato per monitorare le prestazioni delle applicazioni e identificare eventuali problemi di prestazioni;
    \item \textbf{analisi delle serie temporali}: utilizzato per visualizzare e analizzare dati di serie temporali, come metriche di monitoraggio, log e dati di sensori;
    \item \textbf{business intelligence}: utilizzato per creare dashboard personalizzate per l'analisi dei dati aziendali e la visualizzazione delle metriche chiave.
\end{itemize}
\subsubsubsection{Impiego nel progetto}
\begin{itemize}
    \item \textbf{Visualizzazione dei dati}: utilizzato per creare dashboard interattive che visualizzano i dati provenienti da ClickHouse;
    \item \textbf{analisi dei dati}: utilizzato per analizzare i dati e identificare tendenze, pattern e anomalie;
    \item \textbf{monitoraggio degli allarmi}: utilizzato per monitorare e visualizzare gli allarmi generati dai simulatori e dagli altri componenti del sistema;
    \item \textbf{notifiche}: utilizzato per inviare notifiche in caso di superamento di soglie critiche.
\end{itemize}