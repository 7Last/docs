\section{Tecnologie}
Questa sezione si occupa di fornire una panoramica delle tecnologie utilizzate per implementare il sistema software.
In particolare, delinea le piattaforme, gli strumenti, i linguaggi di programmazione, i framework e altre risorse tecnologiche che sono state impiegate durante lo sviluppo.

\subsection{Docker}
È una piattaforma di virtualizzazione leggera che semplifica lo sviluppo, il testing e il rilascio delle applicazioni fornendo un ambiente isolato e riproducibile.
È utilizzato per creare ambienti di sviluppo standardizzati, facilitare la scalabilità delle applicazioni e semplificare la gestione delle risorse.

\subsubsection{Ambienti}
Per lo sviluppo di questo progetto sono stati ipotizzati i due seguenti scenari di esecuzione, separati grazie all'utilizzo di profili diversi di Docker Compose:
\begin{itemize}
	\item \texttt{local}: utilizzato dagli sviluppatori per testare e sviluppare le funzionalità dell'applicazione sui propri computer.
	      Questo ambiente permette di eseguire tutti i componenti del sistema all'interno di un container Docker, ad eccezione del simulatore Python.
	      Esso viene eseguito direttamente sul sistema operativo dell'utente, in modo da facilitare il debugging e il testing delle funzionalità,
	      senza dover necessariamente eseguire la \textit{build} dell'immagine Docker ad ogni modifica del codice;
	\item \texttt{release}: utilizzato quando si desidera simulare un ipotetico ambiente di produzione o non è necessario modificare il codice Python. Consente di non dover manualmente
	      installare le dipendenze o configurare l'ambiente di esecuzione. In questo caso, tutti i componenti del sistema vengono eseguiti all'interno di container Docker.

\end{itemize}

\subsubsection{Immagini Docker}
Nello sviluppo di questo progetto abbiamo utilizzato diverse immagini Docker, di seguito elencate.
Le porte sono indicate nel formato \texttt{<porta\_host>:<porta\_container>}, dove \texttt{<porta\_host>} è la porta del sistema host a cui si mappa la porta del container \texttt{<porta\_container>}.
\begin{itemize}
	\item \textbf{Simulator - Python}
	      \begin{itemize}
		      \item \textbf{Immagine}: python:3.11.9-alpine;
		      \item \textbf{Riferimento}: \underline{\href{https://hub.docker.com/_/python}{Python Docker Image}} [Ultima consultazione: 2024-06-02].
		      \item \textbf{Ambiente}: \texttt{release};
	      \end{itemize}

	\item \textbf{Redpanda \mySkip{Init}}:
	      l'immagine di \texttt{alpine} viene utilizzata per creare un container che si occupa di inizializzare il broker Redpanda.
	      \begin{itemize}
		      \item \textbf{Immagine}: alpine:3.20.1;
		      \item \textbf{Riferimento}: \underline{\href{https://hub.docker.com/_/alpine}{Alpine}} [Ultima consultazione: 2024-06-25].
		      \item \textbf{Ambiente}: \texttt{local}, \texttt{release}.
	      \end{itemize}

	\item \textbf{Redpanda}
	      \begin{itemize}
		      \item \textbf{Immagine}: docker.redpanda.com/redpandadata/redpanda:v23.3.11;
		      \item \textbf{Riferimento}: \underline{\href{https://hub.docker.com/r/redpandadata/redpanda}{Redpanda Docker Image}} [Ultima consultazione: 2024-06-02].
		      \item \textbf{Ambiente}: \texttt{local}, \texttt{release};
		      \item \textbf{Porte}:
		            \begin{itemize}
			            \item 18081:18081: \textit{schema registry};
			            \item 18082:18082: \textit{proxy};
			            \item 19092:19092: \textit{broker}.
		            \end{itemize}
	      \end{itemize}

	\item \textbf{Redpanda console}
	      \begin{itemize}
		      \item \textbf{Immagine}: docker.redpanda.com/redpandadata/console:v2.4.6;
		      \item \textbf{Riferimento}: \underline{\href{https://hub.docker.com/r/redpandadata/redpanda}{Redpanda Console Docker Image}} [Ultima consultazione: 2024-06-02].
		      \item \textbf{Ambiente}: \texttt{local}, \texttt{release};
		      \item \textbf{Porte}:
		            \begin{itemize}
			            \item 8080:8080.
		            \end{itemize}
	      \end{itemize}

	\item \textbf{Connectors}
	      \begin{itemize}
		      \item \textbf{Immagine}: docker.redpanda.com/redpandadata/connectors:v1.0.27;
		      \item \textbf{Riferimento}: \underline{\href{https://hub.docker.com/r/redpandadata/connectors}{Redpanda Connectors Docker Image}} [Ultima consultazione: 2024-06-02].
		      \item \textbf{Ambiente}: \texttt{local}, \texttt{release};
		      \item \textbf{Porte}:
		            \begin{itemize}
			            \item 8083:8083.
		            \end{itemize}
	      \end{itemize}

	\item \textbf{ClickHouse}
	      \begin{itemize}
		      \item \textbf{Immagine}: clickhouse/clickhouse-server:24-alpine;
		      \item \textbf{Riferimento}: \underline{\href{https://hub.docker.com/r/clickhouse/clickhouse-server}{ClickHouse Docker Image}} [Ultima consultazione: 2024-06-02].
		      \item \textbf{Ambiente}: \texttt{local}, \texttt{release}.
		      \item \textbf{Porte}:
		            \begin{itemize}
			            \item 8123:8123.
		            \end{itemize}
	      \end{itemize}

	\item \textbf{Apache Flink}
	      \begin{itemize}
		      \item \textbf{Immagine}: flink:1.18.1-java17;
		      \item \textbf{Riferimento}: \underline{\href{https://hub.docker.com/_/flink}{Flink Docker Image}} [Ultima consultazione: 2024-06-02].
		      \item \textbf{Ambiente}: \texttt{local}, \texttt{release}.
		      \item \textbf{Porte}:
		            \begin{itemize}
			            \item servizio \texttt{jobmanager}: 9001:8081 interfaccia \textit{web};
			            \item servizio \texttt{taskmanager}: 6121:6121.
		            \end{itemize}
	      \end{itemize}

	\item \textbf{Grafana}
	      \begin{itemize}
		      \item \textbf{Immagine}: grafana/grafana-oss:10.3.0;
		      \item \textbf{Riferimento}: \underline{\href{https://hub.docker.com/r/grafana/grafana-oss}{Grafana Docker Image}} [Ultima consultazione: 2024-06-02].
		      \item \textbf{Ambiente}: \texttt{local}, \texttt{release}.
		      \item \textbf{Porte}:
		            \begin{itemize}
			            \item 3000:3000.
		            \end{itemize}
	      \end{itemize}
\end{itemize}

\subsection{Linguaggi e formato dati}
\begin{longtable}{|>{\centering\arraybackslash}m{0.10\textwidth}|>{\centering\arraybackslash}m{0.10\textwidth}|>{\centering\arraybackslash}m{0.35\textwidth}|>{\centering\arraybackslash}m{0.35\textwidth}|}
	\hline
	\textbf{Nome} & \textbf{Versione} & \textbf{Descrizione}                                                                                                                                 & \textbf{Impiego}                                                                                                             \\\hline
	\endfirsthead
	\hline
	\textbf{Nome} & \textbf{Versione} & \textbf{Descrizione}                                                                                                                                 & \textbf{Impiego}                                                                                                             \\\hline
	\endhead
	Python        & 3.11.9            & Linguaggio di programmazione ad alto livello, interpretato e multiparadigma.                                                                         & Simulatore di sensori, \textit{testing}, \textit{script} per automatizzare il \textit{deployment} dei \textit{job} di Flink. \\\hline
	JSON          & -                 & Formato di dati semplice da interpretare e generare, ampiamente utilizzato per lo scambio di dati tra applicazioni.                                  & Configurazione \textit{dashboard} Grafana, definizione di schemi Avro.                                                       \\\hline
	YAML          & -                 & Linguaggio di serializzazione dei dati leggibile sia per gli esseri umani sia per le macchine.                                                       & Docker Compose, provisioning Grafana e configurazione \textit{alert}, file di \textit{workflow} per le GitHub Actions.       \\\hline
	SQL           & Ansi SQL          & Linguaggio di programmazione specificamente progettato per la gestione e la manipolazione di dati all'interno di sistemi di gestione di database.    & \textit{Query} e gestione database ClickHouse.                                                                               \\\hline
	TOML          & 1.0.0             & Linguaggio di \textit{markup} progettato per essere più leggibile e facile da scrivere rispetto ad altri formati di configurazione come JSON e YAML. & Configurazione e gestione dei sensori simulati.                                                                              \\\hline
	Java          & 17                & Linguaggio di programmazione ad alto livello, orientato agli oggetti.                                                                                & Creazione di \textit{job} per le aggregazioni dei dati in Apache Flink.                                                      \\\hline
	\caption{Linguaggi e formato dati}
\end{longtable}

\subsection{Librerie}
\begin{longtable}{|>{\centering\arraybackslash}m{0.35\textwidth}|>{\centering\arraybackslash}m{0.10\textwidth}|>{\centering\arraybackslash}m{0.45\textwidth}|}
	\hline
	\multicolumn{3}{|c|}{\textbf{Python}}                                                                                                          \\
	\hline
	\textbf{Nome}                          & \textbf{Versione} & \textbf{Impiego}                                                                  \\\hline
	\endfirsthead
	\hline
	\textbf{Nome}                          & \textbf{Versione} & \textbf{Impiego}                                                                  \\
	\endhead
	% \texttt{avro} & 1.11.3            & Libreria per la serializzazione dei dati in formato Avro. & Serializzazione dei dati in formato Avro. \\\hline
	\texttt{confluent\_avro}               & 1.8.0             & Serializzazione dei dati in formato Confluent Avro.                                         \\\hline
	\texttt{coverage}                      & 7.5.1             & Strumento per misurare la percentuale di linee di codice e rami coperti dai test. \\\hline
	\texttt{\mySkip{isodate}}                       & 0.6.1             & Libreria per la manipolazione delle date e delle ore in formato ISO8601.          \\\hline
	\texttt{kafka-python-\mySkip{ng}}               & 2.2.2             & Client Kafka per Python.                                                          \\\hline
	% \texttt{requests}          & 2.32.3            & Libreria per effettuare richieste HTTP.                                           \\\hline
	\texttt{ruff}                          & 0.3.5             & Libreria per l'analisi statica del codice.                                        \\\hline
	\texttt{toml}                          & 0.10.2            & Libreria per effettuare il parsing dei file di configurazione in formato TOML.    \\\hline

	\multicolumn{3}{|c|}{\textbf{Java}}                                                                                                            \\\hline
	% TODO: controllare quali effettivamente necessarie/utilizzate
	\texttt{flink-streaming-java}          & 1.18.0            & Utilizzo di DataStream API di Flink.                                              \\\hline
	\texttt{flink-connector-kafka}         & 3.1.0-1.18        & Connessione di Flink a Kafka.                                                     \\\hline
	\texttt{flink-clients}                 & 1.18.0            & Creazione di \textit{job} di Flink.                                               \\\hline
	\texttt{flink-java}                    & 1.18.0            & Creazione di \textit{job} di Flink.                                               \\\hline
	\texttt{flink-avro-confluent-registry} & 1.18.0            & Connessione di Flink a uno \textit{schema registry} che utilizza Avro.            \\\hline
	% \texttt{flink-avro}            & 1.18.0            & Libreria per la connessione di Flink ad Avro.                                      \\\hline
	\texttt{flink-shaded-guava}            & 31.1-jre-17.0     & Gestione delle dipendenze di Flink.                                               \\\hline
	\texttt{\mySkip{slf}4j-simple}                  & 1.7.36            & Implementazione di \mySkip{SLF4J}.                                                         \\\hline
	\texttt{lombok}                        & 1.18.32           & Libreria per la generazione di codice \textit{boilerplate}.                       \\\hline
	\texttt{maven-assembly-plugin}         & 3.7.1             & Plugin Maven per la creazione di un \textit{fat jar}.                             \\\hline
	\caption{Librerie utilizzate}
\end{longtable}

\subsection{Servizi}
\subsubsection{Redpanda}
Redpanda è una piattaforma di streaming sviluppata in C++. Il suo obiettivo è fornire una soluzione leggera, semplice e performante, pensata per essere un'alternativa ad Apache Kafka. Viene utilizzato per disaccoppiare i dati provenienti dal simulatore.

\begin{itemize}
	\item \textbf{Versione}: v23.3.11;
	\item \textbf{documentazione}: \url{https://docs.redpanda.com/current/home/} [Ultima consultazione: 2024-06-02].
\end{itemize}

\subsubsubsection{Vantaggi}
I vantaggi nell'utilizzo di questo strumento consistono in:
\begin{itemize}
	\item \textbf{\textit{performance}}: è scritto in C++ e utilizza il \textit{framework} Seastar, offrendo un'architettura \textit{thread-per-core} ad alte prestazioni.
	      Ciò permette di ottenere un'elevata \textit{throughput} e latenze costantemente basse, evitando cambi di contesto e blocchi.
	      Inoltre, è progettato per sfruttare l'\textit{hardware} moderno, tra cui unità NVMe, processori \textit{multi-core} e interfacce di rete ad alta velocità;
	\item \textbf{semplicità di configurazione}: oltre al \textit{message broker}, contiene anche un \textit{proxy} HTTP e uno \textit{schema registry};
	\item \textbf{minore richiesta di risorse}: rispetto ad Apache Kafka, richiede meno risorse per l'esecuzione in locale, rendendolo più adatto per l'esecuzione su \textit{hardware} meno potente;
	\item \textbf{compatibilità con le API di Kafka}: è compatibile con le API di Apache Kafka, consentendo di utilizzare le librerie e gli strumenti esistenti;
\end{itemize}
\subsubsubsection{Casi d'uso}
Tra i casi d'uso di Redpanda si possono citare:
\begin{itemize}
	\item \textbf{\textit{streaming} di eventi}, permettendo la gestione e l'elaborazione di flussi di dati in tempo reale;
	\item \textbf{\textit{data integration}}, agisce come un intermediario flessibile e robusto per l'integrazione dei dati, consentendo la raccolta, il trasporto e la trasformazione dei dati provenienti da diverse sorgenti verso varie destinazioni;
	\item \textbf{elaborazione di \textit{big data}}, permette di gestire e processare enormi volumi di dati in modo efficiente e scalabile;
	\item \textbf{messaggistica \textit{real time}}, supporta la messaggistica in tempo reale tra applicazioni e sistemi distribuiti.
\end{itemize}

\subsubsubsection{Impiego nel progetto}
Il \textbf{broker} Redpanda gestisce i dati provenienti dai simulatori e li rende disponibili per i due consumatori. Inoltre,
con lo \textit{schema registry} integrato è possibile garantire la compatibilità tra i dati prodotti dai simulatori e i consumatori.
I consumatori sono:
\begin{itemize}
	\item Il \textbf{connector sink ClickHouse}, che salva i dati nelle tabelle di ClickHouse;
	\item \textbf{Apache Flink}, che elabora i dati in tempo reale.
\end{itemize}

\subsubsection{ClickHouse}
ClickHouse è un sistema di gestione di database colonnare \textit{open-source} progettato per l'analisi dei dati in tempo reale e l'elaborazione di grandi volumi di dati.
\begin{itemize}
	\item \textbf{Versione}: v24-alpine;
	\item \textbf{documentazione}: \url{https://clickhouse.com/docs/en/intro} [Ultima consultazione: 2024-06-02].
\end{itemize}

\subsubsubsection{Vantaggi}
I vantaggi nell'utilizzo di questo strumento consistono in:
\begin{itemize}
	\item \textbf{alte prestazioni}, è progettato per eseguire query analitiche complesse in modo estremamente rapido;
	\item \textbf{scalabilità orizzontale}, può essere scalato orizzontalmente su più nodi, permettendo di gestire grandi volumi di dati;
	\item \textbf{elaborazione in tempo reale}, è in grado di gestire l'ingestione e l'elaborazione dei dati in tempo reale, rendendolo ideale per applicazioni che richiedono l'analisi immediata dei dati appena arrivano;
	\item \textbf{compressione efficiente}, utilizza algoritmi di compressione avanzati per ridurre lo spazio di archiviazione e migliorare l'efficienza I/O;
	\item \textbf{facilità di integrazione}, si integra facilmente con molti strumenti di visualizzazione dei dati e piattaforme di business intelligence come Grafana;
	\item \textbf{partizionamento e indici}, supporta il partizionamento dei dati e l'uso di indici per ottimizzare le query;
\end{itemize}
\subsubsubsection{Casi d'uso}
ClickHouse è utilizzato in una varietà di casi d'uso, tra cui:
\begin{itemize}
	\item \textbf{analisi dei \textit{log} e monitoraggio}, utilizzato per l'analisi e il monitoraggio dei log in tempo reale;
	\item \textbf{\textit{business intelligence}}, impiegato in applicazioni di BI per eseguire analisi approfondite dei dati aziendali, supportando la presa di decisioni basata sui dati;
	\item \textbf{\textit{data warehousing}}, funziona come data warehouse per memorizzare e analizzare grandi volumi di dati.
\end{itemize}
\subsubsubsection{Impiego nel progetto}
ClickHouse viene utilizzato per memorizzare i dati grezzi provenienti dai simulatori; attraverso il \textit{connector sink} di Redpanda, i \textit{record} pubblicati nei \textit{topic} vengono
salvati in tabelle di ClickHouse.
Inoltre, tramite l'utilizzo di Materialized Views, vengono effettuate delle semplici aggregazioni sui dati, come ad esempio la media oraria o giornaliera, le quali sono poi memorizzate in apposite tabelle,
in modo da poterne monitorare l'andamento nel tempo.\\
Le aggregazioni più complesse che coinvolgono dati provenienti da sensori differenti sono invece effettuate utilizzando Apache Flink, come meglio descritto nella sezione \ref{sec:flink}.\\
ClickHouse si integra semplicemente con Grafana, attraverso l'utilizzo del plugin \\
\texttt{datasource-clickhouse}, fornito da Grafana Labs.

\subsubsection{Apache Flink} \label{sec:flink}
Apache Flink è un framework open-source per l'elaborazione dei dati in tempo reale e in \textit{batch}.
Sviluppato in Java e Scala, è progettato per gestire \textit{data stream} in modo efficiente, consentendo l'elaborazione di grandi volumi di dati in tempo reale.
Flink si distingue per la sua capacità di fornire elaborazione a bassa latenza, esecuzione \textit{fault-tolerant} e scalabilità orizzontale.
Come sistema di \textit{build} e gestione delle dipendenze, è stato utilizzato Maven.

\begin{itemize}
	\item \textbf{Versione}: v1.18.1;
	\item \textbf{documentazione}: \url{https://flink.apache.org} [Ultima consultazione: 2024-06-25].
\end{itemize}

\subsubsubsection{Vantaggi}
\begin{itemize}
	\item \textbf{elaborazione a bassa latenza}: Flink è progettato per elaborare i dati in tempo reale con latenza estremamente bassa, rendendolo ideale per applicazioni che richiedono risposte rapide ai cambiamenti dei dati.
	\item \textbf{\textit{fault tolerance}}: Flink utilizza una tecnologia chiamata \textit{Stateful Stream Processing} che garantisce che lo stato dell'applicazione venga memorizzato in modo sicuro e possa essere recuperato in caso di guasti. Questo consente un'elaborazione affidabile e continua anche in presenza di errori hardware o software.
	\item \textbf{scalabilità}: Flink può scalare orizzontalmente su cluster di grandi dimensioni, distribuendo il carico di lavoro tra molteplici nodi per gestire volumi di dati crescenti senza compromettere le prestazioni.
	\item \textbf{modello di programmazione flessibile}: Flink offre due tipologie di API per l'elaborazione dei dati: DataStream API, utilizzato per l'elaborazione di flussi di dati non strutturati in tempo reale, e Table API, un'astrazione di livello superiore per manipolare dati strutturati come tabelle, facilitando l'uso di operazioni simili a SQL;
	\item \textbf{supporto per analisi complesse}: Flink fornisce potenti funzionalità di analisi come aggregazioni, join e \textit{windowing}, che consentono di realizzare analisi complesse sui flussi di dati.
\end{itemize}

\subsubsubsection{Casi d'uso}
Tra i principali casi d'uso di Apache Flink si trovano:
\begin{itemize}
	\item \textbf{applicazioni \textit{event-driven}}: applicazioni \textit{stateful} che elaborano eventi provenienti da uno o più flussi di eventi e reagiscono agli eventi in ingresso attivando calcoli, aggiornamenti di stato o azioni esterne;
	\item \textbf{\textit{data analytics}}: estrazione di informazioni e \textit{insight} a partire dall'elaborazione dei dati grezzi, sia in \textit{real time} che in modalità \textit{batch};
	\item \textbf{\textit{data pipeline}}: ad esempio per la costruzione di \textit{Extract-transform-load} (ETL) o integrazione di dati provenienti da sorgenti differenti.
\end{itemize}

\subsubsubsection{Impiego nel progetto}
Per ciascuno degli indici in seguito elencati è stato sviluppato un \textit{job} in Apache Flink; i dettagli implementativi di ciascuno di essi sono meglio discussi nella sezione 3.5. %TODO: numero corretto
\begin{itemize}
	\item \textbf{\textit{Heat Index}}: una misura che combina la temperatura dell'aria e l'umidità relativa per determinare la temperatura percepita dall'uomo. Questa misura riflette meglio il livello di disagio che una persona potrebbe sperimentare rispetto alla sola temperatura dell'aria;
	\item \textbf{\textit{Charging Efficiency}}: combinando i dati provenienti dai sensori di occupazione dei parcheggi e all'utilizzo delle colonnine di ricarica (relative
	      allo stesso parcheggio), si calcola l'efficienza di utilizzo di queste ultime; viene calcolato sia un indice di utilizzo (\texttt{utilization\_rate}) nel periodo
	      di tempo considerato, sia un indice di efficienza (\texttt{efficiency\_rate}) che rappresenta quanto tempo la colonnina è stata effettivamente utilizzata rispetto
	      al tempo in cui il parcheggio è stato occupato.
\end{itemize}

\subsubsection{Grafana}
È una potente piattaforma di visualizzazione dei dati progettata per creare, esplorare e condividere \textit{dashboard}  interattive che visualizzano metriche, \textit{log} e altri dati di monitoraggio in tempo reale.

\begin{itemize}
	\item \textbf{Versione}: v10.3.0;
	\item \textbf{documentazione}: \url{https://grafana.com/docs/grafana/v10.4/} [Ultima consultazione: 2024-06-02].
\end{itemize}

\subsubsubsection{Vantaggi}
\begin{itemize}
	\item \textbf{Facilità d'uso}: possiede un'interfaccia intuitiva che rende facile la creazione e la gestione delle dashboard;
	\item \textbf{flessibilità}: La capacità di integrarsi con molteplici sorgenti dati e l'ampia gamma di plugin disponibili la rendono estremamente flessibile;
	\item \textbf{personalizzazione}: permette una personalizzazione completa delle dashboard, soddisfando ogni possibile necessità di visualizzazione dei dati;
	\item \textbf{gestione degli accessi}: offre funzionalità avanzate di gestione degli accessi e delle autorizzazioni, consentendo di controllare chi può accedere alle \textit{dashboard} e quali azioni possono eseguire.
\end{itemize}
\subsubsubsection{Casi d'uso}
\begin{itemize}
	\item \textbf{Monitoraggio delle infrastrutture}: utilizzato per monitorare le prestazioni e la disponibilità delle infrastrutture IT, inclusi server, database, servizi cloud e altro;
	\item \textbf{analisi delle performance delle applicazioni}: utilizzato per monitorare le prestazioni delle applicazioni e identificare eventuali problemi di prestazioni;
	\item \textbf{analisi delle serie temporali}: utilizzato per visualizzare e analizzare dati di serie temporali, come metriche di monitoraggio, log e dati di sensori;
	\item \textbf{business intelligence}: utilizzato per creare \textit{dashboard} personalizzate per l'analisi dei dati aziendali e la visualizzazione delle metriche chiave.
\end{itemize}
\subsubsubsection{Impiego nel progetto}
\begin{itemize}
	\item \textbf{Visualizzazione dei dati}: creazione \textit{dashboard} interattive che visualizzano i dati salvati su ClickHouse;
	\item \textbf{notifiche superamento soglie}: invio di notifiche nel caso in cui vengano superate delle soglie prestabilite, che rappresentano situazioni di eventuale pericolo, forte disagio o disservizio per i cittadini.
\end{itemize}
