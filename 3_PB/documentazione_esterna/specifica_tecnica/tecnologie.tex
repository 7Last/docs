\section{Tecnologie}
Questa sezione si occupa di fornire una panoramica delle tecnologie utilizzate per implementare il sistema software.
In particolare, delinea le piattaforme, gli strumenti, i linguaggi di programmazione, i framework e altre risorse tecnologiche che sono state impiegate durante lo sviluppo.

\subsection{Docker}
È una piattaforma di virtualizzazione leggera che semplifica lo sviluppo, il testing e il rilascio delle applicazioni fornendo un ambiente isolato e riproducibile.
È utilizzato per creare ambienti di sviluppo standardizzati, facilitare la scalabilità delle applicazioni e semplificare la gestione delle risorse.

\subsubsection{Ambienti}
Per lo sviluppo di questo progetto sono stati ipotizzati i due seguenti scenari di esecuzione, separati grazie all'utilizzo di profili diversi di Docker Compose:
\begin{itemize}
	\item \texttt{local}: utilizzato dagli sviluppatori per testare e sviluppare le funzionalità dell'applicazione sui propri computer.
	      Questo ambiente permette di eseguire tutti i componenti del sistema all'interno di un container Docker, ad eccezione del simulatore Python.
	      Esso viene eseguito direttamente sul sistema operativo dell'utente, in modo da facilitare il debugging e il testing delle funzionalità,
	      senza dover necessariamente eseguire la \textit{build} dell'immagine Docker ad ogni modifica del codice;
	\item \texttt{release}: utilizzato quando si desidera simulare un ipotetico ambiente di produzione o non è necessario modificare il codice Python. Consente di non dover manualmente
	      installare le dipendenze o configurare l'ambiente di esecuzione. In questo caso, tutti i componenti del sistema vengono eseguiti all'interno di container Docker.

\end{itemize}

\subsubsection{Immagini Docker}
Nello sviluppo di questo progetto \textit{7Last} ha utilizzato diverse immagini Docker di seguito elencate.
\begin{itemize}
	\item \textbf{Simulator - Python}
	      \begin{itemize}
		      \item \textbf{Immagine}: 3.11.9-alpine;
		      \item \textbf{Riferimento}: \underline{\href{https://hub.docker.com/_/python}{Python Docker Image}} [Ultima consultazione: 2024-06-02].
		      \item \textbf{Ambiente}: \texttt{release};
	      \end{itemize}

	\item \textbf{Redpanda}
	      \begin{itemize}
		      \item \textbf{Immagine}: docker.redpanda.com/redpandadata/redpanda:v23.3.11;
		      \item \textbf{Riferimento}: \underline{\href{https://hub.docker.com/r/redpandadata/redpanda}{Redpanda Docker Image}} [Ultima consultazione: 2024-06-02].
		      \item \textbf{Ambiente}: \texttt{local}, \texttt{release}.
	      \end{itemize}

	\item \textbf{Redpanda console}
	      \begin{itemize}
		      \item \textbf{Immagine}: docker.redpanda.com/redpandadata/console:v2.4.6;
		      \item \textbf{Riferimento}: \underline{\href{https://hub.docker.com/r/redpandadata/redpanda}{Redpanda Console Docker Image}} [Ultima consultazione: 2024-06-02].
		      \item \textbf{Ambiente}: \texttt{local}, \texttt{release}.
	      \end{itemize}

	\item \textbf{Connectors}
	      \begin{itemize}
		      \item \textbf{Immagine}: docker.redpanda.com/redpandadata/connectors:v1.0.27;
		      \item \textbf{Riferimento}: \underline{\href{https://hub.docker.com/r/redpandadata/connectors}{Redpanda Connectors Docker Image}} [Ultima consultazione: 2024-06-02].
		      \item \textbf{Ambiente}: \texttt{local}, \texttt{release}.
	      \end{itemize}

	\item \textbf{ClickHouse}
	      \begin{itemize}
		      \item \textbf{Immagine}: clickhouse/clickhouse-server:24-alpine;
		      \item \textbf{Riferimento}: \underline{\href{https://hub.docker.com/r/clickhouse/clickhouse-server}{ClickHouse Docker Image}} [Ultima consultazione: 2024-06-02].
		      \item \textbf{Ambiente}: \texttt{local}, \texttt{release}.
	      \end{itemize}

	\item \textbf{Grafana}
	      \begin{itemize}
		      \item \textbf{Immagine}: grafana/grafana-oss:10.3.0;
		      \item \textbf{Riferimento}: \underline{\href{https://hub.docker.com/r/grafana/grafana-oss}{Grafana Docker Image}} [Ultima consultazione: 2024-06-02].
		      \item \textbf{Ambiente}: \texttt{local}, \texttt{release}.
	      \end{itemize}

	\item \textbf{Apache Flink}
	      \begin{itemize}
		      \item \textbf{Immagine}: flink:1.18.1-java17;
		      \item \textbf{Riferimento}: \underline{\href{https://hub.docker.com/_/flink}{Flink Docker Image}} [Ultima consultazione: 2024-06-02].
		      \item \textbf{Ambiente}: \texttt{local}, \texttt{release}.
	      \end{itemize}
\end{itemize}

\subsection{Linguaggi e formato dati}
\begin{longtable}{|>{\centering\arraybackslash}m{0.10\textwidth}|>{\centering\arraybackslash}m{0.10\textwidth}|>{\centering\arraybackslash}m{0.35\textwidth}|>{\centering\arraybackslash}m{0.35\textwidth}|}
	\hline
	\textbf{Nome} & \textbf{Versione} & \textbf{Descrizione}                                                                                                                                 & \textbf{Impiego}                                                                                                             \\\hline
	\endfirsthead
	\hline
	\textbf{Nome} & \textbf{Versione} & \textbf{Descrizione}                                                                                                                                 & \textbf{Impiego}                                                                                                             \\\hline
	\endhead
	Python        & 3.11.9            & Linguaggio di programmazione ad alto livello, interpretato e multiparadigma.                                                                         & Simulatore di sensori, \textit{testing}, \textit{script} per automatizzare il \textit{deployment} dei \textit{job} di Flink. \\\hline
	JSON          & -                 & Formato di dati semplice da interpretare e generare, ampiamente utilizzato per lo scambio di dati tra applicazioni.                                  & Configurazione dashboard Grafana.                                                                                            \\\hline
	YAML          & -                 & Linguaggio di serializzazione dei dati leggibile sia per gli esseri umani sia per le macchine.                                                       & Docker Compose, provisioning Grafana e configurazione \textit{alert}, file di \textit{workflow} per le GitHub Actions.       \\\hline
	SQL           & Ansi SQL          & Linguaggio di programmazione specificamente progettato per la gestione e la manipolazione di dati all'interno di sistemi di gestione di database.    & \textit{Query} e gestione database ClickHouse.                                                                               \\\hline
	TOML          & 1.0.0             & Linguaggio di \textit{markup} progettato per essere più leggibile e facile da scrivere rispetto ad altri formati di configurazione come JSON e YAML. & Configurazione e gestione dei sensori simulati.                                                                              \\\hline
	Java          & 17                & Linguaggio di programmazione ad alto livello, orientato agli oggetti.                                                                                & Creazione di job per le aggregazioni dei dati di Flink.                                                                      \\\hline
	\caption{Linguaggi e formato dati}
\end{longtable}

\subsection{Librerie}
\begin{longtable}{|>{\centering\arraybackslash}m{0.35\textwidth}|>{\centering\arraybackslash}m{0.10\textwidth}|>{\centering\arraybackslash}m{0.45\textwidth}|}
	\hline
	\multicolumn{3}{|c|}{\textbf{Python}}                                                                                                          \\
	\hline
	\textbf{Nome}                          & \textbf{Versione} & \textbf{Impiego}                                                                  \\\hline
	\endfirsthead
	\hline
	\textbf{Nome}                          & \textbf{Versione} & \textbf{Impiego}                                                                  \\
	\endhead
	% \texttt{avro} & 1.11.3            & Libreria per la serializzazione dei dati in formato Avro. & Serializzazione dei dati in formato Avro. \\\hline
	\texttt{confluent\_avro}               & 1.8.0             & Serializzazione dei dati in formato Avro.                                         \\\hline
	\texttt{coverage}                      & 7.5.1             & Strumento per misurare la percentuale di linee di codice e rami coperti dai test. \\\hline
	\texttt{isodate}                       & 0.6.1             & Libreria per la manipolazione delle date e delle ore in formato ISO8601.          \\\hline
	\texttt{kafka-python-ng}               & 2.2.2             & Client Kafka per Python.                                                          \\\hline
	% \texttt{requests}          & 2.32.3            & Libreria per effettuare richieste HTTP.                                           \\\hline
	\texttt{ruff}                          & 0.3.5             & Libreria per l'analisi statica del codice.                                        \\\hline
	\texttt{toml}                          & 0.10.2            & Libreria per effettuare il parsing dei file di configurazione in formato TOML.    \\\hline

	\multicolumn{3}{|c|}{\textbf{Java}}                                                                                                            \\\hline
	% TODO: controllare quali effettivamente necessarie/utilizzate
	\texttt{flink-streaming-java}          & 1.18.0            & Utilizzo di DataStream API di Flink.                                              \\\hline
	\texttt{flink-connector-kafka}         & 3.1.0-1.18        & Connessione di Flink a Kafka.                                                     \\\hline
	\texttt{flink-clients}                 & 1.18.0            & Creazione di \textit{job} di Flink.                                               \\\hline
	\texttt{flink-java}                    & 1.18.0            & Creazione di \textit{job} di Flink.                                               \\\hline
	\texttt{flink-avro-confluent-registry} & 1.18.0            & Connessione di Flink a uno \textit{schema registry} che utilizza Avro.            \\\hline
	% \texttt{flink-avro}            & 1.18.0            & Libreria per la connessione di Flink ad Avro.                                      \\\hline
	\texttt{flink-shaded-guava}            & 31.1-jre-17.0     & Gestione delle dipendenze di Flink.                                               \\\hline
	\texttt{slf4j-simple}                  & 1.7.36            & Implementazione di SLF4J.                                                         \\\hline
	\texttt{lombok}                        & 1.18.32           & Libreria per la generazione di codice \textit{boilerplate}.                       \\\hline
	\texttt{maven-assembly-plugin}         & 3.7.1             & Plugin Maven per la creazione di un \textit{fat jar}.                             \\\hline
	\caption{Librerie utilizzate}
\end{longtable}

\subsection{Servizi}
\subsubsection{Redpanda}
Redpanda è una piattaforma di streaming sviluppata in C++. Il suo obiettivo è fornire una soluzione leggera, semplice e performante, pensata per essere un'alternativa ad Apache Kafka. Viene utilizzato per disaccoppiare i dati provenienti dal simulatore.
% Ciascun tipo di dato viene inviato su un topic specifico, in modo da poter essere elaborato in modo indipendente.
\subsubsubsection{Versione}
La versione impiegata per lo sviluppo del progetto è la v23.3.11.
\subsubsubsection{Documentazione}
\url{https://docs.redpanda.com/current/home/} [Ultima consultazione: 2024-06-02].
\subsubsubsection{Vantaggi}
I vantaggi nell'utilizzo di questo strumento consistono in:
\begin{itemize}
	\item \textbf{\textit{performance}}: è scritto in C++ e utilizza il \textit{framework} Seastar, offrendo un'architettura \textit{thread-per-core} ad alte prestazioni.
	      Ciò permette di ottenere un'elevata \textit{throughput} e latenze costantemente basse, evitando cambi di contesto e blocchi.
	      Inoltre, è progettato per sfruttare l'\textit{hardware} moderno, tra cui unità NVMe, processori \textit{multi-core} e interfacce di rete ad alta velocità;
	\item \textbf{semplicità di configurazione}: oltre al \textit{message broker}, contiene anche un \textit{proxy} HTTP e uno \textit{schema registry};
	\item \textbf{minore richiesta di risorse}: rispetto ad Apache Kafka, richiede meno risorse per l'esecuzione in locale, rendendolo più adatto per l'esecuzione su \textit{hardware} meno potente;
	\item \textbf{compatibilità con le API di Kafka}: è compatibile con le API di Apache Kafka, consentendo di utilizzare le librerie e gli strumenti esistenti;
\end{itemize}
\subsubsubsection{Casi d'Uso}
Tra i casi d'uso di Redpanda si possono citare:
\begin{itemize}
	\item \textbf{\textit{streaming} di eventi}, permettendo la gestione e l'elaborazione di flussi di dati in tempo reale;
	\item \textbf{\textit{data integration}}, agisce come un intermediario flessibile e robusto per l'integrazione dei dati, consentendo la raccolta, il trasporto e la trasformazione dei dati provenienti da diverse sorgenti verso varie destinazioni;
	\item \textbf{elaborazione di \textit{big data}}, permette di gestire e processare enormi volumi di dati in modo efficiente e scalabile;
	\item \textbf{messaggistica \textit{real time}}, supporta la messaggistica in tempo reale tra applicazioni e sistemi distribuiti.
\end{itemize}

\subsubsubsection{Impiego nel progetto}
Il \textbf{broker} Redpanda gestisce i dati provenienti dai simulatori e li rende disponibili per i due consumatori. Inoltre,
con lo \textit{schema registry} integrato è possibile garantire la compatibilità tra i dati prodotti dai simulatori e i consumatori.
I consumatori sono:
\begin{itemize}
	\item Il \textbf{connector sink ClickHouse}, che salva i dati nelle tabelle di ClickHouse;
	\item \textbf{Apache Flink}, che elabora i dati in tempo reale.
\end{itemize}

\subsubsubsection{Formato dei dati e Schema Registry}
% TODO: parlare di Avro e del formato di serializzazione delle chiavi
% TODO: parlare di topic name strategy
% TODO: parlare di partitioning strategy


\subsubsection{ClickHouse}
ClickHouse è un sistema di gestione di database colonnare \textit{open-source} progettato per l'analisi dei dati in tempo reale e l'elaborazione di grandi volumi di dati.
\subsubsubsection{Versione}
La versione impiegata per lo sviluppo del progetto è la 24.3.2.23.
\subsubsubsection{Documentazione}
\url{https://clickhouse.com/docs/en/intro} [Ultima consultazione: 2024-06-02].
\subsubsubsection{Vantaggi}
I vantaggi nell'utilizzo di questo strumento consistono in:
\begin{itemize}
	\item \textbf{alte prestazioni}, è progettato per eseguire query analitiche complesse in modo estremamente rapido;
	\item \textbf{scalabilità orizzontale}, può essere scalato orizzontalmente su più nodi, permettendo di gestire grandi volumi di dati;
	\item \textbf{elaborazione in tempo reale}, è in grado di gestire l'ingestione e l'elaborazione dei dati in tempo reale, rendendolo ideale per applicazioni che richiedono l'analisi immediata dei dati appena arrivano;
	\item \textbf{compressione efficiente}, utilizza algoritmi di compressione avanzati per ridurre lo spazio di archiviazione e migliorare l'efficienza I/O;
	\item \textbf{facilità di integrazione}, si integra facilmente con molti strumenti di visualizzazione dei dati e piattaforme di business intelligence come Grafana;
	\item \textbf{partizionamento e indici}, supporta il partizionamento dei dati e l'uso di indici per ottimizzare le query;
\end{itemize}
\subsubsubsection{Casi d'Uso}
ClickHouse è utilizzato in una varietà di casi d'uso, tra cui:
\begin{itemize}
	\item \textbf{analisi dei log e monitoraggio}, utilizzato per l'analisi e il monitoraggio dei log in tempo reale;
	\item \textbf{\textit{business intelligence}}, impiegato in applicazioni di BI per eseguire analisi approfondite dei dati aziendali, supportando la presa di decisioni basata sui dati;
	\item \textbf{\textit{data warehousing}}, funziona come data warehouse per memorizzare e analizzare grandi volumi di dati.
\end{itemize}
\subsubsubsection{Impiego nel progetto}
\begin{itemize}
	\item \textbf{Organizzazione efficiente dei dati}: il tipo di architettura di \textit{ClickHouse} permette di comprimere i dati in modo più efficace e di leggere solo le colonne necessarie durante l'esecuzione delle query.
	\item \textbf{Integrazione con Redpanda}: può essere utilizzato in sinergia con Redpanda, una piattaforma di streaming dati compatibile con Apache Kafka. Questa integrazione permette di ingestire, elaborare e analizzare flussi di dati in tempo reale.
	\item \textbf{Aggregazione rapida dei dati}: è progettato per eseguire aggregazioni di dati in modo estremamente veloce. Grazie alle sue capacità di elaborazione colonnare, alle tecniche avanzate di compressione e all'uso di indici, ClickHouse può eseguire calcoli aggregati su grandi dataset in tempi molto ridotti.
	\item \textbf{Integrazione con Grafana}: si integra facilmente con Grafana, una delle piattaforme di visualizzazione dei dati più popolari.
\end{itemize}

\subsubsection{Grafana}
È una potente piattaforma di visualizzazione dei dati progettata per creare, esplorare e condividere dashboard interattive che visualizzano metriche, log e altri dati di monitoraggio in tempo reale.
\subsubsubsection{Versione}
La versione impiegata per lo sviluppo del progetto è la 10.3.0.
\subsubsubsection{Documentazione}
\url{https://grafana.com/docs/grafana/v10.4/} [Ultima consultazione: 2024-06-02].

\subsubsubsection{Vantaggi}
\begin{itemize}
	\item \textbf{Facilità d'uso}: possiede un'interfaccia intuitiva che rende facile la creazione e la gestione delle dashboard;
	\item \textbf{flessibilità}: La capacità di integrarsi con molteplici sorgenti dati e l'ampia gamma di plugin disponibili la rendono estremamente flessibile;
	\item \textbf{personalizzazione}: permette una personalizzazione completa delle dashboard, soddisfando ogni possibile necessità di visualizzazione dei dati;
	\item \textbf{gestione degli accessi}: offre funzionalità avanzate di gestione degli accessi e delle autorizzazioni, consentendo di controllare chi può accedere alle dashboard e quali azioni possono eseguire.
\end{itemize}
\subsubsubsection{Casi d'Uso}
\begin{itemize}
	\item \textbf{Monitoraggio delle infrastrutture}: utilizzato per monitorare le prestazioni e la disponibilità delle infrastrutture IT, inclusi server, database, servizi cloud e altro;
	\item \textbf{analisi delle performance delle applicazioni}: utilizzato per monitorare le prestazioni delle applicazioni e identificare eventuali problemi di prestazioni;
	\item \textbf{analisi delle serie temporali}: utilizzato per visualizzare e analizzare dati di serie temporali, come metriche di monitoraggio, log e dati di sensori;
	\item \textbf{business intelligence}: utilizzato per creare dashboard personalizzate per l'analisi dei dati aziendali e la visualizzazione delle metriche chiave.
\end{itemize}
\subsubsubsection{Impiego nel progetto}
\begin{itemize}
	\item \textbf{Visualizzazione dei dati}: utilizzato per creare dashboard interattive che visualizzano i dati provenienti da ClickHouse;
	\item \textbf{analisi dei dati}: utilizzato per analizzare i dati e identificare tendenze, pattern e anomalie;
	\item \textbf{monitoraggio degli allarmi}: utilizzato per monitorare e visualizzare gli allarmi generati dai simulatori e dagli altri componenti del sistema;
	\item \textbf{notifiche}: utilizzato per inviare notifiche in caso di superamento di soglie critiche.
\end{itemize}
