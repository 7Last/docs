\section{Metriche di qualità}
La qualità di processo è un criterio fondamentale ed è alla base di ogni prodotto che rispecchi lo stato dell'arte. Per raggiungere tale obiettivo è necessario sfruttare delle pratiche rigorose che consentano lo svolgimento di ogni attività in maniera ottimale. Al fine di valutare nel miglior modo possibile la qualità del prodotto e l'efficacia dei processi, sono state definite delle metriche, meglio specificate di seguito. Il contenuto di questa sezione è necessario per identificare i parametri che le metriche devono rispettare per essere considerate accettabili o ottime. Esse sono state suddivise utilizzando lo \textit{standard ISO/IEC 12207:1995}, il quale separa i processi di ciclo di vita del software, in tre categorie: % TODO aggiungere versione documento
\begin{itemize}
	\item processi di base e/o primari;
	\item processi di supporto;
	\item processi organizzativi.
\end{itemize}

\subsection{Processi di base e/o primari}

\subsubsection{Fornitura}
Nella fase di fornitura si definiscono le procedure e le risorse necessarie per la consegna del prodotto. Per valutare la qualità di tale processo, sono state definite le seguenti metriche.

\hypertarget{1M}{\subsubsubsection*{1M-PV Planned Value}}
\begin{itemize}
	\item \textbf{Definizione}: il \textit{Planned Value} (o Valore Pianificato) rappresenta il valore del lavoro programmato per essere completato fino a un determinato momento. Si tratta del budget preventivato per lo \href{https://7last.github.io/docs/pb/documentazione-interna/glossario\#sprint}{sprint\textsubscript{G}} in corso.
	\item \textbf{Come si calcola}: \begin{equation*}PV = BAC \times LP\end{equation*} dove:
		\begin{itemize}
			\item $BAC$: Budget At Completion;
			\item $LP$: percentuale di lavoro pianificato.
		\end{itemize}
	\item \textbf{Valore ammissibile}: \begin{equation*}PV \geq 0\end{equation*}
	\item \textbf{Valore ottimo}: \begin{equation*}PV \leq BAC\end{equation*}
\end{itemize}

\hypertarget{2M}{\subsubsubsection*{2M-EV Earned Value}}
\begin{itemize}
	\item \textbf{Definizione}: l'\textit{Earned Value} (o Valore Guadagnato) rappresenta il valore del lavoro effettivamente completato fino al periodo in analisi.
	\item \textbf{Come si calcola}: \begin{equation*}EV = BAC \times LC\end{equation*} dove:
		\begin{itemize}
			\item $BAC$: Budget At Completion;
			\item $LC$: percentuale di lavoro completato.
		\end{itemize}
	\item \textbf{Valore ammissibile}: \begin{equation*}EV \geq 0\end{equation*}
	\item \textbf{Valore ottimo}: \begin{equation*}EV \leq EAC\ (Estimated\ At\ Completion)\end{equation*}
\end{itemize}

\hypertarget{3M}{\subsubsubsection*{3M-AC Actual Cost}}
\begin{itemize}
	\item \textbf{Definizione}: l'\textit{Actual Cost} (o Costo Effettivo) rappresenta il costo effettivamente sostenuto per completare il lavoro fino al periodo in analisi.
	\item \textbf{Come si calcola}: si ottiene sommando tutti i costi effettivi sostenuti fino a quella data.
	\item \textbf{Valore ammissibile}: \begin{equation*}AC \geq 0\end{equation*}
	\item \textbf{Valore ottimo}: \begin{equation*}AC \leq EAC\end{equation*}
\end{itemize}

\hypertarget{4M}{\subsubsubsection*{4M-SV Schedule Variance}}
\begin{itemize}
	\item \textbf{Definizione}: la \textit{Schedule Variance} (o Variazione di Programma) rappresenta la differenza tra il valore del lavoro effettivamente completato e il valore del lavoro pianificato, calcolata in percentuale.
	\item \textbf{Come si calcola}: \begin{equation*}SV = \frac{EV - PV}{EV}\end{equation*}
	\item \textbf{Valore ammissibile}: \begin{equation*}SV \geq -10\%\end{equation*}
	\item \textbf{Valore ottimo}: \begin{equation*}SV \geq 0\%\end{equation*}
\end{itemize}

\hypertarget{5M}{\subsubsubsection*{5M-CV Cost Variance}}
\begin{itemize}
	\item \textbf{Definizione}: la \textit{Cost Variance} (o Variazione dei Costi) rappresenta la differenza tra il valore del lavoro effettivamente completato e il costo effettivamente sostenuto per completarlo, calcolata in percentuale.
	\item \textbf{Come si calcola}: \begin{equation*}CV = \frac{EV - AC}{EV}\end{equation*}
	\item \textbf{Valore ammissibile}: \begin{equation*}CV \geq -10\%\end{equation*}
	\item \textbf{Valore ottimo}: \begin{equation*}CV \geq 0\%\end{equation*}
\end{itemize}

\hypertarget{6M}{\subsubsubsection*{6M-CPI Cost Performance Index}}
\begin{itemize}
	\item \textbf{Definizione}: il \textit{Cost Performance Index} rappresenta il rapporto tra il valore del lavoro effettivamente completato e i costi sostenuti per completarlo.
	\item \textbf{Come si calcola}: \begin{equation*}CPI = \frac{EV}{AC}\end{equation*}
	\item \textbf{Valore ammissibile}: \begin{equation*}CPI \geq 0.8\end{equation*}
	\item \textbf{Valore ottimo}: \begin{equation*}CPI \geq 1\end{equation*}
\end{itemize}

\hypertarget{7M}{\subsubsubsection*{7M-SPI Schedule Performance Index}}
\begin{itemize}
	\item \textbf{Definizione}: lo \textit{Schedule Performance Index} rappresenta l'efficienza con cui il progetto sta rispettando il programma.
	\item \textbf{Come si calcola}: \begin{equation*}SPI = \frac{EV}{PV}\end{equation*}
	\item \textbf{Valore ammissibile}: \begin{equation*}SPI \geq 0.8\end{equation*}
	\item \textbf{Valore ottimo}: \begin{equation*}SPI \geq 1\end{equation*}
\end{itemize}

\hypertarget{8M}{\subsubsubsection*{8M-EAC Estimate At Completion}}
\begin{itemize}
	\item \textbf{Definizione}: l'\textit{Estimate at Completion} (o Stima al Completamento) rappresenta una previsione aggiornata del costo totale del progetto basata sulle performance attuali, calcolata in base ai costi effettivamente sostenuti e ai costi stimati per completare il lavoro rimanente.
	\item \textbf{Come si calcola}: \begin{equation*}EAC = AC + ETC\end{equation*}
	\item \textbf{Valore ammissibile}: \begin{equation*}EAC \leq BAC + 5\%\end{equation*}
	\item \textbf{Valore ottimo}: \begin{equation*}EAC \leq BAC\end{equation*}
\end{itemize}

\hypertarget{9M}{\subsubsubsection*{9M-ETC Estimate To Complete}}
\begin{itemize}
	\item \textbf{Definizione}: l'\textit{Estimate to Complete} (o Stima al Completamento) rappresenta una previsione del costo necessario per completare le attività rimanenti del progetto basata sulle performance attuali.
	\item \textbf{Come si calcola}: \begin{equation*}ETC = EAC - AC\end{equation*}
	\item \textbf{Valore ammissibile}: \begin{equation*}ETC \geq 0\end{equation*}
	\item \textbf{Valore ottimo}: \begin{equation*}ETC \leq EAC\end{equation*}
\end{itemize}

\hypertarget{10M}{\subsubsubsection*{10M-OTDR On-Time Delivery Rate}}
\begin{itemize}
	\item \textbf{Definizione}: l'\textit{On-Time Delivery Rate} (o Tasso di Consegna nei Tempi) rappresenta la percentuale di attività completate entro la data di scadenza.
	\item \textbf{Come si calcola}: \begin{equation*}OTDR = \frac{AP}{AT}\end{equation*} dove:
		\begin{itemize}
			\item $AP$: attività completate entro la data di scadenza;
			\item $AT$: attività totali.
		\end{itemize}
	\item \textbf{Valore ammissibile}: \begin{equation*}OTDR \geq 90\%\end{equation*}
	\item \textbf{Valore ottimo}: \begin{equation*}OTDR \geq 95\%\end{equation*}
\end{itemize}



\subsubsection{Sviluppo}
Nella fase di sviluppo si realizza il prodotto software, seguendo le specifiche definite in fase di progettazione.

\subsubsubsection{Analisi dei requisiti}
Questa fase consiste nell'esaminare le richieste della \href{https://7last.github.io/docs/pb/documentazione-interna/glossario\#proponente}{proponente\textsubscript{G}} e nel definire i requisiti che il prodotto dovrà soddisfare. Per valutare la qualità di tale processo, sono state definite le seguenti metriche.

\hypertarget{11M}{\subsubsubsection*{11M-PRO Percentuale Requisiti Obbligatori}}
\begin{itemize}
	\item \textbf{Definizione}: rappresenta la percentuale di requisiti obbligatori soddisfatti secondo quanto definito nel documento \href{https://7last.github.io/docs/pb/documentazione-interna/glossario\#analisi-dei-requisiti}{\textit{Analisi dei Requisiti}\textsubscript{G}}.
	\item \textbf{Come si calcola}: \begin{equation*}PRO = \frac{ROS}{ROT}\end{equation*} dove:
		\begin{itemize}
			\item $ROS$: Requisiti Obbligatori Soddisfatti;
			\item $ROT$: Requisiti Obbligatori Totali.
		\end{itemize}
	\item \textbf{Valore ammissibile}: \begin{equation*}PRO \geq 100\%\end{equation*}
	\item \textbf{Valore ottimo}: \begin{equation*}PRO = 100\%\end{equation*}
\end{itemize}

\hypertarget{12M}{\subsubsubsection*{12M-PRD Percentuale Requisiti Desiderabili}}
\begin{itemize}
	\item \textbf{Definizione}: rappresenta la percentuale di requisiti desiderabili soddisfatti secondo quanto definito nel documento \href{https://7last.github.io/docs/pb/documentazione-interna/glossario\#analisi-dei-requisiti}{\textit{Analisi dei Requisiti}\textsubscript{G}}.
	\item \textbf{Come si calcola}: \begin{equation*}PRD = \frac{RDS}{RDT}\end{equation*} dove:
		\begin{itemize}
			\item $RDS$: Requisiti Desiderabili Soddisfatti;
			\item $RDT$: Requisiti Desiderabili Totali.
		\end{itemize}
	\item \textbf{Valore ammissibile}: \begin{equation*}PRD \geq 35\%\end{equation*}
	\item \textbf{Valore ottimo}: \begin{equation*}PRD = 100\%\end{equation*}
\end{itemize}

\hypertarget{13M}{\subsubsubsection*{13M-PRO Percentuale Requisiti Opzionali}}
\begin{itemize}
	\item \textbf{Definizione}: rappresenta la percentuale di requisiti opzionali soddisfatti secondo quanto definito nel documento \href{https://7last.github.io/docs/pb/documentazione-interna/glossario\#analisi-dei-requisiti}{\textit{Analisi dei Requisiti}\textsubscript{G}}.
	\item \textbf{Come si calcola}: \begin{equation*}PRO = \frac{ROS}{ROT}\end{equation*} dove:
		\begin{itemize}
			\item $ROS$: Requisiti Opzionali Soddisfatti;
			\item $ROT$: Requisiti Opzionali Totali.
		\end{itemize}
	\item \textbf{Valore ammissibile}: \begin{equation*}PRO \geq 0\%\end{equation*}
	\item \textbf{Valore ottimo}: \begin{equation*}PRO \geq 100\%\end{equation*}
\end{itemize}

\subsubsubsection{Progettazione}
Questa fase consiste nel definire l'architettura del prodotto software, in modo da soddisfare i requisiti definiti in fase di analisi. Per valutare la qualità di tale processo, sono state definite le seguenti metriche.
\hypertarget{14M}{\subsubsubsection*{14M-PG Profondità delle Gerarchie}}
\begin{itemize}
	\item \textbf{Definizione}: la profondità delle gerarchie rappresenta il numero massimo di livelli di annidamento delle classi.
	\item \textbf{Come si calcola}: si conta il numero massimo di livelli di annidamento delle classi.
	\item \textbf{Valore ammissibile}: \begin{equation*}PG \leq 7\end{equation*}
	\item \textbf{Valore ottimo}: \begin{equation*}PG \leq 5\end{equation*}
\end{itemize}

\subsubsubsection{Codifica}
Queste metriche aiutano a valutare la qualità del codice, la complessità e la manutenibilità.

\hypertarget{15M}{\subsubsubsection*{15M-PPM Parametri Per Metodo}}
\begin{itemize}
	\item \textbf{Definizione}: numero medio di parametri passati ai metodi. Un numero elevato di parametri può indicare che un metodo è troppo complesso o che potrebbe essere suddiviso in metodi più piccoli.
	\item \textbf{Come si calcola}: \begin{equation*}PPM = \frac{P}{M}\end{equation*} dove:
		\begin{itemize}
			\item $P$: numero totale di parametri;
			\item $M$: numero totale di metodi.
		\end{itemize}
	\item \textbf{Valore ammissibile}: \begin{equation*}PPM \leq 7\end{equation*}
	\item \textbf{Valore ottimo}: \begin{equation*}PPM \leq 5\end{equation*}
\end{itemize}

\hypertarget{16M}{\subsubsubsection*{16M-CPC Campi Per Classe}}
\begin{itemize}
	\item \textbf{Definizione}: numero medio di campi (variabili di istanza) per classe. Un numero elevato di campi dati può indicare che una classe sta facendo troppo e che potrebbe essere suddivisa in classi più piccole.
	\item \textbf{Come si calcola}: \begin{equation*}CPC = \frac{CD}{CL}\end{equation*} dove:
		\begin{itemize}
			\item $CD$: numero totale di campi dati;
			\item $CL$: numero totale di classi.
		\end{itemize}
	\item \textbf{Valore ammissibile}: \begin{equation*}CPC \leq 8\end{equation*}
	\item \textbf{Valore ottimo}: \begin{equation*}CPC \leq 5\end{equation*}
\end{itemize}

\hypertarget{17M}{\subsubsubsection*{17M-LCPM Linee di Codice Per Metodo}}
\begin{itemize}
	\item \textbf{Definizione}: numero medio di linee di codice per metodo. Metodi troppo lunghi possono essere difficili da leggere, capire e mantenere.
	\item \textbf{Come si calcola}: \begin{equation*}LCPM = \frac{LC}{M}\end{equation*} dove:
		\begin{itemize}
			\item $LC$: numero totale di linee di codice;
			\item $M$: numero totale di metodi.
		\end{itemize}
	\item \textbf{Valore ammissibile}: \begin{equation*}LCPM \leq 50\end{equation*}
	\item \textbf{Valore ottimo}: \begin{equation*}LCPM \leq 20\end{equation*}
\end{itemize}

\hypertarget{18M}{\subsubsubsection*{18M-CCM Complessità CicloMatica}}
\begin{itemize}
	\item \textbf{Definizione}: la \textit{Complessità CicloMatica} rappresenta la complessità di un programma sulla base del numero di percorsi lineari indipendenti attraverso il codice sorgente. Un valore elevato indica un codice più complesso e potenzialmente più difficile da mantenere.
	\item \textbf{Come si calcola}: \begin{equation*}CCM = E - N + 2P\end{equation*} dove:
		\begin{itemize}
			\item $E$: numero di archi del grafo;
			\item $N$: numero di nodi del grafo;
			\item $P$: numero di componenti connesse.
		\end{itemize}
	\item \textbf{Valore ammissibile}: \begin{equation*}CCM \leq 6\end{equation*}
	\item \textbf{Valore ottimo}: \begin{equation*}CCM \leq 3\end{equation*}
\end{itemize}



\subsection{Processi di supporto}
I processi di supporto si affiancano ai processi primari per garantire il corretto svolgimento delle attività.

\subsubsection{Documentazione}
La documentazione è un aspetto fondamentale per la comprensione del prodotto e per la sua manutenibilità. A livello pratico consiste nella redazione di manuali e documenti tecnici che descrivano il funzionamento del prodotto e le scelte progettuali adottate. Per valutare la qualità di tale processo, sono state definite le seguenti metriche.

\hypertarget{19M}{\subsubsubsection*{19M-IG Indice Gulpease}}
\begin{itemize}
	\item \textbf{Definizione}: l'Indice Gulpease è un indice di leggibilità di un testo tarato sulla lingua italiana. Misura la lunghezza delle parole e delle frasi rispetto al numero di lettere.
	\item \textbf{Come si calcola}: \begin{equation*}IG = 89 + \frac{300 \times F - 10 \times L}{P}\end{equation*} dove:
		\begin{itemize}
			\item $F$: numero totale di frasi nel documento;
			\item $L$: numero totale di lettere nel documento;
			\item $P$: numero totale di parole nel documento.
		\end{itemize}
	\item \textbf{Valore ammissibile}: \begin{equation*}IG \geq 50\%\end{equation*}
	\item \textbf{Valore ottimo}: \begin{equation*}IG \geq 75\%\end{equation*}
\end{itemize}

\hypertarget{20M}{\subsubsubsection*{20M-CO Correttezza Ortografica}}
\begin{itemize}
	\item \textbf{Definizione}: la correttezza ortografica indica la presenza di errori ortografici nei documenti.
	\item \textbf{Come si calcola}: si contano gli errori ortografici presenti nei documenti.
	\item \textbf{Valore ammissibile}: \begin{equation*}0\ errori\end{equation*}
	\item \textbf{Valore ottimo}: \begin{equation*}0\ errori\end{equation*}
\end{itemize}


\subsubsection{Gestione della qualità}
La gestione della qualità è un processo che si occupa di definire una metodologia per garantire la qualità del prodotto. Per valutare la qualità di tale processo, sono state definite le seguenti metriche.

\hypertarget{21M}{\subsubsubsection*{21M-FU Facilità di Utilizzo}}
\begin{itemize}
	\item \textbf{Definizione}: rappresenta il livello di usabilità del prodotto software mediante il numero di errori riscontrati durante l'utilizzo del prodotto da parte di un utente generico.
	\item \textbf{Come si calcola}: si contano gli errori riscontrati durante l'utilizzo del prodotto da parte di un utente che non ha conoscenze pregresse sul prodotto software.
	\item \textbf{Valore ammissibile}: \begin{equation*}FU \leq 3\ errori\end{equation*}
	\item \textbf{Valore ottimo}: \begin{equation*}FU = 0\ errori\end{equation*}
\end{itemize}

\hypertarget{22M}{\subsubsubsection*{22M-TA Tempo di Apprendimento}}
\begin{itemize}
	\item \textbf{Definizione}: indica il tempo massimo richiesto da parte di un utente generico per apprendere l'utilizzo del prodotto.
	\item \textbf{Come si calcola}: si misura il tempo necessario per apprendere l'utilizzo del prodotto da parte di un utente che non ha conoscenze pregresse sul prodotto software.
	\item \textbf{Valore ammissibile}: \begin{equation*}TA \leq 12\ minuti\end{equation*}
	\item \textbf{Valore ottimo}: \begin{equation*}TA \leq 7\ minuti\end{equation*}
\end{itemize}

\hypertarget{23M}{\subsubsubsection*{23M-TR Tempo di Risposta}}
\begin{itemize}
	\item \textbf{Definizione}: indica il tempo massimo di risposta del sistema sotto carico rilevato.
	\item \textbf{Come si calcola}: si misura il tempo massimo necessario per ottenere una risposta dal sistema.
	\item \textbf{Valore ammissibile}: \begin{equation*}TR \leq 8\ s\end{equation*}
	\item \textbf{Valore ottimo}: \begin{equation*}TR \leq 4\ s\end{equation*}
\end{itemize}

\hypertarget{24M}{\subsubsubsection*{24M-TE Tempo di Elaborazione}}
\begin{itemize}
	\item \textbf{Definizione}: indica il tempo massimo di elaborazione di un dato grezzo fino alla sua presentazione rilevato.
	\item \textbf{Come si calcola}: si misura il tempo massimo di elaborazione di un dato grezzo dal momento della sua comparsa nel sistema fino alla sua presentazione all'utente.
	\item \textbf{Valore ammissibile}: \begin{equation*}TE \leq 10\ s\end{equation*}
	\item \textbf{Valore ottimo}: \begin{equation*}TE \leq 5\ s\end{equation*}
\end{itemize}

\hypertarget{25M}{\subsubsubsection*{25M-QMS Metriche di Qualità Soddisfatte}}
\begin{itemize}
	\item \textbf{Definizione}: indica il numero di metriche implementate e soddisfatte, tra quelle definite.
	\item \textbf{Come si calcola}: \begin{equation*}QMS = \frac{MS}{MT}\end{equation*} dove:
		\begin{itemize}
			\item $MS$: metriche soddisfatte;
			\item $MT$: metriche totali.
		\end{itemize}
	\item \textbf{Valore ammissibile}: \begin{equation*}QMS \geq 90\%\end{equation*}
	\item \textbf{Valore ottimo}: \begin{equation*}QMS = 100\%\end{equation*}
\end{itemize}

\subsubsection{Verifica}
La verifica è un processo che si occupa di controllare che il prodotto soddisfi i requisiti stabiliti
e sia pienamente funzionante. Per valutare la qualità di tale processo, sono state definite le seguenti metriche.

\hypertarget{26M}{\subsubsubsection*{26M-CC Code Coverage}}
\begin{itemize}
	\item \textbf{Definizione}: la \textit{Code Coverage} indica quale percentuale del codice sorgente è stata eseguita durante i test. Serve per capire quanto del codice è stato verificato dai test automatizzati.
	\item \textbf{Come si calcola}: \begin{equation*}CC = \frac{LE}{LT}\end{equation*} dove:
		\begin{itemize}
			\item $LE$: linee di codice eseguite;
			\item $LT$: linee di codice totali.
		\end{itemize}
	\item \textbf{Valore ammissibile}: \begin{equation*}CC \geq 80\%\end{equation*}
	\item \textbf{Valore ottimo}: \begin{equation*}CC = 100\%\end{equation*}
\end{itemize}

\hypertarget{27M}{\subsubsubsection*{27M-BC Branch Coverage}}
\begin{itemize}
	\item \textbf{Definizione}: la \textit{Branch Coverage} indica quale percentuale dei rami decisionali (percorsi derivanti da istruzioni condizionali come \textit{if}, \textit{for}, \textit{while}) del codice è stata eseguita durante i test.
	\item \textbf{Come si calcola}: \begin{equation*}BC = \frac{BE}{BT}\end{equation*} dove:
		\begin{itemize}
			\item $BE$: rami eseguiti;
			\item $BT$: rami totali.
		\end{itemize}
	\item \textbf{Valore ammissibile}: \begin{equation*}BC \geq 80\%\end{equation*}
	\item \textbf{Valore ottimo}: \begin{equation*}BC = 100\%\end{equation*}
\end{itemize}

\hypertarget{28M}{\subsubsubsection*{28M-SC Statement Coverage}}
\begin{itemize}
	\item \textbf{Definizione}: la \textit{Statement Coverage} indica quale percentuale di istruzioni del codice è stata eseguita durante i test.
	\item \textbf{Come si calcola}: \begin{equation*}SC = \frac{IE}{IT}\end{equation*} dove:
		\begin{itemize}
			\item $IE$: istruzioni eseguite;
			\item $IT$: istruzioni totali.
		\end{itemize}
	\item \textbf{Valore ammissibile}: \begin{equation*}SC \geq 80\%\end{equation*}
	\item \textbf{Valore ottimo}: \begin{equation*}SC = 100\%\end{equation*}
\end{itemize}

\hypertarget{29M}{\subsubsubsection*{29M-FD Failure Density}}
\begin{itemize}
	\item \textbf{Definizione}: la \textit{Failure Density} indica il numero di difetti trovati in un software o in una parte di esso durante il ciclo di sviluppo rispetto alla dimensione del software stesso.
	\item \textbf{Come si calcola}: \begin{equation*}FD = \frac{DF}{LT}\end{equation*} dove:
		\begin{itemize}
			\item $DF$: difetti trovati;
			\item $LT$: linee di codice totali.
		\end{itemize}
	\item \textbf{Valore ammissibile}: \begin{equation*}FD \leq 15\%\end{equation*}
	\item \textbf{Valore ottimo}: \begin{equation*}FD = 0\%\end{equation*}
\end{itemize}

\hypertarget{30M}{\subsubsubsection*{30M-PTCP Passed Test Case Percentage}}
\begin{itemize}
	\item \textbf{Definizione}: la \textit{Passed Test Case Percentage} indica la percentuale di test che sono stati eseguiti con successo su una base di test.
	\item \textbf{Come si calcola}: \begin{equation*}PTCP = \frac{TS}{TT}\end{equation*} dove:
		\begin{itemize}
			\item $TS$: test superati;
			\item $TT$: test totali.
		\end{itemize}
	\item \textbf{Valore ammissibile}: \begin{equation*}PTCP \geq 90\%\end{equation*}
	\item \textbf{Valore ottimo}: \begin{equation*}PTCP = 100\%\end{equation*}
\end{itemize}



\subsubsection{Risoluzione dei problemi}
La risoluzione dei problemi è un processo che mira a identificare, analizzare e risolvere le varie problematiche che possono emergere durante lo sviluppo. La gestione dei rischi, in particolare, si occupa di identificare, analizzare e gestire i rischi che possono insorgere durante lo svolgimento del progetto. Per valutare la qualità di tale processo, sono state definite le seguenti metriche.

\hypertarget{31M}{\subsubsubsection*{31M-RMR Risk Mitigation Rate}}
\begin{itemize}
	\item \textbf{Definizione}: la \textit{Risk Mitigation Rate} indica la percentuale di rischi identificati che sono stati mitigati con successo.
	\item \textbf{Come si calcola}: \begin{equation*}RMR = \frac{RM}{RT}\end{equation*} dove:
		\begin{itemize}
			\item $RM$: rischi mitigati;
			\item $RT$: rischi totali identificati.
		\end{itemize}
	\item \textbf{Valore ammissibile}: \begin{equation*}RMR \geq 80\%\end{equation*}
	\item \textbf{Valore ottimo}: \begin{equation*}RMR = 100\%\end{equation*}
\end{itemize}

\hypertarget{32M}{\subsubsubsection*{32M-NCR Rischi Non Calcolati}}
\begin{itemize}
	\item \textbf{Definizione}: indica il numero di rischi occorsi che non sono stati preventivati durante l'analisi dei rischi.
	\item \textbf{Come si calcola}: si contano i rischi occorsi e non preventivati.
	\item \textbf{Valore ammissibile}: \begin{equation*}NCR \leq 3\end{equation*}
	\item \textbf{Valore ottimo}: \begin{equation*}NCR = 0\end{equation*}
\end{itemize}




\subsection{Processi organizzativi}
I processi organizzativi sono processi che si occupano di definire le linee guida e le procedure da seguire per garantire un'efficace gestione e coordinazione del progetto.

\subsubsection{Pianificazione}
La pianificazione è un processo che si occupa di definire le attività da svolgere e le risorse temporali e umane necessarie per il loro svolgimento. Per valutare la qualità di tale processo, sono state definite le seguenti metriche.

\hypertarget{33M}{\subsubsubsection*{33M-RSI Requirements Stability Index}}
\begin{itemize}
	\item \textbf{Definizione}: il \textit{Requirements Stability Index} (RSI) indica la percentuale di requisiti che sono rimasti invariati rispetto al totale dei requisiti inizialmente definiti. Si tratta di una metrica utilizzata per misurare quanto i requisiti di un progetto rimangono stabili durante il ciclo di vita del progetto stesso, è particolarmente utile per comprendere l'impatto delle modifiche ai requisiti sul progetto.
	\item \textbf{Come si calcola}: \begin{equation*}RSI = \frac{RI - (RA + RR + RC)}{RI}\end{equation*} dove:
		\begin{itemize}
			\item $RI$: requisiti iniziali;
			\item $RA$: requisiti aggiunti;
			\item $RR$: requisiti rimossi;
			\item $RC$: requisiti cambiati.
		\end{itemize}
	\item \textbf{Valore ammissibile}: \begin{equation*}RSI \geq 75\%\end{equation*}
	\item \textbf{Valore ottimo}: \begin{equation*}RSI = 100\%\end{equation*}
\end{itemize}

