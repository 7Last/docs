\section{Standard per la qualità}
Abbiamo scelto di adottare standard internazionali per l'analisi e la valutazione della qualità dei processi e del software. In particolare, la suddivisione dei processi in primari, di supporto e organizzativi sarà guidata dall'adozione dello standard \textit{ISO/IEC 12207:1995}. Inoltre, l'adozione dello standard \textit{ISO/IEC 25010:2023} ci fornirà un quadro completo per la definizione e la classificazione delle metriche di qualità del software. Abbiamo inoltre deciso di applicare solo questi due standard, poiché lo standard \textit{ISO/IEC 9126:2001} è stato ritirato e sostituito dallo standard \textit{ISO/IEC 25010:2023}.
\subsection{Caratteristiche del sistema ISO/IEC 25010:2023}
\subsubsection{Appropriatezza funzionale}
\begin{itemize}
    \item \textbf{Completezza}: il prodotto software deve soddisfare tutti i requisiti definiti e attesi dagli utenti.
    \item \textbf{Correttezza}: il prodotto software deve funzionare come previsto e produrre risultati accurati.
    \item \textbf{Appropriatezza}: il prodotto software deve essere adatto allo scopo previsto e al contesto di utilizzo.
\end{itemize}
\subsubsection{Performance}
\begin{itemize}
    \item \textbf{Tempo}: il prodotto software deve rispettare le scadenze e i tempi di consegna previsti.
    \item \textbf{Risorse}: il prodotto software deve utilizzare le risorse di sistema in modo efficiente e ragionevole.
    \item \textbf{Capacità}: il prodotto software deve essere in grado di gestire il carico di lavoro previsto.
\end{itemize}
\subsubsection{Compatibilità}
\begin{itemize}
    \item \textbf{Coesistenza}: il prodotto software deve essere in grado di coesistere con altri software e sistemi sul computer.
    \item \textbf{Interoperabilità}: il prodotto software deve essere in grado di scambiare informazioni e collaborare con altri software e sistemi.
\end{itemize}
\subsubsection{Usabilità}
\begin{itemize}
    \item \textbf{Riconoscibilità}: il prodotto software deve avere un'interfaccia utente intuitiva e facile da usare.
    \item \textbf{Apprendibilità}: gli utenti devono essere in grado di imparare a utilizzare il prodotto software in modo rapido e semplice.
    \item \textbf{Operabilità}: il prodotto software deve essere facile da usare e da controllare.
    \item \textbf{Protezione} errori: il prodotto software deve essere in grado di rilevare e gestire gli errori in modo efficace.
    \item \textbf{Esteticità}: il prodotto software deve avere un'interfaccia utente piacevole e accattivante.
    \item \textbf{Accessibilità}: il prodotto software deve essere accessibile a persone con disabilità.
\end{itemize}
\subsubsection{Affidabilità}
\begin{itemize}
    \item \textbf{Maturità}: il prodotto software deve essere stabile, affidabile e robusto.
    \item \textbf{Disponibilità}: il prodotto software deve essere disponibile quando necessario.
    \item \textbf{Tolleranza}: il prodotto software deve essere in grado di tollerare errori e condizioni inaspettate.
    \item \textbf{Ricoverabilità}: il prodotto software deve essere in grado di ripristinare i dati e le funzionalità in caso di guasto o errore.
\end{itemize}
\subsubsection{Sicurezza}
\begin{itemize}
    \item \textbf{Riservatezza}: il prodotto software deve proteggere i dati sensibili e le informazioni riservate.
    \item \textbf{Integrità}: il prodotto software deve garantire l'accuratezza e la completezza dei dati.
    \item \textbf{Non ripudio}: il prodotto software deve garantire che le transazioni e le comunicazioni non possano essere negate o ripudiate.
    \item \textbf{Autenticazione}: il prodotto software deve verificare l'identità degli utenti e garantire che solo gli utenti autorizzati possano accedere al sistema.
    \item \textbf{Autenticità}: il prodotto software deve garantire che l'origine dei dati e delle informazioni sia verificabile.
\end{itemize}
\subsubsection{Manutenibilità}
\begin{itemize}
    \item \textbf{Modularità}: il prodotto software deve essere progettato in modo modulare, con componenti indipendenti e ben definiti.
    \item \textbf{Riusabilità}: i componenti del prodotto software devono essere progettati per essere riutilizzati in altri progetti.
    \item \textbf{Analizzabilità}: il prodotto software deve essere progettato in modo da essere facilmente analizzabile e comprensibile.
    \item \textbf{Modificabilità}: il prodotto software deve essere progettato in modo da essere facilmente modificabile e adattabile.
    \item \textbf{Testabilità}: il prodotto software deve essere progettato in modo da essere facilmente testabile.
\end{itemize}
\subsubsection{Portabilità}
\begin{itemize}
    \item \textbf{Adattabilità}: il prodotto software deve essere in grado di adattarsi a nuovi ambienti, requisiti e tecnologie.
    \item \textbf{Installabilità}: il prodotto software deve essere facilmente installabile e configurabile.
    \item \textbf{Sostituibilità}: il prodotto software deve essere facilmente sostituibile con altre soluzioni o versioni più recenti.
\end{itemize}
\newpage
\subsection{Suddivisione secondo standard ISO/IEC 12207:1995}
\subsubsection{Processi primari}
Essenziali per lo sviluppo del software e comprendono:
\begin{itemize}
    \item \textbf{acquisizione}: include tutte le attività necessarie per ottenere un prodotto software o un servizio e comprende la preparazione delle richieste di offerta, la selezione del fornitore e la gestione del contratto;
    \item \textbf{fornitura}: include le attività eseguite per fornire un prodotto software o un servizio al cliente e include la preparazione della proposta, la negoziazione del contratto e la consegna del prodotto;
    \item \textbf{sviluppo}: copre tutte le attività di progettazione e sviluppo del software, dall'\href{https://7last.github.io/docs/pb/documentazione-interna/glossario\#analisi-dei-requisiti}{analisi dei requisiti\textsubscript{G}} alla progettazione, implementazione, test e integrazione;
    \item \textbf{operatività}: riguarda le attività necessarie per utilizzare il sistema software in ambiente operativo e include la gestione delle operazioni quotidiane, il monitoraggio delle prestazioni, la gestione degli utenti, il backup e il ripristino dei dati, nonché il mantenimento della continuità operativa del sistema;
    \item \textbf{manutenzione}: include le attività necessarie per modificare il software dopo la sua consegna, per correggere difetti o migliorare prestazioni;
\end{itemize}
\subsubsection{Processi di supporto}
Forniscono il supporto necessario per i processi primari e comprendono:
\begin{itemize}
    \item \textbf{documentazione}: gestione della documentazione necessaria per supportare le attività del ciclo di vita del software;
    \item \textbf{gestione della configurazione}: controllo delle modifiche al software per mantenere l'integrità e la tracciabilità delle configurazioni del software;
    \item \textbf{assicurazione della qualità}: garantisce che i prodotti e i processi del software soddisfino gli standard richiesti, include attività come la pianificazione della qualità, le revisioni di qualità, le ispezioni e le valutazioni per assicurare la conformità agli standard stabiliti;
    \item \textbf{verifica}: assicura che i prodotti del software soddisfino i requisiti specificati; 
    \item \textbf{validazione}: assicura che il prodotto software soddisfi le esigenze e le aspettative dell'utente;
    \item \textbf{revisioni congiunte con il cliente}: fornisce una revisione periodica dei risultati del progetto con tutte le parti interessate;
    \item \textbf{audit}: verifica che i processi e i prodotti del software siano conformi ai requisiti, agli standard e alle procedure;
    \item \textbf{risoluzione dei problemi}: gestione e risoluzione dei problemi identificati nel software;
\end{itemize}
\subsubsection{Processi organizzativi}
Supportano l'organizzazione nel suo insieme e si compongono di:
\begin{itemize}
    \item \textbf{gestione}: pianificazione, monitoraggio e controllo delle attività del progetto;
    \item \textbf{infrastrutture}: gestione dell'infrastruttura necessaria per supportare i processi di ciclo di vita del software;
    \item \textbf{miglioramento}: miglioramento continuo dei processi di ciclo di vita del software;
    \item \textbf{formazione}: pianificazione e fornitura della formazione necessaria per il personale coinvolto nei processi di ciclo di vita del software.
\end{itemize}