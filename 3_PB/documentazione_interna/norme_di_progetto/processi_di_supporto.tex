\section{Processi di supporto}
\subsection{Documentazione}
\subsubsection{Introduzione}
Il processo di documentazione è una componente fondamentale nella realizzazione e nel rilascio di un prodotto software,
poiché fornisce informazioni utili alle parti coinvolte e tiene traccia di tutte le attività relative al ciclo di vita del software,
comprese scelte e norme adottate dal gruppo durante lo svolgimento del progetto. In particolare, la documentazione è utile per:
\begin{itemize}
	\item permettere una comprensione profonda del prodotto e delle sue funzionalità;
	\item garantire uno standard di qualità all'interno dei processi produttivi.
\end{itemize}
Lo scopo della sezione è:
\begin{itemize}
	\item definire un insieme di regole e convenzioni per garantire la coerenza e la qualità della documentazione prodotta;
	\item creare template per ogni tipologia di documento così da garantire omogeneità.
\end{itemize}

\subsubsection{Versionamento}
La documentazione viene versionata allo stesso modo del codice sorgente, utilizzando la piattaforma \href{https://7last.github.io/docs/pb/documentazione-interna/glossario\#github}{Github\textsubscript{G}}.
Ciò permette di tenere traccia delle modifiche apportate ai documenti e di realizzare procedure automatiche per l'individuazione degli errori grammaticali,
inserimento della "G" a pedice per i termini presenti nel \href{https://7last.github.io/docs/pb/documentazione-interna/glossario\#glossario}{glossario\textsubscript{G}} e la pubblicazione nel sito del progetto.\\
Abbiamo stabilito di utilizzare il branch \texttt{main} per la pubblicazione dei documenti approvati dal \href{https://7last.github.io/docs/pb/documentazione-interna/glossario\#verificatore}{verificatore\textsubscript{G}} e il branch \texttt{develop} per contenere
i file in formato LaTex da cui derivano i documenti finali.

\subsubsubsection{Branch policy}
Sono state definite le seguenti regole per la gestione dei branch:
\begin{itemize}
	\item \texttt{main}: viene aggiornato solamente dalla \href{https://7last.github.io/docs/pb/documentazione-interna/glossario\#github}{\textit{Github\textsubscript{G}} Action} che si occupa di pubblicare i documenti approvati presenti nel branch \texttt{develop};
	\item \texttt{develop}: per la pubblicazione in questo branch è necessario ricevere un'approvazione da parte del \href{https://7last.github.io/docs/pb/documentazione-interna/glossario\#verificatore}{verificatore\textsubscript{G}} tramite \textit{pull request}.
\end{itemize}

\subsubsection{Tipografia e sorgente documenti}
Per la redazione dei documenti abbiamo deciso di utilizzare il linguaggio di markup LaTeX, grazie alla sua flessibilità e possibilità di creare documenti di alta qualità tipografica.
Inoltre, abbiamo stabilito di suddividere ciascun documento in file differenti, in modo da consentire ai membri del team di lavorare contemporaneamente su sezioni diverse,
riducendo il numero di conflitti e semplificando la gestione del versionamento.

\subsubsection{Ciclo di vita}
Ciascun documento segue un ciclo di vita ben definito, che prevede le seguenti fasi:
\begin{enumerate}
	\item \textbf{Creazione branch} a partire da \texttt{develop}, utilizzando \href{https://www.atlassian.com/git/tutorials/comparing-workflows/gitflow-workflow}{\underline{Gitflow}} [Ultima consultazione 2024-05-13];
	\item \textbf{Scelta del template} da utilizzare per il documento, scegliendo tra \textit{documento}, \textit{verbale interno} o \textit{verbale esterno};
	\item \textbf{Redazione} del documento, seguendo le norme tipografiche e le convenzioni stabilite;
	\item \textbf{Commit} delle modifiche effettuate;
	\item \textbf{Apertura \textit{pull request}} per la revisione del documento, che dovrà essere approvata dal \href{https://7last.github.io/docs/pb/documentazione-interna/glossario\#verificatore}{verificatore\textsubscript{G}} e dal \href{https://7last.github.io/docs/pb/documentazione-interna/glossario\#responsabile}{responsabile\textsubscript{G}} per lo \href{https://7last.github.io/docs/pb/documentazione-interna/glossario\#sprint}{sprint\textsubscript{G}} in corso;
	\item \textbf{Chiusura \textit{pull request}} e \textbf{merge} del branch;
	\item \textbf{Completamento task} su \href{https://7last.github.io/docs/pb/documentazione-interna/glossario\#clickup}{\textit{ClickUp}\textsubscript{G}}.
\end{enumerate}

\subsubsection{Convenzioni nomenclatura e struttura di archiviazione}
Tutti i documenti prodotti devono seguire la seguente convenzione di nomenclatura ben definita, in modo da garantire una corretta organizzazione e archiviazione dei file
\begin{center}
	\texttt{nome\_fase/tipo\_documento/nome\_documento/nome\_documento.tex}
\end{center}
Dove:
\begin{itemize}
	\item \textbf{nome\_fase}: rappresenta la fase di progetto in cui il documento è stato prodotto, può assumere i seguenti valori:
	      \begin{itemize}
		      \item \textbf{1\_candidatura};
		      \item \textbf{2\_RTB};
		      \item \textbf{3\_PB};
		      \item \textbf{4\_CA}.
	      \end{itemize}
	\item \textbf{tipo\_documento}: rappresenta la tipologia di documento, può assumere i seguenti valori:
	      \begin{itemize}
		      \item \textbf{documentazione\_interna};
		      \item \textbf{documentazione\_esterna};
		      \item \textbf{verbali\_interni};
		      \item \textbf{verbali\_esterni}.
	      \end{itemize}
	\item \textbf{nome\_documento}: rappresenta il nome del documento, scritto in snake case.
\end{itemize}

\subsubsection{Procedure correlate alla redazione di documenti}
\subsubsubsection{I redattori}
Il redattore si occupa della stesura del documento, seguendo le norme tipografiche e le convenzioni stabilite, in modo da garantire la
chiarezza e la coerenza del testo. Dovrà rispettare le convenzioni sopracitate.

\subsubsubsection{Il resposabile}
Il \href{https://7last.github.io/docs/pb/documentazione-interna/glossario\#responsabile}{responsabile\textsubscript{G}} coordina e gestisce le attività del gruppo. In particolare:
\begin{itemize}
	\item definire quali documenti devono essere prodotti e quali compiti devono essere svolti;
	\item gestisce le risorse e gli strumenti necessari per ultimare tali attività;
	\item approva la versione finale dei documenti;
	\item comunica con gli \href{https://7last.github.io/docs/pb/documentazione-interna/glossario\#stakeholder}{stakeholder\textsubscript{G}}.
\end{itemize}

\subsubsubsection{L'amministratore}
L’\href{https://7last.github.io/docs/pb/documentazione-interna/glossario\#amministratore}{amministratore\textsubscript{G}} è colui che si occupa della configurazione degli strumenti di supporto alla produzione del software; nel nostro caso,
inserisce le attività nell'ITS \href{https://7last.github.io/docs/pb/documentazione-interna/glossario\#clickup}{ClickUp\textsubscript{G}} e le assegna ai membri del gruppo.

\subsubsubsection{I verificatori}
Il compito del \href{https://7last.github.io/docs/pb/documentazione-interna/glossario\#verificatore}{verificatore\textsubscript{G}} è di controllare la correttezza e la qualità dei documenti prodotti, assicurandosi che siano conformi agli standard e alle norme stabilite.

\subsubsection{Struttura del documento}
Tutti i documenti produtti devono seguire una struttura ben definita, che prevede le componenti in seguito elencate.
\subsubsubsection{Prima pagina}
Nella prima pagina di ogni documento è presente un'intestazione contenente le seguenti informazioni:
\begin{itemize}
	\item titolo del documento;
	\item versione del documento;
	\item logo del gruppo;
	\item nome del gruppo.
\end{itemize}

\subsubsubsection{Registro delle modifiche}
La seconda pagina è dedicata al registro delle modifiche, rappresentato mediante una tabella. Il suo scopo è quello di tenere traccia di tutti i cambiamenti effettuati al documento,
in modo da garantire la rintracciabilità e la trasparenza delle modifiche. Ogni riga della tabella corrisponde a un cambiamento, e contiene le seguenti informazioni:
\begin{itemize}
	\item \textbf{versione} del documento, nel formato vX.Y, dove X rappresenta il numero di versione principale e Y il numero di versione secondaria;
	\item \textbf{data} di ultima modifica;
	\item nome del \textbf{redattore};
	\item nome del \href{https://7last.github.io/docs/pb/documentazione-interna/glossario\#verificatore}{\textbf{verificatore}\textsubscript{G}};
	\item \textbf{descrizione} della modifica.
\end{itemize}

\subsubsubsection{Indice}
Ogni documento contiene un indice delle sezioni e delle sottosezioni presenti al suo interno, in modo da facilitare la consultazione e la navigazione.
Tutte le figure e tabelle presenti nei documenti sono visibili negli appositi indici e sono direttamente accessibili tramite link ipertestuali.

\subsubsubsection{Intestazione}
Ogni pagina del documento, ad eccezione della prima, include un'intestazione che presenta, da sinistra a destra, il logo del gruppo, il titolo del documento e la versione.

\subsubsubsection{Struttura dei verbali}
I verbali rappresentano un resoconto dettagliato degli incontri, tracciando gli argomenti discussi, le decisioni prese e le azioni da intraprendere.
Sono suddivisi in verbali interni ed esterni, a seconda che il meeting coinvolga solo i membri del team o persone esterne al gruppo.
La struttura è la medesima per entrambe le tipologie, ad eccezione dell'ultima pagina.
\begin{itemize}
	\item \textbf{prima pagina}:
	      \begin{itemize}
		      \item tipologia verbale (interno / esterno);
		      \item titolo del documento;
		      \item sottotitolo del documento, "Riunione interna settimanale" per i verbali interni e "Colloquio con \textit{Sync Lab}" per i verbali esterni;
		      \item data e versione del documento;
		      \item logo del gruppo
		      \item nome del gruppo.
	      \end{itemize}
	\item \textbf{Registro delle modifiche};
	\item \textbf{Dettagli sulla riunione}:
	      \begin{itemize}
		      \item sede della riunione (piattaforma utilizzata);
		      \item orario di inizio e fine;
		      \item partecipanti del gruppo (interni) con specificazione del ruolo;
		      \item partecipanti esterni.
	      \end{itemize}
	\item \textbf{Corpo del documento}
	      All'interno del corpo del documento sono presenti le seguenti sezioni:
	      \begin{itemize}
		      \item \textbf{ordine del giorno}: elenco di ciò che verrà discusso durante la riunione;
		      \item \textbf{verbale}: resoconto dettagliato dell'incontro, con una sottosezione per ciascun punto dell'ordine del giorno,
		            gli obiettivi prefissati e le decisioni prese;
	      \end{itemize}
	\item \textbf{Ultima pagina}: solo in caso di verbale esterno, l'ultima pagina contiene data e firma di un componente esterno, presente durante l'incontro.
\end{itemize}
Il template dei verbali è disponibile nella repository del gruppo, più precisamente al seguente \href{https://github.com/7Last/docs/tree/develop/0_template}{\underline{link}}. (Ultima consultazione 2024-05-20).
\newpage
\subsubsection{Norme tipografiche}
\textbf{Nomi dei file}\\ I nomi dei documenti devono essere coerenti con il loro contenuto,
scritti in snake case. Nel branch \texttt{develop} tutti i documenti non contengono la versione nel nome, al fine di evitare conflitti durante la modifica.
Nel momento in cui il documento viene approvato, esso viene spostato nel branch \texttt{main} e la versione viene aggiunta automaticamente al nome del file.\\
In particolare, la denominazione dei file nel branch \texttt{main} deve seguire la seguente convenzione:
\begin{itemize}
	\item \textbf{Verbali}: verbale\_esterno/interno\_YYYY\_MM\_DD\_vA.B;
	\item \href{https://7last.github.io/docs/pb/documentazione-interna/glossario\#norme-di-progetto}{\textbf{Norme di Progetto}\textsubscript{G}}: norme\_di\_progetto\_vA.B;
	\item \href{https://7last.github.io/docs/pb/documentazione-interna/glossario\#analisi-dei-requisiti}{\textbf{Analisi dei Requisiti}\textsubscript{G}}: analisi\_dei\_requisiti\_vA.B;
	\item \href{https://7last.github.io/docs/pb/documentazione-interna/glossario\#piano-di-progetto}{\textbf{Piano di Progetto}\textsubscript{G}}: piano\_di\_progetto\_vA.B;
	\item \textbf{Lettera di Presentazione}: lettera\_di\_presentazione;
	\item \href{https://7last.github.io/docs/pb/documentazione-interna/glossario\#glossario}{\textbf{Glossario}\textsubscript{G}}: \href{https://7last.github.io/docs/pb/documentazione-interna/glossario\#glossario}{glossario\textsubscript{G}}\_vA.B.
\end{itemize}
\textbf{Stile del testo}
\begin{itemize}
	\item \textbf{Grassetto}:
	      \begin{itemize}
		      \item titoli di sezione;
		      \item termini importanti;
		      \item parole seguite da descrizione o elenchi puntati.
	      \end{itemize}
	\item \textbf{Corsivo}:
	      \begin{itemize}
		      \item nome del gruppo e dell'azienda \href{https://7last.github.io/docs/pb/documentazione-interna/glossario\#proponente}{proponente\textsubscript{G}};
		      \item riferimenti a documenti esterni.
	      \end{itemize}
	\item \textbf{Maiuscolo}:
	      \begin{itemize}
		      \item acronimi;
		      \item iniziali dei nomi;
	      \end{itemize}
\end{itemize}
\newpage
\textbf{Regole sintattiche}:
\begin{itemize}
	\item negli elenchi ogni voce deve terminare con ";" mentre l'ultima ".";
	\item i numeri razionali vengono scritti mediante l'uso della virgola come separatore tra parte intera e parte decimale;
	\item le date devono seguire lo standard internazionale ISO 8601, rappresentando la data con YYYY-MM-DD (anno, mese, giorno).
	\item le sezioni di un documento devono essere numerate in modo gerarchico, seguendo la struttura X.Y, dove:
	      \begin{itemize}
		      \item X rappresenta il numero della sezione principale;
		      \item Y rappresenta il numero della sottosezione;
	      \end{itemize}
\end{itemize}

\subsubsection{Metriche}

\begin{table}[!h]
	\centering
	\begin{tabular}{|c|l|}
		\hline
		\textbf{Codice} & \textbf{Nome esteso}               					\\
		\hline
		\underline{\hyperlink{19M}{19M-IG}}     & Indice di Gulpease   			\\
		\underline{\hyperlink{20M}{20M-CO}}     & Correttezza Ortografica  		\\
		\hline
	\end{tabular}
	\caption{Metriche inerenti il processo di documentazione}
\end{table}

\subsubsection{Strumenti}
Gli strumenti utilizzati per la redazione dei documenti sono:
\begin{itemize}
	\item \textbf{LaTeX}: impiegato nella stesura dei documenti;
	\item \href{https://7last.github.io/docs/pb/documentazione-interna/glossario\#github}{\textbf{GitHub}\textsubscript{G}}: utilizzato per la gestione delle versioni e per la condivisione dei documenti;
	\item \textbf{Visual Studio Code}: utilizzato come editor di testo per la scrittura dei documenti.
\end{itemize}



\subsection{Accertamento della qualità}
\subsubsection{Introduzione}
L'accertamento della qualità consiste in un insieme di attività e processi volti a garantire che il software sviluppato soddisfi gli standard di qualitativi richiesti. Questo include l'adesione a requisiti concordati con la \href{https://7last.github.io/docs/pb/documentazione-interna/glossario\#proponente}{proponente\textsubscript{G}}, normative ed è cruciale per assicurare che il prodotto finale sia affidabile e performante.

\subsubsection{Attività}
Il processo di accertamento della qualità comprende le seguenti attività:
\begin{enumerate}
	\item \textbf{sviluppo del \href{https://7last.github.io/docs/pb/documentazione-interna/glossario\#piano-di-qualifica}{piano di qualifica\textsubscript{G}}}: vengono definiti gli standard di qualità e le metriche che il progetto deve raggiungere;
	\item \textbf{monitoraggio continuo della qualità}: consiste in attività sistematiche e pianificate per garantire che i processi utilizzati durante lo sviluppo del software siano adeguati e seguiti correttamente;
	\item \textbf{gestione delle configurazioni}: attraverso un sistema di gestione delle configurazioni ci si assicura che ciascuna di esse venga sottoposta ad opportuni controlli e valutazioni;
	\item \textbf{reporting}: in questa fase vengono generati report che descrivono lo stato della qualità del progetto, inclusi i risultati dei test, i difetti trovati e le azioni correttive intraprese;
	\item \textbf{miglioramento continuo}: vengono applicate le modifiche ai processi basate sui feedback e sulle analisi per migliorare continuamente la qualità del software e dei processi di sviluppo;
	\item \textbf{confronto con la \href{https://7last.github.io/docs/pb/documentazione-interna/glossario\#proponente}{proponente\textsubscript{G}}}: viene instaurata una comunicazione con la \href{https://7last.github.io/docs/pb/documentazione-interna/glossario\#proponente}{proponente\textsubscript{G}} per ricevere feedback costante sul lavoro svolto.
\end{enumerate}

\subsubsection{Piano di qualifica}
Il documento \href{https://7last.github.io/docs/pb/documentazione-interna/glossario\#piano-di-qualifica}{\textit{Piano di Qualifica}\textsubscript{G}} è redatto per garantire che il software sviluppato rispetti gli standard di qualità richiesti e soddisfi le aspettative degli \href{https://7last.github.io/docs/pb/documentazione-interna/glossario\#stakeholder}{stakeholder\textsubscript{G}}. Il suo utilizzo e il suo scopo si estendono a diverse aree critiche del progetto, dalle fasi iniziali di pianificazione fino alla consegna finale del prodotto.

\subsubsection{Ciclo di Deming}
Il \textit{ciclo di Deming}, è un ciclo a 4 stadi che consente di apportare determinate migliorie a specifici processi. È composto da:
\begin{itemize}
	\item \textbf{Plan}: in cui si definiscono le attività, scadenze, responsabilità e risorse per raggiungere gli obiettivi di miglioramento;
	\item \textbf{Do}: esecuzione delle attività definite dal punto precedente;
	\item \textbf{Check}: verifica l'esito delle azioni di miglioramento rispetto alle attese;
	\item \textbf{Act}: consolidare il buono e cercare modi per migliorare il resto.
\end{itemize}

È importante specificare che questo ciclo \textbf{non opera} sul prodotto, bensì sul \textbf{\textit{way of working}} del gruppo, per migliorarlo e renderlo più efficiente.

\subsubsection{Struttura e identificazioni metriche}
Ogni metrica presenta la seguente struttura:
\begin{itemize}
	\item \textbf{Metrica}:
	      codice identificativo nel formato:
	      \begin{center}
		      \textbf{[numero]M-[acronimo]}
	      \end{center}
	      Dove:
	      \begin{itemize}
		      \item \textbf{[numero]}: numero progressivo univoco per ogni metrica;
		      \item \textbf{M}: metrica;
		      \item \textbf{[acronimo]}: abbreviazione del nome della metrica.
	      \end{itemize}
	\item \textbf{Nome}: nome della metrica;
	\item \textbf{Valore accettabile}: valore minimo affinché la metrica sia considerabile soddisfacente e conforme agli obiettivi di qualità;
	\item \textbf{Valore ammissibile}: valore ottimale e ideale che dovrebbe essere raggiunto dalla metrica;
	\item \textbf{Valore ottimo}:
	\item \textbf{Descrizione}: breve descrizione della metrica adottata e delle sue funzionalità;
\end{itemize}

\subsubsection{Metriche}
\begin{table}[!h]
	\centering
	\begin{tabular}{ | c | l | }
		\hline
		\textbf{Codice}                      & \textbf{Nome esteso}            \\
		\hline
		\underline{\hyperlink{21M}{21M-FU}}  & Facilità di utilizzo            \\
        \underline{\hyperlink{22M}{22M-TA}}  & Tempo di apprendimento          \\
        \underline{\hyperlink{23M}{23M-TR}}  & Tempo di risposta               \\
        \underline{\hyperlink{24M}{24M-TE}}  & Tempo di elaborazione           \\
        \underline{\hyperlink{25M}{25M-QMS}} & Metriche di qualità soddisfatte \\
		\hline
	\end{tabular}
	\caption{Metriche relative all'accertamento della qualità}
\end{table}

\subsection{Verifica}
\subsubsection{Introduzione}
La verifica nel ciclo di vita del software garantisce l'efficienza e la correttezza delle attività, assicurando che i prodotti soddisfino i requisiti. I \href{https://7last.github.io/docs/pb/documentazione-interna/glossario\#verificatore}{verificatori\textsubscript{G}} applicano tecniche di test seguendo procedure definite all'interno del \href{https://7last.github.io/docs/pb/documentazione-interna/glossario\#piano-di-qualifica}{Piano di Qualifica\textsubscript{G}}, il quale traccia il percorso di verifica, stabilendo obiettivi e criteri di accettazione. Tutti i processi attivi vengono sottoposti a verifica dopo aver raggiunto un buon grado di completamento o a seguito di modifiche dello stato del processo stesso.

\subsubsection{Verifica dei documenti} \label{verifica_dei_documenti}
Nell'ambito della documentazione, la verifica è un'attività cruciale per garantire la correttezza e l'accuratezza dei contenuti. Questo processo coinvolge una serie di controlli sistematici e revisioni finalizzate a garantire che la documentazione sia accurata, completa e conforme agli standard e ai requisiti specificati.
La verifica dei documenti assicura l'accuratezza, identificando e correggendo errori, omissioni e incongruenze. Questo riduce il rischio di malintesi o errori di implementazione, garantendo che tutte le informazioni tecniche e di progetto siano corrette e precise.
Inoltre, la verifica garantisce che la documentazione rispetti gli standard aziendali, di settore e normativi, mantenendo la coerenza e la qualità attraverso tutti i documenti del progetto. Documenti verificati e accurati migliorano la comunicazione tra i membri del team e con gli \href{https://7last.github.io/docs/pb/documentazione-interna/glossario\#stakeholder}{stakeholder\textsubscript{G}} esterni, facilitando la comprensione e la collaborazione e riducendo i tempi di chiarimento e discussione. Essa si suddivide in:
\begin{itemize}
	\item \textbf{correttezza tecnica}: vengono controllate le informazioni tecniche e i contenuti del documento, verificando che siano corretti e precisi;
	\item \textbf{conformità agli standard}: verifica che il documento segua le linee guida e gli standard stabiliti per la formattazione, la struttura e lo stile;
	\item \textbf{verifica della completezza}: si effettua un controllo per assicurare che tutti i requisiti documentali siano stati soddisfatti e che  tutte le sezioni e le informazioni necessarie siano presenti;
	\item \textbf{revisione linguistica}: viene effettuata una correzione di eventuali errori grammaticali, ortografici e di punteggiatura con l'ausilio di controlli automatici;
	\item \textbf{coerenza}: ci si assicura che i termini e le definizioni siano utilizzati in modo uniforme e coerente in tutto il documento.
\end{itemize}

\subsubsubsection{Processo di revisione}
Al momento di completamento delle modifiche in un documento il redattore è tenuto a creare una pull request dal branch in cui ha effettuato le modifiche verso il branch \texttt{develop}, richiedendo una revisione da parte del \href{https://7last.github.io/docs/pb/documentazione-interna/glossario\#verificatore}{verificatore\textsubscript{G}} e del \href{https://7last.github.io/docs/pb/documentazione-interna/glossario\#responsabile}{responsabile\textsubscript{G}}; in caso di approvazione verrà effettuato il merge e la issue creata su \href{https://7last.github.io/docs/pb/documentazione-interna/glossario\#clickup}{ClickUp\textsubscript{G}} verrà spostata da \textit{in progres} a \textit{done}.

\subsubsection{Analisi}
L'analisi è un processo essenziale che mira a garantire la qualità e l'affidabilità del prodotto finale. Questa fase critica comporta un esame dettagliato e sistematico di tutte le componenti del progetto per identificare e risolvere eventuali problemi prima della consegna. Si suddivide in \textbf{analisi dinamica} e \textbf{analisi statica}.

\subsubsubsection{Analisi dinamica}
L'analisi dinamica riguarda l'esame del comportamento del software durante la sua esecuzione. Questo tipo di analisi viene eseguito a tempo di esecuzione e si concentra sull'osservazione di come il software interagisce con il sistema operativo, le risorse di sistema e altre applicazioni. L'analisi dinamica è essenziale per valutare l'efficienza, le prestazioni e la sicurezza del software in condizioni operative reali; viene effettuata tramite l'utilizzo delle seguenti tipologie di test:
\begin{itemize}
	\item test di unità;
	\item test di integrazione;
	\item test di sistema;
	\item test di regressione;
	\item test di accettazione.
\end{itemize}

I benefici dell'analisi dinamica includono:
\begin{itemize}
	\item \textbf{individuazione di errori e difetti}: consente di identificare problemi e difetti nel software durante l'esecuzione, riducendo i rischi di malfunzionamenti e guasti;
	\item \textbf{ottimizzazione delle prestazioni}: permette di valutare le prestazioni del software e di identificare possibili aree di miglioramento;
	\item \textbf{convalida delle funzionalità}: assicura che il software funzioni correttamente e soddisfi i requisiti operativi attraverso i test elencati in precedenza.
\end{itemize}

\hypertarget{testing}{\subsubsection{Testing}}
Il testing è una fase fondamentale dello sviluppo del software, il cui scopo primario è garantire che il prodotto finale sia di alta qualità, soddisfi i requisiti specificati e funzioni correttamente in tutte le condizioni previste. Questo processo si articola in una serie di attività sistematiche e metodiche per identificare difetti, errori e incongruenze nel software, migliorandone così l'affidabilità e le prestazioni complessive.

Per ogni test è necessario definire i seguenti aspetti:
\begin{itemize}
	\item \textbf{identificazione del test}: un identificatore univoco per il caso di test, che faciliti il tracciamento e la referenza durante il processo di testing;
	\item \textbf{descrizione del test}: una descrizione chiara e concisa del caso di test, che spieghi la funzionalità specifica o il requisito che viene testato;
	\item \textbf{input del test}: dettagli sui dati di input necessari per eseguire il test, inclusi eventuali prerequisiti o condizioni iniziali;
	\item \textbf{output atteso}: descrizione degli output o dei risultati attesi del test, inclusi valori attesi, comportamenti del sistema o condizioni post-esecuzione;
	\item \textbf{stato del test}: indicazione dello stato del test, che può essere "superato", "non superato" o "non implementato";
\end{itemize}

\subsubsubsection{Test di unità}
I test di unità consentono di valutare il funzionamento delle singole unità o componenti del codice in modo isolato, senza dipendere dalle altre parti del sistema. Questo approccio mira a garantire che ogni porzione di software, anche la più piccola, operi correttamente e coerentemente con le specifiche definite, indipendentemente dalle interazioni con altre unità.
L'obiettivo principale dei test di unità è di assicurare che ogni unità funzioni come previsto, restituendo risultati attesi per determinate condizioni di input. Ciò significa che, per ciascuna unità, vengono definiti uno o più scenari di test che includono input specifici e i risultati attesi per tali input. Durante l'esecuzione dei test, il comportamento effettivo dell'unità viene confrontato con i risultati attesi, e qualsiasi discrepanza tra i due indica un potenziale difetto nel codice.
La pratica dei test di unità offre diversi vantaggi significativi. Innanzitutto, permette di individuare e risolvere tempestivamente eventuali difetti nel codice, poiché consente di identificare problemi già durante le prime fasi dello sviluppo, quando sono più facili e meno costosi da correggere. Inoltre, l'esecuzione automatica dei test di unità consente di garantire una verifica continua del codice, permettendo agli sviluppatori di individuare rapidamente eventuali regressioni o effetti collaterali indesiderati derivanti da modifiche al codice.
Inoltre, i test di unità forniscono una documentazione vivente del comportamento del software, in quanto ciascun test rappresenta un caso specifico di utilizzo dell'unità. Questo non solo facilita la comprensione del codice da parte di altri sviluppatori, ma può anche servire come una sorta di contratto che definisce il comportamento atteso dell'unità nel tempo.

\subsubsubsection{Test di integrazione}
I test di integrazione sono una fase nel processo di testing del software durante la quale vengono verificati e validati i collegamenti e le interazioni tra le diverse unità o componenti del sistema. L'obiettivo principale dei test di integrazione è accertarsi che le singole parti del software, già testate individualmente durante i test di unità, funzionino correttamente quando integrate insieme come un sistema completo. In pratica, i test di integrazione esaminano come le varie unità o componenti interagiscono tra loro e se cooperano correttamente per fornire le funzionalità desiderate del software. Questi test possono rivelare problemi come:
\begin{itemize}
	\item \textbf{incompatibilità} tra le interfacce delle diverse unità;
	\item \textbf{errori di comunicazione} tra le componenti;
	\item \textbf{mancata sincronizzazione o coerenza} nei dati scambiati tra le unità;
\end{itemize}

\subsubsubsection{Test di sistema}
Durante questa tipologia di test l'intero sistema viene valutato e verificato per garantire che soddisfi i requisiti funzionali e non funzionali specificati. Vengono eseguiti dopo i test di unità e di integrazione e sono progettati per valutare il sistema nel suo complesso, testando le sue funzionalità e le sue prestazioni rispetto agli obiettivi definiti. In particolare, permettono di:
\begin{itemize}
	\item \textbf{verificare} che il sistema \textbf{soddisfi i requisiti specificati};
	\item \textbf{valutare le prestazioni} del sistema;
	\item \textbf{verificare la sicurezza} e la \textbf{robustezza} del sistema;
	\item \textbf{verificare la compatibilità} del sistema con l'ambiente di esecuzione;
	\item \textbf{verificare la facilità d'uso} e l'\textbf{usabilità} del sistema.
\end{itemize}

\subsubsubsection{Test di regressione}
I test di regressione assicurano che le modifiche apportate al codice non abbiano introdotto nuovi difetti o intaccato funzionalità esistenti nel software. Questi test vengono eseguiti dopo che sono state apportate modifiche al software, come nuove funzionalità, correzioni di bug o aggiornamenti del sistema, per assicurare il corretto funzionamento anche dopo tali modifiche.

\subsubsubsection{Test di accettazione}
I test di accettazione sono una fase nel processo di sviluppo del software durante la quale il prodotto viene testato per valutare se soddisfa i requisiti e le aspettative degli utenti finali, dei clienti o degli \href{https://7last.github.io/docs/pb/documentazione-interna/glossario\#stakeholder}{stakeholder\textsubscript{G}}. Questi test vengono eseguiti dopo che il software è stato completamente sviluppato e prima del suo rilascio ufficiale, consentendo agli utenti finali di valutarlo in un ambiente simile a quello di produzione e fornire feedback sulle sue funzionalità, l'usabilità e la conformità ai requisiti.

\subsubsubsection{Sequenza delle fasi di test}
I test vengono eseguiti in sequenza, partendo da quelli di unità e procedendo verso quelli di integrazione, di sistema, di regressione e di accettazione. Questo approccio graduale consente di identificare e risolvere i difetti in modo sistematico e strutturato, garantendo che il software funzioni correttamente e soddisfi i requisiti specificati.

\subsubsubsection{Codici dei test}
Ciascun test deve essere identificato da un codice univoco nel seguente formato:
\begin{center}
	\textbf{[numero\_test]-[tipologia]}
\end{center}
Dove:
\begin{itemize}
	\item \textbf{[numero\_test]}: rappresenta un numero, in ordine crescente, associato al test, univoco all'interno della tipologia;
	\item \textbf{[tipologia]}: indica l'appartenenza del test, può assumere i seguenti valori:
	      \begin{itemize}
		      \item \textbf{U}: test di unità;
		      \item \textbf{I}: test di integrazione;
		      \item \textbf{S}: test di sistema;
		      \item \textbf{R}: test di regressione;
		      \item \textbf{A}: test di accettazione.
	      \end{itemize}

\end{itemize}

\subsubsubsection{Stato dei test}
A ciascun test è associato uno stato che indica l'esito della sua esecuzione:
\begin{itemize}
	\item \textbf{S}: il test è stato superato;
	\item \textbf{NS}: il test non è stato superato;
	\item \textbf{NI}: il test non è stato implementato.
\end{itemize}
Questi risultati saranno riportati nel documento \textit{"\href{https://7last.github.io/docs/pb/documentazione-interna/glossario\#piano-di-qualifica}{Piano di Qualifica\textsubscript{G}}"}, in particolare nella sezione \textit{"Metodologie di Testing"}.

\subsubsubsection{Analisi statica}
L'analisi statica è il processo mediante il quale si esamina il codice sorgente di un programma senza eseguirlo. Questo tipo di analisi viene effettuato utilizzando strumenti automatizzati che analizzano il codice per individuare possibili errori, violazioni delle best practice di codifica e potenziali vulnerabilità di sicurezza. L'analisi statica si concentra su aspetti come la sintassi, la semantica e le strutture di controllo del codice. I benefici di questo tipo di analisi sono molteplici, tra cui:
\begin{itemize}
	\item \textbf{individuazione precoce degli errori}: consente di rilevare errori e difetti nelle prime fasi dello sviluppo, riducendo i costi e i tempi necessari per le correzioni;
	\item \textbf{miglioramento della qualità del codice}: promuove l'adozione di standard di codifica e best practice, migliorando la manutenibilità e la leggibilità del codice;
	\item \textbf{prevenzione di vulnerabilità di sicurezza}: aiuta a individuare vulnerabilità di sicurezza, come buffer overflow e injection flaws, prima che il software venga eseguito.
\end{itemize}

Nell'ambito dell'analisi statica del software, l'inspection e il walkthrough sono tecniche di revisione del codice e della documentazione, ma presentano differenze significative in termini di obiettivi, formalità e procedure.

\subsubsubsubsection{Inspection}
L'inspection è una tecnica formale e strutturata di revisione del codice. Le sue principali caratteristiche sono:
\begin{itemize}
	\item \textbf{formalità}: l'inspection segue un processo rigoroso con ruoli definiti (moderatore, autore, lettore, ispettore, e registratore) e fasi ben precise (pianificazione, overview, preparazione, meeting di inspection, rework, e follow-up);
	\item \textbf{obiettivi}: l'obiettivo principale è identificare difetti nel codice o nella documentazione. Si concentra su aspetti come errori di logica, violazioni degli standard di codifica, problemi di performance e sicurezza;
	\item \textbf{documentazione}: viene prodotta una documentazione dettagliata delle osservazioni e dei difetti riscontrati, che serve come base per la correzione e per tracciare le azioni successive;
	\item \textbf{ruoli e responsabilità}: ogni partecipante ha un ruolo specifico e le attività sono strettamente coordinate. Il moderatore gestisce il processo, l'autore presenta il lavoro, gli ispettori individuano i difetti, il lettore guida la revisione del materiale, e il registratore annota i difetti trovati;
	\item \textbf{preparazione} I partecipanti devono esaminare il materiale da rivedere in anticipo e preparare una lista di potenziali problemi.
\end{itemize}

\subsubsubsubsection{Walkthrough}
Il walkthrough è una tecnica meno formale e più flessibile rispetto all'inspection. Le sue principali caratteristiche sono:
\begin{itemize}
	\item \textbf{informalità}: il walkthrough è meno strutturato e può essere condotto in modo informale. Non richiede ruoli rigidamente definiti o una procedura rigorosa;
	\item \textbf{obiettivi}: l'obiettivo principale è comprendere il codice o la documentazione, discutere possibili miglioramenti e condividere conoscenze tra i membri del team. Anche se l'individuazione dei difetti è uno degli scopi, non è l'unico focus;
	\item \textbf{documentazione}: la documentazione delle osservazioni è meno formale e dettagliata rispetto all'inspection. Può non essere sempre prodotta una documentazione completa delle discussioni e dei difetti riscontrati;
	\item \textbf{ruoli e responsabilità}: non ci sono ruoli formali assegnati. Solitamente, l'autore del codice o della documentazione guida la revisione e i partecipanti contribuiscono con commenti e suggerimenti;
	\item \textbf{preparazione}: la preparazione è meno intensiva. Non è necessario che tutti i partecipanti esaminino il materiale in anticipo, anche se può essere utile.
\end{itemize}

\subsubsection{Metriche}

\begin{table}[!h]
	\centering
	\begin{tabular}{|c|l|}
		\hline
		\textbf{Codice} & \textbf{Nome esteso}               					\\
		\hline
		\underline{\hyperlink{26M}{26M-CC}}     & Code Coverage   				\\
		\underline{\hyperlink{27M}{27M-BC}}     & Branch Coverage   			\\
		\underline{\hyperlink{28M}{28M-SC}}     & Statement Coverage   			\\
		\underline{\hyperlink{29M}{29M-FD}}     & Failure Density   			\\
		\underline{\hyperlink{30M}{30M-PTCP}}   & Passed Test Case Percentage   \\
		\hline
	\end{tabular}
	\caption{Metriche inerenti il processo di verifica}
\end{table}



\subsubsection{Strumenti}

\begin{itemize}
	\item \href{http://aspell.net/}{Aspell} per il controllo ortografico;
	\item \href{https://docs.python.org/3/library/unittest.html}{unittest} libreria di \href{https://7last.github.io/docs/pb/documentazione-interna/glossario\#python}{Python\textsubscript{G}} per lo sviluppo di test.
\end{itemize}



\subsection{Validazione}
\subsubsection{Introduzione}
La validazione rappresenta un momento critico nel ciclo di sviluppo, in quanto sottopone il software a una serie di controlli dettagliati per assicurare la sua conformità ai requisiti stabiliti e la sua idoneità all'utilizzo da parte degli utenti finali. Questo processo non è solo una verifica formale, ma una fase cruciale che assicura che il software sia costruito in modo tale da rispondere pienamente alle esigenze e alle aspettative degli utenti, nonché agli obiettivi del progetto.

\subsubsection{Procedura di validazione}
In questo processo copre un ruolo fondamentale il test di accettazione che mira a garantire la validazione
del prodotto. Infatti i diversi test elencati nella sezione \underline{\hyperlink{testing}{testing}} costituiscono un input
per la validazione. Essi dovranno verificare:
\begin{itemize}
	\item l'implementazione di tutti i casi d'uso;
	\item la conformità del prodotto ai requisiti obbligatori;
	\item il soddisfacimento di altri requisiti concordati con il \href{https://7last.github.io/docs/pb/documentazione-interna/glossario\#committente}{committente\textsubscript{G}}.
\end{itemize}

\subsection{Gestione della configurazione}
\subsubsection{Introduzione}
La gestione della configurazione è il processo di identificazione, controllo e coordinamento dei componenti software e delle risorse associate durante tutto il ciclo di vita del prodotto. Questo processo assicura che il software e i suoi artefatti correlati siano gestiti in modo coerente e controllato, consentendo agli sviluppatori di tracciare le modifiche, gestire le versioni e garantire l'integrità e la coerenza del sistema nel tempo.
\newpage
\subsubsection{Versionamento}
La convenzione di versionamento adottata è nel formato X.Y dove:
\begin{itemize}
	\item \textbf{X}: rappresenta il completamento in vista di una delle fasi del progetto e dunque viene incrementato al raggiungimento di \href{https://7last.github.io/docs/pb/documentazione-interna/glossario\#requirements-and-technology-baseline}{RTB\textsubscript{G}}, \href{https://7last.github.io/docs/pb/documentazione-interna/glossario\#product-baseline}{PB\textsubscript{G}} ed eventuale \href{https://7last.github.io/docs/pb/documentazione-interna/glossario\#customer-acceptance}{CA\textsubscript{G}}.
	\item \textbf{Y}: rappresenta una versione intermedia e viene incrementata ad ogni modifica significativa del documento.
\end{itemize}

\subsubsection{Repository}
Il team utilizza due repository:
\begin{itemize}
	\item \href{https://github.com/7Last/docs.git}{\underline{docs}}: contenente la documentazione prodotta;
	\item \href{https://github.com/7Last/SyncCity}{\href{https://7last.github.io/docs/pb/documentazione-interna/glossario\#synccity}{\underline{SyncCity}\textsubscript{G}}}: contenente il codice del progetto.
\end{itemize}

\subsubsubsection{Struttura repository}
Il repository inerente alla documentazione è così organizzato:
\begin{enumerate}
	\item \textbf{Candidatura}:
	      \begin{itemize}
		      \item \textbf{verbali\_esterni}: al suo interno sono presenti i verbali delle riunioni avute con i membri di \textit{Sync Lab S.r.l.};
		      \item \textbf{verbali\_interni}: contenente i verbali delle riunioni interne svolte;
		      \item \textbf{lettera\_di\_presentazione}: documento di presentazione per la candidatura al \href{https://7last.github.io/docs/pb/documentazione-interna/glossario\#capitolato}{capitolato\textsubscript{G}} scelto;
		      \item \textbf{preventivo\_costi\_assunzione\_impegni}: documento in cui vengono specificati i costi previsti, il totale delle ore per persona e gli impegni assunti;
		      \item \textbf{valutazione\_dei\_capitolati}: contenente il parere personale riguardo i capitolati offerti dalle varie aziende.
	      \end{itemize}
	\item \href{https://7last.github.io/docs/pb/documentazione-interna/glossario\#requirements-and-technology-baseline}{\textbf{RTB}\textsubscript{G}}:
	      \begin{itemize}
		      \item \textbf{documentazione\_esterna}: contenente i seguenti documenti:
		            \begin{itemize}
			            \item \textbf{analisi\_dei\_requisiti};
			            \item \textbf{piano\_di\_progetto};
			            \item \textbf{piano\_di\_qualifica};
		            \end{itemize}
		      \item \textbf{documentazione\_interna}: contenente i seguenti documenti:
		            \begin{itemize}
			            \item \href{https://7last.github.io/docs/pb/documentazione-interna/glossario\#glossario}{\textbf{glossario}\textsubscript{G}};
			            \item \textbf{norme\_di\_progetto}.
		            \end{itemize}
		      \item \textbf{verbali\_esterni}: al suo interno sono presenti i verbali delle riunioni avute con i membri di \textit{Sync Lab S.r.l.};
		      \item \textbf{verbali\_interni}: contenente i verbali delle riunioni interne svolte.
	      \end{itemize}
	\item \href{https://7last.github.io/docs/pb/documentazione-interna/glossario\#product-baseline}{\textbf{PB}\textsubscript{G}} %todo, da implementare
\end{enumerate}

\subsubsection{Sincronizzazione e branching}

\subsubsubsection{Documentazione}	\label{nomenclatura}
L'approccio adottato per la redazione della documentazione segue il workflow noto come \textit{Gitflow}. Questo workflow è un processo strutturato e collaborativo che consente ai membri del team di lavorare in modo efficace alla creazione, revisione e integrazione di documenti all'interno di un progetto. Questo approccio garantisce una gestione ordinata e controllata della redazione dei documenti, consentendo una migliore organizzazione e una maggiore qualità del lavoro finale.
\begin{flushleft}
	\textbf{Nomenclatura dei branch per le attività di redazione e/o modifica di documenti}: \label{convenzioni_nomenclatura}
\end{flushleft}
\begin{itemize}
	\item il nome del nuovo branch deve riportare il titolo del documento da redarre o modificare;
	\item il nome dei verbali deve presentare anche la data della riunione:
	      \begin{itemize}
		      \item \textit{verbale\_interno\_yy\_mm\_dd} (es. verbale\_interno\_24\_03\_05);
	      \end{itemize}
\end{itemize}

\subsubsubsection{Sviluppo}
\textit{7Last} utilizza \href{https://www.atlassian.com/it/git/tutorials/comparing-workflows/gitflow-workflow}{\underline{{Gitflow}}} come flusso di lavoro.
\begin{flushleft}
	Flusso di lavoro Gitflow:
\end{flushleft}
\begin{enumerate}
    \item \textbf{\texttt{main} branch}
    \begin{itemize}
        \item branch principale e stabile;
        \item contiene il codice in produzione;
        \item ogni commit rappresenta una versione rilasciata.
    \end{itemize}

    \item \textbf{\texttt{develop} branch}
    \begin{itemize}
        \item branch per lo sviluppo integrato;
        \item riceve i merge dai feature branches;
        \item rappresenta lo stato intermedio prima del rilascio.
    \end{itemize}

    \item \textbf{feature branch}
    \begin{itemize}
        \item derivano dal branch \texttt{develop};
        \item utilizzati per sviluppare nuove funzionalità;
        \item dopo il completamento, viene fatto il merge nel branch \texttt{develop}.
    \end{itemize}

    \item \textbf{release branch}
    \begin{itemize}
        \item derivano dal branch \texttt{develop};
        \item preparano una nuova versione per il rilascio;
        \item permettono di effettuare correzioni di bug e preparare la documentazione;
        \item dopo il completamento, viene fatto il merge sia nel branch \texttt{main} che nel branch \texttt{develop}.
    \end{itemize}

    \item \textbf{hotfix branch}
    \begin{itemize}
        \item derivano dal branch master;
        \item utilizzati per correggere rapidamente bug critici in produzione;
        \item dopo il completamento, viene fatto il merge sia nel branch \texttt{main} che nel branch \texttt{develop}.
    \end{itemize}
\end{enumerate}

\subsubsection{Consegna e rilascio}
Il processo di consegna e rilascio secondo lo standard \textit{ISO/IEC 12207:1995} è una componente essenziale del ciclo di vita del software, che si concentra sulla gestione delle versioni del software e sulla distribuzione dei prodotti software agli utenti finali. Questo processo assicura che le versioni del software siano rilasciate in modo controllato, pianificato e che soddisfino i requisiti di qualità e sicurezza. Attraverso una serie di attività ben definite, il processo di gestione del rilascio garantisce che il software sia pronto per l'uso, minimizzando i rischi di problemi post-rilascio e massimizzando l'efficienza operativa. La creazione della prima release deve avvenire in concomitanza con la baseline \href{https://7last.github.io/docs/pb/documentazione-interna/glossario\#requirements-and-technology-baseline}{RTB\textsubscript{G}} (\href{https://7last.github.io/docs/pb/documentazione-interna/glossario\#requirements-and-technology-baseline}{Requirements and Technology Baseline\textsubscript{G}}), mentre la creazione delle successive release deve avvenire in concomitanza con la baseline \href{https://7last.github.io/docs/pb/documentazione-interna/glossario\#product-baseline}{PB\textsubscript{G}} (\href{https://7last.github.io/docs/pb/documentazione-interna/glossario\#product-baseline}{Product Baseline\textsubscript{G}}) e, se prevista, \href{https://7last.github.io/docs/pb/documentazione-interna/glossario\#customer-acceptance}{CA\textsubscript{G}} (\href{https://7last.github.io/docs/pb/documentazione-interna/glossario\#customer-acceptance}{Customer Acceptance\textsubscript{G}}).

\subsubsection{Sito del gruppo}
Il gruppo ha sviluppato un sito disponibile al link: \url{https://7last.github.io/}; che ha lo scopo di facilitare la consultazione dei documenti redatti fino ad ora. Tale pagina web viene aggiornata automaticamente per rispecchiare lo stato dei documenti presenti all'interno del branch \texttt{main}. Inoltre facilita la consultazione del \href{https://7last.github.io/docs/pb/documentazione-interna/glossario\#glossario}{glossario\textsubscript{G}} grazie a dei link, presenti nelle parole distinte dalla lettera G al pedice, diretti alla definizione del termine.

\subsubsection{Strumenti}

\begin{itemize}
	\item \href{https://7last.github.io/docs/pb/documentazione-interna/glossario\#github}{\textbf{Github}\textsubscript{G}}: utilizzato per la gestione delle versioni e per la condivisione dei documenti;
	\item \href{https://7last.github.io/docs/pb/documentazione-interna/glossario\#clickup}{\textbf{ClickUp}\textsubscript{G}}: è una piattaforma di gestione del lavoro all-in-one progettata per pianificare, organizzare e collaborare su progetti e attività.
\end{itemize}


\subsection{Analisi congiunta}
\subsubsection{Introduzione}
L'analisi congiunta è un'attività collaborativa cruciale in cui i recensori e i recensiti si incontrano per esaminare e valutare diversi aspetti del progetto.  Nel nostro caso i recensori sono costituiti da \href{https://7last.github.io/docs/pb/documentazione-interna/glossario\#proponente}{proponente\textsubscript{G}}, \href{https://7last.github.io/docs/pb/documentazione-interna/glossario\#committente}{committente\textsubscript{G}} e \href{https://7last.github.io/docs/pb/documentazione-interna/glossario\#stakeholder}{stakeholder\textsubscript{G}}; mentre i recensiti sono rappresentati dal gruppo \textit{7Last}. Lo scopo principale di questa revisione congiunta è garantire che tutte le parti coinvolte abbiano una comprensione comune dello stato attuale del progetto e dei passi successivi necessari per raggiungere gli obiettivi fissati.

\subsubsection{Realizzazione del processo}
Il processo comprende i seguenti impegni:
\subsubsubsection{Revisioni periodiche}
In corrispondenza delle \href{https://7last.github.io/docs/pb/documentazione-interna/glossario\#milestone}{milestone\textsubscript{G}} stabilite saranno effettuate delle revisioni periodiche, come riportato nel
documento \href{https://7last.github.io/docs/pb/documentazione-interna/glossario\#piano-di-progetto}{\textit{Piano di Progetto}\textsubscript{G}}.

\subsubsubsection{Stato avanzamento lavori}
Al termine di ogni \href{https://7last.github.io/docs/pb/documentazione-interna/glossario\#sprint}{sprint\textsubscript{G}} viene svolta una revisione \href{https://7last.github.io/docs/pb/documentazione-interna/glossario\#stato-avanzamento-lavori}{SAL\textsubscript{G}}
(\href{https://7last.github.io/docs/pb/documentazione-interna/glossario\#stato-avanzamento-lavori}{Stato Avanzamento Lavori\textsubscript{G}}) tra il team e la \href{https://7last.github.io/docs/pb/documentazione-interna/glossario\#proponente}{proponente\textsubscript{G}}. Questa revisione ha lo scopo di valutare il lavoro svolto durante lo \href{https://7last.github.io/docs/pb/documentazione-interna/glossario\#sprint}{sprint\textsubscript{G}} precedente, per verificare che gli obiettivi prefissati siano stati correttamente raggiunti secondo le scadenze prefissate.
Inoltre, durante questo incontro, si pianificano le attività per lo \href{https://7last.github.io/docs/pb/documentazione-interna/glossario\#sprint}{sprint\textsubscript{G}} successivo.

\subsubsubsection{Revisioni straordinarie}
Se uno qualsiasi dei soggetti coinvolti tra gli \href{https://7last.github.io/docs/pb/documentazione-interna/glossario\#stakeholder}{stakeholder\textsubscript{G}} lo ritenga opportuno, è possibile istituire una revisione straordinaria per esaminare attentamente lo stato di avanzamento dei lavori. Durante questa revisione, si discutono eventuali problematiche emerse e le relative soluzioni adottabili.

\subsubsubsection{Risorse per le revisioni}
Le risorse coinvolte nelle revisioni possono essere differenti, ad esempio: strumenti hardware, software, documentazione... è fondamentale che tali risorse siano discusse e concordate tra tutte le parti coinvolte.

\subsubsubsection{Organizzazione degli incontri}
Pochi giorni prima della riunione con la \href{https://7last.github.io/docs/pb/documentazione-interna/glossario\#proponente}{proponente\textsubscript{G}} \textit{7Last} si impegna a consegnare un report con tutte le modifiche effettuate e i campi in cui sono state applicate.
In ciascuna revisione rimane la necessità di concordare i seguenti elementi:
\begin{itemize}
	\item agenda della riunione;
	\item prodotti software risultati dall'attività e relative problematiche;
\end{itemize}

\subsubsubsection{Documenti prodotti e decisioni approvate}
Tramite i \textit{Verbali Esterni} vengono documentati i risultati delle revisioni, compresi i problemi individuati, le soluzioni proposte e le azioni correttive da intraprendere. Questi documenti vengono distribuiti a tutte le parti coinvolte per garantire una comprensione comune e una chiara comunicazione. La parte recensente comunicherà alla parte recensita la veridicità di quanto riportato, approvando o disapprovando i documenti citati.

\subsection{Risoluzione dei problemi}
\subsubsection{Introduzione}
La risoluzione dei problemi è un processo che mira a identificare, analizzare e risolvere le varie problematiche che possono emergere durante lo sviluppo. Si riferisce a un insieme di tecniche e metodologie utilizzate per affrontare e risolvere le difficoltà tecniche, di design, di implementazione o di gestione che possono sorgere durante lo sviluppo del software. Questo processo può includere:
\begin{itemize}
	\item \textbf{identificazione del problema}: consiste nel riconoscere e definire chiaramente il problema;
	\item \textbf{analisi del problema}: nella quale si esamina il problema per capire le cause sottostanti e l'impatto;
	\item \textbf{generazione di soluzioni}: vengono proposte diverse possibili soluzioni al problema;
	\item \textbf{valutazione delle soluzioni}: vengono analizzate le soluzioni proposte per determinarne la fattibilità, l'efficacia e l'efficienza;
	\item \textbf{implementazione della soluzione}: la soluzione adottata viene implementata e testata per verificare che risolva il problema in modo efficace.
\end{itemize}

\subsubsection{Gestione dei rischi}
All'interno del documento \href{https://7last.github.io/docs/pb/documentazione-interna/glossario\#piano-di-progetto}{\textit{Piano di Progetto}\textsubscript{G}}, più precisamente nella sezione \textit{Analisi dei rischi} sono contenuti i rischi che potrebbero emergere durante lo svolgimento del progetto, con relativi approfondimenti e strategie di mitigazione.

\subsubsubsection{Codifica dei rischi}
Ogni rischio è identificato da un codice univoco nel seguente formato:
\begin{center}
	\textbf{R[tipologia]-[indice]}: nome identificativo del rischio.
\end{center}
Dove:
\begin{itemize}
	\item \textbf{[tipologia]}: rappresenta la categoria di rischio, la quale può essere organizzativa, tecnologica o comunicativa;
	\item \textbf{[indice]}: un valore numerico incrementale che identifica univocamente il rischio per ogni tipologia.
\end{itemize}

\subsubsection{Metriche}
\begin{table}[h]
	\centering
	\begin{tabular}{|c|c|}
		\hline
		\textbf{Metrica} & \textbf{Nome}        						\\
		\hline
		\underline{\hyperlink{31M}{31M-RMR}}   &  Risk Mitigation Rate  \\
		\underline{\hyperlink{32M}{32M-NCR}}   &  Rischi Non Calcolati  \\
		\hline
	\end{tabular}
	\caption{Metriche relative alla risoluzione dei problemi}
\end{table}

\subsubsection{Strumenti}
\begin{itemize}
	\item \href{https://7last.github.io/docs/pb/documentazione-interna/glossario\#clickup}{\textbf{ClickUp}\textsubscript{G}}: utilizzato per la gestione delle attività e delle issue;
\end{itemize}
