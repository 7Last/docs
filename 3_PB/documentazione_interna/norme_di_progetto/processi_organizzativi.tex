\section{Processi organizzativi}
I processi organizzativi sono fondamentali per garantire che il progetto sia gestito in modo efficace, efficiente e conforme agli standard di qualità. Questi processi servono a coordinare le attività del team, a ottimizzare l'uso delle risorse e a mitigare i rischi, assicurando così il successo del progetto. 
\subsection{Gestione dei processi}
\subsubsection{Introduzione}
La gestione dei processi è un'attività chiave per garantire che il progetto sia completato con successo e in conformità con gli obiettivi e i requisiti stabiliti. Questa fase del progetto si concentra sulla pianificazione, organizzazione, monitoraggio e controllo delle attività coinvolte nel ciclo di vita del software, assicurando che il lavoro svolto sia di alta qualità e soddisfi le aspettative del cliente. Le attività di gestione dei processi sono qui di seguito specificate.
\begin{itemize}
    \item \textbf{Definizione dei processi}: documentazione dei processi chiave che saranno utilizzati nel progetto, inclusi i processi di sviluppo del software, di controllo di versione, di gestione dei cambiamenti e di assicurazione della qualità.
    \item \textbf{Pianificazione dei processi}: vengono definiti gli obiettivi, le fasi del progetto, le risorse necessarie e le scadenze. Questa fase stabilisce anche i criteri di successo e il piano di lavoro dettagliato.
    \item \textbf{Assegnazione delle risorse}: assegnazione dei membri del team alle attività specifiche del progetto in base alle loro competenze e disponibilità.
    \item \textbf{Monitoraggio e controllo}: si effettua un monitoraggio continuo del progresso del progetto rispetto al piano stabilito. Questo include il monitoraggio dei tempi, dei costi e della qualità, nonché l'identificazione e la gestione dei rischi.
    \item \textbf{Gestione dei cambiamenti}: vengono effettuate una valutazione e gestione delle modifiche richieste durante lo sviluppo del software. Questo può includere modifiche ai requisiti, alla pianificazione o alla distribuzione delle risorse.
    \item \textbf{Assicurazione della qualità}: implementazione di processi e procedure per garantire che il prodotto software soddisfi i requisiti e le aspettative del cliente.
    \item \textbf{Comunicazione e coordinamento}: facilitazione della comunicazione tra i membri del team, gli \href{https://7last.github.io/docs/pb/documentazione-interna/glossario\#stakeholder}{stakeholder\textsubscript{G}}. Questo assicura che tutte le parti coinvolte siano informate sullo stato del progetto e sulle decisioni prese.
    \item \textbf{Miglioramento continuo}: consiste in un'analisi dei processi utilizzati nel progetto al fine di identificare aree di miglioramento e implementare azioni correttive per ottimizzare l'efficienza e la qualità complessiva del lavoro svolto.
\end{itemize}

\subsubsection{Pianificazione}
\subsubsubsection{Descrizione}
La pianificazione dei processi implica la definizione, l'organizzazione e il controllo delle attività necessarie per portare a termine con successo il progetto. Si tratta di un'attività strategica che fornisce una guida chiara e una struttura gestionale per tutto il ciclo di vita. La pianificazione dei processi è essenziale per garantire che il software sia completato in tempo, entro il budget e con la qualità richiesta.

\subsubsubsection{Obiettivi}
Lo scopo principale della pianificazione dei processi è garantire che il progetto venga eseguito in modo efficiente, efficace e conforme agli obiettivi e ai requisiti stabiliti, assicurando allo stesso tempo che ciascun membro del team assuma ogni ruolo per almeno una volta. Serve anche a mitigare i rischi e ad affrontare le sfide in modo proattivo, consentendo al team di affrontare eventuali ostacoli lungo il percorso.

\subsubsubsection{Assegnazione dei ruoli}
Durante lo svolgimento del progetto, i membri di \textit{7Last} assumeranno ruoli distinti, svolgendo le mansioni previste e assumendosi le responsabilità proprie del determinato ruolo, di seguito meglio specificate.

\subsubsubsubsection*{\href{https://7last.github.io/docs/pb/documentazione-interna/glossario\#responsabile}{Responsabile\textsubscript{G}}:}
\begin{itemize}
    \item coordina il gruppo di lavoro;
    \item pianifica e controlla le attività;
    \item gestisce le risorse;
    \item gestisce le comunicazioni con l'esterno;
    \item redige il \href{https://7last.github.io/docs/pb/documentazione-interna/glossario\#piano-di-progetto}{\textit{Piano di Progetto}\textsubscript{G}}.
\end{itemize}

\subsubsubsubsection*{\href{https://7last.github.io/docs/pb/documentazione-interna/glossario\#amministratore}{Amministratore\textsubscript{G}}:}
\begin{itemize}
    \item gestisce l'ambiente di lavoro;
    \item gestisce le procedure e le norme;
    \item gestisce la configurazione del prodotto;
    \item redige le \href{https://7last.github.io/docs/pb/documentazione-interna/glossario\#norme-di-progetto}{\textit{Norme di Progetto}\textsubscript{G}}.
\end{itemize}

\subsubsubsubsection*{\href{https://7last.github.io/docs/pb/documentazione-interna/glossario\#analista}{Analista\textsubscript{G}}:}
\begin{itemize}
    \item analizza i requisiti del progetto;
    \item studia il dominio applicativo del problema;
    \item redige l'\href{https://7last.github.io/docs/pb/documentazione-interna/glossario\#analisi-dei-requisiti}{\textit{Analisi dei Requisiti}\textsubscript{G}}.
\end{itemize}

\subsubsubsubsection*{\href{https://7last.github.io/docs/pb/documentazione-interna/glossario\#progettista}{Progettista\textsubscript{G}}:}
\begin{itemize}
    \item progetta l'architettura del prodotto;
    \item prende decisioni tecniche e tecnologiche;
    \item redige la \textit{Specifica Tecnica}.
\end{itemize}

\subsubsubsubsection*{\href{https://7last.github.io/docs/pb/documentazione-interna/glossario\#programmatore}{Programmatore\textsubscript{G}}:}
\begin{itemize}
    \item scrive il codice del prodotto;
    \item implementa le funzionalità richieste;
    \item codifica le componenti dell'architettura del prodotto;
    \item redige il \textit{Manuale Utente}. 
\end{itemize}

\subsubsubsubsection*{\href{https://7last.github.io/docs/pb/documentazione-interna/glossario\#verificatore}{Verificatore\textsubscript{G}}:}
\begin{itemize}
    \item verifica che il lavoro svolto sia conforme alle norme e alle specifiche tecniche del progetto;
    \item ricerca ed eventualmente segnala errori;
    \item redige il \href{https://7last.github.io/docs/pb/documentazione-interna/glossario\#piano-di-qualifica}{\textit{Piano di Qualifica}\textsubscript{G}}.
\end{itemize}

\subsubsubsection{Ticketing}
Il gruppo \textit{7Last} ha deciso di utilizzare \href{https://7last.github.io/docs/pb/documentazione-interna/glossario\#clickup}{ClickUp\textsubscript{G}} come strumento di Issue Tracking System (ITS) per la gestione delle attività. L'\href{https://7last.github.io/docs/pb/documentazione-interna/glossario\#amministratore}{amministratore\textsubscript{G}} si occuperà di creare le attività e assegnarle ai membri del gruppo, i quali dovranno segnalarne lo stato di avanzamento. Su \href{https://7last.github.io/docs/pb/documentazione-interna/glossario\#clickup}{ClickUp\textsubscript{G}} sarà possibile visualizzare le attività assegnate, i compiti da svolgere e le scadenze da rispettare tramite una \href{https://7last.github.io/docs/pb/documentazione-interna/glossario\#dashboard}{dashboard\textsubscript{G}} semplice ed intuitiva. Un'ulteriore funzionalità offerta da \href{https://7last.github.io/docs/pb/documentazione-interna/glossario\#clickup}{ClickUp\textsubscript{G}} è quella del monitoraggio del tempo, strumento utile per il tracciamento del tempo impiegato per svolgere ciascuna attività. Altra caratteristica molto utile è l'integrazione con lo strumento \href{https://7last.github.io/docs/pb/documentazione-interna/glossario\#github}{\textit{GitHub}\textsubscript{G}}, mediante la quale è possibile collegare le attività create su \href{https://7last.github.io/docs/pb/documentazione-interna/glossario\#clickup}{ClickUp\textsubscript{G}} ai branch di \href{https://7last.github.io/docs/pb/documentazione-interna/glossario\#github}{GitHub\textsubscript{G}} e alle relative pull request.
\subsubsubsubsection*{Apertura e ciclo di vita delle attività}
In \href{https://7last.github.io/docs/pb/documentazione-interna/glossario\#clickup}{ClickUp\textsubscript{G}} si possono creare diversi tipi di attività:
\begin{itemize}
    \item \textbf{attività} (predefinito), attività generica corrispondente alle issue di altri ITS;
    \item \textbf{attività cardine}, attività principali che rappresentano le \href{https://7last.github.io/docs/pb/documentazione-interna/glossario\#milestone}{milestone\textsubscript{G}} del progetto;
    \item \href{https://7last.github.io/docs/pb/documentazione-interna/glossario\#product-backlog}{\textbf{Product Backlog\textsubscript{G} }Item} (PBI), strumento chiave per pianificare il lavoro del team e assicurare che il prodotto evolva in modo coerente.
\end{itemize}
Le attività verranno create seguendo una struttura ben definita.
\begin{itemize}
    \item \textbf{Tipo di attività}: indica il tipo di attività da svolgere.
    \item \textbf{Nome}: deve essere chiaro e conciso, in modo da identificare facilmente l'attività.
    \item \textbf{Assegnatari}: membri del team a cui viene assegnata l'attività.
    \item \textbf{Data di scadenza}: data entro la quale l'attività deve essere completata.
    \item \textbf{Priorità}: indica l'importanza dell'attività rispetto alle altre.
    \item \textbf{Stato}: indica lo stato di avanzamento dell'attività.
    \item \textbf{Tag}: etichetta che identifica l'attività in base alla sua tipologia.
    \item \textbf{Descrizione}: descrizione dettagliata dell'attività da svolgere.
    \item \href{https://7last.github.io/docs/pb/documentazione-interna/glossario\#github}{\textbf{GitHub}\textsubscript{G}}: collegamento con il branch di \href{https://7last.github.io/docs/pb/documentazione-interna/glossario\#github}{GitHub\textsubscript{G}} relativo all'attività.
\end{itemize}
Ogni attività avrà un ciclo di vita ben definito, che prevede i seguenti stati:
\begin{itemize}
    \item \textbf{da fare} indica che l'attività è stata assegnata ma non ancora iniziata;
    \item \textbf{in corso} specifica che il membro del team sta lavorando ad essa;
    \item \textbf{completata} evidenzia che l'attività è stata completata con successo e la pull request è stata accettata.
\end{itemize}

\subsubsection{Metriche}
\begin{table}[!h]
	\centering
	\begin{tabular}{ | c | l | }
		\hline
		\textbf{Codice}                      & \textbf{Nome esteso}         \\
		\hline
        \underline{\hyperlink{33M}{33M-RSI}} & Requirements Stability Index \\
		\hline
	\end{tabular}
	\caption{Metriche relative alla pianificazione}
\end{table}

\subsubsubsection{Strumenti}
\begin{itemize}
    \item \href{https://7last.github.io/docs/pb/documentazione-interna/glossario\#github}{\textbf{GitHub}\textsubscript{G}}: utilizzato per la condivisione del codice tra i membri del gruppo;
    \item \href{https://7last.github.io/docs/pb/documentazione-interna/glossario\#clickup}{\textbf{ClickUp}\textsubscript{G}}: piattaforma utilizzata per il tracciamento e la gestione delle issue e dei compiti;
\end{itemize}

\subsubsection{Coordinamento}
\subsubsubsection{Descrizione}
La fase di coordinamento è un'attività cruciale per la gestione della comunicazione e delle risorse all'interno del progetto. Questa fase si concentra sulla creazione di un ambiente di lavoro collaborativo e armonioso, in cui i membri del team possano lavorare insieme in modo efficace e produttivo. Il coordinamento coinvolge la pianificazione, l'organizzazione e il controllo delle attività, garantendo che ogni membro del team sia adeguatamente informato, motivato e allineato sui compiti da svolgere e sugli obiettivi da raggiungere. 

\subsubsubsection{Obiettivi}
Lo scopo principale della fase di coordinamento è quello di creare armonia tra tutte le attività del progetto, integrando e sincronizzando ogni componente in modo da massimizzare l'efficienza e ottimizzare l'utilizzo delle risorse disponibili. Questo processo implica una gestione attenta delle risorse umane, finanziarie e temporali, assicurando che ogni membro del team sia adeguatamente informato, motivato e allineato sui compiti da svolgere e sugli obiettivi da raggiungere. Inoltre, il coordinamento mira a garantire che le risorse siano allocate in modo appropriato, evitando sovrapposizioni o carenze che potrebbero compromettere il successo del progetto. In definitiva, la fase di coordinamento agisce come il collante che tiene insieme tutte le componenti del progetto, consentendo al team di lavorare in modo collaborativo verso il raggiungimento del successo. A tal proposito, il gruppo \textit{7Last} si occupa di mantenere comunicazioni attive, sia interne che esterne, che possono essere sincrone o asincrone.
\subsubsubsection{Comunicazioni asincrone}
\begin{itemize}
    \item \textbf{Comunicazione asincrone interne}: viene deciso di adottare Telegram, applicazione che consente di comunicare mediante l'utilizzo di messaggi testo, media e file in chat private o all'interno di gruppi.
    \item \textbf{Comunicazione asincrone esterne}: vengono adottati due canali differenti per garantire le comunicazioni asincrone esterne, ovvero: 
        \begin{itemize}
            \item \textbf{e-mail}: per comunicazioni formali e ufficiali;
            \item \textbf{Discord}: in caso di necessità di risposta immediata e per comunicazioni informali.
        \end{itemize}
\end{itemize}
\subsubsubsection{Comunicazioni sincrone}
\begin{itemize}
    \item \textbf{Comunicazione sincrone interne}: viene adottato \textbf{Discord} per questo scopo, il quale permette di comunicare tramite chiamate vocali, videochiamate, messaggi di testo, media e file in chat private o all'interno di gruppi.
    \item \textbf{Comunicazione sincrone esterne}: in accordo con l'azienda \textit{Sync Lab S.r.l.} viene scelto di adottare \textbf{Google Meet} per le comunicazioni sincrone esterne.
\end{itemize}

\subsubsubsection{Riunioni interne}
Le riunioni interne avranno luogo ogni mercoledì, tramite Discord. Queste riunioni serviranno a monitorare il progresso delle attività, a discutere eventuali problematiche riscontrate e a pianificare le attività future. L'orario in cui si terranno queste riunioni è dalle 15:00 alle 16:00. Nel caso in cui un membro del gruppo non possa partecipare alla riunione è tenuto a comunicarlo tempestivamente al \href{https://7last.github.io/docs/pb/documentazione-interna/glossario\#responsabile}{responsabile\textsubscript{G}}. Qualora fosse necessario, il \href{https://7last.github.io/docs/pb/documentazione-interna/glossario\#responsabile}{responsabile\textsubscript{G}} provvederà a trovare un'altra data e un altro orario che possa andare bene a tutti i membri del gruppo, in alternativa gli eventuali assenti si informeranno sull'andamento della riunione tramite la lettura dei relativi verbali.
 Le riunioni interne saranno documentate e il verbale sarà redatto dal redattore della riunione. In queste riunioni i compiti del \href{https://7last.github.io/docs/pb/documentazione-interna/glossario\#responsabile}{responsabile\textsubscript{G}} sono:
\begin{itemize}
    \item stabilire l'\textbf{\textit{ordine del giorno}} della riunione;
    \item guidare la riunione e assicurarsi che tutti i membri forniscano il loro parere in modo ordinato;
    \item pianificare e assegnare le nuove attività da svolgere.
\end{itemize}

\subsubsubsection{Riunioni esterne}
Durante lo svolgimento del progetto, è essenziale organizzare vari incontri con i \href{https://7last.github.io/docs/pb/documentazione-interna/glossario\#committente}{committenti\textsubscript{G}} o la \href{https://7last.github.io/docs/pb/documentazione-interna/glossario\#proponente}{proponente\textsubscript{G}} al fine di valutare lo stato di avanzamento del prodotto e chiarire eventuali dubbi o questioni. La responsabilità di convocare tali incontri ricade sul \href{https://7last.github.io/docs/pb/documentazione-interna/glossario\#responsabile}{responsabile\textsubscript{G}}, il quale è incaricato di pianificarli e agevolarne lo svolgimento in modo efficiente ed efficace. Sarà compito sempre del \href{https://7last.github.io/docs/pb/documentazione-interna/glossario\#responsabile}{responsabile\textsubscript{G}} anche l’esposizione dei punti di discussione alla \href{https://7last.github.io/docs/pb/documentazione-interna/glossario\#proponente}{proponente\textsubscript{G}} o al \href{https://7last.github.io/docs/pb/documentazione-interna/glossario\#committente}{committente\textsubscript{G}}, lasciando la parola ai membri del gruppo interessati se necessario. Questo approccio assicura una comunicazione efficace tra le varie parti in causa, garantendo una gestione ottimale del tempo e una registrazione accurata delle informazioni rilevanti emerse durante gli incontri. I membri del gruppo si impegnano a garantire la propria presenza in modo costante alle riunioni, facendo il possibile per riorganizzare eventuali altri impegni al fine di partecipare. Nel caso in cui gli obblighi inderogabili di un membro del gruppo rendessero impossibile la partecipazione, il \href{https://7last.github.io/docs/pb/documentazione-interna/glossario\#responsabile}{responsabile\textsubscript{G}} assicurerà di informare tempestivamente la \href{https://7last.github.io/docs/pb/documentazione-interna/glossario\#proponente}{proponente\textsubscript{G}} o i \href{https://7last.github.io/docs/pb/documentazione-interna/glossario\#committente}{committenti\textsubscript{G}}, richiedendo eventualmente la possibilità di rinviare la riunione ad una data successiva.

\subsubsubsubsection*{Riunioni con l'azienda \href{https://7last.github.io/docs/pb/documentazione-interna/glossario\#proponente}{proponente\textsubscript{G}}}
\textit{7Last} si impegna ad organizzare incontri regolari con \textit{Sync Lab S.r.l.} per monitorare lo stato di avanzamento del progetto e affrontare eventuali dubbi o questioni. In linea con tale impegno, è stato concordato di tenere incontri di \href{https://7last.github.io/docs/pb/documentazione-interna/glossario\#stato-avanzamento-lavori}{Stato Avanzamento Lavori\textsubscript{G}} (\href{https://7last.github.io/docs/pb/documentazione-interna/glossario\#stato-avanzamento-lavori}{SAL\textsubscript{G}}) inizialmente ogni due settimane, con l'intenzione di aumentare la frequenza a uno ogni settimana in seguito. Gli incontri si svolgeranno tramite la piattaforma Google Meet e tratteranno i seguenti argomenti:
\begin{itemize}
    \item discussione e valutazione delle attività svolte nel periodo passato;
    \item pianificazione delle attività per il periodo successivo;
    \item chiarimento di eventuali dubbi emersi nel corso del periodo passato.
\end{itemize}
\begin{flushleft}
\textbf{Verbali esterni}
\end{flushleft}
Come avviene per le riunioni interne, anche per quelle esterne avvenute con la \href{https://7last.github.io/docs/pb/documentazione-interna/glossario\#proponente}{proponente\textsubscript{G}} verrà stilato un verbale secondo le medesime modalità illustrate nella sezione relativa ai verbali Interni. Tuttavia, la distinzione risiede nel fatto che il verbale redatto sarà successivamente inviato all'azienda per l'approvazione e la firma.

\subsubsubsection{Strumenti}
\begin{itemize}
    \item \textbf{Discord}: impiegato per la comunicazione sincrona interna e asincrona esterna;
    \item \textbf{Telegram}: utilizzato per la comunicazione asincrona interna;
    \item \textbf{Google Meet}: adottato per la comunicazione sincrona esterna;
    \item \textbf{Gmail}: come servizio di posta elettronica.
\end{itemize}

\subsection{Miglioramento}
\subsubsection{Introduzione}
Secondo lo standard \textit{ISO/IEC 12207:1995}, il miglioramento continuo è un processo che coinvolge l'identificazione e l'attuazione di azioni correttive e preventive per ottimizzare i processi e i prodotti. Questo processo è essenziale per garantire che il progetto sia completato con successo e in conformità con gli obiettivi e i requisiti stabiliti. Si compone di:
\begin{itemize}
    \item \textbf{analisi dei dati}: questa fase coinvolge la raccolta e l'analisi dei dati pertinenti per identificare le aree che richiedono miglioramento;
    \item \textbf{miglioramento}: attuazione di azioni correttive e preventive per ottimizzare i processi.
\end{itemize}

\subsection{Formazione}
\subsubsection{Introduzione}
La formazione è una componente essenziale per garantire che tutti i membri del team siano adeguatamente preparati per affrontare le sfide tecniche e gestionali del progetto. Questa processo si concentra sullo sviluppo delle competenze e delle conoscenze necessarie per utilizzare strumenti, tecnologie e metodologie specifiche del progetto, promuovendo così l'efficienza e la qualità del lavoro svolto.

\subsubsection{Metodo di formazione}
\subsubsubsection{Individuale}
Ogni individuo del team dovrà compiere un processo di autoformazione per riuscire a svolgere al meglio il ruolo assegnato. La rotazione dei ruoli permetterà al nuovo occupante di un ruolo di apprendere le competenze necessarie da colui che lo ha precedentemente svolto, nel caso avesse delle lacune. Questo metodo permette di avere una formazione continua e di garantire che ogni membro del team sia in grado di svolgere ogni ruolo.

\subsubsubsection{Gruppo}
\textit{Sync Lab S.r.l.} mette a disposizione sessioni formative riguardo le tecnologie adottate nel progetto.
Qualcosa sia necessario, verranno eseguite riunioni per chiarire eventuali dubbi.