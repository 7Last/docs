\documentclass[italian,12pt]{article} %tipo di documento

%--------------variabili------------------%
\def\Title{Norme di Progetto}
\def\Author{7Last}
\def\Version{v0.2}
%-----------------------------------------%


\usepackage[left=2cm, right=2cm, bottom=3cm, top=3cm]{geometry}
\usepackage{fancyhdr}
\usepackage{graphicx}
\graphicspath{ {../../logo/} }
\usepackage{href-ul}
\usepackage{tikz}
\usepackage{tgadventor}
\usepackage[useregional=numeric,showseconds=true,showzone=false]{datetime2}
\usepackage{caption}
\usepackage{longtable}
\usepackage{xcolor}



% Definizione delle nuove classi di titolo
\titleclass{\subsubsubsection}{straight}[\subsection]
\titleclass{\subsubsubsubsection}{straight}[\subsubsubsection]
\titleclass{\subsubsubsubsubsection}{straight}[\subsubsubsubsection] % nuovo livello

% Creazione dei nuovi contatori
\newcounter{subsubsubsection}[subsubsection]
\newcounter{subsubsubsubsection}[subsubsubsection]
\newcounter{subsubsubsubsubsection}[subsubsubsubsection] % nuovo livello

% Rinnovo dei comandi per la formattazione dei numeri delle sezioni
\renewcommand\thesubsubsubsection{\thesubsubsection.\arabic{subsubsubsection}}
\renewcommand\thesubsubsubsubsection{\thesubsubsubsection.\arabic{subsubsubsubsection}}
\renewcommand\thesubsubsubsubsubsection{\thesubsubsubsubsection.\arabic{subsubsubsubsubsection}} % nuovo livello
\renewcommand\theparagraph{\thesubsubsubsubsubsection.\arabic{paragraph}} % opzionale; utile se i paragrafi devono essere numerati

% Formattazione dei titoli delle sezioni
\titleformat{\subsubsubsection}
  {\normalfont\normalsize\bfseries}{\thesubsubsubsection}{1em}{}
\titleformat{\subsubsubsubsection}
  {\normalfont\normalsize\bfseries}{\thesubsubsubsubsection}{1em}{}
\titleformat{\subsubsubsubsubsection} % nuovo livello
  {\normalfont\normalsize\bfseries}{\thesubsubsubsubsubsection}{1em}{} 

% Spaziatura dei titoli delle sezioni
\titlespacing*{\subsubsubsection}
{0pt}{3.25ex plus 1ex minus .2ex}{1.5ex plus .2ex}
\titlespacing*{\subsubsubsubsection}
{0pt}{3.25ex plus 1ex minus .2ex}{1.5ex plus .2ex}
\titlespacing*{\subsubsubsubsubsection} % nuovo livello
{0pt}{3.25ex plus 1ex minus .2ex}{1.5ex plus .2ex}

\makeatletter
% Rinnovo dei comandi per la formattazione dei paragrafi e sottoparagrafi
\renewcommand\paragraph{\@startsection{paragraph}{6}{\z@}%
  {3.25ex \@plus1ex \@minus.2ex}%
  {-1em}%
  {\normalfont\normalsize\bfseries}}
\renewcommand\subparagraph{\@startsection{subparagraph}{7}{\parindent}%
  {3.25ex \@plus1ex \@minus .2ex}%
  {-1em}%
  {\normalfont\normalsize\bfseries}}

% Definizione dei livelli per il Table of Contents
\def\toclevel@subsubsubsection{4}
\def\toclevel@subsubsubsubsection{5}
\def\toclevel@subsubsubsubsubsection{6} % nuovo livello
\def\toclevel@paragraph{7}
\def\toclevel@subparagraph{8}

% Definizione della formattazione per il Table of Contents
\def\l@subsubsubsection{\@dottedtocline{4}{7em}{4em}}
\def\l@subsubsubsubsection{\@dottedtocline{5}{10em}{5em}}
\def\l@subsubsubsubsubsection{\@dottedtocline{6}{14em}{6em}} % nuovo livello
\def\l@paragraph{\@dottedtocline{7}{18em}{7em}}
\def\l@subparagraph{\@dottedtocline{8}{22em}{8em}}
\makeatother

% Impostazione della profondità dei numeri di sezione e del Table of Contents
\setcounter{secnumdepth}{6} % nuovo livello
\setcounter{tocdepth}{6} % nuovo livello


\linespread{1.2}
\captionsetup[table]{name=Tabella}
\geometry{headsep=1.5cm}

\renewcommand{\contentsname}{Indice}%imposto il nome dell'indice
\renewcommand\familydefault{\sfdefault}

\renewcommand{\listtablename}{Indice delle tabelle}%imposto il nome della lista tabelle
\renewcommand\familydefault{\sfdefault}

\renewcommand{\listfigurename}{Indice delle immagini}%imposto il nome della lista immagini
\renewcommand\familydefault{\sfdefault}
%-------------------INIZIO DOCUMENTO--------------
\begin{document}
\newgeometry{left=2cm,right=2cm,bottom=2.1cm,top=2.1cm}
\begin{titlepage}
	\vspace*{.5cm}

	\vspace{2cm}
	{
		\centering
		{\bfseries\huge \Title\par}
		\bigbreak
		{\bfseries\Large \Subtitle\par}
		\bigbreak
		{\bfseries\large \Author\par}
		\bigbreak
		{\Date\;-\;\Version\par}
		\vfill

		\begin{center}
			\begin{tikzpicture}
				\clip (0,0) circle (2cm) node {\includegraphics[width=4cm]{logo.jpg}};
			\end{tikzpicture}
		\end{center}
	}

	\vfill

\end{titlepage}

\restoregeometry





















\newpage
\pagestyle{fancy}
\fancyhead{}
\lhead{
	\begin{tikzpicture}
		\clip (0,0) circle (0.5cm);
		\node at (0,0) {\includegraphics[width=1cm]{./../logo/logo.png}};
	\end{tikzpicture}%
}
\chead{\vspace{\fill}\Title\vspace{\fill}}
\rhead{\vspace{\fill}\Version\vspace{\fill}}

\begin{flushleft}
    Ai professori Tullio Vardanega e Riccardo Cardin.\\
    Con il presente documento, il gruppo \textit{7Last} desidera annunciare la propria intenzione di sostenere la revisione PB (Product Baseline) del progetto 
    \begin{center}
        \textit{SyncCity: A smart city monitoring platform} 
    \end{center}
    proposto dall'azienda \textit{Sync Lab S.r.l.}.
\end{flushleft}
\begin{flushleft}
    La completa documentazione inerente al progetto è raggiungibile al seguente link:  
    \begin{center}
        \url{https://7last.github.io/}
    \end{center}
\end{flushleft}
\begin{flushleft}
    Nello specifico è presente una release \textit{PB} all'interno della quale sono visibili i documenti sviluppati finora, tra cui:
    \begin{itemize}
        \item documenti esterni:
        \begin{itemize}
            \item \textbf{Analisi dei Requisiti v2.0};
            \item \textbf{Piano di Progetto v2.0};
            \item \textbf{Piano di Qualifica v2.0};
            \item \textbf{Manuale Utente v1.0};
            \item \textbf{Specifica Tecnica v1.0};
        \end{itemize}
        \item documenti interni:
        \begin{itemize}
            \item \textbf{Norme di Progetto v2.0};
            \item \textbf{Glossario v2.0};
        \end{itemize}
        \item verbali esterni;
        \item verbali interni.
    \end{itemize}
\end{flushleft}
\newblock

\begin{flushleft}
    Di seguito viene fornito il link al repository GitHub del gruppo contenente il \textbf{Minimum Viable Product} (MVP) del progetto:
    \begin{center}
        \url{https://github.com/7Last/SyncCity}
    \end{center}
    \newpage
    Rispetto al preventivo iniziale di € 12.680,00 il gruppo è riuscito a completare il progetto con un risparmio di € 300,00, arrivando a consumare un totale di € 12.380,00. \\
    Per quanto riguarda le ore totali di attività, alcuni membri hanno totalizzato un numero di ore superiore rispetto a quanto preventivato, mentre altri un numero inferiore. Nella tabella seguente vengono riportate le ore produttive totali per ciascun membro del gruppo:
    \begin{table}[!h]
        \footnotesize
        \begin{center}
            \vspace{0.5cm}
            \begin{tabular}{|l|c|}
                \hline
                \textbf{Membro} & \textbf{Ore produttive} \\
                \hline
                Leonardo Baldo & 80.0 \\
                Antonio Benetazzo & 94.0 \\
                Elena Ferro & 95.0 \\
                Davide Malgarise & 82.0 \\
                Valerio Occhinegro & 94.5 \\
                Raul Seganfreddo & 85.5 \\
                Matteo Tiozzo & 94.0 \\ 
                \hline
                \textbf{Totale} & \textbf{625.0} \\
                \hline
            \end{tabular}
        \end{center}
    \end{table}

    A differenza di quanto previsto inizialmente, \textit{7Last} ha deciso di concludere il progetto al termine della fase di revisione PB, e quindi di non procedere con la fase di Customer Acceptance (CA), come specificato e motivato nel documento \textit{Piano di Progetto v2.0}.
    
\end{flushleft}
\begin{flushleft}
    Di seguito vengono riportarti le matricole e i nomi dei componenti del gruppo:
    \begin{itemize}
        \item 2042372 - \textbf{Leonardo Baldo};
        \item 2034528 - \textbf{Antonio Benetazzo};
        \item 2042328 - \textbf{Elena Ferro};
        \item 2009994 - \textbf{Davide Malgarise};
        \item 2011069 - \textbf{Valerio Occhinegro};
        \item 1226293 - \textbf{Raul Seganfreddo};
        \item 2042882 - \textbf{Matteo Tiozzo}.
    \end{itemize}
    Nell'attesa di un cortese riscontro, porgiamo distinti saluti, \textit{7Last}.
\end{flushleft}
\end{document}