\documentclass[italian,12pt]{article}

%--------------variabili------------------%
\def\Title{Norme di Progetto}
\def\Author{7Last}
\def\Version{v0.2}
%-----------------------------------------%


\usepackage[left=2cm, right=2cm, bottom=3cm, top=3cm]{geometry}
\usepackage{fancyhdr}
\usepackage{graphicx}
\graphicspath{ {../../logo/} }
\usepackage{href-ul}
\usepackage{tikz}
\usepackage{tgadventor}
\usepackage[useregional=numeric,showseconds=true,showzone=false]{datetime2}
\usepackage{caption}
\usepackage{longtable}
\usepackage{xcolor}




\linespread{1.2}
\captionsetup[table]{labelformat=empty}
\geometry{headsep=1.5cm}

\renewcommand{\contentsname}{Indice}
\renewcommand\familydefault{\sfdefault}

\let\oldthepage\thepage
\renewcommand{\thepage}{\sffamily \oldthepage}

\begin{document}

\newgeometry{left=2cm,right=2cm,bottom=2.1cm,top=2.1cm}
\begin{titlepage}
	\vspace*{.5cm}

	\vspace{2cm}
	{
		\centering
		{\bfseries\huge \Title\par}
		\bigbreak
		{\bfseries\Large \Subtitle\par}
		\bigbreak
		{\bfseries\large \Author\par}
		\bigbreak
		{\Date\;-\;\Version\par}
		\vfill

		\begin{center}
			\begin{tikzpicture}
				\clip (0,0) circle (2cm) node {\includegraphics[width=4cm]{logo.jpg}};
			\end{tikzpicture}
		\end{center}
	}

	\vfill

\end{titlepage}

\restoregeometry






















\newpage

\pagestyle{fancy}
\fancyhead{}
\lhead{
	\begin{tikzpicture}
		\clip (0,0) circle (0.5cm);
		\node at (0,0) {\includegraphics[width=1cm]{./../logo/logo.png}};
	\end{tikzpicture}%
}
\chead{\vspace{\fill}\Title\vspace{\fill}}
\rhead{\vspace{\fill}\Version\vspace{\fill}}


\begin{table}[!h]
	\caption{Versioni}
	\footnotesize
	\begin{center}
		\begin{tabular}{ l l l l p{6cm} }
			\hline                                                                     \\[-2ex]
			Ver. & Data       & Redattore        & Verificatore      & Descrizione     \\
			\\[-2ex] \hline \\[-1.5ex]
			1.0  & 2024-07-03 & Leonardo Baldo   & Antonio Benetazzo & Stesura verbale \\
			\\[-1.5ex] \hline
		\end{tabular}
	\end{center}
\end{table}

\newpage

\tableofcontents

\newpage

\section{Dettagli della riunione}

\textbf{Sede della riunione}: Google Meet\\
\textbf{Orario di inizio}: 16:00\\
\textbf{Orario di fine}: 16:30\\

\begin{flushleft}
	\begin{table}[!h]
		\begin{tabular}{ |l|l|l| }
			\hline
			\textbf{Partecipante} & \textbf{Ruolo} & \textbf{Presenza} \\
			\hline
			Antonio Benetazzo     & Verificatore   & Presente          \\
			Davide Malgarise      &                & Presente          \\
			Elena Ferro           &                & Presente          \\
			Leonardo Baldo        & Redattore      & Presente          \\
			Matteo Tiozzo         & Amministratore & Presente          \\
			Raul Seganfreddo      &                & Assente           \\
			Valerio Occhinegro    &                & Assente           \\
			\hline
		\end{tabular}
	\end{table}
	\textbf{Partecipanti esterni}: Fabio Pallaro, Andrea Dorigo, Daniele Zorzi.\\
\end{flushleft}

\subsection*{Ordine del giorno:}
\begin{itemize}
	\item stato di sviluppo del prodotto;
	\item obiettivi per l'undicesimo sprint;
	\item decisioni prese e conclusioni.
\end{itemize}

\newpage

\section{Verbale}

\subsection{Stato di sviluppo del prodotto}
Il gruppo, tramite il responsabile Antonio Benetazzo, presentail lavoro svolto durante il decimo sprint e lo stato del prodotto. Vengono mostrate le funzionalità sviluppate secondo gli obiettivi prefissati al precedente SAL, oltre ad altri avanzamenti che sono stati effettuati anche se non programmati. In particolare sono stati aggiunte le chiavi di partizionamento in \textit{Apache Flink} utilizzando gli UUID, in questo modo siamo riusciti a garantire che i dati vengano elaborati coerentemente. Inoltre sono state riprogettate le dashboard di \textit{Grafana} per separare i grafici di monitoraggio da quelli di analisi: i primi sono stati spostati in una dashboard dedicata, dove verranno mostrati i semplici dati grezzi (grafici time series e dati in real time), mentre i grafici di tipo analitico sono stati orgasnizzati in due dashboard separate, una per la gestione dei dati urbanistici e l'altra per i dati ambientali. Questo è stato fatto per la natura molto differente tra questi due tipi di dati, che richiedono una visualizzazione e un'analisi differente. \\
Il proponente si dichiara molto soddisfatto del lavoro svolto dal gruppo, in particolare per il livello di comprensione dello scopo del capitolato e del possibile utilizzo, dimostrati dalla qualità del prodotto. Visto lo stato di avanzamento del progetto, l'azienda propone al gruppo di fissare un incontro per permetterci di effettuare il collaudo finale e presentare loro l'MVP. Chiede ai membri del gruppo la disponibilità di effettuare l'incontro in presenza presso la loro sede di Padova, in modo da poterci conoscere di persona, e propone come data utile venerdì 12 luglio.

\subsection{Obiettivi per il decimo sprint}
Per il prossimo periodo vengono definiti i seguenti obiettivi:
\begin{itemize}
	\item calcolo dell'efficienza delle colonnine di ricarica con relativi grafici;
	\item raffinamento generale e preparazione per collaudo finale.
\end{itemize}

\newpage

\subsection{Decisioni prese e conclusioni}
Sulla base di quanto detto ci riserviamo di discutere internamente sulla possibile data del collaudo finale e di comunicare la nostra disponibilità al proponente. Ci diamo appuntamento per il prossimo incontro, che avverrà in presenza presso la sede di \textit{Sync Lab} in occasione del collaudo finale del prodotto in data da confermare.

\begin{table}[b]
	\begin{tabular}{@{}p{5cm}p{10cm}@{}}
		Data:                 & \hrulefill \\
		                      &            \\
		                      &            \\
		Firma del proponente: & \hrulefill \\
	\end{tabular}
\end{table}

\end{document}
