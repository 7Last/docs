\documentclass[italian,12pt]{article}

%--------------variabili------------------%
\def\Title{Norme di Progetto}
\def\Author{7Last}
\def\Version{v0.2}
%-----------------------------------------%


\usepackage[left=2cm, right=2cm, bottom=3cm, top=3cm]{geometry}
\usepackage{fancyhdr}
\usepackage{graphicx}
\graphicspath{ {../../logo/} }
\usepackage{href-ul}
\usepackage{tikz}
\usepackage{tgadventor}
\usepackage[useregional=numeric,showseconds=true,showzone=false]{datetime2}
\usepackage{caption}
\usepackage{longtable}
\usepackage{xcolor}




\linespread{1.2}
\captionsetup[table]{labelformat=empty}
\geometry{headsep=1.5cm}

\renewcommand{\contentsname}{Indice}
\renewcommand\familydefault{\sfdefault}

\let\oldthepage\thepage
\renewcommand{\thepage}{\sffamily \oldthepage}

\begin{document}

\newgeometry{left=2cm,right=2cm,bottom=2.1cm,top=2.1cm}
\begin{titlepage}
	\vspace*{.5cm}

	\vspace{2cm}
	{
		\centering
		{\bfseries\huge \Title\par}
		\bigbreak
		{\bfseries\Large \Subtitle\par}
		\bigbreak
		{\bfseries\large \Author\par}
		\bigbreak
		{\Date\;-\;\Version\par}
		\vfill

		\begin{center}
			\begin{tikzpicture}
				\clip (0,0) circle (2cm) node {\includegraphics[width=4cm]{logo.jpg}};
			\end{tikzpicture}
		\end{center}
	}

	\vfill

\end{titlepage}

\restoregeometry






















\newpage

\pagestyle{fancy}
\fancyhead{}
\lhead{
	\begin{tikzpicture}
		\clip (0,0) circle (0.5cm);
		\node at (0,0) {\includegraphics[width=1cm]{./../logo/logo.png}};
	\end{tikzpicture}%
}
\chead{\vspace{\fill}\Title\vspace{\fill}}
\rhead{\vspace{\fill}\Version\vspace{\fill}}


\begin{table}[!h]
	\caption{Versioni}
	\footnotesize
	\begin{center}
		\begin{tabular}{ l l l l p{6cm} }
			\hline                                                                           \\[-2ex]
			Ver. & Data       & Redattore       & Verificatore       & Descrizione           \\
			\\[-2ex] \hline \\[-1.5ex]
			1.0  & 2024-06-16 & Raul Seganfredo & Davide Malgarise & Stesura verbale \\
			\\[-1.5ex] \hline
		\end{tabular}
	\end{center}
\end{table}

\newpage

\tableofcontents

\newpage

\section{Dettagli della riunione}


\textbf{Sede della riunione}: Google Meet\\
\textbf{Orario di inizio}: 16:00\\
\textbf{Orario di fine}: 16:30\\

\begin{flushleft}
	\begin{table}[!h]
		\begin{tabular}{ |l|l|l| }
			\hline
			\textbf{Partecipante} & \textbf{Ruolo} & \textbf{Presenza} \\
			\hline
			Antonio Benetazzo     &                & Presente          \\
			Davide Malgarise      & Verificatore   & Presente          \\
			Elena Ferro           & 			   & Presente          \\
			Leonardo Baldo        & 			   & Assente           \\
			Matteo Tiozzo         & Amministratore & Presente          \\
			Raul Seganfreddo      & Redattore	   & Presente          \\
			Valerio Occhinegro    &                & Presente          \\
			\hline
		\end{tabular}
	\end{table}
	\textbf{Partecipanti esterni}: Andrea Dorigo, Daniele Zorzi, Fabio Pallaro.\\
\end{flushleft}\\
\\
\textbf{Ordine del giorno}:
\begin{itemize}
	\item Stato di sviluppo del prodotto;
	\item Obiettivi per il prossimo sprint;
	\item Decisioni prese e conclusioni.
\end{itemize}

\newpage

\section{Verbale}

\subsection{Stato di sviluppo del prodotto}
Per iniziare la riunione, il gruppo ha presentato lo stato di sviluppo del prodotto.\\
Matteo Tiozzo, che ha fatto le veci di responsabile, vista l'assenza di Leonardo Baldo, ha mostrato al proponente le
dashboard implementate, riguardanti i due nuovi sensori, quello di parcheggio e quello di qualità dell'aria.\\
Il proponente ha apprezzato il lavoro svolto, ma ha richiesto alcune modifiche, in particolare di suddividere
i dati relativi ai nuovi sensori in due dashboard distinte, una per lo stato attuale e una analitica.\\
Infine Elena Ferro ha posto un dubbio riguardante la scelta delle chiavi dei dati Kafka, in particolare, quale fosse
la chiave migliore da utilizzare.\\ Il consiglio del proponente è stato di utilizzare come chiave il nome o
l'identificativo del sensore.

\subsection{Obiettivi per il prossimo sprint}
Sono stati definiti i seguenti obiettivi per il prossimo sprint:
\begin{itemize}
	\item Arricchire i grafici dei nuovi sensori con nuove informazioni (Leonardo Baldo);
	\item Sistemare il processing dei dati con apache Flink (Matteo Tiozzo);
	\item Compilazione verbale esterno (Raul Seganfredo);
	\item Compilazione verbale interno (David Malgarise);
	\item Aggiornamento Piano di Progetto (Elena Ferro).
\end{itemize}

\subsection{Decisioni prese e conclusioni}
Dopo aver posto gli obiettivi per il prossimo sprint, il gruppo ha concordato con il proponente di mantenere la durata
degli sprint a una settimana, pianificando il prossimo SAL in data 2024-06-19.


\newpage
\begin{table}[b]
	\begin{tabular}{@{}p{5cm}p{10cm}@{}}
		Data:                 & \hrulefill \\
		                      &            \\
		                      &            \\
		Firma del proponente: & \hrulefill \\
	\end{tabular}
\end{table}

\end{document}
