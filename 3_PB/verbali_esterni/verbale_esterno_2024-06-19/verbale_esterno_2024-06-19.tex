\documentclass[italian,12pt]{article}

%--------------variabili------------------%
\def\Title{Norme di Progetto}
\def\Author{7Last}
\def\Version{v0.2}
%-----------------------------------------%


\usepackage[left=2cm, right=2cm, bottom=3cm, top=3cm]{geometry}
\usepackage{fancyhdr}
\usepackage{graphicx}
\graphicspath{ {../../logo/} }
\usepackage{href-ul}
\usepackage{tikz}
\usepackage{tgadventor}
\usepackage[useregional=numeric,showseconds=true,showzone=false]{datetime2}
\usepackage{caption}
\usepackage{longtable}
\usepackage{xcolor}




\linespread{1.2}
\captionsetup[table]{labelformat=empty}
\geometry{headsep=1.5cm}

\renewcommand{\contentsname}{Indice}
\renewcommand\familydefault{\sfdefault}

\let\oldthepage\thepage
\renewcommand{\thepage}{\sffamily \oldthepage}

\begin{document}

\newgeometry{left=2cm,right=2cm,bottom=2.1cm,top=2.1cm}
\begin{titlepage}
	\vspace*{.5cm}

	\vspace{2cm}
	{
		\centering
		{\bfseries\huge \Title\par}
		\bigbreak
		{\bfseries\Large \Subtitle\par}
		\bigbreak
		{\bfseries\large \Author\par}
		\bigbreak
		{\Date\;-\;\Version\par}
		\vfill

		\begin{center}
			\begin{tikzpicture}
				\clip (0,0) circle (2cm) node {\includegraphics[width=4cm]{logo.jpg}};
			\end{tikzpicture}
		\end{center}
	}

	\vfill

\end{titlepage}

\restoregeometry






















\newpage

\pagestyle{fancy}
\fancyhead{}
\lhead{
	\begin{tikzpicture}
		\clip (0,0) circle (0.5cm);
		\node at (0,0) {\includegraphics[width=1cm]{./../logo/logo.png}};
	\end{tikzpicture}%
}
\chead{\vspace{\fill}\Title\vspace{\fill}}
\rhead{\vspace{\fill}\Version\vspace{\fill}}


\begin{table}[!h]
	\caption{Versioni}
	\footnotesize
	\begin{center}
		\begin{tabular}{ l l l l p{6cm} }
			\hline                                                                     \\[-2ex]
			Ver. & Data       & Redattore        & Verificatore      & Descrizione     \\
			\\[-2ex] \hline \\[-1.5ex]
			1.0  & 2024-06-22 & Leonardo Baldo   & Raul Seganfreddo  & Stesura verbale \\
			\\[-1.5ex] \hline
		\end{tabular}
	\end{center}
\end{table}

\newpage

\tableofcontents

\newpage

\section{Dettagli della riunione}

\textbf{Sede della riunione}: Google Meet\\
\textbf{Orario di inizio}: 16:00\\
\textbf{Orario di fine}: 16:45\\

\begin{flushleft}
	\begin{table}[!h]
		\begin{tabular}{ |l|l|l| }
			\hline
			\textbf{Partecipante} & \textbf{Ruolo} & \textbf{Presenza} \\
			\hline
			Antonio Benetazzo     &                 & Assente           \\
			Davide Malgarise      &                 & Assente           \\
			Elena Ferro           & Amministratore  & Presente          \\
			Leonardo Baldo        &                 & Presente          \\
			Matteo Tiozzo         & Redattore       & Presente          \\
			Raul Seganfreddo      & Verificatore    & Presente          \\
			Valerio Occhinegro    & 				& Presente          \\
			\hline
		\end{tabular}
	\end{table}
	\textbf{Partecipanti esterni}: Fabio Pallaro.\\
\end{flushleft}

\subsection*{Ordine del giorno:}
\begin{itemize}
	\item stato di sviluppo del prodotto;
	\item obiettivi per il nono sprint;
	\item decisioni prese e conclusioni.
\end{itemize}

\newpage

\section{Verbale}

\subsection{Stato di sviluppo del prodotto}
L'incontro si apre con la presentazione da parte del responsabile, Valerio Occhinegro, delle dashboard visualizzate in \textit{Grafana} realizzate fino a questo momento. Si pone particolare attenzione sui nuovi sensori implementati che monitorano il livello dei fiumi, le colonnine di ricarica e le precipitazioni. Si passa poi a quanto realizzato in \textit{Apache Flink}, per quanto riguarda l'aggregazione dei dati, in particolare finalizzata al calcolo della temperatura percepita. Il proponente approva il lavoro svolto ma suggerisce alcune migliorie soprattutto relative ai titoli dei grafici mostrati in \textit{Grafana} e relativamente alla dashboard della temperatura, dove oltre alla temperatura attuale e percepita si potrebbe mostrare anche l'umidità.

\subsection{Obiettivi per il nono sprint}
Per il prossimo periodo vengono definiti i seguenti obiettivi:
\begin{itemize}
	\item mostrare i sensori dell'umidità nella mappa della temperatura;
	\item sistemare i titoli dei grafici di \textit{Grafana}.
\end{itemize}

\subsection{Decisioni prese e conclusioni}
In accordo con il proponente, si mantiene la durata del periodo di lavoro di una settimana e dunque si fissa il prossimo incontro in data
26 giugno 2024.

\begin{table}[b]
	\begin{tabular}{@{}p{5cm}p{10cm}@{}}
		Data:                 & \hrulefill \\
		                      &            \\
		                      &            \\
		Firma del proponente: & \hrulefill \\
	\end{tabular}
\end{table}

\end{document}
