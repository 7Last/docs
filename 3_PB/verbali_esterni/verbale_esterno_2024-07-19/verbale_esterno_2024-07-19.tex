\documentclass[italian,12pt]{article}

%--------------variabili------------------%
\def\Title{Norme di Progetto}
\def\Author{7Last}
\def\Version{v0.2}
%-----------------------------------------%


\usepackage[left=2cm, right=2cm, bottom=3cm, top=3cm]{geometry}
\usepackage{fancyhdr}
\usepackage{graphicx}
\graphicspath{ {../../logo/} }
\usepackage{href-ul}
\usepackage{tikz}
\usepackage{tgadventor}
\usepackage[useregional=numeric,showseconds=true,showzone=false]{datetime2}
\usepackage{caption}
\usepackage{longtable}
\usepackage{xcolor}




\linespread{1.2}
\captionsetup[table]{labelformat=empty}
\geometry{headsep=1.5cm}

\renewcommand{\contentsname}{Indice}
\renewcommand\familydefault{\sfdefault}

\let\oldthepage\thepage
\renewcommand{\thepage}{\sffamily \oldthepage}

\begin{document}

\newgeometry{left=2cm,right=2cm,bottom=2.1cm,top=2.1cm}
\begin{titlepage}
	\vspace*{.5cm}

	\vspace{2cm}
	{
		\centering
		{\bfseries\huge \Title\par}
		\bigbreak
		{\bfseries\Large \Subtitle\par}
		\bigbreak
		{\bfseries\large \Author\par}
		\bigbreak
		{\Date\;-\;\Version\par}
		\vfill

		\begin{center}
			\begin{tikzpicture}
				\clip (0,0) circle (2cm) node {\includegraphics[width=4cm]{logo.jpg}};
			\end{tikzpicture}
		\end{center}
	}

	\vfill

\end{titlepage}

\restoregeometry






















\newpage

\pagestyle{fancy}
\fancyhead{}
\lhead{
	\begin{tikzpicture}
		\clip (0,0) circle (0.5cm);
		\node at (0,0) {\includegraphics[width=1cm]{./../logo/logo.png}};
	\end{tikzpicture}%
}
\chead{\vspace{\fill}\Title\vspace{\fill}}
\rhead{\vspace{\fill}\Version\vspace{\fill}}


\begin{table}[!h]
	\caption{Versioni}
	\footnotesize
	\begin{center}
		\begin{tabular}{ l l l l p{6cm} }
			\hline                                                                     \\[-2ex]
			Ver. & Data       & Redattore        & Verificatore      & Descrizione     \\
			\\[-2ex] \hline \\[-1.5ex]
			1.0  & 2024-07-19 & Matteo Tiozzo    & Antonio Benetazzo & Stesura verbale \\
			\\[-1.5ex] \hline
		\end{tabular}
	\end{center}
\end{table}

\newpage

\tableofcontents

\newpage

\section{Dettagli della riunione}

\textbf{Sede della riunione}: Sync Lab S.r.l.\\
\textbf{Orario di inizio}: 15:00\\
\textbf{Orario di fine}: 16:00\\

\begin{flushleft}
	\begin{table}[!h]
		\begin{tabular}{ |l|l|l| }
			\hline
			\textbf{Partecipante} & \textbf{Ruolo} & \textbf{Presenza} \\
			\hline
			Antonio Benetazzo     & Verificatore   & Presente          \\
			Davide Malgarise      &                & Presente          \\
			Elena Ferro           & Amministratore & Presente          \\
			Leonardo Baldo        &       		   & Presente          \\
			Matteo Tiozzo         & Redattore      & Presente          \\
			Raul Seganfreddo      &                & Presente          \\
			Valerio Occhinegro    &                & Presente          \\
			\hline
		\end{tabular}
	\end{table}
	\textbf{Partecipanti esterni}: Fabio Pallaro, Andrea Dorigo, Daniele Zorzi.\\
\end{flushleft}

\subsection*{Ordine del giorno:}
\begin{itemize}
	\item collaudo finale con il proponente.
\end{itemize}

\newpage

\section{Verbale}
\subsection{Collaudo finale con il proponente}
Il gruppo presenta il prodotto finale al proponente, elencando le funzionalità principali, le soluzioni, le tecnologie utilizzate per lo sviluppo e il loro funzionamento all'interno dello stesso. Una volta effettuata la presentazione \textit{7Last} procede con la presentazione vera e propria del prodotto. 

\subsection{Conclusioni}
A seguito della dimostrazione del prodotto e dopo aver risposto a tutte le domande e i dubbi del proponente, quest'ultimo si è dichiarato pienamente soddisfatto. Congratulandosi con il team per la presentazione e il lavoro svolto, ha approvato il \textit{Minimum Viable Product}, esprimendo apprezzamento per l'organizzazione, la struttura e le funzionalità del sistema.
\begin{table}[b]
	\begin{tabular}{@{}p{5cm}p{10cm}@{}}
		Data:                 & \hrulefill \\
		                      &            \\
		                      &            \\
		Firma del proponente: & \hrulefill \\
	\end{tabular}
\end{table}
\end{document}