\documentclass[italian,12pt]{article}

%--------------variabili------------------%
\def\Title{Norme di Progetto}
\def\Author{7Last}
\def\Version{v0.2}
%-----------------------------------------%


\usepackage[left=2cm, right=2cm, bottom=3cm, top=3cm]{geometry}
\usepackage{fancyhdr}
\usepackage{graphicx}
\graphicspath{ {../../logo/} }
\usepackage{href-ul}
\usepackage{tikz}
\usepackage{tgadventor}
\usepackage[useregional=numeric,showseconds=true,showzone=false]{datetime2}
\usepackage{caption}
\usepackage{longtable}
\usepackage{xcolor}




\linespread{1.2}
\captionsetup[table]{labelformat=empty}
\geometry{headsep=1.5cm}

\renewcommand{\contentsname}{Indice}
\renewcommand\familydefault{\sfdefault}

\let\oldthepage\thepage
\renewcommand{\thepage}{\sffamily \oldthepage}

\begin{document}

\newgeometry{left=2cm,right=2cm,bottom=2.1cm,top=2.1cm}
\begin{titlepage}
	\vspace*{.5cm}

	\vspace{2cm}
	{
		\centering
		{\bfseries\huge \Title\par}
		\bigbreak
		{\bfseries\Large \Subtitle\par}
		\bigbreak
		{\bfseries\large \Author\par}
		\bigbreak
		{\Date\;-\;\Version\par}
		\vfill

		\begin{center}
			\begin{tikzpicture}
				\clip (0,0) circle (2cm) node {\includegraphics[width=4cm]{logo.jpg}};
			\end{tikzpicture}
		\end{center}
	}

	\vfill

\end{titlepage}

\restoregeometry






















\newpage

\pagestyle{fancy}
\fancyhead{}
\lhead{
	\begin{tikzpicture}
		\clip (0,0) circle (0.5cm);
		\node at (0,0) {\includegraphics[width=1cm]{./../logo/logo.png}};
	\end{tikzpicture}%
}
\chead{\vspace{\fill}\Title\vspace{\fill}}
\rhead{\vspace{\fill}\Version\vspace{\fill}}


\begin{table}[!h]
	\caption{Versioni}
	\footnotesize
	\begin{center}
		\begin{tabular}{ l l l l p{6cm} }
			\hline                                                                     \\[-2ex]
			Ver. & Data       & Redattore        & Verificatore      & Descrizione     \\
			\\[-2ex] \hline \\[-1.5ex]
			1.0  & 2024-06-29 & Davide Malgarise & Matteo Tiozzo     & Stesura verbale \\
			\\[-1.5ex] \hline
		\end{tabular}
	\end{center}
\end{table}

\newpage

\tableofcontents

\newpage

\section{Dettagli della riunione}

\textbf{Sede della riunione}: Google Meet\\
\textbf{Orario di inizio}: 16:00\\
\textbf{Orario di fine}: 16:30\\

\begin{flushleft}
	\begin{table}[!h]
		\begin{tabular}{ |l|l|l| }
			\hline
			\textbf{Partecipante} & \textbf{Ruolo} & \textbf{Presenza} \\
			\hline
			Antonio Benetazzo     & Amministratore  & Presente          \\
			Davide Malgarise      & Redattore       & Presente          \\
			Elena Ferro           & 			    & Presente          \\
			Leonardo Baldo        &                 & Presente          \\
			Matteo Tiozzo         & Verificatore    & Presente          \\
			Raul Seganfreddo      & 			    & Assente           \\
			Valerio Occhinegro    & 				& Assente           \\
			\hline
		\end{tabular}
	\end{table}
	\textbf{Partecipanti esterni}: Fabio Pallaro.\\
\end{flushleft}

\subsection*{Ordine del giorno:}
\begin{itemize}
	\item stato di sviluppo del prodotto;
	\item obiettivi per il decimo sprint;
	\item decisioni prese e conclusioni.
\end{itemize}

\newpage

\section{Verbale}

\subsection{Stato di sviluppo del prodotto}
Inizialmente il responsabile del periodo di lavoro, Elena Ferro, presenta lo stato attuale del prodotto mostrando i progressi fatti
per quanto riguarda le dashboard di \textit{Grafana}, sulle quali il proponente chiede approfondimenti e chiarimenti,
anche in vista della prossima valutazione con il committente.
In particolare si discute della funzione che genera i dati per la temperatura percepita in base al periodo dell'anno e si pensa ad una possibile
distinzione dei veicoli considerati nella ricarica delle colonnine elettriche.

\subsection{Obiettivi per il decimo sprint}
Per il prossimo periodo vengono definiti i seguenti obiettivi:
\begin{itemize}
	\item raffinamento dashboard di \textit{Grafana} tramite separazione dei grafici di monitoraggio e grafici di analisi;
	\item implementazione di chiavi di partizionamento in \textit{Apache Flink}.
\end{itemize}

\subsection{Decisioni prese e conclusioni}
Si discute sulla possibile data in cui avverrà il prossimo incontro. Tenendo conto degli esami universitari che alcuni membri del team dovranno affrontare,
il proponente suggerisce di conferire maggiore durata al prossimo periodo di lavoro, passando dai consuetudinari 7 giorni a 10. Tuttavia il gruppo
ritiene di essere in grado di gestire il seguente rischio e mantiene la durata dello sprint invariata, accordandosi per il 2024-07-03.

\begin{table}[b]
	\begin{tabular}{@{}p{5cm}p{10cm}@{}}
		Data:                 & \hrulefill \\
		                      &            \\
		                      &            \\
		Firma del proponente: & \hrulefill \\
	\end{tabular}
\end{table}

\end{document}
